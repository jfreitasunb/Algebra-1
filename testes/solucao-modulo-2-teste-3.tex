%!TEX program = xelatex
%!TEX encoding = ISO-8859-1
\documentclass[12pt]{article}

\usepackage{amssymb}
\usepackage{amsmath,amsfonts,amsthm,amstext,mathabx}
\usepackage[brazil]{babel}
%\usepackage[latin1]{inputenc}
\usepackage{graphicx}
\graphicspath{{/home/jfreitas/Dropbox/imagens-latex/}{/Volumes/Vader/Dropbox/imagens-latex/}{D:/Dropbox/imagens-latex/}}
\usepackage{enumitem}
\usepackage{multicol}
\usepackage[all]{xy}

\setlength{\topmargin}{-1.0in}
\setlength{\oddsidemargin}{0in}
\setlength{\textheight}{10.1in}
\setlength{\textwidth}{6.5in}
\setlength{\baselineskip}{12mm}

\newcounter{exercicios}
\setcounter{exercicios}{0}
\newcommand{\questao}{
\addtocounter{exercicios}{1}
\noindent{\bf Exerc{\'\i}cio \arabic{exercicios}: }}

\newcommand{\equi}{\Leftrightarrow}
\newcommand{\bic}{\leftrightarrow}
\newcommand{\cond}{\rightarrow}
\newcommand{\impl}{\Rightarrow}
\newcommand{\nao}{\sim}
\newcommand{\sub}{\subseteq}
\newcommand{\e}{\ \wedge\ }
\newcommand{\ou}{\ \vee\ }
\newcommand{\vaz}{\emptyset}
\newcommand{\nsub}{\nsubset}
\renewcommand{\sin}{{\rm sen\,}}

\newcommand{\n}{\mathbb{N}}
\newcommand{\z}{\mathbb{Z}}
\newcommand{\real}{\mathbb{R}}
\newcommand{\vesp}{\vspace{0.2cm}}
\newcommand{\subne}{\subsetneqq}


\newcommand{\compcent}[1]{\vcenter{\hbox{$#1\circ$}}}
\newcommand{\comp}{\mathbin{\mathchoice
{\compcent\scriptstyle}{\compcent\scriptstyle}
{\compcent\scriptscriptstyle}{\compcent\scriptscriptstyle}}}

\begin{document}

\pagestyle{empty}

\begin{figure}[h]
        \begin{minipage}[c]{1.7cm}
        \includegraphics[width=1.7cm]{unb.pdf}
        \end{minipage}%
        \hspace{0pt}
        \begin{minipage}[c]{4in}
          {Universidade de Brasília} \\
          {Departamento de Matemática}
\end{minipage}
\end{figure}
\vspace{-1cm}\hrule


\begin{center}
{\Large\bf {\'A}lgebra 1 - Turma C -- 1$^{o}$/2018} \\ \vspace{9pt} {\large\bf
  $3^{\underline{o}}$ Teste M\'odulo 2 - Resolu\c{c}\~ao}\\
\vspace{9pt} Prof. Jos{\'e} Ant{\^o}nio O. Freitas
\end{center}
\hrule

\vspace{.6cm}

\questao Em $\z \times \z$ defina
\begin{align*}
	(a, b) \oplus (c,d) &= (a + b, c + d)\\
	(a, b) \otimes (c, d) &= (ac, ad + bc)
\end{align*}
para todos $(a, b)$, $(c, d) \in \z \times \z$. Mostre que $(\z \times \z, \oplus, \otimes)$ é um anel. É comutativo? Possui unidade?

\noindent\textbf{Solu\c{c}\~ao:}

	\begin{enumerate}
		\item Para todos $(a, b)$, $(c, d)$, $(e, f) \in \z \times \z$ temos
		\begin{align*}
			[(a, b) \oplus (c, d)] \oplus (e, f) &= (a + c, b + d) \oplus (e, f) = (a + c + e, b + d + f)\\
			(a, b) \oplus [(c, d) \oplus (e, f)] = (a, b) \oplus (c + e, d + f) = (a + c + e, b + d + f)
		\end{align*}
		Logo $[(a, b) \oplus (c, d)] \oplus (e, f) = (a, b) \oplus [(c, d) \oplus (e, f)]$, como queríamos.

		\item Para todos $(a, b)$, $(c, d) \in \z \times \z$ temos
		\begin{align*}
			(a, b) \oplus (c, d) &= (a + c, b + d)\\
			(c, d) \oplus (a, b) &= (c + a, d + b)
		\end{align*}
		Logo $(a, b) \oplus (c, d) = (c, d) \oplus (a, b)$.

		\item Tome $0_{\z \times \z} = (0, 0)$. Para todo $(a, b) \in \z \times \z$ temos
		\[
			(a, b) \oplus 0_{\z \times \z} = (a, b) \oplus (0, 0) = (a + 0, b + 0) = (a, b).
		\]
		Assim $0_{\z \times \z} = (0, 0)$ é o elemento neutro da operação $\oplus$ em $\z \times \z$.

		\item Para $(a, b) \in \z \times \z$ tome $(-a, -b) \in \z \times \z$. Temos
		\[
			(a, b) \oplus (-a, -b) = (a - a, b - b) = (0, 0) = 0_{\z \times \z}.
		\]
		Logo $(-a, -b)$ é o oposto de $(a, b)$ na operação $\oplus$ em $\z \times \z$.

		\item Para todos $(a, b)$, $(c, d)$, $(e, f) \in \z \times \z$ temos
		\begin{align*}
			[(a, b) \otimes (c, d)] \otimes (e, f) &= (ac, ad + bc) \otimes (e, f) = (ace, acf + ade + bce)\\
			(a, b) \otimes [(c, d) \otimes (e, f)] &= (a, b) \otimes (ce, cf + de) = (ace, acf + ade + bce)
		\end{align*}
		Logo $[(a, b) \otimes (c, d)] \otimes (e, f) = (a, b) \otimes [(c, d) \otimes (e, f)]$, como queríamos.

		\item Para todos $(a, b)$, $(c, d)$, $(e, f) \in \z \times \z$ temos
		\begin{align*}
			[(a, b) \oplus (c, d)] \otimes (e, f) &= (a + c, b + d) \otimes (e, f) = (ae + ce, af + cf + be + de)\\
			[(a, b) \otimes (e, f)] \oplus [(c, d) \otimes (e, f)] &= (ae, af + be) \otimes (ce, cf + de) = (ae + ce, af + be + cf + de)
		\end{align*}

		Logo $[(a, b) \oplus (c, d)] \otimes (e, f) = [(a, b) \otimes (e, f)] \oplus [(c, d) \otimes (e, f)]$.

		\item Para todos $(a, b)$, $(c, d)$, $(e, f) \in \z \times \z$ temos
		\begin{align*}
			(a, b) \otimes [(c, d) \oplus (e, f)] &= (a, b) \otimes (c + e, d + f) = (ac + ae, ad + af + bc + be)\\
			[(a, b) \otimes (c, d)] \oplus [(a, b) \otimes (e, f)] &= (ac, ad + bc) \oplus (ae, af + be) = (ac + ae, ad + bc + af + be)
		\end{align*}

		Logo $(a, b) \otimes [(c, d) \oplus (e, f)] = [(a, b) \otimes (c, d)] \oplus [(a, b) \otimes (e, f)]$.
	\end{enumerate}

\end{document}