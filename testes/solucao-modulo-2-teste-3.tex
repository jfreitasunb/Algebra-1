%!TEX program = xelatex
%!TEX encoding = ISO-8859-1
\documentclass[12pt]{article}

\usepackage{amssymb}
\usepackage{amsmath,amsfonts,amsthm,amstext,mathabx}
\usepackage[brazil]{babel}
%\usepackage[latin1]{inputenc}
\usepackage{graphicx}
\graphicspath{{/home/jfreitas/Dropbox/imagens-latex/}{/Volumes/Vader/Dropbox/imagens-latex/}{D:/Dropbox/imagens-latex/}}
\usepackage{enumitem}
\usepackage{multicol}
\usepackage[all]{xy}

\setlength{\topmargin}{-1.0in}
\setlength{\oddsidemargin}{0in}
\setlength{\textheight}{10.1in}
\setlength{\textwidth}{6.5in}
\setlength{\baselineskip}{12mm}

\newcounter{exercicios}
\setcounter{exercicios}{0}
\newcommand{\questao}{
\addtocounter{exercicios}{1}
\noindent{\bf Exerc{\'\i}cio \arabic{exercicios}: }}

\newcommand{\equi}{\Leftrightarrow}
\newcommand{\bic}{\leftrightarrow}
\newcommand{\cond}{\rightarrow}
\newcommand{\impl}{\Rightarrow}
\newcommand{\nao}{\sim}
\newcommand{\sub}{\subseteq}
\newcommand{\e}{\ \wedge\ }
\newcommand{\ou}{\ \vee\ }
\newcommand{\vaz}{\emptyset}
\newcommand{\nsub}{\nsubset}
\renewcommand{\sin}{{\rm sen\,}}

\newcommand{\n}{\mathbb{N}}
\newcommand{\z}{\mathbb{Z}}
\newcommand{\real}{\mathbb{R}}
\newcommand{\vesp}{\vspace{0.2cm}}
\newcommand{\subne}{\subsetneqq}


\newcommand{\compcent}[1]{\vcenter{\hbox{$#1\circ$}}}
\newcommand{\comp}{\mathbin{\mathchoice
{\compcent\scriptstyle}{\compcent\scriptstyle}
{\compcent\scriptscriptstyle}{\compcent\scriptscriptstyle}}}

\begin{document}

\pagestyle{empty}

\begin{figure}[h]
        \begin{minipage}[c]{1.7cm}
        \includegraphics[width=1.7cm]{unb.pdf}
        \end{minipage}%
        \hspace{0pt}
        \begin{minipage}[c]{4in}
          {Universidade de Brasília} \\
          {Departamento de Matemática}
\end{minipage}
\end{figure}
\vspace{-1cm}\hrule


\begin{center}
{\Large\bf {\'A}lgebra 1 - Turma C -- 1$^{o}$/2018} \\ \vspace{9pt} {\large\bf
  $3^{\underline{o}}$ Teste M\'odulo 2 - Resolu\c{c}\~ao}\\
\vspace{9pt} Prof. Jos{\'e} Ant{\^o}nio O. Freitas
\end{center}
\hrule

\vspace{.6cm}

\questao Em $\z \times \z$ defina
\begin{align*}
	(a, b) \oplus (c,d) &= (a + b, c + d)\\
	(a, b) \otimes (c, d) &= (ac, ad + bc)
\end{align*}
para todos $(a, b)$, $(c, d) \in \z \times \z$. Mostre que $(\z \times \z, \oplus, \otimes)$ é um anel. É comutativo? Possui unidade?

\noindent\textbf{Solu\c{c}\~ao:}

	Sejam $x_1$, $x_2 \in A$ tais que $f(x_1) = f(x_2)$. Como $f(x_1)$, $f(x_2) \in B$ e $g : B \to A$ \'e uma fun\c{c}\~ao temos $g(f(x_1)) = g(f(x_2))$. Isto \'e, $(g \comp f)(x_1) = (g \comp f)(x_2)$ e como $g \comp f$ \'e injetora segue que $x_1 = x_2$. Portanto $f$ \'e injetora.

\vspace{.5cm}

\questao Seja $f : A \to B$ uma fun\c{c}\~ao injetora e $P$, $Q \sub A$. Mostre que
\[
	f(P \cap Q) = f(P) \cap f(Q).
\]

\noindent\textbf{Solu\c{c}\~ao:} Primeiro temos
\begin{align*}
	f(P \cap Q) &= \{f(x) \mid x \in P \cap Q\}\\
	f(P) &= \{f(z) \mid z \in P\}\\
	f(Q) &= \{f(t) \mid t \in Q\}.
\end{align*}
Seja $y \in f(P \cap Q)$. Ent\~ao existe $x \in P \cap Q$ tal que $y = f(x)$. Como $x \in P \cap Q$, ent\~ao $x \in P$ e $x \in Q$. Logo $y = f(x)$ com $x \in P$ e $y = f(x)$ com $x \in Q$. Ou seja, $y \in f(P)$ e $y \in f(Q)$. Portanto $y \in f(P) \cap f(Q)$.

Agora seja $y \in f(P) \cap f(Q)$. Da{\'\i} $y \in f(P)$ e $y \in f(Q)$. Assim $y = f(x_1)$ com $x_1 \in P$ e $y = f(x_2)$ com $x_2 \in Q$. Logo $y = f(x_1) = f(x_2)$ e como $f$ \'e injetora, segue que $x_1 = x_2$. Portanto $y = f(x_1)$ com $x_1 = x_1 \in P \cap Q$, isto \'e, $y \in f(P \cap Q)$.

Com isso $f(P \cap Q) = f(P) \cap f(Q)$.

\end{document}