%!TEX program = xelatex
% !TEX encoding = ISO-8859-1
\def\ano{2019}
\def\semestre{2}
\def\disciplina{\'Algebra 1}
\def\turma{C}

\documentclass[12pt]{exam}

\usepackage{caption}
\usepackage{amssymb}
\usepackage{amsmath,amsfonts,amsthm,amstext}
\usepackage[brazil]{babel}
% \usepackage[latin1]{inputenc}
\usepackage{graphicx}
\graphicspath{{/ArquivosLinux/OneDrive/imagens-latex/}{D:/OneDrive - unb.br/imagens-latex/}}
\usepackage{enumitem}
\usepackage{multicol}
\usepackage{answers}
\usepackage{tikz,ifthen}
\usetikzlibrary{lindenmayersystems}
\usetikzlibrary[shadings]
\Newassociation{solucao}{Solution}{ans}
\newtheorem{exercicio}{}

\setlength{\topmargin}{-1.0in}
\setlength{\oddsidemargin}{0in}
\setlength{\textheight}{10.1in}
\setlength{\textwidth}{6.5in}
\setlength{\baselineskip}{12mm}

\extraheadheight{0.7in}
\firstpageheadrule
\runningheadrule
\lhead{
        \begin{minipage}[c]{1.7cm}
        \includegraphics[width=1.7cm]{unb.pdf}
        \end{minipage}%
        \hspace{0pt}
        \begin{minipage}[c]{4in}
          {Universidade de Brasília} --
          {Departamento de Matemática}
\end{minipage}
\vspace*{-0.8cm}
}
% \chead{Universidade de Brasília - Departamento de Matemática}
% \rhead{}
% \vspace*{-2cm}

\extrafootheight{.5in}
\footrule
\lfoot{\disciplina\ - \semestre$^o$/\ano\ - Módulo \numeromodulo}
\cfoot{}
\rfoot{Página \thepage\ de \numpages}

\newcounter{exercicios}
\renewcommand{\theexercicios}{\arabic{exercicios}}

\newenvironment{questao}[1]{
\refstepcounter{exercicios}
\ifx&#1&
\else
   \label{#1}
\fi
\noindent\textbf{Exercício {\theexercicios}:}
}

\newcommand{\resp}[1]{
\noindent{\bf Exercício #1: }}

\def\ano{2024}
\def\semestre{1}
\def\disciplina{Álgebra 1}
\def\nomeabreviado{Álgebra 1}
\def\turma{1}

\newcommand{\im}{{\rm Im\,}}
\newcommand{\dlim}[2]{\displaystyle\lim_{#1\rightarrow #2}}
\newcommand{\minf}{+\infty}
\newcommand{\ninf}{-\infty}
\newcommand{\cp}[1]{\mathbb{#1}}
\newcommand{\sub}{\subseteq}
\newcommand{\n}{\mathbb{N}}
\newcommand{\z}{\mathbb{Z}}
\newcommand{\rac}{\mathbb{Q}}
\newcommand{\real}{\mathbb{R}}
\newcommand{\complex}{\mathbb{C}}

\newcommand{\vesp}[1]{\vspace{ #1  cm}}

\newcommand{\compcent}[1]{\vcenter{\hbox{$#1\circ$}}}
\newcommand{\comp}{\mathbin{\mathchoice
        {\compcent\scriptstyle}{\compcent\scriptstyle}
        {\compcent\scriptscriptstyle}{\compcent\scriptscriptstyle}}}
\renewcommand{\sin}{{\rm sen\,}}
\renewcommand{\tan}{{\rm tg\,}}
\renewcommand{\csc}{{\rm cossec\,}}
\renewcommand{\cot}{{\rm cotg\,}}
\renewcommand{\sinh}{{\rm senh\,}}

\begin{document}

\begin{center}
{\Large\bf \disciplina\ - Turma \turma\ -- \semestre$^{o}$/\ano} \\ \vspace{9pt} {\large\bf
$2^{\underline{o}}$ Teste - M\'odulo 2 - Resolu\c{c}\~ao}\\
\vspace{9pt} Prof. Jos{\'e} Ant{\^o}nio O. Freitas
\end{center}
\hrule

\vspace{.6cm}

\questao Em $\z \times \z$ defina
\begin{align*}
	(a, b) \oplus (c,d) &= (a + b, c + d)\\
	(a, b) \otimes (c, d) &= (ac, ad + bc)
\end{align*}
para todos $(a, b)$, $(c, d) \in \z \times \z$. Mostre que $(\z \times \z, \oplus, \otimes)$ \'e um anel. \'E comutativo? Possui unidade?

\noindent\textbf{Solu\c{c}\~ao:}

	\begin{enumerate}[label={\roman*})]
		\item Para todos $(a, b)$, $(c, d)$, $(e, f) \in \z \times \z$ temos
		\begin{align*}
			[(a, b) \oplus (c, d)] \oplus (e, f) &= (a + c, b + d) \oplus (e, f) = (a + c + e, b + d + f)\\
			(a, b) \oplus [(c, d) \oplus (e, f)] &= (a, b) \oplus (c + e, d + f) = (a + c + e, b + d + f)
		\end{align*}
		Logo $[(a, b) \oplus (c, d)] \oplus (e, f) = (a, b) \oplus [(c, d) \oplus (e, f)]$, como quer{\'\i}amos.

		\item Para todos $(a, b)$, $(c, d) \in \z \times \z$ temos
		\begin{align*}
			(a, b) \oplus (c, d) &= (a + c, b + d)\\
			(c, d) \oplus (a, b) &= (c + a, d + b)
		\end{align*}
		Logo $(a, b) \oplus (c, d) = (c, d) \oplus (a, b)$.

		\item Tome $0_{\z \times \z} = (0, 0)$. Para todo $(a, b) \in \z \times \z$ temos
		\[
			(a, b) \oplus 0_{\z \times \z} = (a, b) \oplus (0, 0) = (a + 0, b + 0) = (a, b).
		\]
		Assim $0_{\z \times \z} = (0, 0)$ \'e o elemento neutro da opera\c{c}\~ao $\oplus$ em $\z \times \z$.

		\item Para $(a, b) \in \z \times \z$ tome $(-a, -b) \in \z \times \z$. Temos
		\[
			(a, b) \oplus (-a, -b) = (a - a, b - b) = (0, 0) = 0_{\z \times \z}.
		\]
		Logo $(-a, -b)$ \'e o oposto de $(a, b)$ na opera\c{c}\~ao $\oplus$ em $\z \times \z$.

		\item Para todos $(a, b)$, $(c, d)$, $(e, f) \in \z \times \z$ temos
		\begin{align*}
			[(a, b) \otimes (c, d)] \otimes (e, f) &= (ac, ad + bc) \otimes (e, f) = (ace, acf + ade + bce)\\
			(a, b) \otimes [(c, d) \otimes (e, f)] &= (a, b) \otimes (ce, cf + de) = (ace, acf + ade + bce)
		\end{align*}
		Logo $[(a, b) \otimes (c, d)] \otimes (e, f) = (a, b) \otimes [(c, d) \otimes (e, f)]$, como quer{\'\i}amos.

		\item Para todos $(a, b)$, $(c, d)$, $(e, f) \in \z \times \z$ temos
		\begin{align*}
			[(a, b) \oplus (c, d)] \otimes (e, f) &= (a + c, b + d) \otimes (e, f) \\& = (ae + ce, af + cf + be + de)\\
			[(a, b) \otimes (e, f)] \oplus [(c, d) \otimes (e, f)] &= (ae, af + be) \otimes (ce, cf + de) \\ &= (ae + ce, af + be + cf + de)
		\end{align*}

		Logo $[(a, b) \oplus (c, d)] \otimes (e, f) = [(a, b) \otimes (e, f)] \oplus [(c, d) \otimes (e, f)]$.

		\item Para todos $(a, b)$, $(c, d)$, $(e, f) \in \z \times \z$ temos
		\begin{align*}
			(a, b) \otimes [(c, d) \oplus (e, f)] &= (a, b) \otimes (c + e, d + f) \\ &= (ac + ae, ad + af + bc + be)\\
			[(a, b) \otimes (c, d)] \oplus [(a, b) \otimes (e, f)] &= (ac, ad + bc) \oplus (ae, af + be) \\ &= (ac + ae, ad + bc + af + be)
		\end{align*}

		Logo $(a, b) \otimes [(c, d) \oplus (e, f)] = [(a, b) \otimes (c, d)] \oplus [(a, b) \otimes (e, f)]$.
	\end{enumerate}

	Portanto $(\z \times \z, \oplus, \otimes)$ \'e um anel.

	Agora para todos $(a, b)$, $(c, d) \in \z \times \z$ temos
	\[
	 	(a, b) \otimes (c, d) = (ac, ad + bc) = (ca, da + cb) = (c, d) \otimes (a, b).
	\]
	Portanto a opera\c{c}\~ao $\otimes$ \'e comutativa. Assim $(\z \times \z, \oplus, \otimes)$ \'e um anel comutativo.

	Tome $1_{\z \times \z} = (1, 0)$. Para todo $(a, b) \in \z \times \z$ temos
	\[
		(a, b) \otimes 1_{\z \times \z} = (a, b) \otimes (1, 0) = (a, b).
	\]
	Logo $1_{\z \times \z} = (1, 0)$ \'e a unidade da opera\c{c}\~ao $\otimes$ em $\z \times \z$. Portanto $(\z \times \z, \oplus, \otimes)$ \'e um anel com unidade.
\end{document}