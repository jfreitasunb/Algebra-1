%!TEX program = xelatex
% !TEX encoding = ISO-8859-1
\def\ano{2019}
\def\semestre{2}
\def\disciplina{\'Algebra 1}
\def\turma{C}
\def\numeroteste{1}
\def\modulo{3}

\documentclass[12pt]{exam}

\usepackage{caption}
\usepackage{amssymb}
\usepackage{amsmath,amsfonts,amsthm,amstext}
\usepackage[brazil]{babel}
% \usepackage[latin1]{inputenc}
\usepackage{graphicx}
\graphicspath{{/ArquivosLinux/OneDrive/imagens-latex/}{D:/OneDrive - unb.br/imagens-latex/}}
\usepackage{enumitem}
\usepackage{multicol}
\usepackage{answers}
\usepackage{tikz,ifthen}
\usetikzlibrary{lindenmayersystems}
\usetikzlibrary[shadings]
\Newassociation{solucao}{Solution}{ans}
\newtheorem{exercicio}{}

\setlength{\topmargin}{-1.0in}
\setlength{\oddsidemargin}{0in}
\setlength{\textheight}{10.1in}
\setlength{\textwidth}{6.5in}
\setlength{\baselineskip}{12mm}

\extraheadheight{0.7in}
\firstpageheadrule
\runningheadrule
\lhead{
        \begin{minipage}[c]{1.7cm}
        \includegraphics[width=1.7cm]{unb.pdf}
        \end{minipage}%
        \hspace{0pt}
        \begin{minipage}[c]{4in}
          {Universidade de Brasília} --
          {Departamento de Matemática}
\end{minipage}
\vspace*{-0.8cm}
}
% \chead{Universidade de Brasília - Departamento de Matemática}
% \rhead{}
% \vspace*{-2cm}

\extrafootheight{.5in}
\footrule
\lfoot{\disciplina\ - \semestre$^o$/\ano\ - Módulo \numeromodulo}
\cfoot{}
\rfoot{Página \thepage\ de \numpages}

\newcounter{exercicios}
\renewcommand{\theexercicios}{\arabic{exercicios}}

\newenvironment{questao}[1]{
\refstepcounter{exercicios}
\ifx&#1&
\else
   \label{#1}
\fi
\noindent\textbf{Exercício {\theexercicios}:}
}

\newcommand{\resp}[1]{
\noindent{\bf Exercício #1: }}

\def\ano{2024}
\def\semestre{1}
\def\disciplina{Álgebra 1}
\def\nomeabreviado{Álgebra 1}
\def\turma{1}

\newcommand{\im}{{\rm Im\,}}
\newcommand{\dlim}[2]{\displaystyle\lim_{#1\rightarrow #2}}
\newcommand{\minf}{+\infty}
\newcommand{\ninf}{-\infty}
\newcommand{\cp}[1]{\mathbb{#1}}
\newcommand{\sub}{\subseteq}
\newcommand{\n}{\mathbb{N}}
\newcommand{\z}{\mathbb{Z}}
\newcommand{\rac}{\mathbb{Q}}
\newcommand{\real}{\mathbb{R}}
\newcommand{\complex}{\mathbb{C}}

\newcommand{\vesp}[1]{\vspace{ #1  cm}}

\newcommand{\compcent}[1]{\vcenter{\hbox{$#1\circ$}}}
\newcommand{\comp}{\mathbin{\mathchoice
        {\compcent\scriptstyle}{\compcent\scriptstyle}
        {\compcent\scriptscriptstyle}{\compcent\scriptscriptstyle}}}
\renewcommand{\sin}{{\rm sen\,}}
\renewcommand{\tan}{{\rm tg\,}}
\renewcommand{\csc}{{\rm cossec\,}}
\renewcommand{\cot}{{\rm cotg\,}}
\renewcommand{\sinh}{{\rm senh\,}}

\begin{document}

\begin{center}
{\Large\bf \disciplina\ - Turma \turma\ -- \semestre$^{o}$/\ano} \\ \vspace{9pt} {\large\bf
$\numeroteste^{\underline{o}}$ Teste - M\'odulo \modulo\ - Resolu\c{c}\~ao}\\
\vspace{9pt} Prof. Jos{\'e} Ant{\^o}nio O. Freitas
\end{center}
\hrule

\vspace{.6cm}

\questao{} Mostre que
\[
	C = \{6k \mid k \in \z\}
\]
é um subanel de $(\rac, \star, \odot)$ onde
\begin{align*}
	x \star y &= x + y - 6\\
	x \odot y &= x + y - \dfrac{xy}{6}
\end{align*}
para $x$, $y \in \rac$.

\noindent\textbf{Solu\c{c}\~ao:} Sabemos que o elemento neutro da operação $\star$ em $\rac$ é $6 = 6\cdot 1$. Logo $6 \in C$ e assim $C \neq \emptyset$.

Agora sejam $a$, $b \in C$. Assim $a = 6k$, $b = 6l$, com $k$, $l \in \z$ e o oposto de $b$ na operação $\star$ de $\rac$ é $-b = 12 - b$. Assim
\begin{align*}
	x \star (-b) &= 6k \star (12 - 6l) = 6k + 12 - 6l - 6 = 6(k - l + 1) \in C\\
	x \odot b &= 6k + 6l - \dfrac{6k6l}{6} = 6(k + l - 6kl) \in C
\end{align*}
pois $k - l + 1 \in \z$ e $k + l - 6kl \in \z$.

Portanto $(C, \star, \odot)$ é um subanel de $(\rac, \star, \odot)$.

\vspace{1.5cm}

\questao{} Mostre que $f : \z \to \z_n$ dada por $f(x) = \overline{x}$ é um homomorfismo de anéis, onde as operações de $\z$ e $\z_n$ são as usuais.

\noindent\textbf{Solu\c{c}\~ao:} Sejam $x$, $y \in \z$. Temos
\begin{align*}
	f(x + y) &= \overline{x + y} = \overline{x} + \overline{y} = f(x) + f(y)\\
	f(x\cdot y) &= \overline{x\cdot y} = \overline{x}\cdot\overline{y} = f(x)\cdot f(y).
\end{align*}
Portanto $f$ é um homomorfismo de anéis.
\end{document}