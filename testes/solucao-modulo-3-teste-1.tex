%!TEX program = xelatex
%!TEX encoding = ISO-8859-1
\documentclass[12pt]{article}

\usepackage{amssymb}
\usepackage{amsmath,amsfonts,amsthm,amstext,mathabx}
\usepackage[brazil]{babel}
%\usepackage[latin1]{inputenc}
\usepackage{graphicx}
\graphicspath{{/home/jfreitas/Dropbox/imagens-latex/}{/Volumes/Vader/Dropbox/imagens-latex/}{D:/Dropbox/imagens-latex/}}
\usepackage{enumitem}
\usepackage{multicol}
\usepackage[all]{xy}

\setlength{\topmargin}{-1.0in}
\setlength{\oddsidemargin}{0in}
\setlength{\textheight}{10.1in}
\setlength{\textwidth}{6.5in}
\setlength{\baselineskip}{12mm}

\newcounter{exercicios}
\setcounter{exercicios}{0}
\newcommand{\questao}{
\addtocounter{exercicios}{1}
\noindent{\bf Exerc{\'\i}cio \arabic{exercicios}: }}

\newcommand{\equi}{\Leftrightarrow}
\newcommand{\bic}{\leftrightarrow}
\newcommand{\cond}{\rightarrow}
\newcommand{\impl}{\Rightarrow}
\newcommand{\nao}{\sim}
\newcommand{\sub}{\subseteq}
\newcommand{\e}{\ \wedge\ }
\newcommand{\ou}{\ \vee\ }
\newcommand{\vaz}{\emptyset}
\newcommand{\nsub}{\nsubset}
\renewcommand{\sin}{{\rm sen\,}}

\newcommand{\n}{\mathbb{N}}
\newcommand{\z}{\mathbb{Z}}
\newcommand{\real}{\mathbb{R}}
\newcommand{\vesp}{\vspace{0.2cm}}
\newcommand{\subne}{\subsetneqq}


\newcommand{\compcent}[1]{\vcenter{\hbox{$#1\circ$}}}
\newcommand{\comp}{\mathbin{\mathchoice
{\compcent\scriptstyle}{\compcent\scriptstyle}
{\compcent\scriptscriptstyle}{\compcent\scriptscriptstyle}}}

\begin{document}

\pagestyle{empty}

\begin{figure}[h]
        \begin{minipage}[c]{1.7cm}
        \includegraphics[width=1.7cm]{unb.pdf}
        \end{minipage}%
        \hspace{0pt}
        \begin{minipage}[c]{4in}
          {Universidade de Brasília} \\
          {Departamento de Matemática}
\end{minipage}
\end{figure}
\vspace{-1cm}\hrule


\begin{center}
{\Large\bf {\'A}lgebra 1 - Turma C -- 1$^{o}$/2018} \\ \vspace{9pt} {\large\bf
  $1^{\underline{o}}$ Teste  - M\'odulo 3 - Resolu\c{c}\~ao}\\
\vspace{9pt} Prof. Jos{\'e} Ant{\^o}nio O. Freitas
\end{center}
\hrule

\vspace{.6cm}

\questao Seja
\[
	M_2(\q) = \left\{ \begin{pmatrix}
		a & b\\c & d
	\end{pmatrix} \mid a,b,c,d \in \q\right\}
\]
um anel com as opera\c{c}\~oes usuais. Mostre que
\[
	B = \left\{ \begin{pmatrix}
		a & 0\\0 & a
	\end{pmatrix} \mid a \in \q\right\}	
\]
\'e um subanel. Esse subanel \'e comutativo?

\noindent\textbf{Solu\c{c}\~ao:} De fato, primeiro tomando $a = 0$ segue que
\[
	\begin{pmatrix}
		0 & 0\\0 & 0
	\end{pmatrix} \in B.
\]
Logo $B \ne \emptyset$.

Agora dados $x$, $y \in B$ ent\~ao
\[
	x = \begin{pmatrix}
		a & 0\\0 & a
	\end{pmatrix}\quad \mbox{ e } \quad y = \begin{pmatrix}
		b & 0\\0 & b
	\end{pmatrix}.
\]
Assim
\begin{align*}
	x - y &= \begin{pmatrix}
		a & 0\\0 & a
	\end{pmatrix} - \begin{pmatrix}
		b & 0\\0 & b
	\end{pmatrix} = \begin{pmatrix}
		a - b & 0\\0 & a - b
	\end{pmatrix} \in B\\
	xy &= \begin{pmatrix}
		a & 0\\0 & a
	\end{pmatrix}\begin{pmatrix}
		b & 0\\0 & b
	\end{pmatrix} = \begin{pmatrix}
		ab & 0\\0 & ab
	\end{pmatrix} \in B.
\end{align*}
Logo $B$ \'e um subanel de $M_2(\q)$.

Agora observe que
\begin{align*}
	xy = \begin{pmatrix}
		a & 0\\0 & a
	\end{pmatrix}\begin{pmatrix}
		b & 0\\0 & b
	\end{pmatrix} = \begin{pmatrix}
		ab & 0\\0 & ab
	\end{pmatrix} = \begin{pmatrix}
		b & 0\\0 & b
	\end{pmatrix}\begin{pmatrix}
		a & 0\\0 & a
	\end{pmatrix} = yx
\end{align*}
para todos $x$, $y \in B$. Logo $B$ \'e um subanel comutativo.


\vspace{.5cm}

\questao Considere o anel $(\z \times \z, \oplus, \otimes)$ cujas opera\c{c}\~oes s\~ao dadas por
\begin{align*}
	(x, y) \oplus (z, t) &= (x + z, y + t)\\
	(x, y) \otimes (z, t) &= (xz, yt)
\end{align*}
para todos $(x, y)$, $(z, t) \in \z \times \z$. Mostre que $f : \z \times \z \to \z \times \z$ dada por
\[
	f(x, y) = (y, x)
\]
\'e um homomorfismo de an\'eis.

\noindent\textbf{Solu\c{c}\~ao:} Dados $(x, y)$, $(z, t) \in \z \times \z$ temos
\begin{align*}
	f((x, y) \oplus (z, t)) &= f(x + z, y + t) = (y + t, x + z) \\ &= (y, x) \oplus (t, z) \\ &= f(x, y) \oplus f(z, t)\\
	f((x, y) \otimes (z, t)) &= f(xz, yt) = (yt, xz) \\ &= (y, x) \otimes (t, z) \\ &= f(x, y) \otimes f(z, t)\\
\end{align*}
\end{document}