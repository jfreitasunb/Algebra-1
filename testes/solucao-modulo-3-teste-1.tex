%!TEX program = xelatex
%!TEX encoding = ISO-8859-1
\documentclass[12pt]{article}

\usepackage{amssymb}
\usepackage{amsmath,amsfonts,amsthm,amstext,mathabx}
\usepackage[brazil]{babel}
%\usepackage[latin1]{inputenc}
\usepackage{graphicx}
\graphicspath{{/home/jfreitas/Dropbox/imagens-latex/}{/Volumes/Vader/Dropbox/imagens-latex/}{D:/Dropbox/imagens-latex/}}
\usepackage{enumitem}
\usepackage{multicol}
\usepackage[all]{xy}

\setlength{\topmargin}{-1.0in}
\setlength{\oddsidemargin}{0in}
\setlength{\textheight}{10.1in}
\setlength{\textwidth}{6.5in}
\setlength{\baselineskip}{12mm}

\newcounter{exercicios}
\setcounter{exercicios}{0}
\newcommand{\questao}{
\addtocounter{exercicios}{1}
\noindent{\bf Exerc{\'\i}cio \arabic{exercicios}: }}

\newcommand{\equi}{\Leftrightarrow}
\newcommand{\bic}{\leftrightarrow}
\newcommand{\cond}{\rightarrow}
\newcommand{\impl}{\Rightarrow}
\newcommand{\nao}{\sim}
\newcommand{\sub}{\subseteq}
\newcommand{\e}{\ \wedge\ }
\newcommand{\ou}{\ \vee\ }
\newcommand{\vaz}{\emptyset}
\newcommand{\nsub}{\nsubset}
\renewcommand{\sin}{{\rm sen\,}}

\newcommand{\n}{\mathbb{N}}
\newcommand{\z}{\mathbb{Z}}
\newcommand{\real}{\mathbb{R}}
\newcommand{\vesp}{\vspace{0.2cm}}
\newcommand{\subne}{\subsetneqq}


\newcommand{\compcent}[1]{\vcenter{\hbox{$#1\circ$}}}
\newcommand{\comp}{\mathbin{\mathchoice
{\compcent\scriptstyle}{\compcent\scriptstyle}
{\compcent\scriptscriptstyle}{\compcent\scriptscriptstyle}}}

\begin{document}

\pagestyle{empty}

\begin{figure}[h]
        \begin{minipage}[c]{1.7cm}
        \includegraphics[width=1.7cm]{unb.pdf}
        \end{minipage}%
        \hspace{0pt}
        \begin{minipage}[c]{4in}
          {Universidade de Brasília} \\
          {Departamento de Matemática}
\end{minipage}
\end{figure}
\vspace{-1cm}\hrule


\begin{center}
{\Large\bf {\'A}lgebra 1 - Turma D -- 2$^{o}$/2017} \\ \vspace{9pt} {\large\bf
  $1^{\underline{o}}$ Teste Módulo 3 - Resolu\c{c}\~ao}\\
\vspace{9pt} Prof. Jos{\'e} Ant{\^o}nio O. Freitas
\end{center}
\hrule

\vspace{.6cm}

\questao Considere os seguintes an{\'e}is: $(\real, +, \cdot)$ e $(\real, \oplus, \otimes)$, sendo $a \oplus b = a + b + 1$ e $a \otimes b = a + b + ab$. Mostre que $f : \real \to \real$ dado por $f(x) = x + 1$, para todo $x \in \real$, {\'e} um homomorfimo de $(\real, \oplus, \otimes)$ em $(\real, +, \cdot)$.

\noindent\textbf{Solu\c{c}\~ao:}
Precisamos mostrar que
\begin{enumerate}[label=({\alph*})]
	\item $f(x \oplus y) = f(x) + f(y)$
	\item $f(x \otimes y) = f(x)\cdot f(y)$
\end{enumerate}
para todos $x$, $y \in \real$.

De fato, para $x$, $y \in \real$ temos
\begin{align*}
	f(x \oplus y) &= f(x + y + 1) = (x + y + 1) + 1 = (x + 1) + (y + 1) = f(x) + f(y)\\
	f(x \otimes y) &=  f(x + y + xy) = (x + y + xy) + 1 = x(1 + y) + (1 + y) \\ &= (x + 1)(y + 1) = f(x)\cdot f(y)
\end{align*}
Logo $f$ é um homomorfismo de $(\real, \oplus, \otimes)$ em $(\real, +, \cdot)$.

\vspace{.5cm}

\questao Seja $f: A \to B$ um homomorfismo de an{\'e}is. Mostre que se $D$ {\'e} um subanel de $B$, ent{\~a}o $f^{-1}(D)$ {\'e} um subanel de $A$.

\noindent\textbf{Solu\c{c}\~ao:} Inicialmente lembre que
\[
	f^{-1}(D) = \{x \in A \mid f(x) \in D\}.
\]

Para mostrar que $f^{-1}(D)$ é um subanel de $A$ precisamos mostrar que $f^{-1}(D) \ne \emptyset$ e que para todos $x$, $y \in f^{-1}(D)$ temos
\begin{enumerate}[label=({\alph*})]
	\item $x - y \in f^{-1}(D)$
	\item $xy \in f^{-1}(D)$.
\end{enumerate}

Primeiro, como $f$ é um homomorfismo de anéis, então $f(0_A) = 0_B$. Agora $D$ sendo um subanel de $B$ então $0_B \in D$. Logo $0_A \in f^{-1}(D)$ pois $f(0_A) \in D$.

Dados $x$, $y \in f^{-1}(D)$ temos que $f(x) \in D$ e $f(y) \in D$. Daí
\begin{align*}
	f(x - y) &= f(x) - f(y) \in D\\
	f(xy) &= f(x)f(y) \in D
\end{align*}
pois $D$ é um subanel de $B$. Portanto $x - y \in f^{-1}(D)$ e $xy \in f^{-1}(D)$ e com isso $f^{-1}(D)$ é um subanel de $A$.
\end{document}