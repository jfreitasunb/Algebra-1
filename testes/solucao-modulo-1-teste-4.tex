%!TEX program = xelatex
%!TEX encoding = ISO-8859-1
\documentclass[12pt]{article}

\usepackage{amssymb}
\usepackage{amsmath,amsfonts,amsthm,amstext,mathabx}
\usepackage[brazil]{babel}
%\usepackage[latin1]{inputenc}
\usepackage{graphicx}
\graphicspath{{/home/jfreitas/Dropbox/imagens-latex/}{/Volumes/Vader/Dropbox/imagens-latex/}{D:/Dropbox/imagens-latex/}}
\usepackage{enumitem}
\usepackage{multicol}
\usepackage[all]{xy}

\setlength{\topmargin}{-1.0in}
\setlength{\oddsidemargin}{0in}
\setlength{\textheight}{10.1in}
\setlength{\textwidth}{6.5in}
\setlength{\baselineskip}{12mm}

\newcounter{exercicios}
\setcounter{exercicios}{0}
\newcommand{\questao}{
\addtocounter{exercicios}{1}
\noindent{\bf Exerc{\'\i}cio \arabic{exercicios}: }}

\newcommand{\equi}{\Leftrightarrow}
\newcommand{\bic}{\leftrightarrow}
\newcommand{\cond}{\rightarrow}
\newcommand{\impl}{\Rightarrow}
\newcommand{\nao}{\sim}
\newcommand{\sub}{\subseteq}
\newcommand{\e}{\ \wedge\ }
\newcommand{\ou}{\ \vee\ }
\newcommand{\vaz}{\emptyset}
\newcommand{\nsub}{\nsubset}
\renewcommand{\sin}{{\rm sen\,}}

\newcommand{\n}{\mathbb{N}}
\newcommand{\z}{\mathbb{Z}}
\newcommand{\real}{\mathbb{R}}
\newcommand{\vesp}{\vspace{0.2cm}}
\newcommand{\subne}{\subsetneqq}


\newcommand{\compcent}[1]{\vcenter{\hbox{$#1\circ$}}}
\newcommand{\comp}{\mathbin{\mathchoice
{\compcent\scriptstyle}{\compcent\scriptstyle}
{\compcent\scriptscriptstyle}{\compcent\scriptscriptstyle}}}

\begin{document}

\pagestyle{empty}

\begin{figure}[h]
        \begin{minipage}[c]{1.7cm}
        \includegraphics[width=1.7cm]{unb.pdf}
        \end{minipage}%
        \hspace{0pt}
        \begin{minipage}[c]{4in}
          {Universidade de Brasília} \\
          {Departamento de Matemática}
\end{minipage}
\end{figure}
\vspace{-1cm}\hrule


\begin{center}
{\Large\bf {\'A}lgebra 1 - Turma D -- 2$^{o}$/2017} \\ \vspace{9pt} {\large\bf
  $4^{\underline{o}}$ Teste - Resolu\c{c}\~ao}\\
\vspace{9pt} Prof. Jos{\'e} Ant{\^o}nio O. Freitas
\end{center}
\hrule

\vspace{.6cm}

\questao Para $(a,b)$, $(c,d) \in \real \times \real$ defina $(a,b)R(c,d)$ quando $a^2 + b^2 = c^2 + d^2$. Mostre que $R$ é uma relação de equivalência sobre $\real \times \real$.

\noindent\textbf{Solu\c{c}\~ao:} Temos
\begin{enumerate}[label={\roman*})]
	\item $(a,b)R(a,b)$ pois $a^2 + b^2 = a^2 + b^2$.
	\item Se $(a,b)R(c,d)$, então $a^2 + b^2 = c^2 + d^2$. Assim $c^2 + d^2 = a^2 +  b^2$. Logo $(c,d)R(a,b)$.
	\item Se $(a,b)R(c,d)$ e $(c,d)R(e,f)$, então $a^2 + b^2 = c^2 + d^2$ e $c^2 + d^2 = e^2 + f^2$. Assim $a^2 + b^2 = e^2 + f^2$. Logo $(a,b)R(c,d)$.
\end{enumerate}

Portanto $R$ é uma relação de equivalência sobre $\real \times \real$.


\vspace{.5cm}

\questao Em $\z^* = \z - \{0\}$ defina $xRy$ quando $x \mid y$ e $y \mid x$. Mostre que $R$ é uma relação de equivalência sobre $\z^*$ e encontre $\z^*/R$.

\noindent\textbf{Solu\c{c}\~ao:} De fato,
	\begin{enumerate}[label={\alph*})]
		\item $xRx$ pois $x \mid x$.

		\item Se $xRy$, então $x \mid y$ e $y \mid x$. Isto é, $y \mid x$ e $x \mid y$. Logo $yRx$.

		\item Se $xRy$ e $yRz$, então $x \mid y$ e $y \mid x$ e $y \mid z$ e $z \mid y$. De $x \mid y$ e $y \mid z$ segue que $x \mid z$. Agora de $z \mid y$ e $y \mid x$ segue que $z \mid x$. Assim $x \mid z$ e $z \mid x$. Logo $xRz$.
	\end{enumerate}

	Portanto $R$ é uma relação de equivalência sobre $\z^*.$

	Agora, dado $a \in \z^*$ temos
	\[
		\overline{a} = \{x \in \z^* : xRa\} = \{ x \in \z^* : x \mid a \mbox{ e } a \mid x\}.
	\]
	Assim existem $k$, $l \in \z$ tais que
	\begin{align*}
		x &= ka\\
		a &= lx
	\end{align*}
	logo
	\begin{align*}
		&x = k(lx)\\
		&x(1 - kl) = 0
	\end{align*}
	e como $x \in \z^*$ segue que $1 - kl = 0$. Assim $k = l = 1$ ou $k = l = -1$. Portanto
	\[
		\overline{a} = \{-a,a\}.
	\]
	Assim
	\[
		\z^*/R = \{\overline{a} \mid a \in \z^*\}.
	\]
\end{document}