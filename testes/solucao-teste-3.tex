%!TEX program = xelatex
%!TEX encoding = ISO-8859-1
\documentclass[12pt]{article}

\usepackage{amssymb}
\usepackage{amsmath,amsfonts,amsthm,amstext,mathabx}
\usepackage[brazil]{babel}
%\usepackage[latin1]{inputenc}
\usepackage{graphicx}
\graphicspath{{/home/jfreitas/Dropbox/imagens-latex/}{/Volumes/Vader/Dropbox/imagens-latex/}{D:/Dropbox/imagens-latex/}}
\usepackage{enumitem}
\usepackage{multicol}
\usepackage[all]{xy}

\setlength{\topmargin}{-1.0in}
\setlength{\oddsidemargin}{0in}
\setlength{\textheight}{10.1in}
\setlength{\textwidth}{6.5in}
\setlength{\baselineskip}{12mm}

\newcounter{exercicios}
\setcounter{exercicios}{0}
\newcommand{\questao}{
\addtocounter{exercicios}{1}
\noindent{\bf Exerc{\'\i}cio \arabic{exercicios}: }}

\newcommand{\equi}{\Leftrightarrow}
\newcommand{\bic}{\leftrightarrow}
\newcommand{\cond}{\rightarrow}
\newcommand{\impl}{\Rightarrow}
\newcommand{\nao}{\sim}
\newcommand{\sub}{\subseteq}
\newcommand{\e}{\ \wedge\ }
\newcommand{\ou}{\ \vee\ }
\newcommand{\vaz}{\emptyset}
\newcommand{\nsub}{\nsubset}
\renewcommand{\sin}{{\rm sen\,}}

\newcommand{\n}{\mathbb{N}}
\newcommand{\z}{\mathbb{Z}}
\newcommand{\real}{\mathbb{R}}
\newcommand{\vesp}{\vspace{0.2cm}}
\newcommand{\subne}{\subsetneqq}


\newcommand{\compcent}[1]{\vcenter{\hbox{$#1\circ$}}}
\newcommand{\comp}{\mathbin{\mathchoice
{\compcent\scriptstyle}{\compcent\scriptstyle}
{\compcent\scriptscriptstyle}{\compcent\scriptscriptstyle}}}

\begin{document}

\pagestyle{empty}

\begin{figure}[h]
        \begin{minipage}[c]{1.7cm}
        \includegraphics[width=1.7cm]{unb.pdf}
        \end{minipage}%
        \hspace{0pt}
        \begin{minipage}[c]{4in}
          {Universidade de Brasília} \\
          {Departamento de Matemática}
\end{minipage}
\end{figure}
\vspace{-1cm}\hrule


\begin{center}
{\Large\bf {\'A}lgebra 1 - Turma D -- 2$^{o}$/2017} \\ \vspace{9pt} {\large\bf
  $3^{\underline{o}}$ Teste - Resolu\c{c}\~ao}\\
\vspace{9pt} Prof. Jos{\'e} Ant{\^o}nio O. Freitas
\end{center}
\hrule

\vspace{.6cm}

\questao Seja $R \sub \q \times \q$ definido por $xRy$ quando $x - y \in \z$. Mostre que $R$ é uma relação de equivalência sobre $\q$.

\noindent\textbf{Solu\c{c}\~ao:} Precisamos mostrar que:
\begin{enumerate}[label={\roman*})]
	\item $(A - B) - C \sub A - (B \cup C)$
	\item $A - (B \cup C) \sub (A - B) - C$
\end{enumerate}


\vspace{.5cm}

\questao Seja $A = \z \times \z^*$, onde $\z^* = \z - \{0\}$. Para $(a,b)$, $(c,d) \in A$ defina
\[
	(a,b)R(c,d) \mbox{ quando } ad=bc.
\].

\begin{enumerate}
	\item Mostre que $R$ é uma rela{\c c}{\~a}o de equival{\^e}ncia sobre $A$.
	\item Encontre a classe de equivalência $\overline{(2,1)}$ e exiba 5 elementos.
\end{enumerate}

\noindent\textbf{Solu\c{c}\~ao:} De fato,
	\begin{itemize}
		\item Para todo $x\in \z$ temos $x - x = m\cdot0$ e com isso $(x,x) \in R$.
		\item Se $(x,y) \in R$ então existe $k \in \z$ tal que $x - y = mk$. Agora $y - x = -(x - y) = -mk = m (-k)$ e como $-k \in \z$ segue que $(y,x) \in R$.
		\item Se $(x,y) \in R$ e $(y,z) \in R$, então existem $k$, $l\in \z$ tais que $x - y = mk$ e $y - z = ml$.
		Somando essas duas equações obtemos
		\begin{align*}
			(x - y) + (y - z) &= mk + ml\\
			x - z &= m(k + l)
		\end{align*}
		e como $k + l \in \z$ segue que $(x,z) \in \z$.
	\end{itemize}
	
	Assim $R$ é uma relação de equivalência.
\end{document}