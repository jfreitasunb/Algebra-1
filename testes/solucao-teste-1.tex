%!TEX program = xelatex
%!TEX encoding = ISO-8859-1
\documentclass[12pt]{article}

\usepackage{amssymb}
\usepackage{amsmath,amsfonts,amsthm,amstext,mathabx}
\usepackage[brazil]{babel}
%\usepackage[latin1]{inputenc}
\usepackage{graphicx}
\graphicspath{{/home/jfreitas/Dropbox/imagens-latex/}{/Volumes/Vader/Dropbox/imagens-latex/}{D:/Dropbox/imagens-latex/}}
\usepackage{enumitem}
\usepackage{multicol}
\usepackage[all]{xy}

\setlength{\topmargin}{-1.0in}
\setlength{\oddsidemargin}{0in}
\setlength{\textheight}{10.1in}
\setlength{\textwidth}{6.5in}
\setlength{\baselineskip}{12mm}

\newcounter{exercicios}
\setcounter{exercicios}{0}
\newcommand{\questao}{
\addtocounter{exercicios}{1}
\noindent{\bf Exerc{\'\i}cio \arabic{exercicios}: }}

\newcommand{\equi}{\Leftrightarrow}
\newcommand{\bic}{\leftrightarrow}
\newcommand{\cond}{\rightarrow}
\newcommand{\impl}{\Rightarrow}
\newcommand{\nao}{\sim}
\newcommand{\sub}{\subseteq}
\newcommand{\e}{\ \wedge\ }
\newcommand{\ou}{\ \vee\ }
\newcommand{\vaz}{\emptyset}
\newcommand{\nsub}{\nsubset}
\renewcommand{\sin}{{\rm sen\,}}

\newcommand{\n}{\mathbb{N}}
\newcommand{\z}{\mathbb{Z}}
\newcommand{\real}{\mathbb{R}}
\newcommand{\vesp}{\vspace{0.2cm}}
\newcommand{\subne}{\subsetneqq}


\newcommand{\compcent}[1]{\vcenter{\hbox{$#1\circ$}}}
\newcommand{\comp}{\mathbin{\mathchoice
{\compcent\scriptstyle}{\compcent\scriptstyle}
{\compcent\scriptscriptstyle}{\compcent\scriptscriptstyle}}}

\begin{document}

\pagestyle{empty}

\begin{figure}[h]
        \begin{minipage}[c]{1.7cm}
        \includegraphics[width=1.7cm]{unb.pdf}
        \end{minipage}%
        \hspace{0pt}
        \begin{minipage}[c]{4in}
          {Universidade de Brasília} \\
          {Departamento de Matemática}
\end{minipage}
\end{figure}
\vspace{-1cm}\hrule


\begin{center}
{\Large\bf {\'A}lgebra 1 - Turma B -- 2$^{o}$/2017} \\ \vspace{9pt} {\large\bf
  $1^{\underline{o}}$ Teste - Resolu\c{c}\~ao}\\
\vspace{9pt} Prof. Jos{\'e} Ant{\^o}nio O. Freitas
\end{center}
\hrule

\vspace{.6cm}

% \textbf{O Exerc{\'\i}cio 1 foi corrigido para atribui\c{c}\~ao da nota.}

\vspace{.6cm}

\questao Sejam $A$ e $B$ conjuntos. Suponha que $x \in A$ e $A \nsub B$. Então $x \notin B$?

\noindent\textbf{Solu\c{c}\~ao:} Falso. Tome por exemplo $A = \{1,2,3\}$, $B = \{3,4\}$ e $x = 3$. Temos $A \nsub B$, $x \in A$ e no entanto $x \in B$.

\vspace{.5cm}

\questao Mostre que $A \cup ( B - C) = (A \cup B) - (C - A)$.

\noindent\textbf{Solu\c{c}\~ao:} Para mostrar a igualdade desses conjuntos precisamos mostrar que:
\begin{enumerate}
	\item [$1^o$)] $A \cup ( B - C) \subseteq (A \cup B) - (C - A)$

	Seja $x \in A \cup (B - C)$. Ent\~ao $x \in A$ ou $x \in (B - C)$. Se $x \in A$, ent\~ao $x \in A \cup B$ e $x \notin (C - A)$. Assim $x \in (A \cup B) - (C - A)$. Com isso $A \cup ( B - C) \subseteq (A \cup B) - (C - A)$. Agora suponha que $x \in (B - C)$. Ent\~ao $x \in B$ e $x \notin C$. Neste caso $x \in A \cup B$ e $x \notin (C - A)$ pois $x \notin C$. Logo $x \in (A \cup B) - (C - A)$ e novamente $A \cup ( B - C) \subseteq (A \cup B) - (C - A)$. Portanto $A \cup ( B - C) \subseteq (A \cup B) - (C - A)$.

	\item [$2^o$)] $(A \cup B) - (C - A) \subseteq A \cup ( B - C)$

	Seja $x \in (A \cup B) - (C - A)$. Assim $x \in A \cup B$ e $x \notin (C - A)$. Se $x \in A$, ent\~ao $x \in A \cup (B - C)$ e acabamos. Suponha que $x \in B$. Como $x \notin (C - A)$, ent\~ao $x \notin C$ ou $x \in A$. Se $x \notin C$, ent\~ao como $x \in B$ obtemos que $x \in B - C$ e com isso $x \in A \cup (B - C)$. Se $x \in A$, ent\~ao $x \in A \cup (B - C)$. Logo sempre teremos $x \in A \cup (B - C)$ e assim $(A \cup B) - (C - A) \subseteq A \cup ( B - C)$. Portanto $(A \cup B) - (C - A) \subseteq A \cup ( B - C)$.
\end{enumerate}
Com isso provamos que $(A \cup B) - (C - A) = A \cup ( B - C)$, como quer{\'\i}amos.

\end{document}