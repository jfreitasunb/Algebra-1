%!TEX program = xelatex
% !TEX encoding = ISO-8859-1
\def\ano{2020}
\def\semestre{1}
\def\disciplina{\'Algebra 1}
\def\turma{C}

\documentclass[12pt]{exam}

\usepackage{caption}
\usepackage{amssymb}
\usepackage{amsmath,amsfonts,amsthm,amstext}
\usepackage[brazil]{babel}
% \usepackage[latin1]{inputenc}
\usepackage{graphicx}
\graphicspath{{/ArquivosLinux/OneDrive/imagens-latex/}{D:/OneDrive - unb.br/imagens-latex/}}
\usepackage{enumitem}
\usepackage{multicol}
\usepackage{answers}
\usepackage{tikz,ifthen}
\usetikzlibrary{lindenmayersystems}
\usetikzlibrary[shadings]
\Newassociation{solucao}{Solution}{ans}
\newtheorem{exercicio}{}

\setlength{\topmargin}{-1.0in}
\setlength{\oddsidemargin}{0in}
\setlength{\textheight}{10.1in}
\setlength{\textwidth}{6.5in}
\setlength{\baselineskip}{12mm}

\extraheadheight{0.7in}
\firstpageheadrule
\runningheadrule
\lhead{
        \begin{minipage}[c]{1.7cm}
        \includegraphics[width=1.7cm]{unb.pdf}
        \end{minipage}%
        \hspace{0pt}
        \begin{minipage}[c]{4in}
          {Universidade de Brasília} --
          {Departamento de Matemática}
\end{minipage}
\vspace*{-0.8cm}
}
% \chead{Universidade de Brasília - Departamento de Matemática}
% \rhead{}
% \vspace*{-2cm}

\extrafootheight{.5in}
\footrule
\lfoot{\disciplina\ - \semestre$^o$/\ano\ - Módulo \numeromodulo}
\cfoot{}
\rfoot{Página \thepage\ de \numpages}

\newcounter{exercicios}
\renewcommand{\theexercicios}{\arabic{exercicios}}

\newenvironment{questao}[1]{
\refstepcounter{exercicios}
\ifx&#1&
\else
   \label{#1}
\fi
\noindent\textbf{Exercício {\theexercicios}:}
}

\newcommand{\resp}[1]{
\noindent{\bf Exercício #1: }}

\def\ano{2024}
\def\semestre{1}
\def\disciplina{Álgebra 1}
\def\nomeabreviado{Álgebra 1}
\def\turma{1}

\newcommand{\im}{{\rm Im\,}}
\newcommand{\dlim}[2]{\displaystyle\lim_{#1\rightarrow #2}}
\newcommand{\minf}{+\infty}
\newcommand{\ninf}{-\infty}
\newcommand{\cp}[1]{\mathbb{#1}}
\newcommand{\sub}{\subseteq}
\newcommand{\n}{\mathbb{N}}
\newcommand{\z}{\mathbb{Z}}
\newcommand{\rac}{\mathbb{Q}}
\newcommand{\real}{\mathbb{R}}
\newcommand{\complex}{\mathbb{C}}

\newcommand{\vesp}[1]{\vspace{ #1  cm}}

\newcommand{\compcent}[1]{\vcenter{\hbox{$#1\circ$}}}
\newcommand{\comp}{\mathbin{\mathchoice
        {\compcent\scriptstyle}{\compcent\scriptstyle}
        {\compcent\scriptscriptstyle}{\compcent\scriptscriptstyle}}}
\renewcommand{\sin}{{\rm sen\,}}
\renewcommand{\tan}{{\rm tg\,}}
\renewcommand{\csc}{{\rm cossec\,}}
\renewcommand{\cot}{{\rm cotg\,}}
\renewcommand{\sinh}{{\rm senh\,}}

\begin{document}
	\begin{center}
	{\Large\bf \disciplina\ - Turma \turma\ -- \semestre$^{o}$/\ano} \\ \vspace{9pt} {\large\bf
	Teste Discursivo 2 - M\'odulo 1 - Resolu\c{c}\~ao}\\
	\vspace{9pt} Prof. Jos{\'e} Ant{\^o}nio O. Freitas
	\end{center}
	\hrule

	\vspace{.6cm}

	\questao{} Considere a seguinte rela\c{c}\~ao sobre $\z$:
	\[
		x R y\quad  \mbox{quando}\quad x^3 - y^3 = 5l, \quad \mbox{para algum}\quad l \in \z
	\]
	\begin{enumerate}[label={\arabic*})]
		\item Mostre que $R$ \'e uma rela\c{c}\~ao de equival\^encia sobre $\z$.
		\item Encontre todas as classes de equival\^encia dessa rela\c{c}\~ao.
		\item A qual classe de equival\^encia sua matr{\'\i}cula pertence?
	\end{enumerate}

	\noindent\textbf{Solu\c{c}\~ao:}

	\begin{enumerate}[label={\arabic*})]
		\item Seja $x \in \z$. Temos
		\[
			x^3 - x^3 = 0 = 5\cdot 0.
		\]
		Logo $xRx$.

		Suponha que $xRy$. Assim existe $l \in \z$ tal que
		\[
			x^3 - y^3 = 5l.
		\]
		Agora,
		\[
			y^3 - x^3 = -(x^3 - y^3) = -5l = 5(-l).
		\]
		Assim $yRx$.

		Finalmente, se $xRy$ e $yRz$, ent\~ao existem $k$, $l \in \z$ tais que
		\[
			x^3 - y^3 = 5k \quad \mbox{e} \quad y^3 - z^3 = 5l.
		\]
		Da{\'\i}
		\[
			x^3 - y^3 = (x^3 - y^3) + (y^3 - z^3) = 5k + 5l = 5(k + l),
		\]
		ou seja, $xRz$.

		Portanto $R$ \'e uma rela\c{c}\~ao de equival\^encia em $\z$.

		\item Seja $a \in \z$. Ent\~ao
		\begin{align*}
			\overline{a} &= \{x \in \z \mid xRa\} \\
			\overline{a} &= \{x \in \z \mid x^3 - a^3 = 5l, l \in \z\}\\
			\overline{a} &= \{x \in \z \mid x^3 = 5l + a^3, l \in \z\}
		\end{align*}
		Note que nesse caso estamos efetuando a divis\~ao inteira de $x^3$ por 5 e o resto \'e dado por $a^3$.
		Agora,
		\begin{align*}
			0^3 &= 0\\
			1^3 &= 1\\
			2^3 &= 8 = 5 + 3\\
			3^3 &= 27 = 25 + 2\\
			4^3 &= 64 = 60 + 4
		\end{align*}
		Assim,
		temos as classes
		\begin{align*}
			\overline{0} &= \{x \in \z \mid x^3 = 5l, l \in \z\}\\
			\overline{1} &= \{x \in \z \mid x^3 = 5l + 1, l \in \z\}\\
			\overline{2} &= \{x \in \z \mid x^3 = 5l + 3, l \in \z\}\\
			\overline{3} &= \{x \in \z \mid x^3 = 5l + 2, l \in \z\}\\
			\overline{4} &= \{x \in \z \mid x^3 = 5l + 4, l \in \z\}.\\
		\end{align*}

		Portanto temos exatamente 5 classes de equival\^encia para a rela\c{c}\~ao $R$.

		\item Seja $x$ a sua matr{\'\i}cula. Para encontrar a qual classe $x$ pertence, basta calcular $x^3$ e efetuar a divis\~ao inteira por 5. O resto dessa divis\~ao determinar\'a a qual classe sua matr{\'\i}cula pertence.
	\end{enumerate}

	\questao{} Considere a seguinte rela\c{c}\~ao sobre $\z$:
	\[
		x S y\quad  \mbox{quando}\quad y^3 - x^3 = 5t, \quad \mbox{para algum}\quad t \in \z
	\]
	\begin{enumerate}[label={\arabic*})]
		\item Mostre que $S$ \'e uma rela\c{c}\~ao de equival\^encia sobre $\z$).
		\item Encontre todas as classes de equival\^encia dessa rela\c{c}\~ao.
		\item A qual classe de equival\^encia sua matr{\'\i}cula pertence?
	\end{enumerate}

	\noindent\textbf{Solu\c{c}\~ao:} Basta repetir os passos do Exerc{\'\i}cio 1, fazendo as substitui\c{c}\~oes necess\'arias.
\end{document}