%!TEX program = xelatex
%!TEX encoding = ISO-8859-1
\documentclass[12pt]{article}

\usepackage{amssymb}
\usepackage{amsmath,amsfonts,amsthm,amstext,mathabx}
\usepackage[brazil]{babel}
%\usepackage[latin1]{inputenc}
\usepackage{graphicx}
\graphicspath{{/home/jfreitas/Dropbox/imagens-latex/}{/Volumes/Vader/Dropbox/imagens-latex/}{D:/Dropbox/imagens-latex/}}
\usepackage{enumitem}
\usepackage{multicol}
\usepackage[all]{xy}

\setlength{\topmargin}{-1.0in}
\setlength{\oddsidemargin}{0in}
\setlength{\textheight}{10.1in}
\setlength{\textwidth}{6.5in}
\setlength{\baselineskip}{12mm}

\newcounter{exercicios}
\setcounter{exercicios}{0}
\newcommand{\questao}{
\addtocounter{exercicios}{1}
\noindent{\bf Exerc{\'\i}cio \arabic{exercicios}: }}

\newcommand{\equi}{\Leftrightarrow}
\newcommand{\bic}{\leftrightarrow}
\newcommand{\cond}{\rightarrow}
\newcommand{\impl}{\Rightarrow}
\newcommand{\nao}{\sim}
\newcommand{\sub}{\subseteq}
\newcommand{\e}{\ \wedge\ }
\newcommand{\ou}{\ \vee\ }
\newcommand{\vaz}{\emptyset}
\newcommand{\nsub}{\nsubset}
\renewcommand{\sin}{{\rm sen\,}}

\newcommand{\n}{\mathbb{N}}
\newcommand{\z}{\mathbb{Z}}
\newcommand{\real}{\mathbb{R}}
\newcommand{\vesp}{\vspace{0.2cm}}
\newcommand{\subne}{\subsetneqq}


\newcommand{\compcent}[1]{\vcenter{\hbox{$#1\circ$}}}
\newcommand{\comp}{\mathbin{\mathchoice
{\compcent\scriptstyle}{\compcent\scriptstyle}
{\compcent\scriptscriptstyle}{\compcent\scriptscriptstyle}}}

\begin{document}

\pagestyle{empty}

\begin{figure}[h]
        \begin{minipage}[c]{1.7cm}
        \includegraphics[width=1.7cm]{unb.pdf}
        \end{minipage}%
        \hspace{0pt}
        \begin{minipage}[c]{4in}
          {Universidade de Brasília} \\
          {Departamento de Matemática}
\end{minipage}
\end{figure}
\vspace{-1cm}\hrule


\begin{center}
{\Large\bf {\'A}lgebra 1 - Turma C -- 1$^{o}$/2018} \\ \vspace{9pt} {\large\bf
  $3^{\underline{o}}$ Teste - Resolu\c{c}\~ao}\\
\vspace{9pt} Prof. Jos{\'e} Ant{\^o}nio O. Freitas
\end{center}
\hrule

\vspace{.6cm}

\questao Para mostrar que $R$ é relação de equivalência sobre $A$ precisamos mostrar que
\begin{enumerate}
	\item Para todo $x \in A$, $xRx$.
	\item Se $xRy$, então $yRx$.
	\item Se $xRy$ e $yRz$, então $XRz$.
\end{enumerate}

Como $A = \n \times \n^*$, então os elementos que aparecem na definição de relação de equivalência serão pares ordenados. Nesse caso particular precisamos verificar que:
\begin{enumerate}
	\item Para todo $x = (a,b) \in A$, $(a,b)R(a,b)$.
	\item Para $x = (a,b)$ e $y = (c,d)$, se $(a,b)R(c,d)$, então $(c,d)R(a,b)$.
	\item Para $x = (a,b)$, $y = (c,d)$ e $z = (e,f)$ se $(a,b)R(c,d)$ e $(c,d)R(e,f)$, então $(a,b)R(e,f)$.
\end{enumerate}

De fato,
\begin{enumerate}
	\item Para $(a,b) \in A$ $a + b = a + b$, logo $(a,b)R(a,b)$.
	\item Se $(a,b)R(c,d)$ então, por definição $a + b = c + d$. Ou seja $c + d = a + b$ e com isso $(c,d)R(a,b)$.
	\item se $(a,b)R(c,d)$ e $(c,d)R(e,f)$ então, por definição $a + b = c + d$  e $c + d = e + f$. Logo $a + b = e + f$, isto é, $(a,b)R(e,f)$.
\end{enumerate}

Portanto, $R$ é uma relação de equivalência sobre $A$ como queríamos.

\[
	(a,b)R(c,d) \mbox{ se, e só se, } a + b = c + d.
\]
Mostre que $R$ é uma relação de equivalência sobre $A$.

\vspace{.5cm}

\questao Considere a seguinte relação sobre $\z$
\[
	xRy \mbox{ quando } x^2 - y^2 = 4k, \mbox{para algum} k \in \z.
\]

\begin{enumerate}[label={\alph*})]
	\item Mostre que $R$ é uma rela{\c c}{\~a}o de equival{\^e}ncia sobre $\z$.
	\item Encontre $\overline{0}$, $\overline{1}$ e $\overline{2}$.
\end{enumerate}

\noindent\textbf{Solu\c{c}\~ao:} De fato,
	\begin{enumerate}[label={\alph*})]
		\item Temos
		\begin{enumerate}[label={\roman*})]
			\item Para todo $x \in \x$, $x^2 - y^2 = 0 = 4\cdot 0$. Logo $xRx$.
			\item Se $xRy$, então $x^2 - y^2 = 4k$. Assim $y^2 - x^2 = -(x^2 - y^2) = -4k = 4(-k)$, isto é, $yRx$.
			\item Se $xRy$ e $yRz$ então
			\begin{align}
				x^2 - y^2 &= 4k\label{primeira_equacao}\\
				y^2 - z^2 &= 4l\label{segunda_equacao}
			\end{align}
			para $k$, $l \in \z$.
			Somando as duas equações obtemos
			\begin{align}\label{terceira_equacao}
				x^2 - y^2 + y^2 - z^2 &= 4k + 4l\\
				x^2 - z^2 &= 4(k + l)
			\end{align}
			Assim $xRz$.
		\end{enumerate}

		Portanto $R$ é uma relação de equivalência sobre $\z$.

		\item Temos:
		\begin{align*}
			\overline{0} &= \{x \in \z \mid xR0\} = \{x \in \z \mid x^2 - 0^2 = 4k,\ k \in \z\} \\
			&= \{x \in \z \mid x^2 = 4k, \ k \in \z\} = \{\dots, -4,-2,0,2,4, \dots\}\\
			\overline{1} &= \{x \in \z \mid xR1\} = \{x \in \z \mid x^2 - 1^2 = 4k,\ k \in \z\} \\
			&= \{x \in \z \mid x^2 = 4k + 1, \ k \in \z\} = \{\dots, -3,-1,1,3, \dots\}\\
			\overline{2} &= \{x \in \z \mid xR2\} = \{x \in \z \mid x^2 - 2^2 = 4k,\ k \in \z\} \\
			&= \{x \in \z \mid x^2 = 4k + 4, \ k \in \z\} = \{x \in \z \mid x^2 = 4(k + 1), \ k \in \z\} = \overline{0}.
		\end{align*}
	\end{enumerate}
\end{document}