%!TEX program = xelatex
%!TEX encoding = ISO-8859-1
\documentclass[12pt]{article}

\usepackage{amssymb}
\usepackage{amsmath,amsfonts,amsthm,amstext,mathabx}
\usepackage[brazil]{babel}
%\usepackage[latin1]{inputenc}
\usepackage{graphicx}
\graphicspath{{/home/jfreitas/Dropbox/imagens-latex/}{/Volumes/Vader/Dropbox/imagens-latex/}{D:/Dropbox/imagens-latex/}}
\usepackage{enumitem}
\usepackage{multicol}
\usepackage[all]{xy}

\setlength{\topmargin}{-1.0in}
\setlength{\oddsidemargin}{0in}
\setlength{\textheight}{10.1in}
\setlength{\textwidth}{6.5in}
\setlength{\baselineskip}{12mm}

\newcounter{exercicios}
\setcounter{exercicios}{0}
\newcommand{\questao}{
\addtocounter{exercicios}{1}
\noindent{\bf Exerc{\'\i}cio \arabic{exercicios}: }}

\newcommand{\equi}{\Leftrightarrow}
\newcommand{\bic}{\leftrightarrow}
\newcommand{\cond}{\rightarrow}
\newcommand{\impl}{\Rightarrow}
\newcommand{\nao}{\sim}
\newcommand{\sub}{\subseteq}
\newcommand{\e}{\ \wedge\ }
\newcommand{\ou}{\ \vee\ }
\newcommand{\vaz}{\emptyset}
\newcommand{\nsub}{\nsubset}
\renewcommand{\sin}{{\rm sen\,}}

\newcommand{\n}{\mathbb{N}}
\newcommand{\z}{\mathbb{Z}}
\newcommand{\real}{\mathbb{R}}
\newcommand{\vesp}{\vspace{0.2cm}}
\newcommand{\subne}{\subsetneqq}


\newcommand{\compcent}[1]{\vcenter{\hbox{$#1\circ$}}}
\newcommand{\comp}{\mathbin{\mathchoice
{\compcent\scriptstyle}{\compcent\scriptstyle}
{\compcent\scriptscriptstyle}{\compcent\scriptscriptstyle}}}

\begin{document}

\pagestyle{empty}

\begin{figure}[h]
        \begin{minipage}[c]{1.7cm}
        \includegraphics[width=1.7cm]{unb.pdf}
        \end{minipage}%
        \hspace{0pt}
        \begin{minipage}[c]{4in}
          {Universidade de Brasília} \\
          {Departamento de Matemática}
\end{minipage}
\end{figure}
\vspace{-1cm}\hrule


\begin{center}
{\Large\bf {\'A}lgebra 1 - Turma C -- 1$^{o}$/2018} \\ \vspace{9pt} {\large\bf
  $3^{\underline{o}}$ Teste - Resolu\c{c}\~ao}\\
\vspace{9pt} Prof. Jos{\'e} Ant{\^o}nio O. Freitas
\end{center}
\hrule

\vspace{.6cm}

\questao Seja $A = \n \times \n^*$ e considere a relação sobre $A$ dada por
\[
	(a,b)R(c,d) \mbox{ se, e só se, } a + b = c + d.
\]
Mostre que $R$ é uma relação de equivalência sobre $A$.

\noindent\textbf{Solu\c{c}\~ao:} Temos
\begin{enumerate}[label={\roman*})]
	\item $xRx$ pois $x - x = 0 \in \z$.
	\item Se $xRy$, então $x - y = k$, com $k \in \z$. Assim $y - x = -(x - y) = -k \in \z$. Logo $yRx$.
	\item Se $xRy$ e $yRz$, então $x - y = k$ e $y - z = l$, com $k$, $l \in \z$. Assim $x - z = (x - y) + (y - z) = k + l \in \z$. Logo $xRz$.
\end{enumerate}

Portanto $R$ é uma relação de equivalência sobre $\q$.


\vspace{.5cm}

\questao Seja $A = \z \times \z^*$, onde $\z^* = \z - \{0\}$. Para $(a,b)$, $(c,d) \in A$ defina
\[
	(a,b)R(c,d) \mbox{ quando } ad=bc.
\]

\begin{enumerate}[label={\alph*})]
	\item Mostre que $R$ é uma rela{\c c}{\~a}o de equival{\^e}ncia sobre $A$.
	\item Encontre a classe de equivalência $\overline{(2,1)}$ e exiba 5 elementos.
\end{enumerate}

\noindent\textbf{Solu\c{c}\~ao:} De fato,
	\begin{enumerate}[label={\alph*})]
		\item Primeiro observe que
		\[
			R = \{((a,b);(c,d)) \in A \times A \mid ad = bc\}
		\]
		Assim um elemento $x \in A = \z\times\z^*$ é da forma $x = (a,b)$.
		\begin{enumerate}[label={\roman*})]
			\item Para $(a,b) \in A$, $ab = ab$ logo $(a,b)R(a,b)$.
			\item Se $(a,b)R(c,d)$, então $ad = bc$. Assim $bc = ad$, isto é, $(c,d)R(a,b)$.
			\item Se $(a,b)R(c,d)$ e $(c,d)R(e,f)$ então
			\begin{align}
				ad &= bc\label{primeira_equacao}\\
				cf &= de.\label{segunda_equacao}
			\end{align}
			Multiplicando a equação \eqref{primeira_equacao} por $f$ obtemos
			\begin{align}\label{terceira_equacao}
				adf = bcf.
			\end{align}
			Substituindo o valor de $cf$ da equação \eqref{segunda_equacao} na equação \eqref{terceira_equacao} obtemos
			\begin{align*}
				adf &= bde\\
				adf - bde &= 0\\
				d(af - be) &= 0.
			\end{align*}
			Como $d \in \z^*$, $d \ne 0$, logo $af - be = 0$. Assim $af = be$ e com isso $(a,b)R(e,f)$.
		\end{enumerate}

		Portanto $R$ é uma relação de equivalência sobre $A$.

		\item A classe de equivalência de $\overline{(2,1)}$ é dada por:
		\begin{align*}
			\overline{(2,1)} &= \{(x,y) \in A \mid (x,y)R(2,1)\} = \{(x,y) \in A \mid x = 2y\} \\
			&= \{\cdots; (2,1); (4,2); (-2,-1);(-4,-2);(6,3);\cdots\}.
		\end{align*}
	\end{enumerate}
\end{document}