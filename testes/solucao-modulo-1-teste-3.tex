%!TEX program = xelatex
%!TEX encoding = ISO-8859-1
%!TEX program = xelatex
% !TEX encoding = ISO-8859-1
\def\ano{2019}
\def\semestre{2}
\def\disciplina{\'Algebra 1}
\def\turma{C}

\documentclass[12pt]{exam}

\usepackage{caption}
\usepackage{amssymb}
\usepackage{amsmath,amsfonts,amsthm,amstext}
\usepackage[brazil]{babel}
% \usepackage[latin1]{inputenc}
\usepackage{graphicx}
\graphicspath{{/ArquivosLinux/OneDrive/imagens-latex/}{D:/OneDrive - unb.br/imagens-latex/}}
\usepackage{enumitem}
\usepackage{multicol}
\usepackage{answers}
\usepackage{tikz,ifthen}
\usetikzlibrary{lindenmayersystems}
\usetikzlibrary[shadings]
\Newassociation{solucao}{Solution}{ans}
\newtheorem{exercicio}{}

\setlength{\topmargin}{-1.0in}
\setlength{\oddsidemargin}{0in}
\setlength{\textheight}{10.1in}
\setlength{\textwidth}{6.5in}
\setlength{\baselineskip}{12mm}

\extraheadheight{0.7in}
\firstpageheadrule
\runningheadrule
\lhead{
        \begin{minipage}[c]{1.7cm}
        \includegraphics[width=1.7cm]{unb.pdf}
        \end{minipage}%
        \hspace{0pt}
        \begin{minipage}[c]{4in}
          {Universidade de Brasília} --
          {Departamento de Matemática}
\end{minipage}
\vspace*{-0.8cm}
}
% \chead{Universidade de Brasília - Departamento de Matemática}
% \rhead{}
% \vspace*{-2cm}

\extrafootheight{.5in}
\footrule
\lfoot{\disciplina\ - \semestre$^o$/\ano\ - Módulo \numeromodulo}
\cfoot{}
\rfoot{Página \thepage\ de \numpages}

\newcounter{exercicios}
\renewcommand{\theexercicios}{\arabic{exercicios}}

\newenvironment{questao}[1]{
\refstepcounter{exercicios}
\ifx&#1&
\else
   \label{#1}
\fi
\noindent\textbf{Exercício {\theexercicios}:}
}

\newcommand{\resp}[1]{
\noindent{\bf Exercício #1: }}

\def\ano{2024}
\def\semestre{1}
\def\disciplina{Álgebra 1}
\def\nomeabreviado{Álgebra 1}
\def\turma{1}

\newcommand{\im}{{\rm Im\,}}
\newcommand{\dlim}[2]{\displaystyle\lim_{#1\rightarrow #2}}
\newcommand{\minf}{+\infty}
\newcommand{\ninf}{-\infty}
\newcommand{\cp}[1]{\mathbb{#1}}
\newcommand{\sub}{\subseteq}
\newcommand{\n}{\mathbb{N}}
\newcommand{\z}{\mathbb{Z}}
\newcommand{\rac}{\mathbb{Q}}
\newcommand{\real}{\mathbb{R}}
\newcommand{\complex}{\mathbb{C}}

\newcommand{\vesp}[1]{\vspace{ #1  cm}}

\newcommand{\compcent}[1]{\vcenter{\hbox{$#1\circ$}}}
\newcommand{\comp}{\mathbin{\mathchoice
        {\compcent\scriptstyle}{\compcent\scriptstyle}
        {\compcent\scriptscriptstyle}{\compcent\scriptscriptstyle}}}
\renewcommand{\sin}{{\rm sen\,}}
\renewcommand{\tan}{{\rm tg\,}}
\renewcommand{\csc}{{\rm cossec\,}}
\renewcommand{\cot}{{\rm cotg\,}}
\renewcommand{\sinh}{{\rm senh\,}}

\begin{document}

\begin{center}
{\Large\bf \disciplina\ - Turma \turma\ -- \semestre$^{o}$/\ano} \\ \vspace{9pt} {\large\bf
$3^{\underline{o}}$ Teste - M\'odulo 1 - Resolu\c{c}\~ao}\\
\vspace{9pt} Prof. Jos{\'e} Ant{\^o}nio O. Freitas
\end{center}
\hrule

\vspace{.6cm}

\questao Seja $m \in \z$, $m > 1$. Defina $R \sub \z \times \z$ por
\[
	R = \{ (x, y) \in \z \times \z \mid x - y = km,\ k \in \z\}.
\]

Mostre que $R$ \'e uma rela\c{c}\~ao de equival\^encia sobre $\z$.

\noindent\textbf{Solu\c{c}\~ao:} Para mostrar que $R$ \'e rela\c{c}\~ao de equival\^encia sobre um conjunto $A \ne \vaz$ precisamos mostrar que:
\begin{enumerate}[label={\roman*})]
	\item Para todo $x \in A$, $(x, x) \in R$.
	\item Se $(x, y) \in R$, ent\~ao $(y, x) \in R$.
	\item Se $(x, y) \in R$ e $(y, z) \in R$, ent\~ao $(x, z) \in R$.
\end{enumerate}

De fato,
\begin{enumerate}[label={\roman*})]
	\item Para $x \in \z$ temos $x - x = 0 = m\cdot 0$, logo $(x, x) \in R$.
	\item Se $(x, y)\in R$ ent\~ao, para algum $k \in \z$ temos $x - y  = mk$. Assim
	\begin{align*}
		y - x = -(x - y) = -(mm) = m(-k)
	\end{align*}
	logo $(y, x) \in R$.
	\item Se $(x, y) \in R$ e $(y, z) \in R$ ent\~ao, existem $k$, $l \in \z$ tais que
	\begin{align*}
		x - y &= km\\
		y - z = lm.
	\end{align*}
	Assim
	\begin{align*}
		x - z = (x - y) + (y - z) = km + lm = m(k + l)
	\end{align*}
	e então $(x, z) \in R$.
\end{enumerate}

Portanto, $R$ \'e uma rela\c{c}\~ao de equival\^encia sobre $\z$ como quer{\'\i}amos.

\vspace{.5cm}

\questao Considere a seguinte rela\c{c}\~ao sobre $\z$
\[
	xRy \mbox{ quando } x^2 - y^2 = 4k, \mbox{ para algum } k \in \z.
\]

\begin{enumerate}[label={\alph*})]
	\item Mostre que $R$ \'e uma rela{\c c}{\~a}o de equival{\^e}ncia sobre $\z$.
	\item Encontre $\overline{0}$, $\overline{1}$ e $\overline{2}$.
\end{enumerate}

\noindent\textbf{Solu\c{c}\~ao:}
	\begin{enumerate}[label={\alph*})]
		\item De fato,
		\begin{enumerate}[label={\roman*})]
			\item Para todo $x \in \z$, $x^2 - y^2 = 0 = 4\cdot 0$. Logo $xRx$.
			\item Se $xRy$, ent\~ao $x^2 - y^2 = 4k$. Assim $y^2 - x^2 = -(x^2 - y^2) = -4k = 4(-k)$, isto \'e, $yRx$.
			\item Se $xRy$ e $yRz$ ent\~ao
			\begin{align*}
				x^2 - y^2 &= 4k\\
				y^2 - z^2 &= 4l
			\end{align*}
			para $k$, $l \in \z$.
			Somando as duas equa\c{c}\~oes obtemos
			\begin{align*}\label{terceira_equacao}
				x^2 - y^2 + y^2 - z^2 &= 4k + 4l\\
				x^2 - z^2 &= 4(k + l)
			\end{align*}
			Assim $xRz$.
		\end{enumerate}

		Portanto $R$ \'e uma rela\c{c}\~ao de equival\^encia sobre $\z$.

		\item Temos:
		\begin{align*}
			\overline{0} &= \{x \in \z \mid xR0\} = \{x \in \z \mid x^2 - 0^2 = 4k,\ k \in \z\} \\
			&= \{x \in \z \mid x^2 = 4k, \ k \in \z\} = \{\dots, -4,-2,0,2,4, \dots\}\\
			\overline{1} &= \{x \in \z \mid xR1\} = \{x \in \z \mid x^2 - 1^2 = 4k,\ k \in \z\} \\
			&= \{x \in \z \mid x^2 = 4k + 1, \ k \in \z\} = \{\dots, -3,-1,1,3, \dots\}\\
			\overline{2} &= \{x \in \z \mid xR2\} = \{x \in \z \mid x^2 - 2^2 = 4k,\ k \in \z\} \\
			&= \{x \in \z \mid x^2 = 4k + 4, \ k \in \z\} = \{x \in \z \mid x^2 = 4(k + 1), \ k \in \z\} = \overline{0}.
		\end{align*}
	\end{enumerate}
\end{document}