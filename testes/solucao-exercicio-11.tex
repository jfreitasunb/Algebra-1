%!TEX program = xelatex
% !TEX encoding = ISO-8859-1
\def\ano{2019}
\def\semestre{2}
\def\disciplina{\'Algebra 1}
\def\turma{C}

\documentclass[12pt]{exam}

\usepackage{caption}
\usepackage{amssymb}
\usepackage{amsmath,amsfonts,amsthm,amstext}
\usepackage[brazil]{babel}
% \usepackage[latin1]{inputenc}
\usepackage{graphicx}
\graphicspath{{/home/jfreitas/GitHub_Repos/Algebra-1/Pictures/}{D:/OneDrive - unb.br/imagens-latex/}}
\usepackage{enumitem}
\usepackage{multicol}
\usepackage{answers}
\usepackage{tikz,ifthen}
\usetikzlibrary{lindenmayersystems}
\usetikzlibrary[shadings]
\newcommand{\nsub}{\varsubsetneq}
\newcommand{\vaz}{\emptyset}
\newcommand{\im}{{\rm Im\,}}
\newcommand{\sub}{\subseteq}
\newcommand{\n}{\mathbb{N}}
\newcommand{\z}{\mathbb{Z}}
\newcommand{\rac}{\mathbb{Q}}
\newcommand{\real}{\mathbb{R}}
\newcommand{\complex}{\mathbb{C}}
\newcommand{\cp}[1]{\mathbb{#1}}
\newcommand{\ch}{\mbox{\textrm{car\,}}\nobreak}
\newcommand{\vesp}[1]{\vspace{ #1  cm}}
\newcommand{\compcent}[1]{\vcenter{\hbox{$#1\circ$}}}
\newcommand{\comp}{\mathbin{\mathchoice
{\compcent\scriptstyle}{\compcent\scriptstyle}
{\compcent\scriptscriptstyle}{\compcent\scriptscriptstyle}}}

\begin{document}

Seja $G = [a]$ um grupo c{\'\i}clico de ordem $s$ e $H = [b]$ um grupo c{\'\i}clico de ordem $t$. Monstre que se $sk$ \'e um m\'ultiplo de $t$, então $\phi : G \to H$, definida por $\phi(a) = b^k$ \'e um homomorfismo de grupos.

\textbf{Ideia de solução:} Nesse caso devemos assumir que $\phi(a^l) = b^{lk}$. Primeiro é preciso mostrar que $\phi$ está bem definida, ou seja, se
\[
    a^i = a^j
\]
para $i \ne j$ então
\[
    b^{ik} = b^{jk}.
\]

Como $G$ é um grupo finito, para que $a^i = a^j$ é preciso que $i \equiv j \pmod s$. Isto é $s | (i - j)$ e com isso podemos escrever $i = \beta s + j$ com $\beta \in \z$. Com isso
\begin{align*}
    b^{ki} = b^{k(\beta s + j)} = b^{\beta ks + kj}.
\end{align*}
Como $ks$ é um múltiplo de $t$, então $ks = \alpha t$ com $\alpha \in \z$. Logo
\[
    b^{ki} = b^{k(\beta s + j)} = b^{\beta ks + kj} = (b^{ks})^\beta(b^{kj}) = (b^{\alpha t})^\beta(b^{kj}) = b^{kj}.
\]
Assim $\phi(a^i) = \phi(a^j)$ e $\phi$ está bem definida.

Agora é só que $\phi$ é um homomorfismo.

\end{document}