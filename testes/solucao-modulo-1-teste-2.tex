%!TEX program = xelatex
% !TEX encoding = ISO-8859-1
\def\ano{2019}
\def\semestre{2}
\def\disciplina{\'Algebra 1}
\def\turma{C}

\documentclass[12pt]{exam}

\usepackage{caption}
\usepackage{amssymb}
\usepackage{amsmath,amsfonts,amsthm,amstext}
\usepackage[brazil]{babel}
% \usepackage[latin1]{inputenc}
\usepackage{graphicx}
\graphicspath{{/ArquivosLinux/OneDrive/imagens-latex/}{D:/OneDrive - unb.br/imagens-latex/}}
\usepackage{enumitem}
\usepackage{multicol}
\usepackage{answers}
\usepackage{tikz,ifthen}
\usetikzlibrary{lindenmayersystems}
\usetikzlibrary[shadings]
\Newassociation{solucao}{Solution}{ans}
\newtheorem{exercicio}{}

\setlength{\topmargin}{-1.0in}
\setlength{\oddsidemargin}{0in}
\setlength{\textheight}{10.1in}
\setlength{\textwidth}{6.5in}
\setlength{\baselineskip}{12mm}

\extraheadheight{0.7in}
\firstpageheadrule
\runningheadrule
\lhead{
        \begin{minipage}[c]{1.7cm}
        \includegraphics[width=1.7cm]{unb.pdf}
        \end{minipage}%
        \hspace{0pt}
        \begin{minipage}[c]{4in}
          {Universidade de Brasília} --
          {Departamento de Matemática}
\end{minipage}
\vspace*{-0.8cm}
}
% \chead{Universidade de Brasília - Departamento de Matemática}
% \rhead{}
% \vspace*{-2cm}

\extrafootheight{.5in}
\footrule
\lfoot{\disciplina\ - \semestre$^o$/\ano\ - Módulo \numeromodulo}
\cfoot{}
\rfoot{Página \thepage\ de \numpages}

\newcounter{exercicios}
\renewcommand{\theexercicios}{\arabic{exercicios}}

\newenvironment{questao}[1]{
\refstepcounter{exercicios}
\ifx&#1&
\else
   \label{#1}
\fi
\noindent\textbf{Exercício {\theexercicios}:}
}

\newcommand{\resp}[1]{
\noindent{\bf Exercício #1: }}

\def\ano{2024}
\def\semestre{1}
\def\disciplina{Álgebra 1}
\def\nomeabreviado{Álgebra 1}
\def\turma{1}

\newcommand{\im}{{\rm Im\,}}
\newcommand{\dlim}[2]{\displaystyle\lim_{#1\rightarrow #2}}
\newcommand{\minf}{+\infty}
\newcommand{\ninf}{-\infty}
\newcommand{\cp}[1]{\mathbb{#1}}
\newcommand{\sub}{\subseteq}
\newcommand{\n}{\mathbb{N}}
\newcommand{\z}{\mathbb{Z}}
\newcommand{\rac}{\mathbb{Q}}
\newcommand{\real}{\mathbb{R}}
\newcommand{\complex}{\mathbb{C}}

\newcommand{\vesp}[1]{\vspace{ #1  cm}}

\newcommand{\compcent}[1]{\vcenter{\hbox{$#1\circ$}}}
\newcommand{\comp}{\mathbin{\mathchoice
        {\compcent\scriptstyle}{\compcent\scriptstyle}
        {\compcent\scriptscriptstyle}{\compcent\scriptscriptstyle}}}
\renewcommand{\sin}{{\rm sen\,}}
\renewcommand{\tan}{{\rm tg\,}}
\renewcommand{\csc}{{\rm cossec\,}}
\renewcommand{\cot}{{\rm cotg\,}}
\renewcommand{\sinh}{{\rm senh\,}}

\begin{document}

\begin{center}
{\Large\bf \disciplina\ - Turma \turma\ -- \semestre$^{o}$/\ano} \\ \vspace{9pt} {\large\bf
$2^{\underline{o}}$ Teste - M\'odulo 1 - Resolu\c{c}\~ao}\\
\vspace{9pt} Prof. Jos{\'e} Ant{\^o}nio O. Freitas
\end{center}
\hrule

\vspace{.6cm}

\questao Suponha que $A \sub E$, $B \sub E$. Se $A \cap B = \vaz$, ent\~ao $A \cup C_E(B) = C_E(B)$.

\noindent\textbf{Solu\c{c}\~ao:} Precisamos mostrar que
\begin{enumerate}[label={\roman*})]
    \item $A \cup C_E(B) \sub C_E(B)$
    \item $C_E(B) \sub A \cup C_E(B)$
\end{enumerate}

Para a primeira inclus\~ao seja $x \in A \cup C_E(B)$. Ent\~ao $x \in A$ ou $x \in C_E(B)$. Caso $x \in A$, como $A \cap B = \vaz$, devemos ter $x \notin B$, isto \'e, $x \in C_E(B)$. Caso $x \in C_E(B)$, ent\~ao temos o que quer{\'\i}amos. Logo $A \cup C_E(B) \sub C_E(B)$.

A segunda inclus\~ao \'e imediata, pois todo conjunto est\'a contido nele mesmo uni\~ao com qualquer outro.

Portanto $A \cup C_E(B) = C_E(B)$.

\vspace{.5cm}

\questao Mostre que $A - (A - B) = A \cap B$.

\noindent\textbf{Solu\c{c}\~ao:} Precisamos mostrar que:
\begin{enumerate}[label={\roman*})]
    \item $A - (A - B) \sub A \cap B$
    \item $A \cap B \sub A - (A - B)$
\end{enumerate}

Para a primeira inclus\~ao seja $x \in A - (A - B)$. Assim $x \in A$ e $x \notin A - B$. Mas $x \notin A - B$ significa que $x \notin A$ ou $x \in B$. Mas como $x \in A$, ent\~ao $x \in B$ e com isso $x \in A \cap B$. Logo $A - (A - B) \sub A \cap B$.

Agora para a segunda inclus\~ao seja $y \in A \cap B$. Da{\'\i} $y \in A$ e $y \in B$. Mas se $y \in B$, ent\~ao $y \notin A - B$. Como $y \in A$ e $y \notin A - B$, ent\~ao $y \in A - (A - B)$. Logo $A \cap B \sub A - (A - B)$.

Portanto, $A - (A - B) = A \cap B$.
\end{document}