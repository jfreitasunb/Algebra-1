%!TEX program = xelatex
%!TEX encoding = ISO-8859-1
\documentclass[12pt]{article}

\usepackage{amssymb}
\usepackage{amsmath,amsfonts,amsthm,amstext,mathabx}
\usepackage[brazil]{babel}
%\usepackage[latin1]{inputenc}
\usepackage{graphicx}
\graphicspath{{/home/jfreitas/Dropbox/imagens-latex/}{/Volumes/Vader/Dropbox/imagens-latex/}{D:/Dropbox/imagens-latex/}}
\usepackage{enumitem}
\usepackage{multicol}
\usepackage[all]{xy}

\setlength{\topmargin}{-1.0in}
\setlength{\oddsidemargin}{0in}
\setlength{\textheight}{10.1in}
\setlength{\textwidth}{6.5in}
\setlength{\baselineskip}{12mm}

\newcounter{exercicios}
\setcounter{exercicios}{0}
\newcommand{\questao}{
\addtocounter{exercicios}{1}
\noindent{\bf Exerc{\'\i}cio \arabic{exercicios}: }}

\newcommand{\equi}{\Leftrightarrow}
\newcommand{\bic}{\leftrightarrow}
\newcommand{\cond}{\rightarrow}
\newcommand{\impl}{\Rightarrow}
\newcommand{\nao}{\sim}
\newcommand{\sub}{\subseteq}
\newcommand{\e}{\ \wedge\ }
\newcommand{\ou}{\ \vee\ }
\newcommand{\vaz}{\emptyset}
\newcommand{\nsub}{\nsubset}
\renewcommand{\sin}{{\rm sen\,}}

\newcommand{\n}{\mathbb{N}}
\newcommand{\z}{\mathbb{Z}}
\newcommand{\real}{\mathbb{R}}
\newcommand{\vesp}{\vspace{0.2cm}}
\newcommand{\subne}{\subsetneqq}


\newcommand{\compcent}[1]{\vcenter{\hbox{$#1\circ$}}}
\newcommand{\comp}{\mathbin{\mathchoice
{\compcent\scriptstyle}{\compcent\scriptstyle}
{\compcent\scriptscriptstyle}{\compcent\scriptscriptstyle}}}

\begin{document}

\pagestyle{empty}

\begin{figure}[h]
        \begin{minipage}[c]{1.7cm}
        \includegraphics[width=1.7cm]{unb.pdf}
        \end{minipage}%
        \hspace{0pt}
        \begin{minipage}[c]{4in}
          {Universidade de Brasília} \\
          {Departamento de Matemática}
\end{minipage}
\end{figure}
\vspace{-1cm}\hrule


\begin{center}
{\Large\bf {\'A}lgebra 1 - Turma C -- 1$^{o}$/2018} \\ \vspace{9pt} {\large\bf
  $2^{\underline{o}}$ Teste - Resolu\c{c}\~ao}\\
\vspace{9pt} Prof. Jos{\'e} Ant{\^o}nio O. Freitas
\end{center}
\hrule

\vspace{.6cm}

\questao Sejam $A$, $B$ e $C$ conjuntos. Mostre que $(A - B) - C = A - (B \cup C)$.

\noindent\textbf{Solu\c{c}\~ao:} Precisamos mostrar que:
\begin{enumerate}[label={\roman*})]
	\item $(A - B) - C \sub A - (B \cup C)$
	\item $A - (B \cup C) \sub (A - B) - C$
\end{enumerate}

Para a primeira inclusão seja $x \in (A - B) - C$. Assim, da definição de diferença de conjuntos, $x \in (A - B)$ e $x \notin C$. Novamente, pela definição de diferença de conjuntos, $x \in A$ e $x \notin B$. Como $x \notin B$ e $x \notin C$, então $x \notin (B \cup C)$. Daí $x \in A$ e $x \notin (B \cup C)$, ou seja, $x \in A - (B \cup C)$. Logo $(A - B) - C \sub A - (B \cup C)$.

Agora para a segunda inclusão seja $y \in A - (B \cup C)$. Daí, pela definição de diferença de conjuntos, $x \in A$ e $x \notin (B \cup C)$. Como $x \notin (B \cup C)$, então $x \notin B$ e $x \notin C$. Assim de $x \in A$ e $x \notin B$ segue que $x \in A - B$. Agora, $x \in A - B$ e $x \notin C$ logo $x \in (A - B) - C$. Logo $A - (B \cup C) \sub (A - B) - C$.

Portanto

\[
	(A - B) - C = A - (B \cup C),
\]
como queríamos.

\vspace{.5cm}

\questao Sejam $A = \z$, $m \in \z$, $m > 1$ fixo e $R = \{(x,y)\in \z \times \z \mid x - y = mk, \mbox{ para algum } k \in \z\}$. Mostre que $R$
é uma rela{\c c}{\~a}o de equival{\^e}ncia sobre $\z$.

\noindent\textbf{Solu\c{c}\~ao:} De fato,
	\begin{itemize}
		\item Para todo $x\in \z$ temos $x - x = m\cdot0$ e com isso $(x,x) \in R$.
		\item Se $(x,y) \in R$ então existe $k \in \z$ tal que $x - y = mk$. Agora $y - x = -(x - y) = -mk = m (-k)$ e como $-k \in \z$ segue que $(y,x) \in R$.
		\item Se $(x,y) \in R$ e $(y,z) \in R$, então existem $k$, $l\in \z$ tais que $x - y = mk$ e $y - z = ml$.
		Somando essas duas equações obtemos
		\begin{align*}
			(x - y) + (y - z) &= mk + ml\\
			x - z &= m(k + l)
		\end{align*}
		e como $k + l \in \z$ segue que $(x,z) \in \z$.
	\end{itemize}
	
	Assim $R$ é uma relação de equivalência.
\end{document}