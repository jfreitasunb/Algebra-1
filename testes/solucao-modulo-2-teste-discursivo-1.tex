%!TEX program = xelatex
%!TEX encoding = ISO-8859-1
\def\ano{2020}
\def\semestre{1}
\def\disciplina{\'Algebra 1}
\def\turma{C}

\documentclass[12pt]{exam}

\usepackage{caption}
\usepackage{amssymb}
\usepackage{amsmath,amsfonts,amsthm,amstext}
\usepackage[brazil]{babel}
% \usepackage[latin1]{inputenc}
\usepackage{graphicx}
\graphicspath{{/ArquivosLinux/OneDrive/imagens-latex/}{D:/OneDrive - unb.br/imagens-latex/}}
\usepackage{enumitem}
\usepackage{multicol}
\usepackage{answers}
\usepackage{tikz,ifthen}
\usetikzlibrary{lindenmayersystems}
\usetikzlibrary[shadings]
\Newassociation{solucao}{Solution}{ans}
\newtheorem{exercicio}{}

\setlength{\topmargin}{-1.0in}
\setlength{\oddsidemargin}{0in}
\setlength{\textheight}{10.1in}
\setlength{\textwidth}{6.5in}
\setlength{\baselineskip}{12mm}

\extraheadheight{0.7in}
\firstpageheadrule
\runningheadrule
\lhead{
        \begin{minipage}[c]{1.7cm}
        \includegraphics[width=1.7cm]{unb.pdf}
        \end{minipage}%
        \hspace{0pt}
        \begin{minipage}[c]{4in}
          {Universidade de Brasília} --
          {Departamento de Matemática}
\end{minipage}
\vspace*{-0.8cm}
}
% \chead{Universidade de Brasília - Departamento de Matemática}
% \rhead{}
% \vspace*{-2cm}

\extrafootheight{.5in}
\footrule
\lfoot{\disciplina\ - \semestre$^o$/\ano\ - Módulo \numeromodulo}
\cfoot{}
\rfoot{Página \thepage\ de \numpages}

\newcounter{exercicios}
\renewcommand{\theexercicios}{\arabic{exercicios}}

\newenvironment{questao}[1]{
\refstepcounter{exercicios}
\ifx&#1&
\else
   \label{#1}
\fi
\noindent\textbf{Exercício {\theexercicios}:}
}

\newcommand{\resp}[1]{
\noindent{\bf Exercício #1: }}

\def\ano{2024}
\def\semestre{1}
\def\disciplina{Álgebra 1}
\def\nomeabreviado{Álgebra 1}
\def\turma{1}

\newcommand{\im}{{\rm Im\,}}
\newcommand{\dlim}[2]{\displaystyle\lim_{#1\rightarrow #2}}
\newcommand{\minf}{+\infty}
\newcommand{\ninf}{-\infty}
\newcommand{\cp}[1]{\mathbb{#1}}
\newcommand{\sub}{\subseteq}
\newcommand{\n}{\mathbb{N}}
\newcommand{\z}{\mathbb{Z}}
\newcommand{\rac}{\mathbb{Q}}
\newcommand{\real}{\mathbb{R}}
\newcommand{\complex}{\mathbb{C}}

\newcommand{\vesp}[1]{\vspace{ #1  cm}}

\newcommand{\compcent}[1]{\vcenter{\hbox{$#1\circ$}}}
\newcommand{\comp}{\mathbin{\mathchoice
        {\compcent\scriptstyle}{\compcent\scriptstyle}
        {\compcent\scriptscriptstyle}{\compcent\scriptscriptstyle}}}
\renewcommand{\sin}{{\rm sen\,}}
\renewcommand{\tan}{{\rm tg\,}}
\renewcommand{\csc}{{\rm cossec\,}}
\renewcommand{\cot}{{\rm cotg\,}}
\renewcommand{\sinh}{{\rm senh\,}}

\begin{document}
	\begin{center}
	{\Large\bf \disciplina\ - Turma \turma\ -- \semestre$^{o}$/\ano} \\ \vspace{9pt} {\large\bf
	Teste Discursivo 1 - M\'odulo 2 - Resolu\c{c}\~ao}\\
	\vspace{9pt} Prof. Jos{\'e} Ant{\^o}nio O. Freitas
	\end{center}
	\hrule

	\vspace{.6cm}

	Seja $f : A \to B$ uma função e $P \subset A$ e $X \subset B$. Mostre que:
	\begin{enumerate}[label={\arabic*})]
		\item $f^{-1}(X^C) = [f^{-1}(X)]^C$

		\item $P \subset f^{-1}(f(P))$
	\end{enumerate}

	\noindent\textbf{Solu\c{c}\~ao:}

	\begin{enumerate}[label={\arabic*})]
		\item Precisamos mostrar que
		\begin{enumerate}
			\item $f^{-1}(X^C) \subset [f^{-1}(X)]^C$

			\item $[f^{-1}(X)]^C \subset f^{-1}(X^C)$
		\end{enumerate}

		Para mostrar \textit{(a)} seja $z \in f^{-1}(X^C)$. Assim $f(z) \in X^C$. Logo $f(z) \notin	X$. Com isso $z \notin f^{-1}(X)$ e então $z \in [f^{-1}(X)]^C$.

		Agora para mostrar \textit{(b)} seja $t \in [f^{-1}(X)]^C$. Daí $t \notin f^{-1}(X)$. Logo $f(t) \notin X$. Isto é, $f(t) \in X^C$ e então $t \in f^{-1}(X^C)$.

		Portanto $f^{-1}(X^C) = [f^{-1}(X)]^C$, como queríamos.

		\item Seja $t \in P$. Assim $f(t) \in f(P)$ e $f(P) \subset B$. Isto é, $t \in f^{-1}(f(P))$.

		Portanto $P \subset f^{-1}(f(P))$, como queríamos.
	\end{enumerate}
\end{document}