%!TEX program = xelatex
%!TEX encoding = ISO-8859-1
\documentclass[12pt]{article}

\usepackage{amssymb}
\usepackage{amsmath,amsfonts,amsthm,amstext,mathabx}
\usepackage[brazil]{babel}
%\usepackage[latin1]{inputenc}
\usepackage{graphicx}
\graphicspath{{/home/jfreitas/Dropbox/imagens-latex/}{/Volumes/Vader/Dropbox/imagens-latex/}{D:/Dropbox/imagens-latex/}}
\usepackage{enumitem}
\usepackage{multicol}
\usepackage[all]{xy}

\setlength{\topmargin}{-1.0in}
\setlength{\oddsidemargin}{0in}
\setlength{\textheight}{10.1in}
\setlength{\textwidth}{6.5in}
\setlength{\baselineskip}{12mm}

\newcounter{exercicios}
\setcounter{exercicios}{0}
\newcommand{\questao}{
\addtocounter{exercicios}{1}
\noindent{\bf Exerc{\'\i}cio \arabic{exercicios}: }}

\newcommand{\equi}{\Leftrightarrow}
\newcommand{\bic}{\leftrightarrow}
\newcommand{\cond}{\rightarrow}
\newcommand{\impl}{\Rightarrow}
\newcommand{\nao}{\sim}
\newcommand{\sub}{\subseteq}
\newcommand{\e}{\ \wedge\ }
\newcommand{\ou}{\ \vee\ }
\newcommand{\vaz}{\emptyset}
\newcommand{\nsub}{\nsubset}
\renewcommand{\sin}{{\rm sen\,}}

\newcommand{\n}{\mathbb{N}}
\newcommand{\z}{\mathbb{Z}}
\newcommand{\real}{\mathbb{R}}
\newcommand{\vesp}{\vspace{0.2cm}}
\newcommand{\subne}{\subsetneqq}


\newcommand{\compcent}[1]{\vcenter{\hbox{$#1\circ$}}}
\newcommand{\comp}{\mathbin{\mathchoice
{\compcent\scriptstyle}{\compcent\scriptstyle}
{\compcent\scriptscriptstyle}{\compcent\scriptscriptstyle}}}

\begin{document}

\pagestyle{empty}

\begin{figure}[h]
        \begin{minipage}[c]{1.7cm}
        \includegraphics[width=1.7cm]{unb.pdf}
        \end{minipage}%
        \hspace{0pt}
        \begin{minipage}[c]{4in}
          {Universidade de Brasília} \\
          {Departamento de Matemática}
\end{minipage}
\end{figure}
\vspace{-1cm}\hrule


\begin{center}
{\Large\bf {\'A}lgebra 1 - Turma C -- 1$^{o}$/2018} \\ \vspace{9pt} {\large\bf
  $1^{\underline{o}}$ Teste - Resolu\c{c}\~ao}\\
\vspace{9pt} Prof. Jos{\'e} Ant{\^o}nio O. Freitas
\end{center}
\hrule

\vspace{.6cm}

% \textbf{O Exerc{\'\i}cio 1 foi corrigido para atribui\c{c}\~ao da nota.}

\vspace{.6cm}

\questao Sejam $A$, $B$ e $C$ conjuntos. Suponha que $A \nsubseteq B$ e $B \subset C$. Ent�o $A \nsubseteq C$?

\noindent\textbf{Solu\c{c}\~ao:} Falso. Tome por exemplo $A = \{1,2,3\}$, $B = \{3,4\}$ e $C = \{1,2,3,4\}$. Temos $A \nsubseteq B$, $B \subset C$ e no entanto $A \subseteq C$.

\vspace{.5cm}

\questao Sejam $A$, $B$, $C$ e $D$ conjuntos. Suponha que $A \subseteq B$ e $C \subseteq D$. Ent�o $A \cap C \subseteq B \cap D$?

\noindent\textbf{Solu\c{c}\~ao:} Verdadeiro. Seja $x \in A \cap C$. Ent�o pela defini��o de interse��o de conjuntos temos $x \in A$ e $x \in C$. Agora, por hip�tese, $A \subseteq B$ e de $x \in A$ podemos concluir que $x \in B$. Al�m disso, por hip�tese tamb�m temos $C \subseteq D$. Da� de $x \in C$ podemos concluir que $x \in D$. Logo $x \in C$ e $x \in D$, ou seja, $x \in B \cap D$. Portanto para todo $x \in A \cap C$ temos $x \in B \cap D$, isto �, $A \cap C \subseteq B \cap D$.

\end{document}