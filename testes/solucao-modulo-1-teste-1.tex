%!TEX program = xelatex
%!TEX encoding = ISO-8859-1
%!TEX program = xelatex
% !TEX encoding = ISO-8859-1
\def\ano{2019}
\def\semestre{2}
\def\disciplina{\'Algebra 1}
\def\turma{C}

\documentclass[12pt]{exam}

\usepackage{caption}
\usepackage{amssymb}
\usepackage{amsmath,amsfonts,amsthm,amstext}
\usepackage[brazil]{babel}
% \usepackage[latin1]{inputenc}
\usepackage{graphicx}
\graphicspath{{/ArquivosLinux/OneDrive/imagens-latex/}{D:/OneDrive - unb.br/imagens-latex/}}
\usepackage{enumitem}
\usepackage{multicol}
\usepackage{answers}
\usepackage{tikz,ifthen}
\usetikzlibrary{lindenmayersystems}
\usetikzlibrary[shadings]
\Newassociation{solucao}{Solution}{ans}
\newtheorem{exercicio}{}

\setlength{\topmargin}{-1.0in}
\setlength{\oddsidemargin}{0in}
\setlength{\textheight}{10.1in}
\setlength{\textwidth}{6.5in}
\setlength{\baselineskip}{12mm}

\extraheadheight{0.7in}
\firstpageheadrule
\runningheadrule
\lhead{
        \begin{minipage}[c]{1.7cm}
        \includegraphics[width=1.7cm]{unb.pdf}
        \end{minipage}%
        \hspace{0pt}
        \begin{minipage}[c]{4in}
          {Universidade de Brasília} --
          {Departamento de Matemática}
\end{minipage}
\vspace*{-0.8cm}
}
% \chead{Universidade de Brasília - Departamento de Matemática}
% \rhead{}
% \vspace*{-2cm}

\extrafootheight{.5in}
\footrule
\lfoot{\disciplina\ - \semestre$^o$/\ano\ - Módulo \numeromodulo}
\cfoot{}
\rfoot{Página \thepage\ de \numpages}

\newcounter{exercicios}
\renewcommand{\theexercicios}{\arabic{exercicios}}

\newenvironment{questao}[1]{
\refstepcounter{exercicios}
\ifx&#1&
\else
   \label{#1}
\fi
\noindent\textbf{Exercício {\theexercicios}:}
}

\newcommand{\resp}[1]{
\noindent{\bf Exercício #1: }}

\def\ano{2024}
\def\semestre{1}
\def\disciplina{Álgebra 1}
\def\nomeabreviado{Álgebra 1}
\def\turma{1}

\newcommand{\im}{{\rm Im\,}}
\newcommand{\dlim}[2]{\displaystyle\lim_{#1\rightarrow #2}}
\newcommand{\minf}{+\infty}
\newcommand{\ninf}{-\infty}
\newcommand{\cp}[1]{\mathbb{#1}}
\newcommand{\sub}{\subseteq}
\newcommand{\n}{\mathbb{N}}
\newcommand{\z}{\mathbb{Z}}
\newcommand{\rac}{\mathbb{Q}}
\newcommand{\real}{\mathbb{R}}
\newcommand{\complex}{\mathbb{C}}

\newcommand{\vesp}[1]{\vspace{ #1  cm}}

\newcommand{\compcent}[1]{\vcenter{\hbox{$#1\circ$}}}
\newcommand{\comp}{\mathbin{\mathchoice
        {\compcent\scriptstyle}{\compcent\scriptstyle}
        {\compcent\scriptscriptstyle}{\compcent\scriptscriptstyle}}}
\renewcommand{\sin}{{\rm sen\,}}
\renewcommand{\tan}{{\rm tg\,}}
\renewcommand{\csc}{{\rm cossec\,}}
\renewcommand{\cot}{{\rm cotg\,}}
\renewcommand{\sinh}{{\rm senh\,}}

\begin{document}

\begin{center}
{\Large\bf \disciplina\ - Turma \turma\ -- \semestre$^{o}$/\ano} \\ \vspace{9pt} {\large\bf
$1^{\underline{o}}$ Teste - Resolu\c{c}\~ao}\\
\vspace{9pt} Prof. Jos{\'e} Ant{\^o}nio O. Freitas
\end{center}
\hrule

\vspace{.6cm}

% \textbf{O Exerc{\'\i}cio 1 foi corrigido para atribui\c{c}\~ao da nota.}

\vspace{.6cm}

\questao Verdadeiro ou falso? Justifique.
\begin{center}
    Se $A \nsubseteq B$ e $B \nsubseteq C$, ent\~ao $A \nsubseteq C$?
\end{center}

\noindent\textbf{Solu\c{c}\~ao:} Falso. Tome por exemplo $A = \{1, 2, 3\}$, $B = \{3, 4, 5\}$ e $C = \{1, 2, 3, 4\}$. Temos $A \nsubseteq B$, $B \nsubseteq C$ e no entanto $A \subseteq C$.

\vspace{.5cm}

\questao Mostre que se $A \subseteq B$ e $C = B - A$, ent\~ao $A = B - C$.

\noindent\textbf{Solu\c{c}\~ao:} Como a teste envolve uma igualdade de conjuntos precisamos mostrar que:
\begin{enumerate}
    \item $A \sub B - C$\label{primeira_inclusao}
    \item $B - C \sub A$\label{segunda_inclusao}
\end{enumerate}

Para mostrar \eqref{primeira_inclusao} seja $x \in A$. Como, por hipótese, $A \sub B$ então $x \in B$. Mas, também por hipótese, temos $C = B - A$. Assim como $x \in A$, então $x \notin C$. Logo, $x \in B$ e $x \notin C$, isto é, $x \in B - C$. Com isso, $A \sub B - C$.

Agora, para mostrar \eqref{segunda_inclusao} seja $y \in B - C$. Daí, por definição, $x \in B$ e $x \notin C$. Mas, por hipótese, $C = B - A$, então como $x \notin C = B - A$ e $x \in B$, devemos ter $x \in A$. Logo $B - C \sub A$.

Portanto, de \eqref{primeira_inclusao} e \eqref{segunda_inclusao} segue que $A = B - C$, como queríamos.

\end{document}