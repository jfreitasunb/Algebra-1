%!TEX program = xelatex
%!TEX encoding = ISO-8859-1
\documentclass[12pt]{article}

\usepackage{amssymb}
\usepackage{amsmath,amsfonts,amsthm,amstext,mathabx}
\usepackage[brazil]{babel}
%\usepackage[latin1]{inputenc}
\usepackage{graphicx}
\graphicspath{{/home/jfreitas/Dropbox/imagens-latex/}{/Volumes/Vader/Dropbox/imagens-latex/}{D:/Dropbox/imagens-latex/}}
\usepackage{enumitem}
\usepackage{multicol}
\usepackage[all]{xy}

\setlength{\topmargin}{-1.0in}
\setlength{\oddsidemargin}{0in}
\setlength{\textheight}{10.1in}
\setlength{\textwidth}{6.5in}
\setlength{\baselineskip}{12mm}

\newcounter{exercicios}
\setcounter{exercicios}{0}
\newcommand{\questao}{
\addtocounter{exercicios}{1}
\noindent{\bf Exerc{\'\i}cio \arabic{exercicios}: }}

\newcommand{\equi}{\Leftrightarrow}
\newcommand{\bic}{\leftrightarrow}
\newcommand{\cond}{\rightarrow}
\newcommand{\impl}{\Rightarrow}
\newcommand{\nao}{\sim}
\newcommand{\sub}{\subseteq}
\newcommand{\e}{\ \wedge\ }
\newcommand{\ou}{\ \vee\ }
\newcommand{\vaz}{\emptyset}
\newcommand{\nsub}{\nsubset}
\renewcommand{\sin}{{\rm sen\,}}

\newcommand{\n}{\mathbb{N}}
\newcommand{\z}{\mathbb{Z}}
\newcommand{\real}{\mathbb{R}}
\newcommand{\vesp}{\vspace{0.2cm}}
\newcommand{\subne}{\subsetneqq}


\newcommand{\compcent}[1]{\vcenter{\hbox{$#1\circ$}}}
\newcommand{\comp}{\mathbin{\mathchoice
{\compcent\scriptstyle}{\compcent\scriptstyle}
{\compcent\scriptscriptstyle}{\compcent\scriptscriptstyle}}}

\begin{document}

\pagestyle{empty}

\begin{figure}[h]
        \begin{minipage}[c]{1.7cm}
        \includegraphics[width=1.7cm]{unb.pdf}
        \end{minipage}%
        \hspace{0pt}
        \begin{minipage}[c]{4in}
          {Universidade de Brasília} \\
          {Departamento de Matemática}
\end{minipage}
\end{figure}
\vspace{-1cm}\hrule


\begin{center}
{\Large\bf {\'A}lgebra 1 - Turma C -- 1$^{o}$/2018} \\ \vspace{9pt} {\large\bf
  $2^{\underline{o}}$ Teste - Módulo 3 - Resolu\c{c}\~ao}\\
\vspace{9pt} Prof. Jos{\'e} Ant{\^o}nio O. Freitas
\end{center}
\hrule

\vspace{.6cm}

\questao Seja $f : \complex \to M_2(\real)$ dada por
\[
	f(a + bi) = \begin{pmatrix}
		\overline{a} & -b\\
		b & a
	\end{pmatrix}.
\]
Mostre que $f$ é um homomorfismo de anéis. Determine $\ker(f)$.

\noindent\textbf{Solu\c{c}\~ao:}

Sejam $a + bi$, $c + di \in \complex$. Então
\begin{align*}
	f((a + bi) + (c + di)) = f((a + c) + (b + d)i) &= \begin{pmatrix}
		a + c & -(b + d)\\
		(b + d) & a + c
	\end{pmatrix} = \begin{pmatrix}
		a & -b\\
		b & a
	\end{pmatrix} + \begin{pmatrix}
		c & -d\\
		d & c
	\end{pmatrix} = f(a + bi) + f(c + di)\\
	f((a + bi)\cdot (c + di)) = f(ac + adi + bci + bdi^2) = f((ac - bd) + (ad + bc)i)&= \begin{pmatrix}
		ac - bd & -(ad + bc)\\
		ad + bc & ac - bd
	\end{pmatrix} = \begin{pmatrix}
		a & -b\\
		b & a
	\end{pmatrix} \begin{pmatrix}
		c & -d\\
		d & c
	\end{pmatrix} = f(x)f(y)
\end{align*}
Logo $f$ é um homomorfismo de anéis.

Agora,
\[
	ker(f) = \left\{a + bi \in \complex \mid f(a + bi) = \begin{pmatrix}
		0 & 0\\
		0 & 0
	\end{pmatrix}\right\}.
\]

Assim
\[
	f(a + bi) = \begin{pmatrix}
		a & -b\\
		b & a
	\end{pmatrix} = \begin{pmatrix}
		0 & 0\\
		0 & 0
	\end{pmatrix}
\]
logo $a = 0$ e $b = 0$. Isto é, $a + bi = 0$. Logo 
\[
	\ker(f) = \{0\}.
\]

\vspace{1cm}

\questao Em $\z \times\z$ considere a operação
\[
	(x, y) + (z, t) = (x + z, y + t)
\]
para todos $(x,y)$, $(z,t) \in \z\times\z$. Mostre que $(\z\times\z, +)$ é um grupo abeliano.

\noindent\textbf{Solu\c{c}\~ao:}

Primeiro dados $(x,y)$, $(z,t) \in \z\times\z$ temos
\[
	(x,y) + (z,t) = (x+z,y+t) = (z+x,t+y) = (z,t) + (x,y).
\]
Assim a operação em $\z\times\z$ é comutativa.

Sejam $(x,y)$, $(z,t)$ e $(r,s) \in \z\times\z$. Então
\begin{align*}
	[(x,y) + (z,t)] + (r,s) &= (x + z, y+t) + (r,s) = (x+z+r,y+t+s)\\
	(x,y) + [(z,t)+(r,s)] &= (x,y) + (z+r,t+s) = (x+z+r,y+t+s).
\end{align*}
Logo $[(x,y)+(z,t)] + (r,s) = (x,y) + [(z,t) + (r,s)]$.

Para todo $(x,y) \in \z\times\z$ temos
\[
	(x,y) + (0,0) = (x+0,y+0) = (x,y).
\]
Assim $(0,0)$ é o elemento neutro da operação em $\z\times\z$.

Agora dado $(x,y) \in \z\times\z$ tome $(-x,-y) \in \z\times\z$. Então
\[
	(x,y) + (-x,-y) = (x+(-x),y+(-y)) = (0,0).
\]
Daí $(-x,-y)$ é o oposto de $(x,y)$ em $\z\times\z$.

Portanto $(\z\times\z,+)$ é um grupo abeliano.

\end{document}