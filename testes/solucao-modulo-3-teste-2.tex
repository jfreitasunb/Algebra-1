%!TEX program = xelatex
%!TEX encoding = ISO-8859-1
\documentclass[12pt]{article}

\usepackage{amssymb}
\usepackage{amsmath,amsfonts,amsthm,amstext,mathabx}
\usepackage[brazil]{babel}
%\usepackage[latin1]{inputenc}
\usepackage{graphicx}
\graphicspath{{/home/jfreitas/Dropbox/imagens-latex/}{/Volumes/Vader/Dropbox/imagens-latex/}{D:/Dropbox/imagens-latex/}}
\usepackage{enumitem}
\usepackage{multicol}
\usepackage[all]{xy}

\setlength{\topmargin}{-1.0in}
\setlength{\oddsidemargin}{0in}
\setlength{\textheight}{10.1in}
\setlength{\textwidth}{6.5in}
\setlength{\baselineskip}{12mm}

\newcounter{exercicios}
\setcounter{exercicios}{0}
\newcommand{\questao}{
\addtocounter{exercicios}{1}
\noindent{\bf Exerc{\'\i}cio \arabic{exercicios}: }}

\newcommand{\equi}{\Leftrightarrow}
\newcommand{\bic}{\leftrightarrow}
\newcommand{\cond}{\rightarrow}
\newcommand{\impl}{\Rightarrow}
\newcommand{\nao}{\sim}
\newcommand{\sub}{\subseteq}
\newcommand{\e}{\ \wedge\ }
\newcommand{\ou}{\ \vee\ }
\newcommand{\vaz}{\emptyset}
\newcommand{\nsub}{\nsubset}
\renewcommand{\sin}{{\rm sen\,}}

\newcommand{\n}{\mathbb{N}}
\newcommand{\z}{\mathbb{Z}}
\newcommand{\real}{\mathbb{R}}
\newcommand{\vesp}{\vspace{0.2cm}}
\newcommand{\subne}{\subsetneqq}


\newcommand{\compcent}[1]{\vcenter{\hbox{$#1\circ$}}}
\newcommand{\comp}{\mathbin{\mathchoice
{\compcent\scriptstyle}{\compcent\scriptstyle}
{\compcent\scriptscriptstyle}{\compcent\scriptscriptstyle}}}

\begin{document}

\pagestyle{empty}

\begin{figure}[h]
        \begin{minipage}[c]{1.7cm}
        \includegraphics[width=1.7cm]{unb.pdf}
        \end{minipage}%
        \hspace{0pt}
        \begin{minipage}[c]{4in}
          {Universidade de Brasília} \\
          {Departamento de Matemática}
\end{minipage}
\end{figure}
\vspace{-1cm}\hrule


\begin{center}
{\Large\bf {\'A}lgebra 1 - Turma C -- 1$^{o}$/2018} \\ \vspace{9pt} {\large\bf
  $2^{\underline{o}}$ Teste - Módulo 3 - Resolu\c{c}\~ao}\\
\vspace{9pt} Prof. Jos{\'e} Ant{\^o}nio O. Freitas
\end{center}
\hrule

\vspace{.6cm}

Seja $f : \z \to M_3(\z_8)$ dada por
\[
	f(x) = \begin{pmatrix}
		\overline{x} & \overline{0} & \overline{0}\\
		\overline{0} & \overline{x} & \overline{0}\\
		\overline{0} & \overline{0} & \overline{x}
	\end{pmatrix}.
\]

\questao Mostre que $f$ é um homomorfismo de anéis.

\noindent\textbf{Solu\c{c}\~ao:}

Sejam $x$, $y \in \z$. Então
\begin{align*}
	f(x + y) &= \begin{pmatrix}
		\overline{x + y} & \overline{0} & \overline{0}\\
		\overline{0} & \overline{x + y} & \overline{0}\\
		\overline{0} & \overline{0} & \overline{x + y}
	\end{pmatrix} = \begin{pmatrix}
		\overline{x} & \overline{0} & \overline{0}\\
		\overline{0} & \overline{x} & \overline{0}\\
		\overline{0} & \overline{0} & \overline{x}
	\end{pmatrix} + \begin{pmatrix}
		\overline{y} & \overline{0} & \overline{0}\\
		\overline{0} & \overline{y} & \overline{0}\\
		\overline{0} & \overline{0} & \overline{y}
	\end{pmatrix} = f(x) + f(y)\\
	f(x\cdot y) &= \begin{pmatrix}
		\overline{xy} & \overline{0} & \overline{0}\\
		\overline{0} & \overline{xy} & \overline{0}\\
		\overline{0} & \overline{0} & \overline{xy}
	\end{pmatrix} = \begin{pmatrix}
		\overline{x} & \overline{0} & \overline{0}\\
		\overline{0} & \overline{x} & \overline{0}\\
		\overline{0} & \overline{0} & \overline{x}
	\end{pmatrix} \begin{pmatrix}
		\overline{y} & \overline{0} & \overline{0}\\
		\overline{0} & \overline{y} & \overline{0}\\
		\overline{0} & \overline{0} & \overline{y}
	\end{pmatrix} = f(x)f(y)
\end{align*}
Logo $f$ é um homomorfismo de anéis.

\vspace{1cm}

\questao Determine $\ker(f)$. $f$ é injetora?

\noindent\textbf{Solu\c{c}\~ao:} Temos
\[
	ker(f) = \left\{x \in \z \mid f(x) = \begin{pmatrix}
		\overline{0} & \overline{0} & \overline{0}\\
		\overline{0} & \overline{0} & \overline{0}\\
		\overline{0} & \overline{0} & \overline{0}
	\end{pmatrix}\right\}.
\]

Assim
\[
	f(x) = \begin{pmatrix}
		\overline{x} & \overline{0} & \overline{0}\\
		\overline{0} & \overline{x} & \overline{0}\\
		\overline{0} & \overline{0} & \overline{x}
	\end{pmatrix} = \begin{pmatrix}
		\overline{0} & \overline{0} & \overline{0}\\
		\overline{0} & \overline{0} & \overline{0}\\
		\overline{0} & \overline{0} & \overline{0}
	\end{pmatrix}
\]
logo $\overline{x} = \overline{0}$. Isto é, $x \equiv 0 \pmod 8$. Logo 
\[
	\ker(f) = \{x \in \z \mid x = 8k,\ k \in \z\} = \{0, \pm 8, \pm 16, \pm 24, \dots\}.
\]

Como $\ker(f) \ne \{0\}$ então $f$ não é injetora.
\end{document}