%!TEX program = xelatex
% !TEX encoding = ISO-8859-1
\def\ano{2019}
\def\semestre{2}
\def\disciplina{\'Algebra 1}
\def\turma{C}
\def\numeroteste{2}
\def\modulo{3}

\documentclass[12pt]{exam}

\usepackage{caption}
\usepackage{amssymb}
\usepackage{amsmath,amsfonts,amsthm,amstext}
\usepackage[brazil]{babel}
% \usepackage[latin1]{inputenc}
\usepackage{graphicx}
\graphicspath{{/ArquivosLinux/OneDrive/imagens-latex/}{D:/OneDrive - unb.br/imagens-latex/}}
\usepackage{enumitem}
\usepackage{multicol}
\usepackage{answers}
\usepackage{tikz,ifthen}
\usetikzlibrary{lindenmayersystems}
\usetikzlibrary[shadings]
\Newassociation{solucao}{Solution}{ans}
\newtheorem{exercicio}{}

\setlength{\topmargin}{-1.0in}
\setlength{\oddsidemargin}{0in}
\setlength{\textheight}{10.1in}
\setlength{\textwidth}{6.5in}
\setlength{\baselineskip}{12mm}

\extraheadheight{0.7in}
\firstpageheadrule
\runningheadrule
\lhead{
        \begin{minipage}[c]{1.7cm}
        \includegraphics[width=1.7cm]{unb.pdf}
        \end{minipage}%
        \hspace{0pt}
        \begin{minipage}[c]{4in}
          {Universidade de Brasília} --
          {Departamento de Matemática}
\end{minipage}
\vspace*{-0.8cm}
}
% \chead{Universidade de Brasília - Departamento de Matemática}
% \rhead{}
% \vspace*{-2cm}

\extrafootheight{.5in}
\footrule
\lfoot{\disciplina\ - \semestre$^o$/\ano\ - Módulo \numeromodulo}
\cfoot{}
\rfoot{Página \thepage\ de \numpages}

\newcounter{exercicios}
\renewcommand{\theexercicios}{\arabic{exercicios}}

\newenvironment{questao}[1]{
\refstepcounter{exercicios}
\ifx&#1&
\else
   \label{#1}
\fi
\noindent\textbf{Exercício {\theexercicios}:}
}

\newcommand{\resp}[1]{
\noindent{\bf Exercício #1: }}

\def\ano{2024}
\def\semestre{1}
\def\disciplina{Álgebra 1}
\def\nomeabreviado{Álgebra 1}
\def\turma{1}

\newcommand{\im}{{\rm Im\,}}
\newcommand{\dlim}[2]{\displaystyle\lim_{#1\rightarrow #2}}
\newcommand{\minf}{+\infty}
\newcommand{\ninf}{-\infty}
\newcommand{\cp}[1]{\mathbb{#1}}
\newcommand{\sub}{\subseteq}
\newcommand{\n}{\mathbb{N}}
\newcommand{\z}{\mathbb{Z}}
\newcommand{\rac}{\mathbb{Q}}
\newcommand{\real}{\mathbb{R}}
\newcommand{\complex}{\mathbb{C}}

\newcommand{\vesp}[1]{\vspace{ #1  cm}}

\newcommand{\compcent}[1]{\vcenter{\hbox{$#1\circ$}}}
\newcommand{\comp}{\mathbin{\mathchoice
        {\compcent\scriptstyle}{\compcent\scriptstyle}
        {\compcent\scriptscriptstyle}{\compcent\scriptscriptstyle}}}
\renewcommand{\sin}{{\rm sen\,}}
\renewcommand{\tan}{{\rm tg\,}}
\renewcommand{\csc}{{\rm cossec\,}}
\renewcommand{\cot}{{\rm cotg\,}}
\renewcommand{\sinh}{{\rm senh\,}}

\begin{document}

\begin{center}
{\Large\bf \disciplina\ - Turma \turma\ -- \semestre$^{o}$/\ano} \\ \vspace{9pt} {\large\bf
$\numeroteste^{\underline{o}}$ Teste - M\'odulo \modulo\ - Resolu\c{c}\~ao}\\
\vspace{9pt} Prof. Jos{\'e} Ant{\^o}nio O. Freitas
\end{center}
\hrule

\vspace{.6cm}

\questao{} Seja $f : \z \to M_2(\z_3)$ dada por
\[
	f(x) = \begin{bmatrix}
		\overline{x} & \overline{0}\\
		\overline{0} & \overline{x}
	\end{bmatrix}.
\]
Mostre que $f$ \'e um homomorfismo de an\'eis e encontre $\ker(f)$.

\noindent\textbf{Solu\c{c}\~ao:}

Sejam $x$, $y \in \z$. Ent\~ao
\begin{align*}
	f(x + y) &= \begin{bmatrix}
		\overline{x + y} & \overline{0}\\
		\overline{0} & \overline{x + y}
	\end{bmatrix} = \begin{bmatrix}
		\overline{x} & \overline{0}\\
		\overline{0} & \overline{x}
	\end{bmatrix} + \begin{bmatrix}
		\overline{y} & \overline{0}\\
		\overline{0} & \overline{y}
	\end{bmatrix}\\ &= f(x) + f(y)\\
	f(x\cdot y) &= \begin{bmatrix}
		\overline{xy} & \overline{0}\\
		\overline{0} & \overline{xy}
	\end{bmatrix} = \begin{bmatrix}
		\overline{x} & \overline{0}\\
		\overline{0} & \overline{x}
	\end{bmatrix} \begin{bmatrix}
		\overline{y} & \overline{0}\\
		\overline{0} & \overline{y}
	\end{bmatrix}\\ &= f(x)f(y)
\end{align*}
Logo $f$ \'e um homomorfismo de an\'eis.

Agora,
\[
	ker(f) = \left\{x \in \z \mid f(x) = \begin{bmatrix}
		\overline{0} & \overline{0}\\
		\overline{0} & \overline{0}
	\end{bmatrix}\right\}.
\]

Assim
\[
	f(x) = \begin{bmatrix}
		\overline{x} & \overline{0}\\
		\overline{0} & \overline{x}
	\end{bmatrix} = \begin{bmatrix}
		\overline{0} & \overline{0}\\
		\overline{0} & \overline{0}
	\end{bmatrix}
\]
logo $\overline{x} = \overline{0}$. Isto \'e, $x = 3k$, $k \in \z$. Logo 
\[
	\ker(f) = \{3k \mid k \in \z\} = \{0, \pm 3, \pm 6, \pm 9, \cdots\}.
\]

\vspace{1cm}

\questao{} Em $(\real , \star)$ considere a opera\c{c}\~ao
\[
	x\star y = \sqrt[3]{x^3 + y^3}
\]
para todos $x$, $y \in \real$. Mostre que $(\real, +)$ \'e um grupo. Esse grupo \'e abeliano?

\noindent\textbf{Solu\c{c}\~ao:}

Primeiro dados $x$, $y \in \real$ temos
\[
	x \star y = \sqrt[3]{x^3 + y^3} = \sqrt[3]{y^3 + x^3} = y \star x.
\]
Assim a opera\c{c}\~ao $\star$ em $\real$ \'e comutativa.

Agora sejam $x$, $y$ e $z \in \real$. Ent\~ao
\begin{align*}
	(x \star y) \star z &= (\sqrt[3]{x^3 + y^3})\star z = \sqrt[3]{(\sqrt[3]{x^3 + y^3})^3 + z^3} = \sqrt[3]{x^3 + y^3 + z^3}\\
	x \star (y \star z) &= x \star (\sqrt[3]{y^3 + z^3}) = \sqrt[3]{x^3 + (\sqrt[3]{y^3 + z^3})^3} = \sqrt[3]{x^3 + y^3 + z^3}.
\end{align*}
Logo $(x \star y) \star z = x \star (y \star z)$.

Para todo $x \in \real$ temos
\[
	x \star 0 = \sqrt[3]{x^3 + 0^3} = x.
\]
Assim $0$ \'e o elemento neutro da opera\c{c}\~ao $\star$ em $\real$.

Agora dado $x \in \real$ tome $-x \in \real$. Ent\~ao
\[
	x \star (-x) = \sqrt[3]{x^3 + (-x)^3} = 0.
\]
Da{\'\i} $-x$ \'e o oposto de $x$ na opera\c{c}\~ao $\star$ em $\real$.

Portanto $(\real,+)$ \'e um grupo e esse grupo \'e abeliano.

\end{document}