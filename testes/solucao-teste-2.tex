%!TEX program = xelatex
%!TEX encoding = ISO-8859-1
\documentclass[12pt]{article}

\usepackage{amssymb}
\usepackage{amsmath,amsfonts,amsthm,amstext,mathabx}
\usepackage[brazil]{babel}
%\usepackage[latin1]{inputenc}
\usepackage{graphicx}
\graphicspath{{/home/jfreitas/Dropbox/imagens-latex/}{/Users/jfreitas/Dropbox/imagens-latex/}{D:/Dropbox/imagens-latex/}}
\usepackage{enumitem}
\usepackage{multicol}
\usepackage[all]{xy}

\setlength{\topmargin}{-1.0in}
\setlength{\oddsidemargin}{0in}
\setlength{\textheight}{10.1in}
\setlength{\textwidth}{6.5in}
\setlength{\baselineskip}{12mm}

\newcounter{exercicios}
\setcounter{exercicios}{0}
\newcommand{\questao}{
\addtocounter{exercicios}{1}
\noindent{\bf Exerc{\'\i}cio \arabic{exercicios}: }}

\newcommand{\equi}{\Leftrightarrow}
\newcommand{\bic}{\leftrightarrow}
\newcommand{\cond}{\rightarrow}
\newcommand{\impl}{\Rightarrow}
\newcommand{\nao}{\sim}
\newcommand{\sub}{\subseteq}
\newcommand{\e}{\ \wedge\ }
\newcommand{\ou}{\ \vee\ }
\newcommand{\vaz}{\emptyset}
\newcommand{\nsub}{\nsubset}
\renewcommand{\sin}{{\rm sen\,}}

\newcommand{\n}{\mathbb{N}}
\newcommand{\z}{\mathbb{Z}}
\newcommand{\q}{\mathbb{Q}}
\newcommand{\cmp}{\mathbb{C}}
\newcommand{\real}{\mathbb{R}}
\newcommand{\vesp}{\vspace{0.2cm}}
\newcommand{\subne}{\subsetneqq}


\newcommand{\compcent}[1]{\vcenter{\hbox{$#1\circ$}}}
\newcommand{\comp}{\mathbin{\mathchoice
{\compcent\scriptstyle}{\compcent\scriptstyle}
{\compcent\scriptscriptstyle}{\compcent\scriptscriptstyle}}}

\begin{document}
\pagestyle{empty}

\begin{figure}[h]
        \begin{minipage}[c]{1.7cm}
        \includegraphics[width=1.7cm]{unb.pdf}
        \end{minipage}%
        \hspace{0pt}
        \begin{minipage}[c]{4in}
          {Universidade de Bras{\'\i}lia} \\
          {Departamento de Matem{\'a}tica}
\end{minipage}
\end{figure}
\vspace{-1cm}\hrule

\begin{center}
{\Large\bf {\'A}lgebra 1 - Turma B -- 2$^{o}$/2015} \\ \vspace{9pt} {\large\bf
  $2^{\underline{o}}$ Teste - Resolu\c{c}\~ao}\\
\vspace{9pt} Prof. Jos{\'e} Ant{\^o}nio O. Freitas
\end{center}
\hrule

\vspace{.6cm}

\textbf{O Exerc{\'\i}cio 2 foi corrigido para atribui\c{c}\~ao da nota.}

\vspace{.6cm}

\questao Mostre que $9^n - 1$ {\'e} m\'ultiplo de 8 para todo $n \ge 0$.

\noindent\textbf{Solu\c{c}\~ao:} Primeiro devemos verificar que a base da indu\c{c}\~ao \'e v\'alida. Para isso tomamos $n = 0$:
\[
	9^0 - 1 = 0 = 8\cdot 0.
\]
Assim a base da indu\c{c}\~ao \'e verdadeira.

Agora suponha que para um certo $k \ge 0$ tenhamos:
\[
	(H.I)\ 9^k - 1 = 8l
\]
para algum $l \in \z$. Vamos provar que para $k + 1$ tamb\'em obtemos um m\'ultiplo de 8. Temos
\begin{align*}
	9^{k + 1} - 1 &= 9^k\cdot 9 - 1 \\& = 9^k\cdot 9 - (9 - 8) \\ &= 9(9^k - 1) - 8
\end{align*}
Agora pela hip\'otese de indu\c{c}\~ao, $9^k - 1 = 8l$ assim a \'ultima equa\c{c}\~ao pode ser escrita como:
\begin{align*}
	9^{k + 1} - 1 &= 9\cdot 8l - 8 \\ &= 8(9l - 1),
\end{align*}
ou seja, um m\'ultiplo de 8. Portanto pelo Princ{\'\i}pio da Indu\c{c}\~ao Finita, a afirma\c{c}\~ao \'e verdadeira.

\vspace{.5cm}

\questao Seja $A=\z\times \z^*$, onde $\mathbb{Z}^*=\mathbb{Z}\setminus \{0\}$. Para $(x,y), (z,t) \in
A$, considere a seguinte rela{\c c}{\~a}o
\[
(x,y)R(z,t) \Leftrightarrow xt = yz.
\]
\begin{enumerate}[label={\alph*})]
\item Mostre que $R$ {\'e} uma rela{\c c}{\~a}o de equival{\^e}ncia sobre $A$.

\noindent\textbf{Solu\c{c}\~ao:} Para $(x,y)$, $(z,t)$ e $(r,s) \in A$ temos
\begin{enumerate}
	\item $(x,y)R(x,y)$ pois $xy = xy$ para todo $(x,y) \in A$. Assim $R$ \'e reflexiva.
	\item Se $(x,y)R(z,t)$, ent\~ao $xt = yz$, ou seja, $yz = xt$. Logo $(z,t)R(x,y)$ e com isso $R$ \'e sim\'etrica.
	\item Se $(x,y)R(z,t)$ e $(z,t)R(r,s)$ ent\~ao
	\begin{align}
		xt = yz\\
		zs = tr.
	\end{align}
	Multiplicando a segunda equa\c{c}\~ao por $y$ e organizando os fatores obtemos:
	\begin{align*}
		(yz)s = t(yr)\\
		(xt)s = t(yr)\\
		t(xs) = t(yr)\\
		t(xs - yr) = 0.
	\end{align*}
	Como $t \in \z^*$, devemos ter $xs - yr = 0$, ou seja, $xs = yr$. Logo $(x,y)R(r,s)$, e com isso $R$ \'e transitiva.
	Portanto, $R$ \'e uma rela\c{c}\~ao de equival\^encia.
\end{enumerate}

\item Descreva a classe de equival{\^e}ncia $\overline{(0,1)}$, $\overline{(1,1)}$, $\overline{(1,2)}$, $\overline{(2,1)}$, $\overline{(2,2)}$, $\overline{(2,3)}$.

\noindent\textbf{Solu\c{c}\~ao:}
\begin{align*}
	\overline{(0,1)} &= \{ (x, y) \in A \mid (x,y)R(0,1)\} = \{ (x,y) \in R \mid x\cdot 0 = y\cdot 1\} \\ &= \{ (x,y) \in R \mid y = 0 \} = \{(0, y) \mid y \in \z^*\} \\ &= \{ (0,1), (0,-1), (0,\pm 2) , (0, \pm 3), \cdots\}\\
	\overline{(1,1)} &= \{ (x, y) \in A \mid (x,y)R(1,1)\} = \{ (x,y) \in R \mid x\cdot 1 = y\cdot 1\} \\ &= \{ (x,y) \in R \mid x = y \} = \{(y, y) \mid y \in \z^*\} \\ &= \{\cdots, (-2,-2), (-1,-1), (1,1), (2,2),\cdots\}\\
	\overline{(1,2)} &= \{ (x, y) \in A \mid (x,y)R(1,2)\} = \{ (x,y) \in R \mid x\cdot 2 = y\cdot 1\} \\ &= \{ (x,y) \in R \mid 2x = y \} = \{(x, 2x) \mid x \in \z^*\} \\ &= \{\cdots, (-2,-4), (-1,-2), (1,2), (2,4), \cdots\}\\
	\overline{(2,1)} &= \{ (x, y) \in A \mid (x,y)R(2,1)\} = \{ (x,y) \in R \mid x\cdot 1 = y\cdot 2\} \\ &= \{ (x,y) \in R \mid x = 2y \} = \{(2y, y) \mid y \in \z^*\} \\ &= \{\cdots, (-4,-2), (-2,-1), (2,1), (4,2), \cdots\}\\
	\overline{(2,2)} &= \{ (x, y) \in A \mid (x,y)R(2,2)\} = \{ (x,y) \in R \mid x\cdot 2 = y\cdot 2\} \\ &= \{ (x,y) \in R \mid 2x = 2y \} = \{(y, y) \mid y \in \z^*\} = \overline{(1,1)}\\
	\overline{(2,3)} &= \{ (x, y) \in A \mid (x,y)R(2,3)\} = \{ (x,y) \in R \mid x\cdot 3 = y\cdot 2\} \\ &= \{ \cdots,(-8,-12),(-4,-6(=),(-2,-3), (2,3), (4,6), (8,12), \cdots\}
\end{align*}
\end{enumerate}

\end{document}