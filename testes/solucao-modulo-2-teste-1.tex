%!TEX program = xelatex
%!TEX encoding = ISO-8859-1
\documentclass[12pt]{article}

\usepackage{amssymb}
\usepackage{amsmath,amsfonts,amsthm,amstext,mathabx}
\usepackage[brazil]{babel}
%\usepackage[latin1]{inputenc}
\usepackage{graphicx}
\graphicspath{{/home/jfreitas/Dropbox/imagens-latex/}{/Volumes/Vader/Dropbox/imagens-latex/}{D:/Dropbox/imagens-latex/}}
\usepackage{enumitem}
\usepackage{multicol}
\usepackage[all]{xy}

\setlength{\topmargin}{-1.0in}
\setlength{\oddsidemargin}{0in}
\setlength{\textheight}{10.1in}
\setlength{\textwidth}{6.5in}
\setlength{\baselineskip}{12mm}

\newcounter{exercicios}
\setcounter{exercicios}{0}
\newcommand{\questao}{
\addtocounter{exercicios}{1}
\noindent{\bf Exerc{\'\i}cio \arabic{exercicios}: }}

\newcommand{\equi}{\Leftrightarrow}
\newcommand{\bic}{\leftrightarrow}
\newcommand{\cond}{\rightarrow}
\newcommand{\impl}{\Rightarrow}
\newcommand{\nao}{\sim}
\newcommand{\sub}{\subseteq}
\newcommand{\e}{\ \wedge\ }
\newcommand{\ou}{\ \vee\ }
\newcommand{\vaz}{\emptyset}
\newcommand{\nsub}{\nsubset}
\renewcommand{\sin}{{\rm sen\,}}

\newcommand{\n}{\mathbb{N}}
\newcommand{\z}{\mathbb{Z}}
\newcommand{\real}{\mathbb{R}}
\newcommand{\vesp}{\vspace{0.2cm}}
\newcommand{\subne}{\subsetneqq}


\newcommand{\compcent}[1]{\vcenter{\hbox{$#1\circ$}}}
\newcommand{\comp}{\mathbin{\mathchoice
{\compcent\scriptstyle}{\compcent\scriptstyle}
{\compcent\scriptscriptstyle}{\compcent\scriptscriptstyle}}}

\begin{document}

\pagestyle{empty}

\begin{figure}[h]
        \begin{minipage}[c]{1.7cm}
        \includegraphics[width=1.7cm]{unb.pdf}
        \end{minipage}%
        \hspace{0pt}
        \begin{minipage}[c]{4in}
          {Universidade de Brasília} \\
          {Departamento de Matemática}
\end{minipage}
\end{figure}
\vspace{-1cm}\hrule


\begin{center}
{\Large\bf {\'A}lgebra 1 - Turma D -- 2$^{o}$/2017} \\ \vspace{9pt} {\large\bf
  $1^{\underline{o}}$ Teste Módulo 2 - Resolu\c{c}\~ao}\\
\vspace{9pt} Prof. Jos{\'e} Ant{\^o}nio O. Freitas
\end{center}
\hrule

\vspace{.6cm}

Seja $f : \real - \left\{-\dfrac{d}{c}\right\} \to \real  - \left\{\dfrac{a}{c}\right\}$ dada por
\[
	f(x) =  \dfrac{ax + b}{cx + d},
\]
onde $a$, $b$, $c$, $d$ s{\~a}o n{\'u}meros reais constantes, $ad - bc \ne 0$.
\vspace{.5cm}

\questao Mostre que $f$ é injetora.

\noindent\textbf{Solu\c{c}\~ao:} Sejam $x_1$, $x_2 \in \real - \left\{-\dfrac{d}{c}\right\}$ tais que $f(x_1) = f(x_2)$. Temos
\begin{align*}
	f(x_1) &= f(x_2)\\
	\dfrac{ax_1 + b}{cx_1 + d} &= \dfrac{ax_2 + b}{cx_2 + d}\\
	(ax_1 + b)(cx_2 + d) &= (ax_2 + b)(cx_1 + d)\\
	acx_1x_2 + adx_1 + bcx_2 + bd &= acx_1x_2 + adx_2 + bcx_1 + bd\\
	adx_1 + bcx_2 - adx_2 - bcx_1 &= 0\\
	ad(x_1 - x_2) - bc(x_1 - x_2)&= 0\\
	(ad - bc)(x_1 - x_2) &= 0
\end{align*}
Como $ad - bc \ne 0$, então $x_1 - x_2 = 0$. Logo $x_1 = x_2$. Portanto $f$ é injetora.

\vspace{.5cm}

\questao Mostre que $f$ é sobrejetora.

\noindent\textbf{Solu\c{c}\~ao:} A função $f$ será sobrejetora se para todo $y \in \real$ existir $x \in \real$ tal que $f(x) = y$. Assim precisamos determinar se a equação $f(x) = y$ tem solução para todo $y \in \real$.
Temos
\begin{align*}
	f(x) &= y\\
	\dfrac{4x + 3}{3 - x} &= y\\
	4x + 3 &= y(3 - x)\\
	4x + 3 &= 3y - yx\\
	x(4 + y) &= 3y - 3
\end{align*}
Mas na última equação só podemos isolar $x$ se $y \ne -4$ e como $-4 \in \real$ isso indica que essa função não será sobrejetora.

Tome então $y = -4$ se existisse $x \in \real - \{3\}$ tal que $f(x) = -4$ teríamos
\begin{align*}
	f(x) &= -4\\
	\dfrac{4x + 3}{3 - x} &= -4\\
	4x + 3 &= -4(3 - x)\\
	4x + 3 &= -12 + 4x
\end{align*}
e então $3 = -12$, o que é falso. Portanto não existe $x \ in \real - \{3\}$ tal que $f(x) = 4$ e com isso $f$ não é sobrejetora.

\end{document}