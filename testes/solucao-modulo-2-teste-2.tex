%!TEX program = xelatex
%!TEX encoding = ISO-8859-1
\documentclass[12pt]{article}

\usepackage{amssymb}
\usepackage{amsmath,amsfonts,amsthm,amstext,mathabx}
\usepackage[brazil]{babel}
%\usepackage[latin1]{inputenc}
\usepackage{graphicx}
\graphicspath{{/home/jfreitas/Dropbox/imagens-latex/}{/Volumes/Vader/Dropbox/imagens-latex/}{D:/Dropbox/imagens-latex/}}
\usepackage{enumitem}
\usepackage{multicol}
\usepackage[all]{xy}

\setlength{\topmargin}{-1.0in}
\setlength{\oddsidemargin}{0in}
\setlength{\textheight}{10.1in}
\setlength{\textwidth}{6.5in}
\setlength{\baselineskip}{12mm}

\newcounter{exercicios}
\setcounter{exercicios}{0}
\newcommand{\questao}{
\addtocounter{exercicios}{1}
\noindent{\bf Exerc{\'\i}cio \arabic{exercicios}: }}

\newcommand{\equi}{\Leftrightarrow}
\newcommand{\bic}{\leftrightarrow}
\newcommand{\cond}{\rightarrow}
\newcommand{\impl}{\Rightarrow}
\newcommand{\nao}{\sim}
\newcommand{\sub}{\subseteq}
\newcommand{\e}{\ \wedge\ }
\newcommand{\ou}{\ \vee\ }
\newcommand{\vaz}{\emptyset}
\newcommand{\nsub}{\nsubset}
\renewcommand{\sin}{{\rm sen\,}}

\newcommand{\n}{\mathbb{N}}
\newcommand{\z}{\mathbb{Z}}
\newcommand{\real}{\mathbb{R}}
\newcommand{\vesp}{\vspace{0.2cm}}
\newcommand{\subne}{\subsetneqq}


\newcommand{\compcent}[1]{\vcenter{\hbox{$#1\circ$}}}
\newcommand{\comp}{\mathbin{\mathchoice
{\compcent\scriptstyle}{\compcent\scriptstyle}
{\compcent\scriptscriptstyle}{\compcent\scriptscriptstyle}}}

\begin{document}

\pagestyle{empty}

\begin{figure}[h]
        \begin{minipage}[c]{1.7cm}
        \includegraphics[width=1.7cm]{unb.pdf}
        \end{minipage}%
        \hspace{0pt}
        \begin{minipage}[c]{4in}
          {Universidade de Brasília} \\
          {Departamento de Matemática}
\end{minipage}
\end{figure}
\vspace{-1cm}\hrule


\begin{center}
{\Large\bf {\'A}lgebra 1 - Turma C -- 1$^{o}$/2018} \\ \vspace{9pt} {\large\bf
  $2^{\underline{o}}$ Teste Módulo 2 - Resolu\c{c}\~ao}\\
\vspace{9pt} Prof. Jos{\'e} Ant{\^o}nio O. Freitas
\end{center}
\hrule

\vspace{.6cm}

\questao Sejam $f : A \to B$ e $g : B \to A$ funções tais que $g \comp f$ é injetora. Mostre que $f$ é injetora.

\noindent\textbf{Solu\c{c}\~ao:}

	Sejam $x_1$, $x_2 \in A$ tais que $f(x_1) = f(x_2)$. Como $f(x_1)$, $f(x_2) \in B$ e $g : B \to A$ é uma função temos $g(f(x_1)) = g(f(x_2))$. Isto é, $(g \comp f)(x_1) = (g \comp f)(x_2)$ e como $g \comp f$ é injetora segue que $x_1 = x_2$. Portanto $f$ é injetora.

\vspace{.5cm}

\questao Considere o anel $\z$. O conjunto
\[
	B = \left\{ \dfrac{a}{2^n} \mid a \in \z,\ n \in \z\right\}
\]
é um subanel de $\z$?

\noindent\textbf{Solu\c{c}\~ao:} Primeio fazendo $a = 0 \in \z$ vemos que $0 \in B$. Logo $B \ne \emptyset$. Dados $x$ e $y \in B$ existem $a$, $b$, $m$ e $n \in \z$ tais que
\[
	x = \dfrac{a}{2^m} \quad \mbox{e} \quad y = \dfrac{b}{2^n}.
\]
Assim
\begin{align*}
	x - y &= \dfrac{a}{2^m} - \dfrac{b}{2^n} = \dfrac{a2^n - b2^m}{2^m2^n}\\
	xy &= \dfrac{a}{2^m}\dfrac{b}{2^n} = \dfrac{ab}{2^{m + n}}
\end{align*}
como $a2^n - b2^m \in \z$ e $ab \in \z$ segue que $x - y \in B$ e $xy \in B$ para todos $x$, $y \in B$. Portanto $B$ é um subanel de $\z$.

\end{document}