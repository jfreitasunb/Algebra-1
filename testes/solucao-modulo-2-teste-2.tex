%!TEX program = xelatex
%!TEX encoding = ISO-8859-1
\documentclass[12pt]{article}

\usepackage{amssymb}
\usepackage{amsmath,amsfonts,amsthm,amstext,mathabx}
\usepackage[brazil]{babel}
%\usepackage[latin1]{inputenc}
\usepackage{graphicx}
\graphicspath{{/home/jfreitas/Dropbox/imagens-latex/}{/Volumes/Vader/Dropbox/imagens-latex/}{D:/Dropbox/imagens-latex/}}
\usepackage{enumitem}
\usepackage{multicol}
\usepackage[all]{xy}

\setlength{\topmargin}{-1.0in}
\setlength{\oddsidemargin}{0in}
\setlength{\textheight}{10.1in}
\setlength{\textwidth}{6.5in}
\setlength{\baselineskip}{12mm}

\newcounter{exercicios}
\setcounter{exercicios}{0}
\newcommand{\questao}{
\addtocounter{exercicios}{1}
\noindent{\bf Exerc{\'\i}cio \arabic{exercicios}: }}

\newcommand{\equi}{\Leftrightarrow}
\newcommand{\bic}{\leftrightarrow}
\newcommand{\cond}{\rightarrow}
\newcommand{\impl}{\Rightarrow}
\newcommand{\nao}{\sim}
\newcommand{\sub}{\subseteq}
\newcommand{\e}{\ \wedge\ }
\newcommand{\ou}{\ \vee\ }
\newcommand{\vaz}{\emptyset}
\newcommand{\nsub}{\nsubset}
\renewcommand{\sin}{{\rm sen\,}}

\newcommand{\n}{\mathbb{N}}
\newcommand{\z}{\mathbb{Z}}
\newcommand{\real}{\mathbb{R}}
\newcommand{\vesp}{\vspace{0.2cm}}
\newcommand{\subne}{\subsetneqq}


\newcommand{\compcent}[1]{\vcenter{\hbox{$#1\circ$}}}
\newcommand{\comp}{\mathbin{\mathchoice
{\compcent\scriptstyle}{\compcent\scriptstyle}
{\compcent\scriptscriptstyle}{\compcent\scriptscriptstyle}}}

\begin{document}

\pagestyle{empty}

\begin{figure}[h]
        \begin{minipage}[c]{1.7cm}
        \includegraphics[width=1.7cm]{unb.pdf}
        \end{minipage}%
        \hspace{0pt}
        \begin{minipage}[c]{4in}
          {Universidade de Brasília} \\
          {Departamento de Matemática}
\end{minipage}
\end{figure}
\vspace{-1cm}\hrule


\begin{center}
{\Large\bf {\'A}lgebra 1 - Turma D -- 2$^{o}$/2017} \\ \vspace{9pt} {\large\bf
  $2^{\underline{o}}$ - Teste Módulo 2 - Resolu\c{c}\~ao}\\
\vspace{9pt} Prof. Jos{\'e} Ant{\^o}nio O. Freitas
\end{center}
\hrule

\vspace{.6cm}

\questao Defina em $\q$ as seguintes operações
\begin{align*}
	x \oplus y &= x + y - 1\\
	x \odot y &= x + y - xy
\end{align*}
para todos $x$, $y \in \q$. Mostre que $(\q, \otimes, \odot)$ é um anel. Esse anel é comutativo? Possui unidade?

\noindent\textbf{Solu\c{c}\~ao:}
\begin{enumerate}
	\item Sejam $x$, $y$, $z \ in \q$. Temos
	\begin{align*}
		(x \oplus y) \oplus z &= (x + y - 1) \oplus z = (x + y - 1) + z + 1 = x + y + z - 2\\
		x \oplus (y \oplus z) &= x \oplus (y + z - 1) = x + (y + z - 1) - 1 = x + y + z - 2
	\end{align*}
	Comparando esses dois resultados vemos que $(x \oplus y) \oplus z = x \ oplus (y \oplus z)$.

	\item Sejam $x$, $y \in \q$. Temos
	\begin{align*}
		x \oplus y = x + y - 1\\
		y \oplus x = y + x - 1
	\end{align*}
	Como $x$, $y \in \q$ segue que $x + y - 1 = y + x - 1$, isto é, $x \oplus y = y \oplus x$.

	\item Queremos mostrar que existe $0_\q \in \q$ tal que $x \oplus 0_\q = x$ para todo $x \in \q$.
	Assim vamos resolver a equação $x \oplus 0_\q = x$:
	\begin{align*}
		x\oplus 0_\q &= x\\
		x + 0_\q - 1 = x\\
		0_\q = 1
	\end{align*}
	Logo $0_\q = 1 \in \q$ é o elemento neutro da operação $\oplus$ em $\q$ pois para todo $x \in \q$ temos
	\[
		x \oplus 1 = x + 1 - 1 = x
	\]

	\item Para cada $x \in \q$ queremos mostrar que existe $y \in \q$ tal que $x \oplus y = 0_q$, onde $0_q$ é o elemento neutro encontrado no passo anterior. Assim resolvendo
	\begin{align*}
		x \oplus y &= 0_q\\
		x + y - 1 &= 1\\
		y &= 2 - x \in \q
	\end{align*}
	Logo para cada $x \in \q$ tome $y = 2 - x \in \q$. Daí
	\[
		x \oplus y = x \oplus (2 - x) = x + (2 - x) - 1 = 2 = 1
	\]
	Logo $y = 2 - x$ é o oposto de $x$ na operação $\oplus$ definida em $\q$.

	\item Sejam $x$, $y$ e $z \in \q$. Temos
	\begin{align*}
		(x \odot y) \odot z &= (x + y - xy) \odot z = (x + y - xy) + z - (x + y - xy)z = x + y + z - xy - xz - yz + xyz\\
		x \odot (y \odot z) &= x \odot (y + z - yz) = x + (y + z - yz) - x(y + z - yz) = x + y + z - xy - xz - yz + xyz
	\end{align*}
	Comparando esses dois resultados vemos que $(x \odot y) \odot z = x \odot (y \odot z)$.

	\item Sejam $x$, $y$ e $z \in \q$. Temos
	\begin{align*}
		(x \oplus y)\odot z &= (x + y - 1) \odot z = (x + y - 1) + z - (x + y - 1)z = x + y + z - xz - yz + z - 1\\
		(x \odot z) \oplus (y \odot z) = (x + z - xz) \oplus (y + z - yz) = (x + z - xz) + (y + z - yz) - 1 = x + y + z - xz - yz + z - 1
	\end{align*}

	Comparando esses dois últimos resultados vemos que $(x \oplus y)\odot z = (x \odot z) \oplus (y \odot z)$.

	\item Sejam $x$, $y$ e $z \in \q$. Temos
	\begin{align*}
		x \odot (y \oplus z) = x \odot (y + z - 1) = x + (y + z - 1) - x(y + z - 1) = x + y + z - xy - xz + x - 1\\
		(x \odot y) \oplus (x \odot z) = (x + y - xy) \oplus (x + z - xz) = (x + y - xy) + (x + z - xz) - 1 = x + y + z - xy - xz + x - 1
	\end{align*}
	Comparando esses dois últimos resultados vemos que $x \odot (y \oplus z) = (x \odot y) \oplus (x \odot z)$.
\end{enumerate}

Portanto $(\q, \oplus, \odot)$ é um anel.

Além disso para $x$, $y \in \q$ temos
\[
	x \odot y = x + y - xy = y + x - yx = y \odot x
\]
e com isso esse anel é comutativo.

Também temos
\[
	x \odot 0 = x + 0 - x\cdot 0 = x
\]
para todo $x \in \q$. Portanto $1_\q = 0$ é a unidade da multiplicação $\odot$ em $\q$.

Desse modo $(\q, \oplus, \odot)$ é um anel comutativo com unidade.

\vspace{.5cm}

\questao $f$ é sobrejetora?

\noindent\textbf{Solu\c{c}\~ao:} A função $f$ será sobrejetora se para todo $y \in \real$ existir $x \in \real$ tal que $f(x) = y$. Assim precisamos determinar se a equação $f(x) = y$ tem solução para todo $y \in \real$.
Temos
\begin{align*}
	f(x) &= y\\
	\dfrac{4x + 3}{3 - x} &= y\\
	4x + 3 &= y(3 - x)\\
	4x + 3 &= 3y - yx\\
	x(4 + y) &= 3y - 3
\end{align*}
Mas na última equação só podemos isolar $x$ se $y \ne -4$ e como $-4 \in \real$ isso indica que essa função não será sobrejetora.

Tome então $y = -4$ se existisse $x \in \real - \{3\}$ tal que $f(x) = -4$ teríamos
\begin{align*}
	f(x) &= -4\\
	\dfrac{4x + 3}{3 - x} &= -4\\
	4x + 3 &= -4(3 - x)\\
	4x + 3 &= -12 + 4x
\end{align*}
e então $3 = -12$, o que é falso. Portanto não existe $x \ in \real - \{3\}$ tal que $f(x) = 4$ e com isso $f$ não é sobrejetora.

\end{document}