%!TEX program = xelatex
%!TEX encoding = ISO-8859-1
%!TEX program = xelatex
% !TEX encoding = ISO-8859-1
\def\ano{2019}
\def\semestre{2}
\def\disciplina{\'Algebra 1}
\def\turma{C}

\documentclass[12pt]{exam}

\usepackage{caption}
\usepackage{amssymb}
\usepackage{amsmath,amsfonts,amsthm,amstext}
\usepackage[brazil]{babel}
% \usepackage[latin1]{inputenc}
\usepackage{graphicx}
\graphicspath{{/ArquivosLinux/OneDrive/imagens-latex/}{D:/OneDrive - unb.br/imagens-latex/}}
\usepackage{enumitem}
\usepackage{multicol}
\usepackage{answers}
\usepackage{tikz,ifthen}
\usetikzlibrary{lindenmayersystems}
\usetikzlibrary[shadings]
\Newassociation{solucao}{Solution}{ans}
\newtheorem{exercicio}{}

\setlength{\topmargin}{-1.0in}
\setlength{\oddsidemargin}{0in}
\setlength{\textheight}{10.1in}
\setlength{\textwidth}{6.5in}
\setlength{\baselineskip}{12mm}

\extraheadheight{0.7in}
\firstpageheadrule
\runningheadrule
\lhead{
        \begin{minipage}[c]{1.7cm}
        \includegraphics[width=1.7cm]{unb.pdf}
        \end{minipage}%
        \hspace{0pt}
        \begin{minipage}[c]{4in}
          {Universidade de Brasília} --
          {Departamento de Matemática}
\end{minipage}
\vspace*{-0.8cm}
}
% \chead{Universidade de Brasília - Departamento de Matemática}
% \rhead{}
% \vspace*{-2cm}

\extrafootheight{.5in}
\footrule
\lfoot{\disciplina\ - \semestre$^o$/\ano\ - Módulo \numeromodulo}
\cfoot{}
\rfoot{Página \thepage\ de \numpages}

\newcounter{exercicios}
\renewcommand{\theexercicios}{\arabic{exercicios}}

\newenvironment{questao}[1]{
\refstepcounter{exercicios}
\ifx&#1&
\else
   \label{#1}
\fi
\noindent\textbf{Exercício {\theexercicios}:}
}

\newcommand{\resp}[1]{
\noindent{\bf Exercício #1: }}

\def\ano{2024}
\def\semestre{1}
\def\disciplina{Álgebra 1}
\def\nomeabreviado{Álgebra 1}
\def\turma{1}

\newcommand{\im}{{\rm Im\,}}
\newcommand{\dlim}[2]{\displaystyle\lim_{#1\rightarrow #2}}
\newcommand{\minf}{+\infty}
\newcommand{\ninf}{-\infty}
\newcommand{\cp}[1]{\mathbb{#1}}
\newcommand{\sub}{\subseteq}
\newcommand{\n}{\mathbb{N}}
\newcommand{\z}{\mathbb{Z}}
\newcommand{\rac}{\mathbb{Q}}
\newcommand{\real}{\mathbb{R}}
\newcommand{\complex}{\mathbb{C}}

\newcommand{\vesp}[1]{\vspace{ #1  cm}}

\newcommand{\compcent}[1]{\vcenter{\hbox{$#1\circ$}}}
\newcommand{\comp}{\mathbin{\mathchoice
        {\compcent\scriptstyle}{\compcent\scriptstyle}
        {\compcent\scriptscriptstyle}{\compcent\scriptscriptstyle}}}
\renewcommand{\sin}{{\rm sen\,}}
\renewcommand{\tan}{{\rm tg\,}}
\renewcommand{\csc}{{\rm cossec\,}}
\renewcommand{\cot}{{\rm cotg\,}}
\renewcommand{\sinh}{{\rm senh\,}}

\begin{document}

\begin{center}
{\Large\bf \disciplina\ - Turma \turma\ -- \semestre$^{o}$/\ano} \\ \vspace{9pt} {\large\bf
$2^{\underline{o}}$ Teste - M\'odulo 2 - Resolu\c{c}\~ao}\\
\vspace{9pt} Prof. Jos{\'e} Ant{\^o}nio O. Freitas
\end{center}
\hrule

\vspace{.6cm}

\questao Sejam $f : A \to B$ e $g : B \to A$ fun\c{c}\~oes tais que $g \comp f$ \'e injetora. Mostre que $f$ \'e injetora.

\noindent\textbf{Solu\c{c}\~ao:}

	Sejam $x_1$, $x_2 \in A$ tais que $f(x_1) = f(x_2)$. Como $f(x_1)$, $f(x_2) \in B$ e $g : B \to A$ \'e uma fun\c{c}\~ao temos $g(f(x_1)) = g(f(x_2))$. Isto \'e, $(g \comp f)(x_1) = (g \comp f)(x_2)$ e como $g \comp f$ \'e injetora segue que $x_1 = x_2$. Portanto $f$ \'e injetora.

\vspace{.5cm}

\questao Sejam $f : A \to B$ e $g : B \to A$ fun\c{c}\~oes tais que $g \comp f$ \'e sobrejetora. Mostre que $g$ \'e sobrejetora.

\noindent\textbf{Solu\c{c}\~ao:} Primeiro temos
\begin{align*}
	f(P \cap Q) &= \{f(x) \mid x \in P \cap Q\}\\
	f(P) &= \{f(z) \mid z \in P\}\\
	f(Q) &= \{f(t) \mid t \in Q\}.
\end{align*}
Seja $y \in f(P \cap Q)$. Ent\~ao existe $x \in P \cap Q$ tal que $y = f(x)$. Como $x \in P \cap Q$, ent\~ao $x \in P$ e $x \in Q$. Logo $y = f(x)$ com $x \in P$ e $y = f(x)$ com $x \in Q$. Ou seja, $y \in f(P)$ e $y \in f(Q)$. Portanto $y \in f(P) \cap f(Q)$.

Agora seja $y \in f(P) \cap f(Q)$. Da{\'\i} $y \in f(P)$ e $y \in f(Q)$. Assim $y = f(x_1)$ com $x_1 \in P$ e $y = f(x_2)$ com $x_2 \in Q$. Logo $y = f(x_1) = f(x_2)$ e como $f$ \'e injetora, segue que $x_1 = x_2$. Portanto $y = f(x_1)$ com $x_1 = x_1 \in P \cap Q$, isto \'e, $y \in f(P \cap Q)$.

Com isso $f(P \cap Q) = f(P) \cap f(Q)$.

\end{document}