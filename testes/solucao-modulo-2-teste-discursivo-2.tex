%!TEX program = xelatex
% !TEX encoding = ISO-8859-1
\def\ano{2020}
\def\semestre{1}
\def\disciplina{\'Algebra 1}
\def\turma{C}

\documentclass[12pt]{exam}

\usepackage{caption}
\usepackage{amssymb}
\usepackage{amsmath,amsfonts,amsthm,amstext}
\usepackage[brazil]{babel}
% \usepackage[latin1]{inputenc}
\usepackage{graphicx}
\graphicspath{{/ArquivosLinux/OneDrive/imagens-latex/}{D:/OneDrive - unb.br/imagens-latex/}}
\usepackage{enumitem}
\usepackage{multicol}
\usepackage{answers}
\usepackage{tikz,ifthen}
\usetikzlibrary{lindenmayersystems}
\usetikzlibrary[shadings]
\Newassociation{solucao}{Solution}{ans}
\newtheorem{exercicio}{}

\setlength{\topmargin}{-1.0in}
\setlength{\oddsidemargin}{0in}
\setlength{\textheight}{10.1in}
\setlength{\textwidth}{6.5in}
\setlength{\baselineskip}{12mm}

\extraheadheight{0.7in}
\firstpageheadrule
\runningheadrule
\lhead{
        \begin{minipage}[c]{1.7cm}
        \includegraphics[width=1.7cm]{unb.pdf}
        \end{minipage}%
        \hspace{0pt}
        \begin{minipage}[c]{4in}
          {Universidade de Brasília} --
          {Departamento de Matemática}
\end{minipage}
\vspace*{-0.8cm}
}
% \chead{Universidade de Brasília - Departamento de Matemática}
% \rhead{}
% \vspace*{-2cm}

\extrafootheight{.5in}
\footrule
\lfoot{\disciplina\ - \semestre$^o$/\ano\ - Módulo \numeromodulo}
\cfoot{}
\rfoot{Página \thepage\ de \numpages}

\newcounter{exercicios}
\renewcommand{\theexercicios}{\arabic{exercicios}}

\newenvironment{questao}[1]{
\refstepcounter{exercicios}
\ifx&#1&
\else
   \label{#1}
\fi
\noindent\textbf{Exercício {\theexercicios}:}
}

\newcommand{\resp}[1]{
\noindent{\bf Exercício #1: }}

\def\ano{2024}
\def\semestre{1}
\def\disciplina{Álgebra 1}
\def\nomeabreviado{Álgebra 1}
\def\turma{1}

\newcommand{\im}{{\rm Im\,}}
\newcommand{\dlim}[2]{\displaystyle\lim_{#1\rightarrow #2}}
\newcommand{\minf}{+\infty}
\newcommand{\ninf}{-\infty}
\newcommand{\cp}[1]{\mathbb{#1}}
\newcommand{\sub}{\subseteq}
\newcommand{\n}{\mathbb{N}}
\newcommand{\z}{\mathbb{Z}}
\newcommand{\rac}{\mathbb{Q}}
\newcommand{\real}{\mathbb{R}}
\newcommand{\complex}{\mathbb{C}}

\newcommand{\vesp}[1]{\vspace{ #1  cm}}

\newcommand{\compcent}[1]{\vcenter{\hbox{$#1\circ$}}}
\newcommand{\comp}{\mathbin{\mathchoice
        {\compcent\scriptstyle}{\compcent\scriptstyle}
        {\compcent\scriptscriptstyle}{\compcent\scriptscriptstyle}}}
\renewcommand{\sin}{{\rm sen\,}}
\renewcommand{\tan}{{\rm tg\,}}
\renewcommand{\csc}{{\rm cossec\,}}
\renewcommand{\cot}{{\rm cotg\,}}
\renewcommand{\sinh}{{\rm senh\,}}

\begin{document}
	\begin{center}
	{\Large\bf \disciplina\ - Turma \turma\ -- \semestre$^{o}$/\ano} \\ \vspace{9pt} {\large\bf
	Teste Discursivo 2 - M\'odulo 2 - Resolu\c{c}\~ao}\\
	\vspace{9pt} Prof. Jos{\'e} Ant{\^o}nio O. Freitas
	\end{center}
	\hrule

	\vspace{.6cm}

	Seja $f : A \to B$ um homomorfismo de anéis. Mostre que:
	\begin{enumerate}[label={\roman*})]
		\item Se $D$ é um subanel de $B$, então $f^{-1}(D)$ é um subanel de $A$.

		\item Se $J$ é um ideal de $B$, então $f^{-1}(J)$ é um ideal de $A$.
	\end{enumerate}
	
	\noindent\textbf{Solu\c{c}\~ao:}

	\begin{enumerate}[label={\roman*})]
		\item Primeiro precisamos mostrar que $f^{-1}(D) \ne \emptyset$. Mas $D$ é um subanel de $B$ e então $0_B \in D$. Como $f$ é um homomorfismo de anéis, então $f(0_A) = 0_B$. Assim $0_A \in f^{-1}(D)$, isto é, $f^{-1}(D) \ne \emptyset$.

		Agora sejam $x$, $y \in f^{-1}(D)$. Logo
		\begin{align*}
			f(x) &\in D\\
			f(y) &\in D.
		\end{align*}
		Mas $D$ é um subanel de $B$, logo $f(x) + (-f(y)) \in D$ e $f(x)f(y) \in D$. Daí como $f$ é um homomorfismo de anéis:
		\begin{align*}
			f(x) + (-f(y)) &= f(x) + f(-y) = f(x + (-y))\\
			f(x)f(y) &= f(xy),
		\end{align*}
		ou seja, $f(x + (-y)) \in D$ e $f(xy) \in D$. Assim $x + (-y) \in f^{-1}(D)$ e $xy \in f^{-1}(D)$. Portanto $f^{-1}(D)$ é um subanel de $A$, como queríamos.

		\item Primeiro precisamos mostrar que $f^{-1}(J) \ne \emptyset$. Mas como $J$ é ideal de $B$, então $0_B \in J$. Novamente, do fato de $f$ ser um homomorfismo, segue que $f(0_A) = 0_B$, daí $0_A \in f^{-1}(J)$. Agora, para mostrar que $f^{-1}(J)$ é um ideal precisamos que
		\begin{enumerate}
			\item para todos $x$, $y \in f^{-1}(J)$, temos $x - y \in f^{-1}(J)$.
			\item Para todo $\alpha \in A$ e todo $x \in f^{-1}(J)$, temos $\alpha\cdot x \in f^{-1}(J)$.
		\end{enumerate}

		Para o item \textit{(a)} sejam $x$, $y \in f^{-1}(J)$. Assim, $f(x)$, $f(y) \in J$. Mas $J$ é ideal de $B$, logo $f(x) - f(y) \in J$. Agora, como $f$ é homomorfismo, então
		\[
			f(x) - f(y) = f(x - y)
		\]
		e então $f(x - y) \in J$, ou seja, $x - y \in f^{-1}(J)$.

		Para o item \textit{(b)} sejam $\alpha \in A$ e $x \in f^{-1}(J)$. Então $f(x) \in J$. Agora, novamente, como $J$ é ideal e $f(\alpha) \in B$ então $f(\alpha)f(x) \in J$. Usando que $f$ é homomorfismo temos
		\[
			f(\alpha)f(x) = f(\alpha x)
		\]
		e então $f(\alpha x) \in J$. Logo $\alpha x \in f^{-1}(J)$.

		Portanto $f^{-1}(J)$ é um ideal	 de $A$, como queríamos.
	\end{enumerate}
\end{document}