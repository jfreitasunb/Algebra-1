%!TEX program = xelatex
%!TEX root = Algebra_1.tex
%%Usar makeindex -s indexstyle.ist arquivo.idx no terminal para gerar o {\'\i}ndice remissivo agrupado por inicial
%%Ap\'os executar pdflatex arquivo
\chapter{No{\c c}{\~o}es de Teoria de Conjuntos}
\section{Conceitos b{\'a}sicos}

Um conjunto {\'e} uma ``cole{\c c}{\~a}o" ou ``fam{\'\i}lia" de elementos.

Usaremos letras mai{\'u}sculas do alfabeto para denotar os conjuntos e denotaremos elementos por letras min{\'u}sculas do alfabeto.

Dado um conjunto $A$, para indicar o fato de que $x$ {\'e} um elemento de $A$, escrevemos:
\[
x \in A.
\]

Para dizer que um elemento $x$ n{\~a}o pertence ao conjunto $A$, escrevemos:
\[
x \notin A.
\]

Um conjunto sem elementos {\'e} chamado de \textbf{vazio} ou \textbf{conjunto vazio}. Tal conjunto {\'e} denotado por $\emptyset$.

Dado um conjunto $A$ e $x$ um elemento, ocorre sempre o uma das seguintes situações:
\[
x \in A \mbox{ ou } x \notin A.
\]

Al{\'e}m disso, para dois elementos $x$, $y \in A$, ocorre exatamente uma das seguinte situações:
\[
x = y \mbox{ ou } x \neq y.
\]

\section{Descri{\c c}{\~a}o de um conjunto}

Um conjunto $A$ pode ser dado pela simples listagem dos seus elementos, como por exemplo:
\begin{align*}
	A= \{1,2,3,4,5\}\\
	B = \{verdade, falso\}.
\end{align*}

Um conjunto tamb{\'e}m pode ser dado pela descri{\c c}{\~a}o das propriedades dos seus elementos, como por exemplo:
\[
A = \{n \mid n \mbox{ \'e m{\'u}ltiplo de } 2\} = \{2,4,6,...\}.
\]

\section{Alguns conjuntos importantes}
\begin{enumerate}
\item $\n = \{1,2,3,...\}$ o conjunto do n{\'u}meros naturais.
\item $\Z = \{...,-2,-1,0,1,2,...\}$ o conjunto dos n{\'u}meros inteiros.
\item $\n_0 = \{0,1,2,3,...\}$ o conjunto dos n{\'u}meros inteiros n{\~a}o negativos.
\item $\real $ o conjunto dos n{\'u}meros reais.
\item $\real^*$ o conjunto dos n{\'u}meros reais n{\~a}o nulos.
\item $\rac = \left\{\dfrac{p}{q} \mid p,q \in \Z, q \neq 0 \right\}$ o conjunto dos n{\'u}meros racionais.
\end{enumerate}

\section{Propriedades dos conjuntos}

\begin{definicao}
	Dados dois conjuntos $A$ e $B$, dizemos que $A$ e $B$ s{\~a}o iguais se, e somente se, eles t{\^e}m os mesmos elementos. Ou seja, para todo $x \in A$ temos que $x \in B$ e para todo $y \in B$ temos $y \in A$.
\end{definicao}

Se $A$ e $B$ s{\~a}o iguais, escrevemos $A = B$
\[ \{1,2,3,4\} = \{3,2,1,4\} \]
\[ \{1,2,3\} \neq \{2,3\} \]

\begin{definicao}
	Se $A$ e $B$ s{\~a}o dois conjuntos, dizemos que $A$ {\'e} um \textbf{subconjunto} de $B$ ou que $A$ \textbf{está contido} em $B$ ou que $B$ \textbf{contém} $A$ se todo elemento de $A$ for elemento de $B$. Ou seja, se para todo elemento $x \in A$, temos $x \in B$. Nesse caso, escrevemos $A \subseteq B$ ou $B \supseteq A$.
\end{definicao}


Caso $A$ seja um subconjunto de $B$ mas n{\~a}o {\'e} igual a $B$, escrevemos:
\[
A \subsetneq B.
\]

Nesse caso, dizemos que $A$ {\'e} um subconjunto pr{\'o}prio de $B$.

Para dizer que $A$ n{\~a}o est{\'a} contido em $B$, escrevemos $A \nsubseteq B$

Usando a definição de continência podemos definir igualdade de conjuntos da seguinte forma: dois conjuntos $A$ e $B$ são iguais se, e somente se, $A \subseteq B$ e $B \subseteq A$. Ou seja, se $A = B$ ent{\~a}o $A \subseteq B$ e $B \subseteq A$, por outro lado, se $A \subseteq B$ e $B \subseteq A$, ent{\~a}o $A = B$.

Quando $A$ e $B$ n{\~a}o s{\~a}o iguais, escrevemos $A \neq B$. Para que $A \neq B$ devemos ter $A \nsubseteq B$ ou $B \nsubseteq A$.

\subsection{Propriedades da contin{\^e}ncia}
Para quaisquer 3 conjuntos $A,B$ e $C$ temos
\begin{enumerate}
\item $A\subseteq A$ (Lei reflexiva)
\item Se $A\subseteq B \mbox{ e } B\subseteq A$, ent{\~a}o $A=B$ (Lei anti-sim{\'e}trica)
\item Se $A\subseteq B$ e $B\subseteq C$, ent{\~a}o $A\subseteq C$ (Lei transitiva)
\end{enumerate}

Considere os seguintes conjuntos:
\[A = \{ n \in \mathbb{N} \mid n \mbox{ {\'e} m{\'u}ltiplo de } 2\}=\{2,4,6,...\}\]
\[ B = \{n\in\mathbb{N} \mid n \mbox{ {\'e} m{\'u}ltiplo de } 3\}=\{3,6,9,...\}\]

Neste caso, $2 \in A$ e $2 \in B$, logo $A \nsubseteq B$. Por outro lado, $3 \in B$ e $3 \notin A$, logo $B \nsubseteq A$.

Assim, dados 2 conjuntos $A$ e $B$, nem sempre temos que $A \subseteq B$ ou $B \subseteq A$.

\begin{proposicao} Seja $A$ um conjunto. Ent{\~a}o $ \emptyset \subseteq A$.\end{proposicao}

\textbf{Demonstra{\c c}{\~a}o} Suponha que $ \emptyset \nsubseteq A$. Logo existe $x \in \emptyset$ e $x \notin A$. Mas por defini{\c c}{\~a}o o conjunto vazio n{\~a}o cont{\'e}m elementos. Logo {\'e} um absurdo ou uma contradi{\c c}{\~a}o existir $z \in \emptyset$. Portanto, $ \emptyset \subseteq A$, como quer{\'\i}amos demonstrar.\#

\section{Rela{\c c}{\~o}es entre conjuntos}

\subsubsection{Intersec{\c c}{\~a}o}

\begin{definicao}[Intersec{\c c}{\~a}o] Sejam $A$ e $B$ dois conjuntos. Definimos a intersec{\c c}{\~a}o de $A$ e $B$ como sendo o conjunto $A \cap B$ cujos elementos pertencem ao conjunto $A$ e $B$ simultaneamente. Assim,
\[ A \cap B = \{x \mid x \in A\mbox{ e }  x \in B\}\]
\end{definicao}

Exemplo: Sejam
\[ A = \{1,2,3\},\ B = \{2,3,4\}\]
\[A \cap B = \{2,3\}\]

\begin{proposicao} Sejam $A$ e $B$ dois conjuntos. Ent{\~a}o
\[(A \cap B) \subseteq A  \mbox{ e } (A \cap B) \subseteq B.\]
\end{proposicao}

\textbf{Demonstra{\c c}{\~a}o} Seja $x \in A \cap B$ um elemento arbitr{\'a}rio. Logo $x \in A$ e $x \in B$. De $x \in A$ temos que $A \cap B \subseteq A$. De $x \in B$ temos que $A \cap B \subseteq B$. Como quer{\'\i}amos demonstrar.\#

\subsubsection{Uni{\~a}o}

\begin{definicao}[Uni{\~a}o] Sejam $A$ e $B$ dois conjuntos. Definimos a uni{\~a}o de $A$ com $B$ como sendo o conjunto $A \cup B$, cujos elementos pertencem ao conjunto $A$ ou ao conjunto $B$. Assim,
\[A \cup B = \{x \mid x \in A \mbox{ ou } x \in B\}.\]
\end{definicao}

Exemplo: Sejam
\[A = \{1,2,3\},\ B = \{2,3,4\}\]
\[A \cup B = \{1,2,3,4\}\]

O conceito de uni{\~a}o($ \cup $) e intersec{\c c}{\~a}o($ \cap $) pode ser estendido para mais de 2 conjuntos.

\subsubsection{Uni{\~a}o e Intersec{\c c}{\~a}o de m{\'u}ltiplos conjuntos}
\begin{definicao}[Uni{\~a}o e Intersec{\c c}{\~a}o de m{\'u}ltiplos conjuntos] Sejam $A_{1},...,A_{n}$ $n$ conjuntos dados. Ent{\~a}o
\[A_{1} \cup A_{2} \cup \cdots \cup A_{n}= \displaystyle\bigcup_{k=1}^{n} A_{k}\]
{\'e} o conjunto dos elementos $x$ tais que $x$ pertence a pelo menos um dos conjuntos $A_{1}$, ..., $A_{n}$. Agora,
\[A_{1} \cap \cdots \cap A_{n} = \displaystyle\bigcap_{k=1}^{n}A_{k}\]
{\'e} o conjunto dos elementos $x$ que pertencem a todos os conjuntos $A_{1}$, ..., $A_{n}$ simultaneamente.
\end{definicao}

Quando a intersec{\c c}{\~a}o de dois ou mais conjuntos {\'e} vazia, dizemos que eles s{\~a}o conjuntos disjuntos.

Se $C = A \cup B$, tais que $A \cap B = \emptyset$, dizemos que $C$ {\'e} uma uni{\~a}o disjunta de $A$ e $B$. Neste caso, escrevemos
\[C = A \sqcup B\]

\begin{proposicao} Sejam $A,\ B$ e $C$ tr{\^e}s conjuntos, ent{\~a}o
\begin{enumerate}
\item $A\cap(B\cup C)=(A\cap B)\cup(A\cap C)$
\item $A\cup(B\cap C)=(A\cup B)\cap(A\cup C)$
\end{enumerate}
\end{proposicao}

\textbf{Demonstra{\c c}{\~a}o}:
\begin{enumerate}
\item Precisamos mostrar que
\[A\cap(B\cup C)\subseteq(A\cap B)\cup(A\cap C)\]
e
\[(A\cap B)\cup(A\cap C)\subseteq A\cap(B\cup C).\]

Seja $x\in A\cap(B\cup C)$. Logo $x\in A$ e $x\in B\cup C$. Agora, de $x\in B\cup C$, temos que $x\in B$ ou $x\in C$. Suponha que $x\in B$. Como $x\in A$, segue que $x\in A\cap B$. Assim, $x\in(A\cap B)\cup(A\cap C)$, ou seja, $A\cap(B\cup C)\subseteq(A\cap B)\cup(A\cap C)$. Por outro lado, se $x\in C$, ent{\~a}o $x\in C$, ent{\~a}o $x\in A\cap C$ e da{\'\i} $x\in(A\cap B)\cup(A\cap C)$, logo $A\cap(B\cup C)\subseteq(A\cap B)\cup(A\cap C)$.

Portanto,
\[A\cap(B\cup C)\subseteq(A\cap B)\cup(A\cap C).\]

Reciprocamente, seja $x\in(A\cap B)\cup(A\cap C)$. Assim, $x\in A\cap B$ ou $x\in A\cap C$. Suponha que $x\in A\cap B$. Da{\'\i}, $x\in A$ e $x\in B$. Como $x\in B$, segue que $x\in B\cup C$ e ent{\~a}o $x\in A\cap(B\cup C)$, ou seja, $(A\cap B)\cup(A\cap C)\subseteq A\cap(B\cup C)$. Agora, supondo que $x\in A\cap C$. Da{\'\i} $x\in A$ e $x\in C$. Desse modo, $x\in B\cup C$, ent{\~a}o $x\in A\cap(B\cup C)$, isto {\'e}
\[(A\cap B)\cup(A\cap C)\subseteq A\cap(B\cup C).\]

Portanto temos \[A\cap(B\cup C)=(A\cap B)\cup(A\cap C).\]

Como quer{\'\i}amos demonstrar.\#
\item O mesmo racioc{\'\i}nio se aplica, \textit{Mutatis Mutandis}, {\`a} segunda demonstra{\c c}{\~a}o.
\end{enumerate}

\subsubsection{Diferen{\c c}a de Conjuntos}
\begin{definicao}[Diferen{\c c}a de Conjuntos] Dados dois conjuntos $A$ e $B$, definimos a diferen{\c c}a dos conjuntos $A$ e $B$, denotado $A-B$ (ou $A\backslash B)$
\[A - B = \{x | x \in A \mbox{ e } x \notin B\}.\]
\end{definicao}

Exemplos:
\begin{enumerate}
\item $A=\{1,2,3,5,4\}$, $B=\{2,3,6,8\}$, $A-B=\{1,4,5\}$, $B-A=\{6,8\}$
\item $A=\{2,4,6,8,10,...\}$,  $B=\{3,6,9,12,15,...\}$, $A-B=\{2,4,8,10,14,16,...\}$, $B - A=\{3,9,15,21,...\}$
\end{enumerate}

\subsubsection{Complementar}

\begin{definicao}[Complementar] Dados dois conjuntos $A$ e $E$ tais que $A\subseteq E$, definimos o complementar de $A$ em $E$, denotado $A^{C}$ ou $C_{E}(A)$, como
\[C_{E}(A) = \{ x \in E \mid x \notin A \}.\]
\end{definicao}

Observa{\c c}{\~o}es:
\begin{enumerate}
\item Se $A = E$, ent{\~a}o $C_{A}(A)=\{ x \in A \mid x \notin A \}=\emptyset$
\item $(A^{C})^{C}=\{x \in E \mid x \notin A^{C}\} = \{ x \in E \mid x \in A \}=A$
\end{enumerate}

Exemplo:\\
$A=\{1,2,3,4\},\ E=\{1,2,3,5,4,0,8,9\}$\\
$A^{c}=\{0,8,9\}$

\begin{proposicao} Sejam $A,B,E$ conjuntos. Se $A\subseteq B\subseteq E$, ent{\~a}o $C_{E}(B)\subseteq C_{E}(A)$.\end{proposicao}

\textbf{Demonstra{\c c}{\~a}o}: Seja $x\in B^{c}$, logo $x\notin B$ e assim $x\notin A$, pois $A\subseteq B$. Ent{\~a}o $x\in A^{c}$, ou seja, $B^{c}\subseteq A^{c}$. (Q.E.D.).\#\\

\begin{proposicao} Sejam $A,B,E$ tr{\^e}s conjunto tais que $A\subseteq E$ e $B\subseteq E$. Ent{\~a}o:
\begin{enumerate}
\item $(A\cup B)^{c}=A^{c}\cap B^{c}$
\item $(A\cap B)^{c}=A^{c}\cup B^{c}$
\end{enumerate}
\end{proposicao}

\textbf{Demonstra{\c c}{\~a}o}: Seja $x\in(A\cup B)^{c}$. Logo $x\notin A\cup B$, assim $x\notin A\wedge x\notin B$. Da{\'\i}, $x\in A^{c}\wedge x\in B^{c}$, isto {\'e}, $x\in A^{c}\cap B^{c}$. Desse modo,
\begin{equation}
(A\cup B)^{c}\subseteq A^{c}\cap B^{c}
\end{equation}

Por outro lado, se $x\in A^{c}\cap B^{c}$, ent{\~a}o $x\in A^{c}\wedge x\in B^{c}$. Da{\'\i}, $x\notin A\wedge x\notin B$, ou seja, $x\notin A\cup B$, logo $x\in (A\cup B)^{c}$. Desse modo
\begin{equation}
A^{c}\cap B^{c}\subseteq(A\cup B)^{c}
\end{equation}

Portanto, de (2.1) e (2.2) temos \[(A\cup B)^{c}=A^{c}\cap B^{c}\]

(Q.E.D).\#\\

\subsubsection{Produto Cartesiano}

\begin{definicao}[Produto Cartesiano] Dados 2 conjuntos $A$ e $B$, definimos o produto cartesiano $A$ e $B$ por
\begin{center}
$A\times B=\{(a,b)/a\in A,b\in B\}$
\end{center}
\end{definicao}

Como o conjunto dos pares ordenados $(a,b)$, onde $a$ percorre $A$ e $b$ percorre $B$.

Dados $(a,b),(c,d)\in A$x$B$, temos $(a,b)=(c,d)$ se, e somente se, \[a=c\wedge b=d\]

Em geral, $A$x$B\neq B$x$A$.

Exemplo:\\
$A=\{1,2\},\ B=\{3\}$\\
$A$x$B=\{(1,3),(2,3)\}$\\
$B$x$A=\{(3,1),(3,2)\}$

\subsubsection{Conjunto partes}

\begin{definicao}[Conjunto Partes] Para qualquer conjunto $A$, indicamos por $P(A)$\[P(A)=\{X/X\subseteq A\}\] o conjunto das partes de $A$.\end{definicao}

Os elementos desse conjunto s{\~a}o todos os subconjuntos de $A$. Dizer que $x\in P(A)$ significa que $x\in A$. Particularmente, temos $\emptyset\in P(A)$ e $A\in P(A)$.

Exemplos:
\begin{enumerate}
\item $A=\emptyset,\ P(A)=\{\emptyset\}$
\item $B=\{b\},\ P(B)=\{\emptyset,B\}$
\item $C=\{a,b,c\}$\\
$P(C)=\{\emptyset, \{a\}, \{b\},\{c\},\{a,b\},\{a,c\},\{b,c\},C\}$
\item $D=\mathbb{R},P(D)=\{X/X\subseteq\mathbb{R}\}$, por exemplo $\mathbb{Q}\in P(D)$
\end{enumerate}