%!TEX program = xelatex
%!TEX root = Algebra_1.tex
\chapter{An{\'e}is}

\section{Conceitos B\'asicos} % (fold)
\label{sec:conceitos_basicos}

\begin{definicao}
	Seja $A$ um conjunto n{\~a}o vazio. Dizemos que $A$ est{\'a} munido (ou equipado) de uma \textbf{opera{\c c}{\~a}o bin{\'a}ria} quando existe uma fun{\c c}{\~a}o
	\begin{align*}
		&\Delta : A \times A \to A\\
		&(a,b) \longmapsto a\Delta b
	\end{align*}
	Uma opera{\c c}{\~a}o bin{\'a}ria tamb{\'e}m {\'e} chamada de uma \textbf{opera{\c c}{\~a}o interna} em $A$.
\end{definicao}

\begin{exemplos}
	\begin{enumerate}[label={\arabic*})]
		\item A soma usual nos conjuntos $\z$, $\rac$, $\real$ e $\complex$ {\'e} uma opera{\c c}{\~a}o bin{\'a}ria.

		\item A multiplica\c{c}\~ao usual nos conjuntos $\z$, $\rac$, $\real$ e $\complex$ {\'e} uma opera{\c c}{\~a}o bin{\'a}ria.

		\item Seja $m > 1$, $m \in \z$ fixo. A soma \eqref{soma_modulo_m} e a multiplica\c{c}\~ao \eqref{multiplicacao_modulo_m} definidos em $\z_m = \{\overline{0},\overline{1},...,\overline{m-1}\}$ s\~ao uma opera\c{c}\~oes bin\'arias.

		\item A opera\c{c}\~ao $\div$ em $\rac^{*}$ {\'e} uma opera{\c c}{\~a}o bin{\'a}ria.
		\item J\'a em $\n$, $\z$, $\z^{*}$ e em $\rac$ a opera\c{c}\~ao $\div$ n{\~a}o {\'e} uma opera{\c c}{\~a}o bin{\'a}ria.
	\end{enumerate}
\end{exemplos}

\begin{definicao}\label{definicaoanel}
	Seja $A$ um conjunto n{\~a}o vazio no qual est\~ao definidas duas opera{\c c}{\~o}es bin\'arias $\oplus$ e $\otimes$, chamadas \textit{soma} e \textit{produto}.  Dizemos que $(A, \oplus, \otimes)$ {\'e} um \textbf{anel} quando as seguintes condi{\c c}{\~o}es s{\~a}o verdadeiras:
	\begin{enumerate}[label={\roman*})]
		\item \textbf{Associatividade}: para todos $x$, $y$, $z \in A$ vale que
		\[
			(x \oplus y) \oplus z = x \oplus (y \oplus z).
		\]
		Essa propriedade {\'e} chamada \textbf{propriedade associativa} da soma.\label{associatividadesoma}

		\item \textbf{Comutatividade}: Para todos $x$, $y \in A$ vale\label{comutatividadesoma}
		\[
			x \oplus y = y \oplus x.
		\]

		\item \textbf{Elemento Neutro}: Existe em $A$ um elemento denotado por $0$ (zero) ou $0_{A}$ tal que para todo elemento $x \in A$ vale
		\[
			x \oplus 0_A = x = 0_A \oplus x.
		\]
		Tal elemento $0_A$ \'e chamado de \textbf{elemento neutro da soma} ou simplesmente \textbf{elemento neutro}.

		\item \textbf{Elemento Oposto}: Para cada elemento $x \in A$, existe $y \in A$ tal que
		\[
			x \oplus y = 0_A = y \oplus x.
		\]
		Tal elemento $y$ \'e chamado de \textbf{oposto aditivo} de $x$ ou simplesmente \textbf{oposto} de $x$.

		\item \textbf{Associatividade}: Para todos $x$, $y$, $z \in A$, vale que\label{associatividadeproduto}
		\[
			(x\otimes y) \otimes z = x\otimes (y\otimes z).
		\]

		\item \textbf{Distributividade}: Para todos $x$, $y$, $z \in A$ vale
		\[
			(x \oplus y)\otimes z = x\otimes z \oplus y\otimes z.
		\]
		Essa propriedade {\'e} chamada \textbf{distributiva da soma em rela{\c c}{\~a}o ao produto}.\label{distributividadesomaproduto}

		\item \textbf{Distributividade}: Para todos $x$, $y$, $z \in A$ vale
		\[
			x\otimes(y \oplus z) = x\otimes y \oplus x\otimes z.
		\]
		Essa {\'e} a propriedade \textbf{distributiva do produto em rela{\c c}{\~a}o {\`a} soma}.\label{distributividadeprodutosoma}
	\end{enumerate}
\end{definicao}

\begin{observacoes}
	Seja $(A, \oplus, \otimes)$ um anel.
	\begin{enumerate}[label={\arabic*})]
		\item \textbf{Comutatividade}: Se para todos $x$, $y \in A$ vale
		\[
			x \otimes y = y \otimes x.
		\]
		Dizemos que $(A, \oplus, \otimes)$ {\'e} um \textbf{anel comutativo}.

		\item \textbf{Unidade}: Se existe em $A$ um elemento denotado por $1$ ou $1_{A}$ tal que
		\[
			x \otimes 1 = x = 1 \otimes x,
		\]
		para todo $x \in A$, ent{\~a}o dizemos que $(A, \oplus, \otimes)$ \'e um \textbf{anel com unidade} ou um \textbf{anel unit{\'a}rio}. O elemento $1_A$ {\'e} chamado de \textbf{unidade} de $A$ ou \textbf{elemento neutro da multiplica\c{c}\~ao} de $A$.

		\item Se um anel $(A, \oplus, \otimes)$ satisfaz as duas propriedades anteriores dizemos que $(A, \oplus, \otimes)$ \'e um \textbf{anel comutativo com unidade} ou um \textbf{anel comutativo unit\'ario}.

		\item Seja $(A, \oplus, \otimes)$ um anel. Quando n\~ao houver chance de confus\~ao com rela\c{c}\~ao \`as opera\c{c}\~oes envolvidas diremos simplesmente que $A$ \'e um anel.
	\end{enumerate}
\end{observacoes}

\begin{exemplos}
	\begin{enumerate}[label={\arabic*})]
		\item $(\z,+,.)$, $(\rac,+,.)$, $(\real,+,.)$, $(\complex,+,.)$, $(\z_m, \oplus, \otimes)$ s{\~a}o an{\'e}is comutativos e com unidade.

		\item  Seja $A = \z =\{f : \z \to \z \mid f \mbox{ {\'e} uma fun{\c c}{\~a}o}\}$. Dadas duas fun{\c c}{\~o}es quaisquer $f$, $g \in A$, definimos $f\oplus g:\z \to \z$ e $f \otimes g:\z \to \z$ como:
		\begin{align*}
			(f\oplus g)(x) &= f(x) + g(x)\\
			(f\otimes g)(x) &= f(x)g(x)
		\end{align*}
		para todo $x \in \z$. Assim $(A, \oplus, \otimes)$ \'e um anel. De fato:
		\begin{enumerate}[label={\roman*})]
			\item Para todo $x \in \z$
			\begin{align*}
				[(f \oplus g) \oplus h](x) &= (f \oplus g)(x) + h(x) = (f(x) + g(x)) + h(x)\\
				&= f(x) + (g(x) + h(x)) = f(x) + (g \oplus h)(x)\\ &= [f \oplus (g \oplus h)](x)
			\end{align*}
			para todos $f$, $g$ e $h \in A$.

			\item Para todo $x \in \z$
			\[
				(f\oplus g)(x) = f(x) + g(x) = g(x) + f(x) = (g\oplus f)(x),
			\]
			portanto $f\oplus g = g\oplus f$ para todos $f$, $g \in A$.

			\item $0_A : \z \to \z$ dada por $0_A(x) = 0$ para todo $x \in \z$. Da{\'\i} para todo $x \in \z$
			\[
				(f \oplus 0_A)(x) = f(x) + 0_A(x) = f(x) + 0 = f(x)
			\]
			para todo $f \in A$. Logo $f + 0_A = f$ para todo $f \in A$. Logo $0_A$ \'e o elemento neutro da soma em $A$.

			\item Dada $f \in A$, defina $g : \z \to \z$ por $g(x) = -f(x)$ para todo $x \in \z$. Da{\'\i} para todo $x \in \z$ temos
			\[
				(f \oplus g)(x) = f(x) + g(x) = f(x) + (-f(x)) = 0.
			\]
			Logo $g(x) = -f(x)$ \'e o oposto de $f \in A$.

			\item Para todo $x \in \z$
			\begin{align*}
				[(f \otimes g)\otimes h](x) &= (f \otimes g)(x)h(x) = (f(x)g(x))h(x)\\ &= f(x)(g(x)h(x)) = f(x)(g \otimes h)(x)\\ &= [f\otimes (g \otimes h)](x)
			\end{align*}
			para todos $f$, $g$ e $h \in \z$.

			\item Para todo $x \in \z$
			\begin{align*}
				[(f \oplus g)\otimes h](x) &= (f \oplus g)(x)h(x) = (f(x) + g(x))h(x) \\ &= f(x)h(x) + g(x)h(x)
				= (f\otimes g)(x) + (g \otimes h)(x)\\
				&= [(f \otimes g) \oplus (g \otimes h)](x)
			\end{align*}
			para todos $f$, $g$ e $h \in A$.

			\item Para todo $x \in \z$
			\begin{align*}
				[f\otimes (g \oplus h)](x) &= f(x)(g\oplus h)(x) = f(x)(g(x) + h(x))\\
				&= f(x)g(x) + f(x)h(x) = (f\otimes g)(x) + (f\otimes h)(x)\\
				&= [(f \otimes g) \oplus (f\otimes h)](x)
			\end{align*}
			para todos $f$, $g$ e $h \in A$.
		\end{enumerate}
		Assim $(A, \oplus, \otimes)$ \'e um anel. Al\'em disso, para todo $x \in \z$
		\[
			(f\otimes g)(x) = f(x)g(x) = g(x)f(x) = (g\otimes f)(x)
		\]
		para todos $f$, $g \in A$. Assim a opera\c{c}\~ao $\otimes$ \'e comutativa.

		Mais ainda, definindo $1_A : \z \to \z$ como $1_A(x) = 1$ para todo $x \in \z$ temos
		\[
			(f \otimes 1_A)(x) = f(x)1_A(x) = f(x)\cdot 1 = f(x)
		\]
		para todo $f \in A$. Logo $1_A$ \'e a unidade de $A$.

		Portanto $(A, \oplus, \otimes)$ \'e um anel comutativo com unidade.
	\end{enumerate}
\end{exemplos}

\begin{observacao}
	Seja $(A, \oplus, \cdot)$ um anel. Para simplificar a nota\c{c}\~ao vamos denotar a opera\c{c}\~ao $\oplus$
	por $+$ e a opera\c{c}\~ao $\otimes$ por $\cdot$ e assim escrever simplesmente que $(A, +, \cdot)$ \'e um anel.
\end{observacao}

\begin{proposicao}
	Seja $(A, + , \cdot)$ um anel. Ent\~ao:
	\begin{enumerate}[label={\roman*})]
		\item O elemento neutro {\'e} {\'u}nico.
		\item Para cada $x \in A$ existe um {\'u}nico oposto.
		\item Para todo $x \in A$, $-(-x) = x$.
		\item Dados $x_{1}$, $x_{2}$, \dots, $x_n \in A$, $n \geqslant 2$, ent{\~a}o
		\[
			-(x_1 + x_2 + \dots + x_n) = (-x_1) + (-x_2) + \dots + (-x_n).
		\]
		\item Para todos $\alpha$, $x$, $y \in A$, se $\alpha + x = \alpha + y$, ent{\~a}o $x = y$.
		\item Para todo $x \in A$, $x\cdot 0_A = 0_A = 0_A\cdot x$.
		\item Para todos $x$, $y \in A$, temos $x(-y) = (-x)y = -(xy)$.
		\item Para todos $x$, $y \in A$, $xy = (-x)(-y)$.
	\end{enumerate}
\end{proposicao}
\begin{prova}
	\begin{enumerate}[label={\roman*})]
		\item Suponha que existam $0_1$, $0_2\in A$ elementos neutros de $A$. Assim
		\[
			x + 0_1 = x \quad \mbox{e}\quad x + 0_2 = x
		\]
		para todo $x \in A$. Assim
		\[
			0_1 = 0_1 + 0_2 = 0_2
		\]
		e portanto o elemento neutro \'e \'unico.

		\item De fato, dado $x \in A$ suponha que existam $y_1$, $y_2\in A$ tais que
		\[
			x + y_1 = 0_A \quad \mbox{e}\quad x + y_2 = 0_A.
		\]
		Da{\'\i}
		\[
			y_1 = y_2 + 0_A = y_1 + (x + y_2) = (y_1 + x) + y_2 = 0_A + y_2 =y_2.
		\]
		Logo o oposto de $x$ \'e \'unico  e da{\'\i} ser\'a denotado por $-x$.

		\item Dado $x \in A$, ent\~ao $-x$ {\'e} oposto de $x$, isto {\'e}, $x + (-x) = 0_A$. Logo o oposto de $(-x)$ {\'e} $x$, ou seja, $-(-x) = x$.

		\item Segue usando indu\c{c}\~ao sobre $n$.

		\item Suponha que $\alpha + x = \alpha + y$. Seja $-\alpha$ o oposto de $\alpha$ da{\'\i}
		\begin{align*}
			x &= 0_A + x \\ &= [(-\alpha) + \alpha] + x\\ &= (-\alpha) + (\alpha + x) \\ &= (-\alpha) + (\alpha + y) \\ &= [(-\alpha) + \alpha] + y \\ &= 0_A + y = y
		\end{align*}
		como quer{\'\i}amos.

		\item Temos $x\cdot 0_A + 0_A = x\cdot 0_A = x\cdot(0_A + 0_A) = x\cdot 0_A + x\cdot 0_A$. Assim do item anterior segue que $x\cdot 0_A = 0_A$.

		\item Provemos que $x(-y) = -(xy)$:
		\[
			x(-y) + xy = x((-y) + y) = x\cdot 0_A = 0_A,
		\]
		portanto $-xy = x(-y)$.

		\item Basta usar o caso anterior.
	\end{enumerate}
\end{prova}

\section{An\'eis de Integridade e Suban\'eis} % (fold)
\label{sec:aneis_de_integridade_e_subaneis}

\begin{definicao}
	Um anel comutativo $(A, + , \cdot)$ {\'e} dito ser um \textbf{anel de integridade} quando para todos
	$x$, $y \in A$, se $xy = 0_A$, ent{\~a}o $x = 0_A$ ou $y = 0_A$. Um anel de integridade tamb{\'e}m {\'e} chamado de \textbf{dom{\'\i}nio de integridade} ou simplesmente de \textbf{dom{\'\i}nio}.
\end{definicao}

\begin{observacao}
	Se $x$ e $y$ s{\~a}o elementos n{\~a}o nulos de um anel $A$ tais que $xy = 0_A$, ent{\~a}o $x$ e $y$ s{\~a}o chamados de \textbf{divisores pr{\'o}prios de zero}.
\end{observacao}


\begin{exemplos}
	\begin{enumerate}[label={\arabic*})]
		\item Os an{\'e}is $\z$, $\rac$, $\real$, $\complex$ s{\~a}o an{\'e}is de integridade.

		\item Em geral $\z_m$ n{\~a}o {\'e} anel de integridade, por exemplo, em $\z_4$, $\overline{2} \neq \overline{0}$, no entanto $\overline{2}\otimes \overline{2} = \overline{4} = \overline{0}$.

		\item $M_{n}(\real)$ n{\~a}o {\'e} um anel de integridade, por exemplo, em $M_{2}(\real)$
		\begin{align*}
			A &= \begin{bmatrix}
				1 & 0\\
				0 & 0
			\end{bmatrix} \neq \begin{bmatrix}
				0 & 0\\
				0 & 0
			\end{bmatrix},\qquad
			B = \begin{bmatrix}
				0 & 0\\
				1 & 0
			\end{bmatrix} \neq \begin{bmatrix}
				0 & 0\\
				0 & 0
			\end{bmatrix}\\
			AB & =\begin{bmatrix}
				0 & 0\\
				0 & 0
			\end{bmatrix}.
		\end{align*}

		\item Suponha que $m = nk$, $m > n > 1$ e $m > k > 1$. Logo, em $\z_m$, $\overline{n} \neq \overline{0}$ e $\overline{k} \neq \overline{0}$ e no entanto $\overline{n} \otimes \overline{k} = \overline{m} = \overline{0}$. Logo, se $m$ n{\~a}o {\'e} primo, ent{\~a}o $\z_m$ n{\~a}o {\'e} um anel de integridade. Agora, suponha que $m = p$ primo. Sejam $\overline{x}$, $\overline{y} \in \z_m$ tais que $\overline{x}\otimes \overline{y} = \overline{0}$, ou seja, $xy \equiv 0 \pmod p$. Da{\'\i} $p\mid xy$. Logo $p\mid x$ ou $p\mid y$. Portanto, $\overline{x} = \overline{0}$ ou $\overline{y} = \bar{0}$. Assim, $\z_m$ {\'e} anel de integridade se, e somente se, $m$ {\'e} primo.
	\end{enumerate}
\end{exemplos}



\begin{definicao}
	Seja $(A, +, \cdot)$ um anel. Dizemos que um subconjunto n{\~a}o vazio $B\subseteq A$ {\'e} um \textbf{subanel} de $A$ quando $(B, +, \cdot)$ \'e um anel.
\end{definicao}

\begin{exemplos}
	\begin{enumerate}[label={\arabic*})]
		\item Todo anel $A$ sempre tem dois suban{\'e}is: $\{0_{A}\}$ e $A$, que s{\~a}o chamados de \textbf{suban{\'e}is triviais}.
		\item Em $(\z_4,\oplus,\otimes)$ o conjunto $B = \{\overline{0}, \overline{2}\}$ \'e um subanel.
		\item No anel $\z$, o conjunto $m\z$, $m > 1$ {\'e} um subanel de $\z$.
	\end{enumerate}
\end{exemplos}

\begin{proposicao}\label{proposicao_subanel}
	Seja $(A, +,\cdot)$ um anel. Um subconjunto n{\~a}o vazio $B\subseteq A$ {\'e} um subanel de $A$ se, e somente se, $x + (-y) \in B$ e $x\cdot y \in B$ para todos $x$, $y \in B$.
\end{proposicao}
\begin{prova}
    Precisamos mostrar que
    \begin{enumerate}[label={\roman*})]
        \item Se $B$ é um subanel de $A$, então $x + (-y) \in B$, $x\cdot y \in B$ para todos $x$, $y \in B$.\label{primeiroladosubanel}

        \item Se $x + (-y) \in B$, $x\cdot y \in B$ para todos $x$, $y \in B$, então $B$ é um subanel de $A$.\label{segundoladosubanel}
    \end{enumerate}

    A prova de \textit{\ref{primeiroladosubanel}} é uma consequência direta da definição de subanel.

    Agora para provar \textit{\ref{segundoladosubanel}}, vamos mostrar que $(B, +, \cdot)$ é um anel. Inicialmente obserque que na definição de anel, Definição \ref{definicaoanel}, os itens \ref{associatividadesoma}, \ref{comutatividadesoma}, \ref{associatividadeproduto}, \ref{distributividadesomaproduto} e \ref{distributividadeprodutosoma} são válidos para todos os elementos de $A$ e como $B \subseteq A$, então essas propriedades também são válidas em $B$.

    Como por hipótese $x\cdot y \in B$ para todos $x$, $y \in B$, então o produto $\cdot$ é uma operação binária em $B$.

    Usando agora que $B \ne \emptyset$, seja $x \in B$. Como $x + (-y) \in B$ para todos $x$, $y \in B$, então
    \[
        0_A = x + (-x) \in B.
    \]
    Além disso, como $x \in B$ e $0_A \in B$, então
    \[
        -x = 0_A + (-x) \in B.
    \]
    Finalmente, dados $x$, $y \in B$. Sabemos que $-y \in B$, daí
    \[
        x + y = x + [-(-y)] \in B.
    \]

    Portanto, $(B, +, \cdot)$ é um anel e com isso $(B, +, \cdot)$ é um subanel de $A$.
\end{prova}

\begin{exemplo}
    No conjunto $\z \times \z$ considere as operações $\oplus$ e $\otimes$ definidas por
    \begin{align*}
        (a, b) \oplus (c, d) = (a + c, b + d)\\
        (a ,b) \otimes (c, d) = (ac - bd, ad + bc).
    \end{align*}
    onde $(a, b)$, $(c, d) \in \z \times \z$. Então é fácil ver que $(\z \times \z, \oplus, \otimes)$ é um anel. Além disso, esse anel é comutativo e possui unidade. Sabendo que
    $(\z \times \z, \oplus, \otimes)$ é um anel, quais dos seguintes subconjuntos de $\z \times \z$ s\~ao suban\'eis?
    \begin{enumerate}[label={\alph*})]
        \item $A = \{(x, y) \in \z \times \z \mid x = 0\}$

        \item $B = \{(x, y) \in \z \times \z \mid x + y = 2k,\ k \in \z\}$

        \item $C =  \{(1, y) \in \z \times \z \mid y \in \z\}$
    \end{enumerate}
    \begin{solucao}
        Inicialmente observe que no anel $(\z \times \z, \oplus, \otimes)$ o elemento neutro é $(0,0)$ e o oposto de $(a, b)$ é dado por
        \[
            -(a, b) = (-a, -b).
        \]
        \begin{enumerate}[label={\alph*})]
            \item Observe que apesar de $(0, 0) \in A$, o subconjunto $A$ não é um subanel de $A$ uma vez que tomando $(0, 1)$, $(0, 2) \in A$ temos
                \[
                    (0, 1) \otimes (0, 2) = (0\cdot 0 - 1\cdot 2, 0\cdot 2 + 0\cdot 1) = (-2, 0) \notin A.
                \]

            \item Primeiro note que $(0, 0) \in B$, logo $B \ne \emptyset$ e assim podemos aplicar a Proposição \ref{proposicao_subanel}. Para isso sejam $(a, b)$, $(c, d) \in B$.
                Temos:
                \begin{align*}
                    (a, b) \oplus [-(c, d)] &= (a, b) \oplus (-c, -d) = (a - c, b - d)\\
                    (a, b) \otimes (c, d) &= (ac - bd, ad + bc)
                \end{align*}
                Como $(a, b)$ e $(c, d) \in B$ então
                \begin{align}
                    a + b &= 2k\label{primeira_condicao}\\
                    c + d &= 2l\label{segunda_condicao}
                \end{align}
                onde $k$, $l \in \z$. Assim
                \[
                    (a - c) + (b - d) = (a + b) - (c + d) = 2k - 2l = 2(k - l)
                \]
                e com isso $(a, b) \oplus [-(c, d)] \in B$.

                Agora observe que de \eqref{primeira_condicao} segue que $a$ e $b$ devem ter a mesma paridade, isto é, $a$ e $b$ são
                ambos pares ou ambos ímpares. De modo análogo, de \eqref{segunda_condicao}, o mesmo ocorre com $c$ e $d$. Assim se $a$ e $b$ ou $c$ e $d$ são pares, então
                \[
                    (ac - bd) + (ad + bc)
                \]
                é sempre um número par.

                Agora se $a$, $b$, $c$ e $d$ são todos ímpares, então $ac$, $bd$, $ad$ e $bc$ também são ímpares e daí
                \[
                    (ac - bd) + (ad + bc)
                \]
                é par. Logo $(a, b) \otimes (c, d) \in B$.

                Portanto $B$ é um subanel de $(\z \times \z, \oplus, \otimes)$.

            \item Como $(0, 0) \notin C$ então $C$ não é um subanel de $(\z \times \z, \oplus, \otimes)$.


        \end{enumerate}
    \end{solucao}
\end{exemplo}


\section{Homomorfismo de An\'eis} % (fold)
\label{sec:homomorfismo_de_aneis}

\begin{definicao}
	Um homomorfismo do anel $(A, +, \cdot)$ no anel $(B, \oplus, \otimes)$ {\'e} uma fun{\c c}{\~a}o $f : A \to B$ que satisfaz:
	\begin{enumerate}[label={\roman*})]
		\item $f(x + y) = f(x) \oplus f(y)$, para todos $x$, $y \in A$;
		\item $f(x \cdot y) = f(x)\otimes f(y)$, para todos $x$, $y \in A$.
		% \item $f(1_{A})=1_{B}$, onde $1_{A}$ {\'e} a unidade de A e $1_{B}$ {\'e} a unidade de B
	\end{enumerate}
\end{definicao}

\begin{proposicao}
	Sejam $(A, +, \cdot)$ e $(B, \oplus, \otimes)$ an\'eis e seja $f : A \to B$ um homomorfismo. Ent{\~a}o:
	\begin{enumerate}[label={\roman*})]
		\item $f(0_{A}) = 0_{B}$
		\item $f(-x) = -f(x)$, para todo $x \in A$.
	\end{enumerate}
\end{proposicao}
\begin{prova}
	\begin{enumerate}[label={\roman*})]
		\item Fazendo $x = y = 0_{A}$, temos
		\[
			f(0_A) = f(0_A + 0_A) = f(0_A) \oplus f(0_A)
		\]
		Somando $-f(0_A)$ em ambos os lados obtemos
		\begin{align*}
			f(0_A) \oplus (-f(0_A)) &= (f(0_A)\oplus f(0_A)) \oplus (-f(0_A))\\
			0_B &= f(0_A) \oplus 0_B\\
			f(0_A) &= 0_B.
		\end{align*}

		\item Temos $0_B = f(0_A) = f(x + (-x)) = f(x)\oplus f(-x)$. Assim somando $-f(x)$ em ambos os lados obtemos
		\begin{align*}
			0_B\oplus(-f(x)) &= [f(x)\oplus f(-x)] + (-f(x))\\
			-f(x) &= f(-x) \oplus (f(x) \oplus (-f(x)))\\
			f(-x) &= -f(x)
		\end{align*}
		como quer{\'\i}amos.
	\end{enumerate}
\end{prova}

\begin{exemplo}
    Seja $(\z\times\z, +, \cdot)$ um anel com as seguintes opera\c{c}\~oes
    \begin{align*}
        (a, b) + (c, d) &= (a + c, b + d)\\
        (a, b)\cdot (c, d) &= (ac, ad + bc)
    \end{align*}
    para todos $(a, b)$, $(c, d) \in \z\times\z$.
    Quais das seguintes funções $f : \z\times\z \to \z$ \'e um homomorfismo?
    \begin{enumerate}[label={\alph*})]
        \item $f(a, b) = a$

        \item $g(x, y) = 2x$
    \end{enumerate}
    \begin{solucao}
        Inicialmente observe que o elemento neutro do anel $\z \times \z$ com as operações dadas é $(0, 0)$.
        \begin{enumerate}[label={\alph*})]
            \item Primeiro veja que $f(0, 0) = 0$, então $f$ pode ser um homomorfismo de anéis. Para confirmar sejam $(a, b)$, $(c, d) \in \z \times \z$. Temos
                \begin{align*}
                    f((a, b) + (c, d)) &= f(a + c, b + d) = a + c = f(a, b) + f(c, d)\\
                    f((a, b) \cdot (c, d)) &= f(ac, ad + bc) = ac = f(a, b)f(c, d),
                \end{align*}
                logo $f$ é realmente um homomorfismo de anéis.

            \item Nesse caso também temos $g(0, 0) = 0$ mas nesse caso tomando $(1, 1)$ e $(2,0) \in \z \times \z$ temos
                \[
                    f((1, 1) \cdot (2, 0)) = f(2, 2) = 4 \ne f(1,1)f(2,0)=2\cdot 4.
                \]
                Logo $g$ não é um homomorfismo de anéis.
        \end{enumerate}
    \end{solucao}
\end{exemplo}


\begin{definicao}Seja $f:A\rightarrow B$ um homomorfismo, onde $A$ e $B$ s{\~a}o an{\'e}is. Dizemos que
	\begin{enumerate}[label={\roman*})]
		\item $f$ {\'e} um epimorfismo se $f$ for sobrejetora.
		\item $f$ {\'e} um monomorfismo se $f$ for injetora.
		\item $f$ {\'e} um isomorfismo se $f$ for bijetora.
		\item Quando $A=B$ e $f$ {\'e} um isomorfismo, ent{\~a}o $f$ {\'e} um automorfismo.
	\end{enumerate}
\end{definicao}

\begin{definicao}
	Sejam $(A, +, \cdot)$ e $(B, \oplus, \otimes)$ an\'eis e $f : A \to B$ um homomorfismo de an\'eis. Ent\~ao o subconjunto de $A$ definido por
	\[
		\ker(f) = \{ x \in A \mid f(x) = 0_B\}
	\]
	\'e chamado de \textbf{kernel} ou \textbf{n\'ucleo} de $f$.
\end{definicao}

\begin{exemplo}
    Seja $f : \complex \to M_2(\real)$ dada por
    \[
        f(a + bi) = \begin{bmatrix}
            a & -b\\
            b & a
        \end{bmatrix}.
    \]
    Determine $\ker(f)$. Esse homomorfismo é injetor?
\end{exemplo}

\begin{proposicao}
	Sejam $(A, +, \cdot)$ e $(B, \oplus, \otimes)$ an\'eis e $f : A \to B$ um homomorfismo de an\'eis. Ent\~ao:
	\begin{enumerate}[label={\roman*})]
		\item $\ker(f)$ \'e um subanel de $A$.
		\item $f$ \'e injetora se, e somente se, $\ker(f) = \{0_A\}$.
	\end{enumerate}
\end{proposicao}
\begin{prova}
	\begin{enumerate}[label={\roman*})]
		\item Primeiro note que sendo $f$ \'e um homomorfismo ent\~ao $f(0_A) = 0_B$. Logo $0_A \in \ker(f)$, isto \'e, $\ker(f) \ne \emptyset$.

		Agora dados $x$, $y \in \ker(f)$ precisamos mostrar que $x - y \in \ker(f)$ e $xy \in \ker(f)$, e para mostrar isso basta mostrar que $f(x - y) = 0_B$ e$f(xy) = 0_B$. Inicialmente como $x$, $y \in \ker(f)$ da{\'\i} $f(x) = f(y) = 0_B$. Assim
		\begin{align*}
			f(x - y) &= f(x + (-y)) = f(x) \oplus f(-y) = f(x) \oplus (-f(y)) = 0_B \oplus 0_B = 0_B\\
			f(xy) &= f(x)\otimes f(y) = 0_B \otimes 0_B = 0_B.
		\end{align*}
		Logo $x - y \in \ker(f)$ e $xy \in \ker(f)$. Portanto $\ker(f)$ \'e um subanel de $A$.

		\item Primeiro suponha que $f$ \'e injetora e vamos mostrar que $\ker(f) = \{0_A\}$. Para isso seja $x \in \ker(f)$. Ent\~ao
		\[
			f(x) = 0_B,
		\]
		mas $f$ sendo um homomorfismo temos $f(0_A) = 0_B$. Da{\'\i}
		\[
			f(x) = 0_B = f(0_A).
		\]
		E como $f$ \'e injetora, por hip\'otese, segue que $x = 0_A$. Logo $\ker(f) = \{0_A\}$.

		Agora suponha que $\ker(f) = \{0_A\}$ e vamos mostrar que $f$ \'e injetora. Para isso sejam $x_1$, $x_2 \in A$ tais que $f(x_1) = f(x_2)$. Da{\'\i}
		\begin{align*}
			&f(x_1) = f(x_2)\\
			&f(x_1) \oplus (-f(x_2)) = 0_B\\
			&f(x_1) \oplus f(-x_2) = 0_B\\
			&f(x_1 - x_2) = 0_B.
		\end{align*}
		Logo $x_1 - x_2 \in \ker(f) = \{0_A\}$. Com isso $x_1 - x_2 = 0_A$, isto \'e, $x_1 = x_2$. Portanto $f$ \'e injetora.
	\end{enumerate}
\end{prova}

\begin{proposicao}
	Sejam $(A, +, \cdot)$ e $(B, \oplus, \otimes)$ an\'eis e seja $f : A \to B$ um homomorfismo sobrejetor de an\'eis.
	\begin{enumerate}[label={\roman*})]
		\item Se $A$ tem unidade, ent\~ao $B$ tem unidade e
		\[
			f(1_A) = 1_B.
		\]
		\item Se $A$ tem unidade e $x \in A$ possui inverso multiplicativo, ent\~ao $f(x)$ tem inverso e
		\[
			[f(x)]^{-1} = f(x^{-1}).
		\]
	\end{enumerate}
\end{proposicao}
\begin{prova}
	\begin{enumerate}[label={\roman*})]
		\item Inicialmente como num anel a unidade \'e \'unica, para mostrar que $B$ possui unidade basta mostrar que
		\[
			y\otimes f(1_A) = y = f(1_A)\otimes y
		\]
		para todo $y \in B$. Sendo assim, seja $y \in B$. Como $f$ \'e sobrejetor ent\~ao existe $x \in A$ tal que $f(x) = y$. Assim
		\begin{align*}
			y\otimes f(1_A) &= f(x) \otimes f(1_A) = f(x\cdot 1_A) = f(x) = y\\
			f(1_A)\otimes y &= f(1_A) \otimes f(x) = f(1_A\cdot x) = f(x) = y
		\end{align*}
		para todo $y \in B$. Portanto $B$ possui unidade e $1_B = f(1_A)$.

		\item Novamente, devido \`a unicidade do inverso em um anel, para mostrar que $f(x)$ possui inverso basta mostrar que
		\[
			f(x)\otimes f(x^{-1}) = 1_B = f(x^{-1})\otimes f(x)
		\]
		desde que $x \in A$ possua inverso multiplicativo. Sendo assim suponha que $x \in A$ possui inverso multiplicativo. Seja $x^{-1}$ o inverso multiplicativo de $x$ em $A$. Temos
		\begin{align*}
			f(x)\otimes f(x^{-1}) &= f(x\cdot x^{-1}) = f(1_A) = 1_B\\
			f(x^{-1})\otimes f(x) &= f(x^{-1}\cdot x) = f(1_A) = 1_B
		\end{align*}
		Portanto $f(x)$ possui inverso multiplicativo e $[f(x)]^{-1} = f(x^{-1})$, como quer{\'\i}amos.
	\end{enumerate}
\end{prova}

\section{Ideais} % (fold)
\label{sec:ideais}

\begin{definicao}
	Seja $(A, +, \cdot)$ um anel comutativo. Um subconjunto n\~ao-vazio $I \sub A$ {\'e} chamado de \textbf{ideal} de $A$ se:
	\begin{enumerate}[label={\roman*})]
		\item para todos $x$, $y \in I$, temos $x - y \in I$.
		\item Para todo $\alpha \in A$ e todo $x \in I$, temos $\alpha\cdot x \in I$.
	\end{enumerate}
\end{definicao}

\begin{observacao}
	Quando $I = A$ ou $I = \{0_A\}$, dizemos que $I$ {\'e} um \textbf{ideal trivial}.
\end{observacao}

\begin{exemplo}
    \begin{enumerate}[label={\arabic*})]
		\item Em $\z$ todos os ideais n{\~a}o triviais s{\~a}o da forma $m\z$, $m > 1$.
		\item No anel $\z_p$, onde $p$ {\'e} um n{\'u}mero primo, os {\'u}nicos ideais  s{\~a}o os triviais $\{\overline{0}\}$ e $\z_p$.
        \begin{solucao}
            De fato, seja $I \sub \z_p$ um ideal, $I \neq \{\overline{0}\}$. Provemos que $I = \z_p$. Para isso, vamos
            provar que $\overline{1} \in I$. Seja $\overline{a} \in I$, $\overline{a} \neq \overline{0}$, pois $I \neq
            \{\overline{0}\}$. Como $p$ {\'e} primo, $mdc(a,p) = 1$, da{\'\i} existe $\overline{b} \in \z_p$,
            $\overline{b} \neq \overline{0}$, tal que $\overline{1} = \overline{a} \otimes \overline{b}$. Mas $I$ {\'e}
            ideal e $\overline{a} \in I$, logo $\overline{1} = \overline{a} \otimes \overline{b} \in I$.

            Portanto $I = \z_p$.
        \end{solucao}

		\item Os {\'u}nicos ideais n{\~a}o triviais de $\z_8 = \{\overline{0}, \overline{1}, \overline{2}, \overline{3}, \overline{4}, \overline{5}, \overline{6}, \overline{7}\}$ s{\~a}o:
		\begin{align*}
			I_1 &= \{\overline{0}, \overline{2}, \overline{4}, \overline{6}\}\\
			I_2 &=\{\overline{0}, \overline{4}\}
		\end{align*}

        \item Verifique se s\~ao ideiais:
            \begin{enumerate}[label=({\alph*})]
                \item $I = m\z \times n\z$ no anel $\z \times \z$, em que $m$, $n \in \z$ e as operações consideradas são as usuais;

                \item $J = \z$ no anel $(\rac, \oplus, \otimes)$ em que a $a \oplus b = a + b - 1$ e $a \otimes b = a + b - ab$, para todos $a$, $b \in \rac$;

                \item $K = 4\z$ no anel $(\z, +, \odot)$ em que a adi\c{c}\~ao \'e a usual e $a \odot b = 0$, para quaisquer $a$, $b \in \z$.
            \end{enumerate}
            \begin{solucao}
                \begin{enumerate}[label=({\alph*})]
                    \item Primeiro observe que o elemento neutro do anel $\z \times \z$ com as operações usuais é o elemento $(0, 0)$ e $(0, 0) \in I$ e que o oposto de $(mk, nl)$
                        é dado por
                        \[
                            -(mk, nl) = (-mk, -nl).
                        \]
                        Assim dados $x = (km, nl) \in I$ e $y = (mp, nq) \in I$ temos
                        \[
                            x + (-y) = (mk, nl) + (-mp, -nq) = (m(k - p), n(l - q)) \in I.
                        \]
                        Agora tomando $(\alpha, \beta) \in \z \times \z$ temos
                        \[
                            (\alpha, \beta) \cdot (mk, nl) = (\alpha(mk), \beta(nl)) = (m(\alpha k), n(\beta l)) \in I.
                        \]
                        Portanto $I$ é um ideal de $\z \times \z$.

                    \item No anel $(\rac, \oplus, \otimes)$ o elemento neutro da soma $\oplus$ é 1 e $1 \in J$. Agora observe que tomando $\alpha = \dfrac{1}{3} \in \rac$ e $x =
                        3 \in \z$ temos
                        \[
                            \alpha \otimes x = \dfrac{1}{3} \otimes 3 = \dfrac{1}{3} + 3 - \dfrac{3}{3} = \dfrac{7}{2} \notin \z,
                        \]
                        logo $J$ não é um ideal de $(\rac, \oplus, \otimes)$.

                    \item No anel $(\z, +, \odot)$ o elemento neutro a soma é o 0 e $0 \in K$. Agora sejam $x$, $y \in K$. Assim $x = 4p$ e $y = 4q$ com $p$, $q \in \z$. Daí
                        \[
                            x - y = 4p - 4q = 4(p - q) \in K.
                        \]
                        Agora tomando $\alpha \in \z$ temos
                        \[
                            \alpha \odot x = \alpha \odot 4q = 0 \in K,
                        \]
                        logo $K$ é um ideal de $(\z, +, \odot)$.
                \end{enumerate}
            \end{solucao}
    \end{enumerate}
\end{exemplo}

\begin{proposicao}
	Seja $A$ um anel comutativo e $I$ um ideal de $A$. Ent{\~a}o:
	\begin{enumerate}[label={\roman*})]
	 	\item $0_{A}\in I$.
	 	\item $-x \in I$ para todo $x \in I$.
	 	\item Se $1_A \in I$, ent\~ao $I = A$.
	\end{enumerate}
\end{proposicao}
\begin{prova}
	\begin{enumerate}[label={\roman*})]
		\item Da defini\c{c}\~ao de ideal temos $\alpha \cdot x \in I$ para todo $x \in I$ e todo $\alpha \in A$.
		Assim dado $x \in I$ $0_A = 0_A \cdot x \in I$.

		\item Como $0_A \in I$, dado $x \in I$ da defini\c{c}\~ao de ideal segue que $0_A - x \in I$, isto \'e, $-x \in I$.

		\item Suponha que $1_A \in I$. Como $I$ {\'e} ideal, para todo $\alpha \in A$ e todo $x \in I$ devemos ter $\alpha\cdot x \in I$. Assim, em particular, $1_A \cdot x \in I$ para todo $x \in A$. Logo, $A\sub I$ e como $I\sub A$, ent{\~a}o $I = A$.
	\end{enumerate}
\end{prova}

\begin{definicao}
	Seja $I$ um ideal de um anel $(A, +, \cdot)$. Dados $x$, $y \in A$ dizemos que $x$ \textbf{\'e congruente a} $y$ \textbf{m\'odulo} $I$ quando $x-y \in I$. Neste caso, escrevemos $x\equiv y \pmod I$.
\end{definicao}

\begin{proposicao}
	A congru{\^e}ncia m{\'o}dulo $I$ {\'e} uma rela{\c c}{\~a}o de equival{\^e}ncia em $A \times A$, onde $A$ anel comutativo unit{\'a}rio.
\end{proposicao}
\begin{prova}
	Como $0 = 0_{A} \in I$ e para todo $x \in I$, $x - x = 0 \in I$, ent{\~a}o $x \equiv x \pmod I$.

	Suponha que $x\equiv y \pmod I$. Ent{\~a}o $x - y \in I$. Como $-1 \in A$, $y - x = -(x - y) = -[(x - y)1] = (x - y)(-1) \in I$, ou seja, $y\equiv x \pmod I$.

	Agora, se $x\equiv y \pmod I$ e $y\equiv z \pmod I$, ent{\~a}o $x - y \in I$ e $y - z \in I$. Da{\'\i}, $x - z = (x - z) + (y - z)\in I$, ou seja, $x\equiv z \pmod I$.

	Logo, {\'e} uma rela{\c c}{\~a}o de equival{\^e}ncia.
\end{prova}

Seja $y \in A$. A classe de equival{\^e}ncia m{\'o}dulo $I$ de $y$ {\'e}
\[
	C(y) = \{x \in A \mid x\equiv y \pmod I\} = \{x \in A \mid x - y \in I\}.
\]

Agora, $x - y \in I$ significa que existe $t \in I$, tal que $x - y = t$. Logo, $x = y + t$, onde $t \in I$.

Assim,
\[
	C(y) = \{y + t\mid t \in I\} = y + I.
\]

\begin{observacao}
	Denotamos por $y + I$ (ou $I + y$) a classe de equival{\^e}ncia m{\'o}dulo $I$ de $y \in A$. Denotamos por $\dfrac{A}{I}$ o conjunto de todas as classes de equival{\^e}ncia, tal conjunto {\'e} chamado de \textbf{quociente do anel $A$ pelo ideal $I$}.
\end{observacao}

\begin{exemplos}
	\begin{enumerate}[label={\arabic*})]
		\item Seja $A$ um anel com unidade e $I_{1} = \{0\}$ e $I_{2} = A$ ideais. Ent\~ao:
		\begin{enumerate}[label={\roman*})]
		\item Dado $x \in A$:
		\[
			C(x) = x + I_{1} = \{x + 0\} = \{x\}.
		\]
		Assim $\dfrac{A}{I_{1}} = \{x + I \mid x \in A\}$, logo existem tantas classes de equival{\^e}ncia quantos forem os elementos de $A$.

		\item Para $I_{2} = A$ temos:
		\[
			C(0_A) = 0_A + I = \{0_A + t \mid t \in I_{2}\}.
		\]
		Como $I_2 = A$, para todo $x \in A$ temos $x \in C(0_A)$ logo existem uma \'unica classe de equival\^encia
		e $\dfrac{A}{I_{2}} = \{0_{A} + I\}$.
	\end{enumerate}

	\item Seja $A = \z$. Sabemos que os ideais de $\z$ s{\~a}o da forma $m\z$, $m > 1$. Seja $I = m\z$ um ideal de $\z$. Assim $x\equiv y \pmod I$ se, e s\'o se, $x - y \in I$. Mais isso ocorre se, e somente se, $x - y = mk $, para algum $k \in \z$. Logo $x\equiv y \pmod I$ se, e s\'o se, $m\mid (x - y)$. Portanto, $\dfrac{\z}{I} = \z_m$.
	\end{enumerate}
\end{exemplos}


Agora seja $I$ ideal e $A$ um anel comutativo unitário. Temos
\[
	\dfrac{A}{I} = \{y + I \mid y \in A\}\\
\]
onde $y + I = \{y + t \mid t \in I\}$ e $y \in A$.

Vamos definir uma soma $\oplus$ e um produto $\otimes$ em $\dfrac{A}{I}$ por
\begin{align*}
	(x + I)\oplus(y + I) &= (x + y) + I\\
	(x + I)\otimes(y + I) &= (xy) + I
\end{align*}
para $x + I$, $y + I \in \dfrac{A}{I}$.

Verifiquemos que a soma e o produto em $\dfrac{A}{I}$ n{\~a}o dependem do representante da classe de equival{\^e}ncia.
Para isso sejam $x_1 + I$, $x_2 + I$, $y_1 + I$, $y_2 + I \in \dfrac{A}{I}$ tais que
\begin{align*}
	x_1 + I &= x_2 + I\\
	y_1 + I &= y_2 + I
\end{align*}

Ent{\~a}o
\begin{align*}
	(x_1 + I) \oplus (y_1 + I) &= (x_1 + y_1) + I\\
	(x_2 + I) \oplus (y_2 + I) &= (x_2 + y_2) + I
\end{align*}

Como $x_1 + I = x_2 + I$, ent{\~a}o $x_1 - x_2 \in I$ e como $y_1 + I = y_2 + I$, ent{\~a}o $y_1 = y_2 \in I$. Mas $I$ {\'e} ideal, logo $(x_1 - x_2) + (y_1 - y_2) = (x_1 + y_1) - (x_2 + y_2) \in I$, ou seja
\[
	(x_1 + I) \oplus (y_1 + I) = (x_2 + I) \oplus (y_2 + I).
\]

Agora,
\begin{align*}
	(x_1 + I) \otimes (y_1 + I) &= (x_1y_1) + I\\
	(x_2 + I) \otimes (y_2 + I) &= (x_2y_2) + I
\end{align*}

Como $(x_1 - x_2)y_1 \in I$ e $(y_1 - y_2)x_2 \in I$ ent\~ao
\begin{align*}
	&(x_1 - x_2)y_1 + (y_1 - y_2)x_2 \in I\\
	&x_1y_2-\underbrace{x_2y_1 + y_1x_2}_{= 0} - y_2x_2 \in I\\
	&x_1y_1 - x_2y_2\in I,
\end{align*}
ou seja, $xy + I = x_2y_2 + I$. Portanto,
\[
	(x_1 + I) \otimes (y + I) = (x_2 + I) \otimes (y_2 + I).
\]

\begin{teorema}
	Seja $(A, +, \cdot)$ um anel comutativo e com unidade. Ent{\~a}o, se $I$ {\'e} um ideal de $A$,
	o quociente $\dfrac{A}{I}$ com as opera{\c c}{\~o}es $\oplus$ e $\otimes$ {\'e} um anel comutativo e com unidade. O elemento neutro da soma {\'e} a classe $0_{A} + I$ e a unidade do produto {\'e} $1_{A} + I$.
\end{teorema}
