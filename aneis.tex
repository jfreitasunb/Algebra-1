%!TEX program = xelatex
%!TEX root = Algebra_1.tex
\chapter{An{\'e}is}

\begin{definicao}
	Seja $A$ um conjunto n{\~a}o vazio. Dizemos que $A$ est{\'a} munido (ou equipado) de uma \textbf{opera{\c c}{\~a}o bin{\'a}ria} quando existe uma fun{\c c}{\~a}o
	\begin{align*}
		&\Delta : A \times A \to A\\
		&(a,b) \longmapsto a\Delta b		
	\end{align*}
	Uma opera{\c c}{\~a}o bin{\'a}ria tamb{\'e}m {\'e} chamada de uma \textbf{opera{\c c}{\~a}o interna} em $A$.
\end{definicao}

\begin{exemplos}
	\begin{enumerate}[label={\arabic*})]
		\item A soma usual nos conjuntos $\z$, $\rac$, $\real$ e $\complex$ {\'e} uma opera{\c c}{\~a}o bin{\'a}ria.

		\item A multiplicação usual nos conjuntos dos $\z$, $\rac$, $\real$ e $\complex$ {\'e} uma opera{\c c}{\~a}o bin{\'a}ria.

		\item Seja $m > 1$, $m \in \z$ fixo. A soma \eqref{soma_modulo_m} e a multiplicação \eqref{multiplicacao_modulo_m} definidos em $\z_m = \{\overline{0},\overline{1},...,\overline{m-1}\}$ é uma operação binária.

		\item A operação $\div$ em $\rac^{*}$ {\'e} uma opera{\c c}{\~a}o bin{\'a}ria.
		\item Já em $\n$, $\z$, $\z^{*}$ e em $\rac$ a operação $\div$ n{\~a}o {\'e} uma opera{\c c}{\~a}o bin{\'a}ria.
\end{enumerate}	
\end{exemplos}

\begin{definicao}
	Seja $A$ um conjunto n{\~a}o vazio $A$ no qual estão definidas duas opera{\c c}{\~o}es binárias $\oplus$ e $\otimes$, chamadas \textit{soma} e \textit{produto}.  Dizemos que $(A, \oplus, \otimes)$ {\'e} um \textbf{anel} quando as seguintes condi{\c c}{\~o}es s{\~a}o verdadeiras:
	\begin{enumerate}[label={\roman*})]
		\item \textbf{Associatividade}: para todos $x$, $y$, $x\in A$ vale que
		\[
			(x \oplus y) \oplus z = x \oplus (y \oplus z)
		\]
		Essa propriedade {\'e} chamada \textbf{propriedade associativa} da soma.

		\item \textbf{Comutatividade}: Para todos $x$, $y \in A$ vale
		\[
			x \oplus y = y \oplus x
		\]

		\item \textbf{Elemento Neutro}: Existe em $A$ um elemento denotado por $0$ (zero) ou $0_{A}$ tal que para todo elemento $x \in A$ vale
		\[
			x \oplus 0_A = x = 0_A \oplus x
		\]
		Tal elemento $0_A$ é chamado de \textbf{elemento neutro da soma} ou simplesmente \textbf{elemento neutro}.

		\item \textbf{Elemento Oposto}: Para cada elemento $x \in A$, existe $y \in A$ tal que
		\[
			x \oplus y = 0_A = y \oplus x
		\]
		Tal elemento $y$ é chamado de \textbf{oposto aditivo} de $x$ ou simplesmente \textbf{oposto} de $x$.

		\item \textbf{Associatividade}: Para todos $x$, $y$, $z \in A$, vale que
		\[
			x\otimes (y\otimes z) = x\otimes (y\otimes z)
		\]

		\item \textbf{Distributividade}: Para todos $x$, $y$, $x \in A$ vale
		\[
			(x \oplus y)\otimes z = x\otimes z \oplus y\otimes z
		\]
		Essa propriedade {\'e} chamada \textbf{distributiva da soma em rela{\c c}{\~a}o ao produto}.
		
		\item \textbf{Distributividade}: Para todos $x$, $y$, $z \in A$ vale
		\[
			x\otimes(y \oplus z) = x\otimes y \oplus x\otimes z.
		\]
		Essa {\'e} a propriedade \textbf{distributiva do produto em rela{\c c}{\~a}o {\`a} soma}.
	\end{enumerate}
\end{definicao}

\begin{observacoes}
	Seja $(A, \oplus, \otimes)$ uma anel.
	\begin{enumerate}
		\item \textbf{Comutatividade}: Se para todos $x$, $y \in A$ vale
		\[
			x \otimes y = y \otimes x
		\]
		Dizemos que $(A, \oplus, \otimes)$ {\'e} um \textbf{anel comutativo}.

		\item \textbf{Unidade}: Se existe em $A$ um elemento denotado por $1$ ou $1_{A}$ tal que
		\[
			x \otimes 1 = x = 1 \otimes x,
		\]
		para todo $x \in A$, ent{\~a}o dizemos que $(A, \oplus, \otimes)$ é um \textbf{anel com unidade} ou um \textbf{anel unit{\'a}rio}. O elemento $1_A$ {\'e} chamado de \textbf{unidade} de $A$ ou \textbf{elemento neutro da multiplicação} de $A$.

		\item Se um anel $(A, \oplus, \otimes)$ satisfaz as duas propriedades anteriores dizemos que $(A, \oplus, \otimes)$ é um \textbf{anel comutativo com unidade} ou um \textbf{anel comutativo unitário}.

		\item Seja $(A, \oplus, \otimes)$ uma anel. Quando não houver chance de confusão com relação às operações envolvidas diremos simplesmente que $A$ é uma anel.
	\end{enumerate}
\end{observacoes}

\begin{exemplos}
	\begin{enumerate}[label={\arabic*})]
		\item $(\z,+,.)$, $(\rac,+,.)$, $(\real,+,.)$, $(\complex,+,.)$, $(\z_m, \oplus, \otimes)$ s{\~a}o an{\'e}is associativos, comutativos e com unidade.

		\item  Seja $A = \z =\{f : \z \to \z \mid f \mbox{ {\'e} uma fun{\c c}{\~a}o}\}$. Dadas duas fun{\c c}{\~o}es quaisquer $f$, $g \in A$, definimos $f\oplus g:\z \to \z$ e $f \otimes g:\z \to \z$ como:
		\begin{align*}
			(f\oplus g)(x) &= f(x) + g(x)\\
			(f\otimes g)(x) &= f(x)g(x)
		\end{align*}
		para todo $x \in \z$. Assim $(A, \oplus, \otimes)$ é um anel. De fato:
		\begin{enumerate}[label={\roman*})]
			\item Para todo $x \in \z$
			\begin{align*}
				[(f \oplus g) \oplus h](x) &= (f \oplus g)(x) + h(x) = (f(x) + g(x)) + h(x)\\ 
				&= f(x) + (g(x) + h(x)) = f(x) + (g \oplus h)(x)\\ &= [f \oplus (g \oplus h)](x)
			\end{align*}
			para todos $f$, $g$ e $h \in A$.
			
			\item Para todo $x \in \z$
			\[
				(f\oplus g)(x) = f(x) + g(x) = g(x) + f(x) = (g\oplus f)(x),
			\]
			portanto $f\oplus g = g\oplus f$ para todos $f$, $g \in A$.

			\item $0_A : \z \to \z$ dada por $0_A(x) = 0$ para todo $x \in \z$. Daí para todo $x \in \z$
			\[
				(f \oplus 0_A)(x) = f(x) + 0_A(x) = f(x) + 0 = f(x)
			\]
			para todo $f \in A$. Logo $f + 0_A = f$ para todo $f \in A$. Logo $0_A$ é o elemento neutro da soma em $A$.

			\item Dada $f \in A$, defina $g : \z \to \z$ por $g(x) = -f(x)$ para todo $x \in \z$. Daí para todo $x \in \z$ temos
			\[
				(f \oplus g)(x) = f(x) + g(x) = f(x) + (-f(x)) = 0.
			\]
			Logo $g(x) = -f(x)$ é o oposto de $f \in A$.

			\item Para todo $x \in \z$
			\begin{align*}
				[(f \otimes g)\otimes h](x) &= (f \otimes g)(x)h(x) = (f(x)g(x))h(x)\\ &= f(x)(g(x)h(x)) = f(x)(g \otimes h)(x)\\ &= [f\otimes (g \otimes h)](x)
			\end{align*}
			para todos $f$, $g$ e $h \in \z$.

			\item Para todo $x \in \z$
			\begin{align*}
				[(f \oplus g)\otimes h](x) &= (f \oplus g)(x)h(x) = (f(x) + g(x))h(x) \\ &= f(x)h(x) + g(x)h(x)
				= (f\otimes g)(x) + (g \otimes h)(x)\\
				&= [(f \otimes g) \oplus (g \otimes h)](x)
			\end{align*}
			para todos $f$, $g$ e $h \in A$.

			\item Para todo $x \in \z$
			\begin{align*}
				[f\otimes (g \oplus h)](x) &= f(x)(g\oplus h)(x) = f(x)(g(x) + h(x))\\ 
				&= f(x)g(x) + f(x)h(x) = (f\otimes g)(x) + (f\otimes h)(x)\\
				&= [(f \otimes g) \oplus (f\otimes h)](x)
			\end{align*}
			para todos $f$, $g$ e $h \in A$.
		\end{enumerate}
		Assim $(A, \oplus, \otimes)$ é um anel. Além disso, para todo $x \in \z$
		\[
			(f\otimes g)(x) = f(x)g(x) = g(x)f(x) = (g\otimes f)(x)
		\]
		para todos $f$, $g \in A$. Assim a operação $\otimes$ é comutativa.

		Mais ainda, definindo $1_A : \z \to \z$ como $1_A(x) = 1$ para todo $x \in \z$ temos
		\[
			(f \otimes 1_A)(x) = f(x)1_A(x) = f(x)\cdot 1 = f(x)
		\]
		para todo $f \in A$. Logo $1_A$ é a unidade de $A$.

		Portanto $(A, \oplus, \otimes)$ é um anel comutativo com unidade.
	\end{enumerate}
\end{exemplos}

\begin{observacao}
	Seja $(A, \oplus, \cdot)$ um anel. Para simplificar a notação vamos denotar a operação $\oplus$
	por $+$ e a operação $\otimes$ por $\cdot$ e assim escrever simplesmente que $(A, +, \cdot)$ é um anel.
\end{observacao}

\begin{proposicao}
	Seja $(A, + , \cdot)$ uma anel. Então:
	\begin{enumerate}[label={\roman*})]
		\item O elemento neutro {\'e} {\'u}nico.
		\item Para cada $x \in A$ existe um {\'u}nico oposto.
		\item Para todo $x \in A$, $-(-x) = x$.
		\item Dados $x_{1}$, $x_{2}$, \dots, $x_n \in A$, $n\leq 2$, ent{\~a}o
		\[
			-(x_1 + x_2 + \dots + x_n) = (-x_1) + (-x_2) + \dots + (-x_n).
		\]
		\item Para todos $a$, $x$, $y \in A$, se $a + x = a + y$, ent{\~a}o $x = y$.
		\item Para todo $x \in A$, $x\cdot 0_A = 0_A = 0_A\cdot x$.
		\item Para todos $x$, $y \in A$, temos $x(-y) = (-x)y = -(xy)$.
		\item Para todos $x$, $y \in A$, $xy = (-x)(-y)$.
	\end{enumerate}
\end{proposicao}
\begin{prova}
	\begin{enumerate}[label={\roman*})]
		\item Suponha que existam $0_1$, $0_2\in A$ elementos neutros de $A$. Assim
		\[
			x + 0_1 = x \quad \mbox{e}\quad x + 0_2 = x	
		\]
		para todo $x \in A$. Assim
		\[
			0_1 = 0_1 + 0_2 = 0_2
		\]
		e portanto o elemento neutro é único.

		\item De fato, dado $x \in A$ suponha que existam $y_1$, $y_2\in A$ tais que
		\[
			x + y_1 = 0_A \quad \mbox{e}\quad x + y_2 = 0_A.
		\]
		Daí
		\[
			y_1 = y_2 + 0_A = y_1 + (x + y_2) = (y_1 + x) + y_2 = 0_A + y_2 =y_2.
		\]
		Logo o oposto de $x$ é único  e daí será denotado por $-x$.
		
		\item Dado $x \in A$, então $-x$ {\'e} oposto de $x$, isto {\'e}, $x + (-x) = 0_A$. Logo o oposto de $(-x)$ {\'e} $x$, ou seja, $-(-x) = x$.

		\item Segue usando indução sobre $n$.

		\item Suponha que $a + x = a + y$. Seja $-a$ o oposto de $a$ daí
		\begin{align*}
			a + x &= a + y\\
			(-a) + a + x &= (-a) + a + y\\
			0_A + x &= 0_A + y\\
			x & = y
		\end{align*}
		como queríamos.

		\item Temos $0_A + x\cdot 0_A = a\cdot 0_A = a(0_A + 0_A) = a\cdot 0_A + a\cdot 0_A$. Assim do item anterior segue que $x\cdot 0_A = 0_A$.

		\item Provemos que $x(-y) = -(xy)$:
		\[
			x(-y) + xy = x((-y) + y) = x\cdot 0_A = 0_A,
		\]
		portanto $-xy = x(-y)$.

		\item Basta usar o caso anterior.
	\end{enumerate}
\end{prova}

\begin{definicao}
	Um anel comutativo $(A, + , \cdot)$ {\'e} dito ser um \textbf{anel de integridade} quando para todos 
	$x$, $y \in A$, se $xy = 0_A$, ent{\~a}o $x = 0_A$ ou $y = 0_a$. Um anel de integridade tamb{\'e}m {\'e} chamado de \textbf{dom{\'\i}nio de integridade} ou simplesmente de \textbf{dom{\'\i}nio}.
\end{definicao}

\begin{observacao}
	Se $x$ e $y$ s{\~a}o elementos n{\~a}o nulos de um anel $A$ tais que $xy = 0_A$, ent{\~a}o $x$ e $y$ s{\~a}o chamados de \textbf{divisores pr{\'o}prios de zero}.
\end{observacao}


\begin{exemplos}
	\begin{enumerate}[label={\arabic*})]
		\item Os an{\'e}is $\z$, $\rac$, $\real$, $\complex$ s{\~a}o an{\'e}is de integridade.
		
		\item Em geral, $\z_m$ n{\~a}o {\'e} anel de integridade, por exemplo, em $\z_4$, $\overline{2} \neq \overline{0}$, no entanto $\overline{2}\otimes \overline{2} = \overline{4} = \overline{0}$.
		
		\item $M_{n}(\real)$ n{\~a}o {\'e} um anel de integridade, por exemplo, em $M_{2}(\real)$
		\begin{align*}
			A &= \begin{bmatrix}
				1 & 0\\
				0 & 0
			\end{bmatrix} \neq \begin{bmatrix}
				0 & 0\\
				0 & 0		
			\end{bmatrix},\qquad 
			B = \begin{bmatrix}
				0 & 0\\
				1 & 0
			\end{bmatrix} \neq \begin{bmatrix}
				0 & 0\\
				0 & 0
			\end{bmatrix}\\
			AB & =\begin{bmatrix}
				0 & 0\\
				0 & 0
			\end{bmatrix}
		\end{align*}

		\item Suponha que $m = nk$, $m > n > 1$ e $m > k > 1$. Logo, em $\z_m$, $\overline{n} \neq \overline{0}$ e $\overline{k} \neq \overline{0}$ e no entanto $\overline{n} \otimes \overline{k} = \overline{m} = \overline{0}$. Logo, se $m$ n{\~a}o {\'e} primo, ent{\~a}o $\z_m$ n{\~a}o {\'e} um anel de integridade. Agora, suponha que $m = p$ primo. Sejam $\overline{x}$, $\overline{y} \in \z_m$ tais que $\overline{x}\otimes \overline{y} = \overline{0}$, ou seja, $xy \equiv 0 \pmod(p)$. Da{\'\i} $p\mid xy$. Logo $p\mid x$ ou $p\mid y$. Portanto, $\overline{x} = \overline{0}$ ou $\overline{y} = \bar{0}$. Assim, $\z_m$ {\'e} anel de integridade se, e somente se, $m$ {\'e} primo.
	\end{enumerate}
\end{exemplos}



\begin{definicao}
	Seja $(A,+,\cdot)$ um anel. Dizemos que um subconjunto n{\~a}o vazio $B\subseteq A$ {\'e} um \textbf{subanel} quando $(B, + , \cdot)$ \'e um anel.
\end{definicao}

Exemplos:
\begin{enumerate}
\item Todo anel $A$ sempre tem dois suban{\'e}is: $\{0_{A}\}$ e $A$, que s{\~a}o chamados de \textbf{suban{\'e}is triviais}.
\item Em $(\z_4,\oplus,\odot)$ o conjunto $B = \{\overline{0}, \overline{2}\}$ \'e um subanel.
\item No anel $\z$, o conjunto $m\z$, $m > 1$ {\'e} um subanel de $\z$.
\end{enumerate}

\begin{proposicao}
	Seja $(A,+,\cdot)$ um anel. Um subconjunto n{\~a}o vazio $B\subseteq A$ {\'e} um subanel de $A$ se, e somente se, $x - y \in B,$ e $x\cdot y \in B$ para todos $x$, $y\in B$.
\end{proposicao}

$(A,+,.),(B,\oplus,\odot)$ An{\'e}is\\
$f:A\rightarrow B$\\
$a\rightarrow f(a)$\\
$g:\z\rightarrow \displaystyle\frac{\z}{m\z}$\\
$x\rightarrow\bar{x}$

$g(x+y)=\overline{x+y}=\bar{x}\oplus\bar{y}=g(x)\oplus g(y)$\\
$g(x+y)=g(x)\oplus g(y)$

$g(xy)=\overline{xy}=\bar{x}\odot\bar{y}=g(x)\odot g(y)$

$g(1)=\bar{1}$

\section{Homomorfismo}
\subsubsection{Defini{\c c}{\~a}o}

\begin{definicao}[Homomorfismo] Um homomorfismo do anel (A,+,.) no anel $(B,\oplus,\odot)$ {\'e} uma fun{\c c}{\~a}o $f:A\rightarrow B$ que satisfaz:
\begin{enumerate}
\item $f(x+y)=f(x)+f(y),\forall x,y\in A$
\item $f(xy)=f(x)f(y),\forall x,y\in A$
% \item $f(1_{A})=1_{B}$, onde $1_{A}$ {\'e} a unidade de A e $1_{B}$ {\'e} a unidade de B
\end{enumerate}
\end{definicao}

Se (A,+,.) {\'e} um anel, ent{\~a}o $f:A\rightarrow A$ dada por $f(a)=a$ {\'e} um homomorfismo de A em A pois:
\begin{enumerate}
\item $f(x+y)=x+y=f(x)+f(y)$
\item $f(xy)=xy=f(x)f(y)$
\item $f(1_{A})=1_{A}$
\end{enumerate}

\subsubsection{Propriedades}
\begin{proposicao} Seja $f:A\rightarrow B$ homomorfismo do anel A no anel B. Ent{\~a}o:
\begin{enumerate}
\item $f(0_{A})=0_{B}$
\item $f(-a)=-f(a),\forall a\in A$
\end{enumerate}
\end{proposicao}

\textbf{Demonstra{\c c}{\~a}o}:
\begin{enumerate}
\item Da condi{\c c}{\~a}o 1 da defini{\c c}{\~a}o de homomorfismo, fazendo $x=y0_{A}$, temos\\
$f(0_{A}+0_{A})=f(0_{A})\oplus f(0_{A})$\\
mas $0_{A}+0_{A}=0_{A}$. Da{\'\i}\\
$f(0_{A})=f(0_{A})+f(0_{A})$\\
Somando $-f(0_{A})$ em ambos os lados\\
$f(0_{A})\oplus(-f(0_{A}))=(f(0_{A})+f(0_{A}))+(-f(0_{A}))$\\
$0_{B}=f(0_{A})+0_{B}$\\
$f(0_{A})=0_{B}$
\item Temos $0_{B}=f(0_{A})=f(a+(-a))=f(a)\oplus f(-a)$\\
Somando $-f(a)$ em ambos os lados\\
$0_{B}\oplus(-f(a))=[f(a)\oplus f(-a)]+(-f(a))$
$-f(a)=f(-a)\oplus(f(a)\oplus(-f(a)))$\\
$f(-a)=-f(a)$.\#
\end{enumerate}

Seja $f:\z\rightarrow\z$ um homomorfismo. Dado $n\in\z,n\geq 0$. Temos da{\'\i}, \[f(n)=f(\underbrace{1+...+1}_{n\ vezes})=\underbrace{f(1)+...+f(1)}_{n\ vezes}=nf(1)=n1\]
$f(n)=n$, para todo $n \in \z$.

\begin{proposicao}
	Seja $f : A \to B$ um homomorfismo sobrejetor de an\'eis.
	\begin{enumerate}
		\item Se $A$ tem unidade, ent\~ao $B$ tem unidade e $f(1_A) = 1_B$.
		\item Se $A$ tem unidade e $x \in A$ possui inverso multiplicativo, ent\~ao $f(x)$ tem inverso e $f(x^{-1} = (f(x))^{-1}$.
	\end{enumerate}
\end{proposicao}

\subsection{Epimorfismo, monomorfismo e isomorfismo}
\begin{definicao}[Epimorfismo, monomorfismo e isomorfismo] Seja $f:A\rightarrow B$ um homomorfismo, onde $A$ e $B$ s{\~a}o an{\'e}is. Dizemos que
\begin{enumerate}
\item $f$ {\'e} um epimorfismo se $f$ for sobrejetora
\item $f$ {\'e} um monomorfismo se $f$ for injetora
\item $f$ {\'e} um isomorfismo se $f$ for bijetora
\item Quando $A=B$ e $f$ {\'e} um isomorfismo, ent{\~a}o $f$ {\'e} um automorfismo
\end{enumerate}
\end{definicao}

\section{Ideal de um anel}
\subsubsection{Defini{\c c}{\~a}o}
\begin{definicao}[Ideal em um anel] Seja $(A,+,.)$ um anel comutativo. Um ideal em $A$ {\'e} um conjunto n{\~a}o vazio $I$ tal que:
\begin{enumerate}
\item Para todo $a$, $b\in I$, devemos ter $a - b \in I$.
\item Para todo $b\in A$ e todo $x \in I$, $bx \in I$.
\end{enumerate}
\end{definicao}

Quando $I=A$ ou $I=\{0_{A}\},\ I$ {\'e} chamado de ideal trivial.

\subsubsection{Propriedades}
\begin{proposicao} Seja $A$ um anel comutativo e $I$ um ideal de $A$. Ent{\~a}o:
\begin{enumerate}
 	\item $0_{A}\in I$.
 	\item $-a \in I$ para todo $a \in I$.
 	\item Se $1_A \in I$, ent\~ao $I = A$.
 \end{enumerate}
\end{proposicao}

\textbf{Demonstra{\c c}{\~a}o}: Temos que da defini{\c c}{\~a}o de ideal, $ab\in I$, para todo $a,b\in I$.

Assim, dado $a\in I,\ a0_{A}=0_{A}\in I$.\#

\textbf{Demonstra{\c c}{\~a}o}: Como $I$ {\'e} ideal, $1_{A}x\in I$, para todo $x\in A$, ou seja, $x=1_{A}x\in I$ para qualquer $x\in A$, logo, $A\subseteq I$. Como $I\subseteq A$, ent{\~a}o $I=A$.\#


Exemplos:
\begin{enumerate}
\item Em $\z$ todos os ideais n{\~a}o triviais s{\~a}o da forma $m\z,m>1$
\item No anel $\displaystyle\frac{\z}{p\z}$, onde $p$ {\'e} um n{\'u}mero primo, os {\'u}nicos ideais  s{\~a}o os triviais $\{\bar{0}\}$ e $\displaystyle\frac{\z}{p\z}$.

\textbf{Demonstra{\c c}{\~a}o}: Seja $I\subseteq\displaystyle\frac{\z}{p\z}$ um ideal, $I\neq\{\bar{0}\}$. Provemos que 
$I=\displaystyle\frac{\z}{p\z}$. Para isso, vamos provar que $\bar{1}\in I$. Seja $\bar{a}\in I, \bar{a}\neq \bar{0}$, pois $I\neq\{\bar{0}\}$. Mas como $p$ {\'e} primo, $mdc(a,p)=1$, da{\'\i} existe $\bar{b}\in\displaystyle\frac{\z}{p\z},\bar{b}\neq\bar{0}$, tal que $\bar{1}=\bar{a}\bar{b}$. Mas $I$ {\'e} ideal e $\bar{a}\in I$, logo $\bar{1}=\bar{a}\bar{b}\in I$.

Portanto $I=\displaystyle\frac{\z}{p\z}$.\#
\item Os {\'u}nicos ideais n{\~a}o triviais de $\displaystyle\frac{\z}{8\z}=\{\bar{0},\bar{1},\bar{2},\bar{3},\bar{4},\bar{5},\bar{6},\bar{7}\}$ s{\~a}o:\\
$I_{1}=\{\bar{0},\bar{2},\bar{4},\bar{6}\}$ e $I_{2}=\{\bar{0},\bar{4}\}$ 

\end{enumerate}

Observa{\c c}{\~a}o: Num anel $(A,+,.)$, a diferen{\c c}a $a-b$ {\'e} definida como
\[a-b=a+(-b),\ a,b\in A\]

\subsection{Congru{\^e}ncia m{\'o}dulo I}
\subsubsection{Defini{\c c}{\~a}o}
\begin{definicao}[Congru{\^e}ncia m{\'o}dulo I] Seja $I$ um ideal de um anel $(A,+,.)$. Dizemos que $x$ {\'e} congruente a $y$ m{\'o}dulo $I$ quando $x-y\in I$. Neste caso, escrevemos $x\equiv y(mod\ I)$.\end{definicao}

\subsubsection{Propriedades}
\begin{proposicao} A congru{\^e}ncia M{\'o}dulo $I$ {\'e} uma rela{\c c}{\~a}o de equival{\^e}ncia em $A \ times A$($A$ anel unit{\'a}rio).\end{proposicao}

\textbf{Demonstra{\c c}{\~a}o}: Como $0=0_{A}\in I$ e para todo $x\in I$, $x-x=0\in I$, ent{\~a}o $x\equiv x(mod\ I)$.

Suponha que $x\equiv y(mod\ I)$. Ent{\~a}o $x-y\in I$. Como $-1\in A$,$y-x=-(x-y)=-[(x-y)1]=(x-y)(-1)\in I$, ou seja, $y\equiv x(mod\ I)$.

Agora,s e $x\equiv y(mod\ I)$ e $y\equiv z(mod\ I)$, ent{\~a}o $x-y\in I$ e $y-z\in I$. Da{\'\i}, $x-z=(x-z)+(y-z)\in I$, ou seja, $x\equiv z(mod\ I)$.

Logo, {\'e} uma rela{\c c}{\~a}o de equival{\^e}ncia.\#

Seja $y\in A$. A classe de equival{\^e}ncia m{\'o}dulo $I$ {\'e}
\[C(y)=\{x\in A/x\equiv y(mod\ I)\}=\{x\in A/x-y\in I\}\]


Agora, $x-y\in I$ significa que existe $t\in I$, tal que $x-y=t$. Logo, $x=y+t$, onde $t\in I$.

Assim,
\[C(y)=\{y+t,t\in I\}=y+I\]

\begin{nota}[Congru{\^e}ncia M{\'o}dulo I] Denotamos por $y+I$ (ou $I+y$) a classe de equival{\^e}ncia m{\'o}dulo $I$. Denotamos por $\displaystyle\frac{A}{I}$ o conjunto de todas as classes de equival{\^e}ncia, tal conjunto {\'e} chamado quociente do anel $A$ pelo ideal $I$.\end{nota}

Exemplos:
\begin{enumerate}
\item A anel e $I_{1}=\{0\}$ e $I_{2}=A$ ideais.
\begin{enumerate}
\item $\displaystyle\frac{A}{I_{1}};a\in A$\\
$C(a)=a+I_{1}=\{a+0\}=\{a\}$\\
$\displaystyle\frac{A}{I_{1}}=\{a+I, a\in A\}$\\
Tantas classes de equival{\^e}ncia quantos elementos em $A$
\item $\displaystyle\frac{A}{I_{2}}; a\in A, I_{2}=A$\\
$C(a)=a+I=\{a+t/t\in I_{2}\}$\\
$C(0_{A})=0_{A}+I]\{0_{A}+t/t\in I_{2}\}$\\
$0_{A}+I=\{t/t\in I_{2}=A\}$\\
$\displaystyle\frac{A}{I_{2}}=\{0_{A}+I\}$\\
Apenas uma classe de equival{\^e}ncia
\end{enumerate}
\item Seja $A=\z$. Sabemos que os ideais de $\z$ s{\~a}o da forma $m\z,m>1$. Seja $I=m\z$ um ideal de $\z$. Ent{\~a}o\\
\[x\equiv y(mod\ I)\Leftrightarrow x-y\in I \Leftrightarrow x-y=mk,k\in\z\Leftrightarrow m|(x-y)
\Leftrightarrow x\equiv y(mod\ m)\]

Portanto, $\displaystyle\frac{\z}{I}=\displaystyle\frac{\z}{m\z}$.
\end{enumerate}

Agora seja $I$ ideal e $A$ anel. \[\displaystyle\frac{A}{I}\{y+I/y\in A\}\] \[y+I=\{y+t/t\in I\}\]

Vamos definir uma soma $\oplus$ e um produto $\odot$ em $\displaystyle\frac{A}{I}$ por \[(x+I)\oplus(y+I)=(x+y)+I\] \[(x+I)\odot(y+I)=(xy)+I\]

Verifiquemos que a soma e o produto em $\displaystyle\frac{A}{I}$ n{\~a}o dependem do representante da classe de equival{\^e}ncia. Dados $x+I, x_{1}+I,y+I,y_{1}+I\in\displaystyle\frac{A}{I}$ tais que \[x+I=x_{1}+I\] \[y+I=y_{1}+I\]

Ent{\~a}o
\[(x+I)\oplus(y+I)=(x+y)+I\]
\[(x_{1}+I)\oplus(y_{1}+I)=(x_{1}+y_{1})+I\]

Como $x+I=x_{1}+I$, ent{\~a}o $x-x_{1}\in I$ e como $y+I=y_{1}+I$, ent{\~a}o $y=y_{1}\in I$. Mas $I$ {\'e} ideal, logo $(x-x_{1})+(y-y_{1})=(x+y)-(x_{1}+y_{1})\in I$, ou seja \[(x+I)\oplus(y+I)=(x_{1}+I)\oplus(y_{1}+I)\]

Agora, \[(x+I)\odot(y+I)=(xy)+I\] \[(x_{1}+I)\odot(y_{1}+I)=(x_{1}y_{1})+I\]

Como $(x-x_{1})y\in I$ e $(y-y_{1})x_{1}\in I$. Logo, \[(x-x_{1})y+(y-y_{1})x_{1}\in I\] \[xy-\underbrace{x_{1}y+yx_{1}}_{=0}-y_{1}x_{1}\in I\]
\[xy-x_{1}y_{1}\in I\], ou seja, $xy+I+x_{1}y_{1}+I$. Portanto, \[(x+I)\odot(y-I)=(x_{1}+I)\odot(y_{1}+I)\]

\begin{teorema} Seja $(A,+,.)$ um anel associativo, comutativo e com unidade. Ent{\~a}o, se $I$ {\'e} um ideal de $A$, o quociente $\displaystyle\frac{A}{I}$ com as opera{\c c}{\~o}es $\oplus$ e $\odot$ {\'e} um anel associativo, comutativo e com unidade. O elemento zero desse anel {\'e} a classe $0_{A}+I$ e o elemento um de $\displaystyle\frac{A}{I}$ {\'e} $1_{A}+I$. \end{teorema}