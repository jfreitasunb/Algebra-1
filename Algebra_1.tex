%!TEX program = xelatex
%!TEX encoding = ISO-8859-1
\documentclass[portuguese,twoside,12pt]{report}%opcao draft remove os links
\usepackage{amssymb,amsmath,amsfonts,amsthm,amstext}
\usepackage[brazil]{babel}
\usepackage{inputenc}
\usepackage{fancybox}
\usepackage{niceframe}
\usepackage{hyperref}
\usepackage{graphicx}
\graphicspath{{/Users/jfreitas/Dropbox/imagens-latex/}{/home/jfreitas/Dropbox/imagens-latex/}}
\usepackage{makeidx}

%=============================================================================================
%             Nomes para defini{\c c}{\~o}es, teoremas, etc
%=============================================================================================

\newtheorem{teorema}{Teorema}[chapter]
\newtheorem{definicao}{Defini{\c c}{\~a}o}[chapter]
\newtheorem{nota}{Nota{\c c}{\~a}o}[definicao]
\newtheorem{proposicao}{Proposi{\c c}{\~a}o}[definicao]
\newtheorem{lema}{Lema}[proposicao]
\newtheorem{corolario}{Corol{\'a}rio}[proposicao]

%=============================================================================================
%             Cabe{\c c}alhos
%=============================================================================================

\usepackage{fancyhdr}
\pagestyle{fancy} \addtolength{\headwidth}{\marginparsep}
\addtolength{\headwidth}{\marginparwidth}
\renewcommand{\headrulewidth}{1pt}
\renewcommand{\chaptermark}[1]{\markboth{{CAP. \thechapter\ $\bullet$ \; #1}}{}}
\renewcommand{\sectionmark}[1]{\markright{{SE{\c C}{\~A}O \thesection\ $\bullet$ \; #1}}}
\fancyhf{} \fancyhead[RO,RE]{\small\bfseries\textrm \thepage}
\fancyhead[LO]{\small\bfseries\textrm \leftmark}
\fancyhead[LE]{\small\bfseries\textrm \rightmark}

%=============================================================================================
%             Estilo dos t{\'\i}tulos de cap{\'\i}tulo, se{\c c}{\~a}o, etc
%=============================================================================================
\usepackage{sectsty}
\usepackage[Conny]{fncychap}
\sectionfont{\rmfamily\raggedright\sectionrule{0.3in}{1pt}{-0.1in}{1pt}}
\subsectionfont{\rmfamily\raggedright}
\chapterfont{\thispagestyle{empty}}
\ChNameVar{\Huge\rm\bfseries} \ChTitleVar{\Huge\rm\bfseries}

%=============================================================================================
%             Medidas
%=============================================================================================


\setlength{\headsep}{1cm}                    % DIST{\^A}NCIA TEXTO/CABE{\c C}ALHO
\setlength{\textwidth}{16.5 cm}              % LARGURA DO TEXTO
\setlength{\textheight}{21.5 cm}             % ALTURA DO TEXTO
\setlength{\oddsidemargin}{0.1 cm}           % MARGEM {\'I}MPAR
\setlength{\evensidemargin}{0.4 cm}          % MARGEM PAR
\setlength{\topmargin}{0 cm}                 % MARGEM SUPERIOR
\renewcommand{\baselinestretch}{1.4}         % DIST{\^A}NCIA ENTRE LINHAS

%=============================================================================================
%             Comandos Pessoais
%=============================================================================================

\newcommand{\sub}{\subseteq}
\newcommand{\real}{\mathbb{R}}
%=============================================================================================


\makeindex



\title{\Huge \textbf{{\'A}lgebra 1}\\ \vspace{0,5cm} \Large Notas de Aula}
\author{\Large Departamento de Matem{\'a}tica\\ \\ \large Universidade de Bras{\'\i}lia - UnB}
\date{2/2010}

\begin{document}
\maketitle

\begin{center}
\Huge \textbf{Pref{\'a}cio}
\end{center}
\hspace{0,5cm}Essas notas de Aula s{\~a}o referentes {\`a} mat{\'e}ria {\'A}lgebra 1,
ministrada na UnB - Universidade de Bras{\'\i}lia - durante o 2� Semestre de 2010
pelo professor Jos{\'e} Ant{\^o}nio de O. Freitas, Departamento de Matem{\'a}tica. Tais
notas foram transcritas e editadas pelo graduando em Ci{\^e}ncias Econ{\^o}micas
Luiz Eduardo Sol R. da Silva\footnote{luizeduardosol@hotmail.com}.

{\'E} livre a reprodu{\c c}{\~a}o, distribui{\c c}{\~a}o e edi{\c c}{\~a}o deste material, desde que citadas as suas fontes e autores. Cr{\'\i}ticas e sugest{\~o}es s{\~a}o bem vindas.
\vspace{20cm}




\begin{center} \textbf{\Large Nota{\c c}{\~o}es e express{\~o}es}
\end{center}
\begin{minipage}[l]{0,5\textwidth}
\begin{itemize}
\item $\neg$ N{\~a}o
\item $\forall$ Para todo
\item $/$ Tal que
\item $|$ Divide
\item $\Rightarrow$ Implica
\item $\in$ Pertence
\item $\emptyset$ Vazio
\item $\subseteq$ Contido ou igual a
\item $\supseteq$ Cont{\'e}m ou igual a
\item $\wedge$ E
\item $\vee$ Ou
\item $=$ Igual
\item $\neq$ Diferente
\item $\mathbb{Z}$ N{\'u}meros Inteiros
\item $\mathbb{R}$ N{\'u}meros Reais
\item $\cap$ Intersec{\c c}{\~a}o
\item $>$ Maior que
\item $\geq$ Maior ou igual a
\item $\displaystyle\bigcup_{i=1}^{n}$ Uni{\~a}o de $n$ conjuntos
\item $\displaystyle\bigsqcup_{i=1}^{n}$ Uni{\~a}o disjunta de $n$ conjuntos


\end{itemize}
\end{minipage}
\begin{minipage}[r]{0,5\textwidth}
\begin{itemize}

\item $\leftrightarrow$ Se, e somente se
\item $\veebar$ Ou...,ou..., mas nunca ambos
\item $\rightarrow$ Se,... ent{\~a}o...
\item $\exists$ Existe
\item $\Leftrightarrow$ Equivalente a
\item $\notin$ N{\~a}o pertence
\item \# Fim da demonstra{\c c}{\~a}o
\item $\mathbb{N}$ N{\'u}meros Naturais
\item $\mathbb{Q}$ N{\'u}meros Racionais
\item $\nsubseteq$ N{\~a}o cont{\'e}m ou {\'e} igual a
\item $\cup$ Uni{\~a}o
\item $\sqcup$ Uni{\~a}o Disjunta
\item $<$ Menor que
\item $\leq$ Menor ou igual a
\item $\displaystyle\bigcap_{i=1}^{n}$ Intersec{\c c}{\~a}o de $n$ conjuntos
\item Q.E.D. (\textit{Quod Erat Demonstrandum}): Como se queria demonstrar
\item P.B.O.: Princ{\'\i}pio da boa ordena{\c c}{\~a}o
\item H.I.: Hip{\'o}tese de Indu{\c c}{\~a}o
\item \textit{Mutatis Mutandis}:  Mudando o que tem que ser mudado

\end{itemize}
\end{minipage}

\tableofcontents

\chapter{No{\c c}{\~o}es de L{\'o}gica}

\section{Conceitos b{\'a}sicos}

\hspace{0,5cm}O estudo da l{\'o}gica proporciona instrumentos de pensamento para determinar a corre{\c c}{\~a}o ou incorre{\c c}{\~a}o de todos os racioc{\'\i}nios.

A l{\'o}gica pode n{\~a}o nos levar {\`a} verdade no sentido absoluto, mas nos permite descobrir a incoer{\^e}ncia e o erro em um argumento.

\subsection{Proposi{\c c}{\~a}o}

\begin{definicao}[Proposi{\c c}{\~a}o]: Chama-se proposi{\c c}{\~a}o todo conjunto de palavras ou s{\'\i}mbolos que exprime um pensamento de sentido completo.
\end{definicao}

S{\~a}o exemplos de proposi{\c c}{\~o}es:
\begin{enumerate}
\item A Lua {\'e} um sat{\'e}lite da Terra
\item $\pi>\sqrt{5}$
\end{enumerate}

\subsection{Valor de uma proposi{\c c}{\~a}o}
\begin{definicao}[Valor de uma Proposi{\c c}{\~a}o]Chama-se valor de uma proposi{\c c}{\~a}o a verdade se a proposi{\c c}{\~a}o {\'e} verdadeira e a falsidade se for falsa.\end{definicao}

\subsection{Princ{\'\i}pios fundamentais}
A l{\'o}gica matem{\'a}tica adota como regras fundamentais os dois seguintes princ{\'\i}pios (ou axiomas):
\begin{enumerate}
\item \textbf{Princ{\'\i}pio da n{\~a}o contradi{\c c}{\~a}o}: uma proposi{\c c}{\~a}o n{\~a}o pode ser verdadeira e falsa ao mesmo tempo
\item \textbf{Princ{\'\i}pio do Terceiro exclu{\'\i}do}: Toda proposi{\c c}{\~a}o ou {\'e} verdadeira ou {\'e} falsa, isto {\'e}, verifica-se sempre um desses casos e nunca um terceiro
\end{enumerate}
\subsection{Argumento l{\'o}gico}
\begin{definicao}[Argumento l{\'o}gico]: Um argumento l{\'o}gico {\'e} uma seq{\"u}{\^e}ncia de proposi{\c c}{\~o}es , na qual uma das seq{\"u}{\^e}ncias {\'e} a conclus{\~a}o e as demais, chamadas de premissas, formam as provas ou evid{\^e}ncias para a conclus{\~a}o.\end{definicao}

Exemplos:

\textbf{Argumento 1}: [Como todo brasileiro {\'e} sul americano](1� proposi{\c c}{\~a}o/ premissa) e [todo brasiliense {\'e} brasileiro](2� proposi{\c c}{\~a}o/premissa), ent{\~a}o [todo brasiliense {\'e} brasileiro](3� proposi{\c c}{\~a}o/conclus{\~a}o).

\textbf{Argumento 2}: [Como todo matem{\'a}tico {\'e} louco](1� proposi{\c c}{\~a}o/ premissa) e [eu sou matem{\'a}tico](2� proposi{\c c}{\~a}o/ premissa), ent{\~a}o [eu sou louco](3� proposi{\c c}{\~a}o, conclus{\~a}o).

Um argumento {\'e} v{\'a}lido quando suas proposi{\c c}{\~o}es, se verdadeiras, fornecem provas convincenetes para sua conclus{\~a}o, isto {\'e}, quando as proposi{\c c}{\~o}es e a conclus{\~a}o est{\~a}o de tal modo relacionados que {\'e} absolutamente imposs{\'\i}vel as proposi{\c c}{\~o}es serem verdadeiras se a conclus{\~a}o n{\~a}o for.

\subsection{Proposi{\c c}{\~o}es simples e compostas}

As proposi{\c c}{\~o}es podem ser classificadas em simples (ou at{\^o}micas) e compostas.

Chama-se proposi{\c c}{\~a}o simples aquela que n{\~a}o cont{\'e}m nenhuma outra proposi{\c c}{\~a}o como parte integrante de si mesma.

Exemplos:
\begin{enumerate}
\item O n{\'u}mero 25 {\'e} um quadrado perfeito
\item $\pi>\sqrt{5}$
\end{enumerate}

Chama-se proposi{\c c}{\~a}o composta a que {\'e} formada pela combina{\c c}{\~a}o de duas ou mais proposi{\c c}{\~o}es.

Exemplo: $3>4$ ou $4>3$

\subsection{Conectivos}

\begin{definicao}[Conectivos]Chamam-se conectivos palavras que se usam para formar novas proposi{\c c}{\~o}es a partir de outras.\end{definicao}

Em l{\'o}gica matem{\'a}tica os conectivos usuais s{\~a}o os seguintes:
\begin{itemize}
\item ``E" (Conjun{\c c}{\~a}o)
\item ``OU" (Disjun{\c c}{\~a}o)
\item ``N{\~a}o" (Nega{\c c}{\~a}o)
\item ``Se... ent{\~a}o" (Condicional)
\item ``Se, e somente se..." (Bicondicional)
\end{itemize} 

Vamos denotar as proposi{\c c}{\~o}es simples por letras min{\'u}sculas (a,b,c...) e as proposi{\c c}{\~o}es compostas por letras mai{\'u}sculas (A,B,C...)

Exemplo:\\
\textbf{$p$}: o n{\'u}mero 6 {\'e} par\\
\textbf{$q$}: o numero 8 {\'e} um cubo perfeito\\
\textbf{$P$}: O n{\'u}mero 6 {\'e} par E o n{\'u}mero 8 {\'e} um cubo perfeito\\
\textbf{$Q$}: O n{\'u}mero 6 {\'e} par OU 8 {\'e} um cubo perfeito

Se p {\'e} uma proposi{\c c}{\~a}o, vamos denotar seu valor l{\'o}gico por $V(p)$
\begin{center}
p: ou $V(p)=V$ ou $V(p)=F$
\end{center}

Vamos usar a nota{\c c}{\~a}o (Tabela \ref{notacao}), chamada Tabela Verdade (Tabela \ref{tabelavdd})
\begin{table}[h]
   \centering 
   \setlength{\arrayrulewidth}{0,5\arrayrulewidth}
   
   \caption{\it Nota{\c c}{\~a}o}
   \begin{tabular}{|c|} 
      \hline
      p \\
      \hline
      V \\
      \hline
      F \\
      \hline
   \end{tabular}
\label{notacao}
\end{table}


\begin{table}[h]
   \centering 
   \setlength{\arrayrulewidth}{0,5\arrayrulewidth}
   \caption{\it Tabela Verdade}
   \begin{tabular}{|c|c|c|c|c|} 
      \hline
      p & q \\
      \hline
      V & V \\
      \hline
      V & F\\
      \hline
      F & V \\
      \hline
      F & F \\
      \hline
   \end{tabular}
\label{tabelavdd}
\end{table}

\subsubsection{Nega{\c c}{\~a}o}
\begin{definicao}[Nega{\c c}{\~a}o] Chama-se nega{\c c}{\~a}o de uma proposi{\c c}{\~a}o p a proposi{\c c}{\~a}o representada por ``n{\~a}o p"($\neg p$), cujo valor l{\'o}gico {\'e} verdade quando p for falsa e a falsidade quando p for verdadeira.\end{definicao}
\begin{table}[h]
   \centering 
   \setlength{\arrayrulewidth}{0,5\arrayrulewidth}
   \caption{\it Tabela verdade: Nega{\c c}{\~a}o}
   \begin{tabular}{|c|c|c|c|c|} 
      \hline
     $ p$ & $\neg p$ \\
      \hline
      V & F \\
      \hline
      F & V \\
      \hline
   \end{tabular}
\end{table}

Exemplo:\\
\textbf{$p$}: 6 {\'e} par, $V(p)=V$\\
\textbf{$\neg p$}: 6 {\'e} {\'\i}mpar, $V(p)=F$\\
\textbf{$q$}: $4\leq 5$, $V(q)=V$\\
\textbf{$\neg q$}: $4>5$, $V(q)=F$

\subsubsection{Conjuga{\c c}{\~a}o}

\begin{definicao}[Conjuga{\c c}{\~a}o] Chama-se conjuga{\c c}{\~a}o de duas proposi{\c c}{\~o}es $p$ e $q$, a proposi{\c c}{\~a}o representada por ``$p$ e $q$", denotada $p\wedge q$, cujo valor l{\'o}gico {\'e} verdade quando as proposi{\c c}{\~o}es $p$ e $q$ s{\~a}o ambas verdadeiras e falsidade nos demais casos.\end{definicao}
\begin{table}[h]
   \centering 
   \setlength{\arrayrulewidth}{0,5\arrayrulewidth}
   \caption{\it Tabela verdade: Conjuga{\c c}{\~a}o}
   \begin{tabular}{|c|c|c|c|c|} 
      \hline
      $p$ & $q$ & $p\wedge q$ \\
      \hline
      V & V & V \\
      \hline
      V & F & F \\
      \hline
      F & V & F \\
      \hline
      F & F & F \\
      \hline
   \end{tabular}
\end{table}

Exemplo:\\
\textbf{$p$}: a neve {\'e} branca, $V(p)=V$\\
\textbf{$q$}: $2<5$, $V(q)=V$
\begin{center}
\textbf{$p\wedge q$}: A neve {\'e} branca E $2<5$, $V(p\wedge q)=V$
\end{center}

\textbf{$r$}: todo n{\'u}mero primo e {\'\i}mpar, $V(r)=F$
\begin{center}
$V(q\wedge r)=F$
\end{center}

\subsubsection{Disjun{\c c}{\~a}o}
\begin{definicao}[Disjun{\c c}{\~a}o]: Chama-se disjun{\c c}{\~a}o de duas proposi{\c c}{\~o}es $p$ e $q$ a proposi{\c c}{\~a}o representada por ``p ou q", denotada ``$p\vee q$", cujo valor l{\'o}gico {\'e} verdade quando ao menos uma das proposi{\c c}{\~o}es $p$ e $q$ forem verdadeiras, e falsidade quando ambas as proposi{\c c}{\~o}es $p$ e $q$ forem falsas.\end{definicao}

A tabela verdade da disjun{\c c}{\~a}o {\'e} (Tabela \ref{disjuncao}):
\begin{table}[h]
   \centering 
   \setlength{\arrayrulewidth}{0,5\arrayrulewidth}
   \caption{\it Tabela verdade: Disjun{\c c}{\~a}o}
   \begin{tabular}{|c|c|c|c|c|} 
      \hline
      $p$ & $q$ & $p\vee q$ \\
     \hline
      V & V & V \\
      \hline
      V & F & V \\
      \hline
      F & V & V \\
      \hline
      F & F & F \\
      \hline
   \end{tabular}
\label{disjuncao}
\end{table}

Exemplo:\\
\textbf{$P$}: Roma {\'e} a capital da R{\'u}ssia ou 9-5=4, $V(P)=V$\\
\textbf{$Q$}: $\pi=3\vee\sqrt{-1}=1$, $V(Q)=F$

\subsubsection{Disjun{\c c}{\~a}o exclusiva}
\begin{definicao}[Disjun{\c c}{\~a}o Exclusiva] Chama-se disjun{\c c}{\~a}o exclusiva de duas proposi{\c c}{\~o}es $p$ e $q$ a proposi{\c c}{\~a}o representada por $p\veebar q$, que se l{\^e} ``ou $p$ ou $q$", ou tamb{\'e}m ``$p$ ou $q$, mas n{\~a}o ambos", cujo valor l{\'o}gico {\'e} a verdade somente quando $p$ {\'e} verdade ou $q$ {\'e} verdade, mas n{\~a}o quando $p$ e $q$ s{\~a}o ambos verdadeiras, e tem valor l{\'o}gico falsidade nos demais casos.\end{definicao}
\begin{table}[h]
   \centering 
   \setlength{\arrayrulewidth}{0,5\arrayrulewidth}
   \caption{\it Tabela verdade: Disjun{\c c}{\~a}o Exclusiva}
   \begin{tabular}{|c|c|c|c|c|} 
      \hline
      $p$ & $q$ & $p\veebar q$ \\
     \hline
      V & V & F \\
      \hline
      V & F & V \\
      \hline
      F & V & V \\
      \hline
      F & F & F \\
      \hline
   \end{tabular}
\end{table}

Exemplo:\\
\textbf{$P$}: Todo n{\'u}mero inteiro ou {\'e} par ou {\'e} {\'\i}mpar, $V(P)=V$

\subsubsection{Condicional}
\begin{definicao}[Condicional] Chama-se proposi{\c c}{\~a}o condicional ou condicional uma proposi{\c c}{\~a}o representada por ``$p\rightarrow q$", que l{\^e}-se ``se $p$ ent{\~a}o $q$". O valor l{\'o}gico da condicional {\'e} a falsidade no caso em que $p$ {\'e} verdade e $q$ {\'e} falsidade e tem valor l{\'o}gico verdade nos demais casos.\end{definicao}

Na condicional ``$p\rightarrow q$", dizemos que $p$ {\'e} o antecedente e o $q$ {\'e} o conseq{\"u}ente. O s{\'\i}mbolo "$\rightarrow$" {\'e} chamado de s{\'\i}mbolo de implica{\c c}{\~a}o.
\begin{table}[h]
   \centering 
   \setlength{\arrayrulewidth}{0,5\arrayrulewidth}
   \caption{\it Tabela verdade: Condicional}
   \begin{tabular}{|c|c|c|c|c|} 
      \hline
      $p$ & $q$ & $p\rightarrow q$ \\
     \hline
      V & V & V \\
      \hline
      V & F & F \\
      \hline
      F & V & V \\
      \hline
      F & F & V \\
      \hline
   \end{tabular}
\end{table}

Exemplos:\\
\textbf{$P$}: Se o m{\^e}s de maio tem 31 dias, ent{\~a}o a terra {\'e} plana, $V(P)=F$\\
\textbf{$Q$}: Se 3+2=6 ent{\~a}o 4+4=9, $V(Q)=V$\\
\textbf{$R$}: Se (-1)=0, ent{\~a}o $\sin\displaystyle\frac{\pi}{6}=\displaystyle\frac{1}{2}$\\

Considere a seguinte condicional:
\begin{center}
7 {\'e} um n{\'u}mero {\'\i}mpar $\rightarrow$ Bras{\'\i}lia {\'e} uma cidade
\end{center}

Observa{\c c}{\~a}o: Uma condicional n{\~a}o afirma que o conseq{\"u}ente se deduz ou {\'e} uma conseq{\"u}{\^e}ncia do antecedente $p$. Uma condicional afirma unicamente uma rela{\c c}{\~a}o entre valores l{\'o}gicos de $p$ e $q$.

\subsubsection{Bicondicional}

\begin{definicao}[Bicondicional] Chama-se proposi{\c c}{\~a}o bicondicional ou bicondicional uma proposi{\c c}{\~a}o representada por ``Se, e somente se", denotada ``$p\leftrightarrow q(p\rightarrow q\wedge q\rightarrow p)$". O valor l{\'o}gico da bicondicional {\'e} verdade quando $p$ e $q$ s{\~a}o ambas verdade ou falsidade, e tem valor l{\'o}gico falsidade nos demais casos.\end{definicao}
\begin{table}[h]
   \centering 
   \setlength{\arrayrulewidth}{0,5\arrayrulewidth}
   \caption{\it Tabela verdade: Bicondicional}
   \begin{tabular}{|c|c|c|c|c|} 
      \hline
      $p$ & $q$ & $p\leftrightarrow q$ \\
     \hline
      V & V & V \\
      \hline
      V & F & F \\
      \hline
      F & V & F \\
      \hline
      F & F & V \\
      \hline
   \end{tabular}
\end{table}

Exemplos:\\
\textbf{$P$}: Roma fica na Europa se, e somente se, a neve {\'e} branca, $V(P)=V$\\
\textbf{$Q$}: Lisboa {\'e} a capital de Portugal se, e somente se, $\tan\displaystyle\frac{\pi}{4}=3$, $V(Q)=F$
\section{Tabela Verdade}

\hspace{0,5cm}Dadas v{\'a}rias proposi{\c c}{\~o}es simples $p,q,r,...$, podemos combin{\'a}-las formando novas proposi{\c c}{\~o}es atrav{\'e}s do uso dos conectivos l{\'o}gicos. Por exemplo:\\
\textbf{$P$}:$\neg p\vee(p\rightarrow q)$\\
\textbf{$R$}:$(p\leftrightarrow q)\wedge q$\\
\textbf{$S$}:$(p\rightarrow\neg q\vee r)\vee(q\vee(p\leftrightarrow\neg r))$

\subsubsection{Ordem de preced{\^e}ncia}

Na l{\'o}gica matem{\'a}tica, {\'e} convencionado a seguinte ordem de preced{\^e}ncia dos operadores:
\begin{enumerate}
\item $\neg$
\item $\wedge,\vee$
\item $\rightarrow,\leftrightarrow$
\end{enumerate}

Duas regras importantes devem ser observadas
\begin{enumerate}
\item A ordem de preced{\^e}ncia de uma opera{\c c}{\~a}o l{\'o}gica somente pode ser alterada atrav{\'e}s do uso de par{\^e}nteses
\item Operadores diferentes e de mesma prioridade necessariamente devem ter sua ordem indicada pelo uso de par{\^e}nteses
\end{enumerate}

Exemplos
\begin{enumerate}
\item $p\vee q\vee r\leftrightarrow\neg p$ (Correto)
\item $p\vee q\vee(r\leftrightarrow\neg p)$ (Correto)
\item $p\wedge q\vee r$ (Errado)
\item $p\rightarrow q\leftrightarrow p$ (Errado)
\end{enumerate}

Observa{\c c}{\~a}o: A coloca{\c c}{\~a}o de par{\^e}nteses pode alterar o valor l{\'o}gico (e o sentido) de uma proposi{\c c}{\~a}o

\begin{center}
Exemplo
\end{center} 
\begin{minipage}[l]{0,5\textwidth}
I)\\
$V\vee F\rightarrow F$\\
$V\rightarrow F$\\
$F$\\
\end{minipage}
\begin{minipage}[r]{0,5\textwidth}
II)\\
$V\vee(F\rightarrow F)$\\
$V\vee V$\\
$V$

\end{minipage}

\section{Constru{\c c}{\~a}o de Tabelas Verdade}

\hspace{0,5cm}Para construir a tabela verdade de uma proposi{\c c}{\~a}o composta $P(p,q,r,...)$ come{\c c}amos contando o n{\'u}mero de proposi{\c c}{\~o}es simples que comp{\~o}em $P(p,q,r,...)$.

\subsubsection{N{\'u}mero de linhas}
Se $P(p,q,r,...)$ for composta por $n$ proposi{\c c}{\~o}es simples, ent{\~a}o a tabela verdade de $P(p,q,r,...)$ conter{\'a} $2^{n}$ linhas.

Exemplos: Construir a tabela verdade das seguintes proposi{\c c}{\~o}es:
\begin{enumerate}
\item $P:\neg(p\wedge\neg q)$ (Tabela \ref{1})
\begin{table}[h]
   \centering 
   \setlength{\arrayrulewidth}{0,5\arrayrulewidth}
   \caption{\it $P:\neg(p\wedge\neg q)$}
   \begin{tabular}{|c|c|c|c|c|} 
      \hline
      $p$ & $q$ & $\neg q$ & $p\wedge\neg q$ & $\neg(p\wedge\neg q)$ \\
     \hline
      V & V & F & F & V \\
      \hline
      V & F & V & V & F \\
      \hline
      F & V & F & F & V \\
      \hline
      F & F & V & F & V \\
      \hline
   \end{tabular}
\label{1}
\end{table}
\item $Q:\neg(p\wedge q)\vee\neg(p\leftrightarrow p)$ (Tabela \ref{2})
\begin{table}[h]
   \centering 
   \setlength{\arrayrulewidth}{0,5\arrayrulewidth}
   \caption{\it $Q:\neg(p\wedge q)\vee\neg(p\leftrightarrow p)$}
   \begin{tabular}{|c|c|c|c|c|c|c|} 
      \hline
      $p$ & $q$ & $p\wedge q$ & $p\leftrightarrow q$ & $\neg(p\wedge q)$ & $\neg(p\leftrightarrow q)$ & Q \\
     \hline
      V & V & V & V & F & F & F \\
      \hline
      V & F & F & F & V & V & V\\
      \hline
      F & V & F & F & V & V & V \\
      \hline
      F & F & F & V & V & V & V \\
      \hline
   \end{tabular}
\label{2}
\end{table}
\end{enumerate}

\subsection{Tautologia}
\begin{definicao}[Tautologia] Chama-se tautologia a proposi{\c c}{\~a}o composta que {\'e} sempre verdadeira independentemente dos valores l{\'o}gicos das proposi{\c c}{\~o}es que a comp{\~o}em.\end{definicao}

Na tabela verdade de uma tautologia, a {\'u}ltima coluna cont{\'e}m somente o valor l{\'o}gico verdadeiro.

\subsection{Contradi{\c c}{\~a}o}
\begin{definicao}[Contradi{\c c}{\~a}o] Chama-se contradi{\c c}{\~a}o a proposi{\c c}{\~a}o composta que {\'e} sempre falsa independentemente dos valores l{\'o}gicos das proposi{\c c}{\~o}es que a comp{\~o}em.\end{definicao}

Assim, na tabela verdade de uma contradi{\c c}{\~a}o, a {\'u}ltima coluna cont{\'e}m somente o valor l{\'o}gico falso.

\section{Implica{\c c}{\~a}o}
\begin{definicao}[Implica{\c c}{\~a}o] Dizemos que uma proposi{\c c}{\~a}o $P(p,q,r,..)$ implica uma proposi{\c c}{\~a}o composta $Q(p,q r,...)$, denotado "$P\Rightarrow Q$", se para todo valor verdade da primeira, ent{\~a}o a segunda {\'e} verdadeira.\end{definicao}

Assim, $P\Rightarrow Q$ somente se a condicional $P\rightarrow Q$ for uma tautologia.

Exemplo: Sendo $P:p\wedge q$ e $Q:p\vee q$, verificar se $P\Rightarrow Q$.

Precisamos verificar se a condicional $P\rightarrow Q$ {\'e} uma tautologia (Tabela \ref{3}).
\begin{table}[h]
   \centering 
   \setlength{\arrayrulewidth}{0,5\arrayrulewidth}
   \caption{\it $P\rightarrow Q$}
   \begin{tabular}{|c|c|c|c|c|} 
      \hline
      $p$ & $q$ & $p\wedge q$ & $p\vee q$ & $(p\wedge q)\rightarrow(p\vee q)$ \\
     \hline
      V & V & V & V & V \\
      \hline
      V & F & F & V & V \\
      \hline
      F & V & F & V & V \\
      \hline
      F & F & F & F & V \\
      \hline
   \end{tabular}
\label{3}
\end{table}

Observa{\c c}{\~a}o:
\begin{enumerate}
\item Os s{\'\i}mbolos $\rightarrow$ e $\Rightarrow$ s{\~a}o diferentes. A condicional, $\rightarrow$, {\'e} um operador l{\'o}gico que aplicado a duas proposi{\c c}{\~o}es $p$ e $q$, por exemplo, produz uma nova proposi{\c c}{\~a}o $p\rightarrow q$. Por outro lado, a implica{\c c}{\~a}o, $\Rightarrow$, estabelece que $p\rightarrow q$ {\'e} uma tautologia.
\item Toda teorema {\'e} uma aplica{\c c}{\~a}o da forma
\begin{center}
Hip{\'o}tese $\Rightarrow$ Tese
\end{center}

Logo demonstrar um teorema significa mostrar que n{\~a}o ocorre o caso da hip{\'o}tese ser verdadeira e a tese falsa, isto {\'e}, a verdade da hip{\'o}tese {\'e} suficiente para garantir a verdade da tese.
\end{enumerate}

\subsection{Equival{\^e}ncia}

\begin{definicao}[Equival{\^e}ncia] Dizemos que uma proposi{\c c}{\~a}o $P(p,q,r,...)$ {\'e} equivalente a uma proposi{\c c}{\~a}o composta $Q(p,q,r,...)$, denotado por $P\Leftrightarrow Q$, se elas implicarem uma na outra. Assim, $P\Leftrightarrow Q$ se a bicondicional {\'e} uma tautologia.\end{definicao}

Exemplo: Sendo $P:p\leftrightarrow q$ e $Q:(p\rightarrow q)\wedge(q\rightarrow p)$ verificar que $P\Leftrightarrow Q$.\\

Precisamos verificar se a bicondicional $P\leftrightarrow Q$ {\'e} uma tautologia (Tabela \ref{4}).
\begin{table}[h]
   \centering 
   \setlength{\arrayrulewidth}{0,5\arrayrulewidth}
   \caption{\it $P\leftrightarrow Q$}
   \begin{tabular}{|c|c|c|c|c|c|c|} 
      \hline
      $p$ & $q$ & $p\rightarrow q$ & $q\leftrightarrow p$ & $(p\rightarrow q)\wedge(q\rightarrow p)$ & $p\leftrightarrow q$ & $P\leftrightarrow Q$ \\
     \hline
      V & V & V & V & V & V & V \\
      \hline
      V & F & F & V & F & F & V\\
      \hline
      F & V & V & F & F & F & V \\
      \hline
      F & F & V & V & V & V & V \\
      \hline
   \end{tabular}
\label{4}
\end{table}

\chapter{No{\c c}{\~o}es de Teoria de Conjuntos}
\section{Conceitos b{\'a}sicos}

Um conjunto {\'e} uma ``cole{\c c}{\~a}o" ou ``fam{\'\i}lia" de elementos.

Usaremos letras mai{\'u}sculas do alfabeto para denotar os conjuntos e denotaremos elementos por letras min{\'u}sculas do alfabeto.

Dado um conjunto $A$, para indicar o fato de que $a$ {\'e} um elemento de $A$, escrevemos:
\[a \in A\]

Para dizer que um elemento $b$ n{\~a}o pertence ao conjunto $A$, escrevemos:
\[b \notin A\]

Um conjunto sem elementos {\'e} chamado de vazio ou conjunto vazio. Tal conjunto {\'e} denotado por $\emptyset$.

Dado um conjunto $A$ e $a$ um elemento, ocorre sempre o seguinte:
\[a \in A \mbox{ ou } a \notin A\]

Al{\'e}m disso, para dois elementos $a,b\in A$, ocorre exatamente o seguinte:
\[a=b \mbox{ ou } a\neq b\]

\section{Descri{\c c}{\~a}o de um conjunto}

Um conjunto $A$ pode ser dado pela simples listagem dos seus elementos, como por exemplo:
\[ A= \{1,2,3,4,5\}\]
\[B = \{verdade,\ falso\}\]

Um conjunto tamb{\'e}m pode ser dado pela descri{\c c}{\~a}o das propriedades dos seus elementos, como por exemplo:

$A=\{n \mid n$ {\'e} m{\'u}ltiplo de 2$\}=\{2,4,6,...\}$

\section{Alguns conjuntos importantes}
\begin{enumerate}
\item $\mathbb{N}=\{1,2,3,...\}$ o conjunto do n{\'u}meros naturais
\item $\mathbb{Z}=\{...,-2,-1,0,1,2,...\}$ o conjunto dos n{\'u}meros inteiros
\item $\mathbb{N}_{0}=\{0,1,2,3,...\}$ o conjunto dos n{\'u}meros inteiros n{\~a}o negativos
\item $\mathbb{R}$ o conjunto dos n{\'u}meros reais
\item $\mathbb{R}^{*}$ o conjunto dos n{\'u}meros reais n{\~a}o nulos
\item $\mathbb{Q}=\left\{\dfrac{p}{q} \mid p,q \in \mathbb{Z}, q \neq 0 \right\}$ o conjunto dos n{\'u}meros racionais
\end{enumerate}

\section{Propriedades dos conjuntos}

\subsubsection{Igualdade entre conjuntos}

Dados dois conjuntos $A$ e $B$, dizemos que $A$ e $B$ s{\~a}o iguais se, e somente se, eles t{\^e}m os mesmos elementos, ou seja, para todo $a\in A$ temos que $a\in B$ e para todo $b\in B,b\in A$.

Se $A$ e $B$ s{\~a}o iguais, escrevemos $A = B$
\[ \{1,2,3,4\} = \{3,2,1,4\} \]
\[ \{1,2,3\} \neq \{2,3\} \]

\subsubsection{Contin{\^e}ncia de conjuntos}

Se $A$ e $B$ s{\~a}o dois conjuntos, dizemos que A {\'e} um subconjunto de $B$ se todo elemento de $A$ for elemento de $B$, ou seja, se para todo elemento $a$, a seguinte implica{\c c}{\~a}o for verdadeira.
\[ a \in A \Rightarrow a \in B\]

Nesse caso, denota-se $A \subseteq B$ ou $B \supseteq A$.

Para o caso onde $A$ {\'e} um subconjunto de $B$ mas n{\~a}o {\'e} igual a $B$, escrevemos:
\begin{center}
$A \subsetneq B$(Em alguns livros, usa-se a nota{\c c}{\~a}o $A \subset B$ para contin{\^e}ncia pr{\'o}pria)
\end{center}

Nesse caso, dizemos que $A$ {\'e} um subconjunto pr{\'o}prio de $B$.

Para dizer que $A$ n{\~a}o est{\'a} contido em $B$, escrevemos $A \nsubseteq B$

Assim, a igualdade entre dois conjuntos pode ser descrita como
\[ A = B \Leftrightarrow A \subseteq B \mbox{ e } B \subseteq A\]

Ou seja, se $A = B$ ent{\~a}o $A \subseteq B$ e $B \subseteq A$, por outro lado, se $A \subseteq B$ e $B \subseteq A$, ent{\~a}o $A = B$

Quando $A$ e $B$ n{\~a}o s{\~a}o iguais, escrevemos $A \neq B$. Assim, temos
\[ A \neq B \Leftrightarrow A \nsubseteq B \mbox{ ou } B \nsubseteq A\]

\subsection{Propriedades da contin{\^e}ncia}
Para quaisquer 3 conjuntos $A,B$ e $C$ temos
\begin{enumerate}
\item $A\subseteq A$ (Lei reflexiva)
\item Se $A\subseteq B \mbox{ e } B\subseteq A$, ent{\~a}o $A=B$ (Lei anti-sim{\'e}trica)
\item Se $A\subseteq B$ e $B\subseteq C$, ent{\~a}o $A\subseteq C$ (Lei transitiva)
\end{enumerate}

Considere os seguintes conjuntos:
\[A = \{ n \in \mathbb{N} \mid n \mbox{ {\'e} m{\'u}ltiplo de } 2\}=\{2,4,6,...\}\]
\[ B = \{n\in\mathbb{N} \mid n \mbox{ {\'e} m{\'u}ltiplo de } 3\}=\{3,6,9,...\}\]

Neste caso, $2 \in A$ e $2 \in B$, logo $A \nsubseteq B$. Por outro lado, $3 \in B$ e $3 \notin A$, logo $B \nsubseteq A$.

Assim, dados 2 conjuntos $A$ e $B$, nem sempre temos que $A \subseteq B$ ou $B \subseteq A$.

\begin{proposicao} Seja $A$ um conjunto. Ent{\~a}o $ \emptyset \subseteq A$.\end{proposicao}

\textbf{Demonstra{\c c}{\~a}o} Suponha que $ \emptyset \nsubseteq A$. Logo existe $x \in \emptyset$ e $x \in A$. Mas por defini{\c c}{\~a}o o conjunto vazio n{\~a}o cont{\'e}m elementos. Logo {\'e} um absurdo ou uma contradi{\c c}{\~a}o existir $z \in \emptyset$. Portanto, $ \emptyset \subseteq A$, como quer{\'\i}amos demonstrar.\#

\section{Rela{\c c}{\~o}es entre conjuntos}

\subsubsection{Intersec{\c c}{\~a}o}

\begin{definicao}[Intersec{\c c}{\~a}o] Sejam $A$ e $B$ dois conjuntos. Definimos a intersec{\c c}{\~a}o de $A$ e $B$ como sendo o conjunto $A \cap B$ cujos elementos pertencem ao conjunto $A$ e $B$ simultaneamente. Assim,
\[ A \cap B = \{x \mid x \in A\mbox{ e }  x \in B\}\]
\end{definicao}

Exemplo: Sejam
\[ A = \{1,2,3\},\ B = \{2,3,4\}\]
\[A \cap B = \{2,3\}\]

\begin{proposicao} Sejam $A$ e $B$ dois conjuntos. Ent{\~a}o
\[(A \cap B) \subseteq A  \mbox{ e } (A \cap B) \subseteq B.\]
\end{proposicao}

\textbf{Demonstra{\c c}{\~a}o} Seja $x \in A \cap B$ um elemento arbitr{\'a}rio. Logo $x \in A$ e $x \in B$. De $x \in A$ temos que $A \cap B \subseteq A$. De $x \in B$ temos que $A \cap B \subseteq B$. Como quer{\'\i}amos demonstrar.\#

\subsubsection{Uni{\~a}o}

\begin{definicao}[Uni{\~a}o] Sejam $A$ e $B$ dois conjuntos. Definimos a uni{\~a}o de $A$ com $B$ como sendo o conjunto $A \cup B$, cujos elementos pertencem ao conjunto $A$ ou ao conjunto $B$. Assim,
\[A \cup B = \{x \mid x \in A \mbox{ ou } x \in B\}.\]
\end{definicao}

Exemplo: Sejam
\[A = \{1,2,3\},\ B = \{2,3,4\}\]
\[A \cup B = \{1,2,3,4\}\]

O conceito de uni{\~a}o($ \cup $) e intersec{\c c}{\~a}o($ \cap $) pode ser estendido para mais de 2 conjuntos.

\subsubsection{Uni{\~a}o e Intersec{\c c}{\~a}o de m{\'u}ltiplos conjuntos}
\begin{definicao}[Uni{\~a}o e Intersec{\c c}{\~a}o de m{\'u}ltiplos conjuntos] Sejam $A_{1},...,A_{n}$ $n$ conjuntos dados. Ent{\~a}o
\[A_{1} \cup A_{2} \cup \cdots \cup A_{n}= \displaystyle\bigcup_{k=1}^{n} A_{k}\]
{\'e} o conjunto dos elementos $x$ tais que $x$ pertence a pelo menos um dos conjuntos $A_{1}$, ..., $A_{n}$. Agora,
\[A_{1} \cap \cdots \cap A_{n} = \displaystyle\bigcap_{k=1}^{n}A_{k}\]
{\'e} o conjunto dos elementos $x$ que pertencem a todos os conjuntos $A_{1}$, ..., $A_{n}$ simultaneamente.
\end{definicao}

Quando a intersec{\c c}{\~a}o de dois ou mais conjuntos {\'e} vazia, dizemos que eles s{\~a}o conjuntos disjuntos.

Se $C = A \cup B$, tais que $A \cap B = \emptyset$, dizemos que $C$ {\'e} uma uni{\~a}o disjunta de $A$ e $B$. Neste caso, escrevemos
\[C = A \sqcup B\]

\begin{proposicao} Sejam $A,\ B$ e $C$ tr{\^e}s conjuntos, ent{\~a}o
\begin{enumerate}
\item $A\cap(B\cup C)=(A\cap B)\cup(A\cap C)$
\item $A\cup(B\cap C)=(A\cup B)\cap(A\cup C)$
\end{enumerate}
\end{proposicao}

\textbf{Demonstra{\c c}{\~a}o}:
\begin{enumerate}
\item Precisamos mostrar que
\[A\cap(B\cup C)\subseteq(A\cap B)\cup(A\cap C)\]
e
\[(A\cap B)\cup(A\cap C)\subseteq A\cap(B\cup C).\]

Seja $x\in A\cap(B\cup C)$. Logo $x\in A$ e $x\in B\cup C$. Agora, de $x\in B\cup C$, temos que $x\in B$ ou $x\in C$. Suponha que $x\in B$. Como $x\in A$, segue que $x\in A\cap B$. Assim, $x\in(A\cap B)\cup(A\cap C)$, ou seja, $A\cap(B\cup C)\subseteq(A\cap B)\cup(A\cap C)$. Por outro lado, se $x\in C$, ent{\~a}o $x\in C$, ent{\~a}o $x\in A\cap C$ e da{\'\i} $x\in(A\cap B)\cup(A\cap C)$, logo $A\cap(B\cup C)\subseteq(A\cap B)\cup(A\cap C)$.

Portanto,
\[A\cap(B\cup C)\subseteq(A\cap B)\cup(A\cap C).\]

Reciprocamente, seja $x\in(A\cap B)\cup(A\cap C)$. Assim, $x\in A\cap B$ ou $x\in A\cap C$. Suponha que $x\in A\cap B$. Da{\'\i}, $x\in A$ e $x\in B$. Como $x\in B$, segue que $x\in B\cup C$ e ent{\~a}o $x\in A\cap(B\cup C)$, ou seja, $(A\cap B)\cup(A\cap C)\subseteq A\cap(B\cup C)$. Agora, supondo que $x\in A\cap C$. Da{\'\i} $x\in A$ e $x\in C$. Desse modo, $x\in B\cup C$, ent{\~a}o $x\in A\cap(B\cup C)$, isto {\'e}
\[(A\cap B)\cup(A\cap C)\subseteq A\cap(B\cup C).\]

Portanto temos \[A\cap(B\cup C)=(A\cap B)\cup(A\cap C).\]

Como quer{\'\i}amos demonstrar.\#
\item O mesmo racioc{\'\i}nio se aplica, \textit{Mutatis Mutandis}, {\`a} segunda demonstra{\c c}{\~a}o.
\end{enumerate}

\subsubsection{Diferen{\c c}a de Conjuntos}
\begin{definicao}[Diferen{\c c}a de Conjuntos] Dados dois conjuntos $A$ e $B$, definimos a diferen{\c c}a dos conjuntos $A$ e $B$, denotado $A-B$ (ou $A\backslash B)$
\[A - B = \{x | x \in A \mbox{ e } x \notin B\}.\]
\end{definicao}

Exemplos:
\begin{enumerate}
\item $A=\{1,2,3,5,4\}$, $B=\{2,3,6,8\}$, $A-B=\{1,4,5\}$, $B-A=\{6,8\}$
\item $A=\{2,4,6,8,10,...\}$,  $B=\{3,6,9,12,15,...\}$, $A-B=\{2,4,8,10,14,16,...\}$, $B - A=\{3,9,15,21,...\}$
\end{enumerate}

\subsubsection{Complementar}

\begin{definicao}[Complementar] Dados dois conjuntos $A$ e $E$ tais que $A\subseteq E$, definimos o complementar de $A$ em $E$, denotado $A^{C}$ ou $C_{E}(A)$, como
\[C_{E}(A) = \{ x \in E | x \in A \}.\]
\end{definicao}

Observa{\c c}{\~o}es:
\begin{enumerate}
\item Se $A = E$, ent{\~a}o $C_{A}(A)=\{ x \in A \mid x \notin A \}=\emptyset$
\item $(A^{C})^{C}=\{x \in E \mid x \notin A^{C}\} = \{ x \in E \mid x \in A \}=A$
\end{enumerate}

Exemplo:\\
$A=\{1,2,3,4\},\ E=\{1,2,3,5,4,0,8,9\}$\\
$A^{c}=\{0,8,9\}$

\begin{proposicao} Sejam $A,B,E$ conjuntos. Se $A\subseteq B\subseteq E$, ent{\~a}o $C_{E}(B)\subseteq C_{E}(A)$.\end{proposicao}

\textbf{Demonstra{\c c}{\~a}o}: Seja $x\in B^{c}$, logo $x\notin B$ e assim $x\notin A$, pois $A\subseteq B$. Ent{\~a}o $x\in A^{c}$, ou seja, $B^{c}\subseteq A^{c}$. (Q.E.D.).\#\\

\begin{proposicao} Sejam $A,B,E$ tr{\^e}s conjunto tais que $A\subseteq E$ e $B\subseteq E$. Ent{\~a}o:
\begin{enumerate}
\item $(A\cup B)^{c}=A^{c}\cap B^{c}$
\item $(A\cap B)^{c}=A^{c}\cup B^{c}$
\end{enumerate}
\end{proposicao}

\textbf{Demonstra{\c c}{\~a}o}: Seja $x\in(A\cup B)^{c}$. Logo $x\in A\cup B$, assim $x\notin A\wedge x\notin B$. Da{\'\i}, $x\in A^{c}\wedge x\in B^{c}$, isto {\'e}, $x\in A^{c}\cap B^{c}$. Desse modo,
\begin{equation}
(A\cup B)^{c}\subseteq A^{c}\cap B^{c}
\end{equation}

Por outro lado, se $x\in A^{c}\cap B^{c}$, ent{\~a}o $x\in A^{c}\wedge x\in B^{c}$. Da{\'\i}, $x\notin A\wedge x\notin B$, ou seja, $x\notin A\cup B$, logo $x\in (A\cup B)^{c}$. Desse modo
\begin{equation}
A^{c}\cap B^{c}\subseteq(A\cup B)^{c}
\end{equation}

Portanto, de (2.1) e (2.2) temos \[(A\cup B)^{c}=A^{c}\cap B^{c}\]

(Q.E.D).\#\\

\subsubsection{Produto Cartesiano}

\begin{definicao}[Produto Cartesiano] Dados 2 conjuntos $A$ e $B$, definimos o produto cartesiano $A$ e $B$ por
\begin{center}
$A\times B=\{(a,b)/a\in A,b\in B\}$
\end{center}
\end{definicao}

Como o conjunto dos pares ordenados $(a,b)$, onde $a$ percorre $A$ e $b$ percorre $B$.

Dados $(a,b),(c,d)\in A$x$B$, temos $(a,b)=(c,d)$ se, e somente se, \[a=c\wedge b=d\]

Em geral, $A$x$B\neq B$x$A$.

Exemplo:\\
$A=\{1,2\},\ B=\{3\}$\\
$A$x$B=\{(1,3),(2,3)\}$\\
$B$x$A=\{(3,1),(3,2)\}$

\subsubsection{Conjunto partes}

\begin{definicao}[Conjunto Partes] Para qualquer conjunto $A$, indicamos por $P(A)$\[P(A)=\{X/X\subseteq A\}\] o conjunto das partes de $A$.\end{definicao}

Os elementos desse conjunto s{\~a}o todos os subconjuntos de $A$. Dizer que $x\in P(A)$ significa que $x\in A$. Particularmente, temos $\emptyset\in P(A)$ e $A\in P(A)$.

Exemplos:
\begin{enumerate}
\item $A=\emptyset,\ P(A)=\{\emptyset\}$
\item $B=\{b\},\ P(B)=\{\emptyset,B\}$
\item $C=\{a,b,c\}$\\
$P(C)=\{\emptyset, \{a\}, \{b\},\{c\},\{a,b\},\{a,c\},\{b,c\},C\}$
\item $D=\mathbb{R},P(D)=\{X/X\subseteq\mathbb{R}\}$, por exemplo $\mathbb{Q}\in P(D)$
\end{enumerate}

\chapter{N{\'u}meros Inteiros}
\section{Conceitos b{\'a}sicos}

\hspace{0,5cm}Indicaremos por $\mathbb{Z}$ o conjunto dos n{\'u}meros inteiros. Portanto $\mathbb{Z}=\{0,\pm 1,\pm 2,\pm 3, \pm 4,...\}$.

\subsubsection{Propriedades b{\'a}sicas da adi{\c c}{\~a}o e da multiplica{\c c}{\~a}o}

Admitiremos as propriedades b{\'a}sicas da adi{\c c}{\~a}o e da multiplica{\c c}{\~a}o em $\mathbb{Z}$. Assim, dados $a,b,c\in\mathbb{Z}$, temos:\\

\begin{minipage}[l]{0,5\textwidth}
Adi{\c c}{\~a}o
\begin{enumerate}
\item $a+b=b+a$
\item $a(b+c)=(a+b)+c$
\item $a+0=a$
\item $a+(-a)=0$
\end{enumerate}
\end{minipage}
\begin{minipage}[r]{0,5\textwidth}
Multiplica{\c c}{\~a}o
\begin{enumerate}
\item $ab=ba$
\item $a(bc)=(ab)c$
\item $a1=a$
\item $ab=0\rightarrow a=0\vee b=0$
\item $ab=1\rightarrow a=\pm 1\wedge b=\pm 1$
\item $a(b+c)=ab+ac$
\end{enumerate}
\end{minipage}

\subsubsection{Propriedades b{\'a}sicas das desigualdades}

Admitiremos tamb{\'e}m a rela{\c c}{\~a}o "menor ou igual", em $\mathbb{Z}$, denotada por "$\leq$". Dados $a,b,c\in\mathbb{Z}$, valem as seguintes propriedades:
\begin{enumerate}
\item $a\leq A$
\item $a\leq b\wedge b\leq a\rightarrow a=b$
\item $a\leq b\wedge b\leq c\rightarrow a\leq c$
\item $a\leq b\veebar b\leq a$
\item $a\leq b\rightarrow a+c\leq b+c$
\item $0\leq a\wedge 0\leq b\rightarrow 0\leq ab$
\end{enumerate}

Para a rela{\c c}{\~a}o "menor", cujo s{\'\i}mbolo {\'e} "$<$", vale
\begin{enumerate}
\item $0<a\wedge 0<b\rightarrow 0<ab$
\item $0<a\wedge b<0\rightarrow ab<0$
\item $0<a\wedge 0<b\rightarrow 0<ab$
\end{enumerate}

\section{Princ{\'\i}pio da boa ordena{\c c}{\~a}o}

\begin{definicao}[Limite Inferior] Seja $A$ um subconjunto n{\~a}o vazio de $\mathbb{Z}$. Dizemos que $A$ {\'e} limitado inferiormente se $\exists\ell\in\mathbb{Z}/\ell\leq a,\ \forall a\in A$.\end{definicao}

Por exemplo:\\
$A=\{-2,0,1,2,3,...\},\ B=\{...,-6,-4,-2,0\},\ C=\{8,16,24,32\}$

$A$ e $C$ s{\~a}o limitados inferiormente pois $-3\leq a,\ 7\leq c,\forall a\in A\wedge\forall c\in C$.

\textbf{Princ{\'\i}pio da boa ordena{\c c}{\~a}o}: Se $A$ {\'e} um subconjunto n{\~a}o vazio de $\mathbb{Z}$ e $A$ {\'e} limitado inferiormente, ent{\~a}o $\exists a_{0}\in A/a_{0}\leq x,\forall x\in A$.\\

Seja $A\neq\emptyset, A\subseteq\mathbb{Z},\ A$ limitada inferiormente. Pelo P.B.O., $\exists a_{0}\in A/a_{0}\leq x,\forall x\in A$. Suponha que existe $a_{1}\in A/a_{1}\leq x, \forall x\in A$. Logo $a_{0}\leq a_{1}\wedge a_{1}\leq a_{0}$, da{\'\i} $a_{1}=a_{0}$.\\

Logo, o elemento $a_{0}\in A$ do P.B.O. {\'e} {\'u}nico. Chamamos $a_{0}$ de elemento m{\'\i}nimo ou elemento minimal.\\

\section{Princ{\'\i}pio da Indu{\c c}{\~a}o Finita}

\begin{teo}[Indu{\c c}{\~a}o finita (1� vers{\~a}o)] : Dado $a\in\mathbb{Z}$, suponhamos que a cada inteiro $n\geq a$ esteja associada uma proposi{\c c}{\~a}o $P(n)$ que depende de $n$. Ent{\~a}o $P(n)$ ser{\'a} verdadeira para todo $n\geq a$ desde que seja poss{\'\i}vel provar o seguinte:
\begin{enumerate}
\item $P(a)$ {\'e} verdadeira
\item Dado $r>a$, se $P(k)$ {\'e} verdadeira para todo $k$ tal que $a\leq k\leq r$, ent{\~a}o $P(r)$ {\'e} verdadeira.
\end{enumerate}
\end{teo}

\begin{teo}[Indu{\c c}{\~a}o finita (2� vers{\~a}o)] Dado $a\in\mathbb{Z}$, suponhamos que para cada $n\geq a$ esteja associada uma proposi{\c c}{\~a}o $P(n)$. Ent{\~a}o $P(n)$ {\'e} verdadeira $\forall n\geq 1$ desde que seja poss{\'\i}vel provar o seguinte:
\begin{enumerate}
\item $P(a)$ {\'e} verdadeira
\item Se $P(r)$ {\'e} verdadeira para $r\geq a$, ent{\~a}o $P(r+1)$ {\'e} verdadeira
\end{enumerate}
\end{teo}

Exemplo:\\

Mostre que $\forall n\in\mathbb{N}$ vale \[1+2+3+...+n=\displaystyle\frac{n(n+1)}{2}\]

Para $n=1$, temos
\[1=\displaystyle\frac{1(1+1)}{2}\]

Agora, suponha que para $r\geq 1$, temos \[\underbrace{1+2+...+r=\displaystyle\frac{r(r+1)}{2}}_{H.I}\]

Assim, para $r+1$ temos \[1+2+3+...+r+(r+1)\]

Pela Hip{\'o}tese de Indu{\c c}{\~a}o, temos \[1+2+...+r+(r+1)=\displaystyle\frac{r(r+1)}{2}+(r+1)=\displaystyle\frac{r(r+1)+2(r+1)}{2}\] \[=\displaystyle\frac{(r+2)(r+1)}{2}\]

Portanto, pelo princ{\'\i}pio da indu{\c c}{\~a}o finita \[1+2+...+n=\displaystyle\frac{n(n+1)}{2}\]

\begin{teo} Dado $a\in\mathbb{Z}$, suponhamos que cada inteiro $n\geq a$ esteja associado uma proposi{\c c}{\~a}o $P(n)$. Ent{\~a}o $P(n)$ ser{\'a} verdadeira $\forall n\geq a$ desde que seja poss{\'\i}vel provar que:
\begin{enumerate}
\item $P(a)$ {\'e} verdadeira
\item Dado que $r>a$, se $P(k)$ {\'e} verdadeira para todo $k$ tal que $a\leq k\leq r$, ent{\~a}o $P(r)$ {\'e} verdadeira
\end{enumerate}
\end{teo}

\textbf{Demonstra{\c c}{\~a}o}: Seja $F=\{\ell\in\mathbb{Z}/a\leq\ell\wedge P(\ell)$ {\'e} falsa$\}$. Suponha $F\neq\emptyset$. Como $F$ {\'e} limitado inferiormente, pelo princ{\'\i}pio da boa ordena{\c c}{\~a}o, existe $\ell_{0}\in F/\ell_{0}\leq x,\forall x\in F$. Como $\ell_{0}\in F,\ P(\ell_{0})$ {\'e} falsa. Mas $P(a)$ {\'e} verdadeira, assim, $\ell_{0}>a$. Agora, como $\ell_{0}$ {\'e} o m{\'\i}nimo de $F$, ent{\~a}o $P(x)$ {\'e} verdadeira para $a\leq x<\ell_{0}$.\\

De 2 temos que $P(\ell_{0})$ {\'e} verdadeira, o que {\'e} uma contradi{\c c}{\~a}o, pois verificamos anteriormente que $P(\ell_{0})$ {\'e} falso.\\

Portanto $F=\emptyset$ e o teorema est{\'a} demonstrado.\#

\section{Divisibilidade}

\begin{definicao}[Divis{\~a}o] Sejam $a,b$ n{\'u}meros inteiros, $b\neq\emptyset$. Dizemos que $b$ divide $a$ quando existe um inteiro $c$ tal que $a=bc$.\end{definicao}

Exemplos:
\begin{enumerate}
\item Os inteiros 1 e $-1$ dividem todos os n{\'u}meros inteiros $a$, pois \[a=1a,a=(-1)(-a)\]
\item O n{\'u}mero 0 n{\~a}o divide nenhum inteiro $b$, pois n{\~a}o existe $a$ tal que $b=0a$
\item Para todo $b\neq 0$,$b$ divide $\pm b$
\item Para todo inteiro $b\neq 0$, $b$ divide 0, pois $0=b0$
\item 3 n{\~a}o divide 8, mas 17 divide 51
\end{enumerate}

\begin{nota}[Divis{\~a}o] Quando $b$ divide $a$, escrevemos $b|a$. Quando $b$ n{\~a}o divide $a$, escrevemos $b\not{|}a$\end{nota}

\textbf{Propriedades}
\begin{enumerate}
\item $a|a, \forall a\in\mathbb{Z}$
\item Se $a|b$ e $b|a,\ a,b\geq 0\rightarrow a=b$

De fato existe $c,d\in\mathbb{Z}/b=ca\wedge a=bd$. Se $a=0\vee b=0$ ent{\~a}o $b=0\veebar a=0$. Podemos supor $a\neq 0$ e $b\neq 0$.

Assim\\
$b=c(bd)$\\
$b(1-cd)=0$. Da{\'\i}, $1-cd=0$, isto {\'e}, $cd=1$.

Assim, $c=\pm 1\wedge d=\pm 1$. Como $a>0$ e $b>0$, devemos ter $c=d=1$. Portanto $a=b$
\item Se $a|b$ e $b|c$, ent{\~a}o $a|c$

De fato, $b=pa\wedge c=bq \Rightarrow c=(pq)a$, ou seja, $a|c$
\item Se $a|b$ e $a|c$, ent{\~a}o $a|(bx+cy)$, para todos $x,y\in\mathbb{Z}$

Temos $b=ap$ e $c=aq$, $p,q\in\mathbb{Z}$
\[bx+cy=apx+aqy=a\underbrace{(px+qy)}_{\in\mathbb{Z}}\]

Logo $a|(bx+cy)$
\end{enumerate}

\section{Algoritmo de divis{\~a}o de Euclides}

\begin{teo}[Algoritmo de divis{\~a}o de Euclides] Para quaisquer $a,b\in\mathbb{Z}$, com\\ $b>0$, existem {\'u}nicos $q$ e $r$ inteiros tais que $a=bq+r$, com $0\leq r<b$.\end{teo}

\textbf{Demonstra{\c c}{\~a}o}: Vamos mostrar primeiro a exist{\^e}ncia de $q$ e $r$.

Seja $M=\{m\in\mathbb{Z}/m=a-bt,\ t\in\mathbb{Z}\}$, onde $t$ varia sobre todos os inteiros. Temos $m\neq\emptyset$. Al{\'e}m disso, $M^{+}$ {\'e} limitado inferiormente, logo, pelo princ{\'\i}pio da boa ordena{\c c}{\~a}o, existe $r\in M^{+}/r\leq x,\ \forall x\in M^{+}$. Como $r\in m^{+}\subseteq M$, existe $q\in\mathbb{Z}$ tal que $r=a+bq$. Portanto, $a=bq+r,\ q\in\mathbb{Z}$, com $r\leq 0$.

Falta provar que $r<b$.

Suponha ent{\~a}o que $r\geq b$. Logo $r=a-bq\geq b$, ou seja,\[a-bq-b\geq 0\Leftrightarrow a-b(q+1)\geq 0\]

Desse modo, $a-b(q+1)\in M^{+}$.

Agora, \[b>0\Rightarrow bq+b>bq\Rightarrow b(q+1)>bq\] \[-b(q+1)<-bq\Rightarrow a-b(a+1)<a-bq=r\], o que {\'e} uma contradi{\c c}{\~a}o, pois $r$ {\'e} o m{\'\i}nimo de $M^{+}$, logo, $r<b$, ou seja, $a=bq+r,\ q,r\in\mathbb{Z},\ 0\leq r<b$\\

Falta provar a unidade de $q$ e $r$. Assim, suponha que existam \[q_{1},q_{2},r_{1},r_{2}\in\mathbb{Z},\ 0\leq r_{1}<b,\ 0\leq r_{2}<b\], tais que:
\[a=bq_{1}+r_{1}=bq_{2}+r_{2}\]

Suponha $r_{1}\neq r_{2}$. Suponha tamb{\'e}m que $r_{1}>r_{2}$. Assim, \[0\leq r_{1}-r_{2}=b(q_{2}-q_{1})\]

E da{\'\i}, $q_{2}-q_{1}\geq 0$.

Desse modo \[r_{1}=b(q_{2}-q_{1})+r_{2}\]

Mas $r_{1}\geq 0,\ q_{2}-q_{1}\geq 1$, da{\'\i} $r_{1}>b$, o que {\'e} uma contradi{\c c}{\~a}o, logo $r_{1}=r_{2}$ e ent{\~a}o $q_{1}=q_{2}$, o que prova a unicidade.\#

\section{M{\'a}ximo Divisor Comum}

\begin{definicao}[M{\'a}ximo Divisor Comum] Dado $a,b\in\mathbb{Z}$, dizemos que $d\in\mathbb{Z}$ {\'e} o m{\'a}ximo divisor comum entre $a$ e $b$ se
\begin{enumerate}
\item $d\geq 0$
\item $d|a$ e $d|b$
\item Se $d'$ {\'e} um inteiro tal que $d'|a$ e $d'|b$, ent{\~a}o $d'|d$ %REVISAR
\end{enumerate}
\end{definicao}

Observa{\c c}{\~o}es:
\begin{enumerate}
\item Se $d$ e $d_{1}$ s{\~a}o m{\'a}ximos divisores comuns entre $a$ e $b$, ent{\~a}o $d=d_{1}$.\\

De fato, dados $d$ e $d_{1}$ m{\'a}ximos divisores comuns de $a$ e $b$, ent{\~a}o temos que $d|a,d|b,d_{1}|a,d_{1}|b$. Mas pelo item 3 da defini{\c c}{\~a}o temos $d|d_{1}$ e $d_{1}|d$. Agora, como $d_{1}\geq 0$ e $d\geq 0$, segue que $d=d_{1}$
\item Se $a=b=0$, segue que $d=d_{1}$ %REVISAR
\item Se $a=0$ e $b\neq 0$, ent{\~a}o $d=|b|$
\item Se $d$ {\'e} o m{\'a}ximo divisor comum entre $a$ e $b$, ent{\~a}o $d$ tamb{\'e}m {\'e} o m{\'a}ximo divisor comum entre $a$ e $-b$, $-a$ e $b$ e entre $-a$ e $-b$.
\end{enumerate}

\begin{nota}[M{\'a}ximo Divisor Comum] Indicaremos por $mdc(a,b)$ o m{\'a}ximo divisor comum ente $a$ e $b$, que j{\'a} sabemos que {\'e} {\'u}nico quando existe.\end{nota}

\begin{proposicao} Quaisquer que sejam $a,b\in\mathbb{Z}$, existe $d\in\mathbb{Z}$ que {\'e} o m{\'a}ximo divisor comum entre $a$ e $b$.\end{proposicao}

\textbf{Demonstra{\c c}{\~a}o}: Das observa{\c c}{\~o}es anteriores podemos considerar somente o caso em que $a>0$ e $b>0$.

Seja $L=\{ax+by/x,y\in\mathbb{Z}\}$. Temos que $L\neq\emptyset$ pois tomando $x=1$ e $y=0$, temos que $m=a1+b0$, pelo princ{\'\i}pio da boa ordena{\c c}{\~a}o, existe $d\in L^{+}$ tal que $d\leq x$, para todo $x\in L^{+}$.

Mostremos que $d=mdc(a,b)$
\begin{enumerate}
\item $d\geq 0$ pois $d\in L^{+}$
\item Como $d\in L^{+}$, existem $x_{0},y_{0}\in\mathbb{Z}$ tais que $d=ax_{0}+by_{0}$.

Agora usando o algoritmo da divis{\~a}o de Euclides para $a$ e $d$ temos que existem $k,r\in\mathbb{Z},0\leq r<d$ tais que $a=kd+r$.

Assim:\[a=k(ax_{0}+by_{0})+r\] \[r=a(1-kx_{0})+b(-y_{0})k\] Da{\'\i}, $r\in L$, mas $r\geq 0$, ent{\~a}o $r\in L^{+}$. Como $d$ {\'e} o m{\'\i}nimo de $L^{+}$ devemos ter $r=0$ e assim $a=kd$, ou seja, $d|a$.

Analogamente, \textit{Mutatis Mutandis}, mostra-se que $d|b$.
\item Seja $d\in\mathbb{Z}$ tal que $d'|a$ e $d'|b$. Temos que $d'|(ax+by)$, para $x,y\in\mathbb{Z}$, em particular, $d'|(ax_{0}+by_{0})=d$, ou seja, $d'|d$.
\end{enumerate}

Portanto, $d=mdc(a,b)$.\#

Observa{\c c}{\~a}o:
\begin{enumerate}
\item Se $d=mdc(a,b)$, ent{\~a}o $d=ax_{0}+by_{0}$, onde $x_{0},y_{0}\in\mathbb{Z}$. Os elementos $x_{0}$ e $y_{0}$ satisfazem que tal igualdade n{\~a}o {\'e} {\'u}nica.
\item Uma igualdade do tipo $d=ax_{0}+by_{0}$ {\'e} chamada de Identidade de Bezout

Exemplos:
\begin{enumerate}
\item $mdc(2,3)=1$\\
$1=2(-1)+3.1=2.2+3(-1)$
\item $mdc(4,8)=4$\\
\end{enumerate}
\end{enumerate}

Considere os seguintes subconjuntos de $\mathbb{Z}$\\
\[I=\{2k/k\in\mathbb{Z}\}=\{0,\pm 2,\pm 4,\pm 6,...\}\]\[J=\{2r+1/r\in\mathbb{Z}\}=\{\pm 1,\pm 3,\pm 5,...\}\]

Dados quaisquer $a,b\in I$, temos $a+b\in I$. Al{\'e}m disso, dado $n\in\mathbb{Z},\ na\in I$. Por outro lado, $1,3\in J$ mas $1+3=4\notin J$.

\section{Ideais}

\subsubsection{Defini{\c c}{\~a}o}
\begin{definicao}[Ideal] Um subconjunto n{\~a}o vazio $S\subseteq\mathbb{Z}$ {\'e} chamado de um ideal de $\mathbb{Z}$ se valem as seguintes condi{\c c}{\~o}es:
\begin{enumerate}
\item $r_{1}+r_{2}\in S,\forall r_{1},r_{2}\in S$
\item $nr\in S,\forall n\in\mathbb{Z},\forall r\in S$
\end{enumerate}
\end{definicao}

\subsubsection{Propriedades}
Seja S ideal de $\mathbb{Z}$
\begin{enumerate}
\item $r_{1}-r_{2}\in S$, para todos $r_{1}, r_{2}\in S$, pois $r_{1}-r_{2}=r_{1}+(-r_{2})$
\item $0\in S$, pois $0=r-r$, para qualquer $r\in S$
\end{enumerate}

Exemplos:
\begin{enumerate}
\item $S=\{2k/k\in\mathbb{Z}\}$ {\'e} um ideal de $\mathbb{Z}$
\item $S=\{0\}$ e $S=\mathbb{Z}$ s{\~a}o ideais de $\mathbb{Z}$, chamados de ideais triviais
\item Dado $a,b,c\in\mathbb{Z}$, o subconjunto $S=\{ax+by/x,y\in\mathbb{Z}\}$ {\'e} um ideal de $\mathbb{Z}$

$S\neq\emptyset$ pois $0=a0+b0\in S$

Sejam $ax_{1}+by_{1},ax_{2}+by_{2}\in S$. Temos $(ax_{1}+by_{1})+(ax_{2}+by_{2})=a(x_{1}+x_{2})+b(y_{1}+y_{2})\in S$

Agora, sejam $ax_{1}+by{1}\in S$ e $n\in\mathbb{Z}$ temos \[n(ax_{1}+by_{1})=a(nx_{1})+b(ny_{1})\in S\]
\end{enumerate}

De modo geral, dados $a_{1},a_{2},...,a_{n}$ n{\'u}meros inteiros, o subconjunto \[S=\{a_{1}x_{1}+a_{2}x_{2}+...+a_{n}x_{n}/x_{1},...,x_{n}\in\mathbb{Z}\}\] {\'e} um ideal de $\mathbb{Z}$.

Se $S$ {\'e} ideal de $\mathbb{Z}$, ent{\~a}o $S=\{nk/n\in\mathbb{Z}\}$.
\subsubsection{Conjunto dos m{\'u}ltiplos de $g$}
\begin{nota}[Conjunto dos m{\'u}ltiplos de $g$] Se $g\in\mathbb{Z}$, denotamos por $g\mathbb{Z}$, ou $\mathbb{Z}g$, o subconjunto dos inteiros que s{\~a}o m{\'u}ltiplos de $g$ (os inteiros que s{\~a}o divis{\'\i}veis por $g$). Em outras palavras \[g\mathbb{Z}=\{gn/n\in\mathbb{Z}\}=\{0,\pm g,\pm 2g,\pm 3g,...\}\]
\end{nota}

\begin{teo} Seja $S$ um ideal de $\mathbb{Z}$. Ent{\~a}o, existe um n{\'u}mero $g\in\mathbb{Z}$ tal que $S=g\mathbb{Z}$.\end{teo}

\textbf{Demonstra{\c c}{\~a}o}: Se $S=\{0\}$, ent{\~a}o tomamos $g=0$ e da{\'\i} $S=0\mathbb{Z}$. Se $S=\mathbb{Z}$, ent{\~a}o $g=1$ e $S=1\mathbb{Z}$.

Assim podemos supor $S\neq\{0\}$ e $S\neq\mathbb{Z}$. Seja $S^{+}=\{x\in S/x>0\}$. Do {\'\i}tem 2 da defini{\c c}{\~a}o de ideal, segue que $S^{+}\neq\emptyset$. Assim, pelo princ{\'\i}pio da boa ordena{\c c}{\~a}o, existe $g\in S^{+}$ tal que $g\leq x,\forall x\in S^{+}$.

Como $g\in S^{+}\subseteq S$ e $S$ {\'e} um ideal de $\mathbb{Z}$, ent{\~a}o $gn\in S\forall n\in\mathbb{Z}$, ou seja, $g\mathbb{Z}\subseteq S$.

Agora precisamos mostrar que $a=gq$, onde $q\in\mathbb{Z}$. Assim, dado $a\in S$, o algoritmo da divis{\~a}o de Euclides garante que existem $q,r\in\mathbb{Z}$ tais que $a=gq+r$, onde $0\leq r<g$. Como $a,q,g\in S$ e $S$ {\'e} um ideal, ent{\~a}o $r=a-gq\in S$. Se $r>0$, ent{\~a}o como $r<g$ e $g$ {\'e} o m{\'\i}nimo de $S^{+}$ obtemos uma contradi{\c c}{\~a}o. Logo, $r=0$ e $a=gp$. Da{\'\i} $S\subseteq g\mathbb{Z}$. Portanto $s=g\mathbb{Z}$.\#

Exemplo: O conjunto $S=\{2x-5y/x,y\in\mathbb{Z}\}$ {\'e} ideal de $\mathbb{Z}$. Neste caso,\\ $S^{+}=\{1,2,3,...\}$. Assim, $g=1$ e $S=1\mathbb{Z}=\mathbb{Z}$.

\chapter{Rela{\c c}{\~o}es e Fun{\c c}{\~o}es}
\section{Rela{\c c}{\~o}es}
\subsubsection{Defini{\c c}{\~a}o}
Sejam A e B dois conjuntos n{\~a}o vazios. Os subconjuntos de AxB s{\~a}o chamados rela{\c c}{\~o}es, ou seja, uma rela{\c c}{\~a}o em AxB {\'e} um subconjunto desse produto cartesiano.

Quando R {\'e} uma rela{\c c}{\~a}o em AxB, tamb{\'e}m dizemos que R {\'e} uma rela{\c c}{\~a}o de A em B.

Exemplos:
\begin{enumerate}
\item Se A=\{0,1\} e B=\{-1,0,1\}, ent{\~a}o AxB=\{(0,-1),(0,0),(0,1),(1,-1),(1,0),(1,1,)\}\\
S{\~a}o exemplos de rela{\c c}{\~o}es:\\
$R_{1}=\{(0,1)\}$\\
$R_{2}=\emptyset$\\
$R_{3}=\{(1,-1),(1,1)\}$\\
$R_{4}=A$x$B$
\item Se $A=B=\mathbb{R}$, ent{\~a}o AxB {\'e} o conjunto formado por todos pares ordenados de n{\'u}meros reais. Um exemplo de rela{\c c}{\~a}o em $\mathbb{R}$x$\mathbb{R}$ {\'e} o conjunto:\\
$R=\{(x,y)\in \mathbb{R}$x$\mathbb{R}/ y\geq 0\}$
\end{enumerate}

\section{Rela{\c c}{\~o}es de equival{\^e}ncia}
\subsubsection{Defini{\c c}{\~a}o}
\begin{definicao}[Rela{\c c}{\~a}o de equival{\^e}ncia] Seja X um conjunto n{\~a}o vazio e $R\subseteq X \times X$ uma rela{\c c}{\~a}o. Dizemos que R {\'e} uma rela{\c c}{\~a}o de equival{\^e}ncia se:
\begin{enumerate}
\item $\forall a \in X,(a,a)\in R$ (Reflexidade)
\item $(a,b)\in R \rightarrow (b,a)\in R$ (Simetria)
\item $(a,b) \in R \wedge (b,c) \in R \rightarrow (a,c)\in R$ (Transitividade)
\end{enumerate}
\end{definicao}

Quando $R\subseteq X$x$X$ {\'e} uma rela{\c c}{\~a}o de equival{\^e}ncia, dizemos que R {\'e} uma rela{\c c}{\~a}o de equival{\^e}ncia em X. Quando 2 elementos $a,b\in X$ s{\~a}o tais que $(a,b)\in R$, dizemos que a e b s{\~a}o relacionados.\\

\subsection{Equival{\^e}ncia m{\'o}dulo R}

\begin{nota}[Equival{\^e}ncia m{\'o}dulo R] Seja R uma rela{\c c}{\~a}o de equival{\^e}ncia em X. Para exprimirmos que $(a,b)\in R$ usaremos a nota{\c c}{\~a}o $a\equiv b(R)$, que se l{\^e} ``a equivalente a B m{\'o}dulo R", ou ainda a nota{\c c}{\~a}o aRb, com o mesmo significado anterior.\end{nota}

Exemplos:
\begin{enumerate}
\item Seja X=\{1,2,3\}. Temos XxX=\{(1,1),(1,2),(1,3),(2,1),(2,2),(2,3),(3,1),(3,2),(3,3)\}\\
S{\~a}o exemplos de rela{\c c}{\~o}es de equival{\^e}ncia:\\
$R_{1}=X$x$X$\\
$R_{2}=\{(1,1),(2,2),(3,3)\}$\\
$R_{3}=\{(1,1),(2,2),(3,3),(1,2),(2,1)\}$
\item Seja $X=\mathbb{Z}$ e $R\in \mathbb{Z}$x$\mathbb{Z}$ definida por $R=\{(x,y)\in \mathbb{Z}$x$\mathbb{Z} /x=y\}$\\
R {\'e} uma rela{\c c}{\~a}o de equival{\^e}ncia pois:
\begin{itemize}
\item $\forall a \in \mathbb{Z}, (a,a) \in R$ pois a=a
\item $(a,b)\in R \rightarrow a=b \wedge b=a \Leftrightarrow (b,a)\in R$
\item $(a,b),(b,c)\in R \rightarrow a=b=c\Rightarrow (a,c)\in R$
\end{itemize}
\item Tome $R=\{(x,y)\in \mathbb{Z}$x$\mathbb{Z}/ 2|(x-y)\}=\{(x,y)\in\mathbb{Z}$x$\mathbb{Z}/ x-y=2k, k\in\mathbb{Z}\}$\\
R {\'e} uma rela{\c c}{\~a}o de equival{\^e}ncia pois:
\begin{itemize}
\item $\forall x\in\mathbb{Z},xRx$ pois $x-x=2.0$
\item $xRy\rightarrow x-y=2k\Rightarrow y-x=-(x-y)=2.(-k)\Rightarrow yRx$
\item $xRy\wedge yRz\rightarrow x-y=2k\wedge y-z=2q\Rightarrow x+z=x-y+y-z=2k+2q=2(k+q)\rightarrow xRz$

\end{itemize}
\end{enumerate}

\subsection{Classe de equival{\^e}ncia e conjunto quociente}
\begin{definicao}[Classe de Equival{\^e}ncia] Seja R uma rela{\c c}{\~a}o de equival{\^e}ncia sobre um conjunto X. Dado $a\in X$, chamamos classe de equival{\^e}ncia determinada por a m{\'o}dulo R, denotada por $\bar{a}$ ou C(a), o subconjunto constitu{\'\i}do pelos elementos $b\in X$ tais que bRa, ou seja, $\bar{a}=C(a)=\{a\in X/ bRa\}$\end{definicao}

\begin{definicao}[Conjunto quociente] O subconjunto das classes de equival{\^e}ncia m{\'o}dulo R ser{\'a} denotado por x/R e {\'e} chamado conjunto quociente de X por R.\end{definicao}

Observa{\c c}{\~a}o: Dado um conjunto $X\neq\emptyset$ e R uma rela{\c c}{\~a}o de equival{\^e}ncia em X, dado $a\in X$ como R {\'e} uma rela{\c c}{\~a}o de equival{\^e}ncia, aRa, da{\'\i} $\bar{a}\neq\emptyset$, pois $a\in\bar{a}$\\

Exemplos:
\begin{enumerate}
\item Seja X=\{a,b,c\} e R=\{(a,a),(b,b),(c,c),(a,c),(c,a)\}. Temos:\\
$\bar{a}=\{x\in X/xRa\}=\{a,c\}$\\
$\bar{b}=\{x\in X/xRb\}=\{b\}$\\
$\bar{c}=\{x\in X/xRc\}=\{a,c\}$
\item Seja X=\{1,2,3,4\} e a rela{\c c}{\~a}o de equival{\^e}ncia R=\{(1,1),(2,2),(3,3),(4,4)\}\\
$\bar{1}=\{x\in X/xR1\}=\{1\}$\\
$\bar{2}=\{x\in X/xR2\}=\{2\}$\\
$\bar{3}=\{x\in X/xR3\}=\{3\}$\\
$\bar{4}=\{x\in X/xR4\}=\{4\}$
\end{enumerate}

\begin{proposicao} Seja R uma rela{\c c}{\~a}o de equival{\^e}ncia em um conjunto n{\~a}o vazio X, sejam a,b$\in$X. Se $\bar{a}\cap\bar{b}\neq\emptyset$, ent{\~a}o aRb.\end{proposicao}

\textbf{Demonstra{\c c}{\~a}o}: Como  $\bar{a}\cap\bar{b}\neq\emptyset$, existe um $y\in\bar{a}\cap\bar{b}$, logo $y\in\bar{a}\wedge y\in\bar{b}$. Da defini{\c c}{\~a}o de classe de equival{\^e}ncia temos que yRa e yRb. Como R {\'e} rela{\c c}{\~a}o de equival{\^e}ncia temos que aRy e bRy. Por transitividade, aRb, como quer{\'\i}amos demonstrar.\#

\begin{proposicao} Se  $\bar{a}\cap\bar{b}\neq\emptyset$, ent{\~a}o $\bar{a}=\bar{b}$\end{proposicao}

\textbf{Demonstra{\c c}{\~a}o}: Seja $y\in \bar{a}$. Da{\'\i} yRa. Como $\bar{a}\cap\bar{b}\neq\emptyset$, pela proposi{\c c}{\~a}o anterior, aRb. Logo, como yRa e aRb, segue que yRb, ou seja, $y\in\bar{b}$. Da{\'\i} $\bar{a}\subseteq\bar{b}$. Como no caso anterior, mostra-se que $\bar{b}\subseteq\bar{a}$. Portanto $\bar{a}=\bar{b}$.\#

\begin{cor} As classes de equival{\^e}ncia s{\~a}o conjuntos disjuntos ou iguais.\end{cor}

Seja R uma rela{\c c}{\~a}o de equival{\^e}ncia em $X\neq\emptyset$, dado $a\in R$. Se bRa, ent{\~a}o $\bar{b}=\bar{a}$, mais ainda, se dRa ent{\~a}o $\bar{d}=\bar{a}=\bar{b}$. Como por exemplo:\\
X=\{a,b,c,d,e,f,g\}\\
$\bar{a}=\{a,b,c\}$\\
$\bar{e}=\{e\}$\\
$\bar{f}=\{f,g\}$

\begin{definicao}[Representante da Classe de Equival{\^e}ncia] Seja C uma classe de equival{\^e}ncia de uma rela{\c c}{\~a}o de equival{\^e}ncia R. Qualquer elemento $y\in C$ {\'e} chamado representante de C.\end{definicao}

\begin{proposicao} Seja X um conjunto n{\~a}o vazio e R uma rela{\c c}{\~a}o de equival{\^e}ncia em X. Ent{\~a}o X {\'e} a uni{\~a}o disjunta das classes $\bar{a}, a\in X$, ou seja, \[X=\displaystyle\bigsqcup_{a\in X}\bar{a}\].\end{proposicao}

\textbf{Demonstra{\c c}{\~a}o}: Para todo $a\in X, \bar{a}\subseteq X$, logo $\displaystyle\bigsqcup_{a\in X}\bar{a}\subseteq X$. Seja $b\in X$. Logo $b\in\bar{b}$, da{\'\i} $b\in \displaystyle\bigsqcup_{a\in X}\bar{a}$, logo $X\subseteq\displaystyle\bigsqcup_{a\in X}\bar{a}$. Portanto, $X=\displaystyle\bigsqcup_{a\in X}\bar{a}$.\#

Exemplo:\\
Em $\mathbb{Z}$x$\mathbb{Z}$ considere a seguinte rela{\c c}{\~a}o: $R=\{(a,b)\in \mathbb{Z}$x$\mathbb{Z}/2|(a-b)\}$. Mostre que {\'e} uma rela{\c c}{\~a}o de equival{\^e}ncia e mostre suas classes de equival{\^e}ncia.
\begin{enumerate}
\item Dado $a\in \mathbb{Z}$, aRa pois $2|(a-a)=0$.
\item Se aRb, ent{\~a}o $2|(a-b)$, ou seja, a-b=2k, -(a-b)=b-a=2(-k). Logo bRa.
\item Se aRb e bRc, ent{\~a}o a-b=2k e b-c=2q. Logo a-b+b-c=2k+2q=2(k+q). Logo, aRc.

\end{enumerate}

Portanto R {\'e} uma rela{\c c}{\~a}o de equival{\^e}ncia.

Dado $a\in\mathbb{Z}$ , temos:\\
$\bar{a}=\{b\in\mathbb{Z}/bRa\}=\{b\in\mathbb{Z}/2|(a-b)\}$ como $2|(a-b)$, temos que:\\
$a-b=2k\Leftrightarrow b=a+2r, r=-k$

Assim, se a {\'e} {\'\i}mpar, b tamb{\'e}m o {\'e}. Logo:\\
$\bar{a}=\{...,-3,-1,1,3,...\}$

Agora, se a {\'e} par, b tamb{\'e}m {\'e}. Logo:\\
$\bar{a}=\{...,-2,0,2,4,...\}$

\section{Fun{\c c}{\~o}es}

\subsubsection{Defini{\c c}{\~a}o}
\begin{definicao}[Fun{\c c}{\~a}o] Uma fun{\c c}{\~a}o $f$ de um conjunto A em um conjunto B {\'e} uma rela{\c c}{\~a}o $f\subseteq A\times B$ satisfazendo:
\begin{enumerate}
\item $\forall x\in A,\exists y\in B/(x,y)\in f$
\item $(x_{1},y_{1}),(x_{1},y_{2})\in f \rightarrow y_{1}=y_{2}$
\end{enumerate}
\end{definicao}

Geralmente, para dizer que $f$ {\'e} uma fun{\c c}{\~a}o de A em B escrevemos $f:A\rightarrow B$.

\subsubsection{Dom{\'\i}nio e contra-dom{\'\i}nio}
O conjunto A {\'e} chamado de Dom{\'\i}nio de $f$ e o conjunto B {\'e} chamado de contra-dom{\'\i}nio.

Se $f:A\rightarrow B$ {\'e} uma fun{\c c}{\~a}o, escrevemos $f(a)=b$ para dizer que $(a,b)\in f$

Exemplos:
\begin{enumerate}
\item Sejam A=\{0,1,2,3\} e B=\{4,5,6,7,8\}. Quais das seguintes rela{\c c}{\~o}es s{\~a}o fun{\c c}{\~o}es?
\begin{itemize}
\item $R_{1}=\{(0,5),(1,6),(2,7)\}$ - N{\~a}o {\'e} fun{\c c}{\~a}o pois o n{\'u}mero 3 n{\~a}o t{\^e}m valor associado {\`a} ele.
\item $R_{2}=\{(0,4),(1,5),(1,6),(2,7),(3,8)\}$ - N{\~a}o {\'e} fun{\c c}{\~a}o pois o valor 1 tem mais de um valor diferente associado {\`a} ele.
\item $R_{3}=\{(0,4),(1,5),(2,7),(3,8)\}$ - {\'E} fun{\c c}{\~a}o
\item $R_{4}=\{(0,5),(1,5),(2,6),(3,7)\}$ - {\'E} fun{\c c}{\~a}o

\end{itemize}
\item $R_{5}=\{(x,y)\in\mathbb{R}$x$\mathbb{R}/y^{2}=x^{2}\}$ - N{\~a}o {\'e} fun{\c c}{\~a}o, pois $x=\pm \sqrt{y}$
\item $R_{6}=\{(x,y)\in\mathbb{R}$x$\mathbb{R}/x^{2}+y^{2}=1\}$ - N{\~a}o {\'e} fun{\c c}{\~a}o pois quando\\ $x=0,y=1\wedge y=-1$
\item  $R_{7}=\{(x,y)\in\mathbb{R}$x$\mathbb{R}/y=x^{2}\}$ - {\'E} fun{\c c}{\~a}o
\end{enumerate}

\subsection{Tipos de fun{\c c}{\~o}es}

\begin{definicao}[Fun{\c c}{\~a}o sobrejetora]  Uma fun{\c c}{\~a}o $f:A\rightarrow B$ {\'e} sobrejetora se, e somente se, para todo $y\in B$ exista um $x\in A$ tal que $f(x)=y$\end{definicao}

\begin{definicao}[Fun{\c c}{\~a}o injetora] Uma fun{\c c}{\~a}o $f:A\rightarrow B$ {\'e} injetora se, e somente se, para $a_{1}\neq a_{2}$, temos $f(a_{1})\neq f(a_{2}), \forall a_{1},a_{2}\in A$\end{definicao}

\begin{definicao}[Fun{\c c}{\~a}o bijetora] Uma fun{\c c}{\~a}o $f:A\rightarrow B$ que {\'e} simultaneamente injetora e sobrejetora {\'e} chamada de bijetora ou bijetiva.
\end{definicao}

Exemplos:
\begin{enumerate}
\item A fun{\c c}{\~a}o $f:\mathbb{R}\rightarrow\mathbb{R}$ dada por $f(x)=3x+1$ {\'e} injetora e sobrejetora.

Dados $x_{1}, x_{2}\in\mathbb{R}$ tais que $f(x_{1})=f(x_{2})$, temos:
\[3x_{1}+1=3x_{2}+1\]
\[x_{1}=x_{2}\]

Logo $f$ {\'e} injetora

Para verificar se $f$ {\'e} sobrejetora precisamos verificar se dado $y\in\mathbb{R}\\ \ \exists x\in\mathbb{R}/f(x)=y$.

Tome $x=\displaystyle\frac{y-1}{3}\in\mathbb{R}$. Da{\'\i}, $f(x)=y$. Logo $f$ {\'e} sobrejetora.
\item A fun{\c c}{\~a}o $f:\mathbb{R}\rightarrow\mathbb{R}$ dada por $f(x)=x^{2}$ {\'e} injetora? E sobrejetora?

N{\~a}o {\'e} injetora pois $f(-1)=f(1)\wedge 1\neq -1$

N{\~a}o {\'e} sobrejetora pois $\nexists x\in\mathbb{R}/x^{2}=-1$

\end{enumerate}

Dado $f:A\rightarrow B$ uma fun{\c c}{\~a}o, considere a rela{\c c}{\~a}o $f^{-1}\subseteq B$x$A$ tal que $(b,a)\in f^{-1}$ se $(a,b)\in f$, ou seja, $f^{-1}(b)=a$ se $f(a)=b$.

Pode ocorrer que $f^{-1}$ n{\~a}o seja fun{\c c}{\~a}o, mesmo $f$ sendo uma fun{\c c}{\~a}o. Por exemplo:

$f:\{0,1,2,3\}\rightarrow\{4,5,6,7,8\}$ dada por:\\
$f(0)=5$\\
$f(1)=5$\\
$f(2)=6$\\
$f(3)=7$

Neste caso, $f^{-1}$ {\'e} dado por:\\
$f^{-1}(5)=0$\\
$f^{-1}(5)=1$\\
$f^{-1}(6)=1$\\
$f^{-1}(7)=3$

\begin{teo} Dada $f:A\rightarrow B$ fun{\c c}{\~a}o tome $f^{-1}:B\rightarrow A$. Definida com o $f^{-1}(b)=a$ se $f(a)=b$. Ent{\~a}o $f^{-1}$ {\'e} uma fun{\c c}{\~a}o se, e somente se, $f$ {\'e} bijetora.\end{teo}

\textbf{Demonstra{\c c}{\~a}o}: Suponha $f^{-1}$ {\'e} fun{\c c}{\~a}o. Precisamos provar que $f$ {\'e} injetora e sobrejetora.

Dados $a_{1},a_{2}\in A$ tais que $f(a_{1})=b=f(a_{2})$. Como $f(a_{1})=b$ temos $f^{-1}(b)=a_{1}$, al{\'e}m disso, $f^{-1}(b)=a_{2}$. Mas $f^{-1}$ {\'e} fun{\c c}{\~a}o, da{\'\i} $a_{1}=a_{2}$, ou seja, $f$ {\'e} injetora.

Dado $b\in B$, como $f^{-1}$ {\'e} uma fun{\c c}{\~a}o, $\forall b\in B, f^{-1}(b)=a\in A$, logo $f(a)=b$ e assim $f$ {\'e} sobrejetora.

Portanto $f$ {\'e} bijetora.

Agora suponha que $f$ {\'e} bijetora.

Primeiramente, dado $b\in B$, como $f$ {\'e} sobrejetora, existe $a\in A$ tal que $f(a)=b$, ou seja, $f^{-1}(b)=a\in A$.

Suponha que $f^{-1}(b)=a_{1}$ e $f^{-1}(b)=a_{2}$. Da{\'\i}, $f(a_{1})=b\wedge f(a_{2})=b$. Mas $f$ {\'e} injetora, assim $a_{1}=a_{2}$ e ent{\~a}o $f^{-1}(b)=a_{1}=a_{2}$.

Portanto $f^{-1}$ {\'e} fun{\c c}{\~a}o. \#
\subsection{Composi{\c c}{\~a}o de fun{\c c}{\~o}es}

\subsubsection{Defini{\c c}{\~a}o}

\begin{definicao}[Fun{\c c}{\~a}o Composta] Sejam $f:A\rightarrow B$ e $g:B\rightarrow C$ fun{\c c}{\~o}es. Chama-se composta de $g$ e $f$ a fun{\c c}{\~a}o de A em C, denotada $g\circ f$, definida por $g\circ f:A\rightarrow C$.\end{definicao}

Temos ent{\~a}o que $(g\circ f)(x)=g(f(x)), \forall x\in A$.

Observa{\c c}{\~a}o: Se $f:A\rightarrow B$ e $g:B\rightarrow A$ ent{\~a}o existem $f\circ g$ e $g\circ f$. Por{\'e}m, em geral, $f\circ g\neq g\circ f$.

\subsubsection{Propriedades}
\begin{proposicao} Se $f:A\rightarrow B$ e $g:B\rightarrow C$ s{\~a}o fun{\c c}{\~o}es injetoras, ent{\~a}o $g\circ f$ {\'e} injetora.\end{proposicao}

\textbf{Demonstra{\c c}{\~a}o}: Dados $x_{1},x_{2}\in A$ tais que $(g\circ f)(x_{1})=(g\circ f)(x_{2})$ temos que $g(f(x_{1}))=g(f(x_{2}))$. Como $g$ {\'e} injetora, $f(x_{1})=f(x_{2})$. Mas $f$ {\'e} injetora, da{\'\i} $x_{1}=x_{2}$. Logo $g\circ f$ {\'e} injetora.\#

\begin{proposicao} Se $f:A\rightarrow B$ e $g:B\rightarrow C$ s{\~a}o sobrejetoras, ent{\~a}o $g\circ f$ {\'e} sobrejetora.\end{proposicao}

\textbf{Demonstra{\c c}{\~a}o}: Temos que $g\circ f:A\rightarrow C$. Dado $z\in C$. Como $g$ {\'e} sobrejetora, $\exists y\in B/g(y)=z$. Como $f$ {\'e} sobrejetora, $\exists x\in A/f(x)=y$. Assim, $z=g(y)=g(f(x))=(g\circ f)(x)$. Logo $g\circ f$ {\'e} sobrejetora.\#

\subsection{Fun{\c c}{\~a}o Identidade}
\subsubsection{Defini{\c c}{\~a}o}
\begin{definicao}[Fun{\c c}{\~a}o Identidade] Dado um conjunto $A\neq\emptyset$, a fun{\c c}{\~a}o $i_{A}:A\rightarrow A$ dada por $i_{A}(x)=(x)$ {\'e} chamada de fun{\c c}{\~a}o identidade.\end{definicao}

\begin{proposicao} Se $f:E\rightarrow F$ {\'e} bijetora, ent{\~a}o $f\circ f^{-1}=i_{F}\wedge f^{-1}\circ f=i_{E}$\end{proposicao}

\textbf{Demonstra{\c c}{\~a}o}: Temos $i_{F}:F\rightarrow F$ e $i_{E}:E\rightarrow E$. Al{\'e}m disso, $f\circ f^{-1}:F\rightarrow F$ e $f^{-1}\circ f:E\rightarrow E$, da{\'\i} $D(f\circ f^{-1})=D(i_{F})$\footnote{$D(f(x))$ {\'e} o dom{\'\i}nio da fun{\c c}{\~a}o $f$} e $D(f^{-1}\circ f)=D(i_{E})$. Dado $x\in F, (f\circ f^{-1})(x)=f(f^{-1}(x))=x=i_{F}(x)$. Dado $x\in E, (f^{-1}\circ f)(x)=f^{-1}(f(x))=x=i_{E}(x)$.\#

\subsubsection{Propriedades}
\begin{proposicao} Se $f:E\rightarrow F$ e $g:F\rightarrow E$ s{\~a}o fun{\c c}{\~o}es, ent{\~a}o:
\begin{enumerate}
\item $f\circ i_{e}=f, i_{F}\circ f=f, g\circ i_{F}=g, i_{E}\circ g=g$
\item Se $g\circ f=i_{E}$, e $f\circ g=i_{F}$, ent{\~a}o $f$ e $g$ s{\~a}o bijetoras e $g=f^{-1}$
\end{enumerate}
\end{proposicao}

\textbf{Demonstra{\c c}{\~a}o}:
\begin{enumerate}
\item Provemos que $f\circ i_{E}=f$.\\
Primeiro temos $f:E\rightarrow F$ e $i_{E}:E\rightarrow E$. Da{\'\i}, $f\circ i_{E}:E\rightarrow F$, ou seja, $D(f\circ i_{E})=D(f)$. Dado $x\in E$, temos $(f\circ i_{E})(x)=f(i_{E}(x))=f(x)$. Portanto, $f\circ i_{E}=f$
\item Provemos q $f$ {\'e} bijetora.\\
Dados $x_{1},x_{2}\in F/f(x_{1})=f(x_{2})$. Como $f:E\rightarrow F$ e $g:F\rightarrow E$, ent{\~a}o $g(f(x_{1}))=g(f(x_{2}))$, ou seja, $(g\circ f)(x_{1})=(g\circ f)(x_{2})$. Da{\'\i}, $i_{E}(x_{1})=i_{E}(x_{2})$. Logo, $x_{1}=x_{2}$, isto {\'e}, $f$ {\'e} injetora.

Agora, dado $y\in F$, segue que $y=i_{F}(y)$. Mas $i_{F}=f\circ g$. Da{\'\i}, $y=i_{F}(y)=(f\circ g)(y)=f(g(y))$. Assim, $x=g(y)\in E$ e $f(x)=y$.

Logo $f$ {\'e} sobrejetora. Portanto $f$ {\'e} bijetora. Analogamente, prova-se que g {\'e} bijetora. Provemos que $g=f^{-1}$. Temos  $f^{-1}:F\rightarrow E$, da{\'\i}, $D(y)=F=D(f^{-1})$. Agora, $f\circ g=i_{F}=f\circ f^{-1}$. Assim, para todo $x\in F, (f\circ g)(x)=(f\circ f^{-1})(x)$. Isto {\'e}, $f(g(x))=f(f^{-1}(x))$. Portanto, $g(x)=f^{-1}(x)\forall x\in F$. Logo, $g=f^{-1}$. \#
\end{enumerate}

\begin{definicao}
Seja $f : E \to F$ uma fun{\c c}{\~a}o.
\begin{enumerate}
\item Dado $A \sub E$, chama-se {\rm imagem direta} de $A$, segundo $f$ e indica-se por $f(A)$ o subconjunto de $F$ dado por
\[
f(A) = \{f(x) \mid x \in A\},
\]
isto {\'e}, $f(A)$ {\'e} o conjunto das imagens por $f$ dos elementos de $A$.

\item Dado $B \sub F$, chama-se {\rm imagem inversa} de $B$, segundo $f$ e indica-se por $f^{-1}(B)$ o subconjunto de $E$ dado por
\[
f^{-1}(B) = \{x \in E \mid f(x) \in B\},
\]
isto {\'e}, $f^{-1}(B)$ {\'e} o conjunto dos elementos de $E$ que tem imagem em $B$ atrav{\'e}s de $f$.
\end{enumerate}
\end{definicao}

{\it Exemplos:}
\begin{enumerate}
\item Seja $E = \{1, 3, 5, 7, 9 \}$ e $F = \{0, 1, 2, 3, \dots, 10\}$ e $f : E \to F$ dada por $f(x) = x + 1$. Temos que
\begin{itemize}
\item $f(\{3, 5, 7\}) = \{f(3), f(5), f(7)\} = \{4, 6, 8\}$

\item $f(E) = \{f(1), f(3), f(5), f(7), f(9)\} = \{2, 4, 6, 8, 10\}$

\item $f(\emptyset) = \emptyset$

\item $f^{-1}(\{2, 4, 10\}) = \{x \in E \mid f(x) \in \{2, 4, 10\}\} = \{1, 3, 9\}$

\item $f^{-1}(\{0, 1, 3, 5, 7, 9\}) = \{x \in E \mid f(x) \in \{0, 1, 3, 5, 7, 9\}\} = \emptyset$
\end{itemize}

\item Sejam $E = F= \real$ e $f : \real \to \real$ dada por $f(x) = x^2$. Temos
\begin{itemize}
\item $f(\{1, 2, 3\}) = \{1, 4, 9\}$

\item $f([0,2]) = \{f(x) \in \real \mid 0 \le x \le 2 \} = \{x^2 \mid 0 \le x \le 2\} = [0, 4]$

\item $f^{-1}([1, 9]) = \{ x \in \real \mid 1 \le f(x) \le 9\} = \{x \in \real \mid 1 \le x^2 \le 9\} = [-1, -3] \cup [1, 3]$
\end{itemize}
\end{enumerate}

\begin{proposicao}
Seja $f : E \to F$ uma aplica{\c c}{\~a}o (ou fun{\c c}{\~a}o) e sejam $A$, $B \sub E$, $X$, $Y \sub F$.
\begin{enumerate}
\item Se $A \sub B$, ent{\~a}o $f(A) \sub f(B)$.

\item $f^{-1}(X \cup Y) = f^{-1}(X) \cup f^{-1}(Y)$.
\end{enumerate}
\end{proposicao}

\textbf{Demonstra{\c c}{\~a}o}: 
\begin{enumerate}
\item Se $y \in f(A)$, ent{\~a}o existe $x \in A$ tal que $f(x) = y$. Mas como $A \sub B$, ent{\~a}o $x \in B$ e da{\'\i} $y \in f(B)$. Logo $f(A) \sub f(B)$.

\item Seja $x \in f^{-1}(X \cup Y)$. Ent{\~a}o $f(x) \in X \cup Y$. Se $f(x) \in X$, entao $x \in f^{-1}(X)$ e da{\'\i} $x \in f^{-1}(X) \cup f^{-1}(Y)$. Se $f(x) \in Y$, ent{\~a}o $x \in f^{-1}(Y)$ e assim $x \in f^{-1}(X) \cup f^{-1}(Y)$. Logo, $f^{-1}(X \cup Y) \sub f^{-1}(X) \cup f^{-1}(Y)$.

Agora, seja $x \in f^{-1}(X) \cup f^{-1}(Y)$. Se $x \in f^{-1}(X)$, ent{\~a}o $f(x) \in X$, da{\'\i} $f(x) \in X \cup Y$, isto {\'e}, $x \in f^{-1}(X \cup Y)$. Se $x \in f^{-1}(Y)$, ent{\~a}o $f(x) \in Y$ e assim $f(x) \in X \cup Y$, isto {\'e}, $x \in f^{-1}(X \cup Y)$. Logo $f^{-1}(X) \cup f^{-1}(Y) \sub f^{-1}(X \cup Y)$.

Portanto, $f^{-1}(X \cup Y) = f^{-1}(X) \cup f^{-1}(Y)$. \#
\end{enumerate}
 

\chapter{Opera{\c c}{\~o}es em $\dfrac{\mathbb{Z}}{m\mathbb{Z}}$}

Durante esse t{\'o}pico, $m$ denotar{\'a} um n{\'u}mero inteiro positivo.

\section{Rela{\c c}{\~o}es de congru{\^e}ncia}
\subsection{Defini{\c c}{\~a}o}

\begin{definicao}[Congru{\^e}ncia] Sejam $a,b\in\mathbb{Z}$, dizemos que $a$ {\'e} congruente com $b$ m{\'o}dulo $m$ se $m|(a-b)$. Neste caso, escrevemos $a\equiv_{m}b$ ou $a\equiv b(mod\ m)$.\end{definicao}

Exemplos:
\begin{enumerate}
\item $5\equiv 2(mod\ 3)$, pois $3|(5-2)$
\item $3\equiv 1(mod\ 2)$, pois $2|(3-1)$
\item $3\equiv 9(mod\ 3)$, pois $2|(3-9)$

\end{enumerate}
\subsection{Propriedades}
\begin{proposicao} A congru{\^e}ncia m{\'o}dulo $m$ {\'e} uma rela{\c c}{\~a}o de equival{\^e}ncia em $\mathbb{Z}$.\end{proposicao}

\textbf{Demonstra{\c c}{\~a}o}: 
\begin{enumerate}
\item $\forall a\in\mathbb{Z},a\equiv a(mod\ m)$ pois $m|(a-a)$ (Reflexidade)
\item Se $a\equiv b(mod\ m)$, ent{\~a}o $m|(a-b)$. Da{\'\i}, $m|(-(a-b))$, ou seja, $m|(b-a)$. Da{\'\i} $b\equiv a(mod\ a)$ (Simetria)
\item Se $a\equiv b(mod\ m)$ e $b\equiv c(mod\ m)$, ent{\~a}o $m|(a-b)$ e $m|(b-c)$. Assim, $m|[(a-b)+(b-c)]$. Logo, $m|(a-c)$, isto {\'e}, $a\equiv c(mod\ m)$ (Transitividade)

Portanto {\'e} rela{\c c}{\~a}o de equival{\^e}ncia. \#

\end{enumerate}

\begin{teo} A rela{\c c}{\~a}o de congru{\^e}ncia m{\'o}dulo $m$ satisfaz as seguintes propriedades:
\begin{enumerate}
\item $a_{1}\equiv b_{1}(mod\ m)\Leftrightarrow a_{1}-b_{1}\equiv 0(mod\ m)$
\item Se $a_{1}\equiv b_{1}(mod\ m)$ e $a_{2}\equiv b_{2}(mod\ m)$, ent{\~a}o $a_{1}+a_{2}\equiv b_{1}+b_{2}(mod\ m)$
\item Se $a_{1}\equiv b_{2}(mod\ m)$ e $a_{2}\equiv b_{2}(mod\ m)$, ent{\~a}o $a_{1}a_{2}\equiv b_{1}b_{2}(mod\ m)$
\item Se $a\equiv b(mod\ m)$, ent{\~a}o $ax\equiv bx(mod\ m), \forall x\in\mathbb{Z}$
\item Vale a lei do cancelamento: se $d\in\mathbb{Z}$ e $mdc(d,m)=1$ ent{\~a}o\\ $ad\equiv bd(mod\ m)$ implica $a\equiv b(mod\ m)$

\end{enumerate}
\end{teo}

\textbf{Demonstra{\c c}{\~a}o}: Provemos o {\'\i}tem 3

Dizer que $a\equiv b(mod\ m)$ significa dizer que existe $t\in\mathbb{Z}$ tal que $a=b+tm$.

Assim, existem $m,l\in\mathbb{Z}$ tais que $a_{1}=b_{1}+km, a_{2}=b_{2}+lm$. Da{\'\i}\\
\[a_{1}a_{2}=b_{1}b_{2}+lb_{1}m+klm^{2}\]\\
\[a_{1}a_{2}=b_{1}b_{2}+\underbrace{(lb_{1}+kb_{2}+klm)}_{\in\mathbb{Z}}m\]

Ou seja, $a_{1}a_{2}=b_{1}b_{2}+pm$, onde $p=lb_{1}+kb_{2}+klm\in\mathbb{Z}$. Portanto, $a_{1}a_{2}\equiv b_{1}b_{2}(mod\ m)$.

Para o {\'\i}tem 5, se $ad\equiv bd(mod\ m)$, ent{\~a}o $m|d(a-b)$. Mas, $mdc(d,m)=1$, logo $m|(a-b)$, isto {\'e}, $a\equiv b(mod\ m)$.\#

Como a congru{\^e}ncia m{\'o}dulo $m$ {\'e} uma rela{\c c}{\~a}o de equival{\^e}ncia, podemos determinar suas classes de equival{\^e}ncia. Assim, dado $n\in\mathbb{Z}$, temos
\[C(n)=\{x\in\mathbb{Z}/x\equiv n(mod\ m)\}\]

Denotaremos $C(n)$ por $R_{m}(n)$ ou $\bar{n}$, quando n{\~a}o houver possibilidade de confus{\~a}o.

Por exemplo, fixando $m$\\
$R_{m}(0)=\{x\in\mathbb{Z}/x\equiv 0(mod\ m)\}=\{x\in \mathbb{Z}/x=mk, k\in\mathbb{Z}\}=m\mathbb{Z}$\\
$R_{m}(1)=\{x\in\mathbb{Z}/x\equiv 1(mod\ m)\}=\{x\in\mathbb{Z}/x=1+km, k\in\mathbb{Z}\}$\\
$R_{m}(n)=\{x\in\mathbb{Z}/x=n+km, k\in\mathbb{Z}\}$

\subsection{Classes de equival{\^e}ncia m{\'o}dulo $m$}

\begin{proposicao} As classes de equival{\^e}ncia definidas pela congru{\^e}ncia m{\'o}dulo $m$ s{\~a}o determinadas pelos restos da divis{\~a}o euclidiana por $m$. Em outras palavras, $R_{m}(n)$ {\'e} o conjunto dos n{\'u}meros inteiros cujo resto na divis{\~a}o euclidiana por $m$ {\'e} $n$.\end{proposicao}

\textbf{Demonstra{\c c}{\~a}o}: Dado $x\in\mathbb{Z}$, pela divis{\~a}o de Euclides, podemos escrever $x=km+r$ onde $0\leq r\leq m$. Da{\'\i}, $x-r=km$, isto {\'e}, $m|(x-r)$. Logo $x\in R_{m}(r)$. Portanto, se $r=n$, ent{\~a}o $x\in R_{m}(n)$ e neste caso, $x=km+n=n+km$, ou seja, o resto da divis{\~a}o euclidiana de $x$ por $m$ {\'e} $n$.\#	

\begin{cor} $R_{m}(k)=R_{m}(l)$ se, e somente se, $k\equiv l(mod\ m)$.\end{cor}

Exemplos:
\begin{enumerate}
\item Se $m=2$, ent{\~a}o os poss{\'\i}veis restos na divis{\~a}o euclidiana por 2 s{\~a}o 0 e 1. Logo, existem duas classes de equival{\^e}ncia, a saber $R_{2}(0)$ e $R_{2}(1)$
\item Se $m=3$, ent{\~a}o os poss{\'\i}veis restos da divis{\~a}o euclidiana s{\~a}o 0,1 e 2. Da{\'\i}\\
$R_{3}(0)=3\mathbb{Z}$\\
$R_{3}(1)=\{x\in\mathbb{Z}/x=3q+1,q\in\mathbb{Z}\}$\\
$R_{3}(2)=\{x\in\mathbb{Z}/x=3q+2,q\in\mathbb{Z}\}$

\end{enumerate}

\begin{proposicao} Na rela{\c c}{\~a}o de equival{\^e}ncia m{\'o}dulo $m$ existem $m$ classes de equival{\^e}ncia.\end{proposicao}

\textbf{Demonstra{\c c}{\~a}o}: Os poss{\'\i}veis restos na divis{\~a}o euclidiana por $m$ s{\~a}o $0,1,...,(m-1)$. Como cada poss{\'\i}vel resto define uma classe de equival{\^e}ncia diferente, existem exatamente $m$ classes de equival{\^e}ncia.\#

\section{Conjunto quociente $\left(\dfrac{\mathbb{Z}}{m\mathbb{Z}}\right)$}


\begin{nota}[Conjunto quociente] Fixado $m$ inteiro positivo, denotaremos\\
$R_{m}(0)=\bar{0}$\\
$R_{m}(1)=\bar{1}$\\
$\vdots$\\
$R_{m}(m-1)=\overline{m-1}$

O conjunto quociente desta rela{\c c}{\~a}o ser{\'a} denotado por $\displaystyle\frac{\mathbb{Z}}{m\mathbb{Z}}$ e $\displaystyle\frac{\mathbb{Z}}{m\mathbb{Z}}=\{\bar{0},\bar{1},...,\overline{m-1}\}$
\end{nota}

Queremos definir um meio de somar e multiplicar os elementos de $\displaystyle\frac{\mathbb{Z}}{m\mathbb{Z}}$. Por exemplo, em $\displaystyle\frac{\mathbb{Z}}{2\mathbb{Z}}=\{\bar{0},\bar{1}\}$ temos que a soma de pares {\'e} par, soma de par com {\'\i}mpar {\'e} {\'\i}mpar e a soma de {\'\i}mpares {\'e} par.

Podemos escrever
\vspace{0,05cm}\\
$\bar{0}\oplus\bar{0}=\overline{0+0}=\bar{0}$\\
$\bar{0}\oplus\bar{1}=\overline{0+1}=\bar{1}$\\
$\bar{1}\oplus\bar{1}=\overline{1+1}=\bar{0}$

Para multiplica{\c c}{\~a}o, temos
\vspace{0,05cm}\\
$\bar{0}\odot\bar{0}=\overline{0.0}=\bar{0}$\\
$\bar{0}\odot\bar{1}=\overline{0.1}=\bar{0}$\\
$\bar{1}\odot\bar{1}=\overline{1.1}=\bar{1}$

Em $\displaystyle\frac{\mathbb{Z}}{m\mathbb{Z}}$ definimos
\begin{eqnarray}
\bar{a}\oplus\bar{b}=\overline{a+b}\\
\bar{a}\odot\bar{b}=\overline{a.b}
\end{eqnarray}
Para $\bar{a},\bar{b}\in\displaystyle\frac{\mathbb{Z}}{m\mathbb{Z}}$

\begin{proposicao} As opera{\c c}{\~o}es de soma e produto definidas em (5.1) e (5.2) s{\~a}o independentes dos representantes das classes.\end{proposicao}

\textbf{Demonstra{\c c}{\~a}o}: Dadas duas classes com representantes diferentes, $\bar{a}_{1}=\bar{a}_{2},\  \bar{b}_{1}=\bar{b}_{2}, a_{1}\neq a_{2}, b_{1}=b_{2}$, temos:\\
\[\overline{a_{1}+b_{1}}=\bar{a}_{1}\oplus\bar{b}_{1}=\bar{a}_{2}\oplus\bar{b}_{2}=\overline{a_{2}+b_{2}}\]\\
\[\overline{a_{1}b_{1}}=\bar{a}_{1}\odot\bar{b}_{1}=\bar{a}_{2}\odot\bar{b}_{2}=\overline{a_{2}b_{2}}\]\\

C.Q.D.\#

Exemplo: Determine a some e multiplica{\c c}{\~a}o em:

$\displaystyle\frac{\mathbb{Z}}{4\mathbb{Z}}=\{\bar{0},\bar{1},\bar{2},\bar{3}\}$
\begin{table}[h]
   \centering 
   \setlength{\arrayrulewidth}{0,5\arrayrulewidth}
   \caption{\it Soma}
   \begin{tabular}{|c|c|c|c|c|} 
      \hline
      $\oplus$ & $\bar{0}$ & $\bar{1}$ & $\bar{2}$ & $\bar{3}$ \\
      \hline
      $\bar{0}$ & $\bar{0}$ & $\bar{1}$ & $\bar{2}$ & $\bar{3}$ \\
      \hline
      $\bar{1}$ & $\bar{1}$ & $\bar{2}$ & $\bar{3}$ & $\bar{0}$ \\
      \hline
      $\bar{2}$ & $\bar{2}$ & $\bar{3}$ & $\bar{0}$ & $\bar{1}$ \\
      \hline
      $\bar{3}$ & $\bar{3}$ & $\bar{0}$ & $\bar{1}$ & $\bar{2}$ \\
      \hline
   \end{tabular}
\end{table}

\begin{table}[h]
   \centering 
   \setlength{\arrayrulewidth}{0,5\arrayrulewidth}
   \caption{\it Multiplica{\c c}{\~a}o}
   \begin{tabular}{|c|c|c|c|c|} 
      \hline
      $\odot$ & $\bar{0}$ & $\bar{1}$ & $\bar{2}$ & $\bar{3}$ \\
      \hline
      $\bar{0}$ & $\bar{0}$ & $\bar{0}$ & $\bar{0}$ & $\bar{0}$ \\
      \hline
      $\bar{1}$ & $\bar{0}$ & $\bar{1}$ & $\bar{2}$ & $\bar{3}$ \\
      \hline
      $\bar{2}$ & $\bar{0}$ & $\bar{2}$ & $\bar{0}$ & $\bar{2}$ \\
      \hline
      $\bar{3}$ & $\bar{0}$ & $\bar{3}$ & $\bar{2}$ & $\bar{1}$ \\
      \hline
   \end{tabular}
\end{table}


\subsection{Elementos Invers{\'\i}veis de $\displaystyle\frac{\mathbb{Z}}{m\mathbb{Z}}$}

\subsubsection{Inversibilidade}
\begin{definicao}[Inversibilidade] Um elemento $\bar{a}\in\displaystyle\frac{\mathbb{Z}}{m\mathbb{Z}}$ {\'e} invers{\'\i}vel se, e somente se, existem $\bar{b}\in\displaystyle\frac{\mathbb{Z}}{m\mathbb{Z}}$ tal que $\bar{a}\odot\bar{b}=\bar{1}$.\end{definicao}

Neste caso, $\bar{b}$ {\'e} chamado inverso de $\bar{a}$ e denotaremos $\bar{b}=(\bar{a})^{-1}$.

Quando $\bar{b}$ existe, ele {\'e} {\'u}nico. De fato, dado $\bar{a}\in\displaystyle\frac{\mathbb{Z}}{m\mathbb{Z}}$, se existem $\bar{b},\bar{d}\in\displaystyle\frac{\mathbb{Z}}{m\mathbb{Z}}$ tais que $\bar{a}\odot\bar{b}=\bar{1}=\bar{a}\odot\bar{d}$, ent{\~a}o $\bar{b}=\bar{b}\odot\bar{1}=\bar{b}\odot (\bar{a}\odot\bar{d})=(\bar{b}\odot\bar{a})\odot\bar{d}=\bar{1}\odot\bar{d}=\bar{d}$.

\begin{proposicao} Um elemento $\bar{a}\in\displaystyle\frac{\mathbb{Z}}{m\mathbb{Z}}$ {\'e} invers{\'\i}vel se, e somente se,
 \[mdc(a,m)=1\]
\end{proposicao}

\textbf{Demonstra{\c c}{\~a}o}: Suponha que existe $\bar{b}\in\displaystyle\frac{\mathbb{Z}}{m\mathbb{Z}}$ tal que $\bar{a}\odot\bar{b}=\bar{1}$. Assim, $\overline{ab}=\bar{1}$, ou seja, $ab\equiv 1(mod\ m)$. Da{\'\i}, $ab-1=km, k\in\mathbb{Z}$, logo $ab+m(-k)=1$, e ent{\~a}o $mdc(a,m)=1$.

Agora suponha que $mdc(a,m)=1$. Logo, existem $x_{0}, y_{0}\in\mathbb{Z}$ tais que $ax_{0}+my_{0}=1$, isto {\'e}, $ax_{0}-1=m(-y_{0})$. Logo $ax_{0}\equiv 1(mod\ m)$, ou seja, $\overline{ax_{0}}=\bar{1}$. Portanto, $\bar{a}\odot\bar{x_{0}}=\bar{1}$.\#

Exemplos:
\begin{enumerate}
\item Em $\displaystyle\frac{\mathbb{Z}}{4\mathbb{Z}}$ existem dois elementos invers{\'\i}veis que s{\~a}o $\bar{1}$, cujo inverso {\'e} $\bar{1}$, e o $\bar{3}$, cujo inverso {\'e} $\bar{3}$.
\item Em $\displaystyle\frac{\mathbb{Z}}{11\mathbb{Z}} $, todos elementos, exceto $\bar{0}$, possuem inverso:

\begin{table}[h]
   \centering 
   \setlength{\arrayrulewidth}{0,5\arrayrulewidth}
   \caption{\it Inversos em $\displaystyle\frac{\mathbb{Z}}{11\mathbb{Z}}$}
   \begin{tabular}{|c|c|c|c|c|c|c|c|c|c|c|} 
      \hline
      Elemento & $\bar{1}$ & $\bar{2}$ & $\bar{3}$ & $\bar{4}$ & $\bar{5}$ & $\bar{6}$ & $\bar{7}$ & $\bar{8}$ & $\bar{9}$ & $\bar{10}$ \\
      \hline
      Inverso & $\bar{1}$ & $\bar{6}$ & $\bar{4}$ & $\bar{3}$ & $\bar{9}$ & $\bar{2}$ & $\bar{8}$ & $\bar{7}$ & $\bar{5}$ & $\bar{10}$ \\
      \hline
   \end{tabular}
\end{table}
\end{enumerate}

O n{\'u}mero de elementos invers{\'\i}veis de $\displaystyle\frac{\mathbb{Z}}{m\mathbb{Z}}$ {\'e} igual a quantidade de n{\'u}meros coprimos com $m$. Esse n{\'u}mero {\'e} denotado por $\varphi(m)$ e {\'e} chamado fun{\c c}{\~a}o $\varphi$ de Euler. Pode-se demonstrar que
\[\varphi(m)=m\displaystyle\prod_{p/m}\left(1-\displaystyle\frac{1}{p}\right)\]\\
Onde o produto varia sobre todos os divisores primos de m, sem repeti{\c c}{\~a}o.

Por exemplo, para $\displaystyle\frac{\mathbb{Z}}{100\mathbb{Z}}$ temos:\\
$100=2^{2}5^{2}$

Da{\'\i},\\
$\varphi(100)=100\left(1-\displaystyle\frac{1}{2}\right)(1-\displaystyle{1}{5})=40$

Logo, em $\displaystyle\frac{\mathbb{Z}}{100\mathbb{Z}}$ existem 40 elementos invers{\'\i}veis.

\begin{nota}[Conjunto dos elementos invers{\'\i}veis] Denotaremos o conjunto de todos os elementos invers{\'\i}veis de $\displaystyle\frac{\mathbb{Z}}{m\mathbb{Z}}$ por $\left(\displaystyle\frac{\mathbb{Z}}{m\mathbb{Z}}\right)^{*}$, ou ainda $U\left(\displaystyle\frac{\mathbb{Z}}{m\mathbb{Z}}\right)$.\end{nota}

\begin{proposicao} Sejam $\bar{a},\bar{b}\in\left(\displaystyle\frac{\mathbb{Z}}{m\mathbb{Z}}\right)^{*}$. Ent{\~a}o $\bar{a}\odot\bar{b}\in\left(\displaystyle\frac{\mathbb{Z}}{m\mathbb{Z}}\right)^{*}$.\end{proposicao}

\textbf{Demonstra{\c c}{\~a}o}: Por uma proposi{\c c}{\~a}o anterior, basta verificar que\\ $mdc(ab,m)=1$. Para que $\bar{a}\odot\bar{b}\in\left(\displaystyle\frac{\mathbb{Z}}{m\mathbb{Z}}\right)^{*}$.

Como $\bar{a},\bar{b}\in\left(\displaystyle\frac{\mathbb{Z}}{m\mathbb{Z}}\right)^{*}$, ent{\~a}o $mdc(a,m)=1$ e $mdc(b,m)=1$.

Assim, existem $x_{0},y_{0},x_{1},y_{1}\in\mathbb{Z}$ tais que\\
$ax_{0}+my_{0}=1$\\
$bx_{1}+my_{1}=1$

Da{\'\i},
\[abx_{0}x_{1}+max_{0}y_{1}+mbx_{1}y_{0}+m^{2}y_{0}y_{1}=1\]
\[\underbrace{abx_{0}x_{1}}_{\in\mathbb{Z}}+m\underbrace{(ax_{0}y_{1}+bx_{1}y_{0}+my_{0}y_{1})}_{\in\mathbb{Z}}=1\]

Logo, $mdc(ab,m)=1$, ou seja, $\bar{a}\odot\bar{b}\in\left(\displaystyle\frac{\mathbb{Z}}{m\mathbb{Z}}\right)^{*}$.\#

\chapter{An{\'e}is}

\section{Conjunto munido de uma opera{\c c}{\~a}o bin{\'a}ria}
\subsection{Defini{\c c}{\~a}o}

\begin{definicao}[Conjunto munido de uma opera{\c c}{\~a}o bin{\'a}ria] Seja C um conjunto n{\~a}o vazio. Dizemos que C est{\'a} munido (ou equipado) de uma opera{\c c}{\~a}o bin{\'a}ria quando existe uma fun{\c c}{\~a}o
\begin{center}
$\Delta:C \times C\rightarrow C$\\
$(a,b)\longmapsto a\Delta b$
\end{center}
\end{definicao}

A opera{\c c}{\~a}o bin{\'a}ria tamb{\'e}m {\'e} chamada de opera{\c c}{\~a}o interna.

Exemplos:
\begin{enumerate}
\item A soma dos n{\'u}meros dos conjuntos $\mathbb{Z}$ e $\mathbb{Q}$ {\'e} uma opera{\c c}{\~a}o bin{\'a}ria
\item O produto dos $\mathbb{Z}$ e $\mathbb{Q}$ {\'e} uma opera{\c c}{\~a}o bin{\'a}ria
\item $\displaystyle\frac{\mathbb{Z}}{m\mathbb{Z}}=\{\bar{0},\bar{1},...,\overline{m-1}\}$\\
$\bar{a},\bar{b}\in\displaystyle\frac{\mathbb{Z}}{m\mathbb{Z}}$\\
$\bar{a}\oplus\bar{b}=\overline{a+b}\in\displaystyle\frac{\mathbb{Z}}{m\mathbb{Z}}$\\
$\bar{a}\odot\bar{b}=\overline{ab}\in\displaystyle\frac{\mathbb{Z}}{m\mathbb{Z}}$
\item $U(\displaystyle\frac{\mathbb{Z}}{m\mathbb{Z}})=\{\bar{a}\in\displaystyle\frac{\mathbb{Z}}{m\mathbb{Z}}/\exists\bar{b}\in\displaystyle\frac{\mathbb{Z}}{m\mathbb{Z}}/\bar{a}\odot\bar{b}=1\}$\\
"$\odot$" em $U(\displaystyle\frac{\mathbb{Z}}{m\mathbb{Z}})$ {\'e}  uma opera{\c c}{\~a}o bin{\'a}ria\\
$U(\displaystyle\frac{\mathbb{Z}}{4\mathbb{Z}})=\{\bar{1},\bar{3}\}$\\
$\bar{1}\oplus\bar{3}=\bar{0}\notin U(\displaystyle\frac{\mathbb{Z}}{m\mathbb{Z}})$\\
"$\oplus$" em $U(\displaystyle\frac{\mathbb{Z}}{m\mathbb{Z}})$ n{\~a}o {\'e} opera{\c c}{\~a}o bin{\'a}ria.
\item ``$\div$" em $\mathbb{Q}^{*}$ {\'e} uma opera{\c c}{\~a}o bin{\'a}ria
\item ``$\div$" em $\mathbb{N}, \mathbb{Z}, \mathbb{Z}^{*}, \mathbb{Q}$ n{\~a}o {\'e} opera{\c c}{\~a}o bin{\'a}ria

\end{enumerate}

\section{Defini{\c c}{\~a}o de anel}

\begin{definicao}[Anel] Um conjunto n{\~a}o vazio $A$ munido de duas opera{\c c}{\~o}es ``+" e ``.", chamados soma e produto, {\'e} um anel quando as seguintes condi{\c c}{\~o}es s{\~a}o verdadeiras:
\begin{enumerate}
\item \textbf{Elemento Neutro}: Existe em $``A"$ um elemento denotado por 0(zero) ou $0_{A}$ tal que para todo elemento $a\in A$ vale
\[a+0=0+a=a\]
\item \textbf{Elemento Oposto}: Para cada elemento $a\in A$, existe $b\in A$ tal que
\[a+b=b+a=0\]
\item \textbf{Associatividade}: para todos $a,b,c\in A$ vale que
\[(a+b)+c=a+(b+c)\]
Essa propriedade {\'e} chamada propriedade associativa da soma
\item \textbf{Comutatividade}: Para todo $a,b\in A$ vale
\[a+b=b+a\]
\item \textbf{Distributividade}: Para todo $a,b,c\in A$ vale
\[(a+b)c=ac+bc\]
Essa propriedade {\'e} chamada distributiva em rela{\c c}{\~a}o ao produto
\item \textbf{Distributividade}: Para todos $a,b,c\in A$ vale
\[a(b+c)=ab+ac\]
Essa {\'e} a propriedade distributiva do produto em rela{\c c}{\~a}o {\`a} soma

Al{\'e}m disso, se A satisfizer
\item \textbf{Comutatividade}: Se para todos $a,b\in A$ vale
\[ab=ba\]
Dizemos que A {\'e} um anel comutativo
\item \textbf{Elemento um}: Se existe em A um elemento denotado por 1 ou $1_{A}$ tal que
\[a1=1a=a\]
Para todo $a\in A$, ent{\~a}o chamamos de anel com unidade ou anel unit{\'a}rio. O elemento 1 {\'e} chamado elemento um ou unidade
\item \textbf{Associatividade}: Se para todos $a,b,c\in A$, vale que
\[a(bc)=(ab)c\]
Dizemos que A {\'e} anel associativo em $M_{2}(\mathbb{R})=\left\{\left[\begin{array}{cc}
a & b\\
c & d
\end{array}\right]/a,b,c,d\in\mathbb{R}\right\}$
\end{enumerate}
\end{definicao}
Defina a soma como a soma usual de matrizes e defina o produto do seguinte modo: dados $A,B\in M_{2}(\mathbb{R})$
\[[A,B]=\underbrace{\underbrace{AB}-\underbrace{BA}}_{Produto\ usual\ de\ Matrizes}\]

Verificar que $M_{2}(\mathbb{R})$ com a soma usual de matrizes e produto [,] {\'e} um anel, mas n{\~a}o {\'e} associativo.

\[[[A,B],C]\neq [A,[B,C]]\]

Quando A, munido de duas opera{\c c}{\~o}es "+" e "." {\'e} um anel, ele ser{\'a} denotado (A,+,.), para indicar claramente as opera{\c c}{\~o}es bin{\'a}rias em A.

No exemplo anterior:\\
$(M_{2}(\mathbb{R};+;[,])$ {\'e} anel pois

\[\underbrace{\left[\begin{array}{cc}
a & b\\
c & d
\end{array}\right]}_{A},\left[\begin{array}{cc}
x & y\\
z & t
\end{array}\right]\in M_{2}(\mathbb{R})\]
\[\left[\begin{array}{cc}
a & b\\
c & d
\end{array}\right]+\left[\begin{array}{cc}
x & y\\
z & t
\end{array}\right]=\left[\begin{array}{cc}
a+x & b+y\\
c+z & d+t
\end{array}\right]\]
\begin{enumerate}
\item $A\in M_{2}(\mathbb{R})$, existe $0_{2}\in M_{2}(\mathbb{R})$ tal que $A+0_{2}=0_{2}+A$. A saber:\\
\[0_{2}=\left[\begin{array}{cc}
0 & 0\\
0 & 0
\end{array}\right]\]
\item Para todo $A\in M_{2}(\mathbb{R})$ existe $B\in M_{2}(\mathbb{R})$ tal que $A+B=B+A=0_{2}$. A saber:\\
\[B=\left[\begin{array}{cc}
-a & -b\\
-c & -d
\end{array}\right]\]
\item $(A+B)+C=A+(B+C)$ pois em $\mathbb{R}$ vale a associatividade
\item $A+B=B+A$, pois em $\mathbb{R}$ a soma {\'e} comutativa
\item $[(A+B),C]=(A+B)C-C(A+B)$\\
$=AC+BC-CA-CB=AC-CA+BC-CB=[A,C]+[B,C]$
\item $[A,B]=AB-BA, [B,A]=BA-AB \Rightarrow [A,B]=-[B,A]$
\end{enumerate}

Exemplos:
\begin{enumerate}
\item $(\mathbb{Z},+,.),(\mathbb{Q},+,.),(\mathbb{R},+,.),(\mathbb{C},+,.),\left(\displaystyle\frac{\mathbb{Z}}{m\mathbb{Z}}\right)$\\
S{\~a}o an{\'e}is associativos, comutativos e com unidade em $\left(\displaystyle\frac{\mathbb{Z}}{m\mathbb{Z}}\right)$, o elemento neutro {\'e} a classe $\bar{0}$ e a unidade {\'e} a classe $\bar{1}$.
\item  Seja $A=\mathbb{Z}=\{f:\mathbb{Z}\rightarrow\mathbb{Z}/f$ {\'e} fun{\c c}{\~a}o$\}$. Dadas duas fun{\c c}{\~o}es quaisquer $f,g\in\mathbb{Z}$x$\mathbb{Z}$, definimos $f\oplus g:\mathbb{Z}\rightarrow\mathbb{Z}$ e $f\odot g:\mathbb{Z}\rightarrow\mathbb{Z}$ como:


$(f\oplus g)=f(x)+g(x)$\\
$(f\odot g)=f(x)g(x)$
\begin{enumerate}
\item Dado $x\in\mathbb{Z}$\\
$(f\oplus g)(x)=f(x)+g(x)=g(x)+f(x)=(g\oplus f)(x)$, portanto $f\oplus g=g\oplus f$
\item Dado $x\in\mathbb{Z}$\\
$(f\odot g)(x)=f(x)g(x)=g(x)f(x)=(g\odot f)(x)$, portanto $f\odot g=g\odot f$
\item Definida $0\mathbb{Z}\rightarrow\mathbb{Z}$ como $0(x)=0,\forall x\in\mathbb{Z}$, temos\\
$(f(x)\oplus 0)(x)=f(x)\oplus 0(x)=0(x)\oplus f(x)=f(x)$

\end{enumerate}
\end{enumerate}

\section{Propriedades de um Anel}
\begin{enumerate}
\item O elemento neutro {\'e} {\'u}nico.\\
Suponha que exista $0_{1},0_{2}\in A$ tais que\\
$a+0_{1}=0_{1}+a=a;\ b+0_{2}=0_{2}+b=b$, da{\'\i} $0_{1}=0_{1}+0_{2}=0_{2}$
\item Para cada $a\in A$ existe um {\'u}nico oposto.\\
De fato, suponha que existam $b_{1},b_{2}\in A$ tais que\\
$a+b_{1}=0;\ a+b_{2}=0$. Da{\'\i} $b_{1}=b_{2}+0=b_{1}+(a+b_{2})=(b_{1}+a)+b_{2}=0+b_{2}=b_{1}=b_{2}$
\item Para todo $a\in A, -(-a)=a$\\
Dado $a\in A,-a$ {\'e} oposto de $a$, isto {\'e}, $a+(-a)=0$. Logo o oposto de $(-a)$ {\'e} $a$, da{\'\i} $-(-a)=a$.
\item Dados $a_{1},a_{2},...,a_{n}\in A, n\leq 2$, ent{\~a}o\\
$-(a_{1}+a_{2}+...+a_{n})=(-a_{1})+(-a_{2})+...+(-a_{n})$
\item Para todo $a,x,y\in A$, se $a+x=a+y$, ent{\~a}o $x=y$
\item Para todo $a\in A,\ a0=0a=0$\\
Temos $0+0.0=a0=a(0+0)=a0+a0$, da{\'\i}\\
$\underbrace{0.0}_{a0}+0=\underbrace{0.0}_{a0}+a0$. Pela propriedade 5 $a0=0$
\item Para todo $a,b\in A$, temos $a(-b)=-ab$\\
Provemos que $a(-b)=-ab$\\
$a(-b)+ab=a((-b)+b)=a0=0$, portanto $-ab=a(-b)$
\item Para todo $a,b\in A,\ ab=(-a)(-b)$\\
\end{enumerate}

\section{Anel de Integridade}
\subsubsection{Defini{\c c}{\~a}o}

\begin{definicao}[Anel de Integridade] Um anel comutativo A {\'e} um anel de integridade quando para todos $a,b\in A$, se $ab=0$,ent{\~a}o $a=0\vee b=0$. Um anel de integridade tamb{\'e}m {\'e} chamado de dom{\'\i}nio de integridade ou simplesmente de dom{\'\i}nio.\end{definicao}

Se $a$ e $b$ s{\~a}o elementos n{\~a}o nulos de um anel $A$ tais que $ab=0$, ent{\~a}o $a$ e $b$ s{\~a}o chamados de divisores pr{\'o}prios de zero.

Exemplos:
\begin{enumerate}
\item Os an{\'e}is $\mathbb{Z},\mathbb{Q},\mathbb{R},\mathbb{C}$ s{\~a}o an{\'e}is de integridade.
\item Em geral, $\displaystyle\frac{\mathbb{Z}}{m\mathbb{Z}}$ n{\~a}o {\'e} anel de integridade, por exemplo, em $\displaystyle\frac{\mathbb{Z}}{4\mathbb{Z}},\ \bar{2}\neq\bar{0}$, no entanto $\bar{2}\odot\bar{2}=\bar{4}=\bar{0}$
\item $M_{n}(\mathbb{R})$ n{\~a}o {\'e} um anel de integridade, por exemplo, em $M_{2}(\mathbb{R})$\\
\[A=\left[\begin{array}{cc}
1 & 0\\
0 & 0
\end{array}\right]\neq\left[\begin{array}{cc}
0 & 0\\
0 & 0
\end{array}\right],\ B=\left[\begin{array}{cc}
0 & 0\\
1 & 0
\end{array}\right]\neq\left[\begin{array}{cc}
0 & 0\\
0 & 0
\end{array}\right]\]
\[AB=\left[\begin{array}{cc}
0 & 0\\
0 & 0
\end{array}\right]\]

\end{enumerate}

Suponha que $m=nk,\ m>n>1$ e $m>k>$. Logo, em $\displaystyle\frac{\mathbb{Z}}{m\mathbb{Z}}, \bar{n}\neq\bar{0}$ e $\bar{k}\neq\bar{0}$. Logo, se $m$ n{\~a}o {\'e} primo, ent{\~a}o $\displaystyle\frac{\mathbb{Z}}{m\mathbb{Z}}$ n{\~a}o {\'e} um anel de integridade.

Agora, suponha que $m=p$ primo. Sejam $\bar{a},\bar{b}\in\displaystyle\frac{\mathbb{Z}}{m\mathbb{Z}}$ tais que $\bar{a}\odot\bar{b}=\bar{0}$, ou seja, $ab\equiv 0(mod\ p)$. Da{\'\i} $p|ab$. Logo $p|a\vee p|b$. Portanto, $\bar{a}=\bar{0}\vee \bar{b}=\bar{0}$. Assim, $\displaystyle\frac{\mathbb{Z}}{m\mathbb{Z}}$ {\'e} anel de integridade se, e somente se, m {\'e} primo.

\subsubsection{Subconjunto anel}
\begin{definicao}[Subconjunto Anel] Seja (A,+,.) um anel. Dizemos que um subconjunto n{\~a}o vazio $C\subseteq A$ {\'e} um anel.\end{definicao}

Exemplos:
\begin{enumerate}
\item $\displaystyle\frac{\mathbb{Z}}{4\mathbb{Z}}=\{\bar{0},\bar{1},\bar{2},\bar{3}\},\oplus,\odot$\\
$S=\{\bar{0},\bar{2}\},\oplus,\odot$\\
$(S,\oplus,\odot)$ {\'e} um subanel de $\displaystyle\frac{\mathbb{Z}}{4\mathbb{Z}}$
\item Todo anel $A$ sempre tem dois suban{\'e}is: $\{0_{A}\}$ e $A$, que s{\~a}o chamados de an{\'e}is triviais.
\item No anel $\mathbb{Z}$, o conjunto $m\mathbb{Z}(m>1)$ {\'e} um subanel de $\mathbb{Z}$
\end{enumerate}

\begin{proposicao} Seja $(A,+,.)$ um anel. Um subconjunto n{\~a}o vazio $C\subseteq A$ {\'e} um sub- anel de $A$ se, e somente se, $x+y\in C,\ xy\in C\ \forall x,y\in C$.\end{proposicao}

$(A,+,.),(B,\oplus,\odot)$ An{\'e}is\\
$f:A\rightarrow B$\\
$a\rightarrow f(a)$\\
$g:\mathbb{Z}\rightarrow \displaystyle\frac{\mathbb{Z}}{m\mathbb{Z}}$\\
$x\rightarrow\bar{x}$

$g(x+y)=\overline{x+y}=\bar{x}\oplus\bar{y}=g(x)\oplus g(y)$\\
$g(x+y)=g(x)\oplus g(y)$

$g(xy)=\overline{xy}=\bar{x}\odot\bar{y}=g(x)\odot g(y)$

$g(1)=\bar{1}$

\section{Homomorfismo}
\subsubsection{Defini{\c c}{\~a}o}

\begin{definicao}[Homomorfismo] Um homomorfismo do anel (A,+,.) no anel $(B,\oplus,\odot)$ {\'e} uma fun{\c c}{\~a}o $f:A\rightarrow B$ que satisfaz:
\begin{enumerate}
\item $f(x+y)=f(x)+f(y),\forall x,y\in A$
\item $f(xy)=f(x)f(y),\forall x,y\in A$
\item $f(1_{A})=1_{B}$, onde $1_{A}$ {\'e} a unidade de A e $1_{B}$ {\'e} a unidade de B
\end{enumerate}
\end{definicao}

Se (A,+,.) {\'e} um anel, ent{\~a}o $f:A\rightarrow A$ dada por $f(a)=a$ {\'e} um homomorfismo de A em A pois:
\begin{enumerate}
\item $f(x+y)=x+y=f(x)+f(y)$
\item $f(xy)=xy=f(x)f(y)$
\item $f(1_{A})=1_{A}$
\end{enumerate}

\subsubsection{Propriedades}
\begin{proposicao} Seja $f:A\rightarrow B$ homomorfismo do anel A no anel B. Ent{\~a}o:
\begin{enumerate}
\item $f(0_{A})=0_{B}$
\item $f(-a)=-f(a),\forall a\in A$
\end{enumerate}
\end{proposicao}

\textbf{Demonstra{\c c}{\~a}o}:
\begin{enumerate}
\item Da condi{\c c}{\~a}o 1 da defini{\c c}{\~a}o de homomorfismo, fazendo $x=y0_{A}$, temos\\
$f(0_{A}+0_{A})=f(0_{A})\oplus f(0_{A})$\\
mas $0_{A}+0_{A}=0_{A}$. Da{\'\i}\\
$f(0_{A})=f(0_{A})+f(0_{A})$\\
Somando $-f(0_{A})$ em ambos os lados\\
$f(0_{A})\oplus(-f(0_{A}))=(f(0_{A})+f(0_{A}))+(-f(0_{A}))$\\
$0_{B}=f(0_{A})+0_{B}$\\
$f(0_{A})=0_{B}$
\item Temos $0_{B}=f(0_{A})=f(a+(-a))=f(a)\oplus f(-a)$\\
Somando $-f(a)$ em ambos os lados\\
$0_{B}\oplus(-f(a))=[f(a)\oplus f(-a)]+(-f(a))$
$-f(a)=f(-a)\oplus(f(a)\oplus(-f(a)))$\\
$f(-a)=-f(a)$.\#
\end{enumerate}

Seja $f:\mathbb{Z}\rightarrow\mathbb{Z}$ um homomorfismo. Dado $n\in\mathbb{Z},n\geq 0$. Temos da{\'\i}, \[f(n)=f(\underbrace{1+...+1}_{n\ vezes})=\underbrace{f(1)+...+f(1)}_{n\ vezes}=nf(1)=n1\]
$f(n)=n,\forall n\in\mathbb{Z}$

\subsection{Epimorfismo, monomorfismo e isomorfismo}
\begin{definicao}[Epimorfismo, monomorfismo e isomorfismo] Seja $f:A\rightarrow B$ um homomorfismo, onde $A$ e $B$ s{\~a}o an{\'e}is. Dizemos que
\begin{enumerate}
\item $f$ {\'e} um epimorfismo se $f$ for sobrejetora
\item $f$ {\'e} um monomorfismo se $f$ for injetora
\item $f$ {\'e} um isomorfismo se $f$ for bijetora
\item Quando $A=B$ e $f$ {\'e} um isomorfismo, ent{\~a}o $f$ {\'e} um automorfismo
\end{enumerate}
\end{definicao}

\section{Ideal de um anel}
\subsubsection{Defini{\c c}{\~a}o}
\begin{definicao}[Ideal em um anel] Seja $(A,+,.)$ um anel comutativo. Um ideal em $A$ {\'e} um conjunto n{\~a}o vazio $I$ tal que:
\begin{enumerate}
\item Para todo $a,b\in I,a-b\in I$
\item Para todo $a,b\in I, ab=ba\in I$
\end{enumerate}
\end{definicao}

Quando $I=A$ ou $I=\{0_{A}\},\ I$ {\'e} chamado de anel trivial.

\subsubsection{Propriedades}
\begin{proposicao} Seja $A$ um anel comutativo e $I$ um ideal de $A$. Ent{\~a}o \[0_{A}\in I\].\end{proposicao}

\textbf{Demonstra{\c c}{\~a}o}: Temos que da defini{\c c}{\~a}o de ideal, $ab\in I$, para todo $a,b\in I$.

Assim, dado $a\in I,\ a0_{A}=0_{A}\in I$.\#

\begin{proposicao} Sejam $A$ um anel comutativo e unit{\'a}rio e $I$ um ideal de $A$. Se $1_{A}\in I$, ent{\~a}o $I=A$\end{proposicao}

\textbf{Demonstra{\c c}{\~a}o}: Como $I$ {\'e} ideal, $1_{A}x\in I$, para todo $x\in A$, ou seja, $x=1_{A}x\in I$ para qualquer $x\in A$, logo, $A\subseteq I$. Como $I\subseteq A$, ent{\~a}o $I=A$.\#

ACRESCENTAR PROPRIEDADE $-a \in I$, $a-b \in I$.

Exemplos:
\begin{enumerate}
\item Em $\mathbb{Z}$ todos os ideais n{\~a}o triviais s{\~a}o da forma $m\mathbb{Z},m>1$
\item No anel $\displaystyle\frac{\mathbb{Z}}{p\mathbb{Z}}$, onde $p$ {\'e} um n{\'u}mero primo, os {\'u}nicos ideais  s{\~a}o os triviais $\{\bar{0}\}$ e $\displaystyle\frac{\mathbb{Z}}{p\mathbb{Z}}$.

\textbf{Demonstra{\c c}{\~a}o}: Seja $I\subseteq\displaystyle\frac{\mathbb{Z}}{p\mathbb{Z}}$ um ideal, $I\neq\{\bar{0}\}$. Provemos que 
$I=\displaystyle\frac{\mathbb{Z}}{p\mathbb{Z}}$. Para isso, vamos provar que $\bar{1}\in I$. Seja $\bar{a}\in I, \bar{a}\neq \bar{0}$, pois $I\neq\{\bar{0}\}$. Mas como $p$ {\'e} primo, $mdc(a,p)=1$, da{\'\i} existe $\bar{b}\in\displaystyle\frac{\mathbb{Z}}{p\mathbb{Z}},\bar{b}\neq\bar{0}$, tal que $\bar{1}=\bar{a}\bar{b}$. Mas $I$ {\'e} ideal e $\bar{a}\in I$, logo $\bar{1}=\bar{a}\bar{b}\in I$.

Portanto $I=\displaystyle\frac{\mathbb{Z}}{p\mathbb{Z}}$.\#
\item Os {\'u}nicos ideais n{\~a}o triviais de $\displaystyle\frac{\mathbb{Z}}{8\mathbb{Z}}=\{\bar{0},\bar{1},\bar{2},\bar{3},\bar{4},\bar{5},\bar{6},\bar{7}\}$ s{\~a}o:\\
$I_{1}=\{\bar{0},\bar{2},\bar{4},\bar{6}\}$ e $I_{2}=\{\bar{0},\bar{4}\}$ 

\end{enumerate}

Observa{\c c}{\~a}o: Num anel $(A,+,.)$, a diferen{\c c}a $a-b$ {\'e} definida como
\[a-b=a+(-b),\ a,b\in A\]

\subsection{Congru{\^e}ncia m{\'o}dulo I}
\subsubsection{Defini{\c c}{\~a}o}
\begin{definicao}[Congru{\^e}ncia m{\'o}dulo I] Seja $I$ um ideal de um anel $(A,+,.)$. Dizemos que $x$ {\'e} congruente a $y$ m{\'o}dulo $I$ quando $x-y\in I$. Neste caso, escrevemos $x\equiv y(mod\ I)$.\end{definicao}

\subsubsection{Propriedades}
\begin{proposicao} A congru{\^e}ncia M{\'o}dulo $I$ {\'e} uma rela{\c c}{\~a}o de equival{\^e}ncia em $A \ times A$($A$ anel unit{\'a}rio).\end{proposicao}

\textbf{Demonstra{\c c}{\~a}o}: Como $0=0_{A}\in I$ e para todo $x\in I$, $x-x=0\in I$, ent{\~a}o $x\equiv x(mod\ I)$.

Suponha que $x\equiv y(mod\ I)$. Ent{\~a}o $x-y\in I$. Como $-1\in A$,$y-x=-(x-y)=-[(x-y)1]=(x-y)(-1)\in I$, ou seja, $y\equiv x(mod\ I)$.

Agora,s e $x\equiv y(mod\ I)$ e $y\equiv z(mod\ I)$, ent{\~a}o $x-y\in I$ e $y-z\in I$. Da{\'\i}, $x-z=(x-z)+(y-z)\in I$, ou seja, $x\equiv z(mod\ I)$.

Logo, {\'e} uma rela{\c c}{\~a}o de equival{\^e}ncia.\#

Seja $y\in A$. A classe de equival{\^e}ncia m{\'o}dulo $I$ {\'e}
\[C(y)=\{x\in A/x\equiv y(mod\ I)\}=\{x\in A/x-y\in I\}\]


Agora, $x-y\in I$ significa que existe $t\in I$, tal que $x-y=t$. Logo, $x=y+t$, onde $t\in I$.

Assim,
\[C(y)=\{y+t,t\in I\}=y+I\]

\begin{nota}[Congru{\^e}ncia M{\'o}dulo I] Denotamos por $y+I$ (ou $I+y$) a classe de equival{\^e}ncia m{\'o}dulo $I$. Denotamos por $\displaystyle\frac{A}{I}$ o conjunto de todas as classes de equival{\^e}ncia, tal conjunto {\'e} chamado quociente do anel $A$ pelo ideal $I$.\end{nota}

Exemplos:
\begin{enumerate}
\item A anel e $I_{1}=\{0\}$ e $I_{2}=A$ ideais.
\begin{enumerate}
\item $\displaystyle\frac{A}{I_{1}};a\in A$\\
$C(a)=a+I_{1}=\{a+0\}=\{a\}$\\
$\displaystyle\frac{A}{I_{1}}=\{a+I, a\in A\}$\\
Tantas classes de equival{\^e}ncia quantos elementos em $A$
\item $\displaystyle\frac{A}{I_{2}}; a\in A, I_{2}=A$\\
$C(a)=a+I=\{a+t/t\in I_{2}\}$\\
$C(0_{A})=0_{A}+I]\{0_{A}+t/t\in I_{2}\}$\\
$0_{A}+I=\{t/t\in I_{2}=A\}$\\
$\displaystyle\frac{A}{I_{2}}=\{0_{A}+I\}$\\
Apenas uma classe de equival{\^e}ncia
\end{enumerate}
\item Seja $A=\mathbb{Z}$. Sabemos que os ideais de $\mathbb{Z}$ s{\~a}o da forma $m\mathbb{Z},m>1$. Seja $I=m\mathbb{Z}$ um ideal de $\mathbb{Z}$. Ent{\~a}o\\
\[x\equiv y(mod\ I)\Leftrightarrow x-y\in I \Leftrightarrow x-y=mk,k\in\mathbb{Z}\Leftrightarrow m|(x-y)
\Leftrightarrow x\equiv y(mod\ m)\]

Portanto, $\displaystyle\frac{\mathbb{Z}}{I}=\displaystyle\frac{\mathbb{Z}}{m\mathbb{Z}}$.
\end{enumerate}

Agora seja $I$ ideal e $A$ anel. \[\displaystyle\frac{A}{I}\{y+I/y\in A\}\] \[y+I=\{y+t/t\in I\}\]

Vamos definir uma soma $\oplus$ e um produto $\odot$ em $\displaystyle\frac{A}{I}$ por \[(x+I)\oplus(y+I)=(x+y)+I\] \[(x+I)\odot(y+I)=(xy)+I\]

Verifiquemos que a soma e o produto em $\displaystyle\frac{A}{I}$ n{\~a}o dependem do representante da classe de equival{\^e}ncia. Dados $x+I, x_{1}+I,y+I,y_{1}+I\in\displaystyle\frac{A}{I}$ tais que \[x+I=x_{1}+I\] \[y+I=y_{1}+I\]

Ent{\~a}o
\[(x+I)\oplus(y+I)=(x+y)+I\]
\[(x_{1}+I)\oplus(y_{1}+I)=(x_{1}+y_{1})+I\]

Como $x+I=x_{1}+I$, ent{\~a}o $x-x_{1}\in I$ e como $y+I=y_{1}+I$, ent{\~a}o $y=y_{1}\in I$. Mas $I$ {\'e} ideal, logo $(x-x_{1})+(y-y_{1})=(x+y)-(x_{1}+y_{1})\in I$, ou seja \[(x+I)\oplus(y+I)=(x_{1}+I)\oplus(y_{1}+I)\]

Agora, \[(x+I)\odot(y+I)=(xy)+I\] \[(x_{1}+I)\odot(y_{1}+I)=(x_{1}y_{1})+I\]

Como $(x-x_{1})y\in I$ e $(y-y_{1})x_{1}\in I$. Logo, \[(x-x_{1})y+(y-y_{1})x_{1}\in I\] \[xy-\underbrace{x_{1}y+yx_{1}}_{=0}-y_{1}x_{1}\in I\]
\[xy-x_{1}y_{1}\in I\], ou seja, $xy+I+x_{1}y_{1}+I$. Portanto, \[(x+I)\odot(y-I)=(x_{1}+I)\odot(y_{1}+I)\]

\begin{teo} Seja $(A,+,.)$ um anel associativo, comutativo e com unidade. Ent{\~a}o, se $I$ {\'e} um ideal de $A$, o quociente $\displaystyle\frac{A}{I}$ com as opera{\c c}{\~o}es $\oplus$ e $\odot$ {\'e} um anel associativo, comutativo e com unidade. O elemento zero desse anel {\'e} a classe $0_{A}+I$ e o elemento um de $\displaystyle\frac{A}{I}$ {\'e} $1_{A}+I$. \end{teo}

\chapter{Grupos}

\section{Defini{\c c}{\~a}o}
\begin{definicao}[Grupo] Um grupo $G$ {\'e} um conjunto n{\~a}o vazio munido com uma opera{\c c}{\~a}o bin{\'a}ria $*$ tal que
\begin{enumerate}
\item Para todo $x,y,z\in G,\ (x*y)*z=x*(y*z)$ (Associatividade)
\item Existe $e\in G$ tal que $x*e=e*x=x$ para todo $x\in G$. Tal elemento $E$ {\'e} chamado de elemento neutro ou unidade
\item Para cada $x\in G$, existe $x^{-1}\in G$ tal que $x*x^{-1}=x^{-1}*x=e$. O elemento $x^{-1}$ {\'e} chamado de inverso\footnote{$x^{-1}\neq\displaystyle\frac{1}{x}$} de $x$
\end{enumerate}
\end{definicao}

Denotamos um grupo $G$, cuja opera{\c c}{\~a}o bin{\'a}ria {\'e} $*$, por $(G,*)$. Quando $*$ {\'e} a soma, dizemos que $(G,*)$ {\'e} um grupo aditivo. Se $*$ {\'e} a multiplica{\c c}{\~a}o, dizemos que $(G,*)$ {\'e} um grupo multiplicativo.\\

\section{Grupo comutativo ou abeliano}
\begin{definicao}[Grupo comutativo ou abeliano] Um grupo $(G,*)$ {\'e} chamado de grupo comutativo ou abeliano quando $*$ {\'e} comutativa, ou seja, \[x*y=y*x\] para todo $x,y\in G$
\end{definicao}
\vspace{1cm}

Exemplos:
\begin{enumerate}
\item $(\mathbb{Z},+$) {\'e} um grupo abeliano
\item Se $(A,+,.)$ {\'e} um anel, ent{\~a}o $(A,+)$ {\'e} um grupo
\item $\left(\displaystyle\frac{\mathbb{Z}}{m\mathbb{Z}},\oplus\right)$ {\'e} grupo
\item $\left(\displaystyle\frac{\mathbb{Z}}{m\mathbb{Z}}-\{\bar{0}\},\odot\right)$ {\'e} grupo?\\
$\displaystyle\frac{\mathbb{Z}}{4\mathbb{Z}}-\{\bar{0}\}=\{\bar{1},\bar{2},\bar{3}\}=G$\\
$\bar{2}\in G,\ \bar{2}\odot\bar{2}=\bar{0}\notin G$
\item $\left(U\left(\displaystyle\frac{\mathbb{Z}}{m\mathbb{Z}}\right),\odot\right)$ {\'e} um grupo\\
\item Considere o conjunto dos n{\'u}meros reais $\mathbb{R}$ com a opera{\c c}{\~a}o $*$ definida por \[x*y=x+y-3\], $x,y\in\mathbb{R}$. Ent{\~a}o $(\mathbb{R},*)$ {\'e} um grupo abeliano.

De fato
\begin{enumerate}
\item \[(x*y)*g=(x+y-3)*g=(x+y-3)+z-3\]
\[=x+(y-3+z)-3=x+(y+z-3)-3)=x*(y+z-3)\]
\[=x*(y*3)\] para todo $x,y,z\in \mathbb{R}$
\item $x*y=x+y-3=y+x-3=y*x$ para todo $x,y\in\mathbb{R}$. Logo, $*$ {\'e} comutativa
\item Para todo $x\in\mathbb{R}$, temos $x*3=x+3-3=x$. Logo, 3 {\'e} o elemento neutro de $*$.
\item Dado $x\in\mathbb{R}$, tome $x^{-1}=6-x$. Assim \[x*x^{-1}=x+(6-x)-3=3\]

Logo, para $x\in\mathbb{R}$ o inverso de $x$ por $*$ {\'e} $6-x$.

Portanto $(\mathbb{R}, *)$ {\'e} um grupo comutativo
\end{enumerate}
\item Considere um conjunto com dois elementos $G=\{x,y\}$. em $G$, considere a opera{\c c}{\~a}o $\triangle$ dada por (Tabela \ref{6})
\begin{table}[h]
   \centering 
   \setlength{\arrayrulewidth}{0,5\arrayrulewidth}
   \caption{\it Opera{\c c}{\~a}o $\triangle$}
   \begin{tabular}{|c|c|c|c|c|} 
      \hline
      $\triangle$ & $x$ & $y$ \\
     \hline
      $x$ & $x$ & $y$ \\
      \hline
      $y$ & $y$ & $x$ \\
      \hline
   \end{tabular}
\label{6}
\end{table}

$(G,\triangle)$ {\'e} um grupo?

$(x\triangle x)\triangle y= x\triangle(x\triangle y)$

$(x\triangle y)\triangle x=x\triangle(y\triangle x)$

$\vdots$

$\triangle$ {\'e} associativa

$x$ {\'e} neutro para $\triangle$\\
$x^{-1}=x$\\
$y^{-1}=y$

Logo, $(G,\triangle)$ {\'e} um grupo
\end{enumerate}

\section{Propriedades Imediatas de um grupo}

Seja $(G,*)$ um grupo. {\'E} f{\'a}cil ver que
\begin{enumerate}
\item O elemento neutro {\'e} {\'u}nico
\item Existe um {\'u}nico inverso para cada $x\in G$
\item Para todos $x,y\in G,(x*y)^{-1}=y^{-1}*x^{-1}$. Por indu{\c c}{\~a}o, $x_{1},x_{2},...,x_{n-1},x_{n}\in G$, \[(x_{1}*x_{2}*...*x_{n-1}*x_{n})^{-1}\] \[=x^{-1}_{n}*x^{-1}_{n-1}*...*x^{-1}_{2}*x^{-1}_{1}\]
\item Para todo $x\in G, (x^{-1})^{-1}=x$

\end{enumerate}

\section{Ordem de um Grupo}
\begin{definicao}[Ordem de um grupo]
Quando um grupo $(G,*)$, $G$ {\'e} um conjunto com um n{\'u}mero finito de elementos, dizemos que $G$ {\'e} um grupo finito. Denotamos por $|G|$ o n{\'u}mero de elementos de $G$ que ser{\'a} chamado de ordem de $G$ ou cardinalidade de $G$. Quando $G$ n{\~a}o {\'e} finito, dizemos que $G$ {\'e} um grupo infinito.
\end{definicao}

Exemplos:
\begin{enumerate}
\item $\left(\dfrac{\mathbb{Z}}{m\mathbb{Z}}, *\right)$ {\'e} um grupo finito para todo $m>1$
\item $(\mathbb{Z}, +)$ {\'e} um grupo infinito
\end{enumerate}

\section{Subgrupo}
\subsubsection{Defini{\c c}{\~a}o}

\begin{definicao}[Subgrupo]
Seja $(G,*)$ um grupo. Um subconjunto n{\~a}o vazio $H\subseteq G$ {\'e} um subgrupo se, e somente se, $(H,*)$ {\'e} um grupo.
\end{definicao}

\subsubsection{Propriedades}
\begin{proposicao}
Um subconjunto n{\~a}o vazio $H\subseteq G$ {\'e} um subgrupo de $G$ se, e somente se
\begin{enumerate}
\item $x^{-1}\in H,\forall x\in H$
\item $x*y\in H,\forall x,y\in H$
\end{enumerate}
\end{proposicao}

\textbf{Demonstra{\c c}{\~a}o}: Se $H$ {\'e} subgrupo, ent{\~a}o $H$ {\'e} um grupo. Logo 1 e 2 s{\~a}o satisfeitos.

Agora provemos que se $H$ satisfaz 1 e 2, ent{\~a}o $H$ {\'e} grupo.

Como $G$ {\'e} grupo, ent{\~a}o $*$ {\'e} associativo, logo $*$ {\'e} associativo em $H$.

De 1, $\forall x\in H,x^{-1}\in H$. Mas de 2, $\forall x,y\in H,\ x*y\in H$. Logo, se $x\in H$, ent{\~a}o $e=x*x^{-1}\in H$

Novamente por 1, todo elemento de $H$ possui inverso em $H$.

Logo, $(H,*)$ {\'e} um grupo.\#

Exemplos:
\begin{enumerate}
\item Dado $(G,*)$ grupo, $H=\{e\}$ e $H=G$ s{\~a}o subgrupos de $G$, chamados de subgrupos triviais
\item $(\mathbb{Z},+),\ H=m\mathbb{Z},\ m>1$

Ent{\~a}o $H$ {\'e} subgrupo de $\mathbb{Z}$
\item $G=U\left(\dfrac{\mathbb{Z}}{8\mathbb{Z}}\right)=\{\bar{1},\bar{3},\bar{5},\bar{7}\}$

$(G,\odot)$ {\'e} um grupo

$|G|$=4

$H_{1}=\{\bar{1},\bar{3}\}$ {\'e} subgrupo de G\\
$H_{2}=\{\bar{1},\bar{5}\}$ {\'e} subgrupo de G\\
$H_{3}=\{\bar{1},\bar{7}\}$ {\'e} subgrupo de G
\end{enumerate}

\section{Ordem de um subgrupo}

\begin{teo}[Lagrange]
Seja $G$ um grupo finito. Se $H\subseteq G$ {\'e} um subgrupo, ent{\~a}o $|H|$ divide $|G|$.
\end{teo}

Exemplo: Quais s{\~a}o as poss{\'\i}veis ordens dos subgrupos de um grupo de ordem 48?

Seja $G$ um grupo tal que $|G|=48$. Se $H$ {\'e} um subgrupo de $G$, ent{\~a}o $|H|$ divide $|G|$\\
$48=2^{4}3$ \\
$|H|=2,3,2^{2},2^{3},2^{4},2.3,2^{2}3,2^{2}3$

Observa{\c c}{\~a}o: O teorema n{\~a}o diz que haver{\'a} um subgrupo de ordem $n$ para todo $n$ tal que $n||G|$. Diz apenas que se $H$ {\'e} subgrupo de $G$, ent{\~a}o $|H|$ divide $|G|$.

\begin{cor}
Os {\'u}nicos subgrupos de um grupo de ordem prima s{\~a}o os triviais
\end{cor}

\textbf{Demonstra{\c c}{\~a}o}: Quando $|G|=p$ primo, temos que os {\'u}nicos divisores de $p$ positivos s{\~a}o 1 e $p$.

Ent{\~a}o, se $H$ {\'e} subgrupo de $G$, ent{\~a}o $|H|=1$ ou $|H|=p$.

Portanto, $H=\{e\}$ ou $H=G$.\#




\chapter{Bibliografia}
\begin{enumerate}
\item H.H. Domingues, G.Iezzi: \textit{{\'A}lgebra Moderna}, 2� Ed., Atual, 1982
\item S. Shokranian: \textit{{\'A}lgebra 1}, Ci{\^e}ncia Moderna, 2010
\item Adilson Gon{\c c}alves: \textit{Introdu{\c c}{\~a}o {\`a} {\'A}lgebra}, 5� Ed., IMPA, 2003
\item G. Birkhoff, S. MacLane: \textit{{\'A}lgebra Moderna B{\'a}sica}, 4� Ed., Guanabara Dois, 1980
\item E. A. Filho: \textit{Inicia{\c c}{\~a}o {\`a} L{\'o}gica Matem{\'a}tica}, Nobel, 2002

\end{enumerate}







\printindex

\end{document}
