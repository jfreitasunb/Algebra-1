%!TEX program = xelatex
%!TEX encoding = ISO-8859-1
\documentclass[portuguese,twoside,12pt]{report}%opcao draft remove os links
\usepackage{amssymb,amsmath,amsfonts,amsthm,amstext}
\usepackage[brazil]{babel}
\usepackage{inputenc}
\usepackage{fancybox}
\usepackage{niceframe}
\usepackage{hyperref}
\usepackage{graphicx}
\graphicspath{{/Users/jfreitas/Dropbox/imagens-latex/}{/home/jfreitas/Dropbox/imagens-latex/}}
\usepackage{makeidx}

%=============================================================================================
%             Nomes para defini{\c c}{\~o}es, teoremas, etc
%=============================================================================================

\newtheorem{teorema}{Teorema}[chapter]
\newtheorem{definicao}{Defini{\c c}{\~a}o}[chapter]
\newtheorem{nota}{Nota{\c c}{\~a}o}[definicao]
\newtheorem{proposicao}{Proposi{\c c}{\~a}o}[definicao]
\newtheorem{lema}{Lema}[proposicao]
\newtheorem{corolario}{Corol{\'a}rio}[proposicao]

%=============================================================================================
%             Cabe{\c c}alhos
%=============================================================================================

\usepackage{fancyhdr}
\pagestyle{fancy} \addtolength{\headwidth}{\marginparsep}
\addtolength{\headwidth}{\marginparwidth}
\renewcommand{\headrulewidth}{1pt}
\renewcommand{\chaptermark}[1]{\markboth{{CAP. \thechapter\ $\bullet$ \; #1}}{}}
\renewcommand{\sectionmark}[1]{\markright{{SE{\c C}{\~A}O \thesection\ $\bullet$ \; #1}}}
\fancyhf{} \fancyhead[RO,RE]{\small\bfseries\textrm \thepage}
\fancyhead[LO]{\small\bfseries\textrm \leftmark}
\fancyhead[LE]{\small\bfseries\textrm \rightmark}

%=============================================================================================
%             Estilo dos t{\'\i}tulos de cap{\'\i}tulo, se{\c c}{\~a}o, etc
%=============================================================================================
\usepackage{sectsty}
\usepackage[Conny]{fncychap}
\sectionfont{\rmfamily\raggedright\sectionrule{0.3in}{1pt}{-0.1in}{1pt}}
\subsectionfont{\rmfamily\raggedright}
\chapterfont{\thispagestyle{empty}}
\ChNameVar{\Huge\rm\bfseries} \ChTitleVar{\Huge\rm\bfseries}

%=============================================================================================
%             Medidas
%=============================================================================================


\setlength{\headsep}{1cm}                    % DIST{\^A}NCIA TEXTO/CABE{\c C}ALHO
\setlength{\textwidth}{16.5 cm}              % LARGURA DO TEXTO
\setlength{\textheight}{21.5 cm}             % ALTURA DO TEXTO
\setlength{\oddsidemargin}{0.1 cm}           % MARGEM {\'I}MPAR
\setlength{\evensidemargin}{0.4 cm}          % MARGEM PAR
\setlength{\topmargin}{0 cm}                 % MARGEM SUPERIOR
\renewcommand{\baselinestretch}{1.4}         % DIST{\^A}NCIA ENTRE LINHAS

%=============================================================================================
%             Comandos Pessoais
%=============================================================================================

\newcommand{\sub}{\subseteq}
\newcommand{\real}{\mathbb{R}}
%=============================================================================================


\makeindex



\title{\Huge \textbf{{\'A}lgebra 1}\\ \vspace{0,5cm} \Large Notas de Aula}
\author{\Large Departamento de Matem{\'a}tica\\ \\ \large Universidade de Bras{\'\i}lia - UnB}
\date{2/2010}

\begin{document}
\maketitle

\begin{center}
\Huge \textbf{Pref{\'a}cio}
\end{center}
\hspace{0,5cm}Essas notas de Aula s{\~a}o referentes {\`a} mat{\'e}ria {\'A}lgebra 1,
ministrada na UnB - Universidade de Bras{\'\i}lia - durante o 2� Semestre de 2010
pelo professor Jos{\'e} Ant{\^o}nio de O. Freitas, Departamento de Matem{\'a}tica. Tais
notas foram transcritas e editadas pelo graduando em Ci{\^e}ncias Econ{\^o}micas
Luiz Eduardo Sol R. da Silva\footnote{luizeduardosol@hotmail.com}.

{\'E} livre a reprodu{\c c}{\~a}o, distribui{\c c}{\~a}o e edi{\c c}{\~a}o deste material, desde que citadas as suas fontes e autores. Cr{\'\i}ticas e sugest{\~o}es s{\~a}o bem vindas.
\vspace{20cm}




\begin{center} \textbf{\Large Nota{\c c}{\~o}es e express{\~o}es}
\end{center}
\begin{minipage}[l]{0,5\textwidth}
\begin{itemize}
\item $\neg$ N{\~a}o
\item $\forall$ Para todo
\item $/$ Tal que
\item $|$ Divide
\item $\Rightarrow$ Implica
\item $\in$ Pertence
\item $\emptyset$ Vazio
\item $\subseteq$ Contido ou igual a
\item $\supseteq$ Cont{\'e}m ou igual a
\item $\wedge$ E
\item $\vee$ Ou
\item $=$ Igual
\item $\neq$ Diferente
\item $\mathbb{Z}$ N{\'u}meros Inteiros
\item $\mathbb{R}$ N{\'u}meros Reais
\item $\cap$ Intersec{\c c}{\~a}o
\item $>$ Maior que
\item $\geq$ Maior ou igual a
\item $\displaystyle\bigcup_{i=1}^{n}$ Uni{\~a}o de $n$ conjuntos
\item $\displaystyle\bigsqcup_{i=1}^{n}$ Uni{\~a}o disjunta de $n$ conjuntos


\end{itemize}
\end{minipage}
\begin{minipage}[r]{0,5\textwidth}
\begin{itemize}

\item $\leftrightarrow$ Se, e somente se
\item $\veebar$ Ou...,ou..., mas nunca ambos
\item $\rightarrow$ Se,... ent{\~a}o...
\item $\exists$ Existe
\item $\Leftrightarrow$ Equivalente a
\item $\notin$ N{\~a}o pertence
\item \# Fim da demonstra{\c c}{\~a}o
\item $\mathbb{N}$ N{\'u}meros Naturais
\item $\mathbb{Q}$ N{\'u}meros Racionais
\item $\nsubseteq$ N{\~a}o cont{\'e}m ou {\'e} igual a
\item $\cup$ Uni{\~a}o
\item $\sqcup$ Uni{\~a}o Disjunta
\item $<$ Menor que
\item $\leq$ Menor ou igual a
\item $\displaystyle\bigcap_{i=1}^{n}$ Intersec{\c c}{\~a}o de $n$ conjuntos
\item Q.E.D. (\textit{Quod Erat Demonstrandum}): Como se queria demonstrar
\item P.B.O.: Princ{\'\i}pio da boa ordena{\c c}{\~a}o
\item H.I.: Hip{\'o}tese de Indu{\c c}{\~a}o
\item \textit{Mutatis Mutandis}:  Mudando o que tem que ser mudado

\end{itemize}
\end{minipage}

\tableofcontents




\chapter{Bibliografia}
\begin{enumerate}
\item H.H. Domingues, G.Iezzi: \textit{{\'A}lgebra Moderna}, 2� Ed., Atual, 1982
\item S. Shokranian: \textit{{\'A}lgebra 1}, Ci{\^e}ncia Moderna, 2010
\item Adilson Gon{\c c}alves: \textit{Introdu{\c c}{\~a}o {\`a} {\'A}lgebra}, 5� Ed., IMPA, 2003
\item G. Birkhoff, S. MacLane: \textit{{\'A}lgebra Moderna B{\'a}sica}, 4� Ed., Guanabara Dois, 1980
\item E. A. Filho: \textit{Inicia{\c c}{\~a}o {\`a} L{\'o}gica Matem{\'a}tica}, Nobel, 2002

\end{enumerate}







\printindex

\end{document}
