%!TEX program = xelatex 
%!TEX encoding = ISO-8859-1
\documentclass[portuguese,twoside,12pt]{book}%opcao draft remove os links
\usepackage{amssymb,amsmath,amsfonts,amsthm,amstext}
\usepackage[brazil]{babel}
% \usepackage[latin1]{inputenc}
\usepackage{fancybox}
%\usepackage{niceframe}
\usepackage[unicode,bookmarks=true]{hyperref}
%\usepackage[pdftex]{graphicx}
\usepackage{graphicx}
\usepackage{makeidx}
\usepackage{enumitem}
\usepackage{multicol}
\usepackage{tikz,circuitikz,siunitx}
\usetikzlibrary{scopes}
\usepackage{pstricks-add,pst-coil}
\usepackage{ccicons}
\usepackage{hyperref}
\usepackage{xcolor}
\usepackage{titlesec}
\usepackage{gauss}
%\usepackage{datetime}
%\usepackage[active,tightpage]{preview}

\setcounter{secnumdepth}{3}
\setcounter{tocdepth}{3}

%=============================================================================================
%             Nomes para defini{\c c}{\~o}es, teoremas, etc
%=============================================================================================

\newtheorem{teorema}{Teorema}[chapter]
\newtheorem{definicao}{Defini{\c c}{\~a}o}[chapter]
\newtheorem{notacao}{Nota\c{c}\~ao}[definicao]
\newtheorem{proposicao}{Proposi{\c c}{\~a}o}[definicao]
\newtheorem{lema}{Lema}[proposicao]
\newtheorem{corolario}{Corol{\'a}rio}[teorema]
\newtheorem{observacao}{Observa{\c c}{\~a}o}[definicao]
\newtheorem{propriedades}{Propriedades}[definicao]
\newtheorem{exemplo}{Exemplo}[definicao]
\newtheorem{exemplos}{Exemplos}[definicao]
\newenvironment{prova}[1][Prova]{\noindent\textbf{#1:} }{\hfill$\diamondsuit$}%{\ \rule{0.5em}{0.5em}}
\newtheoremstyle{dotless}{}{}{\itshape}{}{\bfseries}{}{ }{}
\theoremstyle{dotless}
\newtheorem*{solucao}{Solu{\c c}{\~a}o:}
%=============================================================================================
%             Cabe{\c c}alhos
%=============================================================================================

\usepackage{fancyhdr}
\pagestyle{fancy} \addtolength{\headwidth}{\marginparsep}
\addtolength{\headwidth}{\marginparwidth}
\renewcommand{\headrulewidth}{1pt}
\renewcommand{\chaptermark}[1]{\markboth{{CAP. \thechapter\ $\bullet$ \; #1}}{}}
\renewcommand{\sectionmark}[1]{\markright{{SE{\c C}{\~A}O \thesection\ $\bullet$ \; #1}}}
\fancyhf{} \fancyhead[RO,RE]{\small\bfseries\textrm \thepage}
\fancyhead[LO]{\small\bfseries\textrm \leftmark}
\fancyhead[LE]{\small\bfseries\textrm \rightmark}

%=============================================================================================
%             Estilo dos t{\'\i}tulos de cap{\'\i}tulo, se{\c c}{\~a}o, etc
%=============================================================================================
\usepackage{sectsty}
\usepackage[Conny]{fncychap}
\sectionfont{\rmfamily\raggedright\sectionrule{0.3in}{1pt}{-0.1in}{1pt}}
\subsectionfont{\rmfamily\raggedright}
\chapterfont{\thispagestyle{empty}}
\ChNameVar{\Huge\rm\bfseries} \ChTitleVar{\Huge\rm\bfseries}

%=============================================================================================
%             Medidas
%=============================================================================================


\setlength{\headsep}{1cm}                    % DIST{\^A}NCIA TEXTO/CABE{\c C}ALHO
\setlength{\textwidth}{16.5 cm}              % LARGURA DO TEXTO
\setlength{\textheight}{21.5 cm}             % ALTURA DO TEXTO
\setlength{\oddsidemargin}{0.1 cm}           % MARGEM {\'I}MPAR
\setlength{\evensidemargin}{0.4 cm}          % MARGEM PAR
\setlength{\topmargin}{0 cm}                 % MARGEM SUPERIOR
\renewcommand{\baselinestretch}{1.4}         % DIST{\^A}NCIA ENTRE LINHAS

%=============================================================================================
%             Comandos Pessoais
%=============================================================================================
\newcommand{\im}{{\rm Im\,}}
\newcommand{\aut}{{\rm Aut\,}}
\newcommand{\cp}[1]{\mathbb{#1}}
\newcommand{\sub}{\subseteq}
\newcommand{\n}{\mathbb{N}}
\newcommand{\integer}{\mathbb{Z}}
\newcommand{\rac}{\mathbb{Q}}
\newcommand{\real}{\mathbb{R}}
\newcommand{\complex}{\mathbb{C}}
\newcommand{\lap}[1]{\mathcal{L}\left\{#1\right\}}
\newcommand{\lapi}[1]{\mathcal{L}^{-1}\left\{#1\right\}}
\newcommand{\se}[1]{\displaystyle\sum_{n = 1}^\infty{#1}}
\newcommand{\dlim}[2]{\displaystyle\lim_{#1\rightarrow #2}}
\newcommand{\slim}{\displaystyle\lim_{n \rightarrow \infty}}
\newcommand{\seq}[1]{\{{#1_n\}}}
\newcommand{\seg}[1]{\displaystyle\sum_{n = 1}^\infty{#1_n}}
\newcommand{\sei}[2]{\displaystyle\sum_{#1}^\infty{#2}}
\newcommand{\sepc}[3]{\displaystyle\sum_{#1}^\infty{#2(x - #3)^n}}
\newcommand{\imp}[3]{\displaystyle\int_{#1}^{+\infty}{#3}{d #2}}
\newcommand{\dint}[4]{\displaystyle\int_{#1}^{#2}{#4}{d#3}}
\newcommand{\inti}[2]{\displaystyle\int{#1}{d#2}}
\newcommand{\norma}[1]{\left\lVert#1\right\rVert}
\newcommand{\flim}[1]{\displaystyle\lim_{#1\rightarrow \infty}}
\renewcommand{\sin}{{\rm sen\,}}
\renewcommand{\tan}{{\rm tg\,}}
\renewcommand{\csc}{{\rm cossec\,}}
\renewcommand{\cot}{{\rm cotg\,}}
\renewcommand{\sinh}{{\rm senh\,}}


%=============================================================================================
%\'Indice remissivo em 3 colunas
%=============================================================================================
% \makeatletter
% \renewenvironment{theindex}
%  {\if@twocolumn
%   \@restonecolfalse
%  \else
%   \@restonecoltrue
%  \fi
%  \setlength{\columnseprule}{0pt}
%  \setlength{\columnsep}{35pt}
%  \begin{multicols}{3}[\section*{\indexname}]
%  \markboth{\MakeUppercase\indexname}%
%    {\MakeUppercase\indexname}%
%  \thispagestyle{plain}
%  \setlength{\parindent}{0pt}
%  \setlength{\parskip}{0pt plus 0.3pt}
%  \relax
%  \let\item\@idxitem}%
%  {\end{multicols}\if@restonecol\onecolumn\else\clearpage\fi}
% \makeatother

\makeindex
%=============================================================================================
%\'Indice remissivo em 3 colunas
%=============================================================================================
% \makeatletter
% \renewenvironment{theindex}
%  {\if@twocolumn
%   \@restonecolfalse
%  \else
%   \@restonecoltrue
%  \fi
%  \setlength{\columnseprule}{0pt}
%  \setlength{\columnsep}{35pt}
%  \begin{multicols}{3}[\section*{\indexname}]
%  \markboth{\MakeUppercase\indexname}%
%    {\MakeUppercase\indexname}%
%  \thispagestyle{plain}
%  \setlength{\parindent}{0pt}
%  \setlength{\parskip}{0pt plus 0.3pt}
%  \relax
%  \let\item\@idxitem}%
%  {\end{multicols}\if@restonecol\onecolumn\else\clearpage\fi}
% \makeatother

\makeindex



\begin{document}
%\listoftodos
%!TEX program = xelatex
%!TEX root = Algebra_1.tex
%%Usar makeindex -s indexstyle.ist arquivo.idx no terminal para gerar o {\'\i}ndice remissivo agrupado por inicial
%%Ap\'os executar pdflatex arquivo
\begin{titlepage}
\begin{center}
\vspace*{-1.5cm}

Álgebra 1


\vspace*{3.5cm}

{\fontsize{14pt}{14pt}\selectfont
   \textbf{Notas de Aula 2/2015}\footnote{\ccbyncsa\ Este texto est\'a licenciado sob uma \textbf{Licen\c{c}a Creative Commons Atribui\c{c}\~ao-N\~aoComercial-CompartilhaIgual 3.0 Brasil} \href{http://creativecommons.org/licenses/by-nc-sa/3.0/br/deed.pt\_BR}{\textit{http://creativecommons.org/licenses/by-nc-sa/3.0/br/deed.pt\_BR}}.}
   }


\vfill

{\fontsize{14pt}{14pt}\selectfont\textbf{Jos\'e Ant\^onio O. Freitas}\\ Departamento de Matem\'atica\\Universidade de Bras{\'\i}lia - UnB}
\end{center}
\end{titlepage}
\vspace*{-2cm}
\hypersetup{linkcolor=blue}
\tableofcontents
% \listoffigures
\hypersetup{linkcolor=red}
%!TEX program = xelatex
%!TEX root = Algebra1.tex
%%Usar makeindex -s indexstyle.ist arquivo.idx no terminal para gerar o {\'\i}ndice remissivo agrupado por inicial
%%Ap\'os executar pdflatex arquivo
\begin{center}
\Huge \textbf{Pref{\'a}cio}
\end{center}
\hspace{0,5cm}Essas notas de Aula s{\~a}o referentes {\`a} mat{\'e}ria {\'A}lgebra 1,
ministrada na UnB - Universidade de Bras{\'\i}lia - durante o 2º Semestre de 2010
pelo professor Jos{\'e} Ant{\^o}nio de O. Freitas, Departamento de Matem{\'a}tica. Tais
notas foram transcritas e editadas pelo graduando em Ci{\^e}ncias Econ{\^o}micas
Luiz Eduardo Sol R. da Silva\footnote{luizeduardosol@hotmail.com}.

{\'E} livre a reprodu{\c c}{\~a}o, distribui{\c c}{\~a}o e edi{\c c}{\~a}o deste material, desde que citadas as suas fontes e autores. Cr{\'\i}ticas e sugest{\~o}es s{\~a}o bem vindas.
\vspace{20cm}




\begin{center} \textbf{\Large Nota{\c c}{\~o}es e express{\~o}es}
\end{center}
\begin{minipage}[l]{0,5\textwidth}
\begin{itemize}
\item $\neg$ N{\~a}o
\item $\forall$ Para todo
\item $/$ Tal que
\item $|$ Divide
\item $\Rightarrow$ Implica
\item $\in$ Pertence
\item $\emptyset$ Vazio
\item $\subseteq$ Contido ou igual a
\item $\supseteq$ Cont{\'e}m ou igual a
\item $\wedge$ E
\item $\vee$ Ou
\item $=$ Igual
\item $\neq$ Diferente
\item $\mathbb{Z}$ N{\'u}meros Inteiros
\item $\mathbb{R}$ N{\'u}meros Reais
\item $\cap$ Intersec{\c c}{\~a}o
\item $>$ Maior que
\item $\geq$ Maior ou igual a
\item $\displaystyle\bigcup_{i=1}^{n}$ Uni{\~a}o de $n$ conjuntos
\item $\displaystyle\bigsqcup_{i=1}^{n}$ Uni{\~a}o disjunta de $n$ conjuntos


\end{itemize}
\end{minipage}
\begin{minipage}[r]{0,5\textwidth}
\begin{itemize}

\item $\leftrightarrow$ Se, e somente se
\item $\veebar$ Ou...,ou..., mas nunca ambos
\item $\rightarrow$ Se,... ent{\~a}o...
\item $\exists$ Existe
\item $\Leftrightarrow$ Equivalente a
\item $\notin$ N{\~a}o pertence
\item \# Fim da demonstra{\c c}{\~a}o
\item $\mathbb{N}$ N{\'u}meros Naturais
\item $\mathbb{Q}$ N{\'u}meros Racionais
\item $\nsubseteq$ N{\~a}o cont{\'e}m ou {\'e} igual a
\item $\cup$ Uni{\~a}o
\item $\sqcup$ Uni{\~a}o Disjunta
\item $<$ Menor que
\item $\leq$ Menor ou igual a
\item $\displaystyle\bigcap_{i=1}^{n}$ Intersec{\c c}{\~a}o de $n$ conjuntos
\item Q.E.D. (\textit{Quod Erat Demonstrandum}): Como se queria demonstrar
\item P.B.O.: Princ{\'\i}pio da boa ordena{\c c}{\~a}o
\item H.I.: Hip{\'o}tese de Indu{\c c}{\~a}o
\item \textit{Mutatis Mutandis}:  Mudando o que tem que ser mudado

\end{itemize}
\end{minipage}
%%!TEX program = xelatex
%!TEX root = Algebra_1.tex
%%Usar makeindex -s indexstyle.ist arquivo.idx no terminal para gerar o índice remissivo agrupado por inicial
%%Após executar pdflatex arquivo
\chapter{Noções de Lógica}

\section{Conceitos básicos}

\hspace{0,5cm}O estudo da lógica proporciona instrumentos de pensamento para determinar a correão ou incorreão de todos os raciocínios.

A lógica pode não nos levar à verdade no sentido absoluto, mas nos permite descobrir a incoerência e o erro em um argumento.

\subsection{Proposição}

\begin{definicao}[Proposição]: Chama-se proposição todo conjunto de palavras ou símbolos que exprime um pensamento de sentido completo.
\end{definicao}

São exemplos de proposições:
\begin{enumerate}
\item A Lua é um satélite da Terra
\item $\pi>\sqrt{5}$
\end{enumerate}

\subsection{Valor de uma proposição}
\begin{definicao}[Valor de uma Proposição]Chama-se valor de uma proposição a verdade se a proposição é verdadeira e a falsidade se for falsa.\end{definicao}

\subsection{Princípios fundamentais}
A lógica matemática adota como regras fundamentais os dois seguintes princípios (ou axiomas):
\begin{enumerate}
\item \textbf{Princípio da não contradição}: uma proposição não pode ser verdadeira e falsa ao mesmo tempo
\item \textbf{Princípio do Terceiro excluído}: Toda proposição ou é verdadeira ou é falsa, isto é, verifica-se sempre um desses casos e nunca um terceiro
\end{enumerate}
\subsection{Argumento lógico}
\begin{definicao}[Argumento lógico]: Um argumento lógico é uma seqüência de proposições , na qual uma das seqüências é a conclusão e as demais, chamadas de premissas, formam as provas ou evidências para a conclusão.\end{definicao}

Exemplos:

\textbf{Argumento 1}: [Como todo brasileiro é sul americano](1ª proposição/ premissa) e [todo brasiliense é brasileiro](2ª proposição/premissa), então [todo brasiliense é brasileiro](3ª proposição/conclusão).

\textbf{Argumento 2}: [Como todo matemático é louco](1ª proposição/ premissa) e [eu sou matemático](2ª proposição/ premissa), então [eu sou louco](3ª proposição, conclusão).

Um argumento é válido quando suas proposições, se verdadeiras, fornecem provas convincenetes para sua conclusão, isto é, quando as proposições e a conclusão estão de tal modo relacionados que é absolutamente impossível as proposições serem verdadeiras se a conclusão não for.

\subsection{Proposições simples e compostas}

As proposições podem ser classificadas em simples (ou atômicas) e compostas.

Chama-se proposição simples aquela que não contém nenhuma outra proposição como parte integrante de si mesma.

Exemplos:
\begin{enumerate}
\item O número 25 é um quadrado perfeito
\item $\pi>\sqrt{5}$
\end{enumerate}

Chama-se proposição composta a que é formada pela combinaão de duas ou mais proposições.

Exemplo: $3>4$ ou $4>3$

\subsection{Conectivos}

\begin{definicao}[Conectivos]Chamam-se conectivos palavras que se usam para formar novas proposições a partir de outras.\end{definicao}

Em lógica matemática os conectivos usuais são os seguintes:
\begin{itemize}
\item ``E" (Conjunão)
\item ``OU" (Disjunão)
\item ``Não" (Negaão)
\item ``Se... então" (Condicional)
\item ``Se, e somente se..." (Bicondicional)
\end{itemize}

Vamos denotar as proposições simples por letras minúsculas (a,b,c...) e as proposições compostas por letras maiúsculas (A,B,C...)

Exemplo:\\
\textbf{$p$}: o número 6 é par\\
\textbf{$q$}: o numero 8 é um cubo perfeito\\
\textbf{$P$}: O número 6 é par E o número 8 é um cubo perfeito\\
\textbf{$Q$}: O número 6 é par OU 8 é um cubo perfeito

Se p é uma proposição, vamos denotar seu valor lógico por $V(p)$
\begin{center}
p: ou $V(p)=V$ ou $V(p)=F$
\end{center}

Vamos usar a notaão (Tabela \ref{notacao}), chamada Tabela Verdade (Tabela \ref{tabelavdd})
\begin{table}[h]
   \centering
   \setlength{\arrayrulewidth}{0,5\arrayrulewidth}

   \caption{\it Notaão}
   \begin{tabular}{|c|}
      \hline
      p \\
      \hline
      V \\
      \hline
      F \\
      \hline
   \end{tabular}
\label{notacao}
\end{table}


\begin{table}[h]
   \centering
   \setlength{\arrayrulewidth}{0,5\arrayrulewidth}
   \caption{\it Tabela Verdade}
   \begin{tabular}{|c|c|c|c|c|}
      \hline
      p & q \\
      \hline
      V & V \\
      \hline
      V & F\\
      \hline
      F & V \\
      \hline
      F & F \\
      \hline
   \end{tabular}
\label{tabelavdd}
\end{table}

\subsubsection{Negaão}
\begin{definicao}[Negaão] Chama-se negaão de uma proposição p a proposição representada por ``não p"($\neg p$), cujo valor lógico é verdade quando p for falsa e a falsidade quando p for verdadeira.\end{definicao}
\begin{table}[h]
   \centering
   \setlength{\arrayrulewidth}{0,5\arrayrulewidth}
   \caption{\it Tabela verdade: Negaão}
   \begin{tabular}{|c|c|c|c|c|}
      \hline
     $ p$ & $\neg p$ \\
      \hline
      V & F \\
      \hline
      F & V \\
      \hline
   \end{tabular}
\end{table}

Exemplo:\\
\textbf{$p$}: 6 é par, $V(p)=V$\\
\textbf{$\neg p$}: 6 é ímpar, $V(p)=F$\\
\textbf{$q$}: $4\leq 5$, $V(q)=V$\\
\textbf{$\neg q$}: $4>5$, $V(q)=F$

\subsubsection{Conjugaão}

\begin{definicao}[Conjugaão] Chama-se conjugaão de duas proposições $p$ e $q$, a proposição representada por ``$p$ e $q$", denotada $p\wedge q$, cujo valor lógico é verdade quando as proposições $p$ e $q$ são ambas verdadeiras e falsidade nos demais casos.\end{definicao}
\begin{table}[h]
   \centering
   \setlength{\arrayrulewidth}{0,5\arrayrulewidth}
   \caption{\it Tabela verdade: Conjugaão}
   \begin{tabular}{|c|c|c|c|c|}
      \hline
      $p$ & $q$ & $p\wedge q$ \\
      \hline
      V & V & V \\
      \hline
      V & F & F \\
      \hline
      F & V & F \\
      \hline
      F & F & F \\
      \hline
   \end{tabular}
\end{table}

Exemplo:\\
\textbf{$p$}: a neve é branca, $V(p)=V$\\
\textbf{$q$}: $2<5$, $V(q)=V$
\begin{center}
\textbf{$p\wedge q$}: A neve é branca E $2<5$, $V(p\wedge q)=V$
\end{center}

\textbf{$r$}: todo número primo e ímpar, $V(r)=F$
\begin{center}
$V(q\wedge r)=F$
\end{center}

\subsubsection{Disjunão}
\begin{definicao}[Disjunão]: Chama-se disjunão de duas proposições $p$ e $q$ a proposição representada por ``p ou q", denotada ``$p\vee q$", cujo valor lógico é verdade quando ao menos uma das proposições $p$ e $q$ forem verdadeiras, e falsidade quando ambas as proposições $p$ e $q$ forem falsas.\end{definicao}

A tabela verdade da disjunão é (Tabela \ref{disjuncao}):
\begin{table}[h]
   \centering
   \setlength{\arrayrulewidth}{0,5\arrayrulewidth}
   \caption{\it Tabela verdade: Disjunão}
   \begin{tabular}{|c|c|c|c|c|}
      \hline
      $p$ & $q$ & $p\vee q$ \\
     \hline
      V & V & V \\
      \hline
      V & F & V \\
      \hline
      F & V & V \\
      \hline
      F & F & F \\
      \hline
   \end{tabular}
\label{disjuncao}
\end{table}

Exemplo:\\
\textbf{$P$}: Roma é a capital da Rússia ou 9-5=4, $V(P)=V$\\
\textbf{$Q$}: $\pi=3\vee\sqrt{-1}=1$, $V(Q)=F$

\subsubsection{Disjunão exclusiva}
\begin{definicao}[Disjunão Exclusiva] Chama-se disjunão exclusiva de duas proposições $p$ e $q$ a proposição representada por $p\veebar q$, que se lê ``ou $p$ ou $q$", ou também ``$p$ ou $q$, mas não ambos", cujo valor lógico é a verdade somente quando $p$ é verdade ou $q$ é verdade, mas não quando $p$ e $q$ são ambos verdadeiras, e tem valor lógico falsidade nos demais casos.\end{definicao}
\begin{table}[h]
   \centering
   \setlength{\arrayrulewidth}{0,5\arrayrulewidth}
   \caption{\it Tabela verdade: Disjunão Exclusiva}
   \begin{tabular}{|c|c|c|c|c|}
      \hline
      $p$ & $q$ & $p\veebar q$ \\
     \hline
      V & V & F \\
      \hline
      V & F & V \\
      \hline
      F & V & V \\
      \hline
      F & F & F \\
      \hline
   \end{tabular}
\end{table}

Exemplo:\\
\textbf{$P$}: Todo número inteiro ou é par ou é ímpar, $V(P)=V$

\subsubsection{Condicional}
\begin{definicao}[Condicional] Chama-se proposição condicional ou condicional uma proposição representada por ``$p\rightarrow q$", que lê-se ``se $p$ então $q$". O valor lógico da condicional é a falsidade no caso em que $p$ é verdade e $q$ é falsidade e tem valor lógico verdade nos demais casos.\end{definicao}

Na condicional ``$p\rightarrow q$", dizemos que $p$ é o antecedente e o $q$ é o conseqüente. O símbolo "$\rightarrow$" é chamado de símbolo de implicaão.
\begin{table}[h]
   \centering
   \setlength{\arrayrulewidth}{0,5\arrayrulewidth}
   \caption{\it Tabela verdade: Condicional}
   \begin{tabular}{|c|c|c|c|c|}
      \hline
      $p$ & $q$ & $p\rightarrow q$ \\
     \hline
      V & V & V \\
      \hline
      V & F & F \\
      \hline
      F & V & V \\
      \hline
      F & F & V \\
      \hline
   \end{tabular}
\end{table}

Exemplos:\\
\textbf{$P$}: Se o mês de maio tem 31 dias, então a terra é plana, $V(P)=F$\\
\textbf{$Q$}: Se 3+2=6 então 4+4=9, $V(Q)=V$\\
\textbf{$R$}: Se (-1)=0, então $\sin\displaystyle\frac{\pi}{6}=\displaystyle\frac{1}{2}$\\

Considere a seguinte condicional:
\begin{center}
7 é um número ímpar $\rightarrow$ Brasília é uma cidade
\end{center}

Observaão: Uma condicional não afirma que o conseqüente se deduz ou é uma conseqüência do antecedente $p$. Uma condicional afirma unicamente uma relaão entre valores lógicos de $p$ e $q$.

\subsubsection{Bicondicional}

\begin{definicao}[Bicondicional] Chama-se proposição bicondicional ou bicondicional uma proposição representada por ``Se, e somente se", denotada ``$p\leftrightarrow q(p\rightarrow q\wedge q\rightarrow p)$". O valor lógico da bicondicional é verdade quando $p$ e $q$ são ambas verdade ou falsidade, e tem valor lógico falsidade nos demais casos.\end{definicao}
\begin{table}[h]
   \centering
   \setlength{\arrayrulewidth}{0,5\arrayrulewidth}
   \caption{\it Tabela verdade: Bicondicional}
   \begin{tabular}{|c|c|c|c|c|}
      \hline
      $p$ & $q$ & $p\leftrightarrow q$ \\
     \hline
      V & V & V \\
      \hline
      V & F & F \\
      \hline
      F & V & F \\
      \hline
      F & F & V \\
      \hline
   \end{tabular}
\end{table}

Exemplos:\\
\textbf{$P$}: Roma fica na Europa se, e somente se, a neve é branca, $V(P)=V$\\
\textbf{$Q$}: Lisboa é a capital de Portugal se, e somente se, $\tan\displaystyle\frac{\pi}{4}=3$, $V(Q)=F$
\section{Tabela Verdade}

\hspace{0,5cm}Dadas várias proposições simples $p,q,r,...$, podemos combiná-las formando novas proposições através do uso dos conectivos lógicos. Por exemplo:\\
\textbf{$P$}:$\neg p\vee(p\rightarrow q)$\\
\textbf{$R$}:$(p\leftrightarrow q)\wedge q$\\
\textbf{$S$}:$(p\rightarrow\neg q\vee r)\vee(q\vee(p\leftrightarrow\neg r))$

\subsubsection{Ordem de precedência}

Na lógica matemática, é convencionado a seguinte ordem de precedência dos operadores:
\begin{enumerate}
\item $\neg$
\item $\wedge,\vee$
\item $\rightarrow,\leftrightarrow$
\end{enumerate}

Duas regras importantes devem ser observadas
\begin{enumerate}
\item A ordem de precedência de uma operaão lógica somente pode ser alterada através do uso de parênteses
\item Operadores diferentes e de mesma prioridade necessariamente devem ter sua ordem indicada pelo uso de parênteses
\end{enumerate}

Exemplos
\begin{enumerate}
\item $p\vee q\vee r\leftrightarrow\neg p$ (Correto)
\item $p\vee q\vee(r\leftrightarrow\neg p)$ (Correto)
\item $p\wedge q\vee r$ (Errado)
\item $p\rightarrow q\leftrightarrow p$ (Errado)
\end{enumerate}

Observaão: A colocaão de parênteses pode alterar o valor lógico (e o sentido) de uma proposição

\begin{center}
Exemplo
\end{center}
\begin{minipage}[l]{0,5\textwidth}
I)\\
$V\vee F\rightarrow F$\\
$V\rightarrow F$\\
$F$\\
\end{minipage}
\begin{minipage}[r]{0,5\textwidth}
II)\\
$V\vee(F\rightarrow F)$\\
$V\vee V$\\
$V$

\end{minipage}

\section{Construão de Tabelas Verdade}

\hspace{0,5cm}Para construir a tabela verdade de uma proposição composta $P(p,q,r,...)$ começamos contando o número de proposições simples que comp{\~o}em $P(p,q,r,...)$.

\subsubsection{Número de linhas}
Se $P(p,q,r,...)$ for composta por $n$ proposições simples, então a tabela verdade de $P(p,q,r,...)$ conterá $2^{n}$ linhas.

Exemplos: Construir a tabela verdade das seguintes proposições:
\begin{enumerate}
\item $P:\neg(p\wedge\neg q)$ (Tabela \ref{1})
\begin{table}[h]
   \centering
   \setlength{\arrayrulewidth}{0,5\arrayrulewidth}
   \caption{\it $P:\neg(p\wedge\neg q)$}
   \begin{tabular}{|c|c|c|c|c|}
      \hline
      $p$ & $q$ & $\neg q$ & $p\wedge\neg q$ & $\neg(p\wedge\neg q)$ \\
     \hline
      V & V & F & F & V \\
      \hline
      V & F & V & V & F \\
      \hline
      F & V & F & F & V \\
      \hline
      F & F & V & F & V \\
      \hline
   \end{tabular}
\label{1}
\end{table}
\item $Q:\neg(p\wedge q)\vee\neg(p\leftrightarrow p)$ (Tabela \ref{2})
\begin{table}[h]
   \centering
   \setlength{\arrayrulewidth}{0,5\arrayrulewidth}
   \caption{\it $Q:\neg(p\wedge q)\vee\neg(p\leftrightarrow p)$}
   \begin{tabular}{|c|c|c|c|c|c|c|}
      \hline
      $p$ & $q$ & $p\wedge q$ & $p\leftrightarrow q$ & $\neg(p\wedge q)$ & $\neg(p\leftrightarrow q)$ & Q \\
     \hline
      V & V & V & V & F & F & F \\
      \hline
      V & F & F & F & V & V & V\\
      \hline
      F & V & F & F & V & V & V \\
      \hline
      F & F & F & V & V & V & V \\
      \hline
   \end{tabular}
\label{2}
\end{table}
\end{enumerate}

\subsection{Tautologia}
\begin{definicao}[Tautologia] Chama-se tautologia a proposição composta que é sempre verdadeira independentemente dos valores lógicos das proposições que a comp{\~o}em.\end{definicao}

Na tabela verdade de uma tautologia, a última coluna contém somente o valor lógico verdadeiro.

\subsection{Contradição}
\begin{definicao}[Contradição] Chama-se contradição a proposição composta que é sempre falsa independentemente dos valores lógicos das proposições que a comp{\~o}em.\end{definicao}

Assim, na tabela verdade de uma contradição, a última coluna contém somente o valor lógico falso.

\section{Implicaão}
\begin{definicao}[Implicaão] Dizemos que uma proposição $P(p,q,r,..)$ implica uma proposição composta $Q(p,q r,...)$, denotado "$P\Rightarrow Q$", se para todo valor verdade da primeira, então a segunda é verdadeira.\end{definicao}

Assim, $P\Rightarrow Q$ somente se a condicional $P\rightarrow Q$ for uma tautologia.

Exemplo: Sendo $P:p\wedge q$ e $Q:p\vee q$, verificar se $P\Rightarrow Q$.

Precisamos verificar se a condicional $P\rightarrow Q$ é uma tautologia (Tabela \ref{3}).
\begin{table}[h]
   \centering
   \setlength{\arrayrulewidth}{0,5\arrayrulewidth}
   \caption{\it $P\rightarrow Q$}
   \begin{tabular}{|c|c|c|c|c|}
      \hline
      $p$ & $q$ & $p\wedge q$ & $p\vee q$ & $(p\wedge q)\rightarrow(p\vee q)$ \\
     \hline
      V & V & V & V & V \\
      \hline
      V & F & F & V & V \\
      \hline
      F & V & F & V & V \\
      \hline
      F & F & F & F & V \\
      \hline
   \end{tabular}
\label{3}
\end{table}

Observaão:
\begin{enumerate}
\item Os símbolos $\rightarrow$ e $\Rightarrow$ são diferentes. A condicional, $\rightarrow$, é um operador lógico que aplicado a duas proposições $p$ e $q$, por exemplo, produz uma nova proposição $p\rightarrow q$. Por outro lado, a implicaão, $\Rightarrow$, estabelece que $p\rightarrow q$ é uma tautologia.
\item Toda teorema é uma aplicaão da forma
\begin{center}
Hipótese $\Rightarrow$ Tese
\end{center}

Logo demonstrar um teorema significa mostrar que não ocorre o caso da hipótese ser verdadeira e a tese falsa, isto é, a verdade da hipótese é suficiente para garantir a verdade da tese.
\end{enumerate}

\subsection{Equivalência}

\begin{definicao}[Equivalência] Dizemos que uma proposição $P(p,q,r,...)$ é equivalente a uma proposição composta $Q(p,q,r,...)$, denotado por $P\Leftrightarrow Q$, se elas implicarem uma na outra. Assim, $P\Leftrightarrow Q$ se a bicondicional é uma tautologia.\end{definicao}

Exemplo: Sendo $P:p\leftrightarrow q$ e $Q:(p\rightarrow q)\wedge(q\rightarrow p)$ verificar que $P\Leftrightarrow Q$.\\

Precisamos verificar se a bicondicional $P\leftrightarrow Q$ é uma tautologia (Tabela \ref{4}).
\begin{table}[h]
   \centering
   \setlength{\arrayrulewidth}{0,5\arrayrulewidth}
   \caption{\it $P\leftrightarrow Q$}
   \begin{tabular}{|c|c|c|c|c|c|c|}
      \hline
      $p$ & $q$ & $p\rightarrow q$ & $q\leftrightarrow p$ & $(p\rightarrow q)\wedge(q\rightarrow p)$ & $p\leftrightarrow q$ & $P\leftrightarrow Q$ \\
     \hline
      V & V & V & V & V & V & V \\
      \hline
      V & F & F & V & F & F & V\\
      \hline
      F & V & V & F & F & F & V \\
      \hline
      F & F & V & V & V & V & V \\
      \hline
   \end{tabular}
\label{4}
\end{table}
%!TEX program = xelatex
%!TEX root = Algebra_1.tex
%%Usar makeindex -s indexstyle.ist arquivo.idx no terminal para gerar o {\'\i}ndice remissivo agrupado por inicial
%%Ap\'os executar pdflatex arquivo
\chapter{Conceitos B\'asicos} % (fold)
\label{cha:conceitos_basicos}

\begin{definicao}
	Uma \textbf{proposi\c{c}\~ao} \'e todo conjunto de palavras ou s{\'\i}mbolos ao qual podemos atribuir um \textbf{valor l\'ogico}.
\end{definicao}

\begin{definicao}
	Diz-se que o \textbf{valor l\'ogico} de uma proposi\c{c}\~ao \'e ``verdade'' (V) se a proposi\c{c}\~ao \'e verdadeira ou ``falsidade'' (F) se a proposi\c{c}\~ao \'e falsa.
\end{definicao}

\begin{exemplos}
	Julgue se as seguintes sentenças são ou não proposições:
	\begin{enumerate}[label={\arabic*})]
		\item Todo número primo é ímpar.
		Essa setença é uma proposição de valor lógico "Falsidade."
		\item $x^2 + y^2 \ge 0$ para todos $x$, $y \in \real$.
		Esse setença é uma proposição de valor lógico "Verdade".
		\item Amanhã irá chover.
		Essa sentença não é uma proposição. Não é possível atribuir um valor lógico a ela.
	\end{enumerate}

\end{exemplos}

\section{Princ{\'\i}pio da n\~ao contradi\c{c}\~ao e do terceiro exclu{\'\i}do} % (fold)
\label{sec:principio_da_nao_contradicao_e_do_3}
\begin{enumerate}[label={\roman*})]
	\item Uma proposi\c{c}\~ao n\~ao pode ser verdadeira e falsa ao mesmo tempo.
	\item Toda proposi\c{c}\~ao ou \'e verdadeira ou \'e falsa, isto \'e, verifica-se sempre um destes casos e nunca um terceiro.
\end{enumerate}

Assim esses princ{\'\i}pios afirmam que:
\begin{center}
	``Toda proposi\c{c}\~ao tem um, e um s\'o, dos valores l\'ogicos \textbf{verdade} ou \textbf{falsidade}.''
\end{center}

De modo geral vamos trabalhar com proposi\c{c}\~oes da forma:
\begin{enumerate}[label={\roman*})]
	\item Se $\mathcal{H}$, ent\~ao $\mathcal{T}$.

	Aqui $\mathcal{H}$ \'e chamado de hip\'otese e $\mathcal{T}$ de tese. Neste tipo de proposi\c{c}\~ao iremos admitir que $\mathcal{H}$ \'e uma verdade e precisaremos provar que $\mathcal{T}$ \'e verdade. Ou seja precisamos construir um argumento que justifique $\mathcal{T}$ ser verdadeira \`a partir do fato de $\mathcal{H}$ ser verdadeira.

	\item $\mathcal{H}$ se, e somente se, $\mathcal{T}$ ou $\mathcal{H}$ se, e s\'o se, $\mathcal{T}$.

	Esse tipo de proposi\c{c}\~ao ser\'a decomposta em duas proposi\c{c}\~oes no formato anterior. Isto \'e:
	\begin{enumerate}[label={\alph*})]
		\item Se $\mathcal{H}$, ent\~ao $\mathcal{T}$.
		\item Se $\mathcal{T}$, ent\~ao $\mathcal{H}$.
	\end{enumerate}

	No primeiro caso admitimos $\mathcal{H}$ verdadeira e provamos que $\mathcal{T}$ tamb\'em \'e verdadeira e no segundo caso admitimos que $\mathcal{T}$ \'e verdadeira e provamos que $\mathcal{H}$ \'e verdadeira.
\end{enumerate}
% section pr{\'\i}ncipio_da_n\~ao_contradi\c{c}\~ao_e_do_3 (end)

% chapter conceitos_b\'asicos (end)

\chapter{No{\c c}{\~o}es de Teoria de Conjuntos}
\section{Conceitos b{\'a}sicos}

Um conjunto {\'e} uma ``cole{\c c}{\~a} o'' ou ``fam{\'\i}lia'' de elementos.

Usaremos letras mai{\'u}sculas do alfabeto para denotar os conjuntos e denotaremos elementos de um dado conjunto por letras min{\'u}sculas do alfabeto.

Dado um conjunto $A$, para indicar o fato de que $x$ {\'e} um elemento de $A$, escrevemos:
\[
x \in A.
\]

Para dizer que um elemento $x$ n{\~a}o pertence ao conjunto $A$, escrevemos:
\[
x \notin A.
\]

Um conjunto sem elementos {\'e} chamado de \textbf{conjunto vazio}. Tal conjunto {\'e} denotado por $\emptyset$.

Dado um conjunto $A$ e $x$ um elemento, ocorre sempre o uma das seguintes situa\c{c}\~oes:
\[
x \in A \mbox{ ou } x \notin A.
\]

Al{\'e}m disso, para dois elementos $x$, $y \in A$, ocorre exatamente uma das seguinte situa\c{c}\~oes:
\[
x = y \mbox{ ou } x \neq y.
\]

\section{Descri{\c c}{\~a}o de um conjunto}

Um conjunto $A$ pode ser dado pela simples listagem dos seus elementos, como por exemplo:
\begin{align*}
	A= \{1,2,3,4,5\}\\
	B = \{verdade, falso\}.
\end{align*}

Um conjunto tamb{\'e}m pode ser dado pela descri{\c c}{\~a}o das propriedades dos seus elementos, como por exemplo:
\[
A = \{n \mid n \mbox{ \'e m{\'u}ltiplo de } 2\} = \{2,4,6,...\}.
\]

\section{Alguns conjuntos importantes}
\begin{enumerate}[label={\arabic*})]
	\item $\n = \{0,1,2,3,...\}$ o conjunto do n{\'u}meros naturais.
	\item $\z = \{...,-2,-1,0,1,2,...\}$ o conjunto dos n{\'u}meros inteiros.
	\item $\n_0 = \{0,1,2,3,...\}$ o conjunto dos n{\'u}meros inteiros n{\~a}o negativos.
	\item $\real $ o conjunto dos n{\'u}meros reais.
	\item $\real^*$ o conjunto dos n{\'u}meros reais n{\~a}o nulos.
	\item $\rac = \left\{\dfrac{p}{q} \mid p,q \in \z, q \neq 0 \right\}$ o conjunto dos n{\'u}meros racionais.
\end{enumerate}

\section{Propriedades dos conjuntos}

\begin{definicao}
	Dados dois conjuntos $A$ e $B$, dizemos que $A$ e $B$ s{\~a}o \textbf{iguais} se, e somente se, eles t{\^e}m os mesmos elementos. Ou seja, para todo $x \in A$ temos que $x \in B$ e para todo $y \in B$ temos $y \in A$.
\end{definicao}

Se $A$ e $B$ s{\~a}o iguais, escrevemos $A = B$
\begin{align*}
	\{1,2,3,4\} &= \{3,2,1,4\}\\
	\{1,2,3\} &\ne \{2,3\} 
\end{align*}
	
\begin{definicao}
	Se $A$ e $B$ s{\~a}o dois conjuntos, dizemos que $A$ {\'e} um \textbf{subconjunto} de $B$ ou que $A$ \textbf{est\'a contido} em $B$ ou que $B$ \textbf{cont\'em} $A$ se todo elemento de $A$ for elemento de $B$. Ou seja, se para todo elemento $x \in A$, temos $x \in B$. Nesse caso, escrevemos $A \subseteq B$ ou $B \supseteq A$.
\end{definicao}


Caso $A$ seja um subconjunto de $B$ mas n{\~a}o {\'e} igual a $B$, escrevemos:
\[
A \subsetneq B.
\]

Nesse caso, dizemos que $A$ {\'e} um \textbf{subconjunto pr{\'o}prio} de $B$.

Para dizer que $A$ n{\~a}o est{\'a} contido em $B$, escrevemos $A \nsubseteq B$

Usando a defini\c{c}\~ao de contin\^encia de conjuntos podemos definir igualdade de conjuntos da seguinte forma:
\begin{center}
	\textbf{dois conjuntos $A$ e $B$ s\~ao iguais se, e somente se, $A \subseteq B$ e $B \subseteq A$}.
\end{center}

Ou seja,
\begin{center}
	\textbf{se $A = B$ ent{\~a}o $A \subseteq B$ e $B \subseteq A$}.
\end{center}

Além disso,
\begin{center}
	\textbf{se $A \subseteq B$ e $B \subseteq A$, ent{\~a}o $A = B$}.
\end{center}

Quando $A$ e $B$ n{\~a}o s{\~a}o iguais, escrevemos $A \neq B$. Para que $A \neq B$ devemos ter $A \nsubseteq B$ ou $B \nsubseteq A$. Isto é, precisamos encontrar algum elemento $x \in A$ tal que $x \notin B$ ou então encontrar $y \in B$ tal que $y \notin A$.

\begin{proposicao}
	Dados três conjuntos $A$, $B$ e $C$ temos:
	\begin{enumerate}[label={\roman*})]
		\item $A\subseteq A$ (Reflexividade)
		\item Se $A\subseteq B \mbox{ e } B\subseteq A$, ent{\~a}o $A=B$. (Antissimetria)
		\item Se $A\subseteq B$ e $B\subseteq C$, ent{\~a}o $A\subseteq C$. (Transitividade)
	\end{enumerate}
\end{proposicao}


Considere os seguintes conjuntos:
\begin{align*}
	A &= \{ n \in \n \mid n \mbox{ {\'e} m{\'u}ltiplo de } 2\} = \{2,4,6,...\}\\
	B &= \{n \in \n \mid n \mbox{ {\'e} m{\'u}ltiplo de } 3\} = \{3,6,9,...\}.
\end{align*}


Neste caso, $2 \in A$ e $2 \notin B$, logo $A \nsubseteq B$. Por outro lado, $3 \in B$ e $3 \notin A$ e com isso $B \nsubseteq A$. Portanto, dados dois conjuntos $A$ e $B$, nem sempre temos $A \subseteq B$ ou $B \subseteq A$.

\begin{proposicao} 
	Seja $A$ um conjunto. Ent{\~a}o $ \emptyset \subseteq A$.
\end{proposicao}
\begin{prova}
	Suponha que $\emptyset \nsubseteq A$. Logo existe $x \in \emptyset$ tal que $x \notin A$. Mas por defini{\c c}{\~a}o, o conjunto vazio n{\~a}o cont{\'e}m elementos. Logo a exist\^encia de $x \in \emptyset$ {\'e} uma contradi{\c c}{\~a}o. Tal contradi\c{c}\~ao surgiu por termos suposto que $\emptyset \nsubseteq A$. Portanto, $\emptyset \subseteq A$, como quer{\'\i}amos demonstrar.
\end{prova}

\section{Rela{\c c}{\~o}es entre conjuntos}

\begin{definicao}\label{intersecao_conjunto}
Sejam $A$ e $B$ dois conjuntos. Definimos a \textbf{intersec{\c c}{\~a}o} de $A$ e $B$ como sendo o conjunto $A \cap B$ cujos elementos pertencem ao conjunto $A$ e $B$ simultaneamente. Assim,
\[
A \cap B = \{x \mid x \in A\mbox{ e }  x \in B\}.
\]
\end{definicao}

\begin{exemplo}
	Sejam $A = \{1, 2, 3\}$, $B = \{2, 3, 4\}$ e $C = \{r, s, t\}$. Então
	\begin{align*}
		A \cap B &= \{2, 3\}\\
		A \cap C &= \emptyset.
	\end{align*}
\end{exemplo}

\begin{definicao}\label{unicao_conjuntos}
Sejam $A$ e $B$ dois conjuntos. Definimos a \textbf{uni{\~a}o} de $A$ com $B$ como sendo o conjunto $A \cup B$, cujos elementos pertencem ao conjunto $A$ ou ao conjunto $B$. Assim,
\[
A \cup B = \{x \mid x \in A \mbox{ ou } x \in B\}.
\]
\end{definicao}

\begin{exemplo}
	Sejam $A = \{1, 2, 3\}$, $B = \{2, 3, 4\}$ e $C = \{r, s, t\}$. Então
	\begin{align*}
		A \cup B &= \{1,2,3,4\}\\
		A \cup C &= \{1,2,3,r,s,t\}.
	\end{align*}
\end{exemplo}

\begin{proposicao} Sejam $A$ e $B$ dois conjuntos. Ent{\~a}o:
	\begin{enumerate}[label={\roman*})]
		\item $(A \cap B) \subseteq A$;
		\item $(A \cap B) \subseteq B$;
		\item $A \subseteq A \cup B$;
		\item $B \subseteq A \cup B$.
	\end{enumerate}
\end{proposicao}
\begin{prova}
	Para provar a primeira afirmação seja $x \in A \cap B$ um elemento qualquer. Da defini\c{c}\~ao de interse\c{c}\~ao de conjuntos, Definição \ref{intersecao_conjunto}, temos $x \in A$ e $x \in B$. Assim podemos afirmar com certeza que $x \in A$. Logo todo elemente de $A \cap B$ também está em $A$, ou seja, $A \cap B \subseteq A$. De modo análogo prova-se a segunda afirmação sobre interseção.

	Para a terceira afirmação, seja $x \in A$. Da definição de união de conjuntos, Definição \ref{unicao_conjuntos}, segue que $x \in A \cup B$. Logo todo elemento de $A$ também está em $A \cup B$, ou seja, $A \subseteq (A \cup B)$. De modo análogo prova-se a quarta afirmação.
\end{prova}

O conceito de uni{\~a}o ($ \cup $) e intersec{\c c}{\~a}o ($ \cap $) pode ser estendido para mais de dois conjuntos.

\begin{definicao}
Sejam $A_{1}$, \dots, $A_{n}$ conjuntos. Ent{\~a}o
\[
A_{1} \cup A_{2} \cup \cdots \cup A_{n}= \displaystyle\bigcup_{k=1}^n A_{k}
\]
{\'e} o conjunto dos elementos $x$ tais que $x$ pertence a pelo menos um dos conjuntos $A_{1}$, \dots, $A_{n}$. Agora,
\[
A_{1} \cap \cdots \cap A_{n} = \displaystyle\bigcap_{k=1}^{n}A_{k}
\]
{\'e} o conjunto dos elementos $x$ que pertencem a todos os conjuntos $A_{1}$, \dots, $A_{n}$ simultaneamente.
\end{definicao}

\begin{definicao}
	Sejam $A$ e $B$ conjuntos. Se $A \cap B = \emptyset$, dizemos que $A$ e $B$ s{\~a}o \textbf{conjuntos disjuntos}.	
\end{definicao}


Sejam $A$ e $B$ conjuntos tais que $C = A \cup B$ e $A \cap B = \emptyset$. Neste caso dizemos que $C$ {\'e} uma \textbf{uni{\~a}o disjunta} de $A$ e $B$. Denotamos tal fato por
\[
C = A \sqcup B.
\]

\begin{proposicao} Sejam $A,\ B$ e $C$ tr{\^e}s conjuntos, ent{\~a}o:
	\begin{enumerate}[label={\roman*})]
		\item $A\cap(B\cup C)=(A\cap B)\cup(A\cap C)$
		\item $A\cup(B\cap C)=(A\cup B)\cap(A\cup C)$
	\end{enumerate}
\end{proposicao}
\begin{prova}
	\begin{enumerate}[label={\roman*})]
		\item Precisamos mostrar que
		\begin{enumerate}[label={\roman*})]
			\item $A\cap(B\cup C)\subseteq(A\cap B)\cup(A\cap C)$;\label{intersecao_unicao_1}
			\item $(A\cap B)\cup(A\cap C)\subseteq A\cap(B\cup C).$\label{intersecao_unicao_2}
		\end{enumerate}

		Para provar \ref{intersecao_unicao_1} seja $x\in A \cap (B \cup C)$. Logo $x\in A$ e $x\in B\cup C$. Agora, de $x\in B\cup C$, segue que $x\in B$ ou $x\in C$. Suponha que $x\in B$. Como $x\in A$ e $x \in B$, ent\~ao $x\in A\cap B$. Assim, $x\in(A\cap B)\cup(A\cap C)$, ou seja, $A\cap(B\cup C)\subseteq(A\cap B)\cup(A\cap C)$. Por outro lado, se $x\in C$, como $x\in A$, ent{\~a}o $x\in A\cap C$ e da{\'\i} $x\in(A\cap B)\cup(A\cap C)$, logo $A\cap(B\cup C)\subseteq(A\cap B)\cup(A\cap C)$.

		Portanto,
		\[
			A\cap(B\cup C)\subseteq(A\cap B)\cup(A\cap C).
		\]

		Agora para provar \ref{intersecao_unicao_2}, seja $x\in(A\cap B)\cup(A\cap C)$. Da{\'\i}, $x\in A\cap B$ ou $x\in A\cap C$. Suponha que $x\in A\cap B$. Assim, $x\in A$ e $x\in B$. Como $x\in B$, segue que $x\in B\cup C$ e ent{\~a}o $x\in A\cap(B\cup C)$, ou seja, $(A\cap B)\cup(A\cap C)\subseteq A\cap(B\cup C)$. Agora, suponha que $x\in A\cap C$. Com isso $x\in A$ e $x\in C$. Desse modo, $x\in B\cup C$ e ent{\~a}o $x\in A\cap(B\cup C)$ e da{\'\i}
		\[
			(A\cap B)\cup(A\cap C)\subseteq A\cap(B\cup C).
		\]

		Portanto
		\[
			A\cap(B\cup C)=(A\cap B)\cup(A\cap C),
		\]
		como quer{\'\i}amos.
		\item An\'aloga ao caso anterior.
	\end{enumerate}
\end{prova}

\begin{definicao}
	Dados dois conjuntos $A$ e $B$, definimos a \textbf{diferen{\c c}a} dos conjuntos $A$ e $B$, denotada por $A-B$ ou $A\backslash B$ como sendo o conjunto
	\[
		A - B = \{x \mid x \in A \mbox{ e } x \notin B\}.
	\]
\end{definicao}

\begin{exemplos}
	\begin{enumerate}[label={\arabic*})]
		\item Se $A=\{1,2,3,5,4\}$, $B=\{2,3,6,8\}$, então
		\begin{align*}
			A - B &= \{1,4,5\}\\
			B - A &=\{6,8\}.
		\end{align*}
		\item Se $A=\{2,4,6,8,10,...\}$, $B=\{3,6,9,12,15,...\}$, então
		\begin{align*}
		 	A - B &= \{2,4,8,10,14,16,...\}\\
		 	B - A &= \{3,9,15,21,...\}
		 \end{align*}
	\end{enumerate}
	
\end{exemplos}

\begin{proposicao}
	Sejam $A$, $B$ e $C$ conjuntos n\~ao vazios. Ent\~ao
	\[(A \cup B) - C = (A - C) \cup (B - C).\]
\end{proposicao}
\begin{prova}
	Segue da defini\c{c}\~ao de diferen\c{c}a de conjuntos.
\end{prova}

\begin{definicao}
Dados dois conjuntos $A$ e $E$ tais que $A\subseteq E$, definimos o \textbf{complementar} de $A$ em $E$, denotado $A^C$ ou $C_E(A)$, como
\[
	C_E(A) = \{ x \in E \mid x \notin A \}.
\]
\end{definicao}

\begin{observacoes}
	\begin{enumerate}[label={\arabic*})]
		\item Se $A = E$, ent{\~a}o $C_A(A) = \{ x \in A \mid x \notin A \} = \emptyset$.
		\item $(A^C)^C = \{x \in E \mid x \notin A^C\} = \{ x \in E \mid x \in A \} = A$
	\end{enumerate}
	
\end{observacoes}

\begin{exemplo}
	Sejam $A = \{1,2,3,4\}$ e $E = \{1,2,3,5,4,0,8,9\}$. Primeiro note que $A \subseteq E$, daí
	\[
			A^C = C_E(A) = \{0,5,8,9\}.
	\]
\end{exemplo}

\begin{proposicao}
	Sejam $A$, $B$ e $E$ conjuntos. Se $A\subseteq B\subseteq E$, ent{\~a}o $C_E(B)\subseteq C_E(A)$.
\end{proposicao}
\begin{prova}
	Seja $x \in C_E(B)$. Assim $x\notin B$ e como $A \subseteq B$, ent\~ao $x \notin A$. Da{\'\i} por defini\c{c}\~ao $x\in C_E(A)$, ou seja, $C_E(B) \subseteq C_E(A)$.
\end{prova}

\begin{proposicao} Sejam $A$, $B$ e $E$ tr{\^e}s conjunto tais que $A\subseteq E$ e $B\subseteq E$. Ent{\~a}o:
\begin{enumerate}[label={\roman*})]
	\item $(A\cup B)^C = A^C\cap B^C$
	\item $(A\cap B)^C = A^C\cup B^C$
\end{enumerate}
\end{proposicao}
\begin{prova}
	\begin{enumerate}[label={\roman*})]
		\item Seja $x \in (A\cup B)^C$. Logo $x\notin A\cup B$, assim $x\notin A$ e $x\notin B$. Da{\'\i}, $x\in A^C$ e $x\in B^C$, isto {\'e}, $x\in A^C\cap B^C$. Desse modo,
		\begin{equation}\label{complementar_uniao-1}
			(A\cup B)^C \subseteq A^C\cap B^C.
		\end{equation}

		Por outro lado, se $x\in A^C\cap B^C$, ent{\~a}o $x\in A^C$ e $x\in B^C$. Com isso, $x\notin A$ e $x\notin B$, ou seja, $x\notin A\cup B$, logo $x\in (A\cup B)^C$. Desse modo
		\begin{equation}\label{complementar_uniao-2}
			A^C\cap B^C\subseteq(A\cup B)^C.
		\end{equation}

		Portanto, de \eqref{complementar_uniao-1} e \eqref{complementar_uniao-2} temos
		\[
			(A\cup B)^C = A^C\cap B^C.
		\]

		\item Seja $x \in (A\cap B)^C$. Logo $x\notin A\cap B$, assim $x\notin A$ ou $x\notin B$. Ent\~ao $x\in A^C$ ou $x\in B^C$, isto {\'e}, $x\in A^C\cup B^C$. Desse modo,
		\begin{equation}\label{complementar_intersecao-1}
			(A\cap B)^C \subseteq A^C\cup B^C.
		\end{equation}

		Por outro lado, se $x\in A^C\cup B^C$, ent{\~a}o $x\in A^C$ ou $x\in B^C$. Da{\'\i}, $x\notin A$ ou $x\notin B$, ou seja, $x\notin A\cap B$, logo $x\in (A\cap B)^C$. Desse modo
		\begin{equation}\label{complementar_intersecao-2}
			A^C\cup B^C\subseteq(A\cap B)^C.
		\end{equation}

		Portanto, de \eqref{complementar_intersecao-1} e \eqref{complementar_intersecao-2} temos
		\[
			(A\cap B)^C = A^C\cup B^C.
		\]
	\end{enumerate}
\end{prova}

\begin{definicao}
	Dados dois conjuntos $A$ e $B$, definimos o \textbf{produto cartesiano} de $A$ por $B$ como sendo o conjunto
	\[
		A \times B = \{(x,y) \mid x\in A, y\in B\}.
	\]
\end{definicao}

Dados $(x,y)$, $(z,t) \in A\times B$, temos
\begin{center}
	\textbf{$(x,y) = (z,t)$ se, e somente se, $x = z$ e $y = t$}.
\end{center}

\begin{exemplo}\label{exemplo_produto_cartesiano}
	Sejam $A = \{1,2\}$ e $B = \{3,4\}$. Então
	\begin{align*}
		A \times B &= \{(1,3), (1,4), (2,3), (2,4)\}\\
		B \times A &= \{(3,1), (3,2), (4,1), (4,2)\}
\end{align*}
\end{exemplo}

\begin{observacao}
	Do Exemplo \eqref{exemplo_produto_cartesiano} vemos que em geral $A \times B \neq B\times A$.
\end{observacao}

\begin{definicao}
	Para qualquer conjunto $A$, indicamos por $\mathcal{P}(A)$ o conjunto
	\[
		\mathcal{P}(A) = \{ X \mid X\subseteq A\}
	\]
	que \'e chamado de \textbf{conjunto das partes} de $A$.
\end{definicao}

Os elementos desse conjunto s{\~a}o todos os subconjuntos de $A$. Dizer que $Y\in \mathcal{P}(A)$ significa que $Y \subseteq A$. Particularmente, temos $\emptyset\in \mathcal{P}(A)$ e $A\in \mathcal{P}(A)$.

\begin{exemplos}
	\begin{enumerate}[label={\arabic*})]
		\item $A = \emptyset$, $\mathcal{P}(A) = \{\emptyset\}$;
		\item $B = \{x\}$, $\mathcal{P}(B) = \{\emptyset, \{x\}\}$;
		\item $C = \{a,b,c\}$, $\mathcal{P}(C)=\{\emptyset, \{a\}, \{b\},\{c\},\{a,b\},\{a,c\},\{b,c\},C\}$;
		\item $D=\real$, $\mathcal{P}(D)=\{X\mid X \subseteq \real\}$, por exemplo $\rac\in \mathcal{P}(D)$.
	\end{enumerate}	
\end{exemplos}
%!TEX program = xelatex
%!TEX root = Algebra_1.tex
%%Usar makeindex -s indexstyle.ist arquivo.idx no terminal para gerar o {\'\i}ndice remissivo agrupado por inicial
%%Ap\'os executar pdflatex arquivo
\chapter{N{\'u}meros Inteiros}
\section{Conceitos b{\'a}sicos}

\hspace{0,5cm}Indicaremos por $\mathbb{Z}$ o conjunto dos n{\'u}meros inteiros. Portanto $\mathbb{Z}=\{0,\pm 1,\pm 2,\pm 3, \pm 4,...\}$.

\subsubsection{Propriedades b{\'a}sicas da adi{\c c}{\~a}o e da multiplica{\c c}{\~a}o}

Admitiremos as propriedades b{\'a}sicas da adi{\c c}{\~a}o e da multiplica{\c c}{\~a}o em $\mathbb{Z}$. Assim, dados $a,b,c\in\mathbb{Z}$, temos:\\

\begin{minipage}[l]{0,5\textwidth}
Adi{\c c}{\~a}o
\begin{enumerate}
\item $a+b=b+a$
\item $a(b+c)=(a+b)+c$
\item $a+0=a$
\item $a+(-a)=0$
\end{enumerate}
\end{minipage}
\begin{minipage}[r]{0,5\textwidth}
Multiplica{\c c}{\~a}o
\begin{enumerate}
\item $ab=ba$
\item $a(bc)=(ab)c$
\item $a1=a$
\item $ab=0\rightarrow a=0\vee b=0$
\item $ab=1\rightarrow a=\pm 1\wedge b=\pm 1$
\item $a(b+c)=ab+ac$
\end{enumerate}
\end{minipage}

\subsubsection{Propriedades b{\'a}sicas das desigualdades}

Admitiremos tamb{\'e}m a rela{\c c}{\~a}o ``menor ou igual'', em $\Z$, denotada por ``$\leq$''. Dados $a$, $b$, $c\in \Z$, valem as seguintes propriedades:
\begin{enumerate}
\item $a\leq a$
\item $a\leq b\wedge b\leq a\rightarrow a=b$
\item $a\leq b\wedge b\leq c\rightarrow a\leq c$
\item $a\leq b\veebar b\leq a$
\item $a\leq b\rightarrow a+c\leq b+c$
\item $0\leq a\wedge 0\leq b\rightarrow 0\leq ab$
\end{enumerate}

Para a rela{\c c}{\~a}o ``menor'', cujo s{\'\i}mbolo {\'e} ``$<$'', vale:
\begin{enumerate}
\item Se $a> 0$ e $b > 0$, então $ab > 0$.
\item Se $a > 0$ e $b<0$, então $ab<0$.
\end{enumerate}

\section{Princ{\'\i}pio da boa ordena{\c c}{\~a}o}

\begin{definicao}[Limite Inferior] 
	Seja $A$ um subconjunto n{\~a}o vazio de $\Z$. Dizemos que $A$ {\'e} \textbf{limitado inferiormente} se existe $l \in \Z$ tal que $l \leq x$
	, para todo $x\in A$.\index{N\'umeros inteiros!Conjuntos limitados}
\end{definicao}

Por exemplo:\\
$A=\{-2,0,1,2,3,...\},\ B=\{...,-6,-4,-2,0\},\ C=\{8,16,24,32\}$

$A$ e $C$ s{\~a}o limitados inferiormente pois $-3\leq a$, $7\leq c$, para todo $a\in A$ e para todo $c\in C$.

\begin{definicao}[\textbf{Princ{\'\i}pio da boa ordena\c{c}\~ao}]
	Se $A$ {\'e} um subconjunto n{\~a}o vazio de $\Z$ e $A$ {\'e} limitado inferiormente, ent{\~a}o existe $a_{0}\in A$ tal que $a_{0}\leq x$ para todo $x\in A$.\index{Princ{\'\i}pio da boa ordena\c{c}\~ao}
\end{definicao}

Seja $A\neq\emptyset$, $A\subseteq\mathbb{Z}$ e $A$ limitado inferiormente. Pelo P.B.O., existe $a_{0}\in A$ tal que $a_{0}\leq x$, para todo $x\in A$. Suponha que existe $a_{1}\in A$ tal que $a_{1}\leq x$, $x\in A$. Logo devemos ter $a_{0}\leq a_{1}$ e al\'em disso $a_{1}\leq a_{0}$, da{\'\i} $a_{1}=a_{0}$. Ou seja, o elemento $a_{0}\in A$ do P.B.O. {\'e} {\'u}nico. Chamamos $a_{0}$ de elemento \textbf{m{\'\i}nimo} ou \textbf{elemento minimal}.\index{Princ{\'\i}pio da boa ordena\c{c}\~ao!Elemento m{\'\i}nimo}

\section{Princ{\'\i}pio da Indu{\c c}{\~a}o Finita}

\begin{teorema}[Indu{\c c}{\~a}o finita ($1^a$ vers{\~a}o)]
Dado $a \in \Z$, suponhamos que a cada inteiro $n\geq a$ esteja associada uma proposi{\c c}{\~a}o $P(n)$ que depende de $n$. Ent{\~a}o $P(n)$ ser{\'a} verdadeira para todo $n\geq a$ desde que seja poss{\'\i}vel provar o seguinte:
\begin{enumerate}
\item $P(a)$ {\'e} verdadeira.
\item Dado $r > a$, se $P(k)$ {\'e} verdadeira para todo $k$ tal que $a \leq k\leq r$, ent{\~a}o $P(r)$ {\'e} verdadeira.
\end{enumerate}
\end{teorema}

\begin{teorema}[Indu{\c c}{\~a}o finita ($2^a$ vers{\~a}o)]
Dado $a \in \Z$, suponhamos que para cada $n\geq a$ esteja associada uma proposi{\c c}{\~a}o $P(n)$. Ent{\~a}o $P(n)$ {\'e} verdadeira para todo $n \geq a$ desde que seja poss{\'\i}vel provar o seguinte:
\begin{enumerate}
\item $P(a)$ {\'e} verdadeira.
\item Se $P(r)$ {\'e} verdadeira para $r \geq a$, ent{\~a}o $P(r+1)$ {\'e} verdadeira.
\end{enumerate}
\end{teorema}

\begin{exemplos}
	\begin{enumerate}
		\item Mostre que para todo $n \in \n$ vale
		\[
			1+2+3+...+n = \frac{n(n+1)}{2}.
		\]
		\begin{solucao}
			Para $n=1$, temos
			\[
				1=\displaystyle\frac{1(1+1)}{2}.
			\]

			Agora, suponha que para $r\geq 1$, temos
			\[
				\underbrace{1+2+...+r = \frac{r(r+1)}{2}}_{H.I}.
			\]
			Assim, para $r+1$ usando a Hip{\'o}tese de Indu{\c c}{\~a}o, obtemos
			\begin{align*}
				1+2+...+r+(r+1) &= \dfrac{r(r+1)}{2}+(r+1) = \dfrac{r(r+1)+2(r+1)}{2}\\ 
				&= \frac{(r+2)(r+1)}{2}.
			\end{align*}
			Portanto, pelo princ{\'\i}pio da indu{\c c}{\~a}o finita a afirmação está provada.
		\end{solucao}

		\item Prove que $(1 + p)^n \ge 1 + np$ para todo $n \in \n$ e $p \ge 0$.
		\begin{solucao}
			Para $n = 1$ temos
			\[
				(1 + p)^1 \ge 1 + p.
			\]
			Suponha então que para $n = k$ temos
			\[
				(1 + p)^k \ge 1 + kp.
			\]
			Para $n = k + 1$ temos
			\begin{align*}
				(1 + p)^{k + 1} &= (1 + p)^r(1 + p) \ge (1 + rp)(1 + p) \\ &= 1 + p + rp + rp^2 \\ &\ge 1 + (r + 1)p.
			\end{align*}
			Logo pelo Princípio da Indução finita a afirmação é verdadeira.
		\end{solucao}
	\end{enumerate}
\end{exemplos}




\begin{teorema}
Dado $a\in\mathbb{Z}$, suponhamos que cada inteiro $n\geq a$ esteja associado uma proposi{\c c}{\~a}o $P(n)$. Ent{\~a}o $P(n)$ ser{\'a} verdadeira $\forall n\geq a$ desde que seja poss{\'\i}vel provar que:
\begin{enumerate}
\item $P(a)$ {\'e} verdadeira.
\item Dado que $r > a$, se $P(k)$ {\'e} verdadeira para todo $k$ tal que $a \leq k \leq r$, ent{\~a}o $P(r)$ {\'e} verdadeira
\end{enumerate}
\end{teorema}

\textbf{Demonstra{\c c}{\~a}o}:
Seja $F=\{l \in \Z \mid a \leq l\ \mbox{e}\ P(l)\ \mbox{{\'e} falsa}\}$. Suponha $F \neq \emptyset$. Como $F$ {\'e} limitado inferiormente, pelo princ{\'\i}pio da boa ordena{\c c}{\~a}o, existe $l_0 \in F$ tal que $l_{0} \leq x$, para todo $x \in F$. Como $l_{0} \in F$, $P(l_0)$ {\'e} falsa. Mas $P(a)$ {\'e} verdadeira, assim, $l_0 > a$. Agora, como $l_0$ {\'e} o m{\'\i}nimo de $F$, ent{\~a}o $P(x)$ {\'e} verdadeira para $a \leq x < l_0$.

Agora pelo item (2) segue que $P(l_0)$ {\'e} verdadeira, o que {\'e} uma contradi{\c c}{\~a}o, pois verificamos anteriormente que $P(l_0)$ {\'e} falso.

Portanto $F=\emptyset$ e o teorema est{\'a} demonstrado.\#

\section{Divisibilidade}

\begin{definicao}[Divis{\~a}o] Sejam $a,b$ n{\'u}meros inteiros, $b\neq\emptyset$. Dizemos que $b$ divide $a$ quando existe um inteiro $c$ tal que $a=bc$.\end{definicao}

Exemplos:
\begin{enumerate}
\item Os inteiros 1 e $-1$ dividem todos os n{\'u}meros inteiros $a$, pois \[a=1a,a=(-1)(-a)\]
\item O n{\'u}mero 0 n{\~a}o divide nenhum inteiro $b$, pois n{\~a}o existe $a$ tal que $b=0a$
\item Para todo $b\neq 0$,$b$ divide $\pm b$
\item Para todo inteiro $b\neq 0$, $b$ divide 0, pois $0=b0$
\item 3 n{\~a}o divide 8, mas 17 divide 51
\end{enumerate}

\begin{nota}[Divis{\~a}o] Quando $b$ divide $a$, escrevemos $b|a$. Quando $b$ n{\~a}o divide $a$, escrevemos $b\not{|}a$\end{nota}

\textbf{Propriedades}
\begin{enumerate}
\item $a|a, \forall a\in\mathbb{Z}$
\item Se $a|b$ e $b|a,\ a,b\geq 0\rightarrow a=b$

De fato existe $c,d\in\mathbb{Z}/b=ca\wedge a=bd$. Se $a=0\vee b=0$ ent{\~a}o $b=0\veebar a=0$. Podemos supor $a\neq 0$ e $b\neq 0$.

Assim\\
$b=c(bd)$\\
$b(1-cd)=0$. Da{\'\i}, $1-cd=0$, isto {\'e}, $cd=1$.

Assim, $c=\pm 1\wedge d=\pm 1$. Como $a>0$ e $b>0$, devemos ter $c=d=1$. Portanto $a=b$
\item Se $a|b$ e $b|c$, ent{\~a}o $a|c$

De fato, $b=pa\wedge c=bq \Rightarrow c=(pq)a$, ou seja, $a|c$
\item Se $a|b$ e $a|c$, ent{\~a}o $a|(bx+cy)$, para todos $x,y\in\mathbb{Z}$

Temos $b=ap$ e $c=aq$, $p,q\in\mathbb{Z}$
\[bx+cy=apx+aqy=a\underbrace{(px+qy)}_{\in\mathbb{Z}}\]

Logo $a|(bx+cy)$
\end{enumerate}

\section{Algoritmo de divis{\~a}o de Euclides}

\begin{teorema}[Algoritmo de divis{\~a}o de Euclides] Para quaisquer $a,b\in\mathbb{Z}$, com\\ $b>0$, existem {\'u}nicos $q$ e $r$ inteiros tais que $a=bq+r$, com $0\leq r<b$.\end{teorema}

\textbf{Demonstra{\c c}{\~a}o}: Vamos mostrar primeiro a exist{\^e}ncia de $q$ e $r$.

Seja $M=\{m\in\mathbb{Z}/m=a-bt,\ t\in\mathbb{Z}\}$, onde $t$ varia sobre todos os inteiros. Temos $m\neq\emptyset$. Al{\'e}m disso, $M^{+}$ {\'e} limitado inferiormente, logo, pelo princ{\'\i}pio da boa ordena{\c c}{\~a}o, existe $r\in M^{+}/r\leq x,\ \forall x\in M^{+}$. Como $r\in m^{+}\subseteq M$, existe $q\in\mathbb{Z}$ tal que $r=a+bq$. Portanto, $a=bq+r,\ q\in\mathbb{Z}$, com $r\leq 0$.

Falta provar que $r<b$.

Suponha ent{\~a}o que $r\geq b$. Logo $r=a-bq\geq b$, ou seja,\[a-bq-b\geq 0\Leftrightarrow a-b(q+1)\geq 0\]

Desse modo, $a-b(q+1)\in M^{+}$.

Agora, \[b>0\Rightarrow bq+b>bq\Rightarrow b(q+1)>bq\] \[-b(q+1)<-bq\Rightarrow a-b(a+1)<a-bq=r\], o que {\'e} uma contradi{\c c}{\~a}o, pois $r$ {\'e} o m{\'\i}nimo de $M^{+}$, logo, $r<b$, ou seja, $a=bq+r,\ q,r\in\mathbb{Z},\ 0\leq r<b$\\

Falta provar a unidade de $q$ e $r$. Assim, suponha que existam \[q_{1},q_{2},r_{1},r_{2}\in\mathbb{Z},\ 0\leq r_{1}<b,\ 0\leq r_{2}<b\], tais que:
\[a=bq_{1}+r_{1}=bq_{2}+r_{2}\]

Suponha $r_{1}\neq r_{2}$. Suponha tamb{\'e}m que $r_{1}>r_{2}$. Assim, \[0\leq r_{1}-r_{2}=b(q_{2}-q_{1})\]

E da{\'\i}, $q_{2}-q_{1}\geq 0$.

Desse modo \[r_{1}=b(q_{2}-q_{1})+r_{2}\]

Mas $r_{1}\geq 0,\ q_{2}-q_{1}\geq 1$, da{\'\i} $r_{1}>b$, o que {\'e} uma contradi{\c c}{\~a}o, logo $r_{1}=r_{2}$ e ent{\~a}o $q_{1}=q_{2}$, o que prova a unicidade.\#

\section{M{\'a}ximo Divisor Comum}

\begin{definicao}[M{\'a}ximo Divisor Comum] Dado $a,b\in\mathbb{Z}$, dizemos que $d\in\mathbb{Z}$ {\'e} o m{\'a}ximo divisor comum entre $a$ e $b$ se
\begin{enumerate}
\item $d\geq 0$
\item $d|a$ e $d|b$
\item Se $d'$ {\'e} um inteiro tal que $d'|a$ e $d'|b$, ent{\~a}o $d'|d$ %REVISAR
\end{enumerate}
\end{definicao}

Observa{\c c}{\~o}es:
\begin{enumerate}
\item Se $d$ e $d_{1}$ s{\~a}o m{\'a}ximos divisores comuns entre $a$ e $b$, ent{\~a}o $d=d_{1}$.\\

De fato, dados $d$ e $d_{1}$ m{\'a}ximos divisores comuns de $a$ e $b$, ent{\~a}o temos que $d|a,d|b,d_{1}|a,d_{1}|b$. Mas pelo item 3 da defini{\c c}{\~a}o temos $d|d_{1}$ e $d_{1}|d$. Agora, como $d_{1}\geq 0$ e $d\geq 0$, segue que $d=d_{1}$
\item Se $a=b=0$, segue que $d=d_{1}$ %REVISAR
\item Se $a=0$ e $b\neq 0$, ent{\~a}o $d=|b|$
\item Se $d$ {\'e} o m{\'a}ximo divisor comum entre $a$ e $b$, ent{\~a}o $d$ tamb{\'e}m {\'e} o m{\'a}ximo divisor comum entre $a$ e $-b$, $-a$ e $b$ e entre $-a$ e $-b$.
\end{enumerate}

\begin{nota}[M{\'a}ximo Divisor Comum] Indicaremos por $mdc(a,b)$ o m{\'a}ximo divisor comum ente $a$ e $b$, que j{\'a} sabemos que {\'e} {\'u}nico quando existe.\end{nota}

\begin{proposicao} Quaisquer que sejam $a,b\in\mathbb{Z}$, existe $d\in\mathbb{Z}$ que {\'e} o m{\'a}ximo divisor comum entre $a$ e $b$.\end{proposicao}

\textbf{Demonstra{\c c}{\~a}o}: Das observa{\c c}{\~o}es anteriores podemos considerar somente o caso em que $a>0$ e $b>0$.

Seja $L=\{ax+by/x,y\in\mathbb{Z}\}$. Temos que $L\neq\emptyset$ pois tomando $x=1$ e $y=0$, temos que $m=a1+b0$, pelo princ{\'\i}pio da boa ordena{\c c}{\~a}o, existe $d\in L^{+}$ tal que $d\leq x$, para todo $x\in L^{+}$.

Mostremos que $d=mdc(a,b)$
\begin{enumerate}
\item $d\geq 0$ pois $d\in L^{+}$
\item Como $d\in L^{+}$, existem $x_{0},y_{0}\in\mathbb{Z}$ tais que $d=ax_{0}+by_{0}$.

Agora usando o algoritmo da divis{\~a}o de Euclides para $a$ e $d$ temos que existem $k,r\in\mathbb{Z},0\leq r<d$ tais que $a=kd+r$.

Assim:\[a=k(ax_{0}+by_{0})+r\] \[r=a(1-kx_{0})+b(-y_{0})k\] Da{\'\i}, $r\in L$, mas $r\geq 0$, ent{\~a}o $r\in L^{+}$. Como $d$ {\'e} o m{\'\i}nimo de $L^{+}$ devemos ter $r=0$ e assim $a=kd$, ou seja, $d|a$.

Analogamente, \textit{Mutatis Mutandis}, mostra-se que $d|b$.
\item Seja $d\in\mathbb{Z}$ tal que $d'|a$ e $d'|b$. Temos que $d'|(ax+by)$, para $x,y\in\mathbb{Z}$, em particular, $d'|(ax_{0}+by_{0})=d$, ou seja, $d'|d$.
\end{enumerate}

Portanto, $d=mdc(a,b)$.\#

Observa{\c c}{\~a}o:
\begin{enumerate}
\item Se $d=mdc(a,b)$, ent{\~a}o $d=ax_{0}+by_{0}$, onde $x_{0},y_{0}\in\mathbb{Z}$. Os elementos $x_{0}$ e $y_{0}$ satisfazem que tal igualdade n{\~a}o {\'e} {\'u}nica.
\item Uma igualdade do tipo $d=ax_{0}+by_{0}$ {\'e} chamada de Identidade de Bezout

Exemplos:
\begin{enumerate}
\item $mdc(2,3)=1$\\
$1=2(-1)+3.1=2.2+3(-1)$
\item $mdc(4,8)=4$\\
\end{enumerate}
\end{enumerate}

Considere os seguintes subconjuntos de $\Z$:
\begin{align*}
	I &= \{2k \mid k\in\Z\} = \{0,\pm 2,\pm 4,\pm 6,...\}\\
	J &= \{2r+1\mid r\in\Z\} = \{\pm 1,\pm 3,\pm 5,...\}.
\end{align*}

Dados quaisquer $a,b\in I$, temos $a+b\in I$. Al{\'e}m disso, dado $n\in\Z,\ na\in I$. Por outro lado, $1,3\in J$ mas $1+3=4\notin J$.

\section{Ideais}

\subsubsection{Defini{\c c}{\~a}o}
\begin{definicao}
Um subconjunto n{\~a}o vazio $S \subseteq\Z$ {\'e} chamado de um \textbf{ideal} de $\Z$ se satisfaz as seguintes condi{\c c}{\~o}es:\index{Ideal}
\begin{enumerate}
\item $r_1 + r_2\in S$, para todos $r_1$, $r_2\in S$,
\item $nr \in S$, para todo $n \in\Z$ e para todo $r \in S$.
\end{enumerate}
\end{definicao}

\subsubsection{Propriedades}
Seja S um ideal de $\mathbb{Z}$. Então:
\begin{enumerate}
\item $r_1 - r_2\in S$, para todos $r_1$, $r_2 \in S$, pois $r_1 - r_2 = r_1 + (-r_2)$.
\item $0 \in S$, pois $0 = r - r$, para qualquer $r\in S$.
\end{enumerate}

Exemplos:
\begin{enumerate}
\item $S = \{2k \mid k \in \Z\}$ {\'e} um ideal de $\mathbb{Z}$.
\item $S=\{0\}$ e $S=\mathbb{Z}$ s{\~a}o ideais de $\mathbb{Z}$, chamados de \textbf{ideais triviais}.
\item Dado $a,b,c\in\mathbb{Z}$, o subconjunto $S=\{ax+by/x,y\in\mathbb{Z}\}$ {\'e} um ideal de $\mathbb{Z}$.

$S\neq\emptyset$ pois $0=a0+b0\in S$

Sejam $ax_{1}+by_{1},ax_{2}+by_{2}\in S$. Temos $(ax_{1}+by_{1})+(ax_{2}+by_{2})=a(x_{1}+x_{2})+b(y_{1}+y_{2})\in S$

Agora, sejam $ax_{1}+by{1}\in S$ e $n\in\mathbb{Z}$ temos \[n(ax_{1}+by_{1})=a(nx_{1})+b(ny_{1})\in S\]
\end{enumerate}

De modo geral, dados $a_{1},a_{2},...,a_{n}$ n{\'u}meros inteiros, o subconjunto \[S=\{a_{1}x_{1}+a_{2}x_{2}+...+a_{n}x_{n}/x_{1},...,x_{n}\in\mathbb{Z}\}\] {\'e} um ideal de $\mathbb{Z}$.

Se $S$ {\'e} ideal de $\mathbb{Z}$, ent{\~a}o $S=\{nk/n\in\mathbb{Z}\}$.
\subsubsection{Conjunto dos m{\'u}ltiplos de $g$}
\begin{nota}[Conjunto dos m{\'u}ltiplos de $g$] Se $g\in\mathbb{Z}$, denotamos por $g\mathbb{Z}$, ou $\mathbb{Z}g$, o subconjunto dos inteiros que s{\~a}o m{\'u}ltiplos de $g$ (os inteiros que s{\~a}o divis{\'\i}veis por $g$). Em outras palavras \[g\mathbb{Z}=\{gn/n\in\mathbb{Z}\}=\{0,\pm g,\pm 2g,\pm 3g,...\}\]
\end{nota}

\begin{teorema} Seja $S$ um ideal de $\mathbb{Z}$. Ent{\~a}o, existe um n{\'u}mero $g\in\mathbb{Z}$ tal que $S=g\mathbb{Z}$.\end{teorema}

\textbf{Demonstra{\c c}{\~a}o}: Se $S=\{0\}$, ent{\~a}o tomamos $g=0$ e da{\'\i} $S=0\mathbb{Z}$. Se $S=\mathbb{Z}$, ent{\~a}o $g=1$ e $S=1\mathbb{Z}$.

Assim podemos supor $S\neq\{0\}$ e $S\neq\mathbb{Z}$. Seja $S^{+}=\{x\in S/x>0\}$. Do {\'\i}tem 2 da defini{\c c}{\~a}o de ideal, segue que $S^{+}\neq\emptyset$. Assim, pelo princ{\'\i}pio da boa ordena{\c c}{\~a}o, existe $g\in S^{+}$ tal que $g\leq x,\forall x\in S^{+}$.

Como $g\in S^{+}\subseteq S$ e $S$ {\'e} um ideal de $\mathbb{Z}$, ent{\~a}o $gn\in S\forall n\in\mathbb{Z}$, ou seja, $g\mathbb{Z}\subseteq S$.

Agora precisamos mostrar que $a=gq$, onde $q\in\mathbb{Z}$. Assim, dado $a\in S$, o algoritmo da divis{\~a}o de Euclides garante que existem $q,r\in\mathbb{Z}$ tais que $a=gq+r$, onde $0\leq r<g$. Como $a,q,g\in S$ e $S$ {\'e} um ideal, ent{\~a}o $r=a-gq\in S$. Se $r>0$, ent{\~a}o como $r<g$ e $g$ {\'e} o m{\'\i}nimo de $S^{+}$ obtemos uma contradi{\c c}{\~a}o. Logo, $r=0$ e $a=gp$. Da{\'\i} $S\subseteq g\mathbb{Z}$. Portanto $s=g\mathbb{Z}$.\#

Exemplo: O conjunto $S=\{2x-5y/x,y\in\mathbb{Z}\}$ {\'e} ideal de $\mathbb{Z}$. Neste caso,\\ $S^{+}=\{1,2,3,...\}$. Assim, $g=1$ e $S=1\mathbb{Z}=\mathbb{Z}$.
%!TEX program = xelatex
%!TEX root = Algebra_1.tex
%%Usar makeindex -s indexstyle.ist arquivo.idx no terminal para gerar o {\'\i}ndice remissivo agrupado por inicial
%%Ap\'os executar pdflatex arquivo
\chapter{Rela{\c c}{\~o}es e Fun{\c c}{\~o}es}
\section{Rela{\c c}{\~o}es}
\subsubsection{Defini{\c c}{\~a}o}
Sejam A e B dois conjuntos n{\~a}o vazios. Os subconjuntos de AxB s{\~a}o chamados rela{\c c}{\~o}es, ou seja, uma rela{\c c}{\~a}o em AxB {\'e} um subconjunto desse produto cartesiano.

Quando $R$ {\'e} uma rela{\c c}{\~a}o em $A \times B$, tamb{\'e}m dizemos que $R$ {\'e} uma rela{\c c}{\~a}o de A em B.

Exemplos:
\begin{enumerate}
\item Se A=\{0,1\} e B=\{-1,0,1\}, ent{\~a}o AxB=\{(0,-1),(0,0),(0,1),(1,-1),(1,0),(1,1,)\}\\
S{\~a}o exemplos de rela{\c c}{\~o}es:\\
$R_{1}=\{(0,1)\}$\\
$R_{2}=\emptyset$\\
$R_{3}=\{(1,-1),(1,1)\}$\\
$R_{4}=A$x$B$
\item Se $A=B=\mathbb{R}$, ent{\~a}o AxB {\'e} o conjunto formado por todos pares ordenados de n{\'u}meros reais. Um exemplo de rela{\c c}{\~a}o em $\mathbb{R}$x$\mathbb{R}$ {\'e} o conjunto:\\
$R=\{(x,y)\in \mathbb{R}$x$\mathbb{R}/ y\geq 0\}$
\end{enumerate}

\section{Rela{\c c}{\~o}es de equival{\^e}ncia}
\subsubsection{Defini{\c c}{\~a}o}
\begin{definicao}[Rela{\c c}{\~a}o de equival{\^e}ncia] Seja X um conjunto n{\~a}o vazio e $R\subseteq X \times X$ uma rela{\c c}{\~a}o. Dizemos que R {\'e} uma rela{\c c}{\~a}o de equival{\^e}ncia se:
\begin{enumerate}
\item Para todo $a \in X$, $(a,a)\in R$.
\item Se $(a, b) \in R$, então $(b, a) \in R$.
\item Se $(a, b) \in R$ e $(b, c) \in R$, então $(a, c)\in R$.
\end{enumerate}
\end{definicao}

Quando $R\subseteq X$x$X$ {\'e} uma rela{\c c}{\~a}o de equival{\^e}ncia, dizemos que R {\'e} uma rela{\c c}{\~a}o de equival{\^e}ncia em X. Quando 2 elementos $a,b\in X$ s{\~a}o tais que $(a,b)\in R$, dizemos que a e b s{\~a}o relacionados.\\

\subsection{Equival{\^e}ncia m{\'o}dulo R}

\begin{nota}[Equival{\^e}ncia m{\'o}dulo R]
Seja R uma rela{\c c}{\~a}o de equival{\^e}ncia em X. Para dizermos que $(a,b)\in R$ usaremos a nota{\c c}{\~a}o $a\equiv b(R)$, que se l{\^e} ``a é equivalente a b m{\'o}dulo R", ou ainda a nota{\c c}{\~a}o $aRb$, com o mesmo significado anterior.
\end{nota}

Exemplos:
\begin{enumerate}
\item Seja X=\{1,2,3\}. Temos $X\times X=\{(1,1),(1,2),(1,3),(2,1),(2,2),(2,3),(3,1),(3,2),(3,3)\}$.
S{\~a}o exemplos de rela{\c c}{\~o}es de equival{\^e}ncia:\\
$R_{1}=X\times X$\\
$R_{2}=\{(1,1),(2,2),(3,3)\}$\\
$R_{3}=\{(1,1),(2,2),(3,3),(1,2),(2,1)\}$
\item Seja $X=\mathbb{Z}$ e $R\subseteq \mathbb{Z}\times \mathbb{Z}$ definida por $R=\{(x,y)\in \mathbb{Z} \times \mathbb{Z} \mid x=y\}$
R {\'e} uma rela{\c c}{\~a}o de equival{\^e}ncia pois:
\begin{itemize}
\item $\forall a \in \mathbb{Z}, (a,a) \in R$ pois a=a
\item $(a,b)\in R \rightarrow a=b \wedge b=a \Leftrightarrow (b,a)\in R$
\item $(a,b),(b,c)\in R \rightarrow a=b=c\Rightarrow (a,c)\in R$
\end{itemize}
\item Tome $R=\{(x,y)\in \mathbb{Z}$x$\mathbb{Z}/ 2|(x-y)\}=\{(x,y)\in\mathbb{Z}$x$\mathbb{Z}/ x-y=2k, k\in\mathbb{Z}\}$\\
R {\'e} uma rela{\c c}{\~a}o de equival{\^e}ncia pois:
\begin{itemize}
\item $\forall x\in\mathbb{Z},xRx$ pois $x-x=2.0$
\item $xRy\rightarrow x-y=2k\Rightarrow y-x=-(x-y)=2.(-k)\Rightarrow yRx$
\item $xRy\wedge yRz\rightarrow x-y=2k\wedge y-z=2q\Rightarrow x+z=x-y+y-z=2k+2q=2(k+q)\rightarrow xRz$

\end{itemize}
\end{enumerate}

\subsection{Classe de equival{\^e}ncia e conjunto quociente}
\begin{definicao}[Classe de Equival{\^e}ncia] Seja R uma rela{\c c}{\~a}o de equival{\^e}ncia sobre um conjunto X. Dado $a\in X$, chamamos classe de equival{\^e}ncia determinada por a m{\'o}dulo R, denotada por $\bar{a}$ ou C(a), o subconjunto constitu{\'\i}do pelos elementos $b\in X$ tais que bRa, ou seja, $\bar{a}=C(a)=\{a\in X/ bRa\}$\end{definicao}

\begin{definicao}[Conjunto quociente]
O conjunto das classes de equival{\^e}ncia m{\'o}dulo R ser{\'a} denotado por $X/R$ e {\'e} chamado conjunto quociente de X por R.
\end{definicao}

Observa{\c c}{\~a}o: Dado um conjunto $X\neq\emptyset$ e R uma rela{\c c}{\~a}o de equival{\^e}ncia em X, dado $a\in X$ como R {\'e} uma rela{\c c}{\~a}o de equival{\^e}ncia, aRa, da{\'\i} $\bar{a}\neq\emptyset$, pois $a\in\bar{a}$\\

Exemplos:
\begin{enumerate}
\item Seja X=\{a,b,c\} e R=\{(a,a),(b,b),(c,c),(a,c),(c,a)\}. Temos:\\
$\bar{a}=\{x\in X/xRa\}=\{a,c\}$\\
$\bar{b}=\{x\in X/xRb\}=\{b\}$\\
$\bar{c}=\{x\in X/xRc\}=\{a,c\}$
\item Seja X=\{1,2,3,4\} e a rela{\c c}{\~a}o de equival{\^e}ncia R=\{(1,1),(2,2),(3,3),(4,4)\}\\
$\bar{1}=\{x\in X/xR1\}=\{1\}$\\
$\bar{2}=\{x\in X/xR2\}=\{2\}$\\
$\bar{3}=\{x\in X/xR3\}=\{3\}$\\
$\bar{4}=\{x\in X/xR4\}=\{4\}$
\end{enumerate}

\begin{proposicao} Seja R uma rela{\c c}{\~a}o de equival{\^e}ncia em um conjunto n{\~a}o vazio X, sejam a,b$\in$X. Se $\bar{a}\cap\bar{b}\neq\emptyset$, ent{\~a}o aRb.\end{proposicao}

\textbf{Demonstra{\c c}{\~a}o}: Como  $\bar{a}\cap\bar{b}\neq\emptyset$, existe um $y\in\bar{a}\cap\bar{b}$, logo $y\in\bar{a}\wedge y\in\bar{b}$. Da defini{\c c}{\~a}o de classe de equival{\^e}ncia temos que yRa e yRb. Como R {\'e} rela{\c c}{\~a}o de equival{\^e}ncia temos que aRy e bRy. Por transitividade, aRb, como quer{\'\i}amos demonstrar.\#

\begin{proposicao} Se  $\bar{a}\cap\bar{b}\neq\emptyset$, ent{\~a}o $\bar{a}=\bar{b}$\end{proposicao}

\textbf{Demonstra{\c c}{\~a}o}: Seja $y\in \bar{a}$. Da{\'\i} yRa. Como $\bar{a}\cap\bar{b}\neq\emptyset$, pela proposi{\c c}{\~a}o anterior, aRb. Logo, como yRa e aRb, segue que yRb, ou seja, $y\in\bar{b}$. Da{\'\i} $\bar{a}\subseteq\bar{b}$. Como no caso anterior, mostra-se que $\bar{b}\subseteq\bar{a}$. Portanto $\bar{a}=\bar{b}$.\#

\begin{corolario} As classes de equival{\^e}ncia s{\~a}o conjuntos disjuntos ou iguais.\end{corolario}

Seja R uma rela{\c c}{\~a}o de equival{\^e}ncia em $X\neq\emptyset$, dado $a\in R$. Se bRa, ent{\~a}o $\bar{b}=\bar{a}$, mais ainda, se dRa ent{\~a}o $\bar{d}=\bar{a}=\bar{b}$. Como por exemplo:\\
X=\{a,b,c,d,e,f,g\}\\
$\bar{a}=\{a,b,c\}$\\
$\bar{e}=\{e\}$\\
$\bar{f}=\{f,g\}$

\begin{definicao}[Representante da Classe de Equival{\^e}ncia] Seja C uma classe de equival{\^e}ncia de uma rela{\c c}{\~a}o de equival{\^e}ncia R. Qualquer elemento $y\in C$ {\'e} chamado representante de C.\end{definicao}

\begin{proposicao} Seja X um conjunto n{\~a}o vazio e R uma rela{\c c}{\~a}o de equival{\^e}ncia em X. Ent{\~a}o X {\'e} a uni{\~a}o disjunta das classes $\bar{a}, a\in X$, ou seja, \[X=\displaystyle\bigsqcup_{a\in X}\bar{a}\].\end{proposicao}

\textbf{Demonstra{\c c}{\~a}o}: Para todo $a\in X, \bar{a}\subseteq X$, logo $\displaystyle\bigsqcup_{a\in X}\bar{a}\subseteq X$. Seja $b\in X$. Logo $b\in\bar{b}$, da{\'\i} $b\in \displaystyle\bigsqcup_{a\in X}\bar{a}$, logo $X\subseteq\displaystyle\bigsqcup_{a\in X}\bar{a}$. Portanto, $X=\displaystyle\bigsqcup_{a\in X}\bar{a}$.\#

Exemplo:\\
Em $\mathbb{Z}$x$\mathbb{Z}$ considere a seguinte rela{\c c}{\~a}o: $R=\{(a,b)\in \mathbb{Z}$x$\mathbb{Z}/2|(a-b)\}$. Mostre que {\'e} uma rela{\c c}{\~a}o de equival{\^e}ncia e mostre suas classes de equival{\^e}ncia.
\begin{enumerate}
\item Dado $a\in \mathbb{Z}$, aRa pois $2|(a-a)=0$.
\item Se aRb, ent{\~a}o $2|(a-b)$, ou seja, a-b=2k, -(a-b)=b-a=2(-k). Logo bRa.
\item Se aRb e bRc, ent{\~a}o a-b=2k e b-c=2q. Logo a-b+b-c=2k+2q=2(k+q). Logo, aRc.

\end{enumerate}

Portanto R {\'e} uma rela{\c c}{\~a}o de equival{\^e}ncia.

Dado $a\in\mathbb{Z}$ , temos:\\
$\bar{a}=\{b\in\mathbb{Z}/bRa\}=\{b\in\mathbb{Z}/2|(a-b)\}$ como $2|(a-b)$, temos que:\\
$a-b=2k\Leftrightarrow b=a+2r, r=-k$

Assim, se a {\'e} {\'\i}mpar, b tamb{\'e}m o {\'e}. Logo:\\
$\bar{a}=\{...,-3,-1,1,3,...\}$

Agora, se a {\'e} par, b tamb{\'e}m {\'e}. Logo:\\
$\bar{a}=\{...,-2,0,2,4,...\}$

\section{Fun{\c c}{\~o}es}

\subsubsection{Defini{\c c}{\~a}o}
\begin{definicao}[Fun{\c c}{\~a}o] Uma fun{\c c}{\~a}o $f$ de um conjunto A em um conjunto B {\'e} uma rela{\c c}{\~a}o $f\subseteq A\times B$ satisfazendo:
\begin{enumerate}
\item $\forall x\in A,\exists y\in B/(x,y)\in f$
\item $(x_{1},y_{1}),(x_{1},y_{2})\in f \rightarrow y_{1}=y_{2}$
\end{enumerate}
\end{definicao}

Geralmente, para dizer que $f$ {\'e} uma fun{\c c}{\~a}o de A em B escrevemos $f:A\rightarrow B$.

\subsubsection{Dom{\'\i}nio e contra-dom{\'\i}nio}
O conjunto A {\'e} chamado de Dom{\'\i}nio de $f$ e o conjunto B {\'e} chamado de contra-dom{\'\i}nio.

Se $f:A\rightarrow B$ {\'e} uma fun{\c c}{\~a}o, escrevemos $f(a)=b$ para dizer que $(a,b)\in f$

Exemplos:
\begin{enumerate}
\item Sejam A=\{0,1,2,3\} e B=\{4,5,6,7,8\}. Quais das seguintes rela{\c c}{\~o}es s{\~a}o fun{\c c}{\~o}es?
\begin{itemize}
\item $R_{1}=\{(0,5),(1,6),(2,7)\}$ - N{\~a}o {\'e} fun{\c c}{\~a}o pois o n{\'u}mero 3 n{\~a}o t{\^e}m valor associado {\`a} ele.
\item $R_{2}=\{(0,4),(1,5),(1,6),(2,7),(3,8)\}$ - N{\~a}o {\'e} fun{\c c}{\~a}o pois o valor 1 tem mais de um valor diferente associado {\`a} ele.
\item $R_{3}=\{(0,4),(1,5),(2,7),(3,8)\}$ - {\'E} fun{\c c}{\~a}o
\item $R_{4}=\{(0,5),(1,5),(2,6),(3,7)\}$ - {\'E} fun{\c c}{\~a}o

\end{itemize}
\item $R_{5}=\{(x,y)\in\mathbb{R}$x$\mathbb{R}/y^{2}=x^{2}\}$ - N{\~a}o {\'e} fun{\c c}{\~a}o, pois $x=\pm \sqrt{y}$
\item $R_{6}=\{(x,y)\in\mathbb{R}$x$\mathbb{R}/x^{2}+y^{2}=1\}$ - N{\~a}o {\'e} fun{\c c}{\~a}o pois quando\\ $x=0,y=1\wedge y=-1$
\item  $R_{7}=\{(x,y)\in\mathbb{R}$x$\mathbb{R}/y=x^{2}\}$ - {\'E} fun{\c c}{\~a}o
\end{enumerate}

\subsection{Tipos de fun{\c c}{\~o}es}

\begin{definicao}[Fun{\c c}{\~a}o sobrejetora]  Uma fun{\c c}{\~a}o $f:A\rightarrow B$ {\'e} sobrejetora se, e somente se, para todo $y\in B$ exista um $x\in A$ tal que $f(x)=y$\end{definicao}

\begin{definicao}[Fun{\c c}{\~a}o injetora] Uma fun{\c c}{\~a}o $f:A\rightarrow B$ {\'e} injetora se, e somente se, para $a_{1}\neq a_{2}$, temos $f(a_{1})\neq f(a_{2}), \forall a_{1},a_{2}\in A$\end{definicao}

\begin{definicao}[Fun{\c c}{\~a}o bijetora] Uma fun{\c c}{\~a}o $f:A\rightarrow B$ que {\'e} simultaneamente injetora e sobrejetora {\'e} chamada de bijetora ou bijetiva.
\end{definicao}

Exemplos:
\begin{enumerate}
\item A fun{\c c}{\~a}o $f:\mathbb{R}\rightarrow\mathbb{R}$ dada por $f(x)=3x+1$ {\'e} injetora e sobrejetora.

Dados $x_{1}, x_{2}\in\mathbb{R}$ tais que $f(x_{1})=f(x_{2})$, temos:
\[3x_{1}+1=3x_{2}+1\]
\[x_{1}=x_{2}\]

Logo $f$ {\'e} injetora

Para verificar se $f$ {\'e} sobrejetora precisamos verificar se dado $y\in\mathbb{R}\\ \ \exists x\in\mathbb{R}/f(x)=y$.

Tome $x=\displaystyle\frac{y-1}{3}\in\mathbb{R}$. Da{\'\i}, $f(x)=y$. Logo $f$ {\'e} sobrejetora.
\item A fun{\c c}{\~a}o $f:\mathbb{R}\rightarrow\mathbb{R}$ dada por $f(x)=x^{2}$ {\'e} injetora? E sobrejetora?

N{\~a}o {\'e} injetora pois $f(-1)=f(1)\wedge 1\neq -1$

N{\~a}o {\'e} sobrejetora pois $\nexists x\in\mathbb{R}/x^{2}=-1$

\end{enumerate}

Dado $f:A\rightarrow B$ uma fun{\c c}{\~a}o, considere a rela{\c c}{\~a}o $f^{-1}\subseteq B$x$A$ tal que $(b,a)\in f^{-1}$ se $(a,b)\in f$, ou seja, $f^{-1}(b)=a$ se $f(a)=b$.

Pode ocorrer que $f^{-1}$ n{\~a}o seja fun{\c c}{\~a}o, mesmo $f$ sendo uma fun{\c c}{\~a}o. Por exemplo:

$f:\{0,1,2,3\}\rightarrow\{4,5,6,7,8\}$ dada por:\\
$f(0)=5$\\
$f(1)=5$\\
$f(2)=6$\\
$f(3)=7$

Neste caso, $f^{-1}$ {\'e} dado por:\\
$f^{-1}(5)=0$\\
$f^{-1}(5)=1$\\
$f^{-1}(6)=2$\\
$f^{-1}(7)=3$

\begin{teorema}
Dada $f:A\rightarrow B$ fun{\c c}{\~a}o tome $f^{-1}:B\rightarrow A$. Definida com o $f^{-1}(b)=a$ se $f(a)=b$. Ent{\~a}o $f^{-1}$ {\'e} uma fun{\c c}{\~a}o se, e somente se, $f$ {\'e} bijetora.
\end{teorema}

\textbf{Demonstra{\c c}{\~a}o}: Suponha $f^{-1}$ {\'e} fun{\c c}{\~a}o. Precisamos provar que $f$ {\'e} injetora e sobrejetora.

Dados $a_{1},a_{2}\in A$ tais que $f(a_{1})=b=f(a_{2})$. Como $f(a_{1})=b$ temos $f^{-1}(b)=a_{1}$, al{\'e}m disso, $f^{-1}(b)=a_{2}$. Mas $f^{-1}$ {\'e} fun{\c c}{\~a}o, da{\'\i} $a_{1}=a_{2}$, ou seja, $f$ {\'e} injetora.

Dado $b\in B$, como $f^{-1}$ {\'e} uma fun{\c c}{\~a}o, $\forall b\in B, f^{-1}(b)=a\in A$, logo $f(a)=b$ e assim $f$ {\'e} sobrejetora.

Portanto $f$ {\'e} bijetora.

Agora suponha que $f$ {\'e} bijetora.

Primeiramente, dado $b\in B$, como $f$ {\'e} sobrejetora, existe $a\in A$ tal que $f(a)=b$, ou seja, $f^{-1}(b)=a\in A$.

Suponha que $f^{-1}(b)=a_{1}$ e $f^{-1}(b)=a_{2}$. Da{\'\i}, $f(a_{1})=b\wedge f(a_{2})=b$. Mas $f$ {\'e} injetora, assim $a_{1}=a_{2}$ e ent{\~a}o $f^{-1}(b)=a_{1}=a_{2}$.

Portanto $f^{-1}$ {\'e} fun{\c c}{\~a}o. \#
\subsection{Composi{\c c}{\~a}o de fun{\c c}{\~o}es}

\subsubsection{Defini{\c c}{\~a}o}

\begin{definicao}[Fun{\c c}{\~a}o Composta] Sejam $f:A\rightarrow B$ e $g:B\rightarrow C$ fun{\c c}{\~o}es. Chama-se composta de $g$ e $f$ a fun{\c c}{\~a}o de A em C, denotada $g\circ f$, definida por $g\circ f:A\rightarrow C$.\end{definicao}

Temos ent{\~a}o que $(g\circ f)(x)=g(f(x)), \forall x\in A$.

Observa{\c c}{\~a}o: Se $f:A\rightarrow B$ e $g:B\rightarrow A$ ent{\~a}o existem $f\circ g$ e $g\circ f$. Por{\'e}m, em geral, $f\circ g\neq g\circ f$.

\subsubsection{Propriedades}
\begin{proposicao} Se $f:A\rightarrow B$ e $g:B\rightarrow C$ s{\~a}o fun{\c c}{\~o}es injetoras, ent{\~a}o $g\circ f$ {\'e} injetora.\end{proposicao}

\textbf{Demonstra{\c c}{\~a}o}: Dados $x_{1},x_{2}\in A$ tais que $(g\circ f)(x_{1})=(g\circ f)(x_{2})$ temos que $g(f(x_{1}))=g(f(x_{2}))$. Como $g$ {\'e} injetora, $f(x_{1})=f(x_{2})$. Mas $f$ {\'e} injetora, da{\'\i} $x_{1}=x_{2}$. Logo $g\circ f$ {\'e} injetora.\#

\begin{proposicao} Se $f:A\rightarrow B$ e $g:B\rightarrow C$ s{\~a}o sobrejetoras, ent{\~a}o $g\circ f$ {\'e} sobrejetora.\end{proposicao}

\textbf{Demonstra{\c c}{\~a}o}: Temos que $g\circ f:A\rightarrow C$. Dado $z\in C$. Como $g$ {\'e} sobrejetora, $\exists y\in B/g(y)=z$. Como $f$ {\'e} sobrejetora, $\exists x\in A/f(x)=y$. Assim, $z=g(y)=g(f(x))=(g\circ f)(x)$. Logo $g\circ f$ {\'e} sobrejetora.\#

\subsection{Fun{\c c}{\~a}o Identidade}
\subsubsection{Defini{\c c}{\~a}o}
\begin{definicao}[Fun{\c c}{\~a}o Identidade] Dado um conjunto $A\neq\emptyset$, a fun{\c c}{\~a}o $i_{A}:A\rightarrow A$ dada por $i_{A}(x)=(x)$ {\'e} chamada de fun{\c c}{\~a}o identidade.\end{definicao}

\begin{proposicao}
Se $f:A\rightarrow B$ {\'e} bijetora, ent{\~a}o $f\circ f^{-1}=i_{B}\wedge f^{-1}\circ f=i_{A}$.
\end{proposicao}

\textbf{Demonstra{\c c}{\~a}o}: Temos $i_{F}:F\rightarrow F$ e $i_{E}:E\rightarrow E$. Al{\'e}m disso, $f\circ f^{-1}:F\rightarrow F$ e $f^{-1}\circ f:E\rightarrow E$, da{\'\i} $D(f\circ f^{-1})=D(i_{F})$\footnote{$D(f(x))$ {\'e} o dom{\'\i}nio da fun{\c c}{\~a}o $f$} e $D(f^{-1}\circ f)=D(i_{E})$. Dado $x\in F, (f\circ f^{-1})(x)=f(f^{-1}(x))=x=i_{F}(x)$. Dado $x\in E, (f^{-1}\circ f)(x)=f^{-1}(f(x))=x=i_{E}(x)$.\#

\subsubsection{Propriedades}
\begin{proposicao} Se $f:A\rightarrow B$ e $g:B\rightarrow A$ s{\~a}o fun{\c c}{\~o}es, ent{\~a}o:
\begin{enumerate}
\item $f\circ i_{A}=f, i_{B}\circ f=f, g\circ i_{B}=g, i_{E}\circ g=g$
\item Se $g\circ f=i_{A}$, e $f\circ g=i_{B}$, ent{\~a}o $f$ e $g$ s{\~a}o bijetoras e $g=f^{-1}$
\end{enumerate}
\end{proposicao}

\textbf{Demonstra{\c c}{\~a}o}:
\begin{enumerate}
\item Provemos que $f\circ i_{A}=f$.\\
Primeiro temos $f:A\rightarrow B$ e $i_{A}:A\rightarrow A$. Da{\'\i}, $f\circ i_{A}:A\rightarrow B$, ou seja, $D(f\circ i_{A})=D(f)$. Dado $x\in A$, temos $(f\circ i_{A})(x)=f(i_{A}(x))=f(x)$. Portanto, $f\circ i_{A}=f$.
\item Provemos que $f$ {\'e} bijetora.\\
Dados $x_{1}$, $x_{2}\in B$ tais que $f(x_{1})=f(x_{2})$. Como $f:A\rightarrow B$ e $g:B\rightarrow A$, ent{\~a}o $g(f(x_{1}))=g(f(x_{2}))$, ou seja, $(g\circ f)(x_{1})=(g\circ f)(x_{2})$. Da{\'\i}, $i_{A}(x_{1})=i_{A}(x_{2})$. Logo, $x_{1}=x_{2}$, isto {\'e}, $f$ {\'e} injetora.

Agora, dado $y\in B$, segue que $y=i_{B}(y)$. Mas $i_{B}=f\circ g$. Da{\'\i}, $y=i_{B}(y)=(f\circ g)(y)=f(g(y))$. Assim, $x=g(y)\in A$ e $f(x)=y$.

Logo $f$ {\'e} sobrejetora. Portanto $f$ {\'e} bijetora. Analogamente, prova-se que g {\'e} bijetora. Provemos que $g=f^{-1}$. Temos  $f^{-1}:B\rightarrow A$, da{\'\i}, $D(g) = B = D(f^{-1})$. Agora, $f\circ g = i_{B} = f\circ f^{-1}$. Assim, para todo $x\in F$, $(f\circ g)(x)=(f\circ f^{-1})(x)$. Isto {\'e}, $f(g(x))=f(f^{-1}(x))$. Portanto, $g(x)=f^{-1}(x)\forall x\in B$. Logo, $g=f^{-1}$. \#
\end{enumerate}

\begin{definicao}
Seja $f : A \to B$ uma fun{\c c}{\~a}o.
\begin{enumerate}
\item Dado $P \sub A$, chama-se {\rm imagem direta} de $P$, segundo $f$ e indica-se por $f(P)$ o subconjunto de $F$ dado por
\[
f(P) = \{f(x) \mid x \in P\},
\]
isto {\'e}, $f(P)$ {\'e} o conjunto das imagens por $f$ dos elementos de $P$.

\item Dado $Q \sub B$, chama-se {\rm imagem inversa} de $Q$, segundo $f$ e indica-se por $f^{-1}(Q)$ o subconjunto de $A$ dado por
\[
f^{-1}(Q) = \{x \in E \mid f(x) \in Q\},
\]
isto {\'e}, $f^{-1}(Q)$ {\'e} o conjunto dos elementos de $A$ que tem imagem em $Q$ atrav{\'e}s de $f$.
\end{enumerate}
\end{definicao}

{\it Exemplos:}
\begin{enumerate}
\item Seja $A = \{1, 3, 5, 7, 9 \}$ e $B = \{0, 1, 2, 3, \dots, 10\}$ e $f : A \to B$ dada por $f(x) = x + 1$. Temos que
\begin{itemize}
\item $f(\{3, 5, 7\}) = \{f(3), f(5), f(7)\} = \{4, 6, 8\}$

\item $f(A) = \{f(1), f(3), f(5), f(7), f(9)\} = \{2, 4, 6, 8, 10\}$

\item $f(\emptyset) = \emptyset$

\item $f^{-1}(\{2, 4, 10\}) = \{x \in A \mid f(x) \in \{2, 4, 10\}\} = \{1, 3, 9\}$

\item $f^{-1}(\{0, 1, 3, 5, 7, 9\}) = \{x \in A \mid f(x) \in \{0, 1, 3, 5, 7, 9\}\} = \emptyset$
\end{itemize}

\item Sejam $A = B= \real$ e $f : \real \to \real$ dada por $f(x) = x^2$. Temos
\begin{itemize}
\item $f(\{1, 2, 3\}) = \{1, 4, 9\}$

\item $f([0,2]) = \{f(x) \in \real \mid 0 \le x \le 2 \} = \{x^2 \mid 0 \le x \le 2\} = [0, 4]$

\item $f^{-1}([1, 9]) = \{ x \in \real \mid 1 \le f(x) \le 9\} = \{x \in \real \mid 1 \le x^2 \le 9\} = [-1, -3] \cup [1, 3]$
\end{itemize}
\end{enumerate}

\begin{proposicao}
Seja $f : A \to B$ uma aplica{\c c}{\~a}o (ou fun{\c c}{\~a}o) e sejam $P$, $Q \sub E$, $X$, $Y \sub B$.
\begin{enumerate}
\item Se $P \sub Q$, ent{\~a}o $f(P) \sub f(Q)$.

\item $f^{-1}(X \cup Y) = f^{-1}(X) \cup f^{-1}(Y)$.
\end{enumerate}
\end{proposicao}

\textbf{Demonstra{\c c}{\~a}o}: 
\begin{enumerate}
\item Se $y \in f(P)$, ent{\~a}o existe $x \in P$ tal que $f(x) = y$. Mas como $P \sub Q$, ent{\~a}o $x \in Q$ e da{\'\i} $y \in f(Q)$. Logo $f(P) \sub f(Q)$.

\item Seja $z \in f^{-1}(X \cup Y)$. Ent{\~a}o $f(z) \in X \cup Y$. Se $f(z) \in X$, entao $z \in f^{-1}(X)$ e da{\'\i} $z \in f^{-1}(X) \cup f^{-1}(Y)$. Se $f(z) \in Y$, ent{\~a}o $z \in f^{-1}(Y)$ e assim $z \in f^{-1}(X) \cup f^{-1}(Y)$. Logo, $f^{-1}(X \cup Y) \sub f^{-1}(X) \cup f^{-1}(Y)$.

Agora, seja $z \in f^{-1}(X) \cup f^{-1}(Y)$. Se $z \in f^{-1}(X)$, ent{\~a}o $f(z) \in X$, da{\'\i} $f(z) \in X \cup Y$, isto {\'e}, $z \in f^{-1}(X \cup Y)$. Se $z \in f^{-1}(Y)$, ent{\~a}o $f(z) \in Y$ e assim $f(z) \in X \cup Y$, isto {\'e}, $z \in f^{-1}(X \cup Y)$. Logo $f^{-1}(X) \cup f^{-1}(Y) \sub f^{-1}(X \cup Y)$.

Portanto, $f^{-1}(X \cup Y) = f^{-1}(X) \cup f^{-1}(Y)$. \#
\end{enumerate}
%!TEX program = xelatex
%!TEX root = Algebra_1.tex
%%Usar makeindex -s indexstyle.ist arquivo.idx no terminal para gerar o {\'\i}ndice remissivo agrupado por inicial
%%Ap\'os executar pdflatex arquivo
\chapter{Opera{\c c}{\~o}es em $\dfrac{\mathbb{Z}}{m\mathbb{Z}}$}

Durante esse t{\'o}pico, $m$ denotar{\'a} um n{\'u}mero inteiro positivo.

\section{Rela{\c c}{\~o}es de congru{\^e}ncia}
\subsection{Defini{\c c}{\~a}o}

\begin{definicao}[Congru{\^e}ncia] Sejam $a,b\in\mathbb{Z}$, dizemos que $a$ {\'e} congruente com $b$ m{\'o}dulo $m$ se $m|(a-b)$. Neste caso, escrevemos $a\equiv_{m}b$ ou $a\equiv b(mod\ m)$.\end{definicao}

Exemplos:
\begin{enumerate}
\item $5\equiv 2(mod\ 3)$, pois $3|(5-2)$
\item $3\equiv 1(mod\ 2)$, pois $2|(3-1)$
\item $3\equiv 9(mod\ 3)$, pois $2|(3-9)$

\end{enumerate}
\subsection{Propriedades}
\begin{proposicao} A congru{\^e}ncia m{\'o}dulo $m$ {\'e} uma rela{\c c}{\~a}o de equival{\^e}ncia em $\mathbb{Z}$.\end{proposicao}

\textbf{Demonstra{\c c}{\~a}o}: 
\begin{enumerate}
\item $\forall a\in\mathbb{Z},a\equiv a(mod\ m)$ pois $m|(a-a)$ (Reflexidade)
\item Se $a\equiv b(mod\ m)$, ent{\~a}o $m|(a-b)$. Da{\'\i}, $m|(-(a-b))$, ou seja, $m|(b-a)$. Da{\'\i} $b\equiv a(mod\ a)$ (Simetria)
\item Se $a\equiv b(mod\ m)$ e $b\equiv c(mod\ m)$, ent{\~a}o $m|(a-b)$ e $m|(b-c)$. Assim, $m|[(a-b)+(b-c)]$. Logo, $m|(a-c)$, isto {\'e}, $a\equiv c(mod\ m)$ (Transitividade)

Portanto {\'e} rela{\c c}{\~a}o de equival{\^e}ncia. \#

\end{enumerate}

\begin{teorema} A rela{\c c}{\~a}o de congru{\^e}ncia m{\'o}dulo $m$ satisfaz as seguintes propriedades:
\begin{enumerate}
\item $a_{1}\equiv b_{1}(mod\ m)\Leftrightarrow a_{1}-b_{1}\equiv 0(mod\ m)$
\item Se $a_{1}\equiv b_{1}(mod\ m)$ e $a_{2}\equiv b_{2}(mod\ m)$, ent{\~a}o $a_{1}+a_{2}\equiv b_{1}+b_{2}(mod\ m)$
\item Se $a_{1}\equiv b_{2}(mod\ m)$ e $a_{2}\equiv b_{2}(mod\ m)$, ent{\~a}o $a_{1}a_{2}\equiv b_{1}b_{2}(mod\ m)$
\item Se $a\equiv b(mod\ m)$, ent{\~a}o $ax\equiv bx(mod\ m), \forall x\in\mathbb{Z}$
\item Vale a lei do cancelamento: se $d\in\mathbb{Z}$ e $mdc(d,m)=1$ ent{\~a}o\\ $ad\equiv bd(mod\ m)$ implica $a\equiv b(mod\ m)$

\end{enumerate}
\end{teorema}

\textbf{Demonstra{\c c}{\~a}o}: Provemos o {\'\i}tem 3

Dizer que $a\equiv b(mod\ m)$ significa dizer que existe $t\in\mathbb{Z}$ tal que $a=b+tm$.

Assim, existem $m,l\in\mathbb{Z}$ tais que $a_{1}=b_{1}+km, a_{2}=b_{2}+lm$. Da{\'\i}\\
\[a_{1}a_{2}=b_{1}b_{2}+lb_{1}m+klm^{2}\]\\
\[a_{1}a_{2}=b_{1}b_{2}+\underbrace{(lb_{1}+kb_{2}+klm)}_{\in\mathbb{Z}}m\]

Ou seja, $a_{1}a_{2}=b_{1}b_{2}+pm$, onde $p=lb_{1}+kb_{2}+klm\in\mathbb{Z}$. Portanto, $a_{1}a_{2}\equiv b_{1}b_{2}(mod\ m)$.

Para o {\'\i}tem 5, se $ad\equiv bd(mod\ m)$, ent{\~a}o $m|d(a-b)$. Mas, $mdc(d,m)=1$, logo $m|(a-b)$, isto {\'e}, $a\equiv b(mod\ m)$.\#

Como a congru{\^e}ncia m{\'o}dulo $m$ {\'e} uma rela{\c c}{\~a}o de equival{\^e}ncia, podemos determinar suas classes de equival{\^e}ncia. Assim, dado $n\in\mathbb{Z}$, temos
\[C(n)=\{x\in\mathbb{Z}/x\equiv n(mod\ m)\}\]

Denotaremos $C(n)$ por $R_{m}(n)$ ou $\bar{n}$, quando n{\~a}o houver possibilidade de confus{\~a}o.

Por exemplo, fixando $m$\\
$R_{m}(0)=\{x\in\mathbb{Z}/x\equiv 0(mod\ m)\}=\{x\in \mathbb{Z}/x=mk, k\in\mathbb{Z}\}=m\mathbb{Z}$\\
$R_{m}(1)=\{x\in\mathbb{Z}/x\equiv 1(mod\ m)\}=\{x\in\mathbb{Z}/x=1+km, k\in\mathbb{Z}\}$\\
$R_{m}(n)=\{x\in\mathbb{Z}/x=n+km, k\in\mathbb{Z}\}$

\subsection{Classes de equival{\^e}ncia m{\'o}dulo $m$}

\begin{proposicao} As classes de equival{\^e}ncia definidas pela congru{\^e}ncia m{\'o}dulo $m$ s{\~a}o determinadas pelos restos da divis{\~a}o euclidiana por $m$. Em outras palavras, $R_{m}(n)$ {\'e} o conjunto dos n{\'u}meros inteiros cujo resto na divis{\~a}o euclidiana por $m$ {\'e} $n$.\end{proposicao}

\textbf{Demonstra{\c c}{\~a}o}: Dado $x\in\mathbb{Z}$, pela divis{\~a}o de Euclides, podemos escrever $x=km+r$ onde $0\leq r\leq m$. Da{\'\i}, $x-r=km$, isto {\'e}, $m|(x-r)$. Logo $x\in R_{m}(r)$. Portanto, se $r=n$, ent{\~a}o $x\in R_{m}(n)$ e neste caso, $x=km+n=n+km$, ou seja, o resto da divis{\~a}o euclidiana de $x$ por $m$ {\'e} $n$.\#  

\begin{corolario} $R_{m}(k)=R_{m}(l)$ se, e somente se, $k\equiv l(mod\ m)$.\end{corolario}

Exemplos:
\begin{enumerate}
\item Se $m=2$, ent{\~a}o os poss{\'\i}veis restos na divis{\~a}o euclidiana por 2 s{\~a}o 0 e 1. Logo, existem duas classes de equival{\^e}ncia, a saber $R_{2}(0)$ e $R_{2}(1)$
\item Se $m=3$, ent{\~a}o os poss{\'\i}veis restos da divis{\~a}o euclidiana s{\~a}o 0,1 e 2. Da{\'\i}\\
$R_{3}(0)=3\mathbb{Z}$\\
$R_{3}(1)=\{x\in\mathbb{Z}/x=3q+1,q\in\mathbb{Z}\}$\\
$R_{3}(2)=\{x\in\mathbb{Z}/x=3q+2,q\in\mathbb{Z}\}$

\end{enumerate}

\begin{proposicao} Na rela{\c c}{\~a}o de equival{\^e}ncia m{\'o}dulo $m$ existem $m$ classes de equival{\^e}ncia.\end{proposicao}

\textbf{Demonstra{\c c}{\~a}o}: Os poss{\'\i}veis restos na divis{\~a}o euclidiana por $m$ s{\~a}o $0,1,...,(m-1)$. Como cada poss{\'\i}vel resto define uma classe de equival{\^e}ncia diferente, existem exatamente $m$ classes de equival{\^e}ncia.\#

\section{Conjunto quociente $\left(\dfrac{\mathbb{Z}}{m\mathbb{Z}}\right)$}


\begin{nota}[Conjunto quociente] Fixado $m$ inteiro positivo, denotaremos\\
$R_{m}(0)=\bar{0}$\\
$R_{m}(1)=\bar{1}$\\
$\vdots$\\
$R_{m}(m-1)=\overline{m-1}$

O conjunto quociente desta rela{\c c}{\~a}o ser{\'a} denotado por $\displaystyle\frac{\mathbb{Z}}{m\mathbb{Z}}$ e $\displaystyle\frac{\mathbb{Z}}{m\mathbb{Z}}=\{\bar{0},\bar{1},...,\overline{m-1}\}$
\end{nota}

Queremos definir um meio de somar e multiplicar os elementos de $\displaystyle\frac{\mathbb{Z}}{m\mathbb{Z}}$. Por exemplo, em $\displaystyle\frac{\mathbb{Z}}{2\mathbb{Z}}=\{\bar{0},\bar{1}\}$ temos que a soma de pares {\'e} par, soma de par com {\'\i}mpar {\'e} {\'\i}mpar e a soma de {\'\i}mpares {\'e} par.

Podemos escrever
\vspace{0,05cm}\\
$\bar{0}\oplus\bar{0}=\overline{0+0}=\bar{0}$\\
$\bar{0}\oplus\bar{1}=\overline{0+1}=\bar{1}$\\
$\bar{1}\oplus\bar{1}=\overline{1+1}=\bar{0}$

Para multiplica{\c c}{\~a}o, temos
\vspace{0,05cm}\\
$\bar{0}\odot\bar{0}=\overline{0.0}=\bar{0}$\\
$\bar{0}\odot\bar{1}=\overline{0.1}=\bar{0}$\\
$\bar{1}\odot\bar{1}=\overline{1.1}=\bar{1}$

Em $\displaystyle\frac{\mathbb{Z}}{m\mathbb{Z}}$ definimos
\begin{eqnarray}
\bar{a}\oplus\bar{b}=\overline{a+b}\\
\bar{a}\odot\bar{b}=\overline{a.b}
\end{eqnarray}
Para $\bar{a},\bar{b}\in\displaystyle\frac{\mathbb{Z}}{m\mathbb{Z}}$

\begin{proposicao} As opera{\c c}{\~o}es de soma e produto definidas em (5.1) e (5.2) s{\~a}o independentes dos representantes das classes.\end{proposicao}

\textbf{Demonstra{\c c}{\~a}o}: Dadas duas classes com representantes diferentes, $\bar{a}_{1}=\bar{a}_{2},\  \bar{b}_{1}=\bar{b}_{2}, a_{1}\neq a_{2}, b_{1}=b_{2}$, temos:\\
\[\overline{a_{1}+b_{1}}=\bar{a}_{1}\oplus\bar{b}_{1}=\bar{a}_{2}\oplus\bar{b}_{2}=\overline{a_{2}+b_{2}}\]\\
\[\overline{a_{1}b_{1}}=\bar{a}_{1}\odot\bar{b}_{1}=\bar{a}_{2}\odot\bar{b}_{2}=\overline{a_{2}b_{2}}\]\\

C.Q.D.\#

Exemplo: Determine a some e multiplica{\c c}{\~a}o em:

$\displaystyle\frac{\mathbb{Z}}{4\mathbb{Z}}=\{\bar{0},\bar{1},\bar{2},\bar{3}\}$
\begin{table}[h]
   \centering 
   \setlength{\arrayrulewidth}{0,5\arrayrulewidth}
   \caption{\it Soma}
   \begin{tabular}{|c|c|c|c|c|} 
      \hline
      $\oplus$ & $\bar{0}$ & $\bar{1}$ & $\bar{2}$ & $\bar{3}$ \\
      \hline
      $\bar{0}$ & $\bar{0}$ & $\bar{1}$ & $\bar{2}$ & $\bar{3}$ \\
      \hline
      $\bar{1}$ & $\bar{1}$ & $\bar{2}$ & $\bar{3}$ & $\bar{0}$ \\
      \hline
      $\bar{2}$ & $\bar{2}$ & $\bar{3}$ & $\bar{0}$ & $\bar{1}$ \\
      \hline
      $\bar{3}$ & $\bar{3}$ & $\bar{0}$ & $\bar{1}$ & $\bar{2}$ \\
      \hline
   \end{tabular}
\end{table}

\begin{table}[h]
   \centering 
   \setlength{\arrayrulewidth}{0,5\arrayrulewidth}
   \caption{\it Multiplica{\c c}{\~a}o}
   \begin{tabular}{|c|c|c|c|c|} 
      \hline
      $\odot$ & $\bar{0}$ & $\bar{1}$ & $\bar{2}$ & $\bar{3}$ \\
      \hline
      $\bar{0}$ & $\bar{0}$ & $\bar{0}$ & $\bar{0}$ & $\bar{0}$ \\
      \hline
      $\bar{1}$ & $\bar{0}$ & $\bar{1}$ & $\bar{2}$ & $\bar{3}$ \\
      \hline
      $\bar{2}$ & $\bar{0}$ & $\bar{2}$ & $\bar{0}$ & $\bar{2}$ \\
      \hline
      $\bar{3}$ & $\bar{0}$ & $\bar{3}$ & $\bar{2}$ & $\bar{1}$ \\
      \hline
   \end{tabular}
\end{table}


\subsection{Elementos Invers{\'\i}veis de $\displaystyle\frac{\mathbb{Z}}{m\mathbb{Z}}$}

\subsubsection{Inversibilidade}
\begin{definicao}[Inversibilidade] Um elemento $\bar{a}\in\displaystyle\frac{\mathbb{Z}}{m\mathbb{Z}}$ {\'e} invers{\'\i}vel se, e somente se, existem $\bar{b}\in\displaystyle\frac{\mathbb{Z}}{m\mathbb{Z}}$ tal que $\bar{a}\odot\bar{b}=\bar{1}$.\end{definicao}

Neste caso, $\bar{b}$ {\'e} chamado inverso de $\bar{a}$ e denotaremos $\bar{b}=(\bar{a})^{-1}$.

Quando $\bar{b}$ existe, ele {\'e} {\'u}nico. De fato, dado $\bar{a}\in\displaystyle\frac{\mathbb{Z}}{m\mathbb{Z}}$, se existem $\bar{b},\bar{d}\in\displaystyle\frac{\mathbb{Z}}{m\mathbb{Z}}$ tais que $\bar{a}\odot\bar{b}=\bar{1}=\bar{a}\odot\bar{d}$, ent{\~a}o $\bar{b}=\bar{b}\odot\bar{1}=\bar{b}\odot (\bar{a}\odot\bar{d})=(\bar{b}\odot\bar{a})\odot\bar{d}=\bar{1}\odot\bar{d}=\bar{d}$.

\begin{proposicao} Um elemento $\bar{a}\in\displaystyle\frac{\mathbb{Z}}{m\mathbb{Z}}$ {\'e} invers{\'\i}vel se, e somente se,
 \[mdc(a,m)=1\]
\end{proposicao}

\textbf{Demonstra{\c c}{\~a}o}: Suponha que existe $\bar{b}\in\displaystyle\frac{\mathbb{Z}}{m\mathbb{Z}}$ tal que $\bar{a}\odot\bar{b}=\bar{1}$. Assim, $\overline{ab}=\bar{1}$, ou seja, $ab\equiv 1(mod\ m)$. Da{\'\i}, $ab-1=km, k\in\mathbb{Z}$, logo $ab+m(-k)=1$, e ent{\~a}o $mdc(a,m)=1$.

Agora suponha que $mdc(a,m)=1$. Logo, existem $x_{0}, y_{0}\in\mathbb{Z}$ tais que $ax_{0}+my_{0}=1$, isto {\'e}, $ax_{0}-1=m(-y_{0})$. Logo $ax_{0}\equiv 1(mod\ m)$, ou seja, $\overline{ax_{0}}=\bar{1}$. Portanto, $\bar{a}\odot\bar{x_{0}}=\bar{1}$.\#

Exemplos:
\begin{enumerate}
\item Em $\displaystyle\frac{\mathbb{Z}}{4\mathbb{Z}}$ existem dois elementos invers{\'\i}veis que s{\~a}o $\bar{1}$, cujo inverso {\'e} $\bar{1}$, e o $\bar{3}$, cujo inverso {\'e} $\bar{3}$.
\item Em $\displaystyle\frac{\mathbb{Z}}{11\mathbb{Z}} $, todos elementos, exceto $\bar{0}$, possuem inverso:

\begin{table}[h]
   \centering 
   \setlength{\arrayrulewidth}{0,5\arrayrulewidth}
   \caption{\it Inversos em $\displaystyle\frac{\mathbb{Z}}{11\mathbb{Z}}$}
   \begin{tabular}{|c|c|c|c|c|c|c|c|c|c|c|} 
      \hline
      Elemento & $\bar{1}$ & $\bar{2}$ & $\bar{3}$ & $\bar{4}$ & $\bar{5}$ & $\bar{6}$ & $\bar{7}$ & $\bar{8}$ & $\bar{9}$ & $\bar{10}$ \\
      \hline
      Inverso & $\bar{1}$ & $\bar{6}$ & $\bar{4}$ & $\bar{3}$ & $\bar{9}$ & $\bar{2}$ & $\bar{8}$ & $\bar{7}$ & $\bar{5}$ & $\bar{10}$ \\
      \hline
   \end{tabular}
\end{table}
\end{enumerate}

O n{\'u}mero de elementos invers{\'\i}veis de $\displaystyle\frac{\mathbb{Z}}{m\mathbb{Z}}$ {\'e} igual a quantidade de n{\'u}meros coprimos com $m$. Esse n{\'u}mero {\'e} denotado por $\varphi(m)$ e {\'e} chamado fun{\c c}{\~a}o $\varphi$ de Euler. Pode-se demonstrar que
\[\varphi(m)=m\displaystyle\prod_{p/m}\left(1-\displaystyle\frac{1}{p}\right)\]\\
Onde o produto varia sobre todos os divisores primos de m, sem repeti{\c c}{\~a}o.

Por exemplo, para $\displaystyle\frac{\mathbb{Z}}{100\mathbb{Z}}$ temos:\\
$100=2^{2}5^{2}$

Da{\'\i},\\
$\varphi(100)=100\left(1-\displaystyle\frac{1}{2}\right)(1-\displaystyle{1}{5})=40$

Logo, em $\displaystyle\frac{\mathbb{Z}}{100\mathbb{Z}}$ existem 40 elementos invers{\'\i}veis.

\begin{nota}[Conjunto dos elementos invers{\'\i}veis] Denotaremos o conjunto de todos os elementos invers{\'\i}veis de $\displaystyle\frac{\mathbb{Z}}{m\mathbb{Z}}$ por $\left(\displaystyle\frac{\mathbb{Z}}{m\mathbb{Z}}\right)^{*}$, ou ainda $U\left(\displaystyle\frac{\mathbb{Z}}{m\mathbb{Z}}\right)$.\end{nota}

\begin{proposicao} Sejam $\bar{a},\bar{b}\in\left(\displaystyle\frac{\mathbb{Z}}{m\mathbb{Z}}\right)^{*}$. Ent{\~a}o $\bar{a}\odot\bar{b}\in\left(\displaystyle\frac{\mathbb{Z}}{m\mathbb{Z}}\right)^{*}$.\end{proposicao}

\textbf{Demonstra{\c c}{\~a}o}: Por uma proposi{\c c}{\~a}o anterior, basta verificar que\\ $mdc(ab,m)=1$. Para que $\bar{a}\odot\bar{b}\in\left(\displaystyle\frac{\mathbb{Z}}{m\mathbb{Z}}\right)^{*}$.

Como $\bar{a},\bar{b}\in\left(\displaystyle\frac{\mathbb{Z}}{m\mathbb{Z}}\right)^{*}$, ent{\~a}o $mdc(a,m)=1$ e $mdc(b,m)=1$.

Assim, existem $x_{0},y_{0},x_{1},y_{1}\in\mathbb{Z}$ tais que\\
$ax_{0}+my_{0}=1$\\
$bx_{1}+my_{1}=1$

Da{\'\i},
\[abx_{0}x_{1}+max_{0}y_{1}+mbx_{1}y_{0}+m^{2}y_{0}y_{1}=1\]
\[\underbrace{abx_{0}x_{1}}_{\in\mathbb{Z}}+m\underbrace{(ax_{0}y_{1}+bx_{1}y_{0}+my_{0}y_{1})}_{\in\mathbb{Z}}=1\]

Logo, $mdc(ab,m)=1$, ou seja, $\bar{a}\odot\bar{b}\in\left(\displaystyle\frac{\mathbb{Z}}{m\mathbb{Z}}\right)^{*}$.\#
%!TEX program = xelatex
%!TEX root = Algebra_1.tex
\chapter{An{\'e}is}

\begin{definicao}
	Seja $A$ um conjunto n{\~a}o vazio. Dizemos que $A$ est{\'a} munido (ou equipado) de uma \textbf{opera{\c c}{\~a}o bin{\'a}ria} quando existe uma fun{\c c}{\~a}o
	\begin{align*}
		&\Delta : A \times A \to A\\
		&(a,b) \longmapsto a\Delta b		
	\end{align*}
	Uma opera{\c c}{\~a}o bin{\'a}ria tamb{\'e}m {\'e} chamada de uma \textbf{opera{\c c}{\~a}o interna} em $A$.
\end{definicao}

\begin{exemplos}
	\begin{enumerate}[label={\arabic*})]
		\item A soma usual nos conjuntos $\z$, $\rac$, $\real$ e $\complex$ {\'e} uma opera{\c c}{\~a}o bin{\'a}ria.

		\item A multiplicação usual nos conjuntos dos $\z$, $\rac$, $\real$ e $\complex$ {\'e} uma opera{\c c}{\~a}o bin{\'a}ria.

		\item Seja $m > 1$, $m \in \z$ fixo. A soma \eqref{soma_modulo_m} e a multiplicação \eqref{multiplicacao_modulo_m} definidos em $\z_m = \{\overline{0},\overline{1},...,\overline{m-1}\}$ é uma operação binária.

		\item A operação $\div$ em $\rac^{*}$ {\'e} uma opera{\c c}{\~a}o bin{\'a}ria.
		\item Já em $\n$, $\z$, $\z^{*}$ e em $\rac$ a operação $\div$ n{\~a}o {\'e} uma opera{\c c}{\~a}o bin{\'a}ria.
\end{enumerate}	
\end{exemplos}

\begin{definicao}
	Seja $A$ um conjunto n{\~a}o vazio $A$ no qual estão definidas duas opera{\c c}{\~o}es binárias $\oplus$ e $\otimes$, chamadas \textit{soma} e \textit{produto}.  Dizemos que $(A, \oplus, \otimes)$ {\'e} um \textbf{anel} quando as seguintes condi{\c c}{\~o}es s{\~a}o verdadeiras:
	\begin{enumerate}[label={\roman*})]
		\item \textbf{Associatividade}: para todos $x$, $y$, $x\in A$ vale que
		\[
			(x \oplus y) \oplus z = x \oplus (y \oplus z)
		\]
		Essa propriedade {\'e} chamada \textbf{propriedade associativa} da soma.

		\item \textbf{Comutatividade}: Para todos $x$, $y \in A$ vale
		\[
			x \oplus y = y \oplus x
		\]

		\item \textbf{Elemento Neutro}: Existe em $A$ um elemento denotado por $0$ (zero) ou $0_{A}$ tal que para todo elemento $x \in A$ vale
		\[
			x \oplus 0_A = x = 0_A \oplus x
		\]
		Tal elemento $0_A$ é chamado de \textbf{elemento neutro da soma} ou simplesmente \textbf{elemento neutro}.

		\item \textbf{Elemento Oposto}: Para cada elemento $x \in A$, existe $y \in A$ tal que
		\[
			x \oplus y = 0_A = y \oplus x
		\]
		Tal elemento $y$ é chamado de \textbf{oposto aditivo} de $x$ ou simplesmente \textbf{oposto} de $x$.

		\item \textbf{Associatividade}: Para todos $x$, $y$, $z \in A$, vale que
		\[
			x\otimes (y\otimes z) = x\otimes (y\otimes z)
		\]

		\item \textbf{Distributividade}: Para todos $x$, $y$, $x \in A$ vale
		\[
			(x \oplus y)\otimes z = x\otimes z \oplus y\otimes z
		\]
		Essa propriedade {\'e} chamada \textbf{distributiva da soma em rela{\c c}{\~a}o ao produto}.
		
		\item \textbf{Distributividade}: Para todos $x$, $y$, $z \in A$ vale
		\[
			x\otimes(y \oplus z) = x\otimes y \oplus x\otimes z.
		\]
		Essa {\'e} a propriedade \textbf{distributiva do produto em rela{\c c}{\~a}o {\`a} soma}.
	\end{enumerate}
\end{definicao}

\begin{observacoes}
	Seja $(A, \oplus, \otimes)$ uma anel.
	\begin{enumerate}
		\item \textbf{Comutatividade}: Se para todos $x$, $y \in A$ vale
		\[
			x \otimes y = y \otimes x
		\]
		Dizemos que $(A, \oplus, \otimes)$ {\'e} um \textbf{anel comutativo}.

		\item \textbf{Unidade}: Se existe em $A$ um elemento denotado por $1$ ou $1_{A}$ tal que
		\[
			x \otimes 1 = x = 1 \otimes x,
		\]
		para todo $x \in A$, ent{\~a}o dizemos que $(A, \oplus, \otimes)$ é um \textbf{anel com unidade} ou um \textbf{anel unit{\'a}rio}. O elemento $1_A$ {\'e} chamado de \textbf{unidade} de $A$ ou \textbf{elemento neutro da multiplicação} de $A$.

		\item Se um anel $(A, \oplus, \otimes)$ satisfaz as duas propriedades anteriores dizemos que $(A, \oplus, \otimes)$ é um \textbf{anel comutativo com unidade} ou um \textbf{anel comutativo unitário}.

		\item Seja $(A, \oplus, \otimes)$ uma anel. Quando não houver chance de confusão com relação às operações envolvidas diremos simplesmente que $A$ é uma anel.
	\end{enumerate}
\end{observacoes}

\begin{exemplos}
	\begin{enumerate}[label={\arabic*})]
		\item $(\z,+,.)$, $(\rac,+,.)$, $(\real,+,.)$, $(\complex,+,.)$, $(\z_m, \oplus, \otimes)$ s{\~a}o an{\'e}is associativos, comutativos e com unidade.

		\item  Seja $A = \z =\{f : \z \to \z \mid f \mbox{ {\'e} uma fun{\c c}{\~a}o}\}$. Dadas duas fun{\c c}{\~o}es quaisquer $f$, $g \in A$, definimos $f\oplus g:\z \to \z$ e $f \otimes g:\z \to \z$ como:
		\begin{align*}
			(f\oplus g)(x) &= f(x) + g(x)\\
			(f\otimes g)(x) &= f(x)g(x)
		\end{align*}
		para todo $x \in \z$. Assim $(A, \oplus, \otimes)$ é um anel. De fato:
		\begin{enumerate}[label={\roman*})]
			\item Para todo $x \in \z$
			\begin{align*}
				[(f \oplus g) \oplus h](x) &= (f \oplus g)(x) + h(x) = (f(x) + g(x)) + h(x)\\ 
				&= f(x) + (g(x) + h(x)) = f(x) + (g \oplus h)(x)\\ &= [f \oplus (g \oplus h)](x)
			\end{align*}
			para todos $f$, $g$ e $h \in A$.
			
			\item Para todo $x \in \z$
			\[
				(f\oplus g)(x) = f(x) + g(x) = g(x) + f(x) = (g\oplus f)(x),
			\]
			portanto $f\oplus g = g\oplus f$ para todos $f$, $g \in A$.

			\item $0_A : \z \to \z$ dada por $0_A(x) = 0$ para todo $x \in \z$. Daí para todo $x \in \z$
			\[
				(f \oplus 0_A)(x) = f(x) + 0_A(x) = f(x) + 0 = f(x)
			\]
			para todo $f \in A$. Logo $f + 0_A = f$ para todo $f \in A$. Logo $0_A$ é o elemento neutro da soma em $A$.

			\item Dada $f \in A$, defina $g : \z \to \z$ por $g(x) = -f(x)$ para todo $x \in \z$. Daí para todo $x \in \z$ temos
			\[
				(f \oplus g)(x) = f(x) + g(x) = f(x) + (-f(x)) = 0.
			\]
			Logo $g(x) = -f(x)$ é o oposto de $f \in A$.

			\item Para todo $x \in \z$
			\begin{align*}
				[(f \otimes g)\otimes h](x) &= (f \otimes g)(x)h(x) = (f(x)g(x))h(x)\\ &= f(x)(g(x)h(x)) = f(x)(g \otimes h)(x)\\ &= [f\otimes (g \otimes h)](x)
			\end{align*}
			para todos $f$, $g$ e $h \in \z$.

			\item Para todo $x \in \z$
			\begin{align*}
				[(f \oplus g)\otimes h](x) &= (f \oplus g)(x)h(x) = (f(x) + g(x))h(x) \\ &= f(x)h(x) + g(x)h(x)
				= (f\otimes g)(x) + (g \otimes h)(x)\\
				&= [(f \otimes g) \oplus (g \otimes h)](x)
			\end{align*}
			para todos $f$, $g$ e $h \in A$.

			\item Para todo $x \in \z$
			\begin{align*}
				[f\otimes (g \oplus h)](x) &= f(x)(g\oplus h)(x) = f(x)(g(x) + h(x))\\ 
				&= f(x)g(x) + f(x)h(x) = (f\otimes g)(x) + (f\otimes h)(x)\\
				&= [(f \otimes g) \oplus (f\otimes h)](x)
			\end{align*}
			para todos $f$, $g$ e $h \in A$.
		\end{enumerate}
		Assim $(A, \oplus, \otimes)$ é um anel. Além disso, para todo $x \in \z$
		\[
			(f\otimes g)(x) = f(x)g(x) = g(x)f(x) = (g\otimes f)(x)
		\]
		para todos $f$, $g \in A$. Assim a operação $\otimes$ é comutativa.

		Mais ainda, definindo $1_A : \z \to \z$ como $1_A(x) = 1$ para todo $x \in \z$ temos
		\[
			(f \otimes 1_A)(x) = f(x)1_A(x) = f(x)\cdot 1 = f(x)
		\]
		para todo $f \in A$. Logo $1_A$ é a unidade de $A$.

		Portanto $(A, \oplus, \otimes)$ é um anel comutativo com unidade.
	\end{enumerate}
\end{exemplos}

\begin{observacao}
	Seja $(A, \oplus, \cdot)$ um anel. Para simplificar a notação vamos denotar a operação $\oplus$
	por $+$ e a operação $\otimes$ por $\cdot$ e assim escrever simplesmente que $(A, +, \cdot)$ é um anel.
\end{observacao}

\begin{proposicao}
	Seja $(A, + , \cdot)$ uma anel. Então:
	\begin{enumerate}[label={\roman*})]
		\item O elemento neutro {\'e} {\'u}nico.
		\item Para cada $x \in A$ existe um {\'u}nico oposto.
		\item Para todo $x \in A$, $-(-x) = x$.
		\item Dados $x_{1}$, $x_{2}$, \dots, $x_n \in A$, $n \geqslant 2$, ent{\~a}o
		\[
			-(x_1 + x_2 + \dots + x_n) = (-x_1) + (-x_2) + \dots + (-x_n).
		\]
		\item Para todos $a$, $x$, $y \in A$, se $a + x = a + y$, ent{\~a}o $x = y$.
		\item Para todo $x \in A$, $x\cdot 0_A = 0_A = 0_A\cdot x$.
		\item Para todos $x$, $y \in A$, temos $x(-y) = (-x)y = -(xy)$.
		\item Para todos $x$, $y \in A$, $xy = (-x)(-y)$.
	\end{enumerate}
\end{proposicao}
\begin{prova}
	\begin{enumerate}[label={\roman*})]
		\item Suponha que existam $0_1$, $0_2\in A$ elementos neutros de $A$. Assim
		\[
			x + 0_1 = x \quad \mbox{e}\quad x + 0_2 = x	
		\]
		para todo $x \in A$. Assim
		\[
			0_1 = 0_1 + 0_2 = 0_2
		\]
		e portanto o elemento neutro é único.

		\item De fato, dado $x \in A$ suponha que existam $y_1$, $y_2\in A$ tais que
		\[
			x + y_1 = 0_A \quad \mbox{e}\quad x + y_2 = 0_A.
		\]
		Daí
		\[
			y_1 = y_2 + 0_A = y_1 + (x + y_2) = (y_1 + x) + y_2 = 0_A + y_2 =y_2.
		\]
		Logo o oposto de $x$ é único  e daí será denotado por $-x$.
		
		\item Dado $x \in A$, então $-x$ {\'e} oposto de $x$, isto {\'e}, $x + (-x) = 0_A$. Logo o oposto de $(-x)$ {\'e} $x$, ou seja, $-(-x) = x$.

		\item Segue usando indução sobre $n$.

		\item Suponha que $a + x = a + y$. Seja $-a$ o oposto de $a$ daí
		\begin{align*}
			a + x &= a + y\\
			(-a) + a + x &= (-a) + a + y\\
			0_A + x &= 0_A + y\\
			x & = y
		\end{align*}
		como queríamos.

		\item Temos $0_A + x\cdot 0_A = a\cdot 0_A = a(0_A + 0_A) = a\cdot 0_A + a\cdot 0_A$. Assim do item anterior segue que $x\cdot 0_A = 0_A$.

		\item Provemos que $x(-y) = -(xy)$:
		\[
			x(-y) + xy = x((-y) + y) = x\cdot 0_A = 0_A,
		\]
		portanto $-xy = x(-y)$.

		\item Basta usar o caso anterior.
	\end{enumerate}
\end{prova}

\begin{definicao}
	Um anel comutativo $(A, + , \cdot)$ {\'e} dito ser um \textbf{anel de integridade} quando para todos 
	$x$, $y \in A$, se $xy = 0_A$, ent{\~a}o $x = 0_A$ ou $y = 0_a$. Um anel de integridade tamb{\'e}m {\'e} chamado de \textbf{dom{\'\i}nio de integridade} ou simplesmente de \textbf{dom{\'\i}nio}.
\end{definicao}

\begin{observacao}
	Se $x$ e $y$ s{\~a}o elementos n{\~a}o nulos de um anel $A$ tais que $xy = 0_A$, ent{\~a}o $x$ e $y$ s{\~a}o chamados de \textbf{divisores pr{\'o}prios de zero}.
\end{observacao}


\begin{exemplos}
	\begin{enumerate}[label={\arabic*})]
		\item Os an{\'e}is $\z$, $\rac$, $\real$, $\complex$ s{\~a}o an{\'e}is de integridade.
		
		\item Em geral $\z_m$ n{\~a}o {\'e} anel de integridade, por exemplo, em $\z_4$, $\overline{2} \neq \overline{0}$, no entanto $\overline{2}\otimes \overline{2} = \overline{4} = \overline{0}$.
		
		\item $M_{n}(\real)$ n{\~a}o {\'e} um anel de integridade, por exemplo, em $M_{2}(\real)$
		\begin{align*}
			A &= \begin{bmatrix}
				1 & 0\\
				0 & 0
			\end{bmatrix} \neq \begin{bmatrix}
				0 & 0\\
				0 & 0		
			\end{bmatrix},\qquad 
			B = \begin{bmatrix}
				0 & 0\\
				1 & 0
			\end{bmatrix} \neq \begin{bmatrix}
				0 & 0\\
				0 & 0
			\end{bmatrix}\\
			AB & =\begin{bmatrix}
				0 & 0\\
				0 & 0
			\end{bmatrix}
		\end{align*}

		\item Suponha que $m = nk$, $m > n > 1$ e $m > k > 1$. Logo, em $\z_m$, $\overline{n} \neq \overline{0}$ e $\overline{k} \neq \overline{0}$ e no entanto $\overline{n} \otimes \overline{k} = \overline{m} = \overline{0}$. Logo, se $m$ n{\~a}o {\'e} primo, ent{\~a}o $\z_m$ n{\~a}o {\'e} um anel de integridade. Agora, suponha que $m = p$ primo. Sejam $\overline{x}$, $\overline{y} \in \z_m$ tais que $\overline{x}\otimes \overline{y} = \overline{0}$, ou seja, $xy \equiv 0 \pmod p$. Da{\'\i} $p\mid xy$. Logo $p\mid x$ ou $p\mid y$. Portanto, $\overline{x} = \overline{0}$ ou $\overline{y} = \bar{0}$. Assim, $\z_m$ {\'e} anel de integridade se, e somente se, $m$ {\'e} primo.
	\end{enumerate}
\end{exemplos}



\begin{definicao}
	Seja $(A, +, \cdot)$ um anel. Dizemos que um subconjunto n{\~a}o vazio $B\subseteq A$ {\'e} um \textbf{subanel} de $A$ quando $(B, +, \cdot)$ \'e um anel.
\end{definicao}

\begin{exemplos}
	\begin{enumerate}[label={\arabic*})]
		\item Todo anel $A$ sempre tem dois suban{\'e}is: $\{0_{A}\}$ e $A$, que s{\~a}o chamados de \textbf{suban{\'e}is triviais}.
		\item Em $(\z_4,\oplus,\otimes)$ o conjunto $B = \{\overline{0}, \overline{2}\}$ \'e um subanel.
		\item No anel $\z$, o conjunto $m\z$, $m > 1$ {\'e} um subanel de $\z$.
	\end{enumerate}	
\end{exemplos}


\begin{proposicao}
	Seja $(A, +,\cdot)$ um anel. Um subconjunto n{\~a}o vazio $B\subseteq A$ {\'e} um subanel de $A$ se, e somente se, $x - y \in B$ e $x\cdot y \in B$ para todos $x$, $y \in B$.
\end{proposicao}

\begin{definicao}
	Um homomorfismo do anel $(A, +, \cdot)$ no anel $(B, \oplus, \otimes)$ {\'e} uma fun{\c c}{\~a}o $f : A \to B$ que satisfaz:
	\begin{enumerate}[label={\roman*})]
		\item $f(x + y) = f(x) \oplus f(y)$, para todos $x$, $y \in A$;
		\item $f(x \cdot y) = f(x)\otimes f(y)$, para todos $x$, $y \in A$.
		% \item $f(1_{A})=1_{B}$, onde $1_{A}$ {\'e} a unidade de A e $1_{B}$ {\'e} a unidade de B
	\end{enumerate}
\end{definicao}

\begin{proposicao}
	Sejam $(A, +, \cdot)$ e $(B, \oplus, \otimes)$ anéis e seja $f : A \to B$ um homomorfismo. Ent{\~a}o:
	\begin{enumerate}[label={\roman*})]
		\item $f(0_{A}) = 0_{B}$
		\item $f(-x) = -f(x)$, para todo $x \in A$.
	\end{enumerate}
\end{proposicao}
\begin{prova}
	\begin{enumerate}[label={\roman*})]
		\item Fazendo $x = y = 0_{A}$, temos
		\[
			f(0_A) = f(0_A + 0_A) = f(0_A) \oplus f(0_A)
		\]
		Somando $-f(0_A)$ em ambos os lados obtemos
		\begin{align*}
			f(0_A) \oplus (-f(0_A)) &= (f(0_A)\oplus f(0_A)) \oplus (-f(0_A))\\
			0_B &= f(0_A) \oplus 0_B\\
			f(0_A) &= 0_B
		\end{align*}

		\item Temos $0_B = f(0_A) = f(x + (-x)) = f(x)\oplus f(-x)$. Assim somando $-f(x)$ em ambos os lados obtemos
		\begin{align*}
			0_B\oplus(-f(x)) &= [f(x)\oplus f(-x)] + (-f(x))\\
			-f(x) &= f(-x) \oplus (f(x) \oplus (-f(x)))\\
			f(-x) &= -f(x)
		\end{align*}
	\end{enumerate}
\end{prova}

\begin{definicao}Seja $f:A\rightarrow B$ um homomorfismo, onde $A$ e $B$ s{\~a}o an{\'e}is. Dizemos que
	\begin{enumerate}[label={\roman*})]
		\item $f$ {\'e} um epimorfismo se $f$ for sobrejetora
		\item $f$ {\'e} um monomorfismo se $f$ for injetora
		\item $f$ {\'e} um isomorfismo se $f$ for bijetora
		\item Quando $A=B$ e $f$ {\'e} um isomorfismo, ent{\~a}o $f$ {\'e} um automorfismo
	\end{enumerate}
\end{definicao}

\begin{proposicao}
	Sejam $(A, +, \cdot)$ e $(B, \oplus, \otimes)$ anéis e seja $f : A \to B$ um homomorfismo sobrejetor de an\'eis.
	\begin{enumerate}[label={\roman*})]
		\item Se $A$ tem unidade, então $B$ tem unidade e $f(1_A) = 1_B$.
		\item Se $A$ tem unidade e $x \in A$ possui inverso multiplicativo, então $f(x)$ tem inverso e $f(x^{-1}) = (f(x))^{-1}$.
	\end{enumerate}
\end{proposicao}

\begin{definicao}
	Seja $(A, +, \cdot)$ um anel comutativo. Um subconjunto não-vazio $I \sub A$ {\'e} chamado de \textbf{ideal} de $A$ se:
	\begin{enumerate}[label={\roman*})]
		\item para todos $x$, $y \in I$, temos $x - y \in I$.
		\item Para todo $\alpha \in A$ e todo $x \in I$, temos $\alpha\cdot x \in I$.
	\end{enumerate}
\end{definicao}

\begin{observacao}
	Quando $I = A$ ou $I = \{0_A\}$, dizemos que $I$ {\'e} um \textbf{ideal trivial}.
\end{observacao}

\begin{proposicao}
	Seja $A$ um anel comutativo e $I$ um ideal de $A$. Ent{\~a}o:
	\begin{enumerate}[label={\roman*})]
	 	\item $0_{A}\in I$.
	 	\item $-x \in I$ para todo $x \in I$.
	 	\item Se $1_A \in I$, então $I = A$.
	\end{enumerate}
\end{proposicao}
\begin{prova}
	\begin{enumerate}[label={\roman*})]
		\item Da definição de ideal temos $\alpha \cdot x \in I$ para todo $x \in I$ e todo $\alpha \in A$.
		Assim dado $x \in I$ $0_A = 0_A \cdot x \in I$.

		\item Como $0_A \in I$, dado $x \in I$ da definição de ideal segue que $0_A - x \in I$, isto é, $-x \in I$.

		\item Suponha que $1_A \in I$. Como $I$ {\'e} ideal, para todo $\alpha \in A$ e todo $x \in I$ devemos ter $\alpha\cdot x \in I$. Assim, em particular, $1_A \cdot x \in I$ para todo $x \in A$. Logo, $A\sub I$ e como $I\sub A$, ent{\~a}o $I = A$.
	\end{enumerate}
\end{prova}

\begin{exemplos}
	\begin{enumerate}[label={\arabic*})]
		\item Em $\z$ todos os ideais n{\~a}o triviais s{\~a}o da forma $m\z$, $m > 1$.
		\item No anel $\z_p$, onde $p$ {\'e} um n{\'u}mero primo, os {\'u}nicos ideais  s{\~a}o os triviais $\{\overline{0}\}$ e $\z_p$.
		
		De fato, seja $I \sub \z_p$ um ideal, $I \neq \{\overline{0}\}$. Provemos que $I = \z_p$. Para isso,
		vamos provar que $\overline{1} \in I$. Seja $\overline{a} \in I$, $\overline{a} \neq \overline{0}$, pois $I \neq \{\overline{0}\}$. Como $p$ {\'e} primo, $mdc(a,p) = 1$, da{\'\i} existe $\overline{b} \in \z_p$, $\overline{b} \neq \overline{0}$, tal que $\overline{1} = \overline{a} \otimes \overline{b}$. Mas $I$ {\'e} ideal e $\overline{a} \in I$, logo $\overline{1} = \overline{a} \otimes \overline{b} \in I$.

		Portanto $I = \z_p$.

		\item Os {\'u}nicos ideais n{\~a}o triviais de $\z_8 = \{\overline{0}, \overline{1}, \overline{2}, \overline{3}, \overline{4}, \overline{5}, \overline{6}, \overline{7}\}$ s{\~a}o:
		\begin{align*}
			I_1 &= \{\overline{0}, \overline{2}, \overline{4}, \overline{6}\}\\
			I_2 &=\{\overline{0}, \overline{4}\}
		\end{align*}
	\end{enumerate}	
\end{exemplos}


\begin{definicao}
	Seja $I$ um ideal de um anel $(A, +, \cdot)$. Dados $x$, $y \in A$ dizemos que $x$ \textbf{é congruente a} $y$ \textbf{módulo} $I$ quando $x-y \in I$. Neste caso, escrevemos $x\equiv y \pmod I$.
\end{definicao}

\begin{proposicao}
	A congru{\^e}ncia m{\'o}dulo $I$ {\'e} uma rela{\c c}{\~a}o de equival{\^e}ncia em $A \times A$, onde $A$ anel unit{\'a}rio.
\end{proposicao}
\begin{prova}
	Como $0 = 0_{A} \in I$ e para todo $x \in I$, $x - x = 0 \in I$, ent{\~a}o $x \equiv x \pmod I$.

	Suponha que $x\equiv y \pmod I$. Ent{\~a}o $x - y \in I$. Como $-1 \in A$, $y - x = -(x - y) = -[(x - y)1] = (x - y)(-1) \in I$, ou seja, $y\equiv x \pmod I$.

	Agora, se $x\equiv y \pmod I$ e $y\equiv z \pmod I$, ent{\~a}o $x - y \in I$ e $y - z \in I$. Da{\'\i}, $x - z = (x - z) + (y - z)\in I$, ou seja, $x\equiv z \pmod I$.

	Logo, {\'e} uma rela{\c c}{\~a}o de equival{\^e}ncia.
\end{prova}

Seja $y \in A$. A classe de equival{\^e}ncia m{\'o}dulo $I$ de $y$ {\'e}
\[
	C(y) = \{x \in A \mid x\equiv y \pmod I\} = \{x \in A \mid x - y \in I\}.
\]

Agora, $x - y \in I$ significa que existe $t \in I$, tal que $x - y = t$. Logo, $x = y + t$, onde $t \in I$.

Assim,
\[
	C(y) = \{y + t\mid t \in I\} = y + I.
\]

\begin{observacao}
	Denotamos por $y + I$ (ou $I + y$) a classe de equival{\^e}ncia m{\'o}dulo $I$ de $y \in A$. Denotamos por $\dfrac{A}{I}$ o conjunto de todas as classes de equival{\^e}ncia, tal conjunto {\'e} chamado de \textbf{quociente do anel $A$ pelo ideal $I$}.
\end{observacao}

\begin{exemplos}
	\begin{enumerate}[label={\arabic*})]
		\item Seja $A$ um anel com unidade e $I_{1} = \{0\}$ e $I_{2} = A$ ideais. Então:
		\begin{enumerate}[label={\roman*})]
		\item Dado $x \in A$:
		\[
			C(x) = x + I_{1} = \{x + 0\} = \{x\}.
		\]
		Assim $\dfrac{A}{I_{1}} = \{x + I \mid x \in A\}$, logo existem tantas classes de equival{\^e}ncia quantos forem os elementos de $A$.

		\item Para $I_{2} = A$ temos:
		\[
			C(0_A) = 0_A + I = \{0_A + t \mid t \in I_{2}\}.
		\]
		Como $I_2 = A$, para todo $x \in A$ temos $x \in C(0_A)$ logo existem uma única classe de equivalência
		e $\dfrac{A}{I_{2}} = \{0_{A} + I\}$.
	\end{enumerate}

	\item Seja $A = \z$. Sabemos que os ideais de $\z$ s{\~a}o da forma $m\z$, $m > 1$. Seja $I = m\z$ um ideal de $\z$. Assim $x\equiv y \pmod I$ se, e só se, $x - y \in I$. Mais isso ocorre se, e somente se, $x - y = mk $, para algum $k \in \z$. Logo $x\equiv y \pmod I$ se, e só se, $m\mid (x - y)$. Portanto, $\dfrac{\z}{I} = \z_m$.
	\end{enumerate}
\end{exemplos}


Agora seja $I$ ideal e $A$ um anel. Temos
\[
	\dfrac{A}{I} = \{y + I \mid y \in A\}\\
\]
onde $y + I = \{y + t \mid t \in I\}$ e $y \in A$.

Vamos definir uma soma $\oplus$ e um produto $\otimes$ em $\dfrac{A}{I}$ por
\begin{align*}
	(x + I)\oplus(y + I) &= (x + y) + I\\
	(x + I)\otimes(y + I) &= (xy) + I
\end{align*}
para $x + I$, $y + I \in \dfrac{A}{I}$.

Verifiquemos que a soma e o produto em $\dfrac{A}{I}$ n{\~a}o dependem do representante da classe de equival{\^e}ncia.
Para isso sejam $x_1 + I$, $x_2 + I$, $y_1 + I$, $y_2 + I \in \dfrac{A}{I}$ tais que
\begin{align*}
	x_1 + I &= x_2 + I\\
	y_1 + I &= y_2 + I	
\end{align*}

Ent{\~a}o
\begin{align*}
	(x_1 + I) \oplus (y_1 + I) &= (x_1 + y_1) + I\\
	(x_2 + I) \oplus (y_2 + I) &= (x_2 + y_2) + I
\end{align*}

Como $x_1 + I = x_2 + I$, ent{\~a}o $x_1 - x_2 \in I$ e como $y_1 + I = y_2 + I$, ent{\~a}o $y_1 = y_2 \in I$. Mas $I$ {\'e} ideal, logo $(x_1 - x_2) + (y_1 - y_2) = (x_1 + y_1) - (x_2 + y_2) \in I$, ou seja
\[
	(x_1 + I) \oplus (y_1 + I) = (x_2 + I) \oplus (y_2 + I).
\]

Agora,
\begin{align*}
	(x_1 + I) \otimes (y_1 + I) &= (x_1y_1) + I\\
	(x_2 + I) \otimes (y_2 + I) &= (x_2y_2) + I
\end{align*}

Como $(x_1 - x_2)y \in I$ e $(y_1 - y_2)x_2 \in I$ então
\begin{align*}
	&(x_1 - x_2)y_1 + (y_1 - y_2)x_2 \in I\\
	&x_1y_2-\underbrace{x_2y_1 + y_1x_2}_{= 0} - y_2x_2 \in I\\
	&x_1y_1 - x_2y_2\in I,
\end{align*}
ou seja, $xy + I = x_2y_2 + I$. Portanto,
\[
	(x_1 + I) \otimes (y + I) = (x_2 + I) \otimes (y_2 + I).
\]

\begin{teorema}
	Seja $(A, +, \cdot)$ um anel associativo, comutativo e com unidade. Ent{\~a}o, se $I$ {\'e} um ideal de $A$,
	o quociente $\dfrac{A}{I}$ com as opera{\c c}{\~o}es $\oplus$ e $\otimes$ {\'e} um anel associativo,
	comutativo e com unidade. O elemento neutro da soma {\'e} a classe $0_{A} + I$ e unidade do produto {\'e} $1_{A} + I$.
\end{teorema}
%!TEX program = xelatex
%!TEX root = Algebra_1.tex
%%Usar makeindex -s indexstyle.ist arquivo.idx no terminal para gerar o {\'\i}ndice remissivo agrupado por inicial
%%Ap\'os executar pdflatex arquivo
\chapter{Grupos}

\begin{definicao}
   Seja $A$ um conjunto n\~ao vazio. Toda fun\c{c}\~ao $f : A \times A \to A$ \'e chamada de uma \textbf{opera\c{c}\~ao bin\'aria} sobre $A$.
\end{definicao}

Nas considera\c{c}\~oes que faremos a seguir uma opera\c{c}\~ao bin\'aria $f$ sobre $A$ que associa a cada par ordenado $(x, y) \in A \times A$ um elemento $f(x, y) \in A$ ser\'a denotada simplesmente por $*$. Assim escreveremos $f(x, y) = x*y$. Por exemplo a opera\c{c}\~ao $* : \n \times \n \to \n$ tal que $x*y = x^y$ est\'a bem definida pois $x^y \in \n$ sempre que $x$, $y \in \n$. Observe que esta opera\c{c}\~ao n\~ao pode ser definida em $\z$ pois por exemplo $2^{-1} \notin \z$. Tamb\'em n\~ao pode ser definida em $\rac$ pois $2^{1/2} \notin \rac$.

\begin{definicao}
	Seja $G$ um conjunto n{\~a}o vazio no qual está definida uma opera{\c c}{\~a}o bin{\'a}ria $*$ tal que:
	\begin{enumerate}[label={\roman*})]
		\item Para todos $x$, $y$, $z\in G$:
		\[
			(x*y)*z=x*(y*z)
		\]
		
		\item Existe $e \in G$ tal que
		\[
			x*e = x = e*x
		\]
		para todo $x \in G$. Tal elemento $e$ {\'e} chamado de \textbf{elemento neutro} ou \textbf{unidade} de $G$.

		\item Para cada $x \in G$, existe $y \in G$ tal que
		\[
			x*y = e = y*x
		\]
		O elemento $y$ {\'e} chamado de \textbf{inverso} ou \textbf{oposto} de $x$.
	\end{enumerate}
	Nesse caso dizemos que o par $(G, *)$ é um \textbf{grupo}.
\end{definicao}

\begin{observacao}
	Quando $*$ {\'e} uma soma, dizemos que $(G,*)$ {\'e} um \textbf{grupo aditivo}. Se $*$ {\'e} uma multiplica{\c c}{\~a}o, dizemos que $(G,*)$ {\'e} um \textbf{grupo multiplicativo}.

	Além disso, quando não houver chance de confusão com relação à operação do grupo $(G, *)$ vamos dizer simplesmente que $G$ é um grupo.
\end{observacao}

\begin{definicao}
	Um grupo $(G,*)$ {\'e} chamado de \textbf{grupo comutativo} ou \textbf{abeliano} quando $*$ {\'e} comutativa, ou seja, quando
	\[
		x*y = y*x
	\]
	para todos $x$, $y \in G$.
\end{definicao}

\begin{exemplos}
	\begin{enumerate}
		\item $(\z,+)$ {\'e} um grupo abeliano.
		\item $(\rac,+)$ {\'e} um grupo abeliano.
		\item $(\rac^*,\cdot)$ {\'e} um grupo abeliano.
		\item $(\real,+)$ {\'e} um grupo abeliano.
		\item $(\real^*,\cdot)$ {\'e} um grupo abeliano.
		\item Considere o conjunto dos n{\'u}meros reais $\mathbb{R}$ com a opera{\c c}{\~a}o $*$ definida por
		\[
			x*y = x + y - 3
		\]
		para $x$, $y \in \mathbb{R}$. Ent{\~a}o $(\mathbb{R}, *)$ {\'e} um grupo abeliano.
		\begin{solucao}
			De fato,
			\begin{enumerate}
				\item Para todos $x$, $y$, $z \real$
				\begin{align*}
					(x*y)*z &= (x+y-3)*z = (x+y-3)+z-3\\
					&= x+(y-3+z)-3 = x+(y+z-3)-3 = x*(y+z-3)\\
					&= x*(y*z)
				\end{align*}

				\item Para todo $x \in \mathbb{R}$, temos $x*3 = x + 3 - 3 = x = 3 * x$. Logo, 3 {\'e} o elemento neutro de $*$.

				\item Dado $x \in \mathbb{R}$, tome $y = 6 - x \in \real$. Assim
				\[
					x*y = x + (6-x)-3 = 3 = y*x.
				\]
				Assim $y = 6 - x$ é o oposto de $x$ na operação $*$ definida em $\real$.
			\end{enumerate}

			Portanto $(\real, *)$ é um grupo.

			Além disso, para todos $x$, $y \in \real$
			\[
				x*y = x + y - 3 = y + x - 3 = y*x
			\]
			Logo, $(\real, *)$ {\'e} um grupo comutativo.
		\end{solucao}

		\item $(\z_m,\oplus)$ {\'e} grupo.

		\item $(\z_m-\{\overline{0}\},\otimes)$ {\'e} grupo?
		\begin{solucao}
			Não pois por exemplo para $m = 4$ temos $\z4-\{\overline{0}\} = \{\overline{1}, \overline{2}, \overline{3}\} = G$ e dados $\overline{2}\in G$ temos $\overline{2} \otimes \overline{2} = \overline{0} \notin G$. Portanto a operação $\otimes$ não é uma operação binária em $G = \z_4$.
		\end{solucao}
	\end{enumerate}
\end{exemplos}


\begin{proposicao}
	Seja $(G,*)$ um grupo. Então:
	\begin{enumerate}[label={\roman*})]
		\item O elemento neutro de $G$ {\'e} {\'u}nico.

		\item Existe um {\'u}nico inverso para cada $x \in G$.

		\item Para todos $x$, $y \in G$,
		\[
			(x*y)^{-1} = y^{-1}*x^{-1}
		\]
		Por indu{\c c}{\~a}o, $x_1$, $x_2$, \dots ,$x_{n-1}$, $x_n \in G$,
		\[
			(x_1*x_2*\cdots *x_{n-1}*x_{n})^{-1} = x^{-1}_{n}*x^{-1}_{n-1}*\cdots *x^{-1}_2*x^{-1}_1
		\]
		\item Para todo $x \in G$, $(x^{-1})^{-1} = x$.
	\end{enumerate}

\end{proposicao}



\section{Ordem de um Grupo}
\begin{definicao}[Ordem de um grupo]
Quando um grupo $(G,*)$, $G$ {\'e} um conjunto com um n{\'u}mero finito de elementos, dizemos que $G$ {\'e} um grupo finito. Denotamos por $|G|$ o n{\'u}mero de elementos de $G$ que ser{\'a} chamado de ordem de $G$ ou cardinalidade de $G$. Quando $G$ n{\~a}o {\'e} finito, dizemos que $G$ {\'e} um grupo infinito.
\end{definicao}

Exemplos:
\begin{enumerate}
\item $(\z_m, +)$ {\'e} um grupo finito para todo $m>1$.
\item $(\z, +)$ {\'e} um grupo infinito.
\end{enumerate}

\section{Subgrupo}
\subsubsection{Defini{\c c}{\~a}o}

\begin{definicao}[Subgrupo]
Seja $(G,*)$ um grupo. Um subconjunto n{\~a}o vazio $H\subseteq G$ {\'e} um subgrupo se, e somente se, $(H,*)$ {\'e} um grupo.
\end{definicao}

\subsubsection{Propriedades}
\begin{proposicao}
Um subconjunto n{\~a}o vazio $H\subseteq G$ {\'e} um subgrupo de $G$ se, e somente se
\begin{enumerate}
\item $x^{-1}\in H,\forall x\in H$
\item $x*y\in H,\forall x,y\in H$
\end{enumerate}
\end{proposicao}

\textbf{Demonstra{\c c}{\~a}o}: Se $H$ {\'e} subgrupo, ent{\~a}o $H$ {\'e} um grupo. Logo 1 e 2 s{\~a}o satisfeitos.

Agora provemos que se $H$ satisfaz 1 e 2, ent{\~a}o $H$ {\'e} grupo.

Como $G$ {\'e} grupo, ent{\~a}o $*$ {\'e} associativo, logo $*$ {\'e} associativo em $H$.

De 1, $\forall x\in H,x^{-1}\in H$. Mas de 2, $\forall x,y\in H,\ x*y\in H$. Logo, se $x\in H$, ent{\~a}o $e=x*x^{-1}\in H$

Novamente por 1, todo elemento de $H$ possui inverso em $H$.

Logo, $(H,*)$ {\'e} um grupo.\#

Exemplos:
\begin{enumerate}
\item Dado $(G,*)$ grupo, $H=\{e\}$ e $H=G$ s{\~a}o subgrupos de $G$, chamados de subgrupos triviais
\item $(\mathbb{Z},+),\ H=m\mathbb{Z},\ m>1$

Ent{\~a}o $H$ {\'e} subgrupo de $\mathbb{Z}$
\item $G=U\left(\dfrac{\mathbb{Z}}{8\mathbb{Z}}\right)=\{\bar{1},\bar{3},\bar{5},\bar{7}\}$

$(G,\odot)$ {\'e} um grupo

$|G|$=4

$H_{1}=\{\bar{1},\bar{3}\}$ {\'e} subgrupo de G\\
$H_{2}=\{\bar{1},\bar{5}\}$ {\'e} subgrupo de G\\
$H_{3}=\{\bar{1},\bar{7}\}$ {\'e} subgrupo de G
\end{enumerate}

\section{Ordem de um subgrupo}

\begin{teorema}[Lagrange]
Seja $G$ um grupo finito. Se $H\subseteq G$ {\'e} um subgrupo, ent{\~a}o $|H|$ divide $|G|$.
\end{teorema}

Exemplo: Quais s{\~a}o as poss{\'\i}veis ordens dos subgrupos de um grupo de ordem 48?

Seja $G$ um grupo tal que $|G|=48$. Se $H$ {\'e} um subgrupo de $G$, ent{\~a}o $|H|$ divide $|G|$\\
$48=2^{4}3$ \\
$|H|=2,3,2^{2},2^{3},2^{4},2.3,2^{2}3,2^{2}3$

Observa{\c c}{\~a}o: O teorema n{\~a}o diz que haver{\'a} um subgrupo de ordem $n$ para todo $n$ tal que $n||G|$. Diz apenas que se $H$ {\'e} subgrupo de $G$, ent{\~a}o $|H|$ divide $|G|$.

\begin{corolario}
Os {\'u}nicos subgrupos de um grupo de ordem prima s{\~a}o os triviais
\end{corolario}

\textbf{Demonstra{\c c}{\~a}o}: Quando $|G|=p$ primo, temos que os {\'u}nicos divisores de $p$ positivos s{\~a}o 1 e $p$.

Ent{\~a}o, se $H$ {\'e} subgrupo de $G$, ent{\~a}o $|H|=1$ ou $|H|=p$.

Portanto, $H=\{e\}$ ou $H=G$.\#

\section{Homomorfimos de Grupos} % (fold)
\label{sec:homomorfimos_de_grupos}

Sejam $(G, *)$ e $(H, \triangle)$ grupos quaisquer. Considere uma fun\c{c}\~ao $f : G \to H$. Entre todas as poss{\'\i}veis fun\c{c}\~oes entre $G$ e $H$ vamos considerar somente aquelas que satisfa\c{c}\~ao a condi\c{c}\~ao
\[
	f(x * y) = f(x)\triangle f(y)
\]
para todos $x$, $y \in G$, ou seja, podemos determinar a imagem de $f(x*y)$ a partir da imagem de $x$ e de $y$,

\begin{definicao}
	Dados doi grupos $(G, *)$ e $(H,\triangle)$ dizemos que uma fun\c{c}\~ao $f : G \to H$ \'e um \textbf{homomorfismo de grupos} se
	\[
		f(x * y) = f(x)\triangle f(y)
	\]
	para todos $x$, $y \in G$.
\end{definicao}

\begin{observacao}
	Sejam $(G, *)$ e $(H, \triangle)$ grupos e $f : G \to H$ um homomorfismo.
	\begin{enumerate}
		\item Se $G = H$, neste caso $f : G \to G$ \'e chamado de um \textbf{endomorfimos} de grupos.
		\item Se $f : G \to H$ \'e uma fun\c{c}\~ao injetora, ent\~ao dizemos que $f$ \'e um \textbf{monomorfismo} de grupos.
		\item Se $f : G \to H$ \'e uma fun\c{c}\~ao sobrejetora, ent\~ao dizemos que $f$ \'e um \textbf{epimorfismo} de grupos.
		\item Se $f : G \to H$ \'e uma fun\c{c}\~ao bijetora, ent\~ao dizemos que $f$ \'e um \textbf{isomorfismo} de grupos.
		\item Se $f : G \to G$ \'e uma fun\c{c}\~ao bijetora, ent\~ao dizemos que $f$ \'e um \textbf{automorfismo} de grupos.
	\end{enumerate}
\end{observacao}

\begin{exemplos}
	\begin{enumerate}
		\item A fun\c{c}\~ao $f : \z \to \complex$ dada por $f(x) = i^x$ \'e um homomorfismo de $(\z, +)$ em $(\complex, \cdot)$. De fato,
		\[
			f(x + y) = i^{x + y} = i^x\cdot i^y = f(x)\cdot f(y)
		\]
		para todos $x$, $y \in \z$.

		\item A fun\c{c}\~ao $f : \real^*_+ \to \real$ dada por $f(x) = \ln(x)$ \'e um homomorfismo de $(\real^*_+, \cdot)$ em $(\real, +)$. De fato,
		\[
			f(xy) = \ln(xy) = \ln(x) + \ln(y) = f(x) + f(y)
		\]
		para todos $x$, $y \in \real^*_+$. Al\'em disso, como $\ln(x)$ \'e uma fun\c{c}\~ao bijetora, ent\~ao $f$ \'e um isomorfismo de grupos.

		\item Sejam $m$ um inteiro positivo fixo. A fun\c{c}\~ao $f: \z \to \z_m$ definida por $f(x) = \overline{x}$ \'e um homomorfimos de $(\z, +)$ em $(\z_m, \oplus)$. De fato,
		\[
			f(x + y) = \overline{x + y} = \overline{x} + \overline{y} = f(x) + f(y).
		\]
		Al\'em disso, esse homomorfismo \'e sobrejetor.
	\end{enumerate}
\end{exemplos}

\begin{proposicao}
	Sejam $(G, *)$ e $(H, \triangle)$ grupos e $f : G \to H$ um homomorfismo. Denote por $1_G$ e $1_H$ os elementos neutros de $G$ e $H$, respectivamente.
	\begin{enumerate}
		\item $f(1_G) = 1_H$
		\item $f(x^{-1}) = (f(x))^{-1}$ para todo $x \in G$.
	\end{enumerate}
\end{proposicao}

\begin{proposicao}
	Sejam $I$ \'e um subgrupo de $G$ e $f : G \to H$ um homomorfismo de grupos. Ent\~ao $f(I)$ \'e um subgrupo de $H$.
\end{proposicao}
\begin{prova}
	Como $I$ \'e um subgrupo de $G$, ent\~ao $1_G \in G$. Agora $f$ \'e um homomorfismo, logo $f(1_G) = 1_H \in f(I)$ e assim $f(I) \ne \emptyset$.

	Agora, dado $y \in f(I)$ precisamos mostrar que $y^{-1} \in f(I)$. Mas se $y \in f(I)$, ent\~ao $y = f(x)$ com $x \in I$. Da{\'\i}
	\[
		y^{-1} = [f(x)]^{-1} = f(x^{-1})
	\]
	e como $I$ \'e um subgrupo de $G$, $x^{-1} \in I$ e como isso $y^{-1} \in f(I)$.

	Finalmente, dados $y$, $z \in f(I)$ existem $x_1$, $x_2 \in I$ tais que $y = f(x_1)$ e $z = f(x_2)$. Mas $f$ \'e homomorfismo, da{\'\i}
	\[
		y\triangle z = f(x_1)\triangle f(x_2) = f(x_1*x_2)
	\]
	e como $I$ \'e subgrupo, $x_1*x_2 \in I$. Logo $y\triangle z \in f(I)$.

	Portanto $f(I)$ \'e um subgrupo de $H$.
\end{prova}

\begin{definicao}
	Sejam $(G, *)$ e $(H, \triangle)$ grupos e $f : G \to H$ um homomorfismo de grupos. Chama-se de \textbf{n\'ucleo} ou \textbf{kernel} de $f$ e denota-se por $N(f)$ ou $\ker(f)$ o seguinte subconjunto de $G$:
	\[
		\ker(f) = \{x \in G \mid f(x) = 1_H\}.
	\]
\end{definicao}

\begin{exemplos}
	\begin{enumerate}
		\item Considere o homomorfismo $f : \z \to \complex^*$ dado por $f(x) = i^x$. Temos
		\[
			\ker(f) = \{x \in \z \mid f(x) = 1\} = \{x \in \z \mid i^x = 1\} = \{0, \pm 4, \pm 8, \cdots\} = 4\z.
		\]

		\item O n\'ucleo do homomorfismo $f : \real^*_+ \to \real$ dado por $f(x) = \ln(x)$. Temos
		\[
			\ker(f) = \{x \in \real^*_+ \mid f(x) = 0\} = \{x \in \real^*_+ \mid \ln(x) = 0\} = \{1\}.
		\]

		\item O n\'ucleo do homomorfismo $f : \z \to \z_m$ dado por $f(x) = \overline{x}$, $m > 0$ fixo. Temos
		\[
			\ker(f) = \{x \in \z \mid f(x) = \overline{0}\} = \{x \in \z \mid \overline{x} = \overline{0}\} = \{0, \pm m, \pm 2m, \cdots\}.
		\]
	\end{enumerate}
\end{exemplos}

\begin{proposicao}
	Sejam $f : G \to H$ um homomorfismo de grupos. Ent\~ao:
	\begin{enumerate}
		\item $\ker(f)$ \'e um subgrupo de $G$.
		\item $f$ \'e um monomorfismo se, e somente se, $\ker(f) = \{1_G\}$.
	\end{enumerate}
\end{proposicao}
\begin{prova}
	\begin{enumerate}
		\item Como $f(1_G) = 1_H$, ent\~ao $1_G \in \ker(f)$ e com isso $\ker(f) \ne \emptyset$. Se $x \in \ker(f)$, ent\~ao $f(x^{-1}) = [f(x)]^{-1} = 1_H^{-1} = 1_H$ e da{\'\i} $x^{-1} \in \ker(f)$. Finalmente se $x$, $y \in \ker(f)$, ent\~ao $f(x*y) = f(x)\triangle f(y) = 1_H \triangle 1_H = 1_H$, ou seja, $x * y \in \ker(f)$.

		Portanto $\ker(f)$ \'e um subgrupo de $G$.

		\item Suponha que $f$ \'e um monomorfismo de grupos. Tome $x \in \ker(f)$. Temos $f(x) = 1_H = f(1_G)$ e como $f$ \'e injetora $x = 1_G$. Logo $\ker(f) = \{1_G\}$.

		Agora suponha que $\ker(f) = \{1_G\}$. Sejam $x$, $y \in G$ tais que
		\begin{align*}
			f(x) &= f(y)\\
			f(x)\triangle f(y)^{-1} & = 1_H\\
			f(x)\triangle f(y^{-1}) &= 1_H\\
			f(x * y^{-1}) &= 1_H
		\end{align*}
		e da{\'\i} $x*y^{-1} \in \ker(f) = \{1_G\}$. Logo $x*y^{-1} = 1_G$, isto \'e, $x = y$. Portanto $f$ \'e injetora.
	\end{enumerate}
\end{prova}




% section homomorfimos_de_grupos (end)

\section{Grupos de Permuta\c{c}\~ao}
Fazer a parte de $S_n$.

\section{Grupos C{\'\i}clicos}
Fazer a parte de grupos c{\'\i}clicos.
%!TEX program = xelatex
%!TEX root = Algebra1.tex
%%Usar makeindex -s indexstyle.ist arquivo.idx no terminal para gerar o {\'\i}ndice remissivo agrupado por inicial
%%Ap\'os executar pdflatex arquivo
\cleardoublepage
\phantomsection
\addcontentsline{toc}{chapter}{Bibliografia}
\renewcommand{\bibname}{Bibliografia}

\begin{thebibliography}{99}
\bibitem{HI} H.H. Domingues, G.Iezzi: \textit{{\'A}lgebra Moderna}, 2ª Ed., Atual, 1982
\bibitem{Shok} S. Shokranian: \textit{{\'A}lgebra 1}, Ci{\^e}ncia Moderna, 2010
\bibitem{AG} Adilson Gon{\c c}alves: \textit{Introdu{\c c}{\~a}o {\`a} {\'A}lgebra}, 5ª Ed., IMPA, 2003
\bibitem{Birk} G. Birkhoff, S. MacLane: \textit{{\'A}lgebra Moderna B{\'a}sica}, 4ª Ed., Guanabara Dois, 1980
\bibitem{Filho} E. A. Filho: \textit{Inicia{\c c}{\~a}o {\`a} L{\'o}gica Matem{\'a}tica}, Nobel, 2002
\end{thebibliography}
\cleardoublepage
\phantomsection
\addcontentsline{toc}{chapter}{\'Indice Remissivo}
\printindex

\end{document}
