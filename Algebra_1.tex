%!TEX program = xelatex
%!TEX encoding = ISO-8859-1
\documentclass[portuguese,twoside,12pt]{report}%opcao draft remove os links
\usepackage{amssymb,amsmath,amsfonts,amsthm,amstext}
\usepackage[brazil]{babel}
\usepackage{inputenc}
\usepackage{fancybox}
\usepackage{niceframe}
\usepackage{hyperref}
\usepackage{graphicx}
\graphicspath{{/Users/jfreitas/Dropbox/imagens-latex/}{/home/jfreitas/Dropbox/imagens-latex/}}
\usepackage{makeidx}

%=============================================================================================
%             Nomes para defini{\c c}{\~o}es, teoremas, etc
%=============================================================================================

\newtheorem{teorema}{Teorema}[chapter]
\newtheorem{definicao}{Defini{\c c}{\~a}o}[chapter]
\newtheorem{nota}{Nota{\c c}{\~a}o}[definicao]
\newtheorem{proposicao}{Proposi{\c c}{\~a}o}[definicao]
\newtheorem{lema}{Lema}[proposicao]
\newtheorem{corolario}{Corol{\'a}rio}[proposicao]

%=============================================================================================
%             Cabe{\c c}alhos
%=============================================================================================

\usepackage{fancyhdr}
\pagestyle{fancy} \addtolength{\headwidth}{\marginparsep}
\addtolength{\headwidth}{\marginparwidth}
\renewcommand{\headrulewidth}{1pt}
\renewcommand{\chaptermark}[1]{\markboth{{CAP. \thechapter\ $\bullet$ \; #1}}{}}
\renewcommand{\sectionmark}[1]{\markright{{SE{\c C}{\~A}O \thesection\ $\bullet$ \; #1}}}
\fancyhf{} \fancyhead[RO,RE]{\small\bfseries\textrm \thepage}
\fancyhead[LO]{\small\bfseries\textrm \leftmark}
\fancyhead[LE]{\small\bfseries\textrm \rightmark}

%=============================================================================================
%             Estilo dos t{\'\i}tulos de cap{\'\i}tulo, se{\c c}{\~a}o, etc
%=============================================================================================
\usepackage{sectsty}
\usepackage[Conny]{fncychap}
\sectionfont{\rmfamily\raggedright\sectionrule{0.3in}{1pt}{-0.1in}{1pt}}
\subsectionfont{\rmfamily\raggedright}
\chapterfont{\thispagestyle{empty}}
\ChNameVar{\Huge\rm\bfseries} \ChTitleVar{\Huge\rm\bfseries}

%=============================================================================================
%             Medidas
%=============================================================================================


\setlength{\headsep}{1cm}                    % DIST{\^A}NCIA TEXTO/CABE{\c C}ALHO
\setlength{\textwidth}{16.5 cm}              % LARGURA DO TEXTO
\setlength{\textheight}{21.5 cm}             % ALTURA DO TEXTO
\setlength{\oddsidemargin}{0.1 cm}           % MARGEM {\'I}MPAR
\setlength{\evensidemargin}{0.4 cm}          % MARGEM PAR
\setlength{\topmargin}{0 cm}                 % MARGEM SUPERIOR
\renewcommand{\baselinestretch}{1.4}         % DIST{\^A}NCIA ENTRE LINHAS

%=============================================================================================
%             Comandos Pessoais
%=============================================================================================

\newcommand{\sub}{\subseteq}
\newcommand{\real}{\mathbb{R}}
%=============================================================================================


\makeindex



\title{\Huge \textbf{{\'A}lgebra 1}\\ \vspace{0,5cm} \Large Notas de Aula}
\author{\Large Departamento de Matem{\'a}tica\\ \\ \large Universidade de Bras{\'\i}lia - UnB}
\date{2/2010}

\begin{document}
\maketitle

\begin{center}
\Huge \textbf{Pref{\'a}cio}
\end{center}
\hspace{0,5cm}Essas notas de Aula s{\~a}o referentes {\`a} mat{\'e}ria {\'A}lgebra 1,
ministrada na UnB - Universidade de Bras{\'\i}lia - durante o 2� Semestre de 2010
pelo professor Jos{\'e} Ant{\^o}nio de O. Freitas, Departamento de Matem{\'a}tica. Tais
notas foram transcritas e editadas pelo graduando em Ci{\^e}ncias Econ{\^o}micas
Luiz Eduardo Sol R. da Silva\footnote{luizeduardosol@hotmail.com}.

{\'E} livre a reprodu{\c c}{\~a}o, distribui{\c c}{\~a}o e edi{\c c}{\~a}o deste material, desde que citadas as suas fontes e autores. Cr{\'\i}ticas e sugest{\~o}es s{\~a}o bem vindas.
\vspace{20cm}




\begin{center} \textbf{\Large Nota{\c c}{\~o}es e express{\~o}es}
\end{center}
\begin{minipage}[l]{0,5\textwidth}
\begin{itemize}
\item $\neg$ N{\~a}o
\item $\forall$ Para todo
\item $/$ Tal que
\item $|$ Divide
\item $\Rightarrow$ Implica
\item $\in$ Pertence
\item $\emptyset$ Vazio
\item $\subseteq$ Contido ou igual a
\item $\supseteq$ Cont{\'e}m ou igual a
\item $\wedge$ E
\item $\vee$ Ou
\item $=$ Igual
\item $\neq$ Diferente
\item $\mathbb{Z}$ N{\'u}meros Inteiros
\item $\mathbb{R}$ N{\'u}meros Reais
\item $\cap$ Intersec{\c c}{\~a}o
\item $>$ Maior que
\item $\geq$ Maior ou igual a
\item $\displaystyle\bigcup_{i=1}^{n}$ Uni{\~a}o de $n$ conjuntos
\item $\displaystyle\bigsqcup_{i=1}^{n}$ Uni{\~a}o disjunta de $n$ conjuntos


\end{itemize}
\end{minipage}
\begin{minipage}[r]{0,5\textwidth}
\begin{itemize}

\item $\leftrightarrow$ Se, e somente se
\item $\veebar$ Ou...,ou..., mas nunca ambos
\item $\rightarrow$ Se,... ent{\~a}o...
\item $\exists$ Existe
\item $\Leftrightarrow$ Equivalente a
\item $\notin$ N{\~a}o pertence
\item \# Fim da demonstra{\c c}{\~a}o
\item $\mathbb{N}$ N{\'u}meros Naturais
\item $\mathbb{Q}$ N{\'u}meros Racionais
\item $\nsubseteq$ N{\~a}o cont{\'e}m ou {\'e} igual a
\item $\cup$ Uni{\~a}o
\item $\sqcup$ Uni{\~a}o Disjunta
\item $<$ Menor que
\item $\leq$ Menor ou igual a
\item $\displaystyle\bigcap_{i=1}^{n}$ Intersec{\c c}{\~a}o de $n$ conjuntos
\item Q.E.D. (\textit{Quod Erat Demonstrandum}): Como se queria demonstrar
\item P.B.O.: Princ{\'\i}pio da boa ordena{\c c}{\~a}o
\item H.I.: Hip{\'o}tese de Indu{\c c}{\~a}o
\item \textit{Mutatis Mutandis}:  Mudando o que tem que ser mudado

\end{itemize}
\end{minipage}

\tableofcontents




\chapter{Grupos}

\section{Defini{\c c}{\~a}o}
\begin{definicao}[Grupo] Um grupo $G$ {\'e} um conjunto n{\~a}o vazio munido com uma opera{\c c}{\~a}o bin{\'a}ria $*$ tal que
\begin{enumerate}
\item Para todo $x,y,z\in G,\ (x*y)*z=x*(y*z)$ (Associatividade)
\item Existe $e\in G$ tal que $x*e=e*x=x$ para todo $x\in G$. Tal elemento $E$ {\'e} chamado de elemento neutro ou unidade
\item Para cada $x\in G$, existe $x^{-1}\in G$ tal que $x*x^{-1}=x^{-1}*x=e$. O elemento $x^{-1}$ {\'e} chamado de inverso\footnote{$x^{-1}\neq\displaystyle\frac{1}{x}$} de $x$
\end{enumerate}
\end{definicao}

Denotamos um grupo $G$, cuja opera{\c c}{\~a}o bin{\'a}ria {\'e} $*$, por $(G,*)$. Quando $*$ {\'e} a soma, dizemos que $(G,*)$ {\'e} um grupo aditivo. Se $*$ {\'e} a multiplica{\c c}{\~a}o, dizemos que $(G,*)$ {\'e} um grupo multiplicativo.\\

\section{Grupo comutativo ou abeliano}
\begin{definicao}[Grupo comutativo ou abeliano] Um grupo $(G,*)$ {\'e} chamado de grupo comutativo ou abeliano quando $*$ {\'e} comutativa, ou seja, \[x*y=y*x\] para todo $x,y\in G$
\end{definicao}
\vspace{1cm}

Exemplos:
\begin{enumerate}
\item $(\mathbb{Z},+$) {\'e} um grupo abeliano
\item Se $(A,+,.)$ {\'e} um anel, ent{\~a}o $(A,+)$ {\'e} um grupo
\item $\left(\displaystyle\frac{\mathbb{Z}}{m\mathbb{Z}},\oplus\right)$ {\'e} grupo
\item $\left(\displaystyle\frac{\mathbb{Z}}{m\mathbb{Z}}-\{\bar{0}\},\odot\right)$ {\'e} grupo?\\
$\displaystyle\frac{\mathbb{Z}}{4\mathbb{Z}}-\{\bar{0}\}=\{\bar{1},\bar{2},\bar{3}\}=G$\\
$\bar{2}\in G,\ \bar{2}\odot\bar{2}=\bar{0}\notin G$
\item $\left(U\left(\displaystyle\frac{\mathbb{Z}}{m\mathbb{Z}}\right),\odot\right)$ {\'e} um grupo\\
\item Considere o conjunto dos n{\'u}meros reais $\mathbb{R}$ com a opera{\c c}{\~a}o $*$ definida por \[x*y=x+y-3\], $x,y\in\mathbb{R}$. Ent{\~a}o $(\mathbb{R},*)$ {\'e} um grupo abeliano.

De fato
\begin{enumerate}
\item \[(x*y)*g=(x+y-3)*g=(x+y-3)+z-3\]
\[=x+(y-3+z)-3=x+(y+z-3)-3)=x*(y+z-3)\]
\[=x*(y*3)\] para todo $x,y,z\in \mathbb{R}$
\item $x*y=x+y-3=y+x-3=y*x$ para todo $x,y\in\mathbb{R}$. Logo, $*$ {\'e} comutativa
\item Para todo $x\in\mathbb{R}$, temos $x*3=x+3-3=x$. Logo, 3 {\'e} o elemento neutro de $*$.
\item Dado $x\in\mathbb{R}$, tome $x^{-1}=6-x$. Assim \[x*x^{-1}=x+(6-x)-3=3\]

Logo, para $x\in\mathbb{R}$ o inverso de $x$ por $*$ {\'e} $6-x$.

Portanto $(\mathbb{R}, *)$ {\'e} um grupo comutativo
\end{enumerate}
\item Considere um conjunto com dois elementos $G=\{x,y\}$. em $G$, considere a opera{\c c}{\~a}o $\triangle$ dada por (Tabela \ref{6})
\begin{table}[h]
   \centering 
   \setlength{\arrayrulewidth}{0,5\arrayrulewidth}
   \caption{\it Opera{\c c}{\~a}o $\triangle$}
   \begin{tabular}{|c|c|c|c|c|} 
      \hline
      $\triangle$ & $x$ & $y$ \\
     \hline
      $x$ & $x$ & $y$ \\
      \hline
      $y$ & $y$ & $x$ \\
      \hline
   \end{tabular}
\label{6}
\end{table}

$(G,\triangle)$ {\'e} um grupo?

$(x\triangle x)\triangle y= x\triangle(x\triangle y)$

$(x\triangle y)\triangle x=x\triangle(y\triangle x)$

$\vdots$

$\triangle$ {\'e} associativa

$x$ {\'e} neutro para $\triangle$\\
$x^{-1}=x$\\
$y^{-1}=y$

Logo, $(G,\triangle)$ {\'e} um grupo
\end{enumerate}

\section{Propriedades Imediatas de um grupo}

Seja $(G,*)$ um grupo. {\'E} f{\'a}cil ver que
\begin{enumerate}
\item O elemento neutro {\'e} {\'u}nico
\item Existe um {\'u}nico inverso para cada $x\in G$
\item Para todos $x,y\in G,(x*y)^{-1}=y^{-1}*x^{-1}$. Por indu{\c c}{\~a}o, $x_{1},x_{2},...,x_{n-1},x_{n}\in G$, \[(x_{1}*x_{2}*...*x_{n-1}*x_{n})^{-1}\] \[=x^{-1}_{n}*x^{-1}_{n-1}*...*x^{-1}_{2}*x^{-1}_{1}\]
\item Para todo $x\in G, (x^{-1})^{-1}=x$

\end{enumerate}

\section{Ordem de um Grupo}
\begin{definicao}[Ordem de um grupo]
Quando um grupo $(G,*)$, $G$ {\'e} um conjunto com um n{\'u}mero finito de elementos, dizemos que $G$ {\'e} um grupo finito. Denotamos por $|G|$ o n{\'u}mero de elementos de $G$ que ser{\'a} chamado de ordem de $G$ ou cardinalidade de $G$. Quando $G$ n{\~a}o {\'e} finito, dizemos que $G$ {\'e} um grupo infinito.
\end{definicao}

Exemplos:
\begin{enumerate}
\item $\left(\dfrac{\mathbb{Z}}{m\mathbb{Z}}, *\right)$ {\'e} um grupo finito para todo $m>1$
\item $(\mathbb{Z}, +)$ {\'e} um grupo infinito
\end{enumerate}

\section{Subgrupo}
\subsubsection{Defini{\c c}{\~a}o}

\begin{definicao}[Subgrupo]
Seja $(G,*)$ um grupo. Um subconjunto n{\~a}o vazio $H\subseteq G$ {\'e} um subgrupo se, e somente se, $(H,*)$ {\'e} um grupo.
\end{definicao}

\subsubsection{Propriedades}
\begin{proposicao}
Um subconjunto n{\~a}o vazio $H\subseteq G$ {\'e} um subgrupo de $G$ se, e somente se
\begin{enumerate}
\item $x^{-1}\in H,\forall x\in H$
\item $x*y\in H,\forall x,y\in H$
\end{enumerate}
\end{proposicao}

\textbf{Demonstra{\c c}{\~a}o}: Se $H$ {\'e} subgrupo, ent{\~a}o $H$ {\'e} um grupo. Logo 1 e 2 s{\~a}o satisfeitos.

Agora provemos que se $H$ satisfaz 1 e 2, ent{\~a}o $H$ {\'e} grupo.

Como $G$ {\'e} grupo, ent{\~a}o $*$ {\'e} associativo, logo $*$ {\'e} associativo em $H$.

De 1, $\forall x\in H,x^{-1}\in H$. Mas de 2, $\forall x,y\in H,\ x*y\in H$. Logo, se $x\in H$, ent{\~a}o $e=x*x^{-1}\in H$

Novamente por 1, todo elemento de $H$ possui inverso em $H$.

Logo, $(H,*)$ {\'e} um grupo.\#

Exemplos:
\begin{enumerate}
\item Dado $(G,*)$ grupo, $H=\{e\}$ e $H=G$ s{\~a}o subgrupos de $G$, chamados de subgrupos triviais
\item $(\mathbb{Z},+),\ H=m\mathbb{Z},\ m>1$

Ent{\~a}o $H$ {\'e} subgrupo de $\mathbb{Z}$
\item $G=U\left(\dfrac{\mathbb{Z}}{8\mathbb{Z}}\right)=\{\bar{1},\bar{3},\bar{5},\bar{7}\}$

$(G,\odot)$ {\'e} um grupo

$|G|$=4

$H_{1}=\{\bar{1},\bar{3}\}$ {\'e} subgrupo de G\\
$H_{2}=\{\bar{1},\bar{5}\}$ {\'e} subgrupo de G\\
$H_{3}=\{\bar{1},\bar{7}\}$ {\'e} subgrupo de G
\end{enumerate}

\section{Ordem de um subgrupo}

\begin{teorema}[Lagrange]
Seja $G$ um grupo finito. Se $H\subseteq G$ {\'e} um subgrupo, ent{\~a}o $|H|$ divide $|G|$.
\end{teorema}

Exemplo: Quais s{\~a}o as poss{\'\i}veis ordens dos subgrupos de um grupo de ordem 48?

Seja $G$ um grupo tal que $|G|=48$. Se $H$ {\'e} um subgrupo de $G$, ent{\~a}o $|H|$ divide $|G|$\\
$48=2^{4}3$ \\
$|H|=2,3,2^{2},2^{3},2^{4},2.3,2^{2}3,2^{2}3$

Observa{\c c}{\~a}o: O teorema n{\~a}o diz que haver{\'a} um subgrupo de ordem $n$ para todo $n$ tal que $n||G|$. Diz apenas que se $H$ {\'e} subgrupo de $G$, ent{\~a}o $|H|$ divide $|G|$.

\begin{corolario}
Os {\'u}nicos subgrupos de um grupo de ordem prima s{\~a}o os triviais
\end{corolario}

\textbf{Demonstra{\c c}{\~a}o}: Quando $|G|=p$ primo, temos que os {\'u}nicos divisores de $p$ positivos s{\~a}o 1 e $p$.

Ent{\~a}o, se $H$ {\'e} subgrupo de $G$, ent{\~a}o $|H|=1$ ou $|H|=p$.

Portanto, $H=\{e\}$ ou $H=G$.\#




\chapter{Bibliografia}
\begin{enumerate}
\item H.H. Domingues, G.Iezzi: \textit{{\'A}lgebra Moderna}, 2� Ed., Atual, 1982
\item S. Shokranian: \textit{{\'A}lgebra 1}, Ci{\^e}ncia Moderna, 2010
\item Adilson Gon{\c c}alves: \textit{Introdu{\c c}{\~a}o {\`a} {\'A}lgebra}, 5� Ed., IMPA, 2003
\item G. Birkhoff, S. MacLane: \textit{{\'A}lgebra Moderna B{\'a}sica}, 4� Ed., Guanabara Dois, 1980
\item E. A. Filho: \textit{Inicia{\c c}{\~a}o {\`a} L{\'o}gica Matem{\'a}tica}, Nobel, 2002

\end{enumerate}







\printindex

\end{document}
