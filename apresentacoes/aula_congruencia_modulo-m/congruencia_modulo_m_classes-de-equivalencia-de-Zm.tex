%!TEX program = xelatex
\def\ano{2020}
\def\semestre{1}
\def\disciplina{Álgebra 1}
\def\turma{C}
\def\autor{José Antônio O. Freitas}
\def\instituto{MAT-UnB}

\documentclass{beamer}
\usetheme{Madrid}
\usecolortheme{beaver}
% \mode<presentation>
\usepackage{caption}
\usepackage{amssymb}
\usepackage{amsmath,amsfonts,amsthm,amstext}
\usepackage[brazil]{babel}
% \usepackage[latin1]{inputenc}
\usepackage{graphicx}
\graphicspath{{/ArquivosLinux/OneDrive/imagens-latex/}{D:/OneDrive - unb.br/imagens-latex/}}
\usepackage{enumitem}
\usepackage{multicol}
\usepackage{answers}
\usepackage{tikz,ifthen}
\usetikzlibrary{lindenmayersystems}
\usetikzlibrary[shadings]
\newtheorem{definicao}{Definição}[section]
\newtheorem{definicoes}{Definições}[section]
\newtheorem{exemplo}{Exemplo}[section]
\newtheorem{exemplos}{Exemplos}[section]
\newtheorem{exercicio}{Exercício}
\newtheorem{observacao}{Observação:}[section]
\newtheorem{observacoes}{Observações:}[section]
\newtheorem*{solucao}{Solução:}
\newtheorem{proposicao}{Proposição}
\newtheorem{lema}{Lema}
\newtheorem{teorema}{Teorema}
\newtheorem{corolario}{Corolário}
\newenvironment{prova}[1][Prova]{\noindent\textbf{#1:} }{\qedsymbol}%{\ \rule{0.5em}{0.5em}}
\def\ano{2024}
\def\semestre{1}
\def\disciplina{Álgebra 1}
\def\nomeabreviado{Álgebra 1}
\def\turma{1}

\newcommand{\im}{{\rm Im\,}}
\newcommand{\dlim}[2]{\displaystyle\lim_{#1\rightarrow #2}}
\newcommand{\minf}{+\infty}
\newcommand{\ninf}{-\infty}
\newcommand{\cp}[1]{\mathbb{#1}}
\newcommand{\sub}{\subseteq}
\newcommand{\n}{\mathbb{N}}
\newcommand{\z}{\mathbb{Z}}
\newcommand{\rac}{\mathbb{Q}}
\newcommand{\real}{\mathbb{R}}
\newcommand{\complex}{\mathbb{C}}

\newcommand{\vesp}[1]{\vspace{ #1  cm}}

\newcommand{\compcent}[1]{\vcenter{\hbox{$#1\circ$}}}
\newcommand{\comp}{\mathbin{\mathchoice
        {\compcent\scriptstyle}{\compcent\scriptstyle}
        {\compcent\scriptscriptstyle}{\compcent\scriptscriptstyle}}}
\renewcommand{\sin}{{\rm sen\,}}
\renewcommand{\tan}{{\rm tg\,}}
\renewcommand{\csc}{{\rm cossec\,}}
\renewcommand{\cot}{{\rm cotg\,}}
\renewcommand{\sinh}{{\rm senh\,}}

\title{Congruência módulo $m$ e relações de equivalência em $\z$}
\author[\autor]{\autor}
\institute[\instituto]{\instituto}
\date{}

\begin{document}
    \begin{frame}
        \maketitle
    \end{frame}

    \logo{\includegraphics[scale=.1]{logo-MAT.png}\vspace*{8.5cm}}

    \begin{frame}
        \begin{definicao}
            Sejam $a$, $b \in \z$. \pause Dizemos que $b$ \textbf{divide} $a$ \pause quando existe um inteiro $k$ tal que $a=bk$. \pause Nesse caso escrevemos $b \mid a$. \pause Quando $b$ \textbf{não divide} $a$, \pause escrevemos $b\not{\mid}a$.
        \end{definicao}
    \end{frame}

    \begin{frame}
        \begin{proposicao}
            \begin{enumerate}[label={\roman*})]
                \item $a\mid a$, para todo $a \in \z$.\pause \vspace{.3cm}
                \item Se $a\mid b$ e $b\mid a$, $a$, $b > 0$ então $a = b$.\pause \vspace{.3cm}
                \item Se $a\mid b$ e $b\mid c$, então $a\mid c$.\pause \vspace{.3cm}
                \item Se $a\mid b$ e $a\mid c$, então $a\mid (bx+cy)$, para todos $x$, $y \in \z$.
            \end{enumerate}
        \end{proposicao}
    \end{frame}

    \begin{frame}

        \begin{definicao}
            Sejam $a$, $b \in\z$, \pause dizemos que $a$ \textbf{é congruente à} $b$ \pause \textbf{módulo} $m$ \pause se $m \mid (a-b)$. \pause Neste caso, escrevemos $a\equiv_{m} b$ \pause ou $a\equiv b \pmod{m}$.
        \end{definicao}
    \end{frame}

    \begin{frame}
        \begin{teorema}
            A relação de congruência módulo $m$ satisfaz as seguintes propriedades:\pause
            \begin{enumerate}[label={\roman*})]
                \item $a_{1}\equiv b_{1}\pmod{m}$ se, e somente se, $a_{1}-b_{1}\equiv 0\pmod{m}$.\pause \vspace{.3cm}

                \item Se $a_{1}\equiv b_{1}\pmod{m}$ e $a_{2}\equiv b_{2}\pmod{m}$, então $a_{1}+a_{2}\equiv b_{1}+b_{2}\pmod{m}$.\pause \vspace{.3cm}

                \item Se $a_{1}\equiv b_{1}\pmod{m}$ e $a_{2}\equiv b_{2}\pmod{m}$, então $a_{1}a_{2}\equiv b_{1}b_{2}\pmod{m}$.\label{item_provado}\pause \vspace{.3cm}

                \item Se $a\equiv b\pmod{m}$, então $ax\equiv bx\pmod{m}$, para todo $x \in \z$.\pause \vspace{.3cm}

                \item Vale a lei do cancelamento: se $d \in \z$ e $mdc(d,m) = 1$ então $ad \equiv bd \pmod{m}$ implica $a\equiv b \pmod{m}$.
            \end{enumerate}
        \end{teorema}
    \end{frame}

    \begin{frame}
        \begin{proposicao}
            A congruência módulo $m$ é uma relação de equivalência em $\z$.
        \end{proposicao}
    \end{frame}

    \begin{frame}
        Dado $n \in \z$, temos\pause
        \[
            \overline{n} = \pause C(n) = \pause \{x \in \z \mid \pause x\equiv n \pmod{m}\}.\pause
        \]

        Vamos dentoar $C(n)$ \pause por $R_{m}(n)$ \pause ou $\overline{n}$, \pause quando não houver possibilidade de confusão. \pause Assim fixando $m > 1$ vamos escrever\pause
        \begin{center}
            \begin{tabular}{l}
                $R_{m}(0) = \pause \{x \in \z \mid \pause x\equiv 0 \pmod{m}\} \pause =\{x \in \z \mid \pause x = mk, k \in \z\}\pause = m\z$\pause\\
                \\
                $R_{m}(1) = \pause \{x \in \z \mid \pause x\equiv 1 \pmod{m}\} \pause = \{x \in \z \mid \pause x = km + 1, k \in \z \}$\pause \\
                \\
                \vdots\\
                \\
                $R_{m}(n) = \pause \{x \in \z \mid \pause x = km + n, k \in \z \}$
            \end{tabular}
        \end{center}
    \end{frame}

    \begin{frame}
        \begin{proposicao}
            As classes de equivalência definidas pela congruência módulo $m$ \pause são determinadas pelos restos da divisão inteira por $m$. \pause Em outras palavras, $R_{m}(n)$ \pause é o conjunto dos números inteiros \pause cujo resto na divisão inteira por $m$ é $n$.\pause
        \end{proposicao}

        \begin{corolario}
            $R_{m}(k) = R_{m}(l)$ \pause se, e somente se, $k\equiv l \pmod{m}$.
        \end{corolario}
    \end{frame}

    \begin{frame}
        \begin{proposicao}
            Na relação de equivalência módulo $m$ existem $m$ classes de equivalência.\pause
        \end{proposicao}
    \end{frame}

    \begin{frame}
        \begin{observacao}
        Fixado $m$ inteiro positivo, \pause denotaremos\pause
        \begin{center}
            \begin{tabular}{l}
                $R_{m}(0) = \overline{0}$\pause \\
                \\
                $R_{m}(1) = \overline{1}$\pause \\
                \\
                $\vdots$\\
                \\
                $R_{m}(m-1) = \overline{m-1}$\pause
            \end{tabular}
        \end{center}

        O conjunto quociente \pause desta relação será denotado por $\displaystyle\frac{\z}{m\z}$ \pause ou $\z_m$. \pause Assim\pause
        \[
            \z_m = \pause \displaystyle\frac{\z}{m\z} = \pause \{\overline{0}, \pause \overline{1}, \pause ..., \pause \overline{m-1}\}.\pause
        \]
        \end{observacao}
    \end{frame}

    \begin{frame}
        \begin{definicao}
            Dados $\overline{a}$, $\overline{b} \in \z_m$ definimos\pause
            \begin{center}
                \begin{tabular}{l}
                    $\overline{a}\oplus\overline{b} = \pause \overline{a + b}\label{soma_modulo_m}$\pause \\
                    \\
                    $\overline{a}\otimes\overline{b} = \pause \overline{ab}.\label{multiplicacao_modulo_m}$\pause
                \end{tabular}
            \end{center}
        \end{definicao}

        \begin{proposicao}
            As operaç{õ}es de soma \pause e a multiplicação \pause definidas acima são independentes dos representantes das classes.
        \end{proposicao}
    \end{frame}

    \begin{frame}
        \begin{proposicao}
            As operações de soma $\oplus$ \pause e multiplicação $\otimes$ \pause em $\z_m$ satisfazem as seguintes propriedades:\pause
            \begin{enumerate}[label={\roman*})]
                \item Para todos $\overline{x}$, $\overline{y} \in \z_m$: \pause $\overline{x} \oplus \overline{y} = \overline{y} \oplus \overline{x}$. \vspace{.2cm} \pause

                \item Para todos $\overline{x}$, $\overline{y}$ e $\overline{z} \in \z_m$: \pause $(\overline{x} \oplus \overline{y}) \oplus \overline{z} = \pause \overline{x} \oplus (\overline{y} \oplus \overline{z})$. \vspace{.2cm} \pause

                \item Para todo $\overline{x} \in \z_m$, \pause $\overline{x} \oplus \overline{0} = \overline{x}$. \vspace{.2cm} \pause

                \item Para todo $\overline{x} \in \z_m$, \pause existe $\overline{y} \in \z_m$ \pause tal que $\overline{x} \oplus \overline{y} = \overline{0}$. \vspace{.2cm} \pause

                \item Para todos $\overline{x}$, $\overline{y} \in \z_m$: \pause $\overline{x} \otimes \overline{y} = \overline{y} \otimes \overline{x}$. \vspace{.2cm} \pause

                \item Para todos $\overline{x}$, $\overline{y}$ e $\overline{z} \in \z_m$: \pause $(\overline{x} \otimes \overline{y}) \otimes \overline{z} = \pause \overline{x} \otimes (\overline{y} \otimes \overline{z})$. \vspace{.2cm} \pause

                \item Para todo $\overline{x} \in \z_m$: \pause $\overline{x} \otimes \overline{1} = \overline{x}$.
            \end{enumerate}
        \end{proposicao}
    \end{frame}

    \begin{frame}
        \begin{definicao}
            Um elemento $\overline{a} \in \z_m$ é \textbf{inversível} \pause se, e somente se, existe $\overline{b} \in \z_m$ \pause tal que $\overline{a} \otimes \overline{b} = \pause \overline{1}$. \pause Neste caso, $\overline{b}$ \pause é chamado \textbf{inverso} de $\overline{a}$ \pause e denotaremos $\overline{b} = (\overline{a})^{-1}$.\pause
        \end{definicao}

        \begin{proposicao}
            Se o inverso existe, \pause então ele é único.
        \end{proposicao}
    \end{frame}

    \begin{frame}
        \begin{proposicao}
            Um elemento $\overline{a} \in \z_m$ é \pause inversível \pause se, e somente se, $mdc(a,m) = 1$.\pause
        \end{proposicao}

        \begin{corolario}
            Se $m$ é um número primo, \pause então para todo $\overline{x} \in \z_m$, \pause $\overline{x} \ne \overline{0}$, \pause existe inverso.
        \end{corolario}
    \end{frame}
\end{document}
