%!TEX program = xelatex
\def\ano{2020}
\def\semestre{2}
\def\disciplina{\'Algebra 1}
\def\turma{C}
\def\autor{Jos\'e Ant\^onio O. Freitas}
\def\instituto{MAT-UnB}

\documentclass{beamer}
\usetheme{Madrid}
\usecolortheme{beaver}
% \mode<presentation>
\usepackage{caption}
\usepackage{amssymb}
\usepackage{amsmath,amsfonts,amsthm,amstext}
\usepackage[brazil]{babel}
% \usepackage[latin1]{inputenc}
\usepackage{graphicx}
\graphicspath{{/ArquivosLinux/OneDrive/imagens-latex/}{D:/OneDrive - unb.br/imagens-latex/}}
\usepackage{enumitem}
\usepackage{multicol}
\usepackage{answers}
\usepackage{tikz,ifthen}
\usetikzlibrary{lindenmayersystems}
\usetikzlibrary[shadings]
\newtheorem{definicao}{Definição}[section]
\newtheorem{definicoes}{Definições}[section]
\newtheorem{exemplo}{Exemplo}[section]
\newtheorem{exemplos}{Exemplos}[section]
\newtheorem{exercicio}{Exercício}
\newtheorem{observacao}{Observação:}[section]
\newtheorem{observacoes}{Observações:}[section]
\newtheorem*{solucao}{Solução:}
\newtheorem{proposicao}{Proposição}
\newtheorem{lema}{Lema}
\newtheorem{teorema}{Teorema}
\newtheorem{corolario}{Corolário}
\newenvironment{prova}[1][Prova]{\noindent\textbf{#1:} }{\qedsymbol}%{\ \rule{0.5em}{0.5em}}
\def\ano{2024}
\def\semestre{1}
\def\disciplina{Álgebra 1}
\def\nomeabreviado{Álgebra 1}
\def\turma{1}

\newcommand{\im}{{\rm Im\,}}
\newcommand{\dlim}[2]{\displaystyle\lim_{#1\rightarrow #2}}
\newcommand{\minf}{+\infty}
\newcommand{\ninf}{-\infty}
\newcommand{\cp}[1]{\mathbb{#1}}
\newcommand{\sub}{\subseteq}
\newcommand{\n}{\mathbb{N}}
\newcommand{\z}{\mathbb{Z}}
\newcommand{\rac}{\mathbb{Q}}
\newcommand{\real}{\mathbb{R}}
\newcommand{\complex}{\mathbb{C}}

\newcommand{\vesp}[1]{\vspace{ #1  cm}}

\newcommand{\compcent}[1]{\vcenter{\hbox{$#1\circ$}}}
\newcommand{\comp}{\mathbin{\mathchoice
        {\compcent\scriptstyle}{\compcent\scriptstyle}
        {\compcent\scriptscriptstyle}{\compcent\scriptscriptstyle}}}
\renewcommand{\sin}{{\rm sen\,}}
\renewcommand{\tan}{{\rm tg\,}}
\renewcommand{\csc}{{\rm cossec\,}}
\renewcommand{\cot}{{\rm cotg\,}}
\renewcommand{\sinh}{{\rm senh\,}}

\title{No\c{c}\~oes de L\'ogica}
\author[\autor]{\autor}
\institute[\instituto]{\instituto}
\date{}

\begin{document}
    \begin{frame}
        \maketitle
    \end{frame}

    \logo{\includegraphics[scale=.1]{logo-MAT.png}\vspace*{8.5cm}}

    \begin{frame}
        \begin{definicao}
            Uma \textbf{proposi\c{c}\~ao} \'e enunciado, \pause por meio de palavras\pause\ ou s{\'\i}mbolos, \pause ao qual podemos atribuir um \textbf{valor l\'ogico}.\pause
        \end{definicao}

        \begin{definicao}
            Diz-se que o \textbf{valor l\'ogico} \pause de uma proposi\c{c}\~ao \'e \pause ``verdade'' (V) \pause se a proposi\c{c}\~ao \'e verdadeira \pause ou ``falsidade'' (F) \pause se a proposi\c{c}\~ao \'e falsa.
        \end{definicao}
    \end{frame}

    \begin{frame}
        \begin{center}
            ``Toda proposi\c{c}\~ao tem um, \pause e um s\'o, \pause dos valores l\'ogicos \textbf{verdade} ou \textbf{falsidade}.''\pause
        \end{center}
        Isso \'e conhecido como \pause \textbf{Princ{\'\i}pio da n\~ao contradi\c{c}\~ao \pause e do terceiro exclu{\'\i}do}.\pause
    \end{frame}

    \begin{frame}
        Vamos considerar com proposi\c{c}\~oes da forma:\pause

        \begin{center}
            Se $\mathbb{H}$, ent\~ao $\mathbb{T}$.\pause
        \end{center}

        $\mathbb{H}$ \'e a hip\'otese\pause

        $\mathbb{T}$ \'e a tese.\pause

        \begin{center}
            $\mathbb{H}$ se, e somente se, $\mathbb{T}$\pause

            ou

            $\mathbb{H}$ se, e s\'o se, $\mathbb{T}$.\pause
        \end{center}

        Essa última proposi\c{c}\~ao poder decomposta em duas proposi\c{c}\~oes da seguinte forma:\pause
        \begin{itemize}
            \item[1)] Se $\mathbb{H}$, \pause ent\~ao $\mathbb{T}$.\pause
            \item[2)] Se $\mathbb{T}$, \pause ent\~ao $\mathbb{H}$.
        \end{itemize}
    \end{frame}

    \begin{frame}
        Temos 3 caminhos para tentar provar uma proposi\c{c}\~ao do tipo:
        \begin{center}
            Se $\mathbb{H}$, ent\~ao $\mathbb{T}$.\pause
        \end{center}

        \begin{itemize}
            \item[1)] Demonstra\c{c}\~ao direta: \pause neste caso admitimos a hip\'otese $\mathbb{H}$ \textbf{verdadeira}, e utilizando de uma
                sequ\^encia de passos cuja veracidade podemos comprovar, e com isso chegar \`a conclus\~ao que a tese $\mathbb{T}$ tamb\'em \'e
                \textbf{verdadeira}.\pause
            \item[2)] Demonstra\c{c}\~ao por contraposi\c{c}\~ao: \pause neste caso supomos que a $\mathbb{T}$ \'e \textbf{falsa} e devemos
                chegar \`a conclus\~ao que a hip\'otese $\mathbb{H}$ tamb\'em \'e \textbf{falsa}. Se conseguirmos chegar \`a essa
                conclus\~ao, ent\~ao a proposi\c{c}\~ao original ser\'a \textbf{verdadeira}.\pause
            \item[3)] Demonstra\c{c}\~ao por contradi\c{c}\~ao ou redu\c{c}\~ao ao absurdo: \pause aqui vamos supor que a hip\'otese
                $\mathbb{H}$ \'e \textbf{verdadeira} \pause e que a tese $\mathbb{T}$ \'e \textbf{falsa}. Usando essas suposi\c{c}\~oes
                devemos chegar \`a alguma conclus\~ao contradit\'oria. \pause Nesse caso, significa que a nossa tese $\mathbb{T}$ deve ser
                obrigatoriamente \textbf{verdadeira}, e com isso a proposi\c{c}\~ao tamb\'em ser\'a \textbf{verdadeira}.\pause
        \end{itemize}
    \end{frame}
\end{document}
