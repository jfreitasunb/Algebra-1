%!TEX program = xelatex
%!TEX encoding = UTF-8
\def\ano{2024}
\def\semestre{1}
\def\disciplina{Álgebra 1}
\def\turma{C}
\def\autor{José Antônio O. Freitas}
\def\instituto{MAT-UnB}

\documentclass{beamer}
\usetheme{Madrid}
\usecolortheme{beaver}
% \mode<presentation>
\usepackage{caption}
\usepackage{amssymb}
\usepackage{amsmath,amsfonts,amsthm,amstext}
\usepackage[brazil]{babel}
% \usepackage[latin1]{inputenc}
\usepackage{graphicx}
\graphicspath{{/ArquivosLinux/OneDrive/imagens-latex/}{D:/OneDrive - unb.br/imagens-latex/}}
\usepackage{enumitem}
\usepackage{multicol}
\usepackage{answers}
\usepackage{tikz,ifthen}
\usetikzlibrary{lindenmayersystems}
\usetikzlibrary[shadings]
\newtheorem{definicao}{Definição}[section]
\newtheorem{definicoes}{Definições}[section]
\newtheorem{exemplo}{Exemplo}[section]
\newtheorem{exemplos}{Exemplos}[section]
\newtheorem{exercicio}{Exercício}
\newtheorem{observacao}{Observação:}[section]
\newtheorem{observacoes}{Observações:}[section]
\newtheorem*{solucao}{Solução:}
\newtheorem{proposicao}{Proposição}
\newtheorem{lema}{Lema}
\newtheorem{teorema}{Teorema}
\newtheorem{corolario}{Corolário}
\newenvironment{prova}[1][Prova]{\noindent\textbf{#1:} }{\qedsymbol}%{\ \rule{0.5em}{0.5em}}
\def\ano{2024}
\def\semestre{1}
\def\disciplina{Álgebra 1}
\def\nomeabreviado{Álgebra 1}
\def\turma{1}

\newcommand{\im}{{\rm Im\,}}
\newcommand{\dlim}[2]{\displaystyle\lim_{#1\rightarrow #2}}
\newcommand{\minf}{+\infty}
\newcommand{\ninf}{-\infty}
\newcommand{\cp}[1]{\mathbb{#1}}
\newcommand{\sub}{\subseteq}
\newcommand{\n}{\mathbb{N}}
\newcommand{\z}{\mathbb{Z}}
\newcommand{\rac}{\mathbb{Q}}
\newcommand{\real}{\mathbb{R}}
\newcommand{\complex}{\mathbb{C}}

\newcommand{\vesp}[1]{\vspace{ #1  cm}}

\newcommand{\compcent}[1]{\vcenter{\hbox{$#1\circ$}}}
\newcommand{\comp}{\mathbin{\mathchoice
        {\compcent\scriptstyle}{\compcent\scriptstyle}
        {\compcent\scriptscriptstyle}{\compcent\scriptscriptstyle}}}
\renewcommand{\sin}{{\rm sen\,}}
\renewcommand{\tan}{{\rm tg\,}}
\renewcommand{\csc}{{\rm cossec\,}}
\renewcommand{\cot}{{\rm cotg\,}}
\renewcommand{\sinh}{{\rm senh\,}}

\title{Noções de Lógica}
\author[\autor]{\autor}
\institute[\instituto]{\instituto}
\date{}

\begin{document}
    \begin{frame}
        \maketitle
    \end{frame}

    \logo{\includegraphics[scale=.1]{logo-MAT.png}\vspace*{8.5cm}}

    \begin{frame}
        \begin{definicao}
            Uma \textbf{proposição} é enunciado, \pause por meio de palavras\pause\ ou símbolos, \pause ao qual podemos atribuir um \textbf{valor lógico}.\pause
        \end{definicao}

        \begin{definicao}
            Diz-se que o \textbf{valor lógico} \pause de uma proposição é \pause ``verdade'' (V) \pause se a proposição é verdadeira \pause ou ``falsidade'' (F) \pause se a proposição é falsa.
        \end{definicao}
    \end{frame}

    \begin{frame}
        \begin{center}
            ``Toda proposição tem um, \pause e um só, \pause dos valores lógicos \textbf{verdade} ou \textbf{falsidade}.''\pause
        \end{center}
        Isso é conhecido como \pause \textbf{Princípio da não contradição \pause e do terceiro excluído}.\pause
    \end{frame}

    \begin{frame}
        Vamos considerar com proposições da forma:\pause

        \begin{center}
            Se $\mathbb{H}$, então $\mathbb{T}$.\pause
        \end{center}

        $\mathbb{H}$ é a hipótese\pause

        $\mathbb{T}$ é a tese.\pause

        \begin{center}
            $\mathbb{H}$ se, e somente se, $\mathbb{T}$\pause

            ou

            $\mathbb{H}$ se, e só se, $\mathbb{T}$.\pause
        \end{center}

        Essa última proposição poder decomposta em duas proposições da seguinte forma:\pause
        \begin{itemize}
            \item[1)] Se $\mathbb{H}$, \pause então $\mathbb{T}$.\pause
            \item[2)] Se $\mathbb{T}$, \pause então $\mathbb{H}$.
        \end{itemize}
    \end{frame}

    \begin{frame}
        Temos 3 caminhos para tentar provar uma proposição do tipo:
        \begin{center}
            Se $\mathbb{H}$, então $\mathbb{T}$.\pause
        \end{center}

        \begin{itemize}
            \item[1)] Demonstração direta: \pause neste caso admitimos a hipótese $\mathbb{H}$ \textbf{verdadeira}, e utilizando de uma
                sequência de passos cuja veracidade podemos comprovar, e com isso chegar à conclusão que a tese $\mathbb{T}$ também é
                \textbf{verdadeira}.\pause
            \item[2)] Demonstração por contraposição: \pause neste caso supomos que a $\mathbb{T}$ é \textbf{falsa} e devemos
                chegar à conclusão que a hipótese $\mathbb{H}$ também é \textbf{falsa}. Se conseguirmos chegar à essa
                conclusão, então a proposição original será \textbf{verdadeira}.\pause
            \item[3)] Demonstração por contradição ou redução ao absurdo: \pause aqui vamos supor que a hipótese
                $\mathbb{H}$ é \textbf{verdadeira} \pause e que a tese $\mathbb{T}$ é \textbf{falsa}. Usando essas suposições
                devemos chegar à alguma conclusão contraditória. \pause Nesse caso, significa que a nossa tese $\mathbb{T}$ deve ser
                obrigatoriamente \textbf{verdadeira}, e com isso a proposição também será \textbf{verdadeira}.\pause
        \end{itemize}
    \end{frame}
\end{document}
