%!TEX program = xelatex
%!TEX encoding = UTF-8
\def\ano{2024}
\def\semestre{1}
\def\disciplina{Álgebra 1}
\def\turma{C}
\def\autor{José Antônio O. Freitas}
\def\instituto{MAT-UnB}

\documentclass{beamer}
\usetheme{Madrid}
\usecolortheme{beaver}
% \mode<presentation>
\usepackage{caption}
\usepackage{amssymb}
\usepackage{amsmath,amsfonts,amsthm,amstext}
\usepackage[brazil]{babel}
% \usepackage[latin1]{inputenc}
\usepackage{graphicx}
\graphicspath{{/ArquivosLinux/OneDrive/imagens-latex/}{D:/OneDrive - unb.br/imagens-latex/}}
\usepackage{enumitem}
\usepackage{multicol}
\usepackage{answers}
\usepackage{tikz,ifthen}
\usetikzlibrary{lindenmayersystems}
\usetikzlibrary[shadings]
\newtheorem{definicao}{Definição}[section]
\newtheorem{definicoes}{Definições}[section]
\newtheorem{exemplo}{Exemplo}[section]
\newtheorem{exemplos}{Exemplos}[section]
\newtheorem{exercicio}{Exercício}
\newtheorem{observacao}{Observação:}[section]
\newtheorem{observacoes}{Observações:}[section]
\newtheorem*{solucao}{Solução:}
\newtheorem{proposicao}{Proposição}
\newtheorem{lema}{Lema}
\newtheorem{teorema}{Teorema}
\newtheorem{corolario}{Corolário}
\newenvironment{prova}[1][Prova]{\noindent\textbf{#1:} }{\qedsymbol}%{\ \rule{0.5em}{0.5em}}
\def\ano{2024}
\def\semestre{1}
\def\disciplina{Álgebra 1}
\def\nomeabreviado{Álgebra 1}
\def\turma{1}

\newcommand{\im}{{\rm Im\,}}
\newcommand{\dlim}[2]{\displaystyle\lim_{#1\rightarrow #2}}
\newcommand{\minf}{+\infty}
\newcommand{\ninf}{-\infty}
\newcommand{\cp}[1]{\mathbb{#1}}
\newcommand{\sub}{\subseteq}
\newcommand{\n}{\mathbb{N}}
\newcommand{\z}{\mathbb{Z}}
\newcommand{\rac}{\mathbb{Q}}
\newcommand{\real}{\mathbb{R}}
\newcommand{\complex}{\mathbb{C}}

\newcommand{\vesp}[1]{\vspace{ #1  cm}}

\newcommand{\compcent}[1]{\vcenter{\hbox{$#1\circ$}}}
\newcommand{\comp}{\mathbin{\mathchoice
        {\compcent\scriptstyle}{\compcent\scriptstyle}
        {\compcent\scriptscriptstyle}{\compcent\scriptscriptstyle}}}
\renewcommand{\sin}{{\rm sen\,}}
\renewcommand{\tan}{{\rm tg\,}}
\renewcommand{\csc}{{\rm cossec\,}}
\renewcommand{\cot}{{\rm cotg\,}}
\renewcommand{\sinh}{{\rm senh\,}}

\title{Grupos}
\author[\autor]{\autor}
\institute[\instituto]{\instituto}
\date{}

\begin{document}
    \begin{frame}
        \maketitle
    \end{frame}

    \logo{\includegraphics[scale=.1]{logo-MAT.png}\vspace*{8.5cm}}

    \begin{frame}
        \begin{definicao}
            Seja $G \ne \emptyset$ \pause um conjunto no qual está definida uma operação binária $*$ \pause tal que:\pause
            \begin{enumerate}
                \item[i)] Para todos $x$, $y$, $z\in G$:\pause
                \[
                    (x*y)*z \pause = x*(y*z).\pause
                \]

                \item[ii)] Existe $e \in G$ \pause tal que\pause
                \[
                    x*e \pause = x = \pause e*x\pause
                \]
                para todo $x \in G$. \pause Tal elemento $e$ \pause é chamado de \textbf{elemento neutro} \pause ou \textbf{unidade} \pause de $G$.\pause

            \end{enumerate}
        \end{definicao}
    \end{frame}

    \begin{frame}
        \begin{definicao}
            \begin{enumerate}
                \item[iii)] Para cada $x \in G$, \pause existe $y \in G$ \pause tal que\pause
                \[
                    x*y \pause = e = \pause y*x.\pause
                \]
                O elemento $y$ \pause é chamado de \textbf{inverso} \pause ou \textbf{oposto} \pause de $x$.\pause
            \end{enumerate}
            Nesse caso dizemos que o par $(G, *)$ \pause é um \textbf{grupo}.\pause
        \end{definicao}
    \end{frame}

    \begin{frame}
        \begin{observacao}
            Quando $*$ é uma ``soma", \pause dizemos que $(G,*)$ é um \textbf{grupo aditivo}.\pause

            \vspace{.3cm}

            Se $*$ é uma ``multiplicação", \pause dizemos que $(G,*)$ é um \textbf{grupo multiplicativo}.\pause

            \vspace{.3cm}
            Além disso, quando não houver chance de confusão com relação à operação do grupo $(G, *)$ \pause vamos dizer simplesmente que $G$ é um grupo.\pause
        \end{observacao}
        \end{frame}

        \begin{frame}
        \begin{definicao}
            Um grupo $(G,*)$ \pause é chamado de \textbf{grupo comutativo} \pause ou \textbf{abeliano} \pause quando $*$ é comutativa, \pause ou seja, quando \pause
            \[
                x*y = \pause y*x \pause
            \]
            para todos $x$, $y \in G$.
        \end{definicao}
    \end{frame}

    \begin{frame}
        \begin{exemplos}
            \begin{enumerate}[label={\arabic*})]
                \item Se $(A,\oplus, \otimes)$ é um anel, então $(A, \oplus)$ é um grupo abeliano.\pause

                \vspace{.3cm}

                \item Denote por $\mathbb{K}$ \pause um dos conjuntos $\z$, \pause $\rac$, \pause $\real$ \pause ou $\complex$, \pause indistintamente. Seja \pause
                \begin{center}
                    $M_{r \times s}(\mathbb{K}) \pause = \{A \mid $ A é uma matriz \pause \\de $r$ linhas \pause por $s$ \pause colunas cujas entradas estão em $\mathbb{K} \pause\}$.
                \end{center}
                Então $(M_{r \times s}(\mathbb{K}), +)$ \pause onde $+$ é a soma de matrizes \pause é um grupo abeliano. \pause

                \vspace{.3cm}

                \item Denote por $\mathbb{K}$ \pause um dos conjuntos $\rac$, \pause $\real$ \pause ou $\complex$, \pause indistintamente. Seja \pause
                \begin{center}
                    $GL_n(\mathbb{K}) \pause = \{A \in M_{n \times n}(\mathbb{K}) \pause \mid \det(A) = 1\}$. \pause
                \end{center}
                Então $(GL_n(\mathbb{K}), \cdot)$ \pause onde $\cdot$ é a multiplicação de matrizes é um grupo \pause não abeliano. \pause
            \end{enumerate}
        \end{exemplos}
    \end{frame}

    \begin{frame}
        \begin{proposicao}
            Seja $(G,*)$ um grupo. \pause Então:\pause
            \begin{enumerate}[label={\roman*})]
                \item O elemento neutro de $G$ é único.\pause

                \vspace{.3cm}

                \item Existe um único inverso para cada $x \in G$.

                \seti
            \end{enumerate}
        \end{proposicao}
    \end{frame}

    \begin{frame}
        \begin{proposicao}
            \begin{enumerate}[label={\roman*})]

                \conti

                \item Para todos $x$, $y \in G$,\pause
                \[
                    (x*y)^{-1} \pause = y^{-1}*x^{-1}.\pause
                \]
                Por indução, \pause $x_1$, $x_2$, \dots ,$x_{n-1}$, $x_n \in G$,\pause
                \[
                    (x_1*x_2*\cdots *x_{n-1}*x_{n})^{-1} = \pause x^{-1}_{n}\pause *x^{-1}_{n-1}\pause *\cdots *\pause x^{-1}_2\pause *x^{-1}_1\pause
                \]
                \item Para todo $x \in G$, \pause
                \[
                    (x^{-1})^{-1} \pause = x.
                \]
            \end{enumerate}
        \end{proposicao}
    \end{frame}

    \begin{frame}
        \begin{definicao}
            Seja $(G,*)$ um grupo. \pause Um subconjunto não vazio \pause $H\sub G$ \pause é chamado de \textbf{subgrupo} de $G$ \pause se, e somente se, $(H,*)$ \pause é um grupo.\pause
        \end{definicao}

        \begin{proposicao}
            Seja $(G, *)$ um grupo. \pause Um subconjunto não vazio \pause $H\subseteq G$ é um subgrupo de $G$ \pause se, e somente se\pause
            \begin{enumerate}[label={\roman*})]
                \item\label{subgrupo_condicao_1} $x^{-1}\in H$, \pause para todo $x \in H$;
                \item\label{subgrupo_condicao_2} $x*y\in H$, \pause para todos $x$, $y \in H$.\pause
            \end{enumerate}
        \end{proposicao}
    \end{frame}

    \begin{frame}
        \begin{exemplos}
            \begin{enumerate}[label={\roman*})]
                \item Dado $(G,*)$ grupo, \pause $H=\{e\}$ \pause e $H=G$ \pause são subgrupos de $G$, \pause chamados de \textbf{subgrupos triviais}.\pause

                \item Seja $(\mathbb{Z},+)$ um grupo. \pause Tomando $H = m\z$, \pause onde $m > 1$, então $H$ é subgrupo de $\z$.\pause

                \item $G = U(\z_8) = \{\overline{1}, \overline{3}, \overline{5}, \overline{7}\}$. \pause Então $(G,\odot)$ é um grupo \pause com $|G| = 4$. \pause Além disso,\pause
                \begin{center}
                    \begin{tabular}{l}
                        $H_1 = \{\overline{1}, \overline{3}\}\pause$\\
                        $H_2 = \{\overline{1}, \overline{5}\}\pause$\\
                        $H_3 = \{\overline{1}, \overline{7}\}\pause$
                    \end{tabular}
                \end{center}
                São subgrupos de $G$.
                \seti
            \end{enumerate}
        \end{exemplos}
    \end{frame}

    \begin{frame}
        \begin{definicao}
            Dados dois grupos $(G, *)$ \pause e $(H,\triangle)$ \pause dizemos que uma função $f : G \to H$ \pause é um \textbf{homomorfismo de grupos} se \pause
            \[
                f(x * y) = \pause f(x)\triangle f(y) \pause
            \]
            para todos $x$, $y \in G$.
        \end{definicao}
    \end{frame}

    \begin{frame}
        \begin{observacao}
            Sejam $(G, *)$ \pause, $(H, \triangle)$ grupos \pause e $f : G \to H$ um homomorfismo. \pause

            \vspace{.5cm}

            \begin{enumerate}[label={\arabic*})]
                \item Se $G = H$, \pause neste caso $f : G \to G$ \pause é chamado de um \textbf{endomorfismo} de grupos.\pause

                \vspace{.5cm}

                \item Se $f : G \to H$ é uma função injetora, \pause então dizemos que $f$ é um \textbf{monomorfismo} de grupos.\pause

                \vspace{.5cm}

                \seti
            \end{enumerate}
        \end{observacao}
    \end{frame}

    \begin{frame}
        \begin{observacao}
            \vspace{.5cm}

            \begin{enumerate}[label={\arabic*})]
                \conti
                \item Se $f : G \to H$ é uma função sobrejetora, \pause então dizemos que $f$ é um \textbf{epimorfismo} de grupos.\pause

                \vspace{.5cm}

                \item Se $f : G \to H$ é uma função bijetora, \pause então dizemos que $f$ é um \textbf{isomorfismo} de grupos.\pause

                \vspace{.5cm}

                \item Se $f : G \to G$ é uma função bijetora, \pause então dizemos que $f$ é um \textbf{automorfismo} de grupos.

                \vspace{.5cm}

            \end{enumerate}
        \end{observacao}
    \end{frame}

    \begin{frame}
        \begin{exemplos}
            \begin{enumerate}[label={\arabic*})]
                \item A função $f : \z \to \complex^*$ \pause dada por $f(x) = i^x$ \pause é um homomorfismo de $(\z, +)$ \pause em $(\complex^*, \cdot)$.
                \seti
            \end{enumerate}

            \vspace{2cm}
        \end{exemplos}
    \end{frame}
    \begin{frame}
        \begin{exemplos}
            \begin{enumerate}[label={\arabic*})]
                \conti

                \item A função $f : \real^*_+ \to \real$ \pause dada por $f(x) = \ln(x)$ \pause é um homomorfismo de $(\real^*_+, \cdot)$ \pause em $(\real, +)$.
                \seti
            \end{enumerate}
            \vspace{2cm}
        \end{exemplos}
    \end{frame}
    \begin{frame}
        \begin{exemplos}
            \begin{enumerate}[label={\arabic*})]
                \conti

                \item Sejam $m$ um inteiro positivo fixo. \pause A função $f: \z \to \z_m$ \pause definida por $f(x) = \overline{x}$ \pause é um homomorfismos de $(\z, +)$ \pause em $(\z_m, \oplus)$.
                \seti
            \end{enumerate}
            \vspace{2cm}
        \end{exemplos}
    \end{frame}
    \begin{frame}
        \begin{exemplos}
            \begin{enumerate}[label={\arabic*})]
                \conti

                \item Sejam $(G, *)$ um grupo, \pause $z\in G$ um elemento fixado \pause e $z^{-1}$ seu inverso. \pause Então a aplicação\pause
                \begin{center}
                    \begin{tabular}{c}
                        $f_z: G\to G$\pause\\
                        $f_z(x) = z^{-1}*x*z$,\pause
                    \end{tabular}
                \end{center}
                 para todo $x \in G$, é um isomorfismo de grupos.
            \end{enumerate}
            \vspace{.5cm}
        \end{exemplos}
    \end{frame}

    \begin{frame}
        \begin{proposicao}
            Sejam $(G, *)$, \pause $(H, \triangle)$ grupos \pause e $f : G \to H$ um homomorfismo. \pause Denote por $1_G$ \pause e $1_H$ \pause os elementos neutros de $G$ e $H$, \pause respectivamente.\pause
            \vspace{.5cm}
            \begin{enumerate}[label={\roman*})]
                \item $f(1_G) = 1_H$\pause

                \vspace{.5cm}

                \item $[f(x)]^{-1} = f(x^{-1})$ \pause para todo $x \in G$.
                \vspace{.5cm}
            \end{enumerate}
        \end{proposicao}
    \end{frame}

    \begin{frame}
        \begin{proposicao}
            Sejam $I$ é um subgrupo de $H$ \pause e $f : G \to H$ \pause um homomorfismo de grupos. \pause Então $f^{-1}(I)$ \pause é um subgrupo de $G$.
        \end{proposicao}

        \vspace{2cm}
    \end{frame}

    \begin{frame}
        \begin{definicao}
            Sejam $(G, *)$, \pause $(H, \triangle)$ grupos \pause e $f : G \to H$ um homomorfismo de grupos. \pause Chama-se de \textbf{núcleo} \pause ou \textbf{kernel} \pause de $f$ e denota-se por \pause $N(f)$ \pause ou $\ker(f)$ \pause o seguinte subconjunto de $G$:\pause
            \[
                \ker(f) = \pause \{x \in G \pause \mid f(x) = 1_H\}.
            \]
        \end{definicao}
    \end{frame}

    \begin{frame}
        \begin{exemplos}
            \begin{enumerate}[label={\arabic*})]
                \item Considere o homomorfismo $f : \z \to \complex^*$ \pause dado por $f(x) = i^x$. \pause O kernel de $f$ é:
                \seti
            \end{enumerate}
        \end{exemplos}
        \vspace{2cm}
    \end{frame}
    \begin{frame}
        \begin{exemplos}
            \begin{enumerate}[label={\arabic*})]
            \conti
                \item Considere o homomorfismo $g : \real^*_+ \to \real$ \pause dado por $g(x) = \ln(x)$. \pause O núcleo de $g$ é:
                \seti
            \end{enumerate}
        \end{exemplos}
        \vspace{2cm}
    \end{frame}
    \begin{frame}
        \begin{exemplos}
            \begin{enumerate}[label={\arabic*})]
            \conti
                \item Considere o homomorfismo $h : \z \to \z_m$ \pause dado por $h(x) = \overline{x}$, \pause $m > 0$ fixo. \pause O kernel de $h$ é:
            \end{enumerate}
        \end{exemplos}
        \vspace{2cm}
    \end{frame}

    \begin{frame}
        \begin{proposicao}
            Sejam $(G, *)$, \pause $(H, \triangle)$ grupos \pause e $f : G \to H$ um homomorfismo de grupos. \pause Então: \pause
            \vspace{.5cm}

            \begin{enumerate}[label={\roman*})]
                \item $\ker(f)$ é um subgrupo de $G$. \pause

                \vspace{.5cm}

                \item $f$ é um monomorfismo se, e somente se, $\ker(f) = \{1_G\}$.

                \vspace{.5cm}
            \end{enumerate}
        \end{proposicao}
    \end{frame}

    \begin{frame}
        \begin{proposicao}
            Sejam $H$, $J$ e $L$ grupos. \pause Se $f : H \to J$ \pause e $g : J \to L$ \pause são homomorfismos de grupos, \pause então $g \circ f : H \to L$ \pause também é um homomorfismo de grupos.
        \end{proposicao}
        \vspace{2cm}
    \end{frame}

    \begin{frame}
        \begin{corolario}
            Se $f$ e $g$ são homomorfismos \pause injetores \pause (sobrejetores), então $g \circ f$ \pause também é um homomorfismo injetor \pause (sobrejetor).
        \end{corolario}
        \vspace{2cm}
    \end{frame}

    \begin{frame}
        \begin{proposicao}
            Se $f : G \to H$ é um isomorfismo de grupos, \pause então $f^{-1} : H \to G$ \pause também é um isomorfismo de grupos.
        \end{proposicao}
    \end{frame}

    \begin{frame}
        \begin{exemplos}
            \begin{enumerate}[label={\roman*})]
                \conti
                \item Considere o grupo aditivo $M_2(\real)$. \pause Mostre que o conjunto\pause
                \[
                    H = \left\{\begin{pmatrix}
                        a & b\\c & d
                    \end{pmatrix} \in M_2(\real) \mid a + d = 0\right\} \pause
                \]
                é um subgrupo de $M_2(\real)$.
            \end{enumerate}
        \end{exemplos}
    \end{frame}

     \begin{frame}
        Seja $A$ um conjunto não vazio.\pause

        \vspace{.3cm}

        Dada uma função $f : A \to A$, sabemos que $f$ possui inversa \pause se, e somente se, $f$ é bijetora.\pause

        \vspace{.3cm}

        Assim considere o conjunto\pause
        \[
            \mathcal{S} = \{ f : A \to A \pause \mid f \mbox{ é bijetora}\}.\pause
        \]

        Em $\mathcal{S}$ vamos considerar a composição de funções $\circ$.\pause

        \vspace{.3cm}

        Como $id : A \to A$ tal que $id(x) = x$ para todo $x \in A$ \pause é uma função bijetora \pause então $id \in \mathcal{S}$ \pause e com isso $\mathcal{S} \ne \emptyset$.
    \end{frame}

    \begin{frame}
        Dadas $f$, $g \in \mathcal{S}$ \pause como $f$ e $g$ são bijetoras, \pause então $f \circ g$ é bijetora \pause e daí $f \circ g \in \mathcal{S}$. \pause Isto é, \pause a composição de funções \pause é uma operação binária em $\mathcal{S}$.

        \vspace{.3cm}

        Agora, sejam $f$, $g$ e $h \in \mathcal{S}$. \pause Para todo $x \in A$ temos\pause
        \begin{center}
            $[(f\circ g) \pause \circ h \pause](x) \pause = (f \circ g) \pause(h(x)) \pause = f(g(h(x))) \pause$\\
            $[f\circ \pause(g\circ h) \pause](x) \pause = f( \pause(g\circ h)(x) \pause) = f(g(h(x))) \pause$
        \end{center}

        Logo $(f\circ g) \pause \circ h  \pause = f\circ \pause (g\circ h)$. \pause

        \vspace{.3cm}

        Agora, para toda $f \in \mathcal{S}$\pause
        \[
            f\circ id  \pause = f  \pause = id\circ f, \pause
        \]
        onde $id : A \to A$ \pause é tal que $id(x) = x$, \pause para todo $x \in A$. \pause Logo $id$ é o elemento neutro da composição.\pause

    \end{frame}

    \begin{frame}

        Finalmente, \pause para toda $f \in \mathcal{S}$, \pause como $f$ é bijetora \pause existe $g \in \mathcal{S}$ \pause tal que \pause
        \[
            f\circ g \pause = id \pause = g \circ f. \pause
        \]
        Logo todo elemento de $\mathcal{S}$ \pause possui inverso. \pause

        \vspace{.3cm}

        Portanto $(\mathcal{S}, \circ)$ \pause é um grupo. \pause Além disso, em geral, esse grupo não é comutativo. \pause

        \vspace{.3cm}

        Vamos considerar agora o caso particular \pause em que $A$ é um conjunto finito. \pause

        \vspace{.3cm}

        Nessa situação podemos supor que $A \sub \n$ \pause para simplificar a notação. \pause

        \vspace{.3cm}

        Vamos ver como é o conjunto $\mathcal{S}$ com essa hip\'otese. \pause
    \end{frame}

    \begin{frame}
        Se $A = \{1\}$, \pause então s\'o existe uma função $f : A \to A$ \pause que é bijetora e essa função é tal que \pause
        \begin{center}
            $f : \{1\} \to \{1\}$ \pause\\
            $f(1) = 1$. \pause
        \end{center}
        Ou seja, $f$ é a função identidade $id$. \pause Nesse caso $\mathcal{S} \pause = S_1 \pause = \{id\}$ \pause e $(S_1, \circ)$ é um grupo, \pause e nesse caso comutativo. \pause
    \end{frame}

    \begin{frame}
        Se $A = \{1, 2\}$ \pause então podemos definir as seguintes funções bijetoras em $A$: \pause
        \begin{multicols}{2}
            \begin{enumerate}
                \item[] \begin{center}
                    $id : A \to A$ \pause\\
                    $id(1) = 1$ \pause\\ $id(2) = 2$
                \end{center}\pause
                \item[]  \begin{center}
                    $f : A \to A$ \pause\\ $f(1) = 2 \pause$\\ $f(2) = 1$
                \end{center}\pause
            \end{enumerate}
        \end{multicols}

        Assim $\mathcal{S} \pause = S_2 \pause = \{id, \pause f\}$ \pause e $(S_2, \circ)$ é um grupo.\pause

        Além disso, esse grupo é comutativo.
    \end{frame}

    \begin{frame}
        Agora, seja $A = \{1, 2, 3\}$. \pause Podemos definir então as seguintes funções bijetoras em $A$: \pause
        \begin{multicols}{3}
            \begin{enumerate}
                \item[] \begin{center}
                    $id : A \to A$ \pause\\
                    $id(1) = 1$ \pause\\
                    $id(2) = 2$ \pause\\
                    $id(3) = 3$ \pause
                \end{center}\pause

                \vspace{.3cm}

                \item[] \begin{center}
                    $f_1 : A \to A$\pause\\
                    $f_1(1) = 2$\pause\\
                    $f_1(2) = 1$\pause\\
                    $f_1(3) = 3$
                \end{center}\pause

                \vspace{.3cm}

                \item[] \begin{center}
                    $f_2 : A \to A$\pause\\
                    $f_2(1) = 3$\pause\\
                    $f_2(2) = 2$\pause\\
                    $f_2(3) = 1$
                \end{center}\pause

                \vspace{.3cm}

                \item[] \begin{center}
                    $f_3 : A \to A$\pause\\
                    $f_3(1) = 1$\pause\\
                    $f_3(2) = 3$\pause\\
                    $f_3(3) = 2$
                \end{center}\pause

                \vspace{.3cm}

                \item[] \begin{center}
                    $f_4 : A \to A$\pause\\
                    $f_4(1) = 2$\pause\\
                    $f_4(2) = 3$\pause\\
                    $f_4(3) = 1$
                \end{center}\pause

                \vspace{.3cm}

                \item[] \begin{center}
                    $f_5 : A \to A$\pause\\
                    $f_5(1) = 3$\pause\\
                    $f_5(2) = 1$\pause\\
                    $f_5(3) = 2$
                \end{center}
            \end{enumerate}
        \end{multicols}
    \end{frame}

    \begin{frame}
        Logo $\mathcal{S} = S_3 = \{id, f_1, f_2, f_3, f_4, f_5\}$ \pause e $(S_3, \circ)$ é um grupo.\pause

        \vspace{.3cm}

        Nesse caso temos\pause
        \begin{center}
            $(f_1 \circ f_4)(1) \pause = f_1(f_4(1)) \pause = f_1(2) \pause = 1$\pause\\
            \vspace{.3cm}
            $(f_4 \circ f_1)(1) \pause = f_4(f_1(1)) \pause = f_4(2) \pause = 3$\pause
        \end{center}
        daí $(f_1 \circ f_4)(1) \pause \ne (f_4 \circ f_1)(1)$ \pause, isto é, \pause $f_1 \circ f_4 \pause \ne f_4 \circ f_1$. \pause Portanto o grupo $(S_3, \circ)$ \pause não é comutativo.\pause

        \vspace{.3cm}

        Note que em $S_2$ \pause temos $2 = 2!$ elementos \pause e em $S_3$ \pause temos $6 = 3!$ elementos.\pause
    \end{frame}

    \begin{frame}
        De modo geral, \pause se $A = \{1, 2, 3, \dots, n\}$ \pause então existem exatamente $n!$ \pause funções $f : A \to A$ bijetoras. \pause

        \vspace{.3cm}

        Assim o grupo $(S_n, \circ)$ \pause possui $n!$ elementos.\pause

        \vspace{.3cm}

        Se $n \geqslant 3$, então \pause $S_n$ é um grupo não comutativo.\pause

        \begin{definicao}
            O grupo $S_n$ é chamado de \pause \textbf{grupo simétrico} \pause ou \textbf{grupo de permutações} \pause em $A = \{1, 2, 3, \dots, n\}$.
        \end{definicao}
    \end{frame}

    \begin{frame}
        Um modo de representar os elementos de $S_n$ é o seguinte: \pause vamos representar as funções $f \in S_n$ \pause na forma de uma matriz contendo 2 linhas \pause e $n$ colunas. \pause A primeira linha é o domínio da função \pause e a segunda contém suas imagens. \pause Assim se $f \in S_n$ escreveremos\pause
        \[
            f = \pause \begin{pmatrix}
                1 & 2 & 3 & \dots & n\pause \\
                f(1) \pause & f(2) \pause & f(3) \pause & \dots \pause & f(n)\pause
            \end{pmatrix}.
        \]
    \end{frame}

    \begin{frame}
        No caso de $S_3$ vamos escrever\pause
        \begin{multicols}{3}
            \begin{enumerate}
                \item[] $id = \begin{pmatrix}
                    1 & 2 & 3\pause\\
                    1 \pause & 2 \pause & 3\pause
                \end{pmatrix}$\pause

                \vspace{.5cm}

                \item[] $f_1 = \begin{pmatrix}
                    1 & 2 & 3 \pause\\
                    2 \pause & 1 \pause & 3 \pause
                \end{pmatrix}$\pause

                \vspace{.5cm}

                \item[] $f_2 = \begin{pmatrix}
                    1 & 2 & 3 \pause\\
                    3 \pause & 2 \pause & 1 \pause
                \end{pmatrix}$\pause

                \vspace{.5cm}

                \item[] $f_3 = \begin{pmatrix}
                    1 & 2 & 3 \pause\\
                    1 \pause & 3 \pause & 2 \pause
                \end{pmatrix}$\pause

                \vspace{.5cm}

                \item[] $f_4 = \begin{pmatrix}
                    1 & 2 & 3 \pause\\
                    2 \pause & 3 \pause & 1 \pause
                \end{pmatrix}$\pause

                \vspace{.5cm}

                \item[] $f_5 = \begin{pmatrix}
                    1 & 2 & 3 \pause\\
                    3 \pause & 1 \pause & 2 \pause
                \end{pmatrix}$\pause
            \end{enumerate}
        \end{multicols}
    \end{frame}

    \begin{frame}
        Assim a composição $f_3 \circ f_4$ pode ser determinada da seguinte forma:
        \[
            f_3\circ f_4 = \begin{pmatrix}
                    1 & 2 & 3\\
                    1 & 3 & 2
                \end{pmatrix} \pause \circ \begin{pmatrix}
                    1 & 2 & 3\\
                    2 & 3 & 1
                \end{pmatrix}
        \]
    \end{frame}

    \begin{frame}
        A composição $f_4 \circ f_5$ é:

        \[
            f_4\circ f_5 = \begin{pmatrix}
                    1 & 2 & 3\\
                    2 & 3 & 1
                \end{pmatrix} \pause \circ \begin{pmatrix}
                    1 & 2 & 3\\
                    3 & 1 & 2
                \end{pmatrix}
        \]
    \end{frame}

    \begin{frame}
        \begin{definicao}
            Seja $(G,*)$ um grupo. \pause Se $G$ é um conjunto com uma quantidade finita de elementos, \pause dizemos que $G$ é um \textbf{grupo finito}. \pause Denotamos por $|G|$ \pause o número de elementos de $G$ \pause e que será chamado de \textbf{ordem} de $G$ \pause ou \textbf{cardinalidade} de $G$. \pause Quando o conjunto $G$ não é finito, \pause dizemos que $G$ é um \textbf{grupo infinito}.\pause
        \end{definicao}

        \begin{exemplos}
            \begin{enumerate}[label={\roman*})]
                \item $(\z_m, +)$ é um grupo finito para todo $m>1$ \pause e $|G| = m$.\pause
                \item $(S_n, \circ)$ é um grupo finito \pause e $|G| = n!$ elementos.\pause
                \item $(\z, +)$ é um grupo infinito.
            \end{enumerate}
        \end{exemplos}
    \end{frame}

    \begin{frame}
        Seja $(G, *)$ um grupo. \pause Para simplificar a notação \pause vamos adotar uma notação multiplicativa \pause e escrever $(G, *) = \pause (G, \cdot)$. \pause Assim, dados $x$, $y \in G$ vamos denotar\pause
        \[
            x * y = \pause x \cdot y = \pause xy.\pause
        \]

        Nesse caso vamos dizer simplesmente que $G$ é um grupo.
    \end{frame}

    \begin{frame}
        \begin{proposicao}
            Seja $G$ um grupo. \pause Dado $H \subset G$ um subgrupo \pause defina \pause
            \[
                x \sim y \pause \mbox{ se, e somente se, } \pause x^{-1}y \in H \pause
            \]
            para todos $x$, $y \in G$. \pause
            \begin{enumerate}[label={\roman*})]
                \item A relação $\sim$ \pause sobre $G$ definida acima é uma relação de equivalência. \pause

                \item Se $a \in G$, \pause então a classe de equivalência determinada por $a$ \pause é o conjunto \pause
                \[
                    aH = \pause \{ah \pause \mid h \in H\}.
                \]
            \end{enumerate}
        \end{proposicao}
    \end{frame}

    \begin{frame}
        \begin{proposicao}
            Seja $H$ um subgrupo de um grupo $G$. \pause Então duas classes laterais quaisquer \pause módulo $H$ \pause são subconjuntos de $G$ que possuem a mesma cardinalidade, \pause isto é, a mesma quantidade de elementos. \pause
        \end{proposicao}
    \end{frame}

    \begin{frame}
        \begin{exemplos}
            \begin{enumerate}[label={\roman*})]
                \item No grupo multiplicativo $G = \{1, -1, i, -i\}$, \pause onde $i^2 = -1$. \pause Considere o conjunto $H = \{1, -1\}$. \pause Então $H$ é um subgrupo de $G$ \pause e as classes laterais serão:

                \seti
            \end{enumerate}
        \end{exemplos}
    \end{frame}

    \begin{frame}
        \begin{exemplos}
            \begin{enumerate}[label={\roman*})]
                \conti
                \item Considere o grupo multiplicativo $\real^*$ \pause e $H = \{ x \in \real^* \mid x > 0\} \pause \subset \real^*$. \pause Então $H$ é subgrupo de $\real^*$ \pause e as classes laterais serão:
                \seti
            \end{enumerate}
        \end{exemplos}
    \end{frame}

    \begin{frame}
        \begin{exemplos}
            \begin{enumerate}[label={\roman*})]
                \conti
                \item Considere agora o grupo simétrico $G = S_3$. \pause Denote por \pause
                \[
                    a = \begin{pmatrix}
                        1 & 2 & 3\\2 & 3 & 1
                    \end{pmatrix}, \pause \quad
                    b = \begin{pmatrix}
                        1 & 2 & 3\\1 & 3 & 2
                    \end{pmatrix}. \pause
                \]
                Fica como exercício verificar que $\{e, a, a^2 , b, ba, ba^2\} = \pause S_3$. \pause Aqui $e$ é a função identidade, \pause $a^2 = a \circ a$, \pause $ba = b \circ a$ e \pause $ba^2 = b\circ(a\circ a)$. \pause Seja $H = \{e, a , a^2\}$. \pause Então $H$ é subgrupo de $S_3$ \pause e as classes laterais serão:
            \end{enumerate}
        \end{exemplos}
    \end{frame}

    \begin{frame}
        Seja $(G, *)$ um grupo. \pause

        Caso a operação $*$ seja do tipo multiplicativa, vamos escrever $(G, *) = \pause (G, \cdot)$. \pause Assim, dados $x$, $y \in G$ vamos denotar\pause
        \[
            x * y = \pause x \cdot y = \pause xy.
        \]

        Caso a operação $*$ seja do tipo aditiva, vamos escrever $(G, *) = \pause (G, +)$. \pause Assim, dados $x$, $y \in G$ vamos denotar\pause
        \[
            x * y = \pause x + y.
        \]

        Com a notação multiplicativa \pause o inverso de um elemento $x \in G$ \pause será denotado por $x^{-1}$ \pause e no caso da notação aditiva \pause o oposto de $x \in G$ \pause será denotado por $-x$.
    \end{frame}

    \begin{frame}
        Seja $G$ um grupo multiplicativo \pause e denote por $e$ o elemento neutro de $G$. \pause Se $x \in G$ \pause e $m \in \z$, \pause a \textbf{potência $m$-ésima} de $x$, \pause ou \textbf{potência de $x$ de expoente $m$}, \pause é o elemento de $G$ denotado por \pause
        \[
            x^m \pause
        \]
        e definido por: \pause
        \[
            x^m = \pause \begin{cases}
                    e, & \mbox{se m = 0}, \pause\\
                    x^{m-1}x, \pause & \mbox{ se } m \ge 1, \pause\\
                    (x^{-m})^{-1}, \pause & \mbox{ se } m < 0.
                   \end{cases}
        \]

    \end{frame}

    \begin{frame}
        \begin{exemplos}
            \begin{enumerate}[label={\roman*})]
                \item No grupo multiplicativo $GL_2(\real)$ \pause seja \pause
                \[
                    A = \begin{pmatrix}
                        1 & 1\\2 & 3
                    \end{pmatrix}. \pause
                \]
                Então:
                \seti
            \end{enumerate}
        \end{exemplos}
    \end{frame}

    \begin{frame}
        \begin{exemplos}
            \begin{enumerate}[label={\roman*})]
                \conti
                \item No grupo multiplicativo $\z_5^*$ \pause seja $a = \overline{2}$. \pause Então:
                \seti
            \end{enumerate}
        \end{exemplos}
    \end{frame}

    \begin{frame}
        \begin{exemplos}
            \begin{enumerate}[label={\roman*})]
                \conti
                \item No grupo multiplicativo $S_3$ \pause seja
                \[
                    a = \begin{pmatrix}
                        1 & 2 & 3\\ 2 & 3 & 1
                    \end{pmatrix}.\pause
                \]
                Então:
            \end{enumerate}
        \end{exemplos}
    \end{frame}

    \begin{frame}
        \begin{proposicao}
            Seja $G$ um grupo multiplicativo. \pause Se $m$ e $n$ são números inteiros \pause e $x \in G$, \pause então \pause
            \begin{enumerate}[label={\roman*})]
                \item $x^mx^n = \pause x^{m + n}$ \pause

                \item $x^{-m} = \pause (x^m)^{-1}$ \pause

                \item $(x^m)^n = \pause x^{mn}$ \pause

                \item $x^mx^n = \pause x^nx^m$
            \end{enumerate}
        \end{proposicao}
    \end{frame}

    \begin{frame}
        Seja $G$ um grupo aditivo \pause e denote por $e$ o elemento neutro de $G$. \pause Se $x \in G$ \pause e $m \in \z$, \pause o \textbf{múltiplo $m$-ésimo} de $x$ \pause é o elemento de $G$ denotado por \pause
        \[
            m \cdot x \pause
        \]
        e definido por:
        \[
            m \cdot x = \pause \begin{cases}
                    e, & \mbox{se m = 0}, \pause\\
                    (m - 1)\cdot x + x, \pause & \mbox{ se } m \ge 1, \pause\\
                    -[(-m) \cdot x], \pause & \mbox{ se } m < 0.
                   \end{cases}
        \]
    \end{frame}

    \begin{frame}
        \begin{proposicao}
            Seja $G$ um grupo aditivo. \pause Se $m$ e $n$ são números inteiros \pause e $x \in G$, então \pause
            \begin{enumerate}[label={\roman*})]
                \item $m \cdot x + n \cdot x = \pause (m + n) \cdot x$ \pause

                \item $(-m) \cdot x = \pause -(m \cdot x)$ \pause

                \item $n\cdot (m \cdot x) = \pause (nm)\cdot x$
            \end{enumerate}
        \end{proposicao}
    \end{frame}

    \begin{frame}
        Seja $G$ um grupo multiplicativo \pause e $x \in G$. \pause Denote por $[x]$ \pause o seguinte conjunto \pause
        \[
            [x] = \pause \{x^m \pause \mid m \in \z\} \pause \subset G.\pause
        \]

        \begin{proposicao}
            Seja $G$ um grupo multiplicativo \pause e $x \in G$. \pause
            \begin{enumerate}[label={\roman*})]
                \item O subconjunto $[x]$ \pause é um subgrupo de $G$.\pause

                \item Se $H$ é um subgrupo de $G$ \pause tal que $x \in H$, \pause então $[x] \subset H$.
            \end{enumerate}
        \end{proposicao}
    \end{frame}

    \begin{frame}
        \begin{definicao}
            Um grupo multiplicativo $G$ \pause será chamado de \textbf{grupo cíclico} \pause se, para algum $x \in G$, \pause vale \pause
            \[
                G = [x]. \pause
            \]
            Nessas condições, o elemento $x$  \pause é chamado de \textbf{gerador} do grupo $G$.
        \end{definicao}
    \end{frame}

    \begin{frame}
        \begin{exemplos}
            \begin{enumerate}[label={\roman*})]
                \item No grupo multiplicativo $\complex^*$, \pause encontre o subgrupo gerado por $i$.
                \seti
            \end{enumerate}
        \end{exemplos}
    \end{frame}

    \begin{frame}
        \begin{exemplos}
            \begin{enumerate}[label={\roman*})]
                \conti
                \item No grupo $S_3$, \pause encontre o subgrupo gerado por
                \[
                    f = \begin{pmatrix}
                        1 & 2 & 3\\
                        2 & 3 & 1
                    \end{pmatrix}.
                \]
            \end{enumerate}
        \end{exemplos}
    \end{frame}

    \begin{frame}
        \begin{proposicao}
            Todo subgrupo de um grupo cíclico é também cíclico.
        \end{proposicao}
    \end{frame}

    \begin{frame}
        \begin{definicao}
            Seja $G$ um grupo com elemento neutro $e$. \pause Dado $x \in G$ \pause se existir um inteiro $h > 0$ \pause tal que \pause
            \begin{enumerate}[label={\roman*})]
                \item $x^h = e$ \pause
                \item $x^r \ne e$ \pause qualquer que seja o inteiro $r$ \pause tal que $0 < r < h$\pause
            \end{enumerate}
            diremos que a \textbf{ordem} \pause ou \textbf{período} \pause de $x$ é $h$. \pause Nesse caso escreveremos $|x| = \pause o(x) = h$. \pause

            Se para qualquer inteiro \pause $r \ne 0$, \pause $x^r \ne e$, \pause diremos que a \textbf{ordem} de $x$ é \textbf{zero}.
        \end{definicao}
    \end{frame}

    \begin{frame}
        \begin{exemplos}
            \begin{enumerate}[label={\roman*})]
                \item No grupo multiplicativo $\complex^*$ temos:
                \seti
            \end{enumerate}
        \end{exemplos}
    \end{frame}

    \begin{frame}
        \begin{exemplos}
            \begin{enumerate}[label={\roman*})]
                \conti
                \item Em $S_3$ temos:
                \seti
            \end{enumerate}
        \end{exemplos}
    \end{frame}

    \begin{frame}
        \begin{exemplos}
            \begin{enumerate}[label={\roman*})]
                \conti

                \item Em $\z_5$ temos:
                \seti
            \end{enumerate}
        \end{exemplos}
    \end{frame}

    \begin{frame}
        \begin{exemplos}
            \begin{enumerate}[label={\roman*})]
                \conti

                \item Em $\z$ \pause o único elemento de ordem diferente de zero \pause é o elemento neutro.
            \end{enumerate}
        \end{exemplos}
    \end{frame}

    \begin{frame}
        \begin{proposicao}
            Seja $x$ um elemento de ordem $h > 0$ \pause de um grupo $G$. \pause Então $x^m = e$ \pause se, e somente se, $h \mid m$.
        \end{proposicao}
    \end{frame}

    \begin{frame}
        Considere o grupo multiplicativo $G = \{1, -1\}$ \pause e o grupo $S_2$ das permutações sobre o conjunto $\{1,2\}$. \pause Aqui
        \[
            S_2 = \left\{id = \begin{pmatrix}
                1 & 2\\1 & 2
            \end{pmatrix}; \pause f = \begin{pmatrix}
                1 & 2\\2 & 1
            \end{pmatrix}\right\}.
        \]
    \end{frame}

    \begin{frame}
        Temos \pause
        \begin{table}
            \begin{minipage}{.5\linewidth}
                \caption*{$G$}
                \centering
                \begin{tabular}{|c|c|c|}
                    \hline
                    $\cdot$ & 1 & -1\\
                    \hline
                    1 & 1 & -1\\
                    \hline
                    -1 & -1 & 1\\
                    \hline
                \end{tabular}
            \end{minipage}%
            \pause
            \begin{minipage}{.5\linewidth}
                \caption*{$S_2$}
                \centering
                \begin{tabular}{|c|c|c|}
                    \hline
                    $\circ$ & $id$ & $f$\\
                    \hline
                    $id$ & $id$ & $f$\\
                    \hline
                    $f$ & $f$ & $id$\\
                    \hline
                \end{tabular}
            \end{minipage}\pause
        \end{table}

        \vspace{.4cm}

        Defina $\sigma : G \to S_2$ por \pause
        \begin{center}
            \begin{tabular}{c}
                $\sigma(1) = id$ \pause\\
                \vspace{.3cm}\\
                $\sigma(-1) = f$.
            \end{tabular}
        \end{center}
    \end{frame}

    \begin{frame}
        Da definição de $\sigma$ \pause é fácil ver que essa função é bijetora. \pause Além disso, \pause
        \begin{center}
            \begin{tabular}{c}
                $\sigma(1) \circ \sigma(1) \pause = id \pause \circ id \pause = id \pause = \sigma(1) \pause = \sigma(1 \cdot 1)$\pause\\
                \\
                $\sigma(1) \circ \sigma(-1) \pause = id \pause \circ f \pause = f \pause = \sigma(-1) \pause = \sigma(1 \cdot -1)$ \pause\\
                \\
                $\sigma(-1) \circ \sigma(1) \pause = f \pause \circ id \pause = f \pause = \sigma(-1) \pause = \sigma(-1 \cdot 1)$ \pause\\
                \\
                $\sigma(-1) \circ \sigma(-1) \pause = f \pause \circ f \pause = id \pause = \sigma(1) \pause = \sigma(-1 \cdot -1)$ \pause\\
            \end{tabular}
        \end{center}
        ou seja, $\sigma(x\cdot y) = \pause \sigma(x) \circ \sigma(y)$ \pause para todos $x$, $y \in G$. \pause Assim função $\sigma$ é um homomorfismo de $G$ em $S_2$. \pause

        \vspace{.3cm}

        Como $\sigma$ também é bijetora, \pause então $\sigma$ é um isomorfismo \pause de $G$ em $S_2$. \pause Nesse caso, dizemos que $G$ e $S_2$ são grupos isomorfos \pause e denotamos isso escrevendo $G \cong S_2$.
    \end{frame}

    \begin{frame}
        \begin{definicao}
            Sejam $(G, *)$ e $(H, \triangle)$ grupos. \pause Se existe $f : G \to H$ um isomorfismo, \pause diremos que $G$ e $H$ são \textbf{grupos isomorfos} \pause e denotaremos esse fato escrevendo $G \cong H$.
        \end{definicao}
    \end{frame}

    \begin{frame}
        \begin{proposicao}
            Sejam $G$ e $H$ grupos multiplicativos. \pause Se $f : G \to H$ é um isomorfismos de grupos, então \pause $G$ é comutativo se, e somente se, $H$ é comutativo.
        \end{proposicao}
    \end{frame}

    \begin{frame}
        \begin{exemplos}
            \vspace{.3cm}
            \begin{enumerate}
                \item[1)] Os grupos $\z_6$ e $S_3$ \pause não são isomorfos, \pause pois $\z_6$ é comutativo \pause e $S_3$ não é comutativo.\pause

                \vspace{.3cm}

                \item[2)] Considere o grupo $S_6$ das permutações em $\{1, 2, \cdots, 6\}$. \pause Tome
                \[
                    f = \begin{pmatrix}
                        1 & 2 & 3 & 4 & 5 & 6\\
                        2 & 3 & 4 & 5 & 6 & 1
                    \end{pmatrix} \in S_6. \pause
                \]
                Seja $H = [f]$. \pause Então $H \cong \z_6$, \pause onde $\phi : H \to \z_6$ dada por $\phi(f^k) = \overline{k}$ \pause é um isomorfismo de grupos.

                \vspace{.3cm}

            \end{enumerate}
        \end{exemplos}
    \end{frame}

    \begin{frame}
        \begin{proposicao}
            Sejam $G$ e $H$ grupos multiplicativos. \pause Seja $f : G \to H$ é um isomorfismos de grupos. \pause Então $x \in G$ \pause é tal que $o(x) = h$ \pause se, e somente se, $o(f(x)) = h$.
        \end{proposicao}
    \end{frame}

    \begin{frame}
        Seja $G = [a]$ um grupo cíclico. \pause Dois casos podem ocorrer: \pause

        \textbf{Caso 1:} $a^r \ne a^s$ \pause sempre que $r \ne s$.
    \end{frame}

    \begin{frame}
        \begin{proposicao}
            Se $G = [a]$ é um grupo cíclico que cumpre a condição do \textbf{Caso 1}, \pause então a função $f : \z \to G$ por $f(r) = a^r$ \pause é um isomorfismo de grupos. \pause Ou seja, $G \cong \z$.
        \end{proposicao}
    \end{frame}

    \begin{frame}
        \textbf{Caso 2:} $a^r = a^s$ \pause para algum par de inteiros distintos, $r$ e $s$.
    \end{frame}

    \begin{frame}
        \begin{proposicao}
            Seja $G = [a]$ um grupo cíclico que cumpre a condição do \textbf{Caso 2}. \pause Então existe um inteiro $m > 0$ tal que \pause
            \begin{enumerate}
                \item[i)] $a^m = e$ \pause

                \item[ii)] $a^r \ne e$, sempre que $0 < r < m$. \pause
            \end{enumerate}
            Nesse caso, a ordem do grupo $G$ é $m$ \pause e
            \[
                G = [a] = \{e, a, a^2, \cdots, a^{m - 1}\}.
            \]
        \end{proposicao}
    \end{frame}

    \begin{frame}
        \begin{corolario}
            Seja $G = [a]$ um grupo cíclico de ordem finita igual a $m$. \pause Então a função $f : \z_m \to G$ \pause dada por $f(\overline{x}) = a^x$ \pause é um isomorfismo de grupos.
        \end{corolario}
    \end{frame}

    \begin{frame}
        Sejam $(G, \cdot)$ um grupo \pause e $A$ e $B$ subconjuntos de $G$. \pause Vamos indicar por \pause
        \[
            AB \pause
        \]
        e chamaremos de \textbf{produto} de $A$ por $B$ \pause o seguinte subconjunto de $G$: \pause
        \begin{center}
            \begin{tabular}{l}
                $AB = \emptyset$, \pause se $A = \emptyset$ ou $B = \emptyset$ \pause\\
                \\
                $AB = \{xy \pause \mid x \in A \pause \mbox{ e } y \in B \pause\}$, se  $A \ne \emptyset$ e $B \ne \emptyset$. \pause
            \end{tabular}
        \end{center}

        \vspace{.3cm}

        Assim o \textbf{produto} de $A$ por $B$ \pause é uma operação sobre o subconjunto das partes de $G$, \pause $\mathcal{P}(G)$, chamada de \textbf{multiplicação de subconjuntos} de $G$. \pause

        \vspace{.3cm}

        Como $G$ é associativo, \pause então a \textbf{multiplicação de subconjuntos} também será associativa. \pause Além disso, caso o grupo $G$ seja comutativo, \pause então \textbf{multiplicação de subconjuntos} também será comutativa.
    \end{frame}

    \begin{frame}
        \begin{exemplos}
            \begin{enumerate}[label=({\arabic*})]
                \item Seja $G = \{e, a, b, c\}$ \pause o grupo tal que \pause
                \begin{table}[h]
                    \begin{tabular}{|c|c|c|c|c|}
                        \hline
                        $\cdot$ & e & a & b & c\\
                        \hline
                        e & e & a & b & c\\
                        \hline
                        a & a & e & c & b\\
                        \hline
                        b & b & c & e & a\\
                        \hline
                        c & c & b & a & e\\
                        \hline
                    \end{tabular}.
                \end{table}
                Esse grupo é chamada de \textbf{grupo de Klein}. \pause

                \vspace{.3cm}

                Se $A = \{e, a\}$ \pause e $B = \{b, c\}$, \pause então:
                \seti
            \end{enumerate}
        \end{exemplos}
    \end{frame}

    \begin{frame}
        \begin{exemplos}
            \begin{enumerate}[label=({\arabic*})]
                \conti
                \item Considere o grupo multiplicativo dos números reais. \pause Se
                \begin{center}
                    $A = \{x \in \real^* \mid x > 0\}$ \pause\\
                    $B = \{x \in \real^* \mid x < 0\}$ \pause
                \end{center}
                então:
            \end{enumerate}
        \end{exemplos}
    \end{frame}

    \begin{frame}
        \begin{definicao}
            Um subgrupo $N$ \pause de um grupo $G$ \pause é chamado de \textbf{subgrupo normal} \pause (ou \textbf{invariante}) \pause se, para todo $x \in G$, \pause vale
            \[
                xN \pause = Nx. \pause
            \]
            Denotaremos esse fato escrevendo $H \unlhd G$.
        \end{definicao}
    \end{frame}

    \begin{frame}
        \begin{exemplos}
            \begin{enumerate}[label=({\arabic*})]
                \item Seja $G = S_3$. \pause Já vimos que se tomamos
                \[
                    Id = \begin{pmatrix}
                        1 & 2 & 3\\
                        1 & 2 & 3
                    \end{pmatrix}, \pause \quad
                    f = \begin{pmatrix}
                        1 & 2 & 3\\
                        2 & 3 & 1
                    \end{pmatrix} \pause \quad \mbox{e}\quad
                    g = \begin{pmatrix}
                        1 & 2 & 3\\
                        1 & 3 & 2
                    \end{pmatrix} \pause
                \]
                então
                \[
                    S_3 = \{Id, f, f^2, g, gf, gf^2\}. \pause
                \]
                Considere o subgrupo $H = [\ f\ ] \pause = \{Id, f, f^2\}$. \pause Então $H$ é um subgrupo normal de $G$.

                \seti
            \end{enumerate}
        \end{exemplos}
    \end{frame}

    \begin{frame}
        \begin{exemplos}
            \begin{enumerate}[label=({\arabic*})]
                \conti

                \item Se $G$ é um grupo abeliano, \pause então todo subgrupo de $G$ é normal.

                \seti
            \end{enumerate}
        \end{exemplos}
    \end{frame}

    \begin{frame}
        \begin{exemplos}
            \begin{enumerate}[label=({\arabic*})]
                \conti

                \item Seja $H$ um subgrupo de $G$ \pause tal que $H$ possui somente duas classes laterais. \pause Então $H$ é um subgrupo normal de $G$.

                \seti
            \end{enumerate}
        \end{exemplos}
    \end{frame}

    \begin{frame}
        \begin{proposicao}
            Seja $G$ um grupo. \pause Se $H$ e $L$ são subgrupos normais de $G$, \pause então $H \cap L$ \pause é um subgrupo normal de $G$.
        \end{proposicao}
    \end{frame}

    \begin{frame}
        \begin{proposicao}
            Seja $N$ um subgrupo normal \pause do grupo $G$. \pause Então, para quaisquer $a$, $b \in G$ temos \pause
            \[
                (aN)(bN) \pause = (ab)N.
            \]
        \end{proposicao}
    \end{frame}

    \begin{frame}
        Seja $N$ um subgrupo normal \pause de um grupo $G$, onde $e$ denota o elemento neutro de $G$. \pause Denote por \pause
        \[
            G/N = \{aN \mid a \in G\} \pause
        \]
        o conjunto das classes de equivalência determinadas por $N$. \pause

        \vspace{.5cm}

        Defina em $G/N$ a operação \pause
        \[
            (aN)(bN)  \pause = (ab)N \pause
        \]
        para todos $aN$, $bN \in G/N$.
    \end{frame}

    \begin{frame}
        Temos: \pause

        \vspace{.3cm}

        \begin{enumerate}[label={\roman*})]
            \item $[(aN)(bN)](cN) \pause = (an)[(bN)(cN)]$ \pause para todos $aN$, $bN$, $cN \in G/N$; \pause

            \vspace{.3cm}

            \item $(aN)(eN) \pause = (ae)N \pause = aN \pause = (ea)N \pause = (eN)(aN)$ \pause para todo $aN \in G/N$; \pause

            \vspace{.3cm}

            \item $(aN)(a^{-1}N) \pause = (aa^{-1})N \pause = eN \pause = (a^{-1}a)N \pause = (a^{-1}N)(aN)$ \pause para todo $aN \in G/N$. \pause

            \vspace{.3cm}
        \end{enumerate}

        Assim, o conjunto $G/N$ é um grupo com a multiplicação de conjuntos. \pause

        \vspace{.3cm}

        Nesse grupo o elemento neutro é $eN$ \pause e $(aN)^{-1} = (a^{-1})N$.
    \end{frame}

    \begin{frame}
        \begin{definicao}
            Sejam $G$ um grupo e $N$ um subgrupo normal de $G$. \pause Nessas condições, \pause o \textbf{grupo quociente} \pause
            de $G$ por $N$ \pause é o par formado pelo conjunto quociente $G/N$ \pause e da operação de multiplicação de conjuntos
            aplicadas aos elementos desse conjunto.
        \end{definicao}
    \end{frame}

    \begin{frame}
        \begin{exemplos}
            \begin{enumerate}[label=({\arabic*})]
                \item Seja $G = \{1, -1, i, -i\}$ um grupo \pause e $N = \{1, -1\}$.

                \seti
            \end{enumerate}
        \end{exemplos}
    \end{frame}

    \begin{frame}
        \begin{exemplos}
            \begin{enumerate}[label=({\arabic*})]
                \conti

                \item Seja $G = \z_6 = \{\overline{0}, \overline{1}, \overline{2}, \overline{3}, \overline{4}, \overline{5}\}$ \pause e $H = \{\overline{0}, \overline{3}\}$.

                \seti
            \end{enumerate}
        \end{exemplos}
    \end{frame}

    \begin{frame}
        \begin{exemplos}
            \begin{enumerate}[label=({\arabic*})]
                \conti

                \item Seja $G = S_3$. \pause Já vimos que se tomamos
                \[
                    Id = \begin{pmatrix}
                        1 & 2 & 3\\
                        1 & 2 & 3
                    \end{pmatrix}, \pause\quad
                    f = \begin{pmatrix}
                        1 & 2 & 3\\
                        2 & 3 & 1
                    \end{pmatrix} \pause \quad \mbox{e}\quad
                    g = \begin{pmatrix}
                        1 & 2 & 3\\
                        1 & 3 & 2
                    \end{pmatrix}
                \]
                então
                \[
                    S_3 = \{Id, f, f^2, g, gf, gf^2\}. \pause
                \]
                Considere o subgrupo $H = [\ f\ ] \pause = \{Id, f, f^2\}$.

                \seti
            \end{enumerate}
        \end{exemplos}
    \end{frame}

    \begin{frame}
        \begin{proposicao}
            Se $N$ é um subgrupo normal de $G$, \pause então a função $\mu : G \to G/N$ \pause definida por $\mu(a) = aN$ \pause é um homomorfismo sobrejetor \pause de grupos tal que \pause
            \[
                \ker(\mu) = N.
            \]
        \end{proposicao}
    \end{frame}

    \begin{frame}
        \begin{definicao}
            Se $N$ é um subgrupo normal de $G$, \pause então o homomorfismo $\mu : G \to G/N$ \pause definido por $\mu(a) = aN$ \pause é chamado de \textbf{homomorfismo can\^onico} \pause de $G$ sobre $G/N$.
        \end{definicao}
    \end{frame}

    \begin{frame}
        \begin{lema}
            Se $f : G \to L$ é um homomorfismo de grupos, \pause então $N = \ker(f)$ \pause é um subgrupo normal de $G$ \pause e, portanto, $G/N$ é um grupo.
        \end{lema}
    \end{frame}

    \begin{frame}
        \begin{teorema}[Teorema do Homomorfismo para Grupos]
            Seja $f : G \to L$ um homomorfismo sobrejetor \pause de grupos. \pause Se $N = \ker(f)$, \pause então o grupo quociente $G/N$ é isomorfo ao grupo $L$.
        \end{teorema}
    \end{frame}

    \begin{frame}
        \begin{exemplo}
            Dado um inteiro $m > 1$, \pause considere o homomorfismo $\rho_m : \z \to \z_m$ \pause definido por $\rho_m(x) = \overline{x}$.
        \end{exemplo}
    \end{frame}

    \begin{frame}
        Seja $G$ um grupo finito. \pause Se $H$ é um subgrupo de $G$, \pause então existirá uma quantidade finita de classes laterais m\'odulo $H$ \pause.
        Assim o conjunto
        \[
            G/H = \{aH \mid a \in G\} \pause
        \]
        é finito. \pause

        O número de elementos de $G/H$ \pause é chamado de \textbf{índice} \pause de $H$ em $G$ \pause e será denotado por
        \[
            [G : H] = |G/H|.
        \]
    \end{frame}

    \begin{frame}
        \begin{exemplos}
            \begin{enumerate}[label=({\arabic*})]
                \item Seja $G = \{1, -1, i, -i\}$ um grupo \pause e $N = \{1, -1\}$ \pause um subgrupo de $G$. \pause Já vimos que as classes laterais de $N$ em $G$ são \pause
                \[
                    N \quad \mbox{e}\quad iN. \pause
                \]
                Daí
                \[
                    G/N = \{N, iN\} \pause
                \]
                e assim $[G : N] = 2$.

                \seti
            \end{enumerate}
        \end{exemplos}
    \end{frame}

    \begin{frame}
        \begin{exemplos}
            \begin{enumerate}[label=({\arabic*})]
                \conti

                \item Seja $G = S_3$. \pause Já vimos que se tomamos
                \[
                    Id = \begin{pmatrix}
                        1 & 2 & 3\\
                        1 & 2 & 3
                    \end{pmatrix}, \pause\quad
                    f = \begin{pmatrix}
                        1 & 2 & 3\\
                        2 & 3 & 1
                    \end{pmatrix} \pause \quad \mbox{e}\quad
                    g = \begin{pmatrix}
                        1 & 2 & 3\\
                        1 & 3 & 2
                    \end{pmatrix}
                \]
                então
                \[
                    S_3 = \{Id, f, f^2, g, gf, gf^2\}. \pause
                \]
                Considere o subgrupo $H = [\ g\ ] \pause = \{Id, g\}$. Então $H$ possui 3 classes laterais que são
                \[
                    H,\ fH,\ f^2H. \pause
                \]
                Daí
                \[
                    G/H = \{H, fH, f^2H\} \pause
                \]
                e então $[G : H] = 3$.
            \end{enumerate}
        \end{exemplos}
    \end{frame}

    \begin{frame}
        \begin{teorema}[Teorema de Lagrange]
            Seja $H$ um subgrupo \pause de um grupo finito $G$. \pause Então $o(G) = o(H)[G:H]$ \pause e, portanto, $o(H) | o(G)$.
        \end{teorema}
    \end{frame}

    \begin{frame}
        \begin{observacao}
            No grupo $S_4$ \pause considere o seguinte subconjunto: \pause
            \[
                L = \left\{\begin{pmatrix}
                    1 & 2 & 3 & 4\\
                    1 & 2 & 3 & 4
                \end{pmatrix}, \pause \begin{pmatrix}
                    1 & 2 & 3 & 4\\
                    1 & 3 & 4 & 2
                \end{pmatrix}\right\}. \pause
            \]
            Observe que o número de elementos de L \pause divide $|S_4| = 4! = 24$, \pause mas $L$ não é um subgrupo de $S_4$ \pause pois
            \[
                \begin{pmatrix}
                    1 & 2 & 3 & 4\\
                    1 & 3 & 4 & 2
                \end{pmatrix}^{-1} \pause = \begin{pmatrix}
                    1 & 2 & 3 & 4\\
                    1 & 4 & 2 & 3
                \end{pmatrix} \pause \notin L.
            \]
        \end{observacao}
    \end{frame}

    \begin{frame}
        \begin{corolario}
            Seja $G$ um grupo finito. \pause Então a ordem de um elemento $x \in G$ \pause divide a ordem de $G$ \pause e o quociente é $[G : H]$, \pause onde $H = [x]$.
        \end{corolario}
    \end{frame}

    \begin{frame}
        \begin{corolario}
            Sejam $G$ um grupo finito \pause e $x \in G$. \pause Então
            \[
                x^{o(G)} \pause = e, \pause
            \]
            onde $e$ denota o elemento neutro de $G$.
        \end{corolario}
    \end{frame}

    \begin{frame}
        \begin{corolario}
            Seja $G$ um grupo finito \pause cuja ordem é um número primo. \pause Então $G$ é um grupo cíclico \pause e os únicos subgrupos de $G$ \pause são os triviais, \pause ou seja, $\{e\}$ e $G$.
        \end{corolario}
    \end{frame}

    \begin{frame}
        \begin{exemplo}
            Determine todos os subgrupos do grupo $S_3$.
        \end{exemplo}
    \end{frame}

    \begin{frame}
        \begin{proposicao}
            Se $G$ é um grupo finito tal que $o(G) \le 5$, então $G$ é abeliano.
        \end{proposicao}
    \end{frame}

    \begin{frame}
        \begin{proposicao}
            Seja $G$ um grupo tal que $|G| = pq$, onde $p$ e $q$ são números primos. Se $G$ é abeliano e $p \ne q$, então $G$ é um grupo cíclico.
        \end{proposicao}
    \end{frame}

\end{document}
