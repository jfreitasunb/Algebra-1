%!TEX program = xelatex
%!TEX encoding = UTF-8
\def\ano{2024}
\def\semestre{1}
\def\disciplina{Álgebra 1}
\def\turma{C}
\def\autor{José Antônio O. Freitas}
\def\instituto{MAT-UnB}

\documentclass{beamer}
\usetheme{Madrid}
\usecolortheme{beaver}
% \mode<presentation>
\usepackage{caption}
\usepackage{amssymb}
\usepackage{amsmath,amsfonts,amsthm,amstext}
\usepackage[brazil]{babel}
% \usepackage[latin1]{inputenc}
\usepackage{graphicx}
\graphicspath{{/ArquivosLinux/OneDrive/imagens-latex/}{D:/OneDrive - unb.br/imagens-latex/}}
\usepackage{enumitem}
\usepackage{multicol}
\usepackage{answers}
\usepackage{tikz,ifthen}
\usetikzlibrary{lindenmayersystems}
\usetikzlibrary[shadings]
\newtheorem{definicao}{Definição}[section]
\newtheorem{definicoes}{Definições}[section]
\newtheorem{exemplo}{Exemplo}[section]
\newtheorem{exemplos}{Exemplos}[section]
\newtheorem{exercicio}{Exercício}
\newtheorem{observacao}{Observação:}[section]
\newtheorem{observacoes}{Observações:}[section]
\newtheorem*{solucao}{Solução:}
\newtheorem{proposicao}{Proposição}
\newtheorem{lema}{Lema}
\newtheorem{teorema}{Teorema}
\newtheorem{corolario}{Corolário}
\newenvironment{prova}[1][Prova]{\noindent\textbf{#1:} }{\qedsymbol}%{\ \rule{0.5em}{0.5em}}
\def\ano{2024}
\def\semestre{1}
\def\disciplina{Álgebra 1}
\def\nomeabreviado{Álgebra 1}
\def\turma{1}

\newcommand{\im}{{\rm Im\,}}
\newcommand{\dlim}[2]{\displaystyle\lim_{#1\rightarrow #2}}
\newcommand{\minf}{+\infty}
\newcommand{\ninf}{-\infty}
\newcommand{\cp}[1]{\mathbb{#1}}
\newcommand{\sub}{\subseteq}
\newcommand{\n}{\mathbb{N}}
\newcommand{\z}{\mathbb{Z}}
\newcommand{\rac}{\mathbb{Q}}
\newcommand{\real}{\mathbb{R}}
\newcommand{\complex}{\mathbb{C}}

\newcommand{\vesp}[1]{\vspace{ #1  cm}}

\newcommand{\compcent}[1]{\vcenter{\hbox{$#1\circ$}}}
\newcommand{\comp}{\mathbin{\mathchoice
        {\compcent\scriptstyle}{\compcent\scriptstyle}
        {\compcent\scriptscriptstyle}{\compcent\scriptscriptstyle}}}
\renewcommand{\sin}{{\rm sen\,}}
\renewcommand{\tan}{{\rm tg\,}}
\renewcommand{\csc}{{\rm cossec\,}}
\renewcommand{\cot}{{\rm cotg\,}}
\renewcommand{\sinh}{{\rm senh\,}}

\title{Grupo Simétrico}
\author[\autor]{\autor}
\institute[\instituto]{\instituto}
\date{}

\begin{document}
    \begin{frame}
        \maketitle
    \end{frame}

    \logo{\includegraphics[scale=.1]{logo-MAT.png}\vspace*{8.5cm}}

    \begin{frame}
        Seja $A$ um conjunto não vazio.\pause

        \vspace{.3cm}

        Dada uma função $f : A \to A$, sabemos que $f$ possui inversa \pause se, e somente se, $f$ é bijetora.\pause

        \vspace{.3cm}

        Assim considere o conjunto\pause
        \[
            \mathcal{S} = \{ f : A \to A \pause \mid f \mbox{ é bijetora}\}.\pause
        \]

        Em $\mathcal{S}$ vamos considerar a composição de funções $\circ$.\pause

        \vspace{.3cm}

        Como $id : A \to A$ tal que $id(x) = x$ para todo $x \in A$ \pause é uma função bijetora \pause então $id \in \mathcal{S}$ \pause e com isso $\mathcal{S} \ne \emptyset$.
    \end{frame}

    \begin{frame}
        Dadas $f$, $g \in \mathcal{S}$ \pause como $f$ e $g$ são bijetoras, \pause então $f \circ g$ é bijetora \pause e daí $f \circ g \in \mathcal{S}$. \pause Isto é, \pause a composição de funções \pause é uma operação binária em $\mathcal{S}$.

        \vspace{.3cm}

        Agora, sejam $f$, $g$ e $h \in \mathcal{S}$. \pause Para todo $x \in A$ temos\pause
        \begin{center}
            $[(f\circ g) \pause \circ h \pause](x) \pause = (f \circ g) \pause(h(x)) \pause = f(g(h(x))) \pause$\\ \vspace{.5cm}
            $[f\circ \pause(g\circ h) \pause](x) \pause = f( \pause(g\circ h)(x) \pause) = f(g(h(x))) \pause$
        \end{center}

        Logo $(f\circ g) \pause \circ h  \pause = f\circ \pause (g\circ h)$. \pause

        \vspace{.3cm}

        Agora, para toda $f \in \mathcal{S}$\pause
        \[
            f\circ id  \pause = f  \pause = id\circ f, \pause
        \]
        onde $id : A \to A$ \pause é tal que $id(x) = x$, \pause para todo $x \in A$. \pause Logo $id$ é o elemento neutro da composição.\pause

    \end{frame}

    \begin{frame}

        Finalmente, \pause para toda $f \in \mathcal{S}$, \pause como $f$ é bijetora \pause existe $g \in \mathcal{S}$ \pause tal que \pause
        \[
            f\circ g \pause = id \pause = g \circ f. \pause
        \]
        Logo todo elemento de $\mathcal{S}$ \pause possui inverso. \pause

        \vspace{.3cm}

        Portanto $(\mathcal{S}, \circ)$ \pause é um grupo.  \pause Além disso, em geral, esse grupo não é comutativo. \pause

        \vspace{.3cm}

        Vamos considerar agora o caso particular \pause em que $A$ é um conjunto finito. \pause

        \vspace{.3cm}

        Nessa situação podemos supor que $A \sub \n$ \pause para simplificar a notação. \pause

        \vspace{.3cm}

        Vamos ver como é o conjunto $\mathcal{S}$ com essa hipótese. \pause
    \end{frame}

    \begin{frame}
        Se $A = \{1\}$, \pause então só existe uma função $f : A \to A$ \pause que é bijetora e essa função é tal que \pause
        \begin{center}
            $f : \{1\} \to \{1\}$ \pause\\
            $f(1) = 1$. \pause
        \end{center}
        Ou seja, $f$ é a função identidade $id$. \pause Nesse caso $\mathcal{S} \pause = S_1 \pause = \{id\}$ \pause e $(S_1, \circ)$ é um grupo, \pause e nesse caso comutativo. \pause
    \end{frame}

    \begin{frame}
        Se $A = \{1, 2\}$, \pause então podemos definir as seguintes funções bijetoras em $A$: \pause
        \begin{multicols}{2}
            \begin{enumerate}
                \item[] \begin{center}
                    $id : A \to A$ \pause\\
                    $id(1) = 1$ \pause\\ \vspace{.3cm} $id(2) = 2$
                \end{center}\pause
                \item[]  \begin{center}
                    $f : A \to A$ \pause\\ $f(1) = 2 \pause$\\ \vspace{.3cm}  $f(2) = 1$
                \end{center}\pause
            \end{enumerate}
        \end{multicols}

        Assim $\mathcal{S} \pause = S_2 \pause = \{id, \pause f\}$ \pause e $(S_2, \circ)$ é um grupo.\pause

        Além disso, esse grupo é comutativo.
    \end{frame}

    \begin{frame}
        Agora, seja $A = \{1, 2, 3\}$. \pause Podemos definir então as seguintes funções bijetoras em $A$: \pause
        \begin{multicols}{3}
            \begin{enumerate}
                \item[] \begin{center}
                    $id : A \to A$ \pause\\
                    $id(1) = 1$ \pause\\ \vspace{.3cm}
                    $id(2) = 2$ \pause\\ \vspace{.3cm}
                    $id(3) = 3$ \pause
                \end{center}\pause

                \vspace{.3cm}

                \item[] \begin{center}
                    $f_1 : A \to A$\pause\\
                    $f_1(1) = 2$\pause\\ \vspace{.3cm}
                    $f_1(2) = 1$\pause\\ \vspace{.3cm}
                    $f_1(3) = 3$
                \end{center}\pause

                \vspace{.3cm}

                \item[] \begin{center}
                    $f_2 : A \to A$\pause\\
                    $f_2(1) = 3$\pause\\ \vspace{.3cm}
                    $f_2(2) = 2$\pause\\ \vspace{.3cm}
                    $f_2(3) = 1$
                \end{center}\pause

                \vspace{.3cm}

                \item[] \begin{center}
                    $f_3 : A \to A$\pause\\
                    $f_3(1) = 1$\pause\\ \vspace{.3cm}
                    $f_3(2) = 3$\pause\\ \vspace{.3cm}
                    $f_3(3) = 2$
                \end{center}\pause

                \vspace{.3cm}

                \item[] \begin{center}
                    $f_4 : A \to A$\pause\\
                    $f_4(1) = 2$\pause\\ \vspace{.3cm}
                    $f_4(2) = 3$\pause\\ \vspace{.3cm}
                    $f_4(3) = 1$
                \end{center}\pause

                \vspace{.3cm}

                \item[] \begin{center}
                    $f_5 : A \to A$\pause\\
                    $f_5(1) = 3$\pause\\ \vspace{.3cm}
                    $f_5(2) = 1$\pause\\ \vspace{.3cm}
                    $f_5(3) = 2$
                \end{center}
            \end{enumerate}
        \end{multicols}
    \end{frame}

    \begin{frame}
        Logo $\mathcal{S} = S_3 = \{id, f_1, f_2, f_3, f_4, f_5\}$ \pause e $(S_3, \circ)$ é um grupo.\pause

        \vspace{.3cm}

        Nesse caso temos\pause
        \begin{center}
            $(f_1 \circ f_4)(1) \pause = f_1(f_4(1)) \pause = f_1(2) \pause = 1$\pause\\
            \vspace{.3cm}
            $(f_4 \circ f_1)(1) \pause = f_4(f_1(1)) \pause = f_4(2) \pause = 3$\pause
        \end{center}
        daí $(f_1 \circ f_4)(1) \pause \ne (f_4 \circ f_1)(1)$ \pause, isto é, \pause $f_1 \circ f_4 \pause \ne f_4 \circ f_1$. \pause Portanto o grupo $(S_3, \circ)$ \pause não é comutativo.\pause

        \vspace{.3cm}

        Note que em $S_2$ \pause temos $2 = 2!$ elementos \pause e em $S_3$ \pause temos $6 = 3!$ elementos.\pause
    \end{frame}

    \begin{frame}
        De modo geral, \pause se $A = \{1, 2, 3, \dots, n\}$ \pause então existem exatamente $n!$ \pause funções $f : A \to A$ bijetoras. \pause

        \vspace{.3cm}

        Assim o grupo $(S_n, \circ)$ \pause possui $n!$ elementos.\pause

        \vspace{.3cm}

        Se $n \geqslant 3$, então \pause $S_n$ é um grupo não comutativo.\pause

        \begin{definicao}
            O grupo $S_n$ é chamado de \pause \textbf{grupo simétrico} \pause ou \textbf{grupo de permutações} \pause em $A = \{1, 2, 3, \dots, n\}$.
        \end{definicao}
    \end{frame}

    \begin{frame}
        Um modo de representar os elementos de $S_n$ é o seguinte: \pause vamos representar as funções $f \in S_n$ \pause na forma de uma matriz contendo 2 linhas \pause e $n$ colunas. \pause A primeira linha é o domínio da função \pause e a segunda contém suas imagens. \pause Assim se $f \in S_n$ escreveremos\pause
        \[
            f = \pause \begin{pmatrix}
                1 & 2 & 3 & \dots & n\pause \\
                f(1) \pause & f(2) \pause & f(3) \pause & \dots \pause & f(n)\pause
            \end{pmatrix}.
        \]
    \end{frame}

    \begin{frame}
        No caso de $S_3$ vamos escrever\pause
        \begin{multicols}{3}
            \begin{enumerate}
                \item[] $id = \begin{pmatrix}
                    1 & 2 & 3\pause\\
                    1 \pause & 2 \pause & 3\pause
                \end{pmatrix}$\pause

                \vspace{.5cm}

                \item[] $f_1 = \begin{pmatrix}
                    1 & 2 & 3 \pause\\
                    2 \pause & 1 \pause & 3 \pause
                \end{pmatrix}$\pause

                \vspace{.5cm}

                \item[] $f_2 = \begin{pmatrix}
                    1 & 2 & 3 \pause\\
                    3 \pause & 2 \pause & 1 \pause
                \end{pmatrix}$\pause

                \vspace{.5cm}

                \item[] $f_3 = \begin{pmatrix}
                    1 & 2 & 3 \pause\\
                    1 \pause & 3 \pause & 2 \pause
                \end{pmatrix}$\pause

                \vspace{.5cm}

                \item[] $f_4 = \begin{pmatrix}
                    1 & 2 & 3 \pause\\
                    2 \pause & 3 \pause & 1 \pause
                \end{pmatrix}$\pause

                \vspace{.5cm}

                \item[] $f_5 = \begin{pmatrix}
                    1 & 2 & 3 \pause\\
                    3 \pause & 1 \pause & 2 \pause
                \end{pmatrix}$\pause
            \end{enumerate}
        \end{multicols}
    \end{frame}

    \begin{frame}
        Assim a composição $f_3 \circ f_4$ pode ser determinada da seguinte forma:
        \[
            f_3\circ f_4 = \begin{pmatrix}
                    1 & 2 & 3\\
                    1 & 3 & 2
                \end{pmatrix} \pause \circ \begin{pmatrix}
                    1 & 2 & 3\\
                    2 & 3 & 1
                \end{pmatrix}
        \]
    \end{frame}

    \begin{frame}
        A composição $f_4 \circ f_5$ é:

        \[
            f_4\circ f_5 = \begin{pmatrix}
                    1 & 2 & 3\\
                    2 & 3 & 1
                \end{pmatrix} \pause \circ \begin{pmatrix}
                    1 & 2 & 3\\
                    3 & 1 & 2
                \end{pmatrix}
        \]
    \end{frame}
\end{document}
