%!TEX program = xelatex
%!TEX encoding = UTF-8
\def\disciplina{Álgebra 1}
\def\autor{José Antônio O. Freitas}
\def\instituto{MAT-UnB}

\documentclass{beamer}
\usetheme{Madrid}
\usecolortheme{beaver}
% \mode<presentation>
\usepackage{caption}
\usepackage{amssymb}
\usepackage{amsmath,amsfonts,amsthm,amstext}
\usepackage[brazil]{babel}
% \usepackage[latin1]{inputenc}
\usepackage{graphicx}
\graphicspath{{/ArquivosLinux/OneDrive/imagens-latex/}{D:/OneDrive - unb.br/imagens-latex/}}
\usepackage{enumitem}
\usepackage{multicol}
\usepackage{answers}
\usepackage{tikz,ifthen}
\usetikzlibrary{lindenmayersystems}
\usetikzlibrary[shadings]
\newtheorem{definicao}{Definição}[section]
\newtheorem{definicoes}{Definições}[section]
\newtheorem{exemplo}{Exemplo}[section]
\newtheorem{exemplos}{Exemplos}[section]
\newtheorem{exercicio}{Exercício}
\newtheorem{observacao}{Observação:}[section]
\newtheorem{observacoes}{Observações:}[section]
\newtheorem*{solucao}{Solução:}
\newtheorem{proposicao}{Proposição}
\newtheorem{lema}{Lema}
\newtheorem{teorema}{Teorema}
\newtheorem{corolario}{Corolário}
\newenvironment{prova}[1][Prova]{\noindent\textbf{#1:} }{\qedsymbol}%{\ \rule{0.5em}{0.5em}}
\def\ano{2024}
\def\semestre{1}
\def\disciplina{Álgebra 1}
\def\nomeabreviado{Álgebra 1}
\def\turma{1}

\newcommand{\im}{{\rm Im\,}}
\newcommand{\dlim}[2]{\displaystyle\lim_{#1\rightarrow #2}}
\newcommand{\minf}{+\infty}
\newcommand{\ninf}{-\infty}
\newcommand{\cp}[1]{\mathbb{#1}}
\newcommand{\sub}{\subseteq}
\newcommand{\n}{\mathbb{N}}
\newcommand{\z}{\mathbb{Z}}
\newcommand{\rac}{\mathbb{Q}}
\newcommand{\real}{\mathbb{R}}
\newcommand{\complex}{\mathbb{C}}

\newcommand{\vesp}[1]{\vspace{ #1  cm}}

\newcommand{\compcent}[1]{\vcenter{\hbox{$#1\circ$}}}
\newcommand{\comp}{\mathbin{\mathchoice
        {\compcent\scriptstyle}{\compcent\scriptstyle}
        {\compcent\scriptscriptstyle}{\compcent\scriptscriptstyle}}}
\renewcommand{\sin}{{\rm sen\,}}
\renewcommand{\tan}{{\rm tg\,}}
\renewcommand{\csc}{{\rm cossec\,}}
\renewcommand{\cot}{{\rm cotg\,}}
\renewcommand{\sinh}{{\rm senh\,}}

\title{Teoria de Conjuntos}
\author[\autor]{\autor}
\institute[\instituto]{\instituto}
\date{}

\begin{document}
    \begin{frame}
        \maketitle
    \end{frame}

    \logo{\includegraphics[scale=.1]{logo-MAT.png}\vspace*{8.5cm}}

    \begin{frame}
        \vspace{.4cm}
        Um conjunto é \pause uma ``coleção'' \pause ou ``família'' \pause de objetos que serão chamados de \textbf{elementos} do conjunto.\pause

        \vspace{.4cm}

        Denotaremos os conjuntos por letras maiúscula \pause e os elementos de um dado conjunto por letras minúsculas.\pause

        \vspace{.4cm}
        Seja $A$ um conjunto, \pause para indicar que $x$ é \pause um elemento de $A$ \pause ou que $x$ \textbf{pertence} ao conjunto $A$, \pause escrevemos:
        \[
            x \in A.\pause
        \]

        Para dizer que um elemento $x$ \pause não pertence ao conjunto $A$, escrevemos:\pause
        \[
            x \notin A.
        \]
    \end{frame}

    \begin{frame}
        Um conjunto sem elementos \pause é chamado de \textbf{conjunto vazio} \pause e é denotado por \pause $\emptyset$ ou \pause $\{\}$.\pause

        \vspace{.4cm}
        Dado um conjunto $A$ \pause e $x$ um elemento,\pause temos:\pause
        \[
            x \in A\pause \mbox{ ou } x \notin A.\pause
        \]

        Além disso, \pause para dois elementos \pause $x$, $y \in A$, \pause sempre ocorre:\pause
        \[
            x = y\pause \mbox{ ou } x \neq y
        \]
    \end{frame}

    \begin{frame}
        \vspace{.4cm}
        Um conjunto $A$ pode ser dado pela simples listagem dos seus elementos, entre chaves:\pause
            \begin{center}
                $A = \{1, bola, carro\}$\pause\\
                $B = \{verdade, falso\}.$\pause
            \end{center}


        Ou pela descrição das propriedades dos seus elementos, também  entre chaves:\pause
        \begin{center}
            $A = \{n \mid n \mbox{ é múltiplo de } 2\}$\pause
        \end{center}
    \end{frame}

    \begin{frame}
        \begin{enumerate}[label={\arabic*})]
            \item $\n = \{0,1,2,3,...\}$ o conjunto dos números naturais.\pause
            \vspace{.3cm}
            \item $\n^* = \{1,2,3,...\}$ o conjunto dos números naturais sem o zero.\pause
            \vspace{.3cm}
            \item $\z = \{...,-2,-1,0,1,2,...\}$ o conjunto dos números inteiros.\pause
            \vspace{.3cm}
            \item $\rac = \left\{\dfrac{p}{q} \mid p,q \in \z, q \neq 0 \right\}$ o conjunto dos números racionais.\pause
            \vspace{.3cm}
            \item $\real $ o conjunto dos números reais.\pause
            \vspace{.3cm}
            \item $\complex = \{a + bi \mid a,b \in \real,\ i^2 = -1\}$ o conjunto dos números complexos.
        \end{enumerate}
    \end{frame}

    \begin{frame}
        \begin{definicao}
            Dados dois conjuntos $A$ e $B$, \pause dizemos que $A$ e $B$ são \textbf{iguais} \pause se, e somente se, eles têm os mesmos elementos. \pause Ou seja, para todo $x \in A$ também vale que $x \in B$ \pause e para todo $y \in B$ também vale que $y \in A$. \pause Se $A$ e $B$ são iguais, \pause escrevemos $A = B$.\pause
        \end{definicao}

        \begin{definicao}
            Se $A$ e $B$ são dois conjuntos, \pause dizemos que $A$ é um \textbf{subconjunto} de $B$ \pause ou que $A$ \textbf{est\'a contido} em $B$ \pause ou que $B$ \textbf{contém} $A$ \pause se todo elemento de $A$ for elemento de $B$. \pause Ou seja, se para todo elemento $x \in A$, \pause temos $x \in B$. \pause Nesse caso, escrevemos $A \subseteq B$ (ou $A \subset B$) \pause ou $B \supseteq A$ (ou $B \supset A$).
        \end{definicao}
    \end{frame}

    \begin{frame}

        \begin{observacao}
            Dados dois conjuntos $A$ e $B$ \pause para que $A$ \textbf{não esteja contido em} $B$ basta \pause que exista $x \in A$ tal que $x \notin B$. \pause Nesse caso escrevemos $A \nsubseteq B$.
        \end{observacao}
    \end{frame}
    \begin{frame}
        \vspace{.4cm}
        Pela definição de continência de conjuntos, \pause podemos reescrever a igualdade de conjuntos, da seguinte forma: \pause dados dois conjuntos $A$ e $B$\pause
        \begin{center}
            $A = B$\quad \textbf{se, e somente se,} \pause\quad $A \subseteq B$ \quad\pause \textbf{e}\quad $B \subseteq A$.\pause
        \end{center}

        Ou seja,
        \begin{center}
            \textbf{se} $A = B$, \textbf{então} $A \subseteq B$ \textbf{e} $B \subseteq A$.\pause
        \end{center}

        Além disso,
        \begin{center}
            \textbf{se} $A \subseteq B$ \textbf{e} $B \subseteq A$, \textbf{então} $A = B$.\pause
        \end{center}

        Quando $A$ e $B$ não são iguais, escrevemos $A \neq B$.\pause

        \begin{proposicao}
            Dados três conjuntos $A$, $B$ e $C$ temos:\pause
            \begin{enumerate}[label={\roman*})]
                \item $A\subseteq A$ (Reflexividade)\pause
                \item Se $A\subseteq B \mbox{ e } B\subseteq A$, então $A=B$. (Antissimetria)\pause
                \item Se $A\subseteq B$ e $B\subseteq C$, então $A\subseteq C$. (Transitividade)
            \end{enumerate}
        \end{proposicao}
    \end{frame}

    \begin{frame}
        Considere os seguintes conjuntos:\pause
        \begin{center}
            $A = \{ n \in \n \mid n \mbox{ é múltiplo de } 2\} = \{2,4,6,...\}$\pause\\
            $B = \{n \in \n \mid n \mbox{ é múltiplo de } 3\} = \{3,6,9,...\}.$\pause
        \end{center}

        Neste caso, $A \nsubseteq B$ \pause e $B \nsubseteq A$. \pause Portanto, dados dois conjuntos $A$ e $B$, nem sempre
        temos $A \subseteq B$ \pause ou $B \subseteq A$.\pause

        \begin{proposicao}
            Seja $A$ um conjunto. \pause Então $ \emptyset \subseteq A$.
        \end{proposicao}
    \end{frame}

        \begin{frame}
        \begin{definicao}
            Sejam $A$ e $B$ dois conjuntos. \pause Definimos a \textbf{interseção} de $A$ e $B$ \pause como sendo o conjunto $A \cap B$ \pause cujos elementos pertencem aos conjuntos $A$ e $B$ simultaneamente. \pause Assim,
            \[
                A \cap B = \{x \mid x \in A\mbox{ e }  x \in B\}.
            \]
        \end{definicao}

    \end{frame}

    \begin{frame}
        \begin{definicao}
            Sejam $A$ e $B$ dois conjuntos. \pause Definimos a \textbf{união} de $A$ com $B$ \pause como sendo o conjunto $A \cup B$, \pause cujos elementos pertencem ao conjunto $A$ ou ao conjunto $B$. \pause Assim,\pause
            \[
                A \cup B = \{x \mid x \in A \mbox{ ou } x \in B\}.
            \]
        \end{definicao}
    \end{frame}

    \begin{frame}
        \begin{proposicao} Sejam $A$ e $B$ dois conjuntos. \pause Então:\pause
            \begin{enumerate}[label={\roman*})]
                \item $(A \cap B) \subseteq A$;\pause
                \item $(A \cap B) \subseteq B$;\pause
                \item $A \subseteq A \cup B$;\pause
                \item $B \subseteq A \cup B$.
            \end{enumerate}
        \end{proposicao}
    \end{frame}

    \begin{frame}
        \begin{definicao}
            Sejam $A_{1}$, \pause $A_2$, \pause \dots, $A_{n}$ \pause conjuntos. \pause Então:\pause
            \[
                A_{1} \cup \pause A_{2} \cup \pause \cdots \cup A_{n} \pause = \displaystyle\bigcup_{k=1}^n \pause A_{k}\pause
            \]
            é o conjunto dos elementos $x$ \pause tais que $x$ pertence a \pause pelo menos um \pause dos conjuntos $A_{1}$, \pause $A_2$, \pause \dots, $A_{n}$. \pause Agora,\pause
            \[
                A_{1} \cap \pause A_2 \cap \pause \cdots \cap A_{n} \pause = \displaystyle\bigcap_{k=1}^{n} \pause A_{k}\pause
            \]
            é o conjunto dos elementos $x$ \pause que pertencem a todos \pause os conjuntos $A_{1}$, \pause $A_2$, \pause \dots, $A_{n}$ \pause simultaneamente.
        \end{definicao}
    \end{frame}

    \begin{frame}
        \begin{definicao}
            Sejam $A$ e $B$ conjuntos. \pause Se $A \cap B = \emptyset$, \pause dizemos que $A$ e $B$ são \pause \textbf{conjuntos disjuntos}.
        \end{definicao}
    \end{frame}

    \begin{frame}
        \begin{proposicao} Sejam $A,$ $B$ e $C$ três conjuntos, \pause então:\pause
            \begin{enumerate}[label={\roman*})]
                \item $A \cap \pause ( B \cup C) \pause = (A \cap B) \pause \cup \pause (A \cap C)$\pause
                \item $A \cup \pause (B \cap C) \pause = (A \cup B) \pause \cap \pause (A \cup C)$\pause
            \end{enumerate}
        \end{proposicao}
    \end{frame}

    \begin{frame}
        \begin{definicao}
            Dados dois conjuntos $A$ e $B$, \pause definimos a \textbf{diferença} \pause dos conjuntos $A$ e $B$, denotada por \pause $A - B$ ou $A \backslash B$ \pause como sendo o conjunto\pause
            \[
                A - B = \pause \{x \pause \mid x \in A \pause \mbox{ e } x \notin B\}.
            \]
        \end{definicao}

    \end{frame}

    \begin{frame}
        \begin{proposicao}
            Sejam $A$, $B$ e $C$ \pause conjuntos não vazios. Então\pause
            \[
                (A \cup B) \pause - C \pause = (A - C) \pause \cup \pause (B - C).
            \]
        \end{proposicao}
    \end{frame}

    \begin{frame}
        \begin{definicao}
        Dados dois conjuntos $A$ e $E$ \pause tais que $A\subseteq E$, \pause definimos o \textbf{complementar} \pause de $A$ em $E$, denotado $A^C$ ou $C_E(A)$, como\pause
        \[
            C_E(A) = \pause \{ x \in E \pause \mid x \notin A \}.
        \]
        \end{definicao}
    \end{frame}

    \begin{frame}
        \begin{proposicao}
            Sejam $A$, $B$ e $E$ conjuntos. \pause Se $A\subseteq B \pause \subseteq E$, \pause então $C_E(B) \pause \subseteq C_E(A)$.
        \end{proposicao}
    \end{frame}

    \begin{frame}
        \begin{proposicao}
            Sejam $A$, $B$ e $E$ três conjuntos \pause tais que $A\subseteq E$ \pause e $B\subseteq E$. \pause Então:\pause
            \begin{enumerate}[label={\roman*})]
                \item $(A\cup B)^C \pause = A^C \pause \cap \pause B^C$.\pause
                \item $(A\cap B)^C \pause = A^C \pause \cup \pause B^C$.
            \end{enumerate}
        \end{proposicao}
    \end{frame}

    \begin{frame}
        \begin{definicao}
            Para qualquer conjunto $A$, \pause indicamos por $\mathcal{P}(A)$ \pause o conjunto\pause
            \[
                \mathcal{P}(A) = \pause \{ X \mid X\subseteq A\}\pause
            \]
            que é chamado de \textbf{conjunto das partes} de $A$.
        \end{definicao}
    \end{frame}

    \begin{frame}
        \begin{definicao}
        Dados dois conjuntos $A$ e $B$, \pause definimos o \textbf{produto cartesiano} \pause de $A$ por $B$ como sendo o conjunto\pause
        \[
            A \times B = \pause \{(x,y) \pause \mid x\in A, y\in B\}.\pause
        \]
        \end{definicao}

        Dados $(x,y)$, \pause $(z,t) \in A\times B$, \pause temos
        \begin{center}
            $(x,y) = (z,t)$ \pause \textbf{se, e somente se,} $x = z$ \pause \textbf{e} $y = t$.
        \end{center}

    \end{frame}

\end{document}
