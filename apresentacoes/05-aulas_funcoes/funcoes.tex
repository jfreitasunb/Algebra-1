%!TEX program = xelatex
\def\ano{2020}
\def\semestre{1}
\def\disciplina{\'Algebra 1}
\def\turma{1}
\def\autor{Jos\'e Ant\^onio O. Freitas}
\def\instituto{MAT-UnB}

\documentclass{beamer}
\usetheme{Madrid}
\usecolortheme{beaver}
\usepackage{pgfplots}
\pgfplotsset{compat=1.15}
\usepackage{mathrsfs}
\usetikzlibrary{arrows}


% \mode<presentation>
\usepackage{caption}
\usepackage{amssymb}
\usepackage{amsmath,amsfonts,amsthm,amstext}
\usepackage[brazil]{babel}
% \usepackage[latin1]{inputenc}
\usepackage{graphicx}
\graphicspath{{/ArquivosLinux/OneDrive/imagens-latex/}{D:/OneDrive - unb.br/imagens-latex/}}
\usepackage{enumitem}
\usepackage{multicol}
\usepackage{answers}
\usepackage{tikz,ifthen}
\usetikzlibrary{lindenmayersystems}
\usetikzlibrary[shadings]
\newtheorem{definicao}{Definição}[section]
\newtheorem{definicoes}{Definições}[section]
\newtheorem{exemplo}{Exemplo}[section]
\newtheorem{exemplos}{Exemplos}[section]
\newtheorem{exercicio}{Exercício}
\newtheorem{observacao}{Observação:}[section]
\newtheorem{observacoes}{Observações:}[section]
\newtheorem*{solucao}{Solução:}
\newtheorem{proposicao}{Proposição}
\newtheorem{lema}{Lema}
\newtheorem{teorema}{Teorema}
\newtheorem{corolario}{Corolário}
\newenvironment{prova}[1][Prova]{\noindent\textbf{#1:} }{\qedsymbol}%{\ \rule{0.5em}{0.5em}}
\def\ano{2024}
\def\semestre{1}
\def\disciplina{Álgebra 1}
\def\nomeabreviado{Álgebra 1}
\def\turma{1}

\newcommand{\im}{{\rm Im\,}}
\newcommand{\dlim}[2]{\displaystyle\lim_{#1\rightarrow #2}}
\newcommand{\minf}{+\infty}
\newcommand{\ninf}{-\infty}
\newcommand{\cp}[1]{\mathbb{#1}}
\newcommand{\sub}{\subseteq}
\newcommand{\n}{\mathbb{N}}
\newcommand{\z}{\mathbb{Z}}
\newcommand{\rac}{\mathbb{Q}}
\newcommand{\real}{\mathbb{R}}
\newcommand{\complex}{\mathbb{C}}

\newcommand{\vesp}[1]{\vspace{ #1  cm}}

\newcommand{\compcent}[1]{\vcenter{\hbox{$#1\circ$}}}
\newcommand{\comp}{\mathbin{\mathchoice
        {\compcent\scriptstyle}{\compcent\scriptstyle}
        {\compcent\scriptscriptstyle}{\compcent\scriptscriptstyle}}}
\renewcommand{\sin}{{\rm sen\,}}
\renewcommand{\tan}{{\rm tg\,}}
\renewcommand{\csc}{{\rm cossec\,}}
\renewcommand{\cot}{{\rm cotg\,}}
\renewcommand{\sinh}{{\rm senh\,}}

\title{Fun\c{c}\~oes}
\author[\autor]{\autor}
\institute[\instituto]{\instituto}
\date{}

\begin{document}
    \begin{frame}
        \maketitle
    \end{frame}

    \logo{\includegraphics[scale=.1]{logo-MAT.png}\vspace*{8.5cm}}

    \begin{frame}
        \begin{definicao}
            Uma \textbf{fun{\c c}{\~a}o} \pause $f : A \to B$, \pause de um conjunto $A$ \pause em um conjunto $B$,
            \pause {\'e} uma rela{\c c}{\~a}o que associa os elementos de $A$ \pause com os elementos em $B$ \pause
            satisfazendo as seguintes condi\c{c}\~oes:\pause
            \begin{enumerate}[label={\roman*})]
                \item Para todo $x \in A$, \pause existe $y \in B$ \pause tal que $f(x) = y$.\pause

                \vspace{.3cm}

                \item  Se $x \in A$ \pause \'e tal que $f(x) = y_1$ \pause e $f(x) = y_2$ \pause com $y_1$, \pause $y_2 \in B$, \pause ent\~ao $y_1 = y_2$.\pause
            \end{enumerate}
            Nesse caso $y$ \'e chamado de \textbf{imagem} \pause de $x$ segundo $f$.\pause
        \end{definicao}

        O conjunto $A$ {\'e} chamado de \textbf{dom{\'\i}nio} \pause de $f$ \pause e ser\'a denotado por $\dom(f)$. \pause O conjunto $B$ {\'e} chamado de \textbf{contra-dom{\'\i}nio} \pause de $f$. \pause O conjunto\pause
        \[
            \im(f) = \pause \{f(x) \pause \mid x \in A\} \pause \sub B\pause
        \]
        \'e chamado \textbf{imagem} de $f$.
    \end{frame}

    \begin{frame}
        \begin{definicao}
            Seja $f : A \to B$ uma fun\c{c}\~ao.\pause
            \begin{enumerate}[label={\roman*})]
                \item Dizemos que $f$ \'e \textbf{injetora} \pause se dados $x_1$, \pause $x_2 \in A$ \pause tais que $f(x_1) = \pause f(x_2)$, \pause ent\~ao $x_1 = x_2$. \pause De modo equivalente, \pause dizemos que $f$ \'e \textbf{injetora} \pause se dados $x_1$, \pause $x_2 \in A$ \pause tais que $x_1 \ne x_2$, \pause ent\~ao $f(x_1) \ne f(x_2)$.\pause

                \vspace{.3cm}

                \item Dizemos que $f$ \'e \textbf{sobrejetora} \pause se para todo $y \in B$, \pause existe $x \in A$ \pause tal que $f(x) = y$.\pause

                \vspace{.3cm}

                \item Dizemos que $f$ é \textbf{bijetora} \pause se $f$ for \textbf{injetora} \pause e \textbf{sobrejetora} \pause simultaneamente.
            \end{enumerate}
        \end{definicao}
    \end{frame}
    \begin{frame}
        \vspace{1cm}
        \begin{definicao}
            Sejam $f : A \to B$ \pause e $g : B \to C$ \pause fun\c{c}\~oes. \pause Definimos a \textbf{fun\c{c}\~ao composta} \pause de $g$ com $f$ \pause como sendo a fun\c{c}\~ao denotada por $g \circ f \pause : A \to C$ \pause tal que \pause $(g\circ f)(x) \pause = g(f(x))$ \pause para todo $x \in A$.
        \end{definicao}
    \end{frame}

    \begin{frame}
        \begin{proposicao}
            Se $f : A \to B$ \pause e $g : B \to C$ \pause s{\~a}o fun{\c c}{\~o}es injetoras, \pause ent{\~a}o $g\circ f : \pause A \to C$ \pause {\'e} injetora.
        \end{proposicao}
    \end{frame}

    \begin{frame}
        \begin{proposicao}
            Se $f : A \to B$ \pause e $g : B \to C$ \pause s{\~a}o fun\c{c}\~oes sobrejetoras, \pause ent{\~a}o $g\circ f : A \to C$ \pause {\'e} sobrejetora.
        \end{proposicao}
    \end{frame}
        \begin{frame}
        Dado $f : A \to B$ \pause uma fun\c{c}{\~a}o, \pause queremos construir uma fun\c{c}\~ao $g : B \to A$ \pause de modo que
        \[
            g(f(x)) = x,\pause
        \]
        para todo $x \in A$. \pause Mas $f(x) = y$ \pause com $y \in B$. \pause Assim podemos tentar definir $g$ \pause como
        \[
            g(y) = x,\ y \in B \pause \quad \mbox{ se, e somente se, } \pause f(x) = y.\pause
        \]
        Com essa defini\c{c}\~ao \pause $g$ \'e uma fun\c{c}\~ao? \pause Vejamos um exemplo: \pause definida $f : \{0,1,2,3\} \to \{4,5,6,7,8\}$ por:\pause
        \begin{center}
            $f(0) = 5$ \pause\\
            \vspace{.3cm}
            $f(1) = 5$\pause\\
            \vspace{.3cm}
            $f(2) = 6$\pause\\
            \vspace{.3cm}
            $f(3) = 7$
        \end{center}
    \end{frame}

    \begin{frame}
        A partir da defini\c{c}\~ao acimas temos\pause
        \begin{center}
            $g(5) = 0 $\pause\\

            \vspace{.3cm}

            $g(5) = 1$\pause\\

            \vspace{.3cm}

            $g(6) = 2$\pause\\

            \vspace{.3cm}

            $g(7) = 3$\pause
        \end{center}

        Assim $g$ definida dessa forma \pause n\~ao \'e uma fun\c{c}\~ao \pause pois $g$ atribui ao n\'umero 5 \pause dois poss{\'\i}veis valores: \pause 0 e 1. \pause Isso ocorre pois $f$ n\~ao \'e injetora. \pause Vamos ent\~ao redefinir $f$ de modo a torn\'a-la injetora:\pause
        \begin{center}
            $f(0) = 5$\pause\\

            \vspace{.3cm}

            $f(1) = 4$\pause\\

            \vspace{.3cm}

            $f(2) = 6$\pause\\

            \vspace{.3cm}

            $f(3) = 7$\pause
        \end{center}
    \end{frame}

    \begin{frame}
        Agora $g$ torna-se:\pause
        \begin{center}
            $g(5) = 0$\pause\\

            \vspace{.3cm}

            $g(4) = 1$\pause\\

            \vspace{.3cm}

            $g(6) = 2$\\

            \vspace{.3cm}

            $g(7) = 3$\pause
        \end{center}

        Ainda assim $g$ n\~ao \'e fun\c{c}\~ao \pause pois $g$ n\~ao associa $8 \in B$ \pause com nenhum elemento em $A$.  Isso ocorre pois \pause $f$ n\~ao \'e sobrejetora.\pause

        \vspace{.3cm}

        Portanto para que a condi\c{c}\~ao dada \pause defina uma fun\c{c}\~ao \pause \'e necess\'ario que $f$ seja bijetora. \pause Temos ent\~ao o seguinte teorema:
    \end{frame}

    \begin{frame}
        \begin{teorema}
            Seja $f: A \to B$ uma fun{\c c}{\~a}o. \pause Defina $g : B \to A$ \pause por\pause
            \[
                g(y) = x,\ y \in B \pause \quad \mbox{ se, e somente se, } \pause f(x) = y.\pause
            \]
            Ent{\~a}o $g$ \pause {\'e} uma fun{\c c}{\~a}o \pause se, e somente se, \pause $f$ {\'e} bijetora.
        \end{teorema}
    \end{frame}

    \begin{frame}
        \begin{definicao}
            A fun\c{c}\~ao $g : B \to A$ \pause do teorema anterior \pause \'e chamada de \textbf{fun\c{c}\~ao inversa} \pause de $f : A \to B$ \pause e ser\'a denotada por $g = f^{-1}$.\pause
        \end{definicao}

        \vspace{.3cm}

        \begin{definicao}
            Sejam $f : A \to B$ \pause e $g : A \to B$ fun\c{c}\~oes. \pause Ent\~ao $f = g$ \pause quando $f(x) = g(x)$ \pause para todo $x \in A$.\pause
        \end{definicao}

        \vspace{.3cm}

        \begin{definicao}
            Dado um conjunto $A \ne \emptyset$, \pause a fun{\c c}{\~a}o $i_{A}: A \to A$ \pause dada por $i_{A}(x) \pause = x$ \pause {\'e} chamada de \pause \textbf{fun{\c c}{\~a}o identidade}.
        \end{definicao}
    \end{frame}

    \begin{frame}
        \begin{proposicao}
            Se $f : A \to B$ \pause {\'e} bijetora, \pause ent{\~a}o $f\circ f^{-1} \pause = i_{B}$ \pause e $f^{-1}\circ f \pause = i_{A}$.
        \end{proposicao}
    \end{frame}

    \begin{frame}
        \begin{proposicao}
            Se $f : A \to B$ \pause e $g : B \to A$ \pause s{\~a}o fun{\c c}{\~o}es, \pause ent{\~a}o:\pause
            \begin{enumerate}[label={\roman*})]
                \item $f\circ i_{A} = f$\pause

                \vspace{.3cm}

                \item $i_{B}\circ f = f$\pause

                \vspace{.3cm}

                \item $g\circ i_{B} = g$\pause

                \vspace{.3cm}

                \item $i_{A}\circ g = g$\pause

                \vspace{.3cm}

                \item Se $g\circ f = i_{A}$ \pause e $f\circ g = i_{B}$, \pause ent{\~a}o \pause $f$ e $g$ s{\~a}o bijetoras \pause e $g=f^{-1}$.
            \end{enumerate}
        \end{proposicao}
    \end{frame}

    \begin{frame}
        \begin{definicao}
            Seja $f : A \to B$ \pause uma fun{\c c}{\~a}o.\pause
            \begin{enumerate}[label={\roman*})]
                \item Dado $P \sub A$, \pause chama-se \textbf{imagem direta} \pause de $P$ \pause \textbf{segundo} $f$ \pause e indica-se por $f(P)$ \pause o subconjunto de $B$ \pause dado por\pause
                \[
                    f(P) = \pause \{f(x) \pause \mid x \in P\},\pause
                \]
                isto {\'e}, \pause $f(P)$ \pause {\'e} o conjunto das imagens por $f$ \pause dos elementos de $P$.\pause

                \vspace{.5cm}

                \item Dado $Q \sub B$, \pause chama-se \textbf{imagem inversa} \pause de $Q$ \textbf{segundo} $f$ \pause e indica-se por \pause $f^{-1}(Q)$ \pause o subconjunto de $A$ \pause dado por\pause
                \[
                    f^{-1}(Q) \pause = \{x \in A \pause \mid f(x) \in Q\},\pause
                \]
                isto {\'e}, \pause $f^{-1}(Q)$ \pause {\'e} o conjunto dos elementos de $A$ \pause que tem imagem em $Q$ \pause atrav{\'e}s de $f$.
            \end{enumerate}
        \end{definicao}
    \end{frame}

    \begin{frame}
        \begin{proposicao}
            Seja $f : A \to B$ uma fun{\c c}{\~a}o \pause e sejam $P$, \pause $Q \sub A$, \pause $R$, \pause $S \sub B$.\pause
            \begin{enumerate}[label={\roman*})]
                \item Se $P \sub Q$, \pause ent{\~a}o $f(P) \sub f(Q)$.\pause

                \vspace{.5cm}

                \item $f^{-1}(R \cup S) \pause = f^{-1}(R) \pause \cup f^{-1}(S)$.
            \end{enumerate}
        \end{proposicao}
    \end{frame}

    \begin{frame}
        Sejam $f : A \to B$ uma função e $X \sub B$. Mostre que
        \[ f^{-1}(X^C) = [f^{-1}(X)]^C, \]
        onde $X^C = C_B(X)$  e $[f^{-1}(X)]^C = C_A(f^{-1}(X))$.
    \end{frame}
\end{document}
