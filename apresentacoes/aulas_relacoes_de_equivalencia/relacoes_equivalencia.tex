%!TEX program = xelatex
%!TEX encoding = UTF-8
\def\autor{Jos\'e Ant\^onio O. Freitas}
\def\instituto{MAT-UnB}

\documentclass{beamer}
\usetheme{Madrid}
\usecolortheme{beaver}
% \mode<presentation>
\usepackage{caption}
\usepackage{amssymb}
\usepackage{amsmath,amsfonts,amsthm,amstext}
\usepackage[brazil]{babel}
% \usepackage[latin1]{inputenc}
\usepackage{graphicx}
\graphicspath{{/ArquivosLinux/OneDrive/imagens-latex/}{D:/OneDrive - unb.br/imagens-latex/}}
\usepackage{enumitem}
\usepackage{multicol}
\usepackage{answers}
\usepackage{tikz,ifthen}
\usetikzlibrary{lindenmayersystems}
\usetikzlibrary[shadings]
\newtheorem{definicao}{Definição}[section]
\newtheorem{definicoes}{Definições}[section]
\newtheorem{exemplo}{Exemplo}[section]
\newtheorem{exemplos}{Exemplos}[section]
\newtheorem{exercicio}{Exercício}
\newtheorem{observacao}{Observação:}[section]
\newtheorem{observacoes}{Observações:}[section]
\newtheorem*{solucao}{Solução:}
\newtheorem{proposicao}{Proposição}
\newtheorem{lema}{Lema}
\newtheorem{teorema}{Teorema}
\newtheorem{corolario}{Corolário}
\newenvironment{prova}[1][Prova]{\noindent\textbf{#1:} }{\qedsymbol}%{\ \rule{0.5em}{0.5em}}
\def\ano{2024}
\def\semestre{1}
\def\disciplina{Álgebra 1}
\def\nomeabreviado{Álgebra 1}
\def\turma{1}

\newcommand{\im}{{\rm Im\,}}
\newcommand{\dlim}[2]{\displaystyle\lim_{#1\rightarrow #2}}
\newcommand{\minf}{+\infty}
\newcommand{\ninf}{-\infty}
\newcommand{\cp}[1]{\mathbb{#1}}
\newcommand{\sub}{\subseteq}
\newcommand{\n}{\mathbb{N}}
\newcommand{\z}{\mathbb{Z}}
\newcommand{\rac}{\mathbb{Q}}
\newcommand{\real}{\mathbb{R}}
\newcommand{\complex}{\mathbb{C}}

\newcommand{\vesp}[1]{\vspace{ #1  cm}}

\newcommand{\compcent}[1]{\vcenter{\hbox{$#1\circ$}}}
\newcommand{\comp}{\mathbin{\mathchoice
        {\compcent\scriptstyle}{\compcent\scriptstyle}
        {\compcent\scriptscriptstyle}{\compcent\scriptscriptstyle}}}
\renewcommand{\sin}{{\rm sen\,}}
\renewcommand{\tan}{{\rm tg\,}}
\renewcommand{\csc}{{\rm cossec\,}}
\renewcommand{\cot}{{\rm cotg\,}}
\renewcommand{\sinh}{{\rm senh\,}}

\title{Relações de Equivalência}
\author[\autor]{\autor}
\institute[\instituto]{\instituto}
\date{}

\begin{document}
    \begin{frame}
        \maketitle
    \end{frame}

    \logo{\includegraphics[scale=.1]{logo-MAT.png}\vspace*{8.5cm}}

    \begin{frame}
        \begin{definicao}
            Seja $A$ um conjunto n{\~a}o vazio \pause e $R\subseteq A \times A$. \pause Dizemos que $R$ \pause {\'e} uma \textbf{rela{\c c}{\~a}o de equival{\^e}ncia} se:\pause
            \begin{enumerate}[label={\roman*})]
                \item Para todo $x \in A$, \pause $(x,x) \in R$. \pause \textit{(Propriedade Reflexiva)}\pause
                \item Se $(x, y) \in R$, \pause ent\~ao $(y, x) \in R$. \pause \textit{(Propriedade Sim\'etrica)}\pause
                \item Se $(x, y) \in R$ \pause e $(y, z) \in R$, \pause ent\~ao $(x, z)\in R$. \pause \textit{(Propriedade Transitiva)}\pause
            \end{enumerate}
        \end{definicao}

        Quando $R\subseteq A \times A$ {\'e} uma rela{\c c}{\~a}o de equival{\^e}ncia, \pause dizemos que $R$ {\'e} uma rela{\c c}{\~a}o de equival{\^e}ncia em $A$. \pause

        Quando dois elementos $x$, $y \in A$ \pause s{\~a}o tais que $(x,y) \in R$, \pause dizemos que $x$ e $y$ \textbf{s{\~a}o relacionados} \pause ou que $x$ e $y$ \textbf{est\~ao relacionados}.
    \end{frame}

    \begin{frame}
        \begin{observacoes}
            Seja $R$ uma rela{\c c}{\~a}o de equival{\^e}ncia em $A$, \pause isto \'e, $R \sub A \times A$.\pause
            \begin{enumerate}[label={\arabic*})]
                \item  Para dizermos que $(x, y) \in R$ \pause usaremos a nota{\c c}{\~a}o $x\equiv y\ (R)$, \pause que deve ser lido como ``$x$ \'e equivalente a $y$ m{\'o}dulo $R$", \pause ou ainda a nota{\c c}{\~a}o $xRy$ \pause, com o mesmo significado anterior.\pause\vspace{.2cm}

                \item Em alguns casos vamos utilizar a nota\c{c}\~ao $\sim$ \pause para representar a rela\c{c}\~ao $R$. \pause Nesse caso, escrevemos $x \sim y$ \pause para dizer que $(x, y) \in R$, \pause ou que, $xRy$.
            \end{enumerate}
        \end{observacoes}
    \end{frame}

    \begin{frame}
        \begin{definicao}
            Seja $A$ um conjunto n{\~a}o vazio \pause e $R\subseteq A \times A$. \pause Dizemos que $R$ {\'e} uma \textbf{rela{\c c}{\~a}o de equival{\^e}ncia} se:\pause
            \begin{enumerate}[label={\roman*})]
                \item Para todo $x \in A$, $xRx$. \textit{(Propriedade Reflexiva)}\pause\vspace{.2cm}

                \item Se $xRy$, ent\~ao $yRx$. \textit{(Propriedade Sim\'etrica)}\pause\vspace{.2cm}

                \item Se $xRy$ e $yRz$, ent\~ao $xRz$. \textit{(Propriedade Transitiva)}
            \end{enumerate}
        \end{definicao}
    \end{frame}

    \begin{frame}
        \begin{definicao}
            Seja $R$ uma rela{\c c}{\~a}o de equival{\^e}ncia sobre um conjunto $A$. \pause Dado $b \in A$, \pause chamamos de \textbf{classe de equival{\^e}ncia \pause determinada por $b$ \pause m{\'o}dulo $R$}\pause, denotada por $\overline{b}$ \pause ou $C(b)$, \pause o subconjunto de $A$ \pause dado por\pause
            \begin{center}
                $\overline{b} =\pause C(b) = \pause \{x \in A \pause \mid (x,b) \in R\}\pause = \{x \in A \pause\mid xRb\}.$
            \end{center}
        \end{definicao}
    \end{frame}

    \begin{frame}
        \begin{proposicao}
            Seja $R$ uma rela{\c c}{\~a}o de equival{\^e}ncia \pause em um conjunto n{\~a}o vazio $A$. \pause Dados $a$, $b \in A$ temos:\pause
            \begin{enumerate}[label={\roman*})]
                \item se $\overline{a} \cap \overline{b} \ne \emptyset$, \pause ent{\~a}o $aRb$.\pause \vspace{.3cm}
                \item se  $\overline{a} \cap \overline{b} \neq \emptyset$, \pause ent{\~a}o $\overline{a} = \overline{b}$.
            \end{enumerate}
        \end{proposicao}

        \begin{corolario}
            Seja $R$ uma rela\c{c}\~ao de equival\^encia sobre um conjunto n\~ao vazio $A$. \pause Dados $a$, $b \in A$ \pause ent\~ao $\overline{a} \cap \overline{b} = \emptyset$ \pause ou $\overline{a} = \overline{b}$.\pause
        \end{corolario}

    \end{frame}

    \begin{frame}
        \begin{definicao}
            Seja $R$ uma rela\c{c}\~ao de equival\^encia sobre um conjunto n\~ao vazio $A$. \pause O conjunto de todas as classes de equival{\^e}ncia \pause determinadas por $R$ ser{\'a} \pause denotado por $A/R$ \pause e {\'e} chamado de \textbf{conjunto quociente} de $A$ por $R$.\pause
        \end{definicao}
    \end{frame}

    \begin{frame}
        \begin{definicao}
            Seja $C$ uma classe de equival{\^e}ncia \pause de uma rela{\c c}{\~a}o de equival{\^e}ncia $R$. \pause Qualquer elemento $y\in C$ \pause {\'e} chamado \textbf{representante} de $C$.\pause
        \end{definicao}

        \begin{proposicao}
            Seja $A$ um conjunto n{\~a}o vazio \pause e $R$ uma rela{\c c}{\~a}o de equival{\^e}ncia em $A$. \pause Ent{\~a}o $A$ {\'e} a uni{\~a}o disjunta das classes $\overline{b}$, $b \in A$, ou seja,\pause
            \[
                A = \bigcup_{b\in A}\overline{b}.
            \]
        \end{proposicao}
    \end{frame}
\end{document}
