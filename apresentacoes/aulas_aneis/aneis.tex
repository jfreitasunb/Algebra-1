%!TEX program = xelatex
\def\ano{2020}
\def\semestre{1}
\def\disciplina{\'Algebra 1}
\def\turma{C}
\def\autor{Jos\'e Ant\^onio O. Freitas}
\def\instituto{MAT-UnB}

\documentclass{beamer}
\usetheme{Madrid}
\usecolortheme{beaver}
% \mode<presentation>
\usepackage{caption}
\usepackage{amssymb}
\usepackage{amsmath,amsfonts,amsthm,amstext}
\usepackage[brazil]{babel}
% \usepackage[latin1]{inputenc}
\usepackage{graphicx}
\graphicspath{{/ArquivosLinux/OneDrive/imagens-latex/}{D:/OneDrive - unb.br/imagens-latex/}}
\usepackage{enumitem}
\usepackage{multicol}
\usepackage{answers}
\usepackage{tikz,ifthen}
\usetikzlibrary{lindenmayersystems}
\usetikzlibrary[shadings]
\newtheorem{definicao}{Definição}[section]
\newtheorem{definicoes}{Definições}[section]
\newtheorem{exemplo}{Exemplo}[section]
\newtheorem{exemplos}{Exemplos}[section]
\newtheorem{exercicio}{Exercício}
\newtheorem{observacao}{Observação:}[section]
\newtheorem{observacoes}{Observações:}[section]
\newtheorem*{solucao}{Solução:}
\newtheorem{proposicao}{Proposição}
\newtheorem{lema}{Lema}
\newtheorem{teorema}{Teorema}
\newtheorem{corolario}{Corolário}
\newenvironment{prova}[1][Prova]{\noindent\textbf{#1:} }{\qedsymbol}%{\ \rule{0.5em}{0.5em}}
\def\ano{2024}
\def\semestre{1}
\def\disciplina{Álgebra 1}
\def\nomeabreviado{Álgebra 1}
\def\turma{1}

\newcommand{\im}{{\rm Im\,}}
\newcommand{\dlim}[2]{\displaystyle\lim_{#1\rightarrow #2}}
\newcommand{\minf}{+\infty}
\newcommand{\ninf}{-\infty}
\newcommand{\cp}[1]{\mathbb{#1}}
\newcommand{\sub}{\subseteq}
\newcommand{\n}{\mathbb{N}}
\newcommand{\z}{\mathbb{Z}}
\newcommand{\rac}{\mathbb{Q}}
\newcommand{\real}{\mathbb{R}}
\newcommand{\complex}{\mathbb{C}}

\newcommand{\vesp}[1]{\vspace{ #1  cm}}

\newcommand{\compcent}[1]{\vcenter{\hbox{$#1\circ$}}}
\newcommand{\comp}{\mathbin{\mathchoice
        {\compcent\scriptstyle}{\compcent\scriptstyle}
        {\compcent\scriptscriptstyle}{\compcent\scriptscriptstyle}}}
\renewcommand{\sin}{{\rm sen\,}}
\renewcommand{\tan}{{\rm tg\,}}
\renewcommand{\csc}{{\rm cossec\,}}
\renewcommand{\cot}{{\rm cotg\,}}
\renewcommand{\sinh}{{\rm senh\,}}

\title{An\'eis}
\author[\autor]{\autor}
\institute[\instituto]{\instituto}
\date{}

\begin{document}
    \begin{frame}
        \maketitle
    \end{frame}

    \logo{\includegraphics[scale=.1]{logo-MAT.png}\vspace*{8.5cm}}

    \begin{frame}
        \begin{definicao}
            Seja $A \ne \emptyset$ um conjunto. \pause Dizemos que $A$ est{\'a} munido \pause (ou equipado) \pause de
            uma \textbf{opera{\c c}{\~a}o bin{\'a}ria} \pause quando existe uma fun{\c c}{\~a}o\pause
            \begin{align*}
                &\Delta : A \times A \to A\\
                &(a,b) \longmapsto a\Delta b
            \end{align*}
            \pause Uma opera{\c c}{\~a}o bin{\'a}ria tamb{\'e}m {\'e} chamada de uma \textbf{opera{\c c}{\~a}o interna} em $A$.\pause
        \end{definicao}
    \end{frame}

    \begin{frame}
        \begin{exemplos}
            \begin{enumerate}[label={\arabic*})]
                \item A soma usual \pause nos conjuntos $\z$, \pause $\rac$, \pause $\real$ \pause e $\complex$ \pause
                    {\'e} uma opera{\c c}{\~a}o bin{\'a}ria.\pause

                \vspace{.3cm}

                \item A multiplica\c{c}\~ao usual \pause nos conjuntos $\z$, \pause $\rac$, \pause $\real$ \pause e
                    $\complex$ {\'e} uma opera{\c c}{\~a}o bin{\'a}ria.\pause

                \vspace{.3cm}

                \item Seja $m > 1$, \pause $m \in \z$ fixo. \pause A soma \pause e a multiplica\c{c}\~ao definidos em
                    $\z_m = \pause \{\overline{0}, \overline{1}, ..., \overline{m-1}\}$ \pause s\~ao opera\c{c}\~oes
                    bin\'arias.\pause

                \vspace{.3cm}

                \item Em $\n$, \pause $\z$, \pause $\z^{*}$ \pause e em $\rac$ \pause a opera\c{c}\~ao $\div$
                    \pause n{\~a}o {\'e} uma opera{\c c}{\~a}o bin{\'a}ria.

                \vspace{.3cm}

                \item A opera\c{c}\~ao $\div$ \pause em $\rac^{*}$ \pause {\'e} uma opera{\c c}{\~a}o bin{\'a}ria.\pause

                \vspace{.3cm}

            \end{enumerate}
        \end{exemplos}
    \end{frame}

    \begin{frame}
        \begin{definicao}
            Seja $A \ne \emptyset$ um conjunto \pause no qual est\~ao definidas duas opera{\c c}{\~o}es bin\'arias
            \pause $\oplus$ \pause e $\otimes$, \pause chamadas \textbf{soma} \pause e \textbf{produto} \pause ou
            \textbf{multiplicação}. \pause Dizemos que $(A, \oplus, \otimes)$ \pause {\'e} um \textbf{anel} \pause
            quando as seguintes condi{\c c}{\~o}es s{\~a}o verdadeiras:\pause
            \begin{enumerate}[label={\roman*})]
                \item \textbf{Associatividade}: \pause para todos $x$, \pause $y$, \pause $z \in A$ \pause vale\pause
                \[
                    (x \oplus y) \pause \oplus z \pause = x \oplus \pause (y \oplus z).\pause
                \]
                Essa propriedade {\'e} chamada de \pause \textbf{propriedade associativa} \pause da soma.\pause

                \vspace{.7cm}

                \item \textbf{Comutatividade}: \pause Para todos $x$, \pause $y \in A$ \pause vale\pause
                \[
                    x \oplus y = \pause y \oplus x.
                \]

                \vspace{.7cm}

                \seti
            \end{enumerate}
        \end{definicao}
    \end{frame}

    \begin{frame}
        \begin{definicao}
            \begin{enumerate}[label={\roman*})]
                \conti

                \item \textbf{Elemento Neutro}: \pause Existe em $A$ \pause um elemento denotado por $0$ \pause (zero)
                    ou $0_{A}$ \pause tal que para todo elemento $x \in A$ \pause vale\pause
                \[
                    x \oplus 0_A \pause = x \pause = 0_A \oplus x.\pause
                \]
                Tal elemento $0_A$ \pause \'e chamado de \textbf{elemento neutro da soma} \pause ou simplesmente
                \textbf{elemento neutro}.\pause

                \vspace{.7cm}

                \item \textbf{Elemento Oposto}: \pause Para cada elemento $x \in A$, \pause existe $y \in A$ \pause tal
                    que\pause
                \[
                    x \oplus y \pause = 0_A \pause = y \oplus x.\pause
                \]
                Tal elemento $y$ \pause \'e chamado de \textbf{oposto aditivo} \pause de $x$ \pause ou simplesmente
                \textbf{oposto} de $x$.

                \vspace{.7cm}

                \seti
            \end{enumerate}
        \end{definicao}
    \end{frame}

    \begin{frame}
        \begin{definicao}
            \begin{enumerate}[label={\roman*})]
                \conti

                \item \textbf{Associatividade}: \pause Para todos $x$, \pause $y$, \pause $z \in A$, \pause vale\pause
                \[
                    (x\otimes y) \pause \otimes z \pause = x \otimes \pause (y\otimes z).\pause
                \]

                \vspace{.7cm}

                \item \textbf{Distributividade}: \pause Para todos $x$, \pause $y$, \pause $z \in A$ \pause vale\pause
                \[
                    (x \oplus y) \pause \otimes z \pause = x \otimes z \pause \oplus \pause y\otimes z.\pause
                \]
                Essa propriedade {\'e} chamada \textbf{distributiva da soma em rela{\c c}{\~a}o ao produto}.

                \vspace{.7cm}

                \seti
            \end{enumerate}
        \end{definicao}
    \end{frame}

    \begin{frame}
        \begin{definicao}
            \begin{enumerate}[label={\roman*})]
                \conti

                \item \textbf{Distributividade}: \pause Para todos $x$, \pause $y$, \pause $z \in A$ \pause vale\pause
                \[
                    x \otimes \pause (y \oplus z) \pause = x \otimes y \pause \oplus \pause x\otimes z.\pause
                \]
                Essa {\'e} a propriedade \textbf{distributiva do produto em rela{\c c}{\~a}o {\`a} soma}.
            \end{enumerate}
        \end{definicao}
    \end{frame}

    \begin{frame}
        \begin{observacoes}
            Seja $(A, \oplus, \otimes)$ \pause um anel.\pause
            \begin{enumerate}[label={\arabic*})]
                \item \textbf{Comutatividade}: Se para todos $x$, \pause $y \in A$ \pause vale
                \[
                    x \otimes y \pause = y \otimes x.\pause
                \]
                Dizemos que $(A, \oplus, \otimes)$ \pause {\'e} um \textbf{anel comutativo}.\pause

                \vspace{.7cm}

                \item \textbf{Unidade}: \pause Se existe em $A$ \pause um elemento denotado por $1$ \pause ou $1_{A}$
                    \pause tal que\pause
                \[
                    x \otimes 1_A \pause = x \pause = 1_A \otimes x,\pause
                \]
                para todo $x \in A$, \pause ent{\~a}o dizemos que $(A, \oplus, \otimes)$ \pause \'e um \textbf{anel com
                unidade} \pause ou um \textbf{anel unit{\'a}rio} ou ainda um \textbf{anel com identidade}. \pause O
                elemento $1_A$ \pause {\'e} chamado de \textbf{unidade} de $A$ \pause ou \textbf{elemento neutro da
                multiplica\c{c}\~ao} \pause de $A$.

                \vspace{.7cm}

                \seti
             \end{enumerate}
        \end{observacoes}
    \end{frame}

    \begin{frame}
        \begin{observacoes}
            \begin{enumerate}[label={\arabic*})]
                \conti

                \item Se um anel $(A, \oplus, \otimes)$ \pause satisfaz as duas propriedades anteriores \pause dizemos
                    que $(A, \oplus, \otimes)$ \'e um \textbf{anel comutativo com unidade} \pause ou um \textbf{anel
                    comutativo unit\'ario}.\pause

                \vspace{.5cm}

                \item Seja $(A, \oplus, \otimes)$ um anel. \pause Quando n\~ao houver chance de confus\~ao com
                    rela\c{c}\~ao \`as opera\c{c}\~oes envolvidas diremos simplesmente que \pause $A$ \'e um anel.\pause
            \end{enumerate}
        \end{observacoes}
    \end{frame}

    \begin{frame}
        \begin{exemplos}
            \begin{enumerate}[label={\arabic*})]
                \item $(\z,+,.)$, \pause $(\rac,+,.)$, \pause $(\real,+,.)$, \pause $(\complex,+,.)$ s{\~a}o an{\'e}is
                    comutativos \pause e com unidade.\pause

                \vspace{1cm}

                \item Seja $m > 1$, $m \in \z$. \pause Então $(\z_m, \oplus, \otimes)$ é um anel \pause comutativo e unitário.\pause
            \end{enumerate}
        \end{exemplos}

        \vspace{0.5cm}

        \begin{observacao}
            Por uma questão de simplicidade vamos denotar a operação de soma $\oplus$ por $+$ \pause e a operação de multiplicação $\otimes$ por $\cdot$.
        \end{observacao}

    \end{frame}

    \begin{frame}
        \begin{proposicao}
            Seja $(A, + , \cdot)$ um anel. \pause Ent\~ao:\pause
            \begin{enumerate}[label={\roman*})]
                \item O elemento neutro {\'e} {\'u}nico.\pause

                \vspace{.5cm}

                \item Para cada $x \in A$ \pause existe um {\'u}nico
                    oposto. \pause Neste caso o \textbf{oposto} de
                    $x \in A$ \pause ser\'a denotado por $-x$.\pause

                \vspace{.5cm}

                \item Para todo $x \in A$, \pause
                \[
                    -(-x) = x.\pause
                \]

                \vspace{.5cm}

                \item Dados $x_{1}$, \pause $x_{2}$, \pause \dots,
                    $x_n \in A$, \pause $n \geqslant 2$, \pause
                    ent{\~a}o\pause
                \[
                    -(x_1 + x_2 + \dots + x_n) \pause = (-x_1) \pause
                    + (-x_2) \pause + \dots + (-x_n).\pause
                \]

                \vspace{.2cm}

                \seti
            \end{enumerate}
        \end{proposicao}
    \end{frame}

    \begin{frame}
        \begin{proposicao}
            \begin{enumerate}[label={\roman*})]
                \conti

                \item Para todos $\alpha$, \pause $x$, \pause $y \in A$, \pause se
                \[
                    \alpha + x \pause = \alpha + y,\pause
                \]
                ent{\~a}o $x = y$.\pause

                \vspace{.5cm}

                \item Para todo $x \in A$, \pause
                \[
                    x\cdot 0_A \pause = 0_A \pause = 0_A\cdot x.\pause
                \]

                \vspace{.5cm}
            \end{enumerate}
        \end{proposicao}
    \end{frame}

    \begin{frame}
        \begin{proposicao}
            \begin{enumerate}
                \item[vii)] Para todos $x$, \pause $y \in A$, \pause temos\pause
                \[
                    x \cdot (-y) \pause = (-x) \cdot y \pause = -(x \cdot y).\pause
                \]

                \vspace{.5cm}

                \item[viii)] Para todos $x$, \pause $y \in A$, \pause
                \[
                    x \cdot y \pause = (-x) \cdot (-y).\pause
                \]

                \vspace{.5cm}
            \end{enumerate}
        \end{proposicao}
    \end{frame}

    \begin{frame}
        \begin{definicao}
            Seja $(A, +, \cdot)$ um anel. \pause Dizemos que um subconjunto n{\~a}o vazio \pause $B\subseteq A$ \pause {\'e} um \textbf{subanel} de $A$ \pause quando $(B, +, \cdot)$ \'e um anel.\pause
        \end{definicao}

        \begin{exemplos}
            \begin{enumerate}[label={\arabic*})]
                \item Todo anel $A$ sempre tem dois suban{\'e}is: \pause $\{0_{A}\}$ \pause e $A$, \pause que s{\~a}o chamados de \textbf{suban{\'e}is triviais}.\pause

                \vspace{.5cm}

                \item Em $(\z_4,\oplus,\otimes)$ \pause o conjunto $B = \{\overline{0}, \overline{2}\}$ \'e um subanel.\pause

                \vspace{.5cm}

                \item No anel $\z$, \pause o conjunto $m\z = \{mk \mid k \in \z\}$, $m > 1$ \pause {\'e} um subanel de $\z$.\pause

                \vspace{.5cm}
            \end{enumerate}
        \end{exemplos}
    \end{frame}

    \begin{frame}
        \begin{proposicao}
            Seja $(A, +,\cdot)$ um anel. \pause Um subconjunto n{\~a}o vazio \pause $B\subseteq A$ \pause {\'e} um subanel de $A$ \pause se, e somente se, \pause $x - y = x + (-y) \in B$ \pause e $x\cdot y \in B$ \pause para todos $x$, $y \in B$.\pause
        \end{proposicao}
    \end{frame}

    \begin{frame}
        \begin{definicao}
            Seja $(A, +, \cdot)$ um anel comutativo. \pause Um subconjunto n\~ao-vazio \pause $I \sub A$ \pause {\'e} chamado de \textbf{ideal} \pause de $A$ se:\pause
            \begin{enumerate}[label={\roman*})]
                \item para todos $x$, $y \in I$, \pause temos $x - y = x + (-y) \in I$;\pause
                \item para todo $\alpha \in A$ \pause e todo $x \in I$, \pause temos $\alpha\cdot x \in I$.\pause
            \end{enumerate}
        \end{definicao}

        \begin{observacao}
            Quando $I = A$ \pause ou $I = \{0_A\}$, \pause dizemos que $I$ \pause {\'e} um \textbf{ideal trivial}.\pause
        \end{observacao}
    \end{frame}

    \begin{frame}
        \begin{proposicao}
            Seja $A$ um anel comutativo \pause e $I$ um ideal de $A$. \pause Ent{\~a}o:\pause
            \begin{enumerate}[label={\roman*})]
                \item $0_{A}\in I$.\pause
                \item $-x \in I$ \pause para todo $x \in I$.\pause
                \item Se $A$ é um anel unitário e $1_A \in I$, \pause ent\~ao $I = A$.\pause
            \end{enumerate}
        \end{proposicao}
    \end{frame}

    \begin{frame}
        \begin{definicao}
            Sejam $(A, +, \cdot)$ \pause e $(B, \oplus, \otimes)$ \pause an\'eis. \pause Uma fun{\c c}{\~a}o $f : A \to B$ \pause \'e chamada de um \textbf{homomorfismo de an\'eis}, \pause ou simplesmente \textbf{homomorfismo}, \pause se satisfaz:\pause
            \begin{enumerate}[label={\roman*})]
                \item $f(x + y) \pause = f(x) \pause \oplus f(y)$\pause

                \vspace{.5cm}

                \item $f(x \cdot y) \pause = f(x) \pause \otimes f(y)$\pause

                \vspace{.5cm}
            \end{enumerate}
            para todos $x$, $y \in A$.\pause
        \end{definicao}
    \end{frame}

    \begin{frame}
        \begin{proposicao}
            Sejam $(A, +, \cdot)$ e $(B, \oplus, \otimes)$ an\'eis. \pause Se $f : A \to B$ \'e um homomorfismo, \pause ent{\~a}o:\pause
            \begin{enumerate}[label={\roman*})]
                \item $f(0_{A}) \pause = 0_{B}$\pause

                \vspace{.5cm}

                \item $f(-x) \pause = -f(x)$, \pause para todo $x \in A$.\pause
            \end{enumerate}
        \end{proposicao}
    \end{frame}

    \begin{frame}
        \begin{definicao}Seja $f:A\rightarrow B$ um homomorfismo, onde $A$ e $B$ s{\~a}o an{\'e}is. Dizemos que:
            \begin{enumerate}[label={\roman*})]
                \item $f$ {\'e} um \textbf{epimorfismo} \pause se $f$ for sobrejetora.\pause

                \vspace{.5cm}

                \item $f$ {\'e} um \textbf{monomorfismo} \pause se $f$ for injetora.\pause

                \vspace{.5cm}

                \item $f$ {\'e} um \textbf{isomorfismo} \pause se $f$ for bijetora.\pause

                \vspace{.5cm}

                \item Quando $A=B$ \pause e $f$ {\'e} um isomorfismo, \pause ent{\~a}o $f$ {\'e} um \textbf{automorfismo}.\pause

                \vspace{.5cm}

            \end{enumerate}
        \end{definicao}
    \end{frame}

    \begin{frame}
        \begin{definicao}
            Sejam $(A, +, \cdot)$ e $(B, \oplus, \otimes)$ an\'eis \pause e $f : A \to B$ um homomorfismo de an\'eis. \pause Ent\~ao o subconjunto de $A$ \pause definido por\pause
            \[
                \ker(f) = \pause N(f) =\pause \{ x \in A \pause \mid f(x) = 0_B\}\pause
            \]
            \'e chamado de \textbf{kernel} \pause ou \textbf{n\'ucleo} \pause de $f$.\pause
        \end{definicao}
    \end{frame}

    \begin{frame}
        \begin{proposicao}
            Sejam $(A, +, \cdot)$ e $(B, \oplus, \otimes)$ an\'eis \pause e $f : A \to B$ um homomorfismo de an\'eis. \pause Ent\~ao:\pause
            \begin{enumerate}[label={\roman*})]
                \item $\ker(f)$ \'e um subanel de $A$.\pause

                \vspace{.5cm}

                \item $f$ \'e injetora \pause se, e somente se, \pause $\ker(f) = \{0_A\}$.\pause

                \vspace{.5cm}
            \end{enumerate}
        \end{proposicao}
    \end{frame}

    \begin{frame}
        \begin{definicao}
            Seja $(A, +, \cdot)$ um anel unit\'ario. \pause Dado $x \in A$, \pause dizemos que $x$ \'e \textbf{invers{\'\i}vel} \pause ou que $x$ \textbf{possui inverso} \pause se existe $y \in A$ \pause tal que\pause
            \[
                x \cdot y = 1_A \pause = y \cdot x.\pause
            \]
        \end{definicao}
    \end{frame}

    \begin{frame}
        \begin{proposicao}
            Seja $(A, +, \cdot)$ um anel unit\'ario. \pause O inverso multiplicativo de um elemento $x \in A$, \pause se existir, \pause \'e \'unico.\pause
        \end{proposicao}
    \end{frame}

    \begin{frame}
        \begin{proposicao}
            Sejam $(A, +, \cdot)$ e $(B, \oplus, \otimes)$ an\'eis \pause e seja $f : A \to B$ um homomorfismo sobrejetor de an\'eis.\pause
            \begin{enumerate}[label={\roman*})]
                \item Se $A$ tem unidade, \pause ent\~ao $B$ tem unidade e\pause
                \[
                    f(1_A) = 1_B.\pause
                \]

                \vspace{.5cm}

                \item Se $A$ tem unidade \pause e $x \in A$ \pause possui inverso multiplicativo, \pause ent\~ao $f(x)$ \pause tem inverso e\pause
                \[
                    [f(x)]^{-1} = f(x^{-1}).\pause
                \]

                \vspace{.5cm}
            \end{enumerate}
        \end{proposicao}
    \end{frame}

    \begin{frame}
        Aplicação de anéis:
        \begin{enumerate}[label={\arabic*})]
            \item Códigos corretores de erros\pause

            \item Método: CRC(cyclic redundancy checks)\pause

            \item Polinômios sobre $\z_2$:\pause
                \[\z_2[x] = \{f(x) = a_0 + a_1x + a_2x^2 + \cdots + a_nx^n \pause \mid a_0, a_1, a_2, \dots, a_n \in \z_2\}\]\pause
            \item $\z_2[x]$ é um anel comutativo e unitário com a soma e a multiplicação de polinônios.\pause

            \seti
        \end{enumerate}
    \end{frame}

    \begin{frame}
        \begin{enumerate}[label={\arabic*})]
            \conti
            \item Explicação do funcionamento computacional do CRC:\\
                \begin{center}
                    \qrcode[height=5cm]{https://youtu.be/izG7qT0EpBw}
                \end{center}
        \end{enumerate}
    \end{frame}

    \begin{frame}
        \begin{enumerate}[label={\arabic*})]
            \conti
            \item Implementação do método:
                \begin{center}
                    \qrcode[height=5cm]{https://youtu.be/sNkERQlK8j8}
                \end{center}
        \end{enumerate}
    \end{frame}

    \begin{frame}
        \begin{enumerate}[label={\arabic*})]
            \item Sejam $(A, +, \cdot)$ um anel e $x \in A$ um elemento fixo. Mostre que o conjunto $N(x) = \{y \in A \mid xy = yx \}$ é um subanel de $A$.

            \vspace{2cm}

            \item Suponha que $(A, +, \cdot)$ é um anel comutativo e que $I$ e $J$ são ideais de $A$. Mostre que $I \cap J$ é um ideal de $A$.
        \end{enumerate}
    \end{frame}

\end{document}
