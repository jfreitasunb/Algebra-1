%!TEX program = xelatex
%!TEX root = Algebra_1.tex
%%Usar makeindex -s indexstyle.ist arquivo.idx no terminal para gerar o {\'\i}ndice remissivo agrupado por inicial
%%Ap\'os executar pdflatex arquivo
\chapter{Rela{\c c}{\~o}es e Fun{\c c}{\~o}es}
% \section{Rela{\c c}{\~o}es}
% \subsubsection{Defini{\c c}{\~a}o}
% Sejam A e B dois conjuntos n{\~a}o vazios. Os subconjuntos de AxB s{\~a}o chamados rela{\c c}{\~o}es, ou seja, uma rela{\c c}{\~a}o em AxB {\'e} um subconjunto desse produto cartesiano.

% Quando $R$ {\'e} uma rela{\c c}{\~a}o em $A \times B$, tamb{\'e}m dizemos que $R$ {\'e} uma rela{\c c}{\~a}o de A em B.

% Exemplos:
% \begin{enumerate}[label={\arabic*})]
% \item Se A=\{0,1\} e B=\{-1,0,1\}, ent{\~a}o AxB=\{(0,-1),(0,0),(0,1),(1,-1),(1,0),(1,1,)\}\\
% S{\~a}o exemplos de rela{\c c}{\~o}es:\\
% $R_{1}=\{(0,1)\}$\\
% $R_{2}=\emptyset$\\
% $R_{3}=\{(1,-1),(1,1)\}$\\
% $R_{4}=A$x$B$
% \item Se $A=B=\mathbb{R}$, ent{\~a}o AxB {\'e} o conjunto formado por todos pares ordenados de n{\'u}meros reais. Um exemplo de rela{\c c}{\~a}o em $\mathbb{R}$x$\mathbb{R}$ {\'e} o conjunto:\\
% $R=\{(x,y)\in \mathbb{R}$x$\mathbb{R}/ y\geq 0\}$
% \end{enumerate}

\section{Rela{\c c}{\~o}es de equival{\^e}ncia}

\begin{definicao}
	Seja $A$ um conjunto n{\~a}o vazio e $R\subseteq A \times A$. Dizemos que $R$ {\'e} uma \textbf{rela{\c c}{\~a}o de equival{\^e}ncia} se:
	\begin{enumerate}[label={\roman*})]
		\item Para todo $x \in A$, $(x,x) \in R$. \textit{(Propriedade Reflexiva)}
		\item Se $(x, y) \in R$, então $(y, x) \in R$. \textit{(Propriedade Simétrica)}
		\item Se $(x, y) \in R$ e $(y, z) \in R$, então $(x, z)\in R$. \textit{(Propriedade Transitiva)}
	\end{enumerate}
\end{definicao}

Quando $R\subseteq A \times A$ {\'e} uma rela{\c c}{\~a}o de equival{\^e}ncia, dizemos que $R$ {\'e} uma rela{\c c}{\~a}o de equival{\^e}ncia em $A$. Quando dois elementos $x$, $y \in A$ s{\~a}o tais que $(x,y) \in R$, dizemos que $x$ e $y$ \textbf{s{\~a}o relacionados} ou que $x$ e $y$ \textbf{estão relacionados}.

\begin{exemplos}\label{exemplos_relacoes_equivalencia}
	\begin{enumerate}
		\item Seja A=\{1,2,3,4\}. Temos
		\begin{align*}
			A\times A = &\{(1,1);(1,2);(1,3);(1,4);(2,1);(2,2);(2,3);(2,4);\\ &(3,1);(3,2);(3,3);(3,4);(4,1);(4,2);(4,3);(4,4)\}.
		\end{align*}
		Quais dos seguintes conjuntos são exemplos de rela{\c c}{\~o}es de equival{\^e}ncia?
		\begin{itemize}
			\item $R_{1}= A\times A$
			\item $R_{2}=\{(1,1);(2,2);(3,3)\}$
			\item $R_{3}=\{(1,1);(2,2);(3,3);(4,4);(1,2);(2,1)\}$
			\item $R_{4}=\{(1,1);(2,2);(3,3);(4,4)\}$
			\item $R_{5}=\{(1,1);(2,2);(3,3);(4,4);(1,2);(2,1);(2,4);(4;2)\}$
		\end{itemize}
		\begin{solucao}
			$R_2$ não é relação de equivalência pois $(4,4) \notin R_2$.

			$R_5$ não é relação de equivalência pois, por exemplo, $(1,4) \notin R_5$.

			Os demais são exemplos de relações de equivalência.
		\end{solucao}
		
		\item Seja $A = \z$ e $R\subseteq \z\times \z$ definida por $R = \{(x,y)\in \z \times \z \mid x = y\}$.
		Então $R$ {\'e} uma rela{\c c}{\~a}o de equival{\^e}ncia.
		\begin{solucao}
			De fato,
			\begin{itemize}
				\item Para todo $x \in \z$ temos $x = x$ daí $(x,x) \in R$.
				\item Se $(x,y)\in R$, então pela definição de $R$ temos $x = y$. Logo $y = x$ e então $(y,x)\in R$.
				\item Se $(x,y) \in R$ e $(y,z) \in R$, então  $x = y$ e $y = z$. Logo $x = z$ e assim $(x,z)\in R$ como queríamos.
			\end{itemize}
			Portanto $R$ é uma relação de equivalência sobre $\z$.
		\end{solucao}
		
		\item Seja $A = \z$ e tome $R = \{(x,y)\in \z \times \z \mid x - y = 2k, \mbox{ para algum } k \in \z\}$. Mostre que $R$
		é uma rela{\c c}{\~a}o de equival{\^e}ncia sobre $\z$.
		\begin{solucao}
			De fato,
			\begin{itemize}
				\item Para todo $x\in\z$ temos $x - x = 2\cdot0$ e com isso $(x,x) \in R$.
				\item Se $(x,y) \in R$ então existe $k \in \z$ tal que $x - y = 2k$. Agora $y - x = -(x - y) = -2k = 2 (-k)$ 
				e como $-k \in \z$ segue que $(y,x) \in R$.
				\item Se $(x,y) \in R$ e $(y,z) \in R$, então existem $k$, $l\in \z$ tais que $x - y = 2k$ e $y - z = 2l$.
				Somando essas duas equações obtemos
				\begin{align*}
					(x - y) + (y - z) &= 2k + 2l\\
					x - z &= 2(k + l)
				\end{align*}
				e como $k + l \in \z$ segue que $(x,z) \in \z)$.
			\end{itemize}
			Assim $R$ é uma relação de equivalência.
		\end{solucao}
	\end{enumerate}
\end{exemplos}
\begin{observacao}
	Seja $R$ uma rela{\c c}{\~a}o de equival{\^e}ncia em $A$. Para dizermos que $(x,y) \in R$ usaremos a nota{\c c}{\~a}o $x\equiv y\ (R)$, que se l{\^e} ``$x$ é equivalente a $y$ m{\'o}dulo $R$", ou ainda a nota{\c c}{\~a}o $xRy$, com o mesmo significado anterior.
\end{observacao}

Em virtude da observação anterior a definição de relação de equivalência pode ser reescrita como:

\begin{definicao}
	Seja $A$ um conjunto n{\~a}o vazio e $R\subseteq A \times A$. Dizemos que $R$ {\'e} uma \textbf{rela{\c c}{\~a}o de equival{\^e}ncia} se:
	\begin{enumerate}[label={\roman*})]
		\item Para todo $x \in A$, $xRx$. \textit{(Propriedade Reflexiva)}
		\item Se $xRy$, então $yRx$. \textit{(Propriedade Simétrica)}
		\item Se $xRy$ e $yRz$, então $xRz$. \textit{(Propriedade Transitiva)}
	\end{enumerate}
\end{definicao}

\begin{definicao}
	Seja $R$ uma rela{\c c}{\~a}o de equival{\^e}ncia sobre um conjunto $A$. Dado $b \in A$, chamamos de \textbf{classe de equival{\^e}ncia determinada por $b$ m{\'o}dulo $R$}, denotada por $\overline{b}$ ou $C(b)$, o subconjunto de $A$ dado por
	\[
		\overline{b} = C(b) = \{x \in A \mid (x,b) \in R\} = \{x \in A \mid xRb\}.
	\]
\end{definicao}

\begin{observacao}
	Seja $A \ne \emptyset$ e $R$ uma relação de equivalência sobre $A$. Segue da definição de relação de equivalência que para todo $b \in A$, $\overline{b} \ne \emptyset$ pois $(b,b) \in R$ logo $b \in \overline{b}$.
\end{observacao}

\begin{exemplos}\label{exemplos_classes_equivalencia}
	Do Exemplo \ref{exemplos_relacoes_equivalencia} temos
	\begin{enumerate}
		\item As classes de equivalência de $R_1$ são:
		\begin{align*}
			\overline{1} &= \{x \in A \mid (x,1) \in R_1\} = \{1,2,3,4\}\\
			\overline{2} &= \{x \in A \mid (x,2) \in R_1\} = \{1,2,3,4\}\\
			\overline{3} &= \{x \in A \mid (x,3) \in R_1\} = \{1,2,3,4\}\\
			\overline{4} &= \{x \in A \mid (x,4) \in R_1\} = \{1,2,3,4\}\\
		\end{align*}
		Nesse caso temos somente uma classe de equivalência.

		\item As classes de equivalência de $R_3$ são:
		\begin{align*}
			\overline{1} &= \{x \in A \mid (x,1) \in R_3\} = \{1,2\}\\
			\overline{2} &= \{x \in A \mid (x,2) \in R_3\} = \{1,2\}\\
			\overline{3} &= \{x \in A \mid (x,3) \in R_3\} = \{3\}\\
			\overline{4} &= \{x \in A \mid (x,4) \in R_3\} = \{4\}\\
		\end{align*}
		Aqui temos três classes de equivalência diferentes.

		\item As classes de equivalência de $R_4$ são:
		\begin{align*}
			\overline{1} &= \{x \in A \mid (x,1) \in R_4\} = \{1\}\\
			\overline{2} &= \{x \in A \mid (x,2) \in R_4\} = \{2\}\\
			\overline{3} &= \{x \in A \mid (x,3) \in R_4\} = \{3\}\\
			\overline{4} &= \{x \in A \mid (x,4) \in R_4\} = \{4\}\\
		\end{align*}
		Aqui temos quatro classes de equivalência diferentes.

		\item Para a relação de equivalência $R = \{(x,y)\in \z \times \z \mid x - y = 2k, \mbox{ para algum } k \in \z\}$ temos:
		\begin{align*}
			\overline{0} &= \{x \in \z \mid xR0 \} = \{x \in \z \mid x - 0 = 2k,\ k \in \z\} \\ 
			\overline{0} &= \{x \in \z \mid x = 2k,\ k \in \z\} = \{0, \pm 2, \pm 4, \pm 6, \dots\}\\
			\overline{1} &= \{x \in \z \mid xR1 \} = \{x \in \z \mid x - 1 = 2k,\ k \in \z\} \\
			\overline{1} &= \{x \in \z \mid x = 2k + 1,\ k \in \z\} = \{\pm 1, \pm 3, \pm 4, \pm 7, \dots\}\\
		\end{align*}
		Neste caso existem somente duas classes de equivalência. (\textit{Por quê?})
	\end{enumerate}
\end{exemplos}

\begin{proposicao}
	Seja $R$ uma rela{\c c}{\~a}o de equival{\^e}ncia em um conjunto n{\~a}o vazio $A$. Dados $a$, $b \in A$ temos:
	\begin{enumerate}
		\item se $\overline{a} \cap \overline{b} \ne \emptyset$, ent{\~a}o $aRb$.
		\item se  $\overline{a} \cap \overline{b} \neq \emptyset$, ent{\~a}o $\overline{a} = \overline{b}$.
	\end{enumerate}
\end{proposicao}
\begin{prova}
	\begin{enumerate}
		\item Como  $\overline{a} \cap \overline{b} \ne \emptyset$, existe um $y \in \overline{a} \cap \overline{b}$, logo $y \in \overline{a}$ e $y \in \overline{b}$. Da defini{\c c}{\~a}o de classe de equival{\^e}ncia temos $yRa$ e $yRb$. Como $R$ {\'e} rela{\c c}{\~a}o de equival{\^e}ncia temos $aRy$ e $bRy$. Pela propriedade transitiva segue que $aRb$, como quer{\'\i}amos.

		\item Precisamos mostrar que $\overline{a} \sub \overline{b}$ e que $\overline{b} \sub \overline{a}.$ Para a primeira inclusão seja $y \in \overline{a}$. Da{\'\i} $yRa$. Mas, por hipótese, $\overline{a}\cap\overline{b}\neq\emptyset$, assim pelo item anterior segue que $aRb$. Logo, como $yRa$ e $aRb$, segue que $yRb$, ou seja, $y \in \overline{b}$. Da{\'\i} $\overline{a}\sub\overline{b}$. Agora para provar a segunda inclusão seja $x \in \overline{b}$. Então $xRb$. Novamente, $\overline{a} \cap \overline{b} \ne \emptyset$ e então pelo item anterior segue que $aRb$. Assim uma vez que $R$ é uma relação de equivalência temos $bRa$ e de $xRb$ obtemos $xRa$, ou seja, $x \in \overline{a}$. Com isso $\overline{b} \sub \overline{a}$. Portanto $\overline{a} = \overline{b}$, como queríamos.
	\end{enumerate}
\end{prova}

\begin{corolario}
	Seja $R$ uma relação de equivalência sobre um conjunto não vazio $A$. Dados $a$, $b \in A$ então $\overline{a} \cap \overline{b} = \emptyset$ ou $\overline{a} = \overline{b}$.
\end{corolario}

\begin{definicao}
	Seja $R$ uma relação de equivalência sobre um conjunto não vazio $A$. O conjunto de todas as classes de equival{\^e}ncia determinadas por $R$ ser{\'a} denotado por $A/R$ e {\'e} chamado de \textbf{conjunto quociente} de $A$ por $R$.
\end{definicao}

\begin{exemplos}
	Do Exemplo \ref{exemplos_classes_equivalencia} temos:
	\begin{enumerate}
		\item $A/R_1 = \{\overline{1}\}$
		\item $A/R_3 = \{\overline{1},\overline{3},\overline{4}\}$
		\item $A/R_4 = \{\overline{1},\overline{2},\overline{3},\overline{4}\}$
		\item $\z/R = \{\overline{0},\overline{1}\}$
	\end{enumerate}
\end{exemplos}

\begin{definicao}
	Seja $C$ uma classe de equival{\^e}ncia de uma rela{\c c}{\~a}o de equival{\^e}ncia $R$. Qualquer elemento $y\in C$ {\'e} chamado \textbf{representante} de $C$.
\end{definicao}

\begin{proposicao}
	Seja $A$ um conjunto n{\~a}o vazio e $R$ uma rela{\c c}{\~a}o de equival{\^e}ncia em $A$. Ent{\~a}o $A$ {\'e} a uni{\~a}o disjunta das classes $\overline{b}$, $b \in A$, ou seja,
	\[
		X = \bigcup_{b\in A}\overline{b}.
	\]
\end{proposicao}
\begin{prova}
	Para todo $b\in A$ temos, pela definição de classe de equivalência, que $\overline{b}\subseteq A$. Logo $\bigcup_{b\in X}\overline{b}\subseteq X$. Agora seja $x\in A$. Logo $x \in \overline{x}$ e da{\'\i} $x\in \bigcup_{b\in A}\overline{b}$. Assim $X\subseteq\bigcup_{a\in X}\overline{a}$. Portanto, $X=\bigcup_{b\in X}\overline{b}$.
\end{prova}

Exemplo:\\
Em $\z$x$\z$ considere a seguinte rela{\c c}{\~a}o: $R=\{(a,b)\in \z$x$\z/2|(a-b)\}$. Mostre que {\'e} uma rela{\c c}{\~a}o de equival{\^e}ncia e mostre suas classes de equival{\^e}ncia.
\begin{enumerate}[label={\alph*})]
\item Dado $a\in \z$, aRa pois $2|(a-a)=0$.
\item Se aRb, ent{\~a}o $2|(a-b)$, ou seja, a-b=2k, -(a-b)=b-a=2(-k). Logo bRa.
\item Se aRb e bRc, ent{\~a}o a-b=2k e b-c=2q. Logo a-b+b-c=2k+2q=2(k+q). Logo, aRc.
\end{enumerate}

Portanto R {\'e} uma rela{\c c}{\~a}o de equival{\^e}ncia.

Dado $a\in\z$ , temos:\\
$\overline{a}=\{b\in\z/bRa\}=\{b\in\z/2|(a-b)\}$ como $2|(a-b)$, temos que:\\
$a-b=2k\Leftrightarrow b=a+2r, r=-k$

Assim, se a {\'e} {\'\i}mpar, b tamb{\'e}m o {\'e}. Logo:\\
$\overline{a}=\{...,-3,-1,1,3,...\}$

Agora, se a {\'e} par, b tamb{\'e}m {\'e}. Logo:\\
$\overline{a}=\{...,-2,0,2,4,...\}$

\section{Fun{\c c}{\~o}es}

\subsubsection{Defini{\c c}{\~a}o}
\begin{definicao}[Fun{\c c}{\~a}o] Uma fun{\c c}{\~a}o $f$ de um conjunto A em um conjunto B {\'e} uma rela{\c c}{\~a}o $f\subseteq A\times B$ satisfazendo:
\begin{enumerate}
\item $\forall x\in A,\exists y\in B/(x,y)\in f$
\item $(x_{1},y_{1}),(x_{1},y_{2})\in f \rightarrow y_{1}=y_{2}$
\end{enumerate}
\end{definicao}

Geralmente, para dizer que $f$ {\'e} uma fun{\c c}{\~a}o de A em B escrevemos $f:A\rightarrow B$.

\subsubsection{Dom{\'\i}nio e contra-dom{\'\i}nio}
O conjunto A {\'e} chamado de Dom{\'\i}nio de $f$ e o conjunto B {\'e} chamado de contra-dom{\'\i}nio.

Se $f:A\rightarrow B$ {\'e} uma fun{\c c}{\~a}o, escrevemos $f(a)=b$ para dizer que $(a,b)\in f$

Exemplos:
\begin{enumerate}
\item Sejam A=\{0,1,2,3\} e B=\{4,5,6,7,8\}. Quais das seguintes rela{\c c}{\~o}es s{\~a}o fun{\c c}{\~o}es?
\begin{itemize}
\item $R_{1}=\{(0,5),(1,6),(2,7)\}$ - N{\~a}o {\'e} fun{\c c}{\~a}o pois o n{\'u}mero 3 n{\~a}o t{\^e}m valor associado {\`a} ele.
\item $R_{2}=\{(0,4),(1,5),(1,6),(2,7),(3,8)\}$ - N{\~a}o {\'e} fun{\c c}{\~a}o pois o valor 1 tem mais de um valor diferente associado {\`a} ele.
\item $R_{3}=\{(0,4),(1,5),(2,7),(3,8)\}$ - {\'E} fun{\c c}{\~a}o
\item $R_{4}=\{(0,5),(1,5),(2,6),(3,7)\}$ - {\'E} fun{\c c}{\~a}o

\end{itemize}
\item $R_{5}=\{(x,y)\in\mathbb{R}$x$\mathbb{R}/y^{2}=x^{2}\}$ - N{\~a}o {\'e} fun{\c c}{\~a}o, pois $x=\pm \sqrt{y}$
\item $R_{6}=\{(x,y)\in\mathbb{R}$x$\mathbb{R}/x^{2}+y^{2}=1\}$ - N{\~a}o {\'e} fun{\c c}{\~a}o pois quando\\ $x=0,y=1\wedge y=-1$
\item  $R_{7}=\{(x,y)\in\mathbb{R}$x$\mathbb{R}/y=x^{2}\}$ - {\'E} fun{\c c}{\~a}o
\end{enumerate}

\subsection{Tipos de fun{\c c}{\~o}es}

\begin{definicao}[Fun{\c c}{\~a}o sobrejetora]  Uma fun{\c c}{\~a}o $f:A\rightarrow B$ {\'e} sobrejetora se, e somente se, para todo $y\in B$ exista um $x\in A$ tal que $f(x)=y$\end{definicao}

\begin{definicao}[Fun{\c c}{\~a}o injetora] Uma fun{\c c}{\~a}o $f:A\rightarrow B$ {\'e} injetora se, e somente se, para $a_{1}\neq a_{2}$, temos $f(a_{1})\neq f(a_{2}), \forall a_{1},a_{2}\in A$\end{definicao}

\begin{definicao}[Fun{\c c}{\~a}o bijetora] Uma fun{\c c}{\~a}o $f:A\rightarrow B$ que {\'e} simultaneamente injetora e sobrejetora {\'e} chamada de bijetora ou bijetiva.
\end{definicao}

Exemplos:
\begin{enumerate}
\item A fun{\c c}{\~a}o $f:\mathbb{R}\rightarrow\mathbb{R}$ dada por $f(x)=3x+1$ {\'e} injetora e sobrejetora.

Dados $x_{1}, x_{2}\in\mathbb{R}$ tais que $f(x_{1})=f(x_{2})$, temos:
\[3x_{1}+1=3x_{2}+1\]
\[x_{1}=x_{2}\]

Logo $f$ {\'e} injetora

Para verificar se $f$ {\'e} sobrejetora precisamos verificar se dado $y\in\mathbb{R}\\ \ \exists x\in\mathbb{R}/f(x)=y$.

Tome $x=\displaystyle\frac{y-1}{3}\in\mathbb{R}$. Da{\'\i}, $f(x)=y$. Logo $f$ {\'e} sobrejetora.
\item A fun{\c c}{\~a}o $f:\mathbb{R}\rightarrow\mathbb{R}$ dada por $f(x)=x^{2}$ {\'e} injetora? E sobrejetora?

N{\~a}o {\'e} injetora pois $f(-1)=f(1)\wedge 1\neq -1$

N{\~a}o {\'e} sobrejetora pois $\nexists x\in\mathbb{R}/x^{2}=-1$

\end{enumerate}

Dado $f:A\rightarrow B$ uma fun{\c c}{\~a}o, considere a rela{\c c}{\~a}o $f^{-1}\subseteq B$x$A$ tal que $(b,a)\in f^{-1}$ se $(a,b)\in f$, ou seja, $f^{-1}(b)=a$ se $f(a)=b$.

Pode ocorrer que $f^{-1}$ n{\~a}o seja fun{\c c}{\~a}o, mesmo $f$ sendo uma fun{\c c}{\~a}o. Por exemplo:

$f:\{0,1,2,3\}\rightarrow\{4,5,6,7,8\}$ dada por:\\
$f(0)=5$\\
$f(1)=5$\\
$f(2)=6$\\
$f(3)=7$

Neste caso, $f^{-1}$ {\'e} dado por:\\
$f^{-1}(5)=0$\\
$f^{-1}(5)=1$\\
$f^{-1}(6)=2$\\
$f^{-1}(7)=3$

\begin{teorema}
Dada $f:A\rightarrow B$ fun{\c c}{\~a}o tome $f^{-1}:B\rightarrow A$. Definida com o $f^{-1}(b)=a$ se $f(a)=b$. Ent{\~a}o $f^{-1}$ {\'e} uma fun{\c c}{\~a}o se, e somente se, $f$ {\'e} bijetora.
\end{teorema}

\textbf{Demonstra{\c c}{\~a}o}: Suponha $f^{-1}$ {\'e} fun{\c c}{\~a}o. Precisamos provar que $f$ {\'e} injetora e sobrejetora.

Dados $a_{1},a_{2}\in A$ tais que $f(a_{1})=b=f(a_{2})$. Como $f(a_{1})=b$ temos $f^{-1}(b)=a_{1}$, al{\'e}m disso, $f^{-1}(b)=a_{2}$. Mas $f^{-1}$ {\'e} fun{\c c}{\~a}o, da{\'\i} $a_{1}=a_{2}$, ou seja, $f$ {\'e} injetora.

Dado $b\in B$, como $f^{-1}$ {\'e} uma fun{\c c}{\~a}o, $\forall b\in B, f^{-1}(b)=a\in A$, logo $f(a)=b$ e assim $f$ {\'e} sobrejetora.

Portanto $f$ {\'e} bijetora.

Agora suponha que $f$ {\'e} bijetora.

Primeiramente, dado $b\in B$, como $f$ {\'e} sobrejetora, existe $a\in A$ tal que $f(a)=b$, ou seja, $f^{-1}(b)=a\in A$.

Suponha que $f^{-1}(b)=a_{1}$ e $f^{-1}(b)=a_{2}$. Da{\'\i}, $f(a_{1})=b\wedge f(a_{2})=b$. Mas $f$ {\'e} injetora, assim $a_{1}=a_{2}$ e ent{\~a}o $f^{-1}(b)=a_{1}=a_{2}$.

Portanto $f^{-1}$ {\'e} fun{\c c}{\~a}o. \#
\subsection{Composi{\c c}{\~a}o de fun{\c c}{\~o}es}

\subsubsection{Defini{\c c}{\~a}o}

\begin{definicao}[Fun{\c c}{\~a}o Composta] Sejam $f:A\rightarrow B$ e $g:B\rightarrow C$ fun{\c c}{\~o}es. Chama-se composta de $g$ e $f$ a fun{\c c}{\~a}o de A em C, denotada $g\circ f$, definida por $g\circ f:A\rightarrow C$.\end{definicao}

Temos ent{\~a}o que $(g\circ f)(x)=g(f(x)), \forall x\in A$.

Observa{\c c}{\~a}o: Se $f:A\rightarrow B$ e $g:B\rightarrow A$ ent{\~a}o existem $f\circ g$ e $g\circ f$. Por{\'e}m, em geral, $f\circ g\neq g\circ f$.

\subsubsection{Propriedades}
\begin{proposicao} Se $f:A\rightarrow B$ e $g:B\rightarrow C$ s{\~a}o fun{\c c}{\~o}es injetoras, ent{\~a}o $g\circ f$ {\'e} injetora.\end{proposicao}

\textbf{Demonstra{\c c}{\~a}o}: Dados $x_{1},x_{2}\in A$ tais que $(g\circ f)(x_{1})=(g\circ f)(x_{2})$ temos que $g(f(x_{1}))=g(f(x_{2}))$. Como $g$ {\'e} injetora, $f(x_{1})=f(x_{2})$. Mas $f$ {\'e} injetora, da{\'\i} $x_{1}=x_{2}$. Logo $g\circ f$ {\'e} injetora.\#

\begin{proposicao} Se $f:A\rightarrow B$ e $g:B\rightarrow C$ s{\~a}o sobrejetoras, ent{\~a}o $g\circ f$ {\'e} sobrejetora.\end{proposicao}

\textbf{Demonstra{\c c}{\~a}o}: Temos que $g\circ f:A\rightarrow C$. Dado $z\in C$. Como $g$ {\'e} sobrejetora, $\exists y\in B/g(y)=z$. Como $f$ {\'e} sobrejetora, $\exists x\in A/f(x)=y$. Assim, $z=g(y)=g(f(x))=(g\circ f)(x)$. Logo $g\circ f$ {\'e} sobrejetora.\#

\subsection{Fun{\c c}{\~a}o Identidade}
\subsubsection{Defini{\c c}{\~a}o}
\begin{definicao}[Fun{\c c}{\~a}o Identidade] Dado um conjunto $A\neq\emptyset$, a fun{\c c}{\~a}o $i_{A}:A\rightarrow A$ dada por $i_{A}(x)=(x)$ {\'e} chamada de fun{\c c}{\~a}o identidade.\end{definicao}

\begin{proposicao}
Se $f:A\rightarrow B$ {\'e} bijetora, ent{\~a}o $f\circ f^{-1}=i_{B}\wedge f^{-1}\circ f=i_{A}$.
\end{proposicao}

\textbf{Demonstra{\c c}{\~a}o}: Temos $i_{F}:F\rightarrow F$ e $i_{E}:E\rightarrow E$. Al{\'e}m disso, $f\circ f^{-1}:F\rightarrow F$ e $f^{-1}\circ f:E\rightarrow E$, da{\'\i} $D(f\circ f^{-1})=D(i_{F})$\footnote{$D(f(x))$ {\'e} o dom{\'\i}nio da fun{\c c}{\~a}o $f$} e $D(f^{-1}\circ f)=D(i_{E})$. Dado $x\in F, (f\circ f^{-1})(x)=f(f^{-1}(x))=x=i_{F}(x)$. Dado $x\in E, (f^{-1}\circ f)(x)=f^{-1}(f(x))=x=i_{E}(x)$.\#

\subsubsection{Propriedades}
\begin{proposicao} Se $f:A\rightarrow B$ e $g:B\rightarrow A$ s{\~a}o fun{\c c}{\~o}es, ent{\~a}o:
\begin{enumerate}
\item $f\circ i_{A}=f, i_{B}\circ f=f, g\circ i_{B}=g, i_{E}\circ g=g$
\item Se $g\circ f=i_{A}$, e $f\circ g=i_{B}$, ent{\~a}o $f$ e $g$ s{\~a}o bijetoras e $g=f^{-1}$
\end{enumerate}
\end{proposicao}

\textbf{Demonstra{\c c}{\~a}o}:
\begin{enumerate}
\item Provemos que $f\circ i_{A}=f$.\\
Primeiro temos $f:A\rightarrow B$ e $i_{A}:A\rightarrow A$. Da{\'\i}, $f\circ i_{A}:A\rightarrow B$, ou seja, $D(f\circ i_{A})=D(f)$. Dado $x\in A$, temos $(f\circ i_{A})(x)=f(i_{A}(x))=f(x)$. Portanto, $f\circ i_{A}=f$.
\item Provemos que $f$ {\'e} bijetora.\\
Dados $x_{1}$, $x_{2}\in B$ tais que $f(x_{1})=f(x_{2})$. Como $f:A\rightarrow B$ e $g:B\rightarrow A$, ent{\~a}o $g(f(x_{1}))=g(f(x_{2}))$, ou seja, $(g\circ f)(x_{1})=(g\circ f)(x_{2})$. Da{\'\i}, $i_{A}(x_{1})=i_{A}(x_{2})$. Logo, $x_{1}=x_{2}$, isto {\'e}, $f$ {\'e} injetora.

Agora, dado $y\in B$, segue que $y=i_{B}(y)$. Mas $i_{B}=f\circ g$. Da{\'\i}, $y=i_{B}(y)=(f\circ g)(y)=f(g(y))$. Assim, $x=g(y)\in A$ e $f(x)=y$.

Logo $f$ {\'e} sobrejetora. Portanto $f$ {\'e} bijetora. Analogamente, prova-se que g {\'e} bijetora. Provemos que $g=f^{-1}$. Temos  $f^{-1}:B\rightarrow A$, da{\'\i}, $D(g) = B = D(f^{-1})$. Agora, $f\circ g = i_{B} = f\circ f^{-1}$. Assim, para todo $x\in F$, $(f\circ g)(x)=(f\circ f^{-1})(x)$. Isto {\'e}, $f(g(x))=f(f^{-1}(x))$. Portanto, $g(x)=f^{-1}(x)\forall x\in B$. Logo, $g=f^{-1}$. \#
\end{enumerate}

\begin{definicao}
	Seja $f : A \to B$ uma fun{\c c}{\~a}o.
	\begin{enumerate}
		\item Dado $P \sub A$, chama-se \textbf{imagem direta} de $P$  \textbf{segundo} $f$ e indica-se por $f(P)$ o subconjunto de $B$ dado por
		\[
			f(P) = \{f(x) \mid x \in P\},
		\]
		isto {\'e}, $f(P)$ {\'e} o conjunto das imagens por $f$ dos elementos de $P$.

		\item Dado $Q \sub B$, chama-se \textbf{imagem inversa} de $Q$ \textbf{segundo} $f$ e indica-se por $f^{-1}(Q)$ o subconjunto de $A$ dado por
		\[
			f^{-1}(Q) = \{x \in E \mid f(x) \in Q\},
		\]
		isto {\'e}, $f^{-1}(Q)$ {\'e} o conjunto dos elementos de $A$ que tem imagem em $Q$ atrav{\'e}s de $f$.
	\end{enumerate}
\end{definicao}

{\it Exemplos:}
\begin{enumerate}
	\item Seja $A = \{1, 3, 5, 7, 9 \}$ e $B = \{0, 1, 2, 3, \dots, 10\}$ e $f : A \to B$ dada por $f(x) = x + 1$. Temos que
	\begin{itemize}
		\item $f(\{3, 5, 7\}) = \{f(3), f(5), f(7)\} = \{4, 6, 8\}$

		\item $f(A) = \{f(1), f(3), f(5), f(7), f(9)\} = \{2, 4, 6, 8, 10\}$

		\item $f(\emptyset) = \emptyset$

		\item $f^{-1}(\{2, 4, 10\}) = \{x \in A \mid f(x) \in \{2, 4, 10\}\} = \{1, 3, 9\}$

		\item $f^{-1}(\{0, 1, 3, 5, 7, 9\}) = \{x \in A \mid f(x) \in \{0, 1, 3, 5, 7, 9\}\} = \emptyset$
	\end{itemize}

	\item Sejam $A = B= \real$ e $f : \real \to \real$ dada por $f(x) = x^2$. Temos
	\begin{itemize}
		\item $f(\{1, 2, 3\}) = \{1, 4, 9\}$

		\item $f([0,2]) = \{f(x) \in \real \mid 0 \le x \le 2 \} = \{x^2 \mid 0 \le x \le 2\} = [0, 4]$

		\item $f^{-1}([1, 9]) = \{ x \in \real \mid 1 \le f(x) \le 9\} = \{x \in \real \mid 1 \le x^2 \le 9\} = [-1, -3] \cup [1, 3]$
	\end{itemize}
\end{enumerate}

\begin{proposicao}
	Seja $f : A \to B$ uma fun{\c c}{\~a}o e sejam $P$, $Q \sub E$, $X$, $Y \sub B$.
	\begin{enumerate}
		\item Se $P \sub Q$, ent{\~a}o $f(P) \sub f(Q)$.

		\item $f^{-1}(X \cup Y) = f^{-1}(X) \cup f^{-1}(Y)$.
	\end{enumerate}
\end{proposicao}
\begin{prova}
	\begin{enumerate}
		\item Se $y \in f(P)$, ent{\~a}o existe $x \in P$ tal que $f(x) = y$. Mas como $P \sub Q$, ent{\~a}o $x \in Q$ e da{\'\i} $y \in f(Q)$. Logo $f(P) \sub f(Q)$.

		\item Seja $z \in f^{-1}(X \cup Y)$. Ent{\~a}o $f(z) \in X \cup Y$. Se $f(z) \in X$, entao $z \in f^{-1}(X)$ e da{\'\i} $z \in f^{-1}(X) \cup f^{-1}(Y)$. Se $f(z) \in Y$, ent{\~a}o $z \in f^{-1}(Y)$ e assim $z \in f^{-1}(X) \cup f^{-1}(Y)$. Logo, $f^{-1}(X \cup Y) \sub f^{-1}(X) \cup f^{-1}(Y)$.

		Agora, seja $z \in f^{-1}(X) \cup f^{-1}(Y)$. Se $z \in f^{-1}(X)$, ent{\~a}o $f(z) \in X$, da{\'\i} $f(z) \in X \cup Y$, isto {\'e}, $z \in f^{-1}(X \cup Y)$. Se $z \in f^{-1}(Y)$, ent{\~a}o $f(z) \in Y$ e assim $f(z) \in X \cup Y$, isto {\'e}, $z \in f^{-1}(X \cup Y)$. Logo $f^{-1}(X) \cup f^{-1}(Y) \sub f^{-1}(X \cup Y)$.

		Portanto, $f^{-1}(X \cup Y) = f^{-1}(X) \cup f^{-1}(Y)$.
	\end{enumerate}
\end{prova}
