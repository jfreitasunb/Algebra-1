%!TEX program = xelatex
%!TEX root = Algebra_1.tex
%%Usar makeindex -s indexstyle.ist arquivo.idx no terminal para gerar o {\'\i}ndice remissivo agrupado por inicial
%%Ap\'os executar pdflatex arquivo
\chapter{Grupos}

\section{Defini{\c c}{\~a}o}
\begin{definicao}[Grupo] Um grupo $G$ {\'e} um conjunto n{\~a}o vazio munido de uma opera{\c c}{\~a}o bin{\'a}ria $*$ tal que:
\begin{enumerate}
\item Para todo $x$, $y$, $z$\in G,$ $(x*y)*z=x*(y*z)$ (Associatividade)
\item Existe $e\in G$ tal que $x*e=e*x=x$ para todo $x\in G$. Tal elemento $e$ {\'e} chamado de elemento neutro ou unidade
\item Para cada $x\in G$, existe $x^{-1}\in G$ tal que $x*x^{-1}=x^{-1}*x=e$. O elemento $x^{-1}$ {\'e} chamado de inverso\footnote{$x^{-1}\neq\displaystyle\frac{1}{x}$} de $x$.
\end{enumerate}
\end{definicao}

Denotamos um grupo $G$, cuja opera{\c c}{\~a}o bin{\'a}ria {\'e} $*$, por $(G,*)$. Quando $*$ {\'e} a soma, dizemos que $(G,*)$ {\'e} um grupo aditivo. Se $*$ {\'e} a multiplica{\c c}{\~a}o, dizemos que $(G,*)$ {\'e} um grupo multiplicativo.\\

\section{Grupo comutativo ou abeliano}
\begin{definicao}[Grupo comutativo ou abeliano] Um grupo $(G,*)$ {\'e} chamado de \textbf{grupo comutativo} ou \textbf{abeliano} quando $*$ {\'e} comutativa, ou seja, \[x*y=y*x\] para todo $x,y\in G$.
\end{definicao}
\vspace{1cm}

Exemplos:
\begin{enumerate}
\item $(\mathbb{Z},+$) {\'e} um grupo abeliano
\item Se $(A,+,.)$ {\'e} um anel, ent{\~a}o $(A,+)$ {\'e} um grupo
\item $\left(\displaystyle\frac{\mathbb{Z}}{m\mathbb{Z}},\oplus\right)$ {\'e} grupo
\item $\left(\displaystyle\frac{\mathbb{Z}}{m\mathbb{Z}}-\{\bar{0}\},\odot\right)$ {\'e} grupo?\\
$\displaystyle\frac{\mathbb{Z}}{4\mathbb{Z}}-\{\bar{0}\}=\{\bar{1},\bar{2},\bar{3}\}=G$\\
$\bar{2}\in G,\ \bar{2}\odot\bar{2}=\bar{0}\notin G$
\item $\left(U\left(\displaystyle\frac{\mathbb{Z}}{m\mathbb{Z}}\right),\odot\right)$ {\'e} um grupo\\
\item Considere o conjunto dos n{\'u}meros reais $\mathbb{R}$ com a opera{\c c}{\~a}o $*$ definida por \[x*y=x+y-3\], $x,y\in\mathbb{R}$. Ent{\~a}o $(\mathbb{R},*)$ {\'e} um grupo abeliano.

De fato
\begin{enumerate}
\item \[(x*y)*g=(x+y-3)*g=(x+y-3)+z-3\]
\[=x+(y-3+z)-3=x+(y+z-3)-3)=x*(y+z-3)\]
\[=x*(y*3)\] para todo $x,y,z\in \mathbb{R}$
\item $x*y=x+y-3=y+x-3=y*x$ para todo $x,y\in\mathbb{R}$. Logo, $*$ {\'e} comutativa
\item Para todo $x\in\mathbb{R}$, temos $x*3=x+3-3=x$. Logo, 3 {\'e} o elemento neutro de $*$.
\item Dado $x\in\mathbb{R}$, tome $x^{-1}=6-x$. Assim \[x*x^{-1}=x+(6-x)-3=3\]

Logo, para $x\in\mathbb{R}$ o inverso de $x$ por $*$ {\'e} $6-x$.

Portanto $(\mathbb{R}, *)$ {\'e} um grupo comutativo
\end{enumerate}
\item Considere um conjunto com dois elementos $G=\{x,y\}$. em $G$, considere a opera{\c c}{\~a}o $\triangle$ dada por (Tabela \ref{6})
\begin{table}[h]
   \centering 
   \setlength{\arrayrulewidth}{0,5\arrayrulewidth}
   \caption{\it Opera{\c c}{\~a}o $\triangle$}
   \begin{tabular}{|c|c|c|c|c|} 
      \hline
      $\triangle$ & $x$ & $y$ \\
     \hline
      $x$ & $x$ & $y$ \\
      \hline
      $y$ & $y$ & $x$ \\
      \hline
   \end{tabular}
\label{6}
\end{table}

$(G,\triangle)$ {\'e} um grupo?

$(x\triangle x)\triangle y= x\triangle(x\triangle y)$

$(x\triangle y)\triangle x=x\triangle(y\triangle x)$

$\vdots$

$\triangle$ {\'e} associativa

$x$ {\'e} neutro para $\triangle$\\
$x^{-1}=x$\\
$y^{-1}=y$

Logo, $(G,\triangle)$ {\'e} um grupo
\end{enumerate}

\section{Propriedades Imediatas de um grupo}

Seja $(G,*)$ um grupo. {\'E} f{\'a}cil ver que
\begin{enumerate}
\item O elemento neutro {\'e} {\'u}nico
\item Existe um {\'u}nico inverso para cada $x\in G$
\item Para todos $x,y\in G,(x*y)^{-1}=y^{-1}*x^{-1}$. Por indu{\c c}{\~a}o, $x_{1},x_{2},...,x_{n-1},x_{n}\in G$, \[(x_{1}*x_{2}*...*x_{n-1}*x_{n})^{-1}\] \[=x^{-1}_{n}*x^{-1}_{n-1}*...*x^{-1}_{2}*x^{-1}_{1}\]
\item Para todo $x\in G, (x^{-1})^{-1}=x$

\end{enumerate}

\section{Ordem de um Grupo}
\begin{definicao}[Ordem de um grupo]
Quando um grupo $(G,*)$, $G$ {\'e} um conjunto com um n{\'u}mero finito de elementos, dizemos que $G$ {\'e} um grupo finito. Denotamos por $|G|$ o n{\'u}mero de elementos de $G$ que ser{\'a} chamado de ordem de $G$ ou cardinalidade de $G$. Quando $G$ n{\~a}o {\'e} finito, dizemos que $G$ {\'e} um grupo infinito.
\end{definicao}

Exemplos:
\begin{enumerate}
\item $\left(\dfrac{\mathbb{Z}}{m\mathbb{Z}}, *\right)$ {\'e} um grupo finito para todo $m>1$
\item $(\mathbb{Z}, +)$ {\'e} um grupo infinito
\end{enumerate}

\section{Subgrupo}
\subsubsection{Defini{\c c}{\~a}o}

\begin{definicao}[Subgrupo]
Seja $(G,*)$ um grupo. Um subconjunto n{\~a}o vazio $H\subseteq G$ {\'e} um subgrupo se, e somente se, $(H,*)$ {\'e} um grupo.
\end{definicao}

\subsubsection{Propriedades}
\begin{proposicao}
Um subconjunto n{\~a}o vazio $H\subseteq G$ {\'e} um subgrupo de $G$ se, e somente se
\begin{enumerate}
\item $x^{-1}\in H,\forall x\in H$
\item $x*y\in H,\forall x,y\in H$
\end{enumerate}
\end{proposicao}

\textbf{Demonstra{\c c}{\~a}o}: Se $H$ {\'e} subgrupo, ent{\~a}o $H$ {\'e} um grupo. Logo 1 e 2 s{\~a}o satisfeitos.

Agora provemos que se $H$ satisfaz 1 e 2, ent{\~a}o $H$ {\'e} grupo.

Como $G$ {\'e} grupo, ent{\~a}o $*$ {\'e} associativo, logo $*$ {\'e} associativo em $H$.

De 1, $\forall x\in H,x^{-1}\in H$. Mas de 2, $\forall x,y\in H,\ x*y\in H$. Logo, se $x\in H$, ent{\~a}o $e=x*x^{-1}\in H$

Novamente por 1, todo elemento de $H$ possui inverso em $H$.

Logo, $(H,*)$ {\'e} um grupo.\#

Exemplos:
\begin{enumerate}
\item Dado $(G,*)$ grupo, $H=\{e\}$ e $H=G$ s{\~a}o subgrupos de $G$, chamados de subgrupos triviais
\item $(\mathbb{Z},+),\ H=m\mathbb{Z},\ m>1$

Ent{\~a}o $H$ {\'e} subgrupo de $\mathbb{Z}$
\item $G=U\left(\dfrac{\mathbb{Z}}{8\mathbb{Z}}\right)=\{\bar{1},\bar{3},\bar{5},\bar{7}\}$

$(G,\odot)$ {\'e} um grupo

$|G|$=4

$H_{1}=\{\bar{1},\bar{3}\}$ {\'e} subgrupo de G\\
$H_{2}=\{\bar{1},\bar{5}\}$ {\'e} subgrupo de G\\
$H_{3}=\{\bar{1},\bar{7}\}$ {\'e} subgrupo de G
\end{enumerate}

\section{Ordem de um subgrupo}

\begin{teorema}[Lagrange]
Seja $G$ um grupo finito. Se $H\subseteq G$ {\'e} um subgrupo, ent{\~a}o $|H|$ divide $|G|$.
\end{teorema}

Exemplo: Quais s{\~a}o as poss{\'\i}veis ordens dos subgrupos de um grupo de ordem 48?

Seja $G$ um grupo tal que $|G|=48$. Se $H$ {\'e} um subgrupo de $G$, ent{\~a}o $|H|$ divide $|G|$\\
$48=2^{4}3$ \\
$|H|=2,3,2^{2},2^{3},2^{4},2.3,2^{2}3,2^{2}3$

Observa{\c c}{\~a}o: O teorema n{\~a}o diz que haver{\'a} um subgrupo de ordem $n$ para todo $n$ tal que $n||G|$. Diz apenas que se $H$ {\'e} subgrupo de $G$, ent{\~a}o $|H|$ divide $|G|$.

\begin{corolario}
Os {\'u}nicos subgrupos de um grupo de ordem prima s{\~a}o os triviais
\end{corolario}

\textbf{Demonstra{\c c}{\~a}o}: Quando $|G|=p$ primo, temos que os {\'u}nicos divisores de $p$ positivos s{\~a}o 1 e $p$.

Ent{\~a}o, se $H$ {\'e} subgrupo de $G$, ent{\~a}o $|H|=1$ ou $|H|=p$.

Portanto, $H=\{e\}$ ou $H=G$.\#