%!TEX program = xelatex
%!TEX root = Algebra_1.tex
%%Usar makeindex -s indexstyle.ist arquivo.idx no terminal para gerar o {\'\i}ndice remissivo agrupado por inicial
%%Ap\'os executar pdflatex arquivo
\chapter{Grupos}

\section{Defini{\c c}{\~a}o}

\begin{definicao}
   Seja $A$ um conjunto n\~ao vazio. Toda fun\c{c}\~ao $f : A \times A \to A$ \'e chamada de uma \textbf{opera\c{c}\~ao bin\'aria} sobre $A$.
\end{definicao}

Nas considera\c{c}\~oes que faremos a seguir uma opere\c{c}\~ao bin\'aria $f$ sobre $A$ associa a cada par ordenado $(x, y) \in A \times A$ um elemento $f(x, y) \in A$ ser\'a denotada simplesmente por $*$. Assim escreveremos $f(x, y) = x*y$. Por exemplo a opera\c{c}\~ao $* : \n \times \n \to \n$ tal que $x*y = x^y$ est\'a bem definida pois $x^y \in \n$ sempre que $x$, $y \in \n$. Observe que esta opera\c{c}\~ao n\~ao pode ser definida em $\Z$ pois por exemplo $2^{-1} \notin \Z$. Tamb\'em n\~ao pode ser definida em $\rac$ pois $2^{1/2} \notin \rac$.

\begin{definicao}[Grupo] Um grupo $G$ {\'e} um conjunto n{\~a}o vazio munido de uma opera{\c c}{\~a}o bin{\'a}ria $*$ tal que:
\begin{enumerate}
\item Para todo $x$, $y$, $z\in G$ temos $(x*y)*z=x*(y*z)$. (Associatividade)
\item Existe $e\in G$ tal que $x*e = e*x = x$ para todo $x\in G$. Tal elemento $e$ {\'e} chamado de \textbf{elemento neutro} ou \textbf{unidade}.
\item Para cada $x\in G$, existe $x^{-1}\in G$ tal que $x*x^{-1} = x^{-1}*x = e$. O elemento $x^{-1}$ {\'e} chamado de \textbf{inverso} ou \textbf{oposto}\footnote{$x^{-1}\neq\displaystyle\frac{1}{x}$} de $x$.
\end{enumerate}
\end{definicao}

Denotamos um grupo $G$, cuja opera{\c c}{\~a}o bin{\'a}ria {\'e} $*$, por $(G,*)$. Quando $*$ {\'e} a soma, dizemos que $(G,*)$ {\'e} um grupo aditivo. Se $*$ {\'e} a multiplica{\c c}{\~a}o, dizemos que $(G,*)$ {\'e} um grupo multiplicativo.\\

\section{Grupo comutativo ou abeliano}
\begin{definicao}[Grupo comutativo ou abeliano] Um grupo $(G,*)$ {\'e} chamado de \textbf{grupo comutativo} ou \textbf{abeliano} quando $*$ {\'e} comutativa, ou seja, \[x*y=y*x\] para todo $x,y\in G$.
\end{definicao}
\vspace{1cm}

Exemplos:
\begin{enumerate}
\item $(\Z,+)$ {\'e} um grupo abeliano.
\item $(\rac,+)$ {\'e} um grupo abeliano.
\item $(\rac^*,\cdot)$ {\'e} um grupo abeliano.
\item $(\real,+)$ {\'e} um grupo abeliano.
\item $(\real^*,\cdot)$ {\'e} um grupo abeliano.
\item $(\complex,+)$ {\'e} um grupo abeliano.
\item $(\complex^*,\cdot)$ {\'e} um grupo abeliano.
\item Considere o conjunto dos n{\'u}meros reais $\mathbb{R}$ com a opera{\c c}{\~a}o $*$ definida por \[x*y=x+y-3\], $x,y\in\mathbb{R}$. Ent{\~a}o $(\mathbb{R},*)$ {\'e} um grupo abeliano.

De fato,

\begin{align*}
(x*y)*z &= (x+y-3)*z = (x+y-3)+z-3\\
&= x+(y-3+z)-3 = x+(y+z-3)-3 = x*(y+z-3)\\
&= x*(y*z)
\end{align*}
para todo $x,y,z\in \mathbb{R}$.
\item $x*y=x+y-3=y+x-3=y*x$ para todo $x,y\in\mathbb{R}$. Logo, $*$ {\'e} comutativa
\item Para todo $x\in\mathbb{R}$, temos $x*3=x+3-3=x$. Logo, 3 {\'e} o elemento neutro de $*$.
\item Dado $x\in\mathbb{R}$, tome $x^{-1}=6-x$. Assim \[x*x^{-1}=x+(6-x)-3=3\]

Logo, para $x\in\mathbb{R}$ o inverso de $x$ por $*$ {\'e} $6-x$.

Portanto $(\mathbb{R}, *)$ {\'e} um grupo comutativo.

\item $\left(\displaystyle\frac{\mathbb{Z}}{m\mathbb{Z}},\oplus\right)$ {\'e} grupo.
\item $\left(\displaystyle\frac{\mathbb{Z}}{m\mathbb{Z}}-\{\bar{0}\},\odot\right)$ {\'e} grupo?\\
$\displaystyle\frac{\mathbb{Z}}{4\mathbb{Z}}-\{\bar{0}\}=\{\bar{1},\bar{2},\bar{3}\}=G$\\
$\bar{2}\in G,\ \bar{2}\odot\bar{2}=\bar{0}\notin G$
\item $\left(U\left(\displaystyle\frac{\mathbb{Z}}{m\mathbb{Z}}\right),\odot\right)$ {\'e} um grupo\\
\end{enumerate}

\section{Propriedades Imediatas de um grupo}

Seja $(G,*)$ um grupo. {\'E} f{\'a}cil ver que
\begin{enumerate}
\item O elemento neutro {\'e} {\'u}nico
\item Existe um {\'u}nico inverso para cada $x\in G$
\item Para todos $x,y\in G,(x*y)^{-1}=y^{-1}*x^{-1}$. Por indu{\c c}{\~a}o, $x_{1},x_{2},...,x_{n-1},x_{n}\in G$, \[(x_{1}*x_{2}*...*x_{n-1}*x_{n})^{-1}\] \[=x^{-1}_{n}*x^{-1}_{n-1}*...*x^{-1}_{2}*x^{-1}_{1}\]
\item Para todo $x\in G, (x^{-1})^{-1}=x$

\end{enumerate}

\section{Ordem de um Grupo}
\begin{definicao}[Ordem de um grupo]
Quando um grupo $(G,*)$, $G$ {\'e} um conjunto com um n{\'u}mero finito de elementos, dizemos que $G$ {\'e} um grupo finito. Denotamos por $|G|$ o n{\'u}mero de elementos de $G$ que ser{\'a} chamado de ordem de $G$ ou cardinalidade de $G$. Quando $G$ n{\~a}o {\'e} finito, dizemos que $G$ {\'e} um grupo infinito.
\end{definicao}

Exemplos:
\begin{enumerate}
\item $(\Z_m, +)$ {\'e} um grupo finito para todo $m>1$.
\item $(\Z, +)$ {\'e} um grupo infinito.
\end{enumerate}

\section{Subgrupo}
\subsubsection{Defini{\c c}{\~a}o}

\begin{definicao}[Subgrupo]
Seja $(G,*)$ um grupo. Um subconjunto n{\~a}o vazio $H\subseteq G$ {\'e} um subgrupo se, e somente se, $(H,*)$ {\'e} um grupo.
\end{definicao}

\subsubsection{Propriedades}
\begin{proposicao}
Um subconjunto n{\~a}o vazio $H\subseteq G$ {\'e} um subgrupo de $G$ se, e somente se
\begin{enumerate}
\item $x^{-1}\in H,\forall x\in H$
\item $x*y\in H,\forall x,y\in H$
\end{enumerate}
\end{proposicao}

\textbf{Demonstra{\c c}{\~a}o}: Se $H$ {\'e} subgrupo, ent{\~a}o $H$ {\'e} um grupo. Logo 1 e 2 s{\~a}o satisfeitos.

Agora provemos que se $H$ satisfaz 1 e 2, ent{\~a}o $H$ {\'e} grupo.

Como $G$ {\'e} grupo, ent{\~a}o $*$ {\'e} associativo, logo $*$ {\'e} associativo em $H$.

De 1, $\forall x\in H,x^{-1}\in H$. Mas de 2, $\forall x,y\in H,\ x*y\in H$. Logo, se $x\in H$, ent{\~a}o $e=x*x^{-1}\in H$

Novamente por 1, todo elemento de $H$ possui inverso em $H$.

Logo, $(H,*)$ {\'e} um grupo.\#

Exemplos:
\begin{enumerate}
\item Dado $(G,*)$ grupo, $H=\{e\}$ e $H=G$ s{\~a}o subgrupos de $G$, chamados de subgrupos triviais
\item $(\mathbb{Z},+),\ H=m\mathbb{Z},\ m>1$

Ent{\~a}o $H$ {\'e} subgrupo de $\mathbb{Z}$
\item $G=U\left(\dfrac{\mathbb{Z}}{8\mathbb{Z}}\right)=\{\bar{1},\bar{3},\bar{5},\bar{7}\}$

$(G,\odot)$ {\'e} um grupo

$|G|$=4

$H_{1}=\{\bar{1},\bar{3}\}$ {\'e} subgrupo de G\\
$H_{2}=\{\bar{1},\bar{5}\}$ {\'e} subgrupo de G\\
$H_{3}=\{\bar{1},\bar{7}\}$ {\'e} subgrupo de G
\end{enumerate}

\section{Ordem de um subgrupo}

\begin{teorema}[Lagrange]
Seja $G$ um grupo finito. Se $H\subseteq G$ {\'e} um subgrupo, ent{\~a}o $|H|$ divide $|G|$.
\end{teorema}

Exemplo: Quais s{\~a}o as poss{\'\i}veis ordens dos subgrupos de um grupo de ordem 48?

Seja $G$ um grupo tal que $|G|=48$. Se $H$ {\'e} um subgrupo de $G$, ent{\~a}o $|H|$ divide $|G|$\\
$48=2^{4}3$ \\
$|H|=2,3,2^{2},2^{3},2^{4},2.3,2^{2}3,2^{2}3$

Observa{\c c}{\~a}o: O teorema n{\~a}o diz que haver{\'a} um subgrupo de ordem $n$ para todo $n$ tal que $n||G|$. Diz apenas que se $H$ {\'e} subgrupo de $G$, ent{\~a}o $|H|$ divide $|G|$.

\begin{corolario}
Os {\'u}nicos subgrupos de um grupo de ordem prima s{\~a}o os triviais
\end{corolario}

\textbf{Demonstra{\c c}{\~a}o}: Quando $|G|=p$ primo, temos que os {\'u}nicos divisores de $p$ positivos s{\~a}o 1 e $p$.

Ent{\~a}o, se $H$ {\'e} subgrupo de $G$, ent{\~a}o $|H|=1$ ou $|H|=p$.

Portanto, $H=\{e\}$ ou $H=G$.\#

\section{Homomorfimos de Grupos} % (fold)
\label{sec:homomorfimos_de_grupos}

Sejam $(G, *)$ e $(H, \triangle)$ grupos quaisquer. Considere uma função $f : G \to H$. Entre todas as possíveis funções entre $G$ e $H$ vamos considerar somente aquelas que satisfação a condição
\[
	f(x * y) = f(x)\triangle f(y)
\]
para todos $x$, $y \in G$, ou seja, podemos determinar a imagem de $f(x*y)$ a partir da imagem de $x$ e de $y$,

\begin{definicao}
	Dados doi grupos $(G, *)$ e $(H,\triangle)$ dizemos que uma função $f : G \to H$ é um \textbf{homomorfismo de grupos} se
	\[
		f(x * y) = f(x)\triangle f(y)
	\]
	para todos $x$, $y \in G$.
\end{definicao}

\begin{observacao}
	Sejam $(G, *)$ e $(H, \triangle)$ grupos e $f : G \to H$ um homomorfismo.
	\begin{enumerate}
		\item Se $G = H$, neste caso $f : G \to G$ é chamado de um \textbf{endomorfimos} de grupos.
		\item Se $f : G \to H$ é uma função injetora, então dizemos que $f$ é um \textbf{monomorfismo} de grupos.
		\item Se $f : G \to H$ é uma função sobrejetora, então dizemos que $f$ é um \textbf{epimorfismo} de grupos.
		\item Se $f : G \to H$ é uma função bijetora, então dizemos que $f$ é um \textbf{isomorfismo} de grupos.
		\item Se $f : G \to G$ é uma função bijetora, então dizemos que $f$ é um \textbf{automorfismo} de grupos.
	\end{enumerate}
\end{observacao}

\begin{exemplos}
	\begin{enumerate}
		\item A função $f : \Z \to \complex$ dada por $f(x) = i^x$ é um homomorfismo de $(\Z, +)$ em $(\complex, \cdot)$. De fato,
		\[
			f(x + y) = i^{x + y} = i^x\cdot i^y = f(x)\cdot f(y)
		\]
		para todos $x$, $y \in \Z$.

		\item A função $f : \real^*_+ \to \real$ dada por $f(x) = \ln(x)$ é um homomorfismo de $(\real^*_+)$ em $(\real, +)$. De fato,
		\[
			f(xy) = \ln(xy) = \ln(x) + \ln(y) = f(x) + f(y)
		\]
		para todos $x$, $y \in \real^*_+$. Além disso, como $\ln(x)$ é uma função bijetora, então $f$ é um isomorfismo de grupos.

		\item Sejam $m$ um inteiro positivo fixo. A função $f: \Z \to \Z_m$ definida por $f(x) = \overline{x}$ é um homomorfimos de $(\Z, +)$ em $(\Z_m, \oplus)$. De fato,
		\[
			f(x + y) = \overline{x + y} = \overline{x} + \overline{y} = f(x) + f(y).
		\]
		Além disso, esse homomorfismo é sobrejetor.
	\end{enumerate}
\end{exemplos}

\begin{proposicao}
	Sejam $(G, *)$ e $(H, \triangle)$ grupos e $f : G \to H$ um homomorfismo. Denote por $1_G$ e $1_H$ os elementos neutros de $G$ e $H$, respectivamente.
	\begin{enumerate}
		\item $f(1_G) = 1_H$
		\item $f(x^{-1}) = (f(x))^{-1}$ para todo $x \in G$.
	\end{enumerate}
\end{proposicao}

\begin{proposicao}
	Sejam $I$ é um subgrupo de $G$ e $f : G \to H$ um homomorfismo de grupos. Então $f(I)$ é um subgrupo de $H$.
\end{proposicao}
\begin{prova}
	Como $I$ é um subgrupo de $G$, então $1_G \in G$. Agora $f$ é um homomorfismo, logo $f(1_G) = 1_H \in f(I)$ e assim $f(I) \ne \emptyset$.

	Agora, dado $y \in f(I)$ precisamos mostrar que $y^{-1} \in f(I)$. Mas se $y \in f(I)$, então $y = f(x)$ com $x \in I$. Daí
	\[
		y^{-1} = [f(x)]^{-1} = f(x^{-1})
	\]
	e como $I$ é um subgrupo de $G$, $x^{-1} \in I$ e como isso $y^{-1} \in f(I)$.

	Finalmente, dados $y$, $z \in f(I)$ existem $x_1$, $x_2 \in I$ tais que $y = f(x_1)$ e $z = f(x_2)$. Mas $f$ é homomorfismo, daí
	\[
		y\triangle z = f(x_1)\triangle f(x_2) = f(x_1*x_2)
	\]
	e como $I$ é subgrupo, $x_1*x_2 \in I$. Logo $y\triangle z \in f(I)$.

	Portanto $f(I)$ é um subgrupo de $H$.
\end{prova}

\begin{definicao}
	Sejam $(G, *)$ e $(H, \triangle)$ grupos e $f : G \to H$ um homomorfismo de grupos. Chama-se de \textbf{núcleo} ou \textbf{kernel} de $f$ e denota-se por $N(f)$ ou $\ker(f)$ o seguinte subconjunto de $G$:
	\[
		\ker(f) = \{x \in G \mid f(x) = 1_H\}.
	\]
\end{definicao}





% section homomorfimos_de_grupos (end)

\section{Grupos de Permuta\c{c}\~ao}
Fazer a parte de $S_n$.

\section{Grupos C{\'\i}clicos}
Fazer a parte de grupos c{\'\i}clicos.