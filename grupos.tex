%!TEX program = xelatex
%!TEX root = Algebra_1.tex
%%Usar makeindex -s indexstyle.ist arquivo.idx no terminal para gerar o {\'\i}ndice remissivo agrupado por inicial
%%Ap\'os executar pdflatex arquivo
\chapter{Grupos}

\section{Primeiras Propriedades} % (fold)
\label{sec:primeiras_propriedades}

\begin{definicao}
   Seja $A$ um conjunto n\~ao vazio. Toda fun\c{c}\~ao $f : A \times A \to A$ \'e chamada de uma \textbf{opera\c{c}\~ao bin\'aria} sobre $A$.
\end{definicao}

Nas considera\c{c}\~oes que faremos a seguir uma opera\c{c}\~ao bin\'aria $f$ sobre $A$ que associa a cada par ordenado $(x, y) \in A \times A$ um elemento $f(x, y) \in A$ ser\'a denotada simplesmente por $*$. Assim escreveremos $f(x, y) = x*y$. Por exemplo a opera\c{c}\~ao $* : \n \times \n \to \n$ tal que $x*y = x^y$ est\'a bem definida pois $x^y \in \n$ sempre que $x$, $y \in \n$. Observe que esta opera\c{c}\~ao n\~ao pode ser definida em $\z$ pois por exemplo $2^{-1} \notin \z$. Tamb\'em n\~ao pode ser definida em $\rac$ pois $2^{1/2} \notin \rac$.

\begin{definicao}
	Seja $G$ um conjunto n{\~a}o vazio no qual est\'a definida uma opera{\c c}{\~a}o bin{\'a}ria $*$ tal que:
	\begin{enumerate}[label={\roman*})]
		\item Para todos $x$, $y$, $z\in G$:
		\[
			(x*y)*z=x*(y*z)
		\]
		
		\item Existe $e \in G$ tal que
		\[
			x*e = x = e*x
		\]
		para todo $x \in G$. Tal elemento $e$ {\'e} chamado de \textbf{elemento neutro} ou \textbf{unidade} de $G$.

		\item Para cada $x \in G$, existe $y \in G$ tal que
		\[
			x*y = e = y*x
		\]
		O elemento $y$ {\'e} chamado de \textbf{inverso} ou \textbf{oposto} de $x$.
	\end{enumerate}
	Nesse caso dizemos que o par $(G, *)$ \'e um \textbf{grupo}.
\end{definicao}

\begin{observacao}
	Quando $*$ {\'e} uma soma, dizemos que $(G,*)$ {\'e} um \textbf{grupo aditivo}. Se $*$ {\'e} uma multiplica{\c c}{\~a}o, dizemos que $(G,*)$ {\'e} um \textbf{grupo multiplicativo}.

	Al\'em disso, quando n\~ao houver chance de confus\~ao com rela\c{c}\~ao \`a opera\c{c}\~ao do grupo $(G, *)$ vamos dizer simplesmente que $G$ \'e um grupo.
\end{observacao}

\begin{definicao}
	Um grupo $(G,*)$ {\'e} chamado de \textbf{grupo comutativo} ou \textbf{abeliano} quando $*$ {\'e} comutativa, ou seja, quando
	\[
		x*y = y*x
	\]
	para todos $x$, $y \in G$.
\end{definicao}

\begin{exemplos}
	\begin{enumerate}[label={\arabic*})]
		\item $(\z,+)$ {\'e} um grupo abeliano.
		\item $(\rac,+)$ {\'e} um grupo abeliano.
		\item $(\rac^*,\cdot)$ {\'e} um grupo abeliano.
		\item $(\real,+)$ {\'e} um grupo abeliano.
		\item $(\real^*,\cdot)$ {\'e} um grupo abeliano.
		\item Considere o conjunto dos n{\'u}meros reais $\mathbb{R}$ com a opera{\c c}{\~a}o $*$ definida por
		\[
			x*y = x + y - 3
		\]
		para $x$, $y \in \mathbb{R}$. Ent{\~a}o $(\mathbb{R}, *)$ {\'e} um grupo abeliano.
		\begin{solucao}
			De fato,
			\begin{enumerate}[label={\roman*})]
				\item Para todos $x$, $y$, $z \in \real$
				\begin{align*}
					(x*y)*z &= (x+y-3)*z = (x+y-3)+z-3\\
					&= x+(y-3+z)-3 = x+(y+z-3)-3 = x*(y+z-3)\\
					&= x*(y*z)
				\end{align*}

				\item Para todo $x \in \mathbb{R}$, temos $x*3 = x + 3 - 3 = x = 3 * x$. Logo, 3 {\'e} o elemento neutro de $*$.

				\item Dado $x \in \mathbb{R}$, tome $y = 6 - x \in \real$. Assim
				\[
					x*y = x + (6-x)-3 = 3 = y*x.
				\]
				Assim $y = 6 - x$ \'e o oposto de $x$ na opera\c{c}\~ao $*$ definida em $\real$.
			\end{enumerate}

			Portanto $(\real, *)$ \'e um grupo.

			Al\'em disso, para todos $x$, $y \in \real$
			\[
				x*y = x + y - 3 = y + x - 3 = y*x
			\]
			Logo, $(\real, *)$ {\'e} um grupo comutativo.
		\end{solucao}

		\item $(\z_m,\oplus)$ {\'e} grupo.

		\item $(\z_m-\{\overline{0}\},\otimes)$ {\'e} grupo?
		\begin{solucao}
			N\~ao, pois por exemplo, para $m = 4$ temos $\z_4-\{\overline{0}\} = \{\overline{1}, \overline{2}, \overline{3}\} = G$ e tomando $\overline{2}\in G$ temos $\overline{2} \otimes \overline{2} = \overline{0} \notin G$. Portanto a opera\c{c}\~ao $\otimes$ n\~ao \'e uma opera\c{c}\~ao bin\'aria em $G = \z_4 - \{\overline{0}\}$.
		\end{solucao}
	\end{enumerate}
\end{exemplos}

\begin{proposicao}
	Seja $(G,*)$ um grupo. Ent\~ao:
	\begin{enumerate}[label={\roman*})]
		\item O elemento neutro de $G$ {\'e} {\'u}nico.

		\item Existe um {\'u}nico inverso para cada $x \in G$.

		\item Para todos $x$, $y \in G$,
		\[
			(x*y)^{-1} = y^{-1}*x^{-1}
		\]
		Por indu{\c c}{\~a}o, $x_1$, $x_2$, \dots ,$x_{n-1}$, $x_n \in G$,
		\[
			(x_1*x_2*\cdots *x_{n-1}*x_{n})^{-1} = x^{-1}_{n}*x^{-1}_{n-1}*\cdots *x^{-1}_2*x^{-1}_1
		\]
		\item Para todo $x \in G$, $(x^{-1})^{-1} = x$.
	\end{enumerate}
\end{proposicao}


\section{Grupo Sim\'etrico} % (fold)
\label{sec:grupo_sim\'etrico}

Seja $A$ um conjunto n\~ao vazio. Dada uma fun\c{c}\~ao $f : A \to A$, sabemos que $f$ possui inversa se, e somente se, $f$ \'e bijetora, Teorema \ref{teorema_funcao_inversa}. Assim considere o conjunto
\[
	\mathcal{S} = \{ f : A \to A \mid f \mbox{ \'e bijetora}\}
\]
com a composi\c{c}\~ao de fun\c{c}\~oes $\circ$. Como $Id : A \to A$ tal que $Id(x) = x$ para todo $x \in A$ \'e uma fun\c{c}\~ao bijetora ent\~ao $\mathcal{S} \ne \emptyset$. Agora sejam $f$, $g$ e $h \in \mathcal{S}$. Para todo $x \in A$ temos
\begin{align*}
	[(f\circ g)\circ h](x) &= (f \circ g)(h(x)) = f(g(h(x)))\\
	[f\circ(g\circ h)](x) &= f((g\circ h)(x)) = f(g(h(x)))
\end{align*}
Logo $(f\circ g)\circ h = f\circ(g\circ h)$.

Agora da Proposi\c{c}\~ao \ref{propriedades_identidade} sabemos que para toda $f \in \mathcal{S}$
\[
	f\circ Id = f = Id\circ f,
\]
logo $Id$ \'e o elemento neutro da composi\c{c}\~ao. Al\'em disso, para toda $f \in \mathcal{S}$ existe $g \in \mathcal{S}$ tal que
\[
	f\circ g = Id = g \circ f
\]
pois $f$ \'e bijetora. Logo todo elemento de $\mathcal{S}$ possui inverso.

Portanto $(\mathcal{S}, \circ)$ \'e um grupo. Al\'em disso, em geral, esse grupo n\~ao \'e comutativo.

Vamos considerar agora o caso particular em que $A \sub \n$ \'e um conjunto finito. Estamos considerando $A \sub \n$ somente para simplificar a nota\c{c}\~ao, poder{\'\i}amos fazer a abordagem seguinte para qualquer conjunto finito.

Se $A = \{1\}$, ent\~ao s\'o existe uma fun\c{c}\~ao $f : A \to A$ que \'e bijetora e essa fun\c{c}\~ao \'e a identidade. Nesse caso $\mathcal{S} = S_1 = \{Id\}$ e $(S_1, \circ)$ \'e um grupo, e nesse caso comutativo.


Se $A = \{1, 2\}$ ent\~ao podemos definir as seguintes fun\c{c}\~oes bijetoras em $A$:
\begin{multicols}{2}
	\begin{enumerate}
		\item[] \begin{align*}
			Id : A &\to A\\ Id(1) &= 1\\ Id(2) &= 2
		\end{align*}
		\item[]  \begin{align*}
			f : A &\to A\\ f(1) &= 2\\ f(2) &= 1
		\end{align*}
	\end{enumerate}
\end{multicols}

Assim $\mathcal{S} = S_2 = \{Id, f\}$ e $(S_2, \circ)$ \'e um grupo.
\begin{table}[!htb]
\centering
	\begin{tabular}{|c|c|c|} 
	    \hline
	    $\circ$ & $Id$ & $f$\T\\
	    \hline
	    $Id$ & $Id$ & $f$\T\\
	    \hline
	    $f$ & $f$ & $Id$\T\\
	    \hline
	\end{tabular}
\end{table}

Al\'em disso, da tabela acima vemos que esse grupo \'e comutativo.

Agora seja $A = \{1, 2, 3\}$. Podemos definir ent\~ao as seguintes fun\c{c}\~oes bijetoras em $A$:
\begin{multicols}{3}
	\begin{enumerate}
		\item[] \begin{align*}
			Id : A &\to A\\
			Id(1) &= 1\\
			Id(2) &= 2\\
			Id(3) &= 3
		\end{align*}
		\item[] \begin{align*}
			f_1 : A &\to A\\
			f_1(1) &= 2\\
			f_1(2) &= 1\\
			f_1(3) &= 3
		\end{align*}
		\item[] \begin{align*}
			f_2 : A &\to A\\
			f_2(1) &= 3\\
			f_2(2) &= 2\\
			f_2(3) &= 1
		\end{align*}
		\item[] \begin{align*}
			f_3 : A &\to A\\
			f_3(1) &= 1\\
			f_3(2) &= 3\\
			f_3(3) &= 2
		\end{align*}
		\item[] \begin{align*}
			f_4 : A &\to A\\
			f_4(1) &= 2\\
			f_4(2) &= 3\\
			f_4(3) &= 1
		\end{align*}
		\item[] \begin{align*}
			f_5 : A &\to A\\
			f_5(1) &= 3\\
			f_5(2) &= 1\\
			f_5(3) &= 2
		\end{align*}
	\end{enumerate}
\end{multicols}

Logo $\mathcal{S} = S_3 = \{Id, f_1, f_2, f_3, f_4, f_5\}$ e $(S_3, \circ)$ \'e um grupo. Nesse caso temos
\begin{align*}
	(f_1 \circ f_4)(1) &= f_1(f_4(1)) = f_1(2) = 1\\
	(f_4 \circ f_1)(1) &= f_4(f_1(1)) = f_4(2) = 3
\end{align*}
da{\'\i} $(f_1 \circ f_4)(1) \ne (f_4 \circ f_1)(1)$, isto \'e, $f_1 \circ f_4 \ne f_4 \circ f_1$. Portanto o grupo $(S_3, \circ)$ n\~ao \'e comutativo.

Note que em $S_2$ temos $2 = 2!$ elementos e em $S_3$ temos $6 = 3!$ elementos.

De modo geral, se $A = \{1, 2, 3, \dots, n\}$ ent\~ao existem exatamente $n!$ fun\c{c}\~oes $f : A \to A$ bijetoras. Assim o grupo $(S_n, \circ)$ possui $n!$ elementos e se $n \geqslant 3$ $S_n$ \'e um grupo n\~ao comutativo.

\begin{definicao}
	O grupo $S_n$ \'e chamado de \textbf{grupo sim\'etrico} ou \textbf{grupo de permuta\c{c}\~oes} em $A = \{1, 2, 3, \dots, n\}$.
\end{definicao}


Um modo de representar os elementos de $S_n$ \'e o seguinte: vamos representar as fun\c{c}\~oes $f \in S_n$ na forma de uma matriz contendo 2 linhas e $n$ colunas. A primeira linha \'e o dom{\'\i}nio da fun\c{c}\~ao e a segunda cont\'em suas imagens. Assim se $f \in S_n$ escreveremos
\[
	f = \begin{pmatrix}
		1 & 2 & 3 & \dots & n\\
		f(1) & f(2) & f(3) & \dots & f(n)
	\end{pmatrix}.
\]

No caso de $S_3$ vamos escrever
\begin{multicols}{3}
	\begin{enumerate}
		\item[] $Id = \begin{pmatrix}
			1 & 2 & 3\\
			1 & 2 & 3
		\end{pmatrix}$
		\item[] $f_1 = \begin{pmatrix}
			1 & 2 & 3\\
			2 & 1 & 3
		\end{pmatrix}$
		\item[] $f_2 = \begin{pmatrix}
			1 & 2 & 3\\
			3 & 2 & 1
		\end{pmatrix}$
		\item[] $f_3 = \begin{pmatrix}
			1 & 2 & 3\\
			1 & 3 & 2
		\end{pmatrix}$
		\item[] $f_4 = \begin{pmatrix}
			1 & 2 & 3\\
			2 & 3 & 1
		\end{pmatrix}$
		\item[] $f_5 = \begin{pmatrix}
			1 & 2 & 3\\
			3 & 1 & 2
		\end{pmatrix}$
	\end{enumerate}
\end{multicols}
e da{\'\i}, por exemplo,
\[
	f_3\circ f_4 = \begin{pmatrix}
			1 & 2 & 3\\
			1 & 3 & 2
		\end{pmatrix} \circ \begin{pmatrix}
			1 & 2 & 3\\
			2 & 3 & 1
		\end{pmatrix} = \begin{pmatrix}
			1 & 2 & 3\\
			3 & 2 & 1
		\end{pmatrix} = f_2.
\]

% section grupo_sim\'etrico (end)


\begin{definicao}
	Seja $(G,*)$ um grupo. Se $G$ {\'e} um conjunto com uma quantidade finita de elementos, dizemos que $G$ {\'e} um \textbf{grupo finito}. Denotamos por $|G|$ o n{\'u}mero de elementos de $G$ e que ser{\'a} chamado de \textbf{ordem} de $G$ ou \textbf{cardinalidade} de $G$. Quando o conjunto $G$ n{\~a}o {\'e} finito, dizemos que $G$ {\'e} um \textbf{grupo infinito}.
\end{definicao}

\begin{exemplos}
	\begin{enumerate}[label={\arabic*})]
		\item $(\z_m, +)$ {\'e} um grupo finito para todo $m>1$.
		\item $(S_n, \circ)$ \'e um grupo finito com $n!$ elementos.
		\item $(\z, +)$ {\'e} um grupo infinito.
	\end{enumerate}	
\end{exemplos}

\section{Subgrupos} % (fold)
\label{sec:subgrupos}

\begin{definicao}
	Seja $(G,*)$ um grupo. Um subconjunto n{\~a}o vazio $H\sub G$ {\'e} chamado de \textbf{subgrupo} de $G$ se, e somente se, $(H,*)$ {\'e} um grupo.
\end{definicao}

\begin{proposicao}
	Seja $G$ um grupo. Um subconjunto n{\~a}o vazio $H\subseteq G$ {\'e} um subgrupo de $G$ se, e somente se
	\begin{enumerate}[label={\roman*})]
		\item\label{subgrupo_condicao_1} $x^{-1}\in H$, para todo $x \in H$;
		\item\label{subgrupo_condicao_2} $x*y\in H$, para todos $x$, $y \in H$.
	\end{enumerate}
\end{proposicao}
\begin{prova}
	Se $H$ {\'e} subgrupo, ent{\~a}o $H$ {\'e} um grupo. Logo \ref{subgrupo_condicao_1} e \ref{subgrupo_condicao_2} s{\~a}o satisfeitos.

	Agora provemos que se $H$ satisfaz \ref{subgrupo_condicao_1} e \ref{subgrupo_condicao_2}, ent{\~a}o $H$ {\'e} grupo.

	Como $G$ {\'e} grupo, ent{\~a}o $*$ {\'e} associativa, logo $*$ {\'e} associativa em $H$.

	De \ref{subgrupo_condicao_1}, para todo $x \in H$, $x^{-1}\in H$. Mas de \ref{subgrupo_condicao_2}, para todos $x$, $y \in H,$ $x*y \in H$. Logo, se $x\in H$, ent{\~a}o $e = x*x^{-1} \in H$.

	Novamente por \ref{subgrupo_condicao_1}, todo elemento de $H$ possui inverso em $H$.

	Portanto, $(H,*)$ {\'e} um grupo.
\end{prova}


\begin{exemplos}
	\begin{enumerate}[label={\arabic*})]
		\item Dado $(G,*)$ grupo, $H=\{e\}$ e $H=G$ s{\~a}o subgrupos de $G$, chamados de \textbf{subgrupos triviais}.
		
		\item Seja $(\mathbb{Z},+)$ um grupo. Tomando $H = m\z$, onde $m > 1$, ent{\~a}o $H$ {\'e} subgrupo de $\z$.
		
		\item $G = U(\z_8) = \{\overline{1}, \overline{3}, \overline{5}, \overline{7}\}$. Ent\~ao $(G,\odot)$ {\'e} um grupo com $|G| = 4$. Al\'em disso,
		\begin{align*}
			H_1 &= \{\overline{1}, \overline{3}\}\\
			H_2 &= \{\overline{1}, \overline{5}\}\\
			H_3 &= \{\overline{1}, \overline{7}\}
		\end{align*}
		S\~ao subgrupos de $G$.

		\item Considere o grupo aditivo $M_2(\real)$. Então o conjunto
        \[
            H = \left\{\begin{pmatrix}
                a & b\\c & d
            \end{pmatrix} \in M_2(\real) \mid a + d = 0\right\} 
        \]
        \'e um subgrupo de $M_2(\real)$.
	\end{enumerate}
\end{exemplos}

Seja $(G, *)$ um grupo. Para simplificar a escrita vamos adotar uma nota\c{c}\~ao multiplicativa e escrever $(G, *) = (G, \cdot)$. Assim, dados $x$, $y \in G$ vamos denotar
\[
    x * y = x \cdot y = xy.
\]

Nesse caso vamos dizer simplesmente que $G$ \'e um grupo.

\begin{proposicao}\label{proposicao_subgrupo_gerado}
    Seja $G$ um grupo. Dado $H \subset G$ um subgrupo defina
    \begin{align}\label{relacao_equivalencia_subgrupo}
        x \sim y \mbox{ se, e somente se, } x^{-1}y \in H
    \end{align}
    para todos $x$, $y \in G$.
    \begin{enumerate}[label={\roman*})]
        \item A rela\c{c}\~ao definida em \eqref{relacao_equivalencia_subgrupo} \'e uma rela\c{c}\~ao de equival\^encia.

        \item Se $a \in G$, ent\~ao a classe de equival\^encia determinada por $a$ \'e o conjunto
        \begin{align*}\label{classe_equivalencia_subgrupo}
            aH = \{al \mid l \in H\}.
        \end{align*}
    \end{enumerate}
\end{proposicao}
\begin{prova}
	\begin{enumerate}[label={\roman*})]
		\item Precisamos mostrar que a relação $\sim$ definida acima satisfaz a Definição \eqref{definicao_relacao_equivalencia}.

		Denote por $e$ o elemento neutro do grupo $G$.

		Primeiro, como $H$ é subgrupo de $G$, então $e \in H$. Mas
		\begin{align*}
			e = x^{-1}x
		\end{align*}
		para todo $x \in G$. Logo $x \sim x$, como queríamos.

		Suponha que $x \sim y$. Daí
		\[
			x^{-1}y \in H.
		\]
		Isto é,
		\[
			x^{-1}y = l
		\]
		onde $l \in H$. Mas $H$ é subgrupo e $l \in H$, então $l^{-1} \in H$. Agora
		\begin{align*}
			l^{-1} = (x^{-1}y)^{-1} = y^{-1}(x^{-1})^{-1} = y^{-1}x,
		\end{align*}
		isto é, $y^{-1}x \in H$. Com isso, $y \sim x$.

		Finalmente, suponha que $x \sim y$ e $y \sim z$. Daí
		\begin{align*}
			x^{-1}y &\in H\\
			y^{-1}z &\in H
		\end{align*}
		e então como $H$ é subgrupo de $G$ devemos ter
		\begin{align*}
			(x^{1}y)(y^{-1}z) &\in H\\
			x^{-1}(yy^{-1})z &\in H\\
			x^{-1}z &\in H.
		\end{align*}

		Ou seja, $x \sim z$.

		Portanto, $\sim$ é uma relação de equivalência sobre $G$.

		\item Seja $a \in G$. Agora, por definição a classe de equivalência de $a$ é dada por
		\[
			\overline{a} = \{ x \in G \mid x \sim a\}.
		\]
		Queremos mostrar que $\overline{a} = aH$, onde
		\[
			aH = \{al \mid l \in H\}.
		\]

		Seja $x \in \overline{a}$. Assim $x \sim a$, isto é, $x^{-1}a \in H$. Logo existe $l \in H$ tal que
		\[
			x^{-1}a = l.
		\]
		Mas então $x = al^{-1}$. Com isso $x \in aH$, uma vez que $H$ é subgrupo e $l^{-1} \in H$.

		Agora seja $y \in aH$. Logo existe $t \in H$ tal que
		\[
			y = at.
		\]
		Então
		\[
			ya^{-1} = t \in H.
		\]
		Logo $a \in y$, ou seja, $yx \in \overline{a}$.

		Portanto $\overline{a} = aH$, como queríamos.
	\end{enumerate}
\end{prova}

\begin{proposicao}
    Seja $H$ um subgrupo de um grupo $G$. Ent\~ao duas classes laterais quaisquer m\'odulo $H$ s\~ao subconjuntos de $G$ que possuem a mesma cardinalidade, isto \'e, a mesma quantidade de elementos.
\end{proposicao}
\begin{prova}
	Seja $H$ um subgrupo de um grupo $G$. Dados $a$, $b \in G$ para mostrar que $aH$ e $bH$ possuem a mesma cardinalidade vamos mostrar que sempre é possível definir uma função bijetora entre esses conjuntos, quaisquer que forem $a$ e $b \in G$.

	Para isso, defina $f : aH \to bH$ por $f(al) = bl$, para $l \in H$. Mostremos que $f$ é bijetora, isto é, que $f$ é injetora e sobrejetora.

	Para mostrar que $f$ é injetora, sejam $al_1$, $al_2 \in aH$ tais que
	\[
		f(al_1) = f(al_2).
	\]
	Daí
	\begin{align*}
		bl_1 &= bl_2\\
		b^{-1}(bl_1) &= b^{-1}(bl_2)\\
		l_1 &= l_2,
	\end{align*}
	com isso $al_1 = al_2$. Logo $f$ é injetora.

	Agora, seja $bt \in bH$. Tome $at \in aH$ e assim
	\[
		f(at) = bt,
	\]
	isto é, $f$ é sobrejetora.

	Portanto $f$ é bijetora e com isso $aH$ e $bH$ têm a mesma cardinalidade, como queríamos.
\end{prova}

\begin{observacao}
	Da proposição anterior, sabemos que duas classes de equivalência posssuem sempre a mesma cardinalidade. Agora, tomando $e \in G$, o elemento neutro, temos
	\[
		eH = \{el \mid l \in H\} = H.
	\]

	Assim a cardinalidade de $aH$ é igual a cardinalidade de $eH = H$, independente de $a \in G$. Ou seja, a cardinalidade de qualquer classe de equivalência é sempre igual à cardinalidade de $H$.
\end{observacao}

\begin{definicao}
	Para cada $a \in G$, a classe de equivalência $aH$ definida pela relação de equivalência \eqref{relacao_equivalencia_subgrupo} é chamada de \textbf{classe lateral à direita, módulo $H$}, determinada por $a$.
\end{definicao}

\begin{exemplos}
	\begin{exemplos}
        \begin{enumerate}[label={\arabic*})]
            \item No grupo multiplicativo $G = \{1, -1, i, -i\}$, onde $i^2 = -1$. Considere o conjunto $H = \{1, -1\}$. Ent\~ao $H$ \'e um sugbrupo de $G$ e as classes laterais ser\~ao:
            \begin{align*}
            	1H &= H = \{1, -1\}\\
            	iH &= \{il \mid l \in H\} = \{i, -i\}.
            \end{align*}

            \item Considere o grupo multiplicativo $\real^*$ e $H = \{ x \in \real^* \mid x > 0\} \subset \real^*$. Ent\~ao $H$ \'e subgrupo de $\real^*$ e as classes laterais ser\~ao:
            \begin{align*}
            	1H &= H = \{x \in \real^* \mid x > 0\}\\
            	aH &= \{al \mid l \in H\}.
            \end{align*}

            Se $a > 0$, então $al > 0$ para todo $l \in H$ e com isso $al \in H$. Logo
            \[
            	aH = H
            \]
            para todo $a > 0$.

            Se $a < 0$, então $al < 0$ para todo $l \in H$. Logo
            \[
            	aH = \{x \in \real^* \mid x < 0\}.
            \]

            Com isso existem somente duas classes laterais que são: $H$ e $aH$, para $a < 0$.

            \item Considere agora o grupo sim\'etrico $G = S_3$. Denote por
                \[
                    a = \begin{pmatrix}
                        1 & 2 & 3\\2 & 3 & 1
                    \end{pmatrix}, \quad
                    b = \begin{pmatrix}
                        1 & 2 & 3\\1 & 3 & 2
                    \end{pmatrix}.
                \]
                Fica como exerc{\'\i}cio verificar que $\{e, a, a^2 , b, ba, ba^2\} = S_3$. Aqui $e$ \'e a fun\c{c}\~ao identidade, $a^2 = a \circ a$, $ba = b \circ a$ e $ba^2 = b\circ(a\circ a)$. Seja $H = \{e, a , a^2\}$. Ent\~ao $H$ \'e subgrupo de $S_3$ e as classes laterais ser\~ao:
                \begin{align*}
                	eH &= H\\
                	bH &= \{bl \mid l \in H\} = \{b, ba, ba^2\}.
                \end{align*}

                Logo existem somente duas classes laterais que são $H$ e $bH$.
        \end{enumerate}
    \end{exemplos}
\end{exemplos}


\section{Grupos Cíclicos}

Seja $(G, *)$ um grupo.

Caso a opera\c{c}\~ao $*$ seja do tipo multiplicativa, vamos escrever $(G, *) = (G, \cdot)$. Assim, dados $x$, $y \in G$ vamos denotar
\[
    x * y = x \cdot y = xy.
\]

Com a nota\c{c}\~ao multiplicativa o inverso de um elemento $x \in G$ ser\'a denotado por $x^{-1}$.

\begin{definicao}
	Seja $G$ um grupo multiplicativo e denote por $e$ o elemento neutro de $G$. Se $x \in G$ e $m \in \z$, a \textbf{pot\^encia $m$-\'esima} de $x$, ou \textbf{pot\^encia de $x$ de expoente $m$}, \'e o elemento de $G$ denotado por
        \[
            x^m
        \]
        e definido por:
        \[
            x^m = \begin{cases}
                    e, & \mbox{se m = 0},\\
                    x^{m-1}x, & \mbox{ se } m \ge 1,\\
                    (x^{-m})^{-1}, & \mbox{ se } m < 0. 
                   \end{cases}
        \]
\end{definicao}

\begin{exemplos}
    \begin{enumerate}[label={\arabic*})]
        \item No grupo multiplicativo $GL_2(\real)$ seja
        \[
            A = \begin{pmatrix}
                1 & 1\\2 & 3
            \end{pmatrix}.
        \]
        Ent\~ao:
        \begin{align*}
        	A^0 &= \begin{pmatrix}
        		1 & 0\\
        		0 & 1
        	\end{pmatrix}\\
        	A^1 &= A\\
        	A^2 &= A\cdot A = \begin{pmatrix}
        		1 & 1\\
        		2 & 3
        	\end{pmatrix}\cdot \begin{pmatrix}
        		1 & 1\\
        		2 & 3
        	\end{pmatrix} = \begin{pmatrix}
        		3 & 4\\
        		8 & 11
        	\end{pmatrix}\\
        	A^{-1} &= \begin{pmatrix}
        		3 & -1\\
        		-2 & 1
        	\end{pmatrix}\\
        	A^{-2} &= (A^2)^{-1}\begin{pmatrix}
        		11 & -4\\
        		-8 & 3
        	\end{pmatrix}\\
        \end{align*}
        e podemos calcular $A^n$ para todo $n \in \z$.

        \item No grupo multiplicativo $\z_5^*$ seja $a = \overline{2}$. Ent\~ao:
        \begin{align*}
        	\overline{2}^0 &= \overline{1}\\
        	\overline{2}^1 &= \overline{2}\\
        	\overline{2}^2 &= \overline{2}\odot \overline{2} = \overline{4}\\
        	\overline{2}^3 &= (\overline{2})^2\odot \overline{2} = \overline{3}\\
        	\overline{2}^4 &= (\overline{2})^3\odot \overline{2} = \overline{1}\\
        	\overline{2}^5 &= (\overline{2})^4\odot \overline{2} = \overline{2}\\
        \end{align*}
        e para $m \ge 5$ os valores se repetem.

        Agora,
        \begin{align*}
        	\overline{2}^{-1} &= \overline{3}\\
        	\overline{2}^{-2} &= (\overline{2}^2)^{-1} = \overline{4}^{-1} = \overline{4}\\
        	\overline{2}^{-3} &= (\overline{2}^3)^{-1} = \overline{3}^{-1} = \overline{2}\\
        	\overline{2}^{-3} &= (\overline{2}^4)^{-1} = \overline{1}^{-1} = \overline{1}\\
        	\overline{2}^{-5} &= (\overline{2}^5)^{-1} = \overline{2}^{-1} = \overline{3}\\
        \end{align*}
        e para $m \le -5$ os valores se repetem.

        \item No grupo multiplicativo $S_3$ seja
        \[
            a = \begin{pmatrix}
                1 & 2 & 3\\ 2 & 3 & 1
            \end{pmatrix}.
        \]
        Ent\~ao:
        \begin{align*}
        	a^0 &= e\\
        	a^1 &= a\\
        	a^2 &= \begin{pmatrix}
        		1 & 2 & 3\\
        		2 & 3 & 1
        	\end{pmatrix} \circ \begin{pmatrix}
        		1 & 2 & 3\\
        		2 & 3 & 1
        	\end{pmatrix} = \begin{pmatrix}
        		1 & 2 & 3\\
        		3 & 1 & 2
        	\end{pmatrix}\\
        	a^3 &= a^2 \circ a = \begin{pmatrix}
        		1 & 2 & 3\\
        		3 & 1 & 2
        	\end{pmatrix} \circ \begin{pmatrix}
        		1 & 2 & 3\\
        		2 & 3 & 1
        	\end{pmatrix} = \begin{pmatrix}
        		1 & 2 & 3\\
        		1 & 2 & 3
        	\end{pmatrix}
        \end{align*}
        e para $m \ge 4$ os valores se repetem.

        Agora,
        \begin{align*}
        	a^{-1} &= a^2\\
        	a^{-2} &= (a^2)^{-1} = a\\
        	a^{-3} &= (a^3)^{-1} = e^{-1} = e\\
        \end{align*}
        e para $m \le -4$ os valores se repetem.
    \end{enumerate}
\end{exemplos}

\begin{proposicao}
	Seja $G$ um grupo multiplicativo. Se $m$ e $n$ s\~ao n\'umeros inteiros e $x \in G$, ent\~ao
	\begin{enumerate}[label={\roman*})]
	    \item $x^mx^n = x^{m + n}$

	    \item $x^{-m} = (x^m)^{-1}$

	    \item $(x^m)^n = x^{mn}$

	    \item $x^mx^n = x^nx^m$.
	\end{enumerate}
\end{proposicao}

Seja $(G, *)$ um grupo.

Caso a opera\c{c}\~ao $*$ seja do tipo aditiva, vamos escrever $(G, *) = (G, +)$. Assim, dados $x$, $y \in G$ vamos denotar
\[
    x * y = x + y.
\]

Com a nota\c{c}\~ao aditiva o oposto de $x \in G$ ser\'a denotado por $-x$.

\begin{definicao}
	Seja $G$ um grupo aditivo e denote por $e$ o elemento neutro de $G$. Se $x \in G$ e $m \in \z$, o \textbf{m\'ultiplo $m$-\'esimo} de $x$ \'e o elemento de $G$ denotado por
	\[
		m \cdot x
	\]
	e definido por:
	\[
		m \cdot x = \begin{cases}
			e, & \mbox{se m = 0},\\
			(m - 1)\cdot x + x, & \mbox{ se } m \ge 1,\\
			-[(-m) \cdot x], & \mbox{ se } m < 0. 
		\end{cases}
	\]
\end{definicao}
    
\begin{proposicao}
    Seja $G$ um grupo aditivo. Se $m$ e $n$ s\~ao n\'umeros inteiros e $x \in G$, ent\~ao
    \begin{enumerate}[label={\roman*})]
        \item $m \cdot x + n \cdot x = (m + n) \cdot x$

        \item $(-m) \cdot x = -(m \cdot x)$

        \item $n\cdot (m \cdot x) = (nm)\cdot x$
    \end{enumerate}
\end{proposicao}

\begin{definicao}
	Seja $G$ um grupo multiplicativo e $x \in G$. Denote por $[x]$ o seguinte conjunto
	\[
	    [x] = \{x^m \mid m \in \z\} \subseteq G.
	\]	
\end{definicao}

\begin{proposicao}
    Seja $G$ um grupo multiplicativo e $x \in G$.
    \begin{enumerate}[label={\roman*})]
        \item O subconjunto $[x]$ \'e um subgrupo de $G$.

        \item Se $H$ \'e um subgrupo de $G$ tal que $x \in H$, ent\~ao $[x] \subseteq H$.
    \end{enumerate}
\end{proposicao}
\begin{prova}
	\begin{enumerate}[label={\roman*})]
		\item Como $x^0 = e$, então $e \in [x]$ e com isso $[x] \ne \emptyset$.

		Agora sejam $a$, $b \in [x]$. Assim existem $l$, $k \in \z$ tais que
		\begin{align*}
			a &= x^l\\
			b &= x^k.
		\end{align*}
		Então
		\begin{align*}
			a^{-1} &= (x^l)^{-1} = x^{-l} \in [x]\\
			ab &= x^lx^k = x^{l + k} \in [x].
		\end{align*}
		Portanto, $[x]$ é um subgrupo de $G$.

		\item Se $x \in H$ e $H$ é um subgrupo de $G$, então como
		\begin{align*}
			x^m &=  x^{m-1}x\\
			&=\underbrace{x\cdot x \cdots x}_{m\ vezes}
		\end{align*}
		segue que $x^m \in H$ para todo $m \in \z$. Logo $[x] \subseteq H$, como queríamos.
	\end{enumerate}
\end{prova}

\begin{definicao}
    Um grupo multiplicativo $G$ ser\'a chamado de \textbf{grupo c{\'\i}clico} se, para algum $x \in G$, vale
    \[
        G = [x].
    \]
    Nessas condi\c{c}\~oes, o elemento $x$  \'e chamado de \textbf{gerador} do grupo $G$.
\end{definicao}

\begin{exemplos}
    \begin{enumerate}[label={\arabic*})]
        \item No grupo multiplicativo $\complex^*$, o subgrupo gerado por $i$ é:
        \[
        	[i] = \{i^m \mid i \in \z\} = \{1, -1, i, -1\}.
        \]
        
        \item No grupo $S_3$, o subgrupo gerado por
        \[
            f = \begin{pmatrix}
                1 & 2 & 3\\
                2 & 3 & 1
            \end{pmatrix}
        \]
        é
        \[
        	[f] = \{f^m \mid m \in \z\} = \{e, f, f^2\}.
        \]
        \item No grupo aditivo $\z$ o subgrupo gerado por $3$ é
        \[
        	[3] = \{3m \mid m \in \z\} = 3\z.
        \]
    \end{enumerate}
\end{exemplos}

\begin{proposicao}
    Todo subgrupo de um grupo c{\'\i}clico \'e tamb\'em c{\'\i}clico.
\end{proposicao}
\begin{prova}
	Seja $G$ um grupo cíclico. Queremos mostrar que se $H \subseteq G$ é um subgrupo, então $H$ também é cíclico.

	Para isso suponha que $G = [x]$. Tome $H \subseteq G$ um subgrupo. Como os elementos de $G$ são todos da forma $x^m$, para $m \in \z$, então os elementos de $H$ também são potências de $x$.

	Se $H = \{e\}$, então $H = [e]$.

	Suponha que $H \ne \{e\}$. Assim existe $x^l \in H$ com $l \ne 0$. Como $H$ é subgrupo, então $(x^l)^{-1} \in H$ para todo $x^l \in H$. Ou seja, existe em $H$ pelo menos um elemento $x^k$ com $k > 0$.

	Seja $\alpha > 0$ o menor número inteiro tal que $x^\alpha \in H$. Denote
	\[
		x^\alpha = b.
	\]
	Vamos mostrar que
	\[
		H = [b].
	\]
	Como $b = x^\alpha \in H$, então pela Proposição \eqref{proposicao_subgrupo_gerado}, segue que $[b] \subseteq H$.

	Agora seja $y \in H \subseteq G = [x]$. Daí $y = x^t$ para algum $t \in \z$.

	Como $\alpha > 0$ podemos efetuar a divisão inteira de $t$ por $\alpha$ obtendo
	\[
		t = q\alpha + r
	\]
	com $0 \le \alpha < r$. Assim
	\begin{align*}
		y = x^t = x^{q\alpha + r} = (x^{\alpha})^qx^r.
	\end{align*}
	Mas $x^\alpha = b \in H$, logo $b^q \in H$ e daí
	\[
		x^r = b^{-q}y\in H
	\]
	pois $b^{-q}$, $y \in H$. Ou seja, $x^r \in H$. Mas $\alpha$ é o menor inteiro positivo tal que $x^\alpha \in H$ e $r < \alpha$. Logo $r = 0$ e com isso
	\begin{align*}
		y = x^{q\alpha + r} = (x^{\alpha})^qx^0 = (x^{\alpha})^q = b^q \in [b].
	\end{align*}
	Logo $y \in [b]$ e portanto
	\[
		H = [b]
	\]
	como queríamos.
\end{prova}

\begin{definicao}
    Seja $G$ um grupo com elemento neutro $e$. Dado $x \in G$ se existir um inteiro $h > 0$ tal que
    \begin{enumerate}[label={\roman*})]
        \item $x^h = e$
        \item $x^r \ne e$ qualquer que seja o inteiro $r$ tal que $0 < r < h$
    \end{enumerate}
    diremos que a \textbf{ordem} ou \textbf{per{\'\i}odo} de $x$ \'e $h$. Nesse caso escreveremos $|x| = o(x) = h$.

    Se para qualquer inteiro $r \ne 0$, $x^r \ne e$, diremos que a \textbf{ordem} de $x$ \'e \textbf{zero}.
\end{definicao}

\begin{exemplos}
    \begin{enumerate}[label={\arabic*})]
        \item No grupo multiplicativo $\complex^*$ temos
        \begin{itemize}
        	\item $o(1) = 1$  pois $1^1 = 1$

        	\item $o(i) = 4$ pois
        	\begin{align*}
        		i^1 &= i\\
        		i^2 &= -1\\
        		i^3 &= -i\\
        		i^4 &= 1
        	\end{align*}

        	\item $o(-i) = 4$

        	\item $o(2i) = 0$ pois para todo $r > 0$ temos $(2i)^r = 2^ri^r$ e $2^r \ne 1$ para todo $r > 0$.

        \end{itemize}
        
        \item Em $S_3$ temos, por exemplo, para
        \[
        	a = \begin{pmatrix}
        		1 & 2 & 3\\
        		2 & 1 & 3
        	\end{pmatrix}
        \]
        que
        \begin{align*}
        	a &\ne e\\
        	a^2 &=  \begin{pmatrix}
        		1 & 2 & 3\\
        		2 & 1 & 3
        	\end{pmatrix} \circ \begin{pmatrix}
        		1 & 2 & 3\\
        		2 & 1 & 3
        	\end{pmatrix} = \begin{pmatrix}
        		1 & 2 & 3\\
        		1 & 2 & 3
        	\end{pmatrix}
        \end{align*}
        e então $o(a) = 2$.

        Agora para
        \[
        	b = \begin{pmatrix}
        		1 & 2 & 3\\
        		3 & 1 & 2
        	\end{pmatrix}
        \]
        temos
        \begin{align*}
        	b &\ne e\\
        	b^2 = \begin{pmatrix}
        		1 & 2 & 3\\
        		3 & 1 & 2
        	\end{pmatrix} \circ \begin{pmatrix}
        		1 & 2 & 3\\
        		3 & 1 & 2
        	\end{pmatrix} = \begin{pmatrix}
        		1 & 2 & 3\\
        		2 & 3 & 1
        	\end{pmatrix}\\
        	b^3 = b^2 \circ b = \begin{pmatrix}
        		1 & 2 & 3\\
        		2 & 3 & 1
        	\end{pmatrix} \circ \begin{pmatrix}
        		1 & 2 & 3\\
        		3 & 1 & 2
        	\end{pmatrix} = \begin{pmatrix}
        		1 & 2 & 3\\
        		1 & 2 & 3
        	\end{pmatrix}
        \end{align*}
        e assim $o(b) = 3$.

        \item Em $\z_5$ com a soma temos
        \begin{itemize}
        	\item $o(\overline{0}) = 1$

        	\item $o(\overline{1}) = 5$ pois
        	\begin{align*}
        		\overline{1} &\ne \overline{0}\\
        		\overline{1} + \overline{1} &\ne \overline{0}\\
        		\overline{1} + \overline{1} + \overline{1} &\ne \overline{0}\\
        		\overline{1} + \overline{1} + \overline{1} + \overline{1} &\ne \overline{0}\\
        		\overline{1} + \overline{1} + \overline{1} + \overline{1} + \overline{1} &= \overline{0}
        	\end{align*}

        	De modo semelhante chega-se à conclusão que
        	\[
        		o(\overline{2}) = o(\overline{3}) = o(\overline{4}) = 5.
        	\]
        \end{itemize}

        \item Em $\z$ o \'unico elemento de ordem diferente de zero \'e o elemento neutro.
    \end{enumerate}
\end{exemplos}


\section{Homomorfismo de Grupos} % (fold)
\label{sec:homomorfismo_de_grupos}

% Sejam $(G, *)$ e $(H, \triangle)$ grupos quaisquer. Considere uma fun\c{c}\~ao $f : G \to H$. Entre todas as poss{\'\i}veis fun\c{c}\~oes entre $G$ e $H$ vamos considerar somente aquelas que satisfa\c{c}\~ao a condi\c{c}\~ao
% \[
% 	f(x * y) = f(x)\triangle f(y)
% \]
% para todos $x$, $y \in G$, ou seja, podemos determinar a imagem de $f(x*y)$ a partir da imagem de $x$ e de $y$,

\begin{definicao}
	Dados doi grupos $(G, *)$ e $(H,\triangle)$ dizemos que uma fun\c{c}\~ao $f : G \to H$ \'e um \textbf{homomorfismo de grupos} se
	\[
		f(x * y) = f(x)\triangle f(y)
	\]
	para todos $x$, $y \in G$.
\end{definicao}

\begin{observacao}
	Sejam $(G, *)$ e $(H, \triangle)$ grupos e $f : G \to H$ um homomorfismo.
	\begin{enumerate}[label={\arabic*})]
		\item Se $G = H$, neste caso $f : G \to G$ \'e chamado de um \textbf{endomorfimos} de grupos.
		\item Se $f : G \to H$ \'e uma fun\c{c}\~ao injetora, ent\~ao dizemos que $f$ \'e um \textbf{monomorfismo} de grupos.
		\item Se $f : G \to H$ \'e uma fun\c{c}\~ao sobrejetora, ent\~ao dizemos que $f$ \'e um \textbf{epimorfismo} de grupos.
		\item Se $f : G \to H$ \'e uma fun\c{c}\~ao bijetora, ent\~ao dizemos que $f$ \'e um \textbf{isomorfismo} de grupos.
		\item Se $f : G \to G$ \'e uma fun\c{c}\~ao bijetora, ent\~ao dizemos que $f$ \'e um \textbf{automorfismo} de grupos.
	\end{enumerate}
\end{observacao}

\begin{exemplos}
	\begin{enumerate}[label={\arabic*})]
		\item A fun\c{c}\~ao $f : \z \to \complex$ dada por $f(x) = i^x$ \'e um homomorfismo de $(\z, +)$ em $(\complex, \cdot)$. De fato,
		\[
			f(x + y) = i^{x + y} = i^x\cdot i^y = f(x)\cdot f(y)
		\]
		para todos $x$, $y \in \z$.

		\item A fun\c{c}\~ao $f : \real^*_+ \to \real$ dada por $f(x) = \ln(x)$ \'e um homomorfismo de $(\real^*_+, \cdot)$ em $(\real, +)$. De fato,
		\[
			f(xy) = \ln(xy) = \ln(x) + \ln(y) = f(x) + f(y)
		\]
		para todos $x$, $y \in \real^*_+$. Al\'em disso, como $\ln(x)$ \'e uma fun\c{c}\~ao bijetora, ent\~ao $f$ \'e um isomorfismo de grupos.

		\item Sejam $m$ um inteiro positivo fixo. A fun\c{c}\~ao $f: \z \to \z_m$ definida por $f(x) = \overline{x}$ \'e um homomorfimos de $(\z, +)$ em $(\z_m, \oplus)$. De fato,
		\[
			f(x + y) = \overline{x + y} = \overline{x} + \overline{y} = f(x) + f(y).
		\]
		Al\'em disso, esse homomorfismo \'e sobrejetor.
	\end{enumerate}
\end{exemplos}

\begin{proposicao}
	Sejam $(G, *)$ e $(H, \triangle)$ grupos e $f : G \to H$ um homomorfismo. Denote por $1_G$ e $1_H$ os elementos neutros de $G$ e $H$, respectivamente.
	\begin{enumerate}[label={\roman*})]
		\item $f(1_G) = 1_H$
		\item $[f(x)]^{-1} = f(x^{-1})$ para todo $x \in G$.
	\end{enumerate}
\end{proposicao}

\begin{proposicao}
	Sejam $I$ \'e um subgrupo de $G$ e $f : G \to H$ um homomorfismo de grupos. Ent\~ao $f(I)$ \'e um subgrupo de $H$.
\end{proposicao}
\begin{prova}
	Como $I$ \'e um subgrupo de $G$, ent\~ao $1_G \in G$. Agora $f$ \'e um homomorfismo, logo $f(1_G) = 1_H \in f(I)$ e assim $f(I) \ne \emptyset$.

	Agora, dado $y \in f(I)$ precisamos mostrar que $y^{-1} \in f(I)$. Mas se $y \in f(I)$, ent\~ao $y = f(x)$ com $x \in I$. Da{\'\i}
	\[
		y^{-1} = [f(x)]^{-1} = f(x^{-1})
	\]
	e como $I$ \'e um subgrupo de $G$, $x^{-1} \in I$ e como isso $y^{-1} \in f(I)$.

	Finalmente, dados $y$, $z \in f(I)$ existem $x_1$, $x_2 \in I$ tais que $y = f(x_1)$ e $z = f(x_2)$. Mas $f$ \'e homomorfismo, da{\'\i}
	\[
		y\triangle z = f(x_1)\triangle f(x_2) = f(x_1*x_2)
	\]
	e como $I$ \'e subgrupo, $x_1*x_2 \in I$. Logo $y\triangle z \in f(I)$.

	Portanto $f(I)$ \'e um subgrupo de $H$.
\end{prova}

\begin{definicao}
	Sejam $(G, *)$ e $(H, \triangle)$ grupos e $f : G \to H$ um homomorfismo de grupos. Chama-se de \textbf{n\'ucleo} ou \textbf{kernel} de $f$ e denota-se por $N(f)$ ou $\ker(f)$ o seguinte subconjunto de $G$:
	\[
		\ker(f) = \{x \in G \mid f(x) = 1_H\}.
	\]
\end{definicao}

\begin{exemplos}
	\begin{enumerate}[label={\roman*})]
		\item Considere o homomorfismo $f : \z \to \complex^*$ dado por $f(x) = i^x$. Temos
		\[
			\ker(f) = \{x \in \z \mid f(x) = 1\} = \{x \in \z \mid i^x = 1\} = \{0, \pm 4, \pm 8, \cdots\} = 4\z.
		\]

		\item O n\'ucleo do homomorfismo $f : \real^*_+ \to \real$ dado por $f(x) = \ln(x)$. Temos
		\[
			\ker(f) = \{x \in \real^*_+ \mid f(x) = 0\} = \{x \in \real^*_+ \mid \ln(x) = 0\} = \{1\}.
		\]

		\item O n\'ucleo do homomorfismo $f : \z \to \z_m$ dado por $f(x) = \overline{x}$, $m > 0$ fixo. Temos
		\[
			\ker(f) = \{x \in \z \mid f(x) = \overline{0}\} = \{x \in \z \mid \overline{x} = \overline{0}\} = \{0, \pm m, \pm 2m, \cdots\}.
		\]
	\end{enumerate}
\end{exemplos}

\begin{proposicao}
	Sejam $(G, *)$ e $(H, \triangle)$ grupos e $f : G \to H$ um homomorfismo de grupos. Ent\~ao:
	\begin{enumerate}[label={\roman*})]
		\item $\ker(f)$ \'e um subgrupo de $G$.
		\item $f$ \'e um monomorfismo se, e somente se, $\ker(f) = \{1_G\}$.
	\end{enumerate}
\end{proposicao}
\begin{prova}
	\begin{enumerate}[label={\roman*})]
		\item Como $f(1_G) = 1_H$, ent\~ao $1_G \in \ker(f)$ e com isso $\ker(f) \ne \emptyset$. Se $x \in \ker(f)$, ent\~ao $f(x^{-1}) = [f(x)]^{-1} = 1_H^{-1} = 1_H$ e da{\'\i} $x^{-1} \in \ker(f)$. Finalmente se $x$, $y \in \ker(f)$, ent\~ao $f(x*y) = f(x)\triangle f(y) = 1_H \triangle 1_H = 1_H$, ou seja, $x * y \in \ker(f)$.

		Portanto $\ker(f)$ \'e um subgrupo de $G$.

		\item Suponha que $f$ \'e um monomorfismo de grupos. Tome $x \in \ker(f)$. Temos $f(x) = 1_H = f(1_G)$ e como $f$ \'e injetora $x = 1_G$. Logo $\ker(f) = \{1_G\}$.

		Agora suponha que $\ker(f) = \{1_G\}$. Sejam $x$, $y \in G$ tais que
		\begin{align*}
			&f(x) = f(y)\\
			&f(x)\triangle f(y)^{-1} = 1_H\\
			&f(x)\triangle f(y^{-1}) = 1_H\\
			&f(x * y^{-1}) = 1_H
		\end{align*}
		e da{\'\i} $x*y^{-1} \in \ker(f) = \{1_G\}$. Logo $x*y^{-1} = 1_G$, isto \'e, $x = y$. Portanto $f$ \'e injetora.
	\end{enumerate}
\end{prova}

\section{Teorema de Lagrange}

\begin{teorema}\label{teorema_de_lagrange}
	Seja $G$ um grupo finito. Se $H\subseteq G$ {\'e} um subgrupo, ent{\~a}o $|H|$ divide $|G|$.
\end{teorema}

\begin{exemplo}
	Quais s{\~a}o as poss{\'\i}veis ordens dos subgrupos de um grupo de ordem 48?
	\begin{solucao}
		Seja $G$ um grupo tal que $|G|=48$. Se $H$ {\'e} um subgrupo pr\'orio de $G$, ent{\~a}o $|H|$ divide $|G|$. Mas $48=2^{4}\cdot 3$, da{\'\i} se $H$ \'e um subgrupo de $G$ ent\~ao $|H|=2$ ou $|H|=3$ ou $|H|= 2^{2}$ ou $|H|=2^{3}$ ou $|H|=2^{4}$ ou $|H|=2\cdot3$ ou $|H|=2^2\cdot 3$ ou $|H|=2^3\cdot 3$.
	\end{solucao}
\end{exemplo}

\begin{observacao}
	O Teorema \ref{teorema_de_lagrange} n{\~a}o diz que haver{\'a} um subgrupo de ordem $n$ para todo $n$ tal que $n||G|$. Diz apenas que se $H$ {\'e} subgrupo de $G$, ent{\~a}o $|H|$ divide $|G|$.
\end{observacao}

\begin{corolario}
	Os {\'u}nicos subgrupos de um grupo de ordem prima s{\~a}o os triviais.
\end{corolario}
\begin{prova}
	Suponha $|G| = p$ \'e um n\'umero primo. Assim os {\'u}nicos divisores positivos de $p$ s{\~a}o 1 e $p$. Logo se $H$ {\'e} um subgrupo de $G$, pelo Teorema \ref{teorema_de_lagrange} ent{\~a}o $|H|$ divide $|G|$. Assim $|H| = 1$ ou $|H| = p$. Portanto, $H=\{e\}$ ou $H = G$.
\end{prova}



% section homomorfimos_de_grupos (end)

% \section{Grupos de Permuta\c{c}\~ao}
% Fazer a parte de $S_n$.

% \section{Grupos C{\'\i}clicos}
% Fazer a parte de grupos c{\'\i}clicos.