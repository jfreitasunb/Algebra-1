% !TEX encoding = ISO-8859-1
\documentclass{article}

\usepackage{amssymb}
\usepackage{amsmath,amsfonts,amsthm,amstext}
\usepackage[brazil]{babel}
\usepackage[latin1]{inputenc}
\usepackage[pdftex]{graphicx}
\usepackage{url}
\usepackage{enumitem}

\setlength{\topmargin}{-1.0in}
\setlength{\oddsidemargin}{0in}
\setlength{\textheight}{10.1in}
\setlength{\textwidth}{6.5in}
\setlength{\baselineskip}{12mm}

\begin{document}
\pagestyle{empty}

\begin{figure}[h]
        \begin{minipage}[c]{1.7cm}
        \includegraphics[width=1.7cm]{../../../imagens/unb.pdf}
        \end{minipage}%
        \hspace{0pt}
        \begin{minipage}[c]{4in}
          {Universidade de Bras{\'\i}lia} \\
          {Departamento de Matem{\'a}tica}
\end{minipage}
\end{figure}
\vspace{-0.35cm} \hrule

\begin{center}
{\Large\bf {\'A}lgebra 1 - Turma A} \\ \vspace{9pt} {\large\bf Plano
de Ensino -- 2$^{o}$/2011}\\ \vspace{9pt} Prof. Jos{\'e} Ant{\^o}nio O. Freitas
%\vspace{0.25cm}
\end{center}
\hrule

\vspace{9pt}
\noindent \textbf{\underline{Programa:}}
\begin{enumerate}[label={\arabic*})]
%\item No{\c c}{\~o}es de L{\'o}gica.

\item Teoria de Conjuntos.

\item N{\'u}meros Inteiros.

\item Rela{\c c}{\~o}es e Fun{\c c}{\~o}es.

\item Estruturas Alg{\'e}bricas: An{\'e}is, Dom{\'\i}nios de
  Integridade, Grupos e Corpos.

\item Polin{\^o}mios em uma Vari{\'a}vel sobre Dom{\'\i}nios de Integridade e Corpos.
\end{enumerate}

\vspace{9pt}
\noindent \textbf{\underline{Bibliografia B{\'a}sica:}}
\vspace{0.15cm}

\begin{enumerate}[label={\arabic*})]
\item S. Shokranian: {\it {\'A}lgebra 1}, Ci{\^e}ncia Moderna, 2010.

\item H. H. Domingues, G. Iezzi: {\it {\'A}lgebra Moderna}, $2^a$
  Ed., Atual, 1982.

\item Adilson Gon{\c c}alves: {\it Introdu{\c c}{\~a}o {\`a} {\'A}lgebra}, $5^a$ Ed., IMPA,
  2003.

\item G. Birkhoff, S. MacLane: {\it {\'A}lgebra Moderna B{\'a}sica}, $4^a$ Ed.,
  Guanabara Dois, 1980.

%\item E. A. Filho: {\it Inicia{\c c}{\~a}o {\`a} L{\'o}gica Matem{\'a}tica}, Nobel, 2002.
\end{enumerate}

\noindent \textbf{\underline{Avalia{\c c}{\~a}o:}}
\vspace{0.15cm}

\noindent Ser{\~a}o aplicadas tr{\^e}s provas $P_1$, $P_2$ e $P_3$. A {\bf nota final (NF)}
ser{\'a} calculada de acordo com a seguinte f{\'o}rmula:
\[
NF = \dfrac{P_1 + 2P_2 + 3P_3}{6}.
\]
Para ser aprovado o aluno
dever{\'a} obter {\bf notal final (NF)} igual ou superior a $5,0$. As provas est{\~a}o
previstas para as datas indicadas abaixo. No entanto, a crit{\'e}rio do
professor, essas datas poder{\~a}o ser alteradas.

\begin{center}
%\vspace{0.5cm}
\begin{tabular}{cccc}
Prova:    & $P_1$    & $P_2$    & $P_3$ \cr
Data: & 29/09/11 & 03/11/11 & 13/12/11 \cr
\end{tabular}
\end{center}

\noindent \textbf{\underline{Observa{\c c}{\~o}es:}}
\vspace{-0.15cm}

\begin{enumerate}[label={\arabic*})]

\item Todos estudantes devem {\bf obrigatoriamente} se cadastrar no Moodle no endere{\c c}o
\url{www.aprender.unb.br}. Em seguida, devem se inscrever na disciplina
\begin{center}
``{\'A}lgebra 1 - Turma A'', com c{\'o}digo de inscri{\c c}{\~a}o {\bf 567329}.
\end{center}

\item As provas ser{\~a}o aplicadas apenas para alunos regularmente
  matriculados na Turma A. A data da prova poder{\'a} ser alterada a crit{\'e}rio do
  professor (de acordo com a realiza{\c c}{\~a}o do cronograma de ensino);
\vspace{-.25cm}

\item O aluno dever{\'a} apresentar documento de identifica{\c c}{\~a}o durante a
  realiza{\c c}{\~a}o da prova;
\vspace{-.25cm}

\item N{\~a}o {\'e} permitida a participa{\c c}{\~a}o de estudantes ouvintes durante as
  aulas;
\vspace{-.25cm}

\item N{\~a}o haver{\'a} prova de reposi{\c c}{\~a}o;
\vspace{-.25cm}

\item As provas ser{\~a}o individuais e sem consulta, sendo proibida a
  utiliza{\c c}{\~a}o de calculadoras e o empr{\'e}stimo de qualquer tipo de material
  entre os alunos;
\vspace{-.25cm}

\item Durante as avalia{\c c}{\~o}es, qualquer aparelho eletr{\^o}nico (incluindo
  celular) dever{\'a} estar desligado, sob pena de anula{\c c}{\~a}o da prova.
\vspace{-.25cm}

\item Haver{\'a} avalia{\c c}{\~a}o quanto {\`a} clareza, apresenta{\c c}{\~a}o e formaliza{\c c}{\~a}o na
  resolu{\c c}{\~a}o das quest{\~o}es de cada prova. A nota do aluno poder{\'a} ser alterada
  em raz{\~a}o da inobserv{\^a}ncia desses par{\^a}metros.

\end{enumerate}

\vfill
\hrule

\end{document}
