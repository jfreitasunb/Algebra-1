%!TEX program = xelatex
%!TEX root = Algebra_1.tex
%%Usar makeindex -s indexstyle.ist arquivo.idx no terminal para gerar o {\'\i}ndice remissivo agrupado por inicial
%%Ap\'os executar pdflatex arquivo
% \chapter{Rela{\c c}{\~o}es e Fun{\c c}{\~o}es}
\chapter{Rela{\c c}{\~o}es}
% \section{Rela{\c c}{\~o}es}
% \subsubsection{Defini{\c c}{\~a}o}
% Sejam A e B dois conjuntos n{\~a}o vazios. Os subconjuntos de AxB s{\~a}o chamados rela{\c c}{\~o}es, ou seja, uma rela{\c c}{\~a}o em AxB {\'e} um subconjunto desse produto cartesiano.

% Quando $R$ {\'e} uma rela{\c c}{\~a}o em $A \times B$, tamb{\'e}m dizemos que $R$ {\'e} uma rela{\c c}{\~a}o de A em B.

% Exemplos:
% \begin{enumerate}[label={\arabic*})]
% \item Se A=\{0,1\} e B=\{-1,0,1\}, ent{\~a}o AxB=\{(0,-1),(0,0),(0,1),(1,-1),(1,0),(1,1,)\}\\
% S{\~a}o exemplos de rela{\c c}{\~o}es:\\
% $R_{1}=\{(0,1)\}$\\
% $R_{2}=\emptyset$\\
% $R_{3}=\{(1,-1),(1,1)\}$\\
% $R_{4}=A$x$B$
% \item Se $A=B=\mathbb{R}$, ent{\~a}o AxB {\'e} o conjunto formado por todos pares ordenados de n{\'u}meros reais. Um exemplo de rela{\c c}{\~a}o em $\mathbb{R}$x$\mathbb{R}$ {\'e} o conjunto:\\
% $R=\{(x,y)\in \mathbb{R}$x$\mathbb{R}/ y\geq 0\}$
% \end{enumerate}

\section{Rela{\c c}{\~o}es de equival{\^e}ncia}

\begin{definicao}
	Seja $A$ um conjunto n{\~a}o vazio e $R\subseteq A \times A$. Dizemos que $R$ {\'e} uma \textbf{rela{\c c}{\~a}o de equival{\^e}ncia} se:
	\begin{enumerate}[label={\roman*})]
		\item Para todo $x \in A$, $(x,x) \in R$. \textit{(Propriedade Reflexiva)}
		\item Se $(x, y) \in R$, ent\~ao $(y, x) \in R$. \textit{(Propriedade Sim\'etrica)}
		\item Se $(x, y) \in R$ e $(y, z) \in R$, ent\~ao $(x, z)\in R$. \textit{(Propriedade Transitiva)}
	\end{enumerate}
\end{definicao}

Quando $R\subseteq A \times A$ {\'e} uma rela{\c c}{\~a}o de equival{\^e}ncia, dizemos que $R$ {\'e} uma rela{\c c}{\~a}o de equival{\^e}ncia em $A$. Quando dois elementos $x$, $y \in A$ s{\~a}o tais que $(x,y) \in R$, dizemos que $x$ e $y$ \textbf{s{\~a}o relacionados} ou que $x$ e $y$ \textbf{est\~ao relacionados}.

\begin{exemplos}\label{exemplos_relacoes_equivalencia}
	\begin{enumerate}[label={\arabic*})]
		\item Seja A=\{1,2,3,4\}. Temos
		\begin{align*}
			A\times A = &\{(1,1);(1,2);(1,3);(1,4);(2,1);(2,2);(2,3);(2,4);\\ &(3,1);(3,2);(3,3);(3,4);(4,1);(4,2);(4,3);(4,4)\}.
		\end{align*}
		Quais dos seguintes conjuntos s\~ao exemplos de rela{\c c}{\~o}es de equival{\^e}ncia?
		\begin{itemize}
			\item $R_{1}= A\times A$
			\item $R_{2}=\{(1,1);(2,2);(3,3)\}$
			\item $R_{3}=\{(1,1);(2,2);(3,3);(4,4);(1,2);(2,1)\}$
			\item $R_{4}=\{(1,1);(2,2);(3,3);(4,4)\}$
			\item $R_{5}=\{(1,1);(2,2);(3,3);(4,4);(1,2);(2,1);(2,4);(4;2)\}$
		\end{itemize}
		\begin{solucao}
			$R_2$ n\~ao \'e rela\c{c}\~ao de equival\^encia pois $(4,4) \notin R_2$.

			$R_5$ n\~ao \'e rela\c{c}\~ao de equival\^encia pois, por exemplo, $(1,4) \notin R_5$.

			Os demais s\~ao exemplos de rela\c{c}\~oes de equival\^encia.
		\end{solucao}
		
		\item Seja $A = \z$ e $R\subseteq \z\times \z$ definida por $R = \{(x,y)\in \z \times \z \mid x = y\}$.
		Ent\~ao $R$ {\'e} uma rela{\c c}{\~a}o de equival{\^e}ncia.
		\begin{solucao}
			De fato,
			\begin{itemize}
				\item Para todo $x \in \z$ temos $x = x$ da{\'\i} $(x,x) \in R$.
				\item Se $(x,y)\in R$, ent\~ao pela defini\c{c}\~ao de $R$ temos $x = y$. Logo $y = x$ e ent\~ao $(y,x)\in R$.
				\item Se $(x,y) \in R$ e $(y,z) \in R$, ent\~ao  $x = y$ e $y = z$. Logo $x = z$ e assim $(x,z)\in R$ como quer{\'\i}amos.
			\end{itemize}
			Portanto $R$ \'e uma rela\c{c}\~ao de equival\^encia sobre $\z$.
		\end{solucao}
		
		\item Seja $A = \z$ e tome $R = \{(x,y)\in \z \times \z \mid x - y = 2k, \mbox{ para algum } k \in \z\}$. Mostre que $R$
		\'e uma rela{\c c}{\~a}o de equival{\^e}ncia sobre $\z$.
		\begin{solucao}
			De fato,
			\begin{itemize}
				\item Para todo $x\in\z$ temos $x - x = 2\cdot0$ e com isso $(x,x) \in R$.
				\item Se $(x,y) \in R$ ent\~ao existe $k \in \z$ tal que $x - y = 2k$. Agora $y - x = -(x - y) = -2k = 2 (-k)$ 
				e como $-k \in \z$ segue que $(y,x) \in R$.
				\item Se $(x,y) \in R$ e $(y,z) \in R$, ent\~ao existem $k$, $l\in \z$ tais que $x - y = 2k$ e $y - z = 2l$.
				Somando essas duas equa\c{c}\~oes obtemos
				\begin{align*}
					(x - y) + (y - z) &= 2k + 2l\\
					x - z &= 2(k + l)
				\end{align*}
				e como $k + l \in \z$ segue que $(x,z) \in \z$.
			\end{itemize}
			Assim $R$ \'e uma rela\c{c}\~ao de equival\^encia.
		\end{solucao}
	\end{enumerate}
\end{exemplos}
\begin{observacoes}
	Seja $R$ uma rela{\c c}{\~a}o de equival{\^e}ncia em $A$, isto \'e, $R \sub A \times A$.
	\begin{enumerate}[label={\arabic*})]
		\item  Para dizermos que $(x,y) \in R$ usaremos a nota{\c c}{\~a}o $x\equiv y\ (R)$, que se l{\^e} ``$x$ \'e equivalente a $y$ m{\'o}dulo $R$", ou ainda a nota{\c c}{\~a}o $xRy$, com o mesmo significado anterior.
		\item Em alguns casos vamos utilizar a nota\c{c}\~ao $\sim$ para representar a rela\c{c}\~ao $R$. Nesse caso, escrevemos $x \sim y$ para dizer que $(x, y) \in R$, ou que, $xRy$.
	\end{enumerate}
\end{observacoes}

Em virtude da observa\c{c}\~ao anterior a defini\c{c}\~ao de rela\c{c}\~ao de equival\^encia pode ser reescrita como:

\begin{definicao}
	Seja $A$ um conjunto n{\~a}o vazio e $R\subseteq A \times A$. Dizemos que $R$ {\'e} uma \textbf{rela{\c c}{\~a}o de equival{\^e}ncia} se:
	\begin{enumerate}[label={\roman*})]
		\item Para todo $x \in A$, $xRx$. \textit{(Propriedade Reflexiva)}
		\item Se $xRy$, ent\~ao $yRx$. \textit{(Propriedade Sim\'etrica)}
		\item Se $xRy$ e $yRz$, ent\~ao $xRz$. \textit{(Propriedade Transitiva)}
	\end{enumerate}
\end{definicao}

\begin{definicao}
	Seja $R$ uma rela{\c c}{\~a}o de equival{\^e}ncia sobre um conjunto $A$. Dado $b \in A$, chamamos de \textbf{classe de equival{\^e}ncia determinada por $b$ m{\'o}dulo $R$}, denotada por $\overline{b}$ ou $C(b)$, o subconjunto de $A$ dado por
	\[
		\overline{b} = C(b) = \{x \in A \mid (x,b) \in R\} = \{x \in A \mid xRb\}.
	\]
\end{definicao}

\begin{observacao}
	Seja $A \ne \emptyset$ e $R$ uma rela\c{c}\~ao de equival\^encia sobre $A$. Segue da defini\c{c}\~ao de rela\c{c}\~ao de equival\^encia que para todo $b \in A$, $\overline{b} \ne \emptyset$ pois $(b,b) \in R$ logo $b \in \overline{b}$.
\end{observacao}

\begin{exemplos}\label{exemplos_classes_equivalencia}
	Do Exemplo \ref{exemplos_relacoes_equivalencia} temos
	\begin{enumerate}[label={\arabic*})]
		\item As classes de equival\^encia de $R_1$ s\~ao:
		\begin{align*}
			\overline{1} &= \{x \in A \mid (x,1) \in R_1\} = \{1,2,3,4\}\\
			\overline{2} &= \{x \in A \mid (x,2) \in R_1\} = \{1,2,3,4\}\\
			\overline{3} &= \{x \in A \mid (x,3) \in R_1\} = \{1,2,3,4\}\\
			\overline{4} &= \{x \in A \mid (x,4) \in R_1\} = \{1,2,3,4\}\\
		\end{align*}
		Nesse caso temos somente uma classe de equival\^encia.

		\item As classes de equival\^encia de $R_3$ s\~ao:
		\begin{align*}
			\overline{1} &= \{x \in A \mid (x,1) \in R_3\} = \{1,2\}\\
			\overline{2} &= \{x \in A \mid (x,2) \in R_3\} = \{1,2\}\\
			\overline{3} &= \{x \in A \mid (x,3) \in R_3\} = \{3\}\\
			\overline{4} &= \{x \in A \mid (x,4) \in R_3\} = \{4\}\\
		\end{align*}
		Aqui temos tr\^es classes de equival\^encia diferentes.

		\item As classes de equival\^encia de $R_4$ s\~ao:
		\begin{align*}
			\overline{1} &= \{x \in A \mid (x,1) \in R_4\} = \{1\}\\
			\overline{2} &= \{x \in A \mid (x,2) \in R_4\} = \{2\}\\
			\overline{3} &= \{x \in A \mid (x,3) \in R_4\} = \{3\}\\
			\overline{4} &= \{x \in A \mid (x,4) \in R_4\} = \{4\}\\
		\end{align*}
		Aqui temos quatro classes de equival\^encia diferentes.

		\item Para a rela\c{c}\~ao de equival\^encia $R = \{(x,y)\in \z \times \z \mid x - y = 2k, \mbox{ para algum } k \in \z\}$ temos:
		\begin{align*}
			\overline{0} &= \{x \in \z \mid xR0 \} = \{x \in \z \mid x - 0 = 2k,\ k \in \z\} \\ 
			\overline{0} &= \{x \in \z \mid x = 2k,\ k \in \z\} = \{0, \pm 2, \pm 4, \pm 6, \dots\}\\
			\overline{1} &= \{x \in \z \mid xR1 \} = \{x \in \z \mid x - 1 = 2k,\ k \in \z\} \\
			\overline{1} &= \{x \in \z \mid x = 2k + 1,\ k \in \z\} = \{\pm 1, \pm 3, \pm 4, \pm 7, \dots\}\\
		\end{align*}
		Neste caso existem somente duas classes de equival\^encia. (\textit{Por qu\^e?})
	\end{enumerate}
\end{exemplos}

\begin{proposicao}
	Seja $R$ uma rela{\c c}{\~a}o de equival{\^e}ncia em um conjunto n{\~a}o vazio $A$. Dados $a$, $b \in A$ temos:
	\begin{enumerate}[label={\roman*})]
		\item se $\overline{a} \cap \overline{b} \ne \emptyset$, ent{\~a}o $aRb$.
		\item se  $\overline{a} \cap \overline{b} \neq \emptyset$, ent{\~a}o $\overline{a} = \overline{b}$.
	\end{enumerate}
\end{proposicao}
\begin{prova}
	\begin{enumerate}[label={\roman*})]
		\item Como  $\overline{a} \cap \overline{b} \ne \emptyset$, existe um $y \in \overline{a} \cap \overline{b}$, logo $y \in \overline{a}$ e $y \in \overline{b}$. Da defini{\c c}{\~a}o de classe de equival{\^e}ncia temos $yRa$ e $yRb$. Como $R$ {\'e} rela{\c c}{\~a}o de equival{\^e}ncia temos $aRy$ e $bRy$. Pela propriedade transitiva segue que $aRb$, como quer{\'\i}amos.

		\item Precisamos mostrar que $\overline{a} \sub \overline{b}$ e que $\overline{b} \sub \overline{a}.$ Para a primeira inclus\~ao seja $y \in \overline{a}$. Da{\'\i} $yRa$. Mas, por hip\'otese, $\overline{a}\cap\overline{b}\neq\emptyset$, assim pelo item anterior segue que $aRb$. Logo, como $yRa$ e $aRb$, segue que $yRb$, ou seja, $y \in \overline{b}$. Da{\'\i} $\overline{a}\sub\overline{b}$. Agora para provar a segunda inclus\~ao seja $x \in \overline{b}$. Ent\~ao $xRb$. Novamente, $\overline{a} \cap \overline{b} \ne \emptyset$ e ent\~ao pelo item anterior segue que $aRb$. Assim uma vez que $R$ \'e uma rela\c{c}\~ao de equival\^encia temos $bRa$ e de $xRb$ obtemos $xRa$, ou seja, $x \in \overline{a}$. Com isso $\overline{b} \sub \overline{a}$. Portanto $\overline{a} = \overline{b}$, como quer{\'\i}amos.
	\end{enumerate}
\end{prova}

\begin{corolario}
	Seja $R$ uma rela\c{c}\~ao de equival\^encia sobre um conjunto n\~ao vazio $A$. Dados $a$, $b \in A$ ent\~ao $\overline{a} \cap \overline{b} = \emptyset$ ou $\overline{a} = \overline{b}$.
\end{corolario}

\begin{definicao}
	Seja $R$ uma rela\c{c}\~ao de equival\^encia sobre um conjunto n\~ao vazio $A$. O conjunto de todas as classes de equival{\^e}ncia determinadas por $R$ ser{\'a} denotado por $A/R$ e {\'e} chamado de \textbf{conjunto quociente} de $A$ por $R$.
\end{definicao}

\begin{exemplos}
	Do Exemplo \ref{exemplos_classes_equivalencia} temos:
	\begin{enumerate}[label={\arabic*})]
		\item $A/R_1 = \{\overline{1}\}$
		\item $A/R_3 = \{\overline{1},\overline{3},\overline{4}\}$
		\item $A/R_4 = \{\overline{1},\overline{2},\overline{3},\overline{4}\}$
		\item $\z/R = \{\overline{0},\overline{1}\}$
	\end{enumerate}
\end{exemplos}

\begin{definicao}
	Seja $C$ uma classe de equival{\^e}ncia de uma rela{\c c}{\~a}o de equival{\^e}ncia $R$. Qualquer elemento $y\in C$ {\'e} chamado \textbf{representante} de $C$.
\end{definicao}

\begin{proposicao}
	Seja $A$ um conjunto n{\~a}o vazio e $R$ uma rela{\c c}{\~a}o de equival{\^e}ncia em $A$. Ent{\~a}o $A$ {\'e} a uni{\~a}o disjunta das classes $\overline{b}$, $b \in A$, ou seja,
	\[
		A = \bigcup_{b\in A}\overline{b}.
	\]
\end{proposicao}
\begin{prova}
	Para todo $b\in A$ temos, pela defini\c{c}\~ao de classe de equival\^encia, que $\overline{b}\subseteq A$. Logo $\bigcup_{b\in A}\overline{b}\subseteq A$. Agora seja $x\in A$. Logo $x \in \overline{x}$ e da{\'\i} $x\in \bigcup_{b\in A}\overline{b}$. Assim $A\subseteq\bigcup_{b\in A}\overline{b}$. Portanto, $A=\bigcup_{b\in A}\overline{b}$.
\end{prova}

\begin{definicao}
	Sejam $a$, $b \in \z$, $b \neq 0$. Dizemos que $b$ \textbf{divide} $a$ quando existe um inteiro $k$ tal que $a=bk$.
	Nesse caso escrevemos $b \mid a$. Quando $b$ \textbf{n{\~a}o divide} $a$, escrevemos $b\not{\mid}a$.
\end{definicao}

\begin{exemplos}
	\begin{enumerate}[label={\arabic*})]
		\item Os inteiros 1 e $-1$ dividem qualquer n{\'u}mero inteiro $a$, pois $a = 1 a$ e $a = (-1)(-a)$.
		\item O n{\'u}mero 0 n{\~a}o divide nenhum inteiro $b$, pois n{\~a}o existe $a \in \z$ tal que $b = 0a$.
		\item Para todo $b\neq 0$, $b$ divide $\pm b$.
		\item Para todo inteiro $b\neq 0$, $b$ divide 0, pois $0 = b0$.
		\item $3 \not{\mid} 8$.
		\item $17 \mid 51$.
	\end{enumerate}	
\end{exemplos}


\begin{proposicao}
	\begin{enumerate}[label={\roman*})]
		\item $a\mid a$, para todo $a \in \z$.
		\item Se $a\mid b$ e $b\mid a$, $a$, $b > 0$ ent\~ao $a = b$.
		\item Se $a\mid b$ e $b\mid c$, ent{\~a}o $a\mid c$.
		\item Se $a\mid b$ e $a\mid c$, ent{\~a}o $a\mid (bx+cy)$, para todos $x$, $y \in \z$.
	\end{enumerate}
\end{proposicao}
\begin{prova}
	\begin{enumerate}[label={\roman*})]
		\item Imediata.
		
		\item De fato, existem $k$, $l \in \z $ tais que $b = ka$ e $a = lb$. Assim $b = klb$, isto \'e, $b(1 - kl) = 0$.
		Como $b \ne 0$ ent\~ao $1 - kl = 0$. Da{\'\i} $kl = 1$ e ent\~ao $k = \pm 1$ e $l = \pm 1$. Mas $a > 0$ e $b > 0$, logo $k = l =1$. Logo $a = b$.

		\item De fato, existem $k$, $l \in \z$ tais que $b = ka$ e $c = bl$. Assim  $c = kal = (kl)a$, ou seja, $a\mid c$.

		\item Temos $b = ka$ e $c = al$, com $k$, $l \in \z$. Da{\'\i} $bx + cy = (ka)x + (al)y = a(kx + ly)$ e como $kx + ly \in \z$ segue que $a \mid (bx + cy)$.
	\end{enumerate}
\end{prova}

\begin{definicao}
	Sejam $a$, $b \in\z$, dizemos que $a$ \textbf{{\'e} congruente \`a} $b$ \textbf{m{\'o}dulo} $m$ se $m \mid (a-b)$. Neste caso, escrevemos $a\equiv_{m} b$ ou $a\equiv b \pmod{m}$.
\end{definicao}

\begin{exemplos}
	\begin{enumerate}[label={\arabic*})]
		\item $5\equiv 2 \pmod{3}$, pois $3 \mid (5-2)$.
		\item $3\equiv 1 \pmod{2}$, pois $2\mid (3-1)$.
		\item $3\equiv 9 \pmod{6}$, pois $6\mid (3-9)$.
	\end{enumerate}	
\end{exemplos}

\begin{proposicao}
	A congru{\^e}ncia m{\'o}dulo $m$ {\'e} uma rela{\c c}{\~a}o de equival{\^e}ncia em $\z$.
\end{proposicao}
\begin{prova}
	\begin{enumerate}[label={\roman*})]
		\item Para todo $a \in \z$, $a\equiv a\pmod{m}$ pois $m\mid (a-a)$.
		\item Se $a\equiv b\pmod{m}$, ent{\~a}o $m\mid (a - b)$. Da{\'\i} existe $k \in \z$, tal que $(a - b) = km$. Agora, $(b - a) = -(a - b) = -(km) = (-k)m$, ou seja, $m \mid (b - a)$. Da{\'\i} $b\equiv a \pmod{m}$.
		\item Se $a\equiv b\pmod{m}$ e $b\equiv c\pmod{m}$, ent{\~a}o $m\mid (a-b)$ e $m\mid (b-c)$. Assim, $m\mid [(a-b)+(b-c)]$. Logo, $m\mid (a-c)$, isto {\'e}, $a\equiv c\pmod{m}$.
	\end{enumerate}

	Portanto a congru{\^e}ncia m{\'o}dulo $m$ {\'e} uma rela{\c c}{\~a}o de equival{\^e}ncia.
\end{prova}

\begin{teorema}
	A rela{\c c}{\~a}o de congru{\^e}ncia m{\'o}dulo $m$ satisfaz as seguintes propriedades:
	\begin{enumerate}[label={\roman*})]
		\item $a_{1}\equiv b_{1}\pmod{m}$ se, e somente se, $a_{1}-b_{1}\equiv 0\pmod{m}$.
		\item Se $a_{1}\equiv b_{1}\pmod{m}$ e $a_{2}\equiv b_{2}\pmod{m}$, ent{\~a}o $a_{1}+a_{2}\equiv b_{1}+b_{2}\pmod{m}$.
		\item Se $a_{1}\equiv b_{2}\pmod{m}$ e $a_{2}\equiv b_{2}\pmod{m}$, ent{\~a}o $a_{1}a_{2}\equiv b_{1}b_{2}\pmod{m}$.\label{item_provado}
		\item Se $a\equiv b\pmod{m}$, ent{\~a}o $ax\equiv bx\pmod{m}$, para todo $x \in \z$.
		\item Vale a lei do cancelamento: se $d \in \z$ e $mdc(d,m) = 1$ ent{\~a}o $ad \equiv bd \pmod{m}$ implica $a\equiv b \pmod{m}$.
	\end{enumerate}
\end{teorema}
\begin{prova}
	Provemos o item \ref{item_provado}.
	
	Como $a_{1}\equiv b_{1}\pmod{m}$ e $a_{2}\equiv b_{2}\pmod{m}$, existem $k$, $l \in \z$ tais que
	\begin{align*}
		a_1 - b_1 &= km\\
		a_2 - b_2 &= lm,
	\end{align*}
	isto \'e,
	\begin{align*}
		a_1 &= b_1 + km\\
		a_2 &= b_2 + lm,
	\end{align*}
	Assim
	\begin{align*}
		a_1a_2 &= (b_1 + km)(b_2 + lm) \\ &= b_1b_2 + b_1lm + b_2km + klm^2 \\ &= b_1b_2 + \underbrace{(lb_{1}+kb_{2}+klm)}_{\in \z}m
	\end{align*}
	
	Ou seja, $a_1a_2 - b_1b_2 = pm$, onde $p = lb_1 + kb_2 + klm \in \z$. Portanto, $a_1a_2 \equiv b_1b_2 \pmod{m}$.
\end{prova}

Como a congru{\^e}ncia m{\'o}dulo $m$ {\'e} uma rela{\c c}{\~a}o de equival{\^e}ncia, podemos determinar suas classes de equival{\^e}ncia. Assim, dado $n \in \z$, temos
\[
	\overline{n} = C(n) = \{x \in \z \mid x\equiv n \pmod{m}\}.
\]

Denotaremos $C(n)$ por $R_{m}(n)$ ou $\overline{n}$, quando n{\~a}o houver possibilidade de confus{\~a}o.

Por exemplo, fixando $m > 1$
\begin{align*}
	R_{m}(0) &= \{x \in \z \mid x\equiv 0 \pmod{m}\}=\{x\in \z \mid x = mk, k\in\z\}=m\z\\
	R_{m}(1) &= \{x\in\z \mid x\equiv 1 \pmod{m}\}=\{x\in\z \mid x = 1 + km, k\in\z\}\\
	R_{m}(n) &= \{x\in\z \mid x = n + km, k\in\z\}
\end{align*}

\begin{proposicao}
	As classes de equival{\^e}ncia definidas pela congru{\^e}ncia m{\'o}dulo $m$ s{\~a}o determinadas pelos restos da divis{\~a}o inteira por $m$. Em outras palavras, $R_{m}(n)$ {\'e} o conjunto dos n{\'u}meros inteiros cujo resto na divis{\~a}o inteira por $m$ {\'e} $n$.
\end{proposicao}

\begin{corolario}
	$R_{m}(k) = R_{m}(l)$ se, e somente se, $k\equiv l \pmod{m}$.
\end{corolario}

\begin{exemplos}
	\begin{enumerate}[label={\arabic*})]
		\item Se $m=2$, ent{\~a}o os poss{\'\i}veis restos na divis{\~a}o inteira por 2 s{\~a}o 0 e 1. Logo, existem duas classes de equival{\^e}ncia, a saber
		\begin{align*}
			R_{2}(0) &= \{x \in \z \mid x\equiv 0 \pmod{2}\} = \{x\in \z \mid x = 2k, k\in\z\}\\
			R_{2}(1) &= \{x\in\z \mid x\equiv 1 \pmod{2}\} = \{x\in\z \mid x = 1 + 2k, k\in\z\}.
		\end{align*}
		
		\item Se $m = 3$, ent{\~a}o os poss{\'\i}veis restos da divis{\~a}o inteira s{\~a}o 0, 1 e 2. Da{\'\i}
		\begin{align*}
			R_{3}(0) &= \{x \in \z \mid x\equiv 0 \pmod{3}\} = \{x\in \z \mid x = 3k, k \in \z\}\\
			R_{3}(1) & = \{x \in \z \mid x\equiv 1 \pmod{3}\} = \{x\in\z \mid x = 3k + 1, k \in \z\}\\
			R_{3}(2) &= \{x \in \z \mid x\equiv 2 \pmod{3}\} = \{x\in\z \mid x = 3k + 2, k \in \z\}
		\end{align*}
	\end{enumerate}	
\end{exemplos}

\begin{proposicao}
	Na rela{\c c}{\~a}o de equival{\^e}ncia m{\'o}dulo $m$ existem $m$ classes de equival{\^e}ncia.
\end{proposicao}
\begin{prova}
	Os poss{\'\i}veis restos na divis{\~a}o inteira por $m$ s{\~a}o $0,1,...,(m-1)$. Como cada poss{\'\i}vel resto define uma classe de equival{\^e}ncia diferente, existem exatamente $m$ classes de equival{\^e}ncia
\end{prova}

\begin{observacao}
Fixado $m$ inteiro positivo, denotaremos
\begin{align*}
	R_{m}(0) &= \overline{0}\\
	R_{m}(1) &= \overline{1}\\
	&\vdots\\
	R_{m}(m-1) &= \overline{m-1}
\end{align*}

O conjunto quociente desta rela{\c c}{\~a}o ser{\'a} denotado por $\displaystyle\frac{\z}{m\z}$ ou $\z_m$. Assim
\[
	\z_m = \displaystyle\frac{\z}{m\z}=\{\overline{0},\overline{1},...,\overline{m-1}\}.
\]
\end{observacao}

Queremos definir um meio de somar e multiplicar os elementos de $\z_m$. Por exemplo, em $\z_2 = \{\overline{0},\overline{1}\}$ temos que a soma de pares {\'e} par, soma de par com {\'\i}mpar {\'e} {\'\i}mpar e a soma de {\'\i}mpares {\'e} par. Assim podemos escrever

\begin{table}[h]
   \centering 
   \setlength{\arrayrulewidth}{0,5\arrayrulewidth}
   \begin{tabular}{|c|c|c|} 
      \hline
      $\oplus$ & $\overline{0}$ & $\overline{1}$ \T\\
      \hline
      $\overline{0}$ & $\overline{0}$ & $\overline{1}$\T\\
      \hline
      $\overline{1}$ & $\overline{1}$ & $\overline{0}$\T\\
      \hline
   \end{tabular}
\end{table}

Para multiplica{\c c}{\~a}o, temos

\begin{table}[h]
   \centering 
   \setlength{\arrayrulewidth}{0,5\arrayrulewidth}
   \begin{tabular}{|c|c|c|} 
      \hline
      $\otimes$ & $\overline{0}$ & $\overline{1}$\T\\
      \hline
      $\overline{0}$ & $\overline{0}$ & $\overline{0}$\T\\
      \hline
      $\overline{1}$ & $\overline{0}$ & $\overline{1}$\T\\
      \hline
   \end{tabular}
\end{table}

\begin{definicao}
	Dados $\overline{a}$, $\overline{b} \in \z_m$ definimos
	\begin{align}
		\overline{a}\oplus\overline{b} &= \overline{a + b}\label{soma_modulo_m}\\
		\overline{a}\otimes\overline{b} &= \overline{ab}.\label{multiplicacao_modulo_m}
	\end{align}
\end{definicao}

\begin{proposicao}
	As opera{\c c}{\~o}es de soma e produto definidas em \eqref{soma_modulo_m} e \eqref{multiplicacao_modulo_m} s{\~a}o independentes dos representantes das classes.
\end{proposicao}
\begin{prova}
	Dadas duas classes em $\z_m$ com representantes diferentes, $\overline{a}_{1} = \overline{a}_{2}$ e  $\overline{b}_{1} = \overline{b}_{2}$, com $a_{1}\ne a_{2}$ e $b_{1}\ne b_{2}$,  temos:
            \begin{align*}
                a_1 &\equiv a_2 \pmod m\\
                b_1 &\equiv b_2 \pmod m.
            \end{align*}
            Daí,
            \begin{align*}
                a_1 + b_1 &\equiv a_2 + b_2 \pmod m\\
                a_1b_1 &\equiv a_2b_2 \pmod m
            \end{align*}
    
        Mas de $a_1 + b_1 \equiv a_2 + b_2 \pmod m$ segue que $\overline{a_1 + b_1} = \overline{a_2 + b_2}$. Assim
        \begin{align*}
            \overline{a}_{1}\oplus \overline{b}_{1} = \overline{a_{1}+b_{1}} = \overline{a_{2} + b_{2}} = \overline{a}_{2}\oplus \overline{b}_{2}.
        \end{align*}

        Agora de $a_1b_1 \equiv a_2b_2 \pmod m$  segue que $\overline{a_1b_2} =  \overline{a_2b_2}$. Assim
        \begin{align*}
            \overline{a}_{1}\otimes \overline{b}_{1} = \overline{a_{1}b_{1}} = \overline{a_{2}b_{2}} = \overline{a}_{2}\otimes\overline{b}_{2}.
        \end{align*}

        Portanto a soma e a multiplicação não dependem dos representantes que escolhemos para as classes de equivalência, como queríamos.\hspace{.3cm}
\end{prova}

\begin{exemplo}
	A soma e a multiplica{\c c}{\~a}o em $\z_4 = \{\overline{0},\overline{1},\overline{2},\overline{3}\}$
	s\~ao dadas nas tabelas abaixo:
		\begin{table}[!htb]
		  \caption{Soma e multiplica\c{c}\~ao em $\z_4$}
		  \begin{minipage}{.5\linewidth}
		    \centering
		 	\begin{tabular}{|c|c|c|c|c|} 
			    \hline
			    $\oplus$ & $\overline{0}$ & $\overline{1}$ & $\overline{2}$ & $\overline{3}$\T\\
			    \hline
			    $\overline{0}$ & $\overline{0}$ & $\overline{1}$ & $\overline{2}$ & $\overline{3}$\T\\
			    \hline
			    $\overline{1}$ & $\overline{1}$ & $\overline{2}$ & $\overline{3}$ & $\overline{0}$\T\\
			    \hline
			    $\overline{2}$ & $\overline{2}$ & $\overline{3}$ & $\overline{0}$ & $\overline{1}$\T\\
			    \hline
			    $\overline{3}$ & $\overline{3}$ & $\overline{0}$ & $\overline{1}$ & $\overline{2}$\T\\
			    \hline
			\end{tabular}
		  \end{minipage}
		  \begin{minipage}{.5\linewidth}
		  \centering
		    \begin{tabular}{|c|c|c|c|c|} 
		      \hline
		      $\otimes$ & $\overline{0}$ & $\overline{1}$ & $\overline{2}$ & $\overline{3}$\T\\
		      \hline
		      $\overline{0}$ & $\overline{0}$ & $\overline{0}$ & $\overline{0}$ & $\overline{0}$\T\\
		      \hline
		      $\overline{1}$ & $\overline{0}$ & $\overline{1}$ & $\overline{2}$ & $\overline{3}$\T\\
		      \hline
		      $\overline{2}$ & $\overline{0}$ & $\overline{2}$ & $\overline{0}$ & $\overline{2}$\T\\
		      \hline
		      $\overline{3}$ & $\overline{0}$ & $\overline{3}$ & $\overline{2}$ & $\overline{1}$\T\\
		      \hline
			\end{tabular}
		\end{minipage}
	\end{table}
\end{exemplo}

\begin{proposicao}
	As opera\c{c}\~oes de soma $\oplus$ e multiplica\c{c}\~ao $\otimes$ em $\z_m$ satisfazem as seguintes propriedades:
	\begin{enumerate}[label={\roman*})]
		\item Para todos $\overline{x}$, $\overline{y} \in \z_m$: $\overline{x} \oplus \overline{y} = \overline{y} \oplus \overline{x}$.
		\item Para todos $\overline{x}$, $\overline{y}$ e $\overline{z} \in \z_m$: $(\overline{x} \oplus \overline{y}) \oplus \overline{z} = \overline{x} \oplus (\overline{y} \oplus \overline{z})$.
		\item Para todo $\overline{x} \in \z_m$, $\overline{x} \oplus \overline{0} = \overline{x}$.
		\item Para todo $\overline{x} \in \z_m$, existe $\overline{y} \in \z_m$ tal que $\overline{x} \oplus \overline{y} = \overline{0}$.
		\item Para todos $\overline{x}$, $\overline{y} \in \z_m$: $\overline{x} \otimes \overline{y} = \overline{y} \otimes \overline{x}$.
		\item Para todos $\overline{x}$, $\overline{y}$ e $\overline{z} \in \z_m$: $(\overline{x} \otimes \overline{y}) \otimes \overline{z} = \overline{x} \otimes (\overline{y} \otimes \overline{z})$.
		\item Para todo $\overline{x} \in \z_m$: $\overline{x} \otimes \overline{1} = \overline{x}$.
	\end{enumerate}
\end{proposicao}
\begin{prova}
	\begin{enumerate}[label={\roman*})]
		\item $\overline{x} \oplus \overline{y} = \overline{x + y} = \overline{y + x} = \overline{y} \oplus \overline{x}$.
		
		\item $(\overline{x} \oplus \overline{y}) \oplus \overline{z} = \overline{x + y} \oplus \overline{z} = \overline{(x + y) + z} = \overline{x + (y + z)} = \overline{x} \oplus \overline{y + z} = \overline{x} \oplus (\overline{y} \oplus \overline{z})$.

		\item $\overline{x} \oplus \overline{0} = \overline{x + 0} = \overline{x}$.

		\item Dado $\overline{x} \in \z_m$ escolha $\overline{y} = \overline{m - x} \in \z_m$. Assim $\overline{x} \oplus \overline{y} = \overline{x} \oplus \overline{m - x} = \overline{x + (m - x)} = \overline{m} = \overline{0}$.

		\item $\overline{x} \otimes \overline{y} = \overline{x \cdot y} = \overline{y \cdot x} = \overline{y} \otimes \overline{x}$.

		\item $(\overline{x} \otimes \overline{y}) \otimes \overline{z} = \overline{x \cdot y} \otimes \overline{z} = \overline{(x \cdot y)\cdot z} = \overline{x\cdot(y \cdot z)} = \overline{x} \otimes \overline{y \cdot z} = \overline{x} \otimes (\overline{y}\otimes \overline{z})$.

		\item $\overline{x} \otimes \overline{1} = \overline{x \cdot 1} = \overline{x}$.
	\end{enumerate}
\end{prova}

\begin{definicao}
	Um elemento $\overline{a} \in \z_m$ {\'e} \textbf{invers{\'\i}vel} se, e somente se, existe $\overline{b} \in \z_m$ tal que $\overline{a} \otimes \overline{b} = \overline{1}$. Neste caso, $\overline{b}$ {\'e} chamado \textbf{inverso} de $\overline{a}$ e denotaremos $\overline{b} = (\overline{a})^{-1}$.
\end{definicao}

\begin{proposicao}
	Se o inverso existe, ent\~ao ele {\'e} {\'u}nico.
\end{proposicao}
\begin{prova}
	De fato, dado $\overline{a} \in \z_m$, suponha que existem $\overline{b}$, $\overline{d} \in \z_m$ tais que $\overline{a} \otimes \overline{b} = \overline{1} = \overline{a} \otimes \overline{d}$, ent{\~a}o
	\begin{align*}
		\overline{b} &= \overline{b} \otimes \overline{1} = \overline{b} \otimes (\overline{a} \otimes \overline{d})\\ &= (\overline{b} \otimes \overline{a}) \otimes \overline{d} = \overline{1} \otimes \overline{d} = \overline{d}
	\end{align*}
\end{prova}

\begin{proposicao}
	Um elemento $\overline{a} \in \z_m$ {\'e} invers{\'\i}vel se, e somente se, $mdc(a,m)=1$.
\end{proposicao}

\begin{corolario}
	Se $m$ \'e um n\'umero primo, ent\~ao para todo $\overline{x} \in \z_m$, $\overline{x} \ne \overline{0}$, existe inverso.
\end{corolario}

\begin{exemplos}
	\begin{enumerate}[label={\arabic*})]
		\item Em $\z_4$ existem dois elementos invers{\'\i}veis que s{\~a}o $\overline{1}$, cujo inverso {\'e} $\overline{1}$, e o $\overline{3}$, cujo inverso {\'e} $\overline{3}$.
		\item Em $\z_{11}$, todos elementos, exceto $\overline{0}$, possuem inverso:

		\begin{table}[h]
   			\centering 
   			\setlength{\arrayrulewidth}{0,5\arrayrulewidth}
   			\caption{\it Inversos em $\z_{11}$}
		   \begin{tabular}{|c|c|c|c|c|c|c|c|c|c|c|} 
		    	\hline
		      	Elemento & $\overline{1}$ & $\overline{2}$ & $\overline{3}$ & $\overline{4}$ & $\overline{5}$ & $\overline{6}$ & $\overline{7}$ & $\overline{8}$ & $\overline{9}$ & $\overline{10}$\T \\
		      	\hline
		      	Inverso & $\overline{1}$ & $\overline{6}$ & $\overline{4}$ & $\overline{3}$ & $\overline{9}$ & $\overline{2}$ & $\overline{8}$ & $\overline{7}$ & $\overline{5}$ & $\overline{10}$\T \\
		      	\hline
		   \end{tabular}
		\end{table}
	\end{enumerate}
\end{exemplos}


% \section{Rela\c{c}\~oes de Ordem} % (fold)
% \label{sec:relacoes_de_ordem}

% \begin{definicao}
% 	Seja $A \ne \emptyset$ e $R \sub A \times A$. Dizemos que $R$ \'e uma \textbf{rela\c{c}\~ao de ordem parcial sobre} $A$ se:
% 	\begin{enumerate}[label={\roman*})]
% 		\item Para todo $x \in A$, $xRx$.
% 		\item Se $xRy$ e $yRx$, ent\~ao $x = y$.
% 		\item Se $xRy$ e $yRz$, ent\~ao $xRz$.
% 	\end{enumerate}
% \end{definicao}

% Quando $R$ \'e uma rela\c{c}\~ao de ordem parcial sobre $A$, para dizer que $(x,y) \in R$ vamos usar a nota\c{c}\~ao $x\preceq y\ (R)$ significando ``$x$ \textit{precede $y$ na rela\c{c}\~ao $R$}''.

% Para denotar que $(x,y) \in R$ e que $x \ne y$ usaremos a nota\c{c}\~ao $x \prec y\ (R)$ significando ``$x$ \textit{precede estritamente $y$ na rela\c{c}\~ao $R$}''.

% \begin{exemplos}\label{exemplos_relacoes_de_ordem}
% 	\begin{enumerate}[label={\arabic*})]
% 		\item Seja $A = \{a,b,c,d\}$. Quais dos seguintes cojuntos s\~ao rela\c{c}\~oes de ordem?
% 		\begin{align*}
% 			R_1 &= \{(a,a);(b,b);(c,c)\}\\
% 			R_2 &= \{(a,a);(b,b);(c,c);(d,d)\}\\
% 			R_3 &= \{(a,a);(b,b);(c,c);(d,d);(a,b);(b,a)\}\\
% 			R_4 &= \{(a,a);(b,b);(c,c);(d,d);(a;b);(c,a);(c,b);(a,d);(c,d)\}\\
% 		\end{align*}
% 		\begin{solucao}
% 			$R_1$ n\~ao \'e rela\c{c}\~ao de ordem pois $(d,d) \notin R_1$.

% 			$R_3$ n\~ao \'e rela\c{c}\~ao de ordem pois $(a,b)$, $(b,a) \in R_3$ no entanto $a \ne b$.

% 			$R_2$ e $R_4$ s\~ao rela\c{c}\~oes de ordem.
% 		\end{solucao}

% 		\item A rela\c{c}\~ao $R$ sobre $\real$ definida por
% 		\[
% 			xRy \mbox{ se, e somente se, } x \leqslant y
% 		\]
% 		\'e uma rela\c{c}\~ao de ordem sobre $\real$.
% 		\begin{solucao}
% 			De fato,
% 			\begin{enumerate}[label={\roman*})]
% 				\item Para todo $x \in \real$, $x \leqslant x$, isto \'e, $xRx$.
% 				\item Se $xRy$ e $yRx$, ent\~ao $x \leqslant y$ e $y \leqslant x$. Logo $x = y$.
% 				\item Se $xRy$ e $yRz$, ent\~ao $x \leqslant y$ e $y \leqslant z$. Logo $x \leqslant z$, isto \'e, $xRz$.
% 			\end{enumerate}
% 		\end{solucao}

% 		\item Seja $A = \n$ e $R \sub A \times A$ definido por
% 		\[
% 			R = \{(x,y) \in \n \times \n : x \mid y\}.
% 		\]
% 		Ent\~ao $R$ \'e uma rela\c{c}\~ao de ordem sobre $\n$.
% 		\begin{solucao}
% 			De fato,
% 			\begin{enumerate}[label={\roman*})]
% 				\item Para todo $x \in \n$, $x \mid x$. Logo $(x,x) \in R$.
% 				\item Se $(x,y)\in R$ e $(y,x) \in R$, ent\~ao $x \mid y$ e $y \mid x$. Como $x$, $y \in \n$ ent\~ao $x = y$.
% 				\item Se $(x,y) \in R$ e $(y,z) \in R$, ent\~ao $x \mid y$ e $y \mid z$. Logo $x \mid z$, isto \'e, $xRz$.
% 			\end{enumerate}
% 		\end{solucao}
% 	\end{enumerate}
% \end{exemplos}

% \begin{definicoes}
% 	Seja $A \ne \emptyset$ e $R$ uma rela\c{c}\~ao de ordem sobre $A$.
% 	\begin{enumerate}[label={\roman*})]
% 		\item Um \textbf{conjunto parcialmente ordenador} \'e um conjunto sobre o qual se definiu uma certa rela\c{c}\~ao de ordem parcial.
% 		\item Dados $x$, $y \in A$ dizemos que $x$ e $y$ s\~ao \textbf{compar\'aveis mediante} $R$ se $x \preceq y$ ou $y \preceq x$.
% 		\item Se quaisquer dois elementos de $A$ forem compar\'aveis mediante $R$, ent\~ao $R$ \'e chamada de \textbf{rela\c{c}\~ao de ordem total sobre} $A$. Nesse caso dizemos que $A$ \'e um \textbf{conjunto totalmente ordenado}.
% 	\end{enumerate}
% \end{definicoes}

% \begin{exemplos}
% 	No Exemplo \ref{exemplos_relacoes_de_ordem} temos:
% 	\begin{enumerate}[label={\arabic*})]
% 		\item $R_4$ \'e uma rela\c{c}\~ao de ordem total sobre $A$.
% 		\item $R$ \'e uma rela\c{c}\~ao de ordem total sobre $\real$.
% 		\item A rela\c{c}\~ao de divibilidade sobre $\n$ n\~ao \'e uma ordem total pois, por exemplo, $2 \not{\mid} 3$ e $3 \not{\mid} 2$.
% 	\end{enumerate}
% \end{exemplos}

% \begin{definicoes}
% 	Seja $A$ um conjunto parcialmente ordenado mediante a rela\c{c}\~ao $\preceq$. Seja $B \sub A$ com $B \ne \emptyset$.
% 	\begin{enumerate}[label={\roman*})]
% 		\item Um elemento $l \in A$ \'e um \textbf{limite superior} de $B$ se para todo $x \in B$ temos $x \preceq l$.
% 		\item Um elemento $m \in A$ \'e um \textbf{limite inferior} de $B$ se para todo $x \in B$, $m \preceq x$.
% 	\end{enumerate}
% \end{definicoes}

% \begin{exemplo}
% 	Considere $\real$ com a ordem $\leqslant$.
% 	\begin{enumerate}[label={\arabic*})]
% 		\item Seja $A = \{x \in \real \mid x < 2\}$. Ent\~ao $l = 2$ \'e um limite superior para $A$. Assim como qualquer n\'umero real maior ou igaul a 2 tamb\'em ser\'a. Note que $A$ n\~ao possui limite inferior.
% 		\item Seja $B = \{x \in \real \mid x \geqslant -3\}$. Ent\~ao $m = -3$ \'e um limite inferior para $B$. Nesse caso n\~ao existe limite superior para $B$.
% 		\item Seja $C = [2,3)$. Aqui $m = 2$ \'e um limite inferior para $C$ e $l = 3$ \'e um limite superior para $C$.
% 	\end{enumerate}
% \end{exemplo}

% \begin{exemplo}
% 	Seja $A = \n$ com a rela\c{c}\~ao de ordem parcial $x \mid y$.
% 	\begin{enumerate}[label={\arabic*})]
% 		\item Seja $A = \{2,4,8,16\}$. Aqui $l = 32$ \'e um limite superior para $A$ pois $x\mid 32$ para todo $x \in A$. Al\'em disso, $m = 2$ \'e um limite inferior para $A$ pois $2 \mid y$ para todo $y \in A$.
% 		\item Seja $B = \{1,3,5,7,9,\dots\}$. Aqui $m = 1$ \'e um limite inferior para $B$ mas n\~ao existe limite superior.
% 	\end{enumerate}
% \end{exemplo}

% \begin{definicoes}
% 	Seja $B$ um subconjunto n\~ao vazio de um conjunto parcialmente ordenado $A$ pela rela\c{c}\~ao $\preceq$.
% 	\begin{enumerate}[label={\roman*})]
% 		\item Um elemento $\alpha \in B$ \'e um \textbf{m\'aximo} de $B$ se para todo $x \in B$, temos $x \preceq \alpha$, isto \'e, quando $\alpha$ \'e um limite superior de $B$ e pertence a $B$.
% 		\item Um elemento $\beta \in B$ \'e um \textbf{m{\'\i}nimo} de $B$ se para todo $y \in B$, temos $\beta \preceq y$, isto \'e, quando $\beta$ \'e um limite inferior de $B$ e pertence a $B$.
% 	\end{enumerate}
% \end{definicoes}

% \begin{proposicao}
% 	Seja $B$ \'e um subconjunto n\~ao vazio do conjunto parcialmente ordenado $A$. Se $B$ possui um m\'aximo, ou m{\'\i}nimo, ent\~ao ele \'e \'unico.
% \end{proposicao}
% \begin{prova}
% 	Faremos a demonstra\c{c}\~ao somente para o m\'aximo. O caso do m{\'\i}nimo \'e an\'alogo.

% 	Como $A$ \'e parcialmente ordenado, existe $R \sub A \times A$ que \'e uma rela\c{c}\~ao de ordem parcial em $A$.
% 	Suponha que $\alpha_1$ e $\alpha_2$ sejam m\'aximos de $B$. Ent\~ao como $\alpha_1$ \'e m\'aximo de $B$ e $\alpha_2 \in B$, ent\~ao
% 	\[
% 		\alpha_2 \preceq \alpha_1.
% 	\]
% 	Agora, como $\alpha_2$ \'e m\'aximo de $B$ e $\alpha_1 \in B$, ent\~ao
% 	\[
% 		\alpha_1 \preceq \alpha_2.
% 	\]
% 	Ou seja, $\alpha_1 R \alpha_2$ e $\alpha_2R\alpha_1$ e como $R$ \'e rela\c{c}\~ao de ordem parcial segue que $\alpha_1 = \alpha_2$, ou seja, o m\'aximo de $B$ \'e \'unico.
% \end{prova}

% section rela\c{c}\~oes_de_ordem (end)