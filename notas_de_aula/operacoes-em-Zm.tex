%!TEX program = xelatex
%!TEX root = Algebra_1.tex
%%Usar makeindex -s indexstyle.ist arquivo.idx no terminal para gerar o índice remissivo agrupado por inicial
%%Ap\'os executar pdflatex arquivo
\chapter{Operações em $\dfrac{\mathbb{Z}}{m\mathbb{Z}}$}

Durante esse tópico, $m$ denotará um número inteiro positivo.

\section{Relações de congruência}
\subsection{Definição}

\begin{definicao}[Congruência] Sejam $a,b\in\mathbb{Z}$, dizemos que $a$ é congruente com $b$ módulo $m$ se $m|(a-b)$. Neste caso, escrevemos $a\equiv_{m}b$ ou $a\equiv b(mod\ m)$.\end{definicao}

Exemplos:
\begin{enumerate}
\item $5\equiv 2(mod\ 3)$, pois $3|(5-2)$
\item $3\equiv 1(mod\ 2)$, pois $2|(3-1)$
\item $3\equiv 9(mod\ 3)$, pois $2|(3-9)$

\end{enumerate}
\subsection{Propriedades}
\begin{proposicao} A congruência módulo $m$ é uma relação de equivalência em $\mathbb{Z}$.\end{proposicao}

\textbf{Demonstração}:
\begin{enumerate}
\item $\forall a\in\mathbb{Z},a\equiv a(mod\ m)$ pois $m|(a-a)$ (Reflexidade)
\item Se $a\equiv b(mod\ m)$, então $m|(a-b)$. Daí, $m|(-(a-b))$, ou seja, $m|(b-a)$. Daí $b\equiv a(mod\ a)$ (Simetria)
\item Se $a\equiv b(mod\ m)$ e $b\equiv c(mod\ m)$, então $m|(a-b)$ e $m|(b-c)$. Assim, $m|[(a-b)+(b-c)]$. Logo, $m|(a-c)$, isto é, $a\equiv c(mod\ m)$ (Transitividade)

Portanto é relação de equivalência. \#

\end{enumerate}

\begin{teorema} A relação de congruência módulo $m$ satisfaz as seguintes propriedades:
\begin{enumerate}
\item $a_{1}\equiv b_{1}(mod\ m)\Leftrightarrow a_{1}-b_{1}\equiv 0(mod\ m)$
\item Se $a_{1}\equiv b_{1}(mod\ m)$ e $a_{2}\equiv b_{2}(mod\ m)$, então $a_{1}+a_{2}\equiv b_{1}+b_{2}(mod\ m)$
\item Se $a_{1}\equiv b_{2}(mod\ m)$ e $a_{2}\equiv b_{2}(mod\ m)$, então $a_{1}a_{2}\equiv b_{1}b_{2}(mod\ m)$
\item Se $a\equiv b(mod\ m)$, então $ax\equiv bx(mod\ m), \forall x\in\mathbb{Z}$
\item Vale a lei do cancelamento: se $d\in\mathbb{Z}$ e $mdc(d,m)=1$ então\\ $ad\equiv bd(mod\ m)$ implica $a\equiv b(mod\ m)$

\end{enumerate}
\end{teorema}

\textbf{Demonstração}: Provemos o ítem 3

Dizer que $a\equiv b(mod\ m)$ significa dizer que existe $t\in\mathbb{Z}$ tal que $a=b+tm$.

Assim, existem $m,l\in\mathbb{Z}$ tais que $a_{1}=b_{1}+km, a_{2}=b_{2}+lm$. Daí\\
\[a_{1}a_{2}=b_{1}b_{2}+lb_{1}m+klm^{2}\]\\
\[a_{1}a_{2}=b_{1}b_{2}+\underbrace{(lb_{1}+kb_{2}+klm)}_{\in\mathbb{Z}}m\]

Ou seja, $a_{1}a_{2}=b_{1}b_{2}+pm$, onde $p=lb_{1}+kb_{2}+klm\in\mathbb{Z}$. Portanto, $a_{1}a_{2}\equiv b_{1}b_{2}(mod\ m)$.

Para o ítem 5, se $ad\equiv bd(mod\ m)$, então $m|d(a-b)$. Mas, $mdc(d,m)=1$, logo $m|(a-b)$, isto é, $a\equiv b(mod\ m)$.\#

Como a congruência módulo $m$ é uma relação de equivalência, podemos determinar suas classes de equivalência. Assim, dado $n\in\mathbb{Z}$, temos
\[C(n)=\{x\in\mathbb{Z}/x\equiv n(mod\ m)\}\]

Denotaremos $C(n)$ por $R_{m}(n)$ ou $\bar{n}$, quando não houver possibilidade de confusão.

Por exemplo, fixando $m$\\
$R_{m}(0)=\{x\in\mathbb{Z}/x\equiv 0(mod\ m)\}=\{x\in \mathbb{Z}/x=mk, k\in\mathbb{Z}\}=m\mathbb{Z}$\\
$R_{m}(1)=\{x\in\mathbb{Z}/x\equiv 1(mod\ m)\}=\{x\in\mathbb{Z}/x=1+km, k\in\mathbb{Z}\}$\\
$R_{m}(n)=\{x\in\mathbb{Z}/x=n+km, k\in\mathbb{Z}\}$

\subsection{Classes de equivalência módulo $m$}

\begin{proposicao} As classes de equivalência definidas pela congruência módulo $m$ são determinadas pelos restos da divisão euclidiana por $m$. Em outras palavras, $R_{m}(n)$ é o conjunto dos números inteiros cujo resto na divisão euclidiana por $m$ é $n$.\end{proposicao}

\textbf{Demonstração}: Dado $x\in\mathbb{Z}$, pela divisão de Euclides, podemos escrever $x=km+r$ onde $0\leq r < m$. Daí, $x-r=km$, isto é, $m|(x-r)$. Logo $x\in R_{m}(r)$. Portanto, se $r=n$, então $x\in R_{m}(n)$ e neste caso, $x=km+n=n+km$, ou seja, o resto da divisão euclidiana de $x$ por $m$ é $n$.\#

\begin{corolario} $R_{m}(k)=R_{m}(l)$ se, e somente se, $k\equiv l(mod\ m)$.\end{corolario}

Exemplos:
\begin{enumerate}
\item Se $m=2$, então os possíveis restos na divisão euclidiana por 2 são 0 e 1. Logo, existem duas classes de equivalência, a saber $R_{2}(0)$ e $R_{2}(1)$
\item Se $m=3$, então os possíveis restos da divisão euclidiana são 0,1 e 2. Daí\\
$R_{3}(0)=3\mathbb{Z}$\\
$R_{3}(1)=\{x\in\mathbb{Z}/x=3q+1,q\in\mathbb{Z}\}$\\
$R_{3}(2)=\{x\in\mathbb{Z}/x=3q+2,q\in\mathbb{Z}\}$

\end{enumerate}

\begin{proposicao} Na relação de equivalência módulo $m$ existem $m$ classes de equivalência.\end{proposicao}

\textbf{Demonstração}: Os possíveis restos na divisão euclidiana por $m$ são $0,1,...,(m-1)$. Como cada possível resto define uma classe de equivalência diferente, existem exatamente $m$ classes de equivalência.\#

\section{Conjunto quociente $\left(\dfrac{\mathbb{Z}}{m\mathbb{Z}}\right)$}


\begin{nota}[Conjunto quociente] Fixado $m$ inteiro positivo, denotaremos\\
$R_{m}(0)=\bar{0}$\\
$R_{m}(1)=\bar{1}$\\
$\vdots$\\
$R_{m}(m-1)=\overline{m-1}$

O conjunto quociente desta relação será denotado por $\displaystyle\frac{\mathbb{Z}}{m\mathbb{Z}}$ e $\displaystyle\frac{\mathbb{Z}}{m\mathbb{Z}}=\{\bar{0},\bar{1},...,\overline{m-1}\}$
\end{nota}

Queremos definir um meio de somar e multiplicar os elementos de $\displaystyle\frac{\mathbb{Z}}{m\mathbb{Z}}$. Por exemplo, em $\displaystyle\frac{\mathbb{Z}}{2\mathbb{Z}}=\{\bar{0},\bar{1}\}$ temos que a soma de pares é par, soma de par com ímpar é ímpar e a soma de ímpares é par.

Podemos escrever
\vspace{0,05cm}\\
$\bar{0}\oplus\bar{0}=\overline{0+0}=\bar{0}$\\
$\bar{0}\oplus\bar{1}=\overline{0+1}=\bar{1}$\\
$\bar{1}\oplus\bar{1}=\overline{1+1}=\bar{0}$

Para multiplicação, temos
\vspace{0,05cm}\\
$\bar{0}\odot\bar{0}=\overline{0.0}=\bar{0}$\\
$\bar{0}\odot\bar{1}=\overline{0.1}=\bar{0}$\\
$\bar{1}\odot\bar{1}=\overline{1.1}=\bar{1}$

Em $\displaystyle\frac{\mathbb{Z}}{m\mathbb{Z}}$ definimos
\begin{eqnarray}
\bar{a}\oplus\bar{b}=\overline{a+b}\\
\bar{a}\odot\bar{b}=\overline{a.b}
\end{eqnarray}
Para $\bar{a},\bar{b}\in\displaystyle\frac{\mathbb{Z}}{m\mathbb{Z}}$

\begin{proposicao} As operações de soma e produto definidas em (5.1) e (5.2) são independentes dos representantes das classes.\end{proposicao}

\textbf{Demonstração}: Dadas duas classes com representantes diferentes, $\bar{a}_{1}=\bar{a}_{2},\  \bar{b}_{1}=\bar{b}_{2}, a_{1}\ne a_{2}, b_{1}\ne b_{2}$, temos:\\
\[\overline{a_{1}+b_{1}}=\bar{a}_{1}\oplus\bar{b}_{1}=\bar{a}_{2}\oplus\bar{b}_{2}=\overline{a_{2}+b_{2}}\]\\
\[\overline{a_{1}b_{1}}=\bar{a}_{1}\odot\bar{b}_{1}=\bar{a}_{2}\odot\bar{b}_{2}=\overline{a_{2}b_{2}}\]\\

C.Q.D.\#

Exemplo: Determine a some e multiplicação em:

$\displaystyle\frac{\mathbb{Z}}{4\mathbb{Z}}=\{\bar{0},\bar{1},\bar{2},\bar{3}\}$
\begin{table}[h]
   \centering
   \setlength{\arrayrulewidth}{0,5\arrayrulewidth}
   \caption{\it Soma}
   \begin{tabular}{|c|c|c|c|c|}
      \hline
      $\oplus$ & $\bar{0}$ & $\bar{1}$ & $\bar{2}$ & $\bar{3}$ \\
      \hline
      $\bar{0}$ & $\bar{0}$ & $\bar{1}$ & $\bar{2}$ & $\bar{3}$ \\
      \hline
      $\bar{1}$ & $\bar{1}$ & $\bar{2}$ & $\bar{3}$ & $\bar{0}$ \\
      \hline
      $\bar{2}$ & $\bar{2}$ & $\bar{3}$ & $\bar{0}$ & $\bar{1}$ \\
      \hline
      $\bar{3}$ & $\bar{3}$ & $\bar{0}$ & $\bar{1}$ & $\bar{2}$ \\
      \hline
   \end{tabular}
\end{table}

\begin{table}[h]
   \centering
   \setlength{\arrayrulewidth}{0,5\arrayrulewidth}
   \caption{\it Multiplicação}
   \begin{tabular}{|c|c|c|c|c|}
      \hline
      $\odot$ & $\bar{0}$ & $\bar{1}$ & $\bar{2}$ & $\bar{3}$ \\
      \hline
      $\bar{0}$ & $\bar{0}$ & $\bar{0}$ & $\bar{0}$ & $\bar{0}$ \\
      \hline
      $\bar{1}$ & $\bar{0}$ & $\bar{1}$ & $\bar{2}$ & $\bar{3}$ \\
      \hline
      $\bar{2}$ & $\bar{0}$ & $\bar{2}$ & $\bar{0}$ & $\bar{2}$ \\
      \hline
      $\bar{3}$ & $\bar{0}$ & $\bar{3}$ & $\bar{2}$ & $\bar{1}$ \\
      \hline
   \end{tabular}
\end{table}


\subsection{Elementos Inversíveis de $\displaystyle\frac{\mathbb{Z}}{m\mathbb{Z}}$}

\subsubsection{Inversibilidade}
\begin{definicao}[Inversibilidade] Um elemento $\bar{a}\in\displaystyle\frac{\mathbb{Z}}{m\mathbb{Z}}$ é inversível se, e somente se, existem $\bar{b}\in\displaystyle\frac{\mathbb{Z}}{m\mathbb{Z}}$ tal que $\bar{a}\odot\bar{b}=\bar{1}$.\end{definicao}

Neste caso, $\bar{b}$ é chamado inverso de $\bar{a}$ e denotaremos $\bar{b}=(\bar{a})^{-1}$.

Quando $\bar{b}$ existe, ele é único. De fato, dado $\bar{a}\in\displaystyle\frac{\mathbb{Z}}{m\mathbb{Z}}$, se existem $\bar{b},\bar{d}\in\displaystyle\frac{\mathbb{Z}}{m\mathbb{Z}}$ tais que $\bar{a}\odot\bar{b}=\bar{1}=\bar{a}\odot\bar{d}$, então $\bar{b}=\bar{b}\odot\bar{1}=\bar{b}\odot (\bar{a}\odot\bar{d})=(\bar{b}\odot\bar{a})\odot\bar{d}=\bar{1}\odot\bar{d}=\bar{d}$.

\begin{proposicao} Um elemento $\bar{a}\in\displaystyle\frac{\mathbb{Z}}{m\mathbb{Z}}$ é inversível se, e somente se,
 \[mdc(a,m)=1\]
\end{proposicao}

\textbf{Demonstração}: Suponha que existe $\bar{b}\in\displaystyle\frac{\mathbb{Z}}{m\mathbb{Z}}$ tal que $\bar{a}\odot\bar{b}=\bar{1}$. Assim, $\overline{ab}=\bar{1}$, ou seja, $ab\equiv 1(mod\ m)$. Daí, $ab-1=km, k\in\mathbb{Z}$, logo $ab+m(-k)=1$, e então $mdc(a,m)=1$.

Agora suponha que $mdc(a,m)=1$. Logo, existem $x_{0}, y_{0}\in\mathbb{Z}$ tais que $ax_{0}+my_{0}=1$, isto é, $ax_{0}-1=m(-y_{0})$. Logo $ax_{0}\equiv 1(mod\ m)$, ou seja, $\overline{ax_{0}}=\bar{1}$. Portanto, $\bar{a}\odot\bar{x_{0}}=\bar{1}$.\#

Exemplos:
\begin{enumerate}
\item Em $\displaystyle\frac{\mathbb{Z}}{4\mathbb{Z}}$ existem dois elementos inversíveis que são $\bar{1}$, cujo inverso é $\bar{1}$, e o $\bar{3}$, cujo inverso é $\bar{3}$.
\item Em $\displaystyle\frac{\mathbb{Z}}{11\mathbb{Z}} $, todos elementos, exceto $\bar{0}$, possuem inverso:

\begin{table}[h]
   \centering
   \setlength{\arrayrulewidth}{0,5\arrayrulewidth}
   \caption{\it Inversos em $\displaystyle\frac{\mathbb{Z}}{11\mathbb{Z}}$}
   \begin{tabular}{|c|c|c|c|c|c|c|c|c|c|c|}
      \hline
      Elemento & $\bar{1}$ & $\bar{2}$ & $\bar{3}$ & $\bar{4}$ & $\bar{5}$ & $\bar{6}$ & $\bar{7}$ & $\bar{8}$ & $\bar{9}$ & $\bar{10}$ \\
      \hline
      Inverso & $\bar{1}$ & $\bar{6}$ & $\bar{4}$ & $\bar{3}$ & $\bar{9}$ & $\bar{2}$ & $\bar{8}$ & $\bar{7}$ & $\bar{5}$ & $\bar{10}$ \\
      \hline
   \end{tabular}
\end{table}
\end{enumerate}

O número de elementos inversíveis de $\displaystyle\frac{\mathbb{Z}}{m\mathbb{Z}}$ é igual a quantidade de números coprimos com $m$. Esse número é denotado por $\varphi(m)$ e é chamado função $\varphi$ de Euler. Pode-se demonstrar que
\[\varphi(m)=m\displaystyle\prod_{p/m}\left(1-\displaystyle\frac{1}{p}\right)\]\\
Onde o produto varia sobre todos os divisores primos de m, sem repetição.

Por exemplo, para $\displaystyle\frac{\mathbb{Z}}{100\mathbb{Z}}$ temos:\\
$100=2^{2}5^{2}$

Daí,\\
$\varphi(100)=100\left(1-\displaystyle\frac{1}{2}\right)(1-\displaystyle{1}{5})=40$

Logo, em $\displaystyle\frac{\mathbb{Z}}{100\mathbb{Z}}$ existem 40 elementos inversíveis.

\begin{nota}[Conjunto dos elementos inversíveis] Denotaremos o conjunto de todos os elementos inversíveis de $\displaystyle\frac{\mathbb{Z}}{m\mathbb{Z}}$ por $\left(\displaystyle\frac{\mathbb{Z}}{m\mathbb{Z}}\right)^{*}$, ou ainda $U\left(\displaystyle\frac{\mathbb{Z}}{m\mathbb{Z}}\right)$.\end{nota}

\begin{proposicao} Sejam $\bar{a},\bar{b}\in\left(\displaystyle\frac{\mathbb{Z}}{m\mathbb{Z}}\right)^{*}$. Então $\bar{a}\odot\bar{b}\in\left(\displaystyle\frac{\mathbb{Z}}{m\mathbb{Z}}\right)^{*}$.\end{proposicao}

\textbf{Demonstração}: Por uma proposição anterior, basta verificar que\\ $mdc(ab,m)=1$. Para que $\bar{a}\odot\bar{b}\in\left(\displaystyle\frac{\mathbb{Z}}{m\mathbb{Z}}\right)^{*}$.

Como $\bar{a},\bar{b}\in\left(\displaystyle\frac{\mathbb{Z}}{m\mathbb{Z}}\right)^{*}$, então $mdc(a,m)=1$ e $mdc(b,m)=1$.

Assim, existem $x_{0},y_{0},x_{1},y_{1}\in\mathbb{Z}$ tais que\\
$ax_{0}+my_{0}=1$\\
$bx_{1}+my_{1}=1$

Daí,
\[abx_{0}x_{1}+max_{0}y_{1}+mbx_{1}y_{0}+m^{2}y_{0}y_{1}=1\]
\[\underbrace{abx_{0}x_{1}}_{\in\mathbb{Z}}+m\underbrace{(ax_{0}y_{1}+bx_{1}y_{0}+my_{0}y_{1})}_{\in\mathbb{Z}}=1\]

Logo, $mdc(ab,m)=1$, ou seja, $\bar{a}\odot\bar{b}\in\left(\displaystyle\frac{\mathbb{Z}}{m\mathbb{Z}}\right)^{*}$.\#