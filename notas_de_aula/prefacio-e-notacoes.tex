%!TEX program = xelatex
%!TEX root = Algebra_1.tex
%%Usar makeindex -s indexstyle.ist arquivo.idx no terminal para gerar o í}ndice remissivo agrupado por inicial
%%Após executar pdflatex arquivo
\begin{center}
\Huge \textbf{Prefácio}
\end{center}

Essas notas de Aula são referentes à matéria Álgebra 1,
ministrada na UnB - Universidade de Brasília - durante o 2$^o$ Semestre de 2010
pelo professor José Antônio O. de Freitas, Departamento de Matemática. Tais
notas foram transcritas e editadas pelo graduando em Ciências Econômicas
Luiz Eduardo Sol R. da Silva\footnote{luizeduardosol@hotmail.com}.

Revisão e ampliação das notas feita por José Antônio O. de Freitas\footnote{jfreitas@mat.unb.br}.


É livre a reprodução, distribuição e edição deste material, desde que citadas as suas fontes e autores. Críticas e sugestões são bem vindas.
\vspace{20cm}




% \begin{center} \textbf{\Large Notações e expressões}
% \end{center}
% \begin{minipage}[l]{0,5\textwidth}
% \begin{itemize}
% \item $\neg$ Não
% \item $\forall$ Para todo
% \item $/$ Tal que
% \item $|$ Divide
% \item $\Rightarrow$ Implica
% \item $\in$ Pertence
% \item $\emptyset$ Vazio
% \item $\subseteq$ Contido ou igual a
% \item $\supseteq$ Contém ou igual a
% \item $\wedge$ E
% \item $\vee$ Ou
% \item $=$ Igual
% \item $\neq$ Diferente
% \item $\mathbb{Z}$ N{\'u}meros Inteiros
% \item $\mathbb{R}$ N{\'u}meros Reais
% \item $\cap$ Intersecção
% \item $>$ Maior que
% \item $\geq$ Maior ou igual a
% \item $\displaystyle\bigcup_{i=1}^{n}$ União de $n$ conjuntos
% \item $\displaystyle\bigsqcup_{i=1}^{n}$ União disjunta de $n$ conjuntos


% \end{itemize}
% \end{minipage}
% \begin{minipage}[r]{0,5\textwidth}
% \begin{itemize}

% \item $\leftrightarrow$ Se, e somente se
% \item $\veebar$ Ou...,ou..., mas nunca ambos
% \item $\rightarrow$ Se,... então...
% \item $\exists$ Existe
% \item $\Leftrightarrow$ Equivalente a
% \item $\notin$ Não pertence
% \item \# Fim da demonstração
% \item $\mathbb{N}$ N{\'u}meros Naturais
% \item $\mathbb{Q}$ N{\'u}meros Racionais
% \item $\nsubseteq$ Não contém ou é igual a
% \item $\cup$ União
% \item $\sqcup$ União Disjunta
% \item $<$ Menor que
% \item $\leq$ Menor ou igual a
% \item $\displaystyle\bigcap_{i=1}^{n}$ Intersecção de $n$ conjuntos
% \item Q.E.D. (\textit{Quod Erat Demonstrandum}): Como se queria demonstrar
% \item P.B.O.: Princí}pio da boa ordenação
% \item H.I.: Hip{ó}tese de Indução
% \item \textit{Mutatis Mutandis}:  Mudando o que tem que ser mudado

% \end{itemize}
% \end{minipage}