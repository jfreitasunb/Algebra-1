%!TEX program = xelatex
%!TEX root = Algebra_1.tex
%%Usar makeindex -s indexstyle.ist arquivo.idx no terminal para gerar o índice remissivo agrupado por inicial
%%Após executar pdflatex arquivo
\chapter{Noções de Lógica}

\section{Conceitos básicos}

\hspace{0,5cm}O estudo da lógica proporciona instrumentos de pensamento para determinar a correão ou incorreão de todos os raciocínios.

A lógica pode não nos levar à verdade no sentido absoluto, mas nos permite descobrir a incoerência e o erro em um argumento.

\subsection{Proposição}

\begin{definicao}[Proposição]: Chama-se proposição todo conjunto de palavras ou símbolos que exprime um pensamento de sentido completo.
\end{definicao}

São exemplos de proposições:
\begin{enumerate}
\item A Lua é um satélite da Terra
\item $\pi>\sqrt{5}$
\end{enumerate}

\subsection{Valor de uma proposição}
\begin{definicao}[Valor de uma Proposição]Chama-se valor de uma proposição a verdade se a proposição é verdadeira e a falsidade se for falsa.\end{definicao}

\subsection{Princípios fundamentais}
A lógica matemática adota como regras fundamentais os dois seguintes princípios (ou axiomas):
\begin{enumerate}
\item \textbf{Princípio da não contradição}: uma proposição não pode ser verdadeira e falsa ao mesmo tempo
\item \textbf{Princípio do Terceiro excluído}: Toda proposição ou é verdadeira ou é falsa, isto é, verifica-se sempre um desses casos e nunca um terceiro
\end{enumerate}
\subsection{Argumento lógico}
\begin{definicao}[Argumento lógico]: Um argumento lógico é uma seqüência de proposições , na qual uma das seqüências é a conclusão e as demais, chamadas de premissas, formam as provas ou evidências para a conclusão.\end{definicao}

Exemplos:

\textbf{Argumento 1}: [Como todo brasileiro é sul americano](1ª proposição/ premissa) e [todo brasiliense é brasileiro](2ª proposição/premissa), então [todo brasiliense é brasileiro](3ª proposição/conclusão).

\textbf{Argumento 2}: [Como todo matemático é louco](1ª proposição/ premissa) e [eu sou matemático](2ª proposição/ premissa), então [eu sou louco](3ª proposição, conclusão).

Um argumento é válido quando suas proposições, se verdadeiras, fornecem provas convincenetes para sua conclusão, isto é, quando as proposições e a conclusão estão de tal modo relacionados que é absolutamente impossível as proposições serem verdadeiras se a conclusão não for.

\subsection{Proposições simples e compostas}

As proposições podem ser classificadas em simples (ou atômicas) e compostas.

Chama-se proposição simples aquela que não contém nenhuma outra proposição como parte integrante de si mesma.

Exemplos:
\begin{enumerate}
\item O número 25 é um quadrado perfeito
\item $\pi>\sqrt{5}$
\end{enumerate}

Chama-se proposição composta a que é formada pela combinaão de duas ou mais proposições.

Exemplo: $3>4$ ou $4>3$

\subsection{Conectivos}

\begin{definicao}[Conectivos]Chamam-se conectivos palavras que se usam para formar novas proposições a partir de outras.\end{definicao}

Em lógica matemática os conectivos usuais são os seguintes:
\begin{itemize}
\item ``E" (Conjunão)
\item ``OU" (Disjunão)
\item ``Não" (Negaão)
\item ``Se... então" (Condicional)
\item ``Se, e somente se..." (Bicondicional)
\end{itemize}

Vamos denotar as proposições simples por letras minúsculas (a,b,c...) e as proposições compostas por letras maiúsculas (A,B,C...)

Exemplo:\\
\textbf{$p$}: o número 6 é par\\
\textbf{$q$}: o numero 8 é um cubo perfeito\\
\textbf{$P$}: O número 6 é par E o número 8 é um cubo perfeito\\
\textbf{$Q$}: O número 6 é par OU 8 é um cubo perfeito

Se p é uma proposição, vamos denotar seu valor lógico por $V(p)$
\begin{center}
p: ou $V(p)=V$ ou $V(p)=F$
\end{center}

Vamos usar a notaão (Tabela \ref{notacao}), chamada Tabela Verdade (Tabela \ref{tabelavdd})
\begin{table}[h]
   \centering
   \setlength{\arrayrulewidth}{0,5\arrayrulewidth}

   \caption{\it Notaão}
   \begin{tabular}{|c|}
      \hline
      p \\
      \hline
      V \\
      \hline
      F \\
      \hline
   \end{tabular}
\label{notacao}
\end{table}


\begin{table}[h]
   \centering
   \setlength{\arrayrulewidth}{0,5\arrayrulewidth}
   \caption{\it Tabela Verdade}
   \begin{tabular}{|c|c|c|c|c|}
      \hline
      p & q \\
      \hline
      V & V \\
      \hline
      V & F\\
      \hline
      F & V \\
      \hline
      F & F \\
      \hline
   \end{tabular}
\label{tabelavdd}
\end{table}

\subsubsection{Negaão}
\begin{definicao}[Negaão] Chama-se negaão de uma proposição p a proposição representada por ``não p"($\neg p$), cujo valor lógico é verdade quando p for falsa e a falsidade quando p for verdadeira.\end{definicao}
\begin{table}[h]
   \centering
   \setlength{\arrayrulewidth}{0,5\arrayrulewidth}
   \caption{\it Tabela verdade: Negaão}
   \begin{tabular}{|c|c|c|c|c|}
      \hline
     $ p$ & $\neg p$ \\
      \hline
      V & F \\
      \hline
      F & V \\
      \hline
   \end{tabular}
\end{table}

Exemplo:\\
\textbf{$p$}: 6 é par, $V(p)=V$\\
\textbf{$\neg p$}: 6 é ímpar, $V(p)=F$\\
\textbf{$q$}: $4\leq 5$, $V(q)=V$\\
\textbf{$\neg q$}: $4>5$, $V(q)=F$

\subsubsection{Conjugaão}

\begin{definicao}[Conjugaão] Chama-se conjugaão de duas proposições $p$ e $q$, a proposição representada por ``$p$ e $q$", denotada $p\wedge q$, cujo valor lógico é verdade quando as proposições $p$ e $q$ são ambas verdadeiras e falsidade nos demais casos.\end{definicao}
\begin{table}[h]
   \centering
   \setlength{\arrayrulewidth}{0,5\arrayrulewidth}
   \caption{\it Tabela verdade: Conjugaão}
   \begin{tabular}{|c|c|c|c|c|}
      \hline
      $p$ & $q$ & $p\wedge q$ \\
      \hline
      V & V & V \\
      \hline
      V & F & F \\
      \hline
      F & V & F \\
      \hline
      F & F & F \\
      \hline
   \end{tabular}
\end{table}

Exemplo:\\
\textbf{$p$}: a neve é branca, $V(p)=V$\\
\textbf{$q$}: $2<5$, $V(q)=V$
\begin{center}
\textbf{$p\wedge q$}: A neve é branca E $2<5$, $V(p\wedge q)=V$
\end{center}

\textbf{$r$}: todo número primo e ímpar, $V(r)=F$
\begin{center}
$V(q\wedge r)=F$
\end{center}

\subsubsection{Disjunão}
\begin{definicao}[Disjunão]: Chama-se disjunão de duas proposições $p$ e $q$ a proposição representada por ``p ou q", denotada ``$p\vee q$", cujo valor lógico é verdade quando ao menos uma das proposições $p$ e $q$ forem verdadeiras, e falsidade quando ambas as proposições $p$ e $q$ forem falsas.\end{definicao}

A tabela verdade da disjunão é (Tabela \ref{disjuncao}):
\begin{table}[h]
   \centering
   \setlength{\arrayrulewidth}{0,5\arrayrulewidth}
   \caption{\it Tabela verdade: Disjunão}
   \begin{tabular}{|c|c|c|c|c|}
      \hline
      $p$ & $q$ & $p\vee q$ \\
     \hline
      V & V & V \\
      \hline
      V & F & V \\
      \hline
      F & V & V \\
      \hline
      F & F & F \\
      \hline
   \end{tabular}
\label{disjuncao}
\end{table}

Exemplo:\\
\textbf{$P$}: Roma é a capital da Rússia ou 9-5=4, $V(P)=V$\\
\textbf{$Q$}: $\pi=3\vee\sqrt{-1}=1$, $V(Q)=F$

\subsubsection{Disjunão exclusiva}
\begin{definicao}[Disjunão Exclusiva] Chama-se disjunão exclusiva de duas proposições $p$ e $q$ a proposição representada por $p\veebar q$, que se lê ``ou $p$ ou $q$", ou também ``$p$ ou $q$, mas não ambos", cujo valor lógico é a verdade somente quando $p$ é verdade ou $q$ é verdade, mas não quando $p$ e $q$ são ambos verdadeiras, e tem valor lógico falsidade nos demais casos.\end{definicao}
\begin{table}[h]
   \centering
   \setlength{\arrayrulewidth}{0,5\arrayrulewidth}
   \caption{\it Tabela verdade: Disjunão Exclusiva}
   \begin{tabular}{|c|c|c|c|c|}
      \hline
      $p$ & $q$ & $p\veebar q$ \\
     \hline
      V & V & F \\
      \hline
      V & F & V \\
      \hline
      F & V & V \\
      \hline
      F & F & F \\
      \hline
   \end{tabular}
\end{table}

Exemplo:\\
\textbf{$P$}: Todo número inteiro ou é par ou é ímpar, $V(P)=V$

\subsubsection{Condicional}
\begin{definicao}[Condicional] Chama-se proposição condicional ou condicional uma proposição representada por ``$p\rightarrow q$", que lê-se ``se $p$ então $q$". O valor lógico da condicional é a falsidade no caso em que $p$ é verdade e $q$ é falsidade e tem valor lógico verdade nos demais casos.\end{definicao}

Na condicional ``$p\rightarrow q$", dizemos que $p$ é o antecedente e o $q$ é o conseqüente. O símbolo "$\rightarrow$" é chamado de símbolo de implicaão.
\begin{table}[h]
   \centering
   \setlength{\arrayrulewidth}{0,5\arrayrulewidth}
   \caption{\it Tabela verdade: Condicional}
   \begin{tabular}{|c|c|c|c|c|}
      \hline
      $p$ & $q$ & $p\rightarrow q$ \\
     \hline
      V & V & V \\
      \hline
      V & F & F \\
      \hline
      F & V & V \\
      \hline
      F & F & V \\
      \hline
   \end{tabular}
\end{table}

Exemplos:\\
\textbf{$P$}: Se o mês de maio tem 31 dias, então a terra é plana, $V(P)=F$\\
\textbf{$Q$}: Se 3+2=6 então 4+4=9, $V(Q)=V$\\
\textbf{$R$}: Se (-1)=0, então $\sin\displaystyle\frac{\pi}{6}=\displaystyle\frac{1}{2}$\\

Considere a seguinte condicional:
\begin{center}
7 é um número ímpar $\rightarrow$ Brasília é uma cidade
\end{center}

Observaão: Uma condicional não afirma que o conseqüente se deduz ou é uma conseqüência do antecedente $p$. Uma condicional afirma unicamente uma relaão entre valores lógicos de $p$ e $q$.

\subsubsection{Bicondicional}

\begin{definicao}[Bicondicional] Chama-se proposição bicondicional ou bicondicional uma proposição representada por ``Se, e somente se", denotada ``$p\leftrightarrow q(p\rightarrow q\wedge q\rightarrow p)$". O valor lógico da bicondicional é verdade quando $p$ e $q$ são ambas verdade ou falsidade, e tem valor lógico falsidade nos demais casos.\end{definicao}
\begin{table}[h]
   \centering
   \setlength{\arrayrulewidth}{0,5\arrayrulewidth}
   \caption{\it Tabela verdade: Bicondicional}
   \begin{tabular}{|c|c|c|c|c|}
      \hline
      $p$ & $q$ & $p\leftrightarrow q$ \\
     \hline
      V & V & V \\
      \hline
      V & F & F \\
      \hline
      F & V & F \\
      \hline
      F & F & V \\
      \hline
   \end{tabular}
\end{table}

Exemplos:\\
\textbf{$P$}: Roma fica na Europa se, e somente se, a neve é branca, $V(P)=V$\\
\textbf{$Q$}: Lisboa é a capital de Portugal se, e somente se, $\tan\displaystyle\frac{\pi}{4}=3$, $V(Q)=F$
\section{Tabela Verdade}

\hspace{0,5cm}Dadas várias proposições simples $p,q,r,...$, podemos combiná-las formando novas proposições através do uso dos conectivos lógicos. Por exemplo:\\
\textbf{$P$}:$\neg p\vee(p\rightarrow q)$\\
\textbf{$R$}:$(p\leftrightarrow q)\wedge q$\\
\textbf{$S$}:$(p\rightarrow\neg q\vee r)\vee(q\vee(p\leftrightarrow\neg r))$

\subsubsection{Ordem de precedência}

Na lógica matemática, é convencionado a seguinte ordem de precedência dos operadores:
\begin{enumerate}
\item $\neg$
\item $\wedge,\vee$
\item $\rightarrow,\leftrightarrow$
\end{enumerate}

Duas regras importantes devem ser observadas
\begin{enumerate}
\item A ordem de precedência de uma operaão lógica somente pode ser alterada através do uso de parênteses
\item Operadores diferentes e de mesma prioridade necessariamente devem ter sua ordem indicada pelo uso de parênteses
\end{enumerate}

Exemplos
\begin{enumerate}
\item $p\vee q\vee r\leftrightarrow\neg p$ (Correto)
\item $p\vee q\vee(r\leftrightarrow\neg p)$ (Correto)
\item $p\wedge q\vee r$ (Errado)
\item $p\rightarrow q\leftrightarrow p$ (Errado)
\end{enumerate}

Observaão: A colocaão de parênteses pode alterar o valor lógico (e o sentido) de uma proposição

\begin{center}
Exemplo
\end{center}
\begin{minipage}[l]{0,5\textwidth}
I)\\
$V\vee F\rightarrow F$\\
$V\rightarrow F$\\
$F$\\
\end{minipage}
\begin{minipage}[r]{0,5\textwidth}
II)\\
$V\vee(F\rightarrow F)$\\
$V\vee V$\\
$V$

\end{minipage}

\section{Construão de Tabelas Verdade}

\hspace{0,5cm}Para construir a tabela verdade de uma proposição composta $P(p,q,r,...)$ começamos contando o número de proposições simples que comp{\~o}em $P(p,q,r,...)$.

\subsubsection{Número de linhas}
Se $P(p,q,r,...)$ for composta por $n$ proposições simples, então a tabela verdade de $P(p,q,r,...)$ conterá $2^{n}$ linhas.

Exemplos: Construir a tabela verdade das seguintes proposições:
\begin{enumerate}
\item $P:\neg(p\wedge\neg q)$ (Tabela \ref{1})
\begin{table}[h]
   \centering
   \setlength{\arrayrulewidth}{0,5\arrayrulewidth}
   \caption{\it $P:\neg(p\wedge\neg q)$}
   \begin{tabular}{|c|c|c|c|c|}
      \hline
      $p$ & $q$ & $\neg q$ & $p\wedge\neg q$ & $\neg(p\wedge\neg q)$ \\
     \hline
      V & V & F & F & V \\
      \hline
      V & F & V & V & F \\
      \hline
      F & V & F & F & V \\
      \hline
      F & F & V & F & V \\
      \hline
   \end{tabular}
\label{1}
\end{table}
\item $Q:\neg(p\wedge q)\vee\neg(p\leftrightarrow p)$ (Tabela \ref{2})
\begin{table}[h]
   \centering
   \setlength{\arrayrulewidth}{0,5\arrayrulewidth}
   \caption{\it $Q:\neg(p\wedge q)\vee\neg(p\leftrightarrow p)$}
   \begin{tabular}{|c|c|c|c|c|c|c|}
      \hline
      $p$ & $q$ & $p\wedge q$ & $p\leftrightarrow q$ & $\neg(p\wedge q)$ & $\neg(p\leftrightarrow q)$ & Q \\
     \hline
      V & V & V & V & F & F & F \\
      \hline
      V & F & F & F & V & V & V\\
      \hline
      F & V & F & F & V & V & V \\
      \hline
      F & F & F & V & V & V & V \\
      \hline
   \end{tabular}
\label{2}
\end{table}
\end{enumerate}

\subsection{Tautologia}
\begin{definicao}[Tautologia] Chama-se tautologia a proposição composta que é sempre verdadeira independentemente dos valores lógicos das proposições que a comp{\~o}em.\end{definicao}

Na tabela verdade de uma tautologia, a última coluna contém somente o valor lógico verdadeiro.

\subsection{Contradição}
\begin{definicao}[Contradição] Chama-se contradição a proposição composta que é sempre falsa independentemente dos valores lógicos das proposições que a comp{\~o}em.\end{definicao}

Assim, na tabela verdade de uma contradição, a última coluna contém somente o valor lógico falso.

\section{Implicaão}
\begin{definicao}[Implicaão] Dizemos que uma proposição $P(p,q,r,..)$ implica uma proposição composta $Q(p,q r,...)$, denotado "$P\Rightarrow Q$", se para todo valor verdade da primeira, então a segunda é verdadeira.\end{definicao}

Assim, $P\Rightarrow Q$ somente se a condicional $P\rightarrow Q$ for uma tautologia.

Exemplo: Sendo $P:p\wedge q$ e $Q:p\vee q$, verificar se $P\Rightarrow Q$.

Precisamos verificar se a condicional $P\rightarrow Q$ é uma tautologia (Tabela \ref{3}).
\begin{table}[h]
   \centering
   \setlength{\arrayrulewidth}{0,5\arrayrulewidth}
   \caption{\it $P\rightarrow Q$}
   \begin{tabular}{|c|c|c|c|c|}
      \hline
      $p$ & $q$ & $p\wedge q$ & $p\vee q$ & $(p\wedge q)\rightarrow(p\vee q)$ \\
     \hline
      V & V & V & V & V \\
      \hline
      V & F & F & V & V \\
      \hline
      F & V & F & V & V \\
      \hline
      F & F & F & F & V \\
      \hline
   \end{tabular}
\label{3}
\end{table}

Observaão:
\begin{enumerate}
\item Os símbolos $\rightarrow$ e $\Rightarrow$ são diferentes. A condicional, $\rightarrow$, é um operador lógico que aplicado a duas proposições $p$ e $q$, por exemplo, produz uma nova proposição $p\rightarrow q$. Por outro lado, a implicaão, $\Rightarrow$, estabelece que $p\rightarrow q$ é uma tautologia.
\item Toda teorema é uma aplicaão da forma
\begin{center}
Hipótese $\Rightarrow$ Tese
\end{center}

Logo demonstrar um teorema significa mostrar que não ocorre o caso da hipótese ser verdadeira e a tese falsa, isto é, a verdade da hipótese é suficiente para garantir a verdade da tese.
\end{enumerate}

\subsection{Equivalência}

\begin{definicao}[Equivalência] Dizemos que uma proposição $P(p,q,r,...)$ é equivalente a uma proposição composta $Q(p,q,r,...)$, denotado por $P\Leftrightarrow Q$, se elas implicarem uma na outra. Assim, $P\Leftrightarrow Q$ se a bicondicional é uma tautologia.\end{definicao}

Exemplo: Sendo $P:p\leftrightarrow q$ e $Q:(p\rightarrow q)\wedge(q\rightarrow p)$ verificar que $P\Leftrightarrow Q$.\\

Precisamos verificar se a bicondicional $P\leftrightarrow Q$ é uma tautologia (Tabela \ref{4}).
\begin{table}[h]
   \centering
   \setlength{\arrayrulewidth}{0,5\arrayrulewidth}
   \caption{\it $P\leftrightarrow Q$}
   \begin{tabular}{|c|c|c|c|c|c|c|}
      \hline
      $p$ & $q$ & $p\rightarrow q$ & $q\leftrightarrow p$ & $(p\rightarrow q)\wedge(q\rightarrow p)$ & $p\leftrightarrow q$ & $P\leftrightarrow Q$ \\
     \hline
      V & V & V & V & V & V & V \\
      \hline
      V & F & F & V & F & F & V\\
      \hline
      F & V & V & F & F & F & V \\
      \hline
      F & F & V & V & V & V & V \\
      \hline
   \end{tabular}
\label{4}
\end{table}