%!TEX program = xelatex
%!TEX root = Algebra_1.tex
%%Usar makeindex -s indexstyle.ist arquivo.idx no terminal para gerar o {\'\i}ndice remissivo agrupado por inicial
%%Ap\'os executar pdflatex arquivo
\chapter{No{\c c}{\~o}es de L{\'o}gica}

\section{Conceitos b{\'a}sicos}

\hspace{0,5cm}O estudo da l{\'o}gica proporciona instrumentos de pensamento para determinar a corre{\c c}{\~a}o ou incorre{\c c}{\~a}o de todos os racioc{\'\i}nios.

A l{\'o}gica pode n{\~a}o nos levar {\`a} verdade no sentido absoluto, mas nos permite descobrir a incoer{\^e}ncia e o erro em um argumento.

\subsection{Proposi{\c c}{\~a}o}

\begin{definicao}[Proposi{\c c}{\~a}o]: Chama-se proposi{\c c}{\~a}o todo conjunto de palavras ou s{\'\i}mbolos que exprime um pensamento de sentido completo.
\end{definicao}

S{\~a}o exemplos de proposi{\c c}{\~o}es:
\begin{enumerate}
\item A Lua {\'e} um sat{\'e}lite da Terra
\item $\pi>\sqrt{5}$
\end{enumerate}

\subsection{Valor de uma proposi{\c c}{\~a}o}
\begin{definicao}[Valor de uma Proposi{\c c}{\~a}o]Chama-se valor de uma proposi{\c c}{\~a}o a verdade se a proposi{\c c}{\~a}o {\'e} verdadeira e a falsidade se for falsa.\end{definicao}

\subsection{Princ{\'\i}pios fundamentais}
A l{\'o}gica matem{\'a}tica adota como regras fundamentais os dois seguintes princ{\'\i}pios (ou axiomas):
\begin{enumerate}
\item \textbf{Princ{\'\i}pio da n{\~a}o contradi{\c c}{\~a}o}: uma proposi{\c c}{\~a}o n{\~a}o pode ser verdadeira e falsa ao mesmo tempo
\item \textbf{Princ{\'\i}pio do Terceiro exclu{\'\i}do}: Toda proposi{\c c}{\~a}o ou {\'e} verdadeira ou {\'e} falsa, isto {\'e}, verifica-se sempre um desses casos e nunca um terceiro
\end{enumerate}
\subsection{Argumento l{\'o}gico}
\begin{definicao}[Argumento l{\'o}gico]: Um argumento l{\'o}gico {\'e} uma seq{\"u}{\^e}ncia de proposi{\c c}{\~o}es , na qual uma das seq{\"u}{\^e}ncias {\'e} a conclus{\~a}o e as demais, chamadas de premissas, formam as provas ou evid{\^e}ncias para a conclus{\~a}o.\end{definicao}

Exemplos:

\textbf{Argumento 1}: [Como todo brasileiro {\'e} sul americano](1ª proposi{\c c}{\~a}o/ premissa) e [todo brasiliense {\'e} brasileiro](2ª proposi{\c c}{\~a}o/premissa), ent{\~a}o [todo brasiliense {\'e} brasileiro](3ª proposi{\c c}{\~a}o/conclus{\~a}o).

\textbf{Argumento 2}: [Como todo matem{\'a}tico {\'e} louco](1ª proposi{\c c}{\~a}o/ premissa) e [eu sou matem{\'a}tico](2ª proposi{\c c}{\~a}o/ premissa), ent{\~a}o [eu sou louco](3ª proposi{\c c}{\~a}o, conclus{\~a}o).

Um argumento {\'e} v{\'a}lido quando suas proposi{\c c}{\~o}es, se verdadeiras, fornecem provas convincenetes para sua conclus{\~a}o, isto {\'e}, quando as proposi{\c c}{\~o}es e a conclus{\~a}o est{\~a}o de tal modo relacionados que {\'e} absolutamente imposs{\'\i}vel as proposi{\c c}{\~o}es serem verdadeiras se a conclus{\~a}o n{\~a}o for.

\subsection{Proposi{\c c}{\~o}es simples e compostas}

As proposi{\c c}{\~o}es podem ser classificadas em simples (ou at{\^o}micas) e compostas.

Chama-se proposi{\c c}{\~a}o simples aquela que n{\~a}o cont{\'e}m nenhuma outra proposi{\c c}{\~a}o como parte integrante de si mesma.

Exemplos:
\begin{enumerate}
\item O n{\'u}mero 25 {\'e} um quadrado perfeito
\item $\pi>\sqrt{5}$
\end{enumerate}

Chama-se proposi{\c c}{\~a}o composta a que {\'e} formada pela combina{\c c}{\~a}o de duas ou mais proposi{\c c}{\~o}es.

Exemplo: $3>4$ ou $4>3$

\subsection{Conectivos}

\begin{definicao}[Conectivos]Chamam-se conectivos palavras que se usam para formar novas proposi{\c c}{\~o}es a partir de outras.\end{definicao}

Em l{\'o}gica matem{\'a}tica os conectivos usuais s{\~a}o os seguintes:
\begin{itemize}
\item ``E" (Conjun{\c c}{\~a}o)
\item ``OU" (Disjun{\c c}{\~a}o)
\item ``N{\~a}o" (Nega{\c c}{\~a}o)
\item ``Se... ent{\~a}o" (Condicional)
\item ``Se, e somente se..." (Bicondicional)
\end{itemize} 

Vamos denotar as proposi{\c c}{\~o}es simples por letras min{\'u}sculas (a,b,c...) e as proposi{\c c}{\~o}es compostas por letras mai{\'u}sculas (A,B,C...)

Exemplo:\\
\textbf{$p$}: o n{\'u}mero 6 {\'e} par\\
\textbf{$q$}: o numero 8 {\'e} um cubo perfeito\\
\textbf{$P$}: O n{\'u}mero 6 {\'e} par E o n{\'u}mero 8 {\'e} um cubo perfeito\\
\textbf{$Q$}: O n{\'u}mero 6 {\'e} par OU 8 {\'e} um cubo perfeito

Se p {\'e} uma proposi{\c c}{\~a}o, vamos denotar seu valor l{\'o}gico por $V(p)$
\begin{center}
p: ou $V(p)=V$ ou $V(p)=F$
\end{center}

Vamos usar a nota{\c c}{\~a}o (Tabela \ref{notacao}), chamada Tabela Verdade (Tabela \ref{tabelavdd})
\begin{table}[h]
   \centering 
   \setlength{\arrayrulewidth}{0,5\arrayrulewidth}
   
   \caption{\it Nota{\c c}{\~a}o}
   \begin{tabular}{|c|} 
      \hline
      p \\
      \hline
      V \\
      \hline
      F \\
      \hline
   \end{tabular}
\label{notacao}
\end{table}


\begin{table}[h]
   \centering 
   \setlength{\arrayrulewidth}{0,5\arrayrulewidth}
   \caption{\it Tabela Verdade}
   \begin{tabular}{|c|c|c|c|c|} 
      \hline
      p & q \\
      \hline
      V & V \\
      \hline
      V & F\\
      \hline
      F & V \\
      \hline
      F & F \\
      \hline
   \end{tabular}
\label{tabelavdd}
\end{table}

\subsubsection{Nega{\c c}{\~a}o}
\begin{definicao}[Nega{\c c}{\~a}o] Chama-se nega{\c c}{\~a}o de uma proposi{\c c}{\~a}o p a proposi{\c c}{\~a}o representada por ``n{\~a}o p"($\neg p$), cujo valor l{\'o}gico {\'e} verdade quando p for falsa e a falsidade quando p for verdadeira.\end{definicao}
\begin{table}[h]
   \centering 
   \setlength{\arrayrulewidth}{0,5\arrayrulewidth}
   \caption{\it Tabela verdade: Nega{\c c}{\~a}o}
   \begin{tabular}{|c|c|c|c|c|} 
      \hline
     $ p$ & $\neg p$ \\
      \hline
      V & F \\
      \hline
      F & V \\
      \hline
   \end{tabular}
\end{table}

Exemplo:\\
\textbf{$p$}: 6 {\'e} par, $V(p)=V$\\
\textbf{$\neg p$}: 6 {\'e} {\'\i}mpar, $V(p)=F$\\
\textbf{$q$}: $4\leq 5$, $V(q)=V$\\
\textbf{$\neg q$}: $4>5$, $V(q)=F$

\subsubsection{Conjuga{\c c}{\~a}o}

\begin{definicao}[Conjuga{\c c}{\~a}o] Chama-se conjuga{\c c}{\~a}o de duas proposi{\c c}{\~o}es $p$ e $q$, a proposi{\c c}{\~a}o representada por ``$p$ e $q$", denotada $p\wedge q$, cujo valor l{\'o}gico {\'e} verdade quando as proposi{\c c}{\~o}es $p$ e $q$ s{\~a}o ambas verdadeiras e falsidade nos demais casos.\end{definicao}
\begin{table}[h]
   \centering 
   \setlength{\arrayrulewidth}{0,5\arrayrulewidth}
   \caption{\it Tabela verdade: Conjuga{\c c}{\~a}o}
   \begin{tabular}{|c|c|c|c|c|} 
      \hline
      $p$ & $q$ & $p\wedge q$ \\
      \hline
      V & V & V \\
      \hline
      V & F & F \\
      \hline
      F & V & F \\
      \hline
      F & F & F \\
      \hline
   \end{tabular}
\end{table}

Exemplo:\\
\textbf{$p$}: a neve {\'e} branca, $V(p)=V$\\
\textbf{$q$}: $2<5$, $V(q)=V$
\begin{center}
\textbf{$p\wedge q$}: A neve {\'e} branca E $2<5$, $V(p\wedge q)=V$
\end{center}

\textbf{$r$}: todo n{\'u}mero primo e {\'\i}mpar, $V(r)=F$
\begin{center}
$V(q\wedge r)=F$
\end{center}

\subsubsection{Disjun{\c c}{\~a}o}
\begin{definicao}[Disjun{\c c}{\~a}o]: Chama-se disjun{\c c}{\~a}o de duas proposi{\c c}{\~o}es $p$ e $q$ a proposi{\c c}{\~a}o representada por ``p ou q", denotada ``$p\vee q$", cujo valor l{\'o}gico {\'e} verdade quando ao menos uma das proposi{\c c}{\~o}es $p$ e $q$ forem verdadeiras, e falsidade quando ambas as proposi{\c c}{\~o}es $p$ e $q$ forem falsas.\end{definicao}

A tabela verdade da disjun{\c c}{\~a}o {\'e} (Tabela \ref{disjuncao}):
\begin{table}[h]
   \centering 
   \setlength{\arrayrulewidth}{0,5\arrayrulewidth}
   \caption{\it Tabela verdade: Disjun{\c c}{\~a}o}
   \begin{tabular}{|c|c|c|c|c|} 
      \hline
      $p$ & $q$ & $p\vee q$ \\
     \hline
      V & V & V \\
      \hline
      V & F & V \\
      \hline
      F & V & V \\
      \hline
      F & F & F \\
      \hline
   \end{tabular}
\label{disjuncao}
\end{table}

Exemplo:\\
\textbf{$P$}: Roma {\'e} a capital da R{\'u}ssia ou 9-5=4, $V(P)=V$\\
\textbf{$Q$}: $\pi=3\vee\sqrt{-1}=1$, $V(Q)=F$

\subsubsection{Disjun{\c c}{\~a}o exclusiva}
\begin{definicao}[Disjun{\c c}{\~a}o Exclusiva] Chama-se disjun{\c c}{\~a}o exclusiva de duas proposi{\c c}{\~o}es $p$ e $q$ a proposi{\c c}{\~a}o representada por $p\veebar q$, que se l{\^e} ``ou $p$ ou $q$", ou tamb{\'e}m ``$p$ ou $q$, mas n{\~a}o ambos", cujo valor l{\'o}gico {\'e} a verdade somente quando $p$ {\'e} verdade ou $q$ {\'e} verdade, mas n{\~a}o quando $p$ e $q$ s{\~a}o ambos verdadeiras, e tem valor l{\'o}gico falsidade nos demais casos.\end{definicao}
\begin{table}[h]
   \centering 
   \setlength{\arrayrulewidth}{0,5\arrayrulewidth}
   \caption{\it Tabela verdade: Disjun{\c c}{\~a}o Exclusiva}
   \begin{tabular}{|c|c|c|c|c|} 
      \hline
      $p$ & $q$ & $p\veebar q$ \\
     \hline
      V & V & F \\
      \hline
      V & F & V \\
      \hline
      F & V & V \\
      \hline
      F & F & F \\
      \hline
   \end{tabular}
\end{table}

Exemplo:\\
\textbf{$P$}: Todo n{\'u}mero inteiro ou {\'e} par ou {\'e} {\'\i}mpar, $V(P)=V$

\subsubsection{Condicional}
\begin{definicao}[Condicional] Chama-se proposi{\c c}{\~a}o condicional ou condicional uma proposi{\c c}{\~a}o representada por ``$p\rightarrow q$", que l{\^e}-se ``se $p$ ent{\~a}o $q$". O valor l{\'o}gico da condicional {\'e} a falsidade no caso em que $p$ {\'e} verdade e $q$ {\'e} falsidade e tem valor l{\'o}gico verdade nos demais casos.\end{definicao}

Na condicional ``$p\rightarrow q$", dizemos que $p$ {\'e} o antecedente e o $q$ {\'e} o conseq{\"u}ente. O s{\'\i}mbolo "$\rightarrow$" {\'e} chamado de s{\'\i}mbolo de implica{\c c}{\~a}o.
\begin{table}[h]
   \centering 
   \setlength{\arrayrulewidth}{0,5\arrayrulewidth}
   \caption{\it Tabela verdade: Condicional}
   \begin{tabular}{|c|c|c|c|c|} 
      \hline
      $p$ & $q$ & $p\rightarrow q$ \\
     \hline
      V & V & V \\
      \hline
      V & F & F \\
      \hline
      F & V & V \\
      \hline
      F & F & V \\
      \hline
   \end{tabular}
\end{table}

Exemplos:\\
\textbf{$P$}: Se o m{\^e}s de maio tem 31 dias, ent{\~a}o a terra {\'e} plana, $V(P)=F$\\
\textbf{$Q$}: Se 3+2=6 ent{\~a}o 4+4=9, $V(Q)=V$\\
\textbf{$R$}: Se (-1)=0, ent{\~a}o $\sin\displaystyle\frac{\pi}{6}=\displaystyle\frac{1}{2}$\\

Considere a seguinte condicional:
\begin{center}
7 {\'e} um n{\'u}mero {\'\i}mpar $\rightarrow$ Bras{\'\i}lia {\'e} uma cidade
\end{center}

Observa{\c c}{\~a}o: Uma condicional n{\~a}o afirma que o conseq{\"u}ente se deduz ou {\'e} uma conseq{\"u}{\^e}ncia do antecedente $p$. Uma condicional afirma unicamente uma rela{\c c}{\~a}o entre valores l{\'o}gicos de $p$ e $q$.

\subsubsection{Bicondicional}

\begin{definicao}[Bicondicional] Chama-se proposi{\c c}{\~a}o bicondicional ou bicondicional uma proposi{\c c}{\~a}o representada por ``Se, e somente se", denotada ``$p\leftrightarrow q(p\rightarrow q\wedge q\rightarrow p)$". O valor l{\'o}gico da bicondicional {\'e} verdade quando $p$ e $q$ s{\~a}o ambas verdade ou falsidade, e tem valor l{\'o}gico falsidade nos demais casos.\end{definicao}
\begin{table}[h]
   \centering 
   \setlength{\arrayrulewidth}{0,5\arrayrulewidth}
   \caption{\it Tabela verdade: Bicondicional}
   \begin{tabular}{|c|c|c|c|c|} 
      \hline
      $p$ & $q$ & $p\leftrightarrow q$ \\
     \hline
      V & V & V \\
      \hline
      V & F & F \\
      \hline
      F & V & F \\
      \hline
      F & F & V \\
      \hline
   \end{tabular}
\end{table}

Exemplos:\\
\textbf{$P$}: Roma fica na Europa se, e somente se, a neve {\'e} branca, $V(P)=V$\\
\textbf{$Q$}: Lisboa {\'e} a capital de Portugal se, e somente se, $\tan\displaystyle\frac{\pi}{4}=3$, $V(Q)=F$
\section{Tabela Verdade}

\hspace{0,5cm}Dadas v{\'a}rias proposi{\c c}{\~o}es simples $p,q,r,...$, podemos combin{\'a}-las formando novas proposi{\c c}{\~o}es atrav{\'e}s do uso dos conectivos l{\'o}gicos. Por exemplo:\\
\textbf{$P$}:$\neg p\vee(p\rightarrow q)$\\
\textbf{$R$}:$(p\leftrightarrow q)\wedge q$\\
\textbf{$S$}:$(p\rightarrow\neg q\vee r)\vee(q\vee(p\leftrightarrow\neg r))$

\subsubsection{Ordem de preced{\^e}ncia}

Na l{\'o}gica matem{\'a}tica, {\'e} convencionado a seguinte ordem de preced{\^e}ncia dos operadores:
\begin{enumerate}
\item $\neg$
\item $\wedge,\vee$
\item $\rightarrow,\leftrightarrow$
\end{enumerate}

Duas regras importantes devem ser observadas
\begin{enumerate}
\item A ordem de preced{\^e}ncia de uma opera{\c c}{\~a}o l{\'o}gica somente pode ser alterada atrav{\'e}s do uso de par{\^e}nteses
\item Operadores diferentes e de mesma prioridade necessariamente devem ter sua ordem indicada pelo uso de par{\^e}nteses
\end{enumerate}

Exemplos
\begin{enumerate}
\item $p\vee q\vee r\leftrightarrow\neg p$ (Correto)
\item $p\vee q\vee(r\leftrightarrow\neg p)$ (Correto)
\item $p\wedge q\vee r$ (Errado)
\item $p\rightarrow q\leftrightarrow p$ (Errado)
\end{enumerate}

Observa{\c c}{\~a}o: A coloca{\c c}{\~a}o de par{\^e}nteses pode alterar o valor l{\'o}gico (e o sentido) de uma proposi{\c c}{\~a}o

\begin{center}
Exemplo
\end{center} 
\begin{minipage}[l]{0,5\textwidth}
I)\\
$V\vee F\rightarrow F$\\
$V\rightarrow F$\\
$F$\\
\end{minipage}
\begin{minipage}[r]{0,5\textwidth}
II)\\
$V\vee(F\rightarrow F)$\\
$V\vee V$\\
$V$

\end{minipage}

\section{Constru{\c c}{\~a}o de Tabelas Verdade}

\hspace{0,5cm}Para construir a tabela verdade de uma proposi{\c c}{\~a}o composta $P(p,q,r,...)$ come{\c c}amos contando o n{\'u}mero de proposi{\c c}{\~o}es simples que comp{\~o}em $P(p,q,r,...)$.

\subsubsection{N{\'u}mero de linhas}
Se $P(p,q,r,...)$ for composta por $n$ proposi{\c c}{\~o}es simples, ent{\~a}o a tabela verdade de $P(p,q,r,...)$ conter{\'a} $2^{n}$ linhas.

Exemplos: Construir a tabela verdade das seguintes proposi{\c c}{\~o}es:
\begin{enumerate}
\item $P:\neg(p\wedge\neg q)$ (Tabela \ref{1})
\begin{table}[h]
   \centering 
   \setlength{\arrayrulewidth}{0,5\arrayrulewidth}
   \caption{\it $P:\neg(p\wedge\neg q)$}
   \begin{tabular}{|c|c|c|c|c|} 
      \hline
      $p$ & $q$ & $\neg q$ & $p\wedge\neg q$ & $\neg(p\wedge\neg q)$ \\
     \hline
      V & V & F & F & V \\
      \hline
      V & F & V & V & F \\
      \hline
      F & V & F & F & V \\
      \hline
      F & F & V & F & V \\
      \hline
   \end{tabular}
\label{1}
\end{table}
\item $Q:\neg(p\wedge q)\vee\neg(p\leftrightarrow p)$ (Tabela \ref{2})
\begin{table}[h]
   \centering 
   \setlength{\arrayrulewidth}{0,5\arrayrulewidth}
   \caption{\it $Q:\neg(p\wedge q)\vee\neg(p\leftrightarrow p)$}
   \begin{tabular}{|c|c|c|c|c|c|c|} 
      \hline
      $p$ & $q$ & $p\wedge q$ & $p\leftrightarrow q$ & $\neg(p\wedge q)$ & $\neg(p\leftrightarrow q)$ & Q \\
     \hline
      V & V & V & V & F & F & F \\
      \hline
      V & F & F & F & V & V & V\\
      \hline
      F & V & F & F & V & V & V \\
      \hline
      F & F & F & V & V & V & V \\
      \hline
   \end{tabular}
\label{2}
\end{table}
\end{enumerate}

\subsection{Tautologia}
\begin{definicao}[Tautologia] Chama-se tautologia a proposi{\c c}{\~a}o composta que {\'e} sempre verdadeira independentemente dos valores l{\'o}gicos das proposi{\c c}{\~o}es que a comp{\~o}em.\end{definicao}

Na tabela verdade de uma tautologia, a {\'u}ltima coluna cont{\'e}m somente o valor l{\'o}gico verdadeiro.

\subsection{Contradi{\c c}{\~a}o}
\begin{definicao}[Contradi{\c c}{\~a}o] Chama-se contradi{\c c}{\~a}o a proposi{\c c}{\~a}o composta que {\'e} sempre falsa independentemente dos valores l{\'o}gicos das proposi{\c c}{\~o}es que a comp{\~o}em.\end{definicao}

Assim, na tabela verdade de uma contradi{\c c}{\~a}o, a {\'u}ltima coluna cont{\'e}m somente o valor l{\'o}gico falso.

\section{Implica{\c c}{\~a}o}
\begin{definicao}[Implica{\c c}{\~a}o] Dizemos que uma proposi{\c c}{\~a}o $P(p,q,r,..)$ implica uma proposi{\c c}{\~a}o composta $Q(p,q r,...)$, denotado "$P\Rightarrow Q$", se para todo valor verdade da primeira, ent{\~a}o a segunda {\'e} verdadeira.\end{definicao}

Assim, $P\Rightarrow Q$ somente se a condicional $P\rightarrow Q$ for uma tautologia.

Exemplo: Sendo $P:p\wedge q$ e $Q:p\vee q$, verificar se $P\Rightarrow Q$.

Precisamos verificar se a condicional $P\rightarrow Q$ {\'e} uma tautologia (Tabela \ref{3}).
\begin{table}[h]
   \centering 
   \setlength{\arrayrulewidth}{0,5\arrayrulewidth}
   \caption{\it $P\rightarrow Q$}
   \begin{tabular}{|c|c|c|c|c|} 
      \hline
      $p$ & $q$ & $p\wedge q$ & $p\vee q$ & $(p\wedge q)\rightarrow(p\vee q)$ \\
     \hline
      V & V & V & V & V \\
      \hline
      V & F & F & V & V \\
      \hline
      F & V & F & V & V \\
      \hline
      F & F & F & F & V \\
      \hline
   \end{tabular}
\label{3}
\end{table}

Observa{\c c}{\~a}o:
\begin{enumerate}
\item Os s{\'\i}mbolos $\rightarrow$ e $\Rightarrow$ s{\~a}o diferentes. A condicional, $\rightarrow$, {\'e} um operador l{\'o}gico que aplicado a duas proposi{\c c}{\~o}es $p$ e $q$, por exemplo, produz uma nova proposi{\c c}{\~a}o $p\rightarrow q$. Por outro lado, a implica{\c c}{\~a}o, $\Rightarrow$, estabelece que $p\rightarrow q$ {\'e} uma tautologia.
\item Toda teorema {\'e} uma aplica{\c c}{\~a}o da forma
\begin{center}
Hip{\'o}tese $\Rightarrow$ Tese
\end{center}

Logo demonstrar um teorema significa mostrar que n{\~a}o ocorre o caso da hip{\'o}tese ser verdadeira e a tese falsa, isto {\'e}, a verdade da hip{\'o}tese {\'e} suficiente para garantir a verdade da tese.
\end{enumerate}

\subsection{Equival{\^e}ncia}

\begin{definicao}[Equival{\^e}ncia] Dizemos que uma proposi{\c c}{\~a}o $P(p,q,r,...)$ {\'e} equivalente a uma proposi{\c c}{\~a}o composta $Q(p,q,r,...)$, denotado por $P\Leftrightarrow Q$, se elas implicarem uma na outra. Assim, $P\Leftrightarrow Q$ se a bicondicional {\'e} uma tautologia.\end{definicao}

Exemplo: Sendo $P:p\leftrightarrow q$ e $Q:(p\rightarrow q)\wedge(q\rightarrow p)$ verificar que $P\Leftrightarrow Q$.\\

Precisamos verificar se a bicondicional $P\leftrightarrow Q$ {\'e} uma tautologia (Tabela \ref{4}).
\begin{table}[h]
   \centering 
   \setlength{\arrayrulewidth}{0,5\arrayrulewidth}
   \caption{\it $P\leftrightarrow Q$}
   \begin{tabular}{|c|c|c|c|c|c|c|} 
      \hline
      $p$ & $q$ & $p\rightarrow q$ & $q\leftrightarrow p$ & $(p\rightarrow q)\wedge(q\rightarrow p)$ & $p\leftrightarrow q$ & $P\leftrightarrow Q$ \\
     \hline
      V & V & V & V & V & V & V \\
      \hline
      V & F & F & V & F & F & V\\
      \hline
      F & V & V & F & F & F & V \\
      \hline
      F & F & V & V & V & V & V \\
      \hline
   \end{tabular}
\label{4}
\end{table}