%!TEX program = xelatex

\def\numerosemana{11}

\documentclass[12pt]{exam}

\def\ano{2022}
\def\semestre{1}
\def\disciplina{\'Algebra 1}
\def\turma{2}

\usepackage{caption}
\usepackage{amssymb}
\usepackage{amsmath,amsfonts,amsthm,amstext}
\usepackage[brazil]{babel}
% \usepackage[latin1]{inputenc}
\usepackage{graphicx}
\graphicspath{{/ArquivosLinux/OneDrive/imagens-latex/}{D:/OneDrive - unb.br/imagens-latex/}}
\usepackage{enumitem}
\usepackage{multicol}
\usepackage{answers}
\usepackage{tikz,ifthen}
\usetikzlibrary{lindenmayersystems}
\usetikzlibrary[shadings]
\Newassociation{solucao}{Solution}{ans}
\newtheorem{exercicio}{}

\setlength{\topmargin}{-1.0in}
\setlength{\oddsidemargin}{0in}
\setlength{\textheight}{10.1in}
\setlength{\textwidth}{6.5in}
\setlength{\baselineskip}{12mm}

\extraheadheight{0.7in}
\firstpageheadrule
\runningheadrule
\lhead{
        \begin{minipage}[c]{1.7cm}
        \includegraphics[width=1.7cm]{unb.pdf}
        \end{minipage}%
        \hspace{0pt}
        \begin{minipage}[c]{4in}
          {Universidade de Brasília} --
          {Departamento de Matemática}
\end{minipage}
\vspace*{-0.8cm}
}
% \chead{Universidade de Brasília - Departamento de Matemática}
% \rhead{}
% \vspace*{-2cm}

\extrafootheight{.5in}
\footrule
\lfoot{\disciplina\ - \semestre$^o$/\ano\ - Módulo \numeromodulo}
\cfoot{}
\rfoot{Página \thepage\ de \numpages}

\newcounter{exercicios}
\renewcommand{\theexercicios}{\arabic{exercicios}}

\newenvironment{questao}[1]{
\refstepcounter{exercicios}
\ifx&#1&
\else
   \label{#1}
\fi
\noindent\textbf{Exercício {\theexercicios}:}
}

\newcommand{\resp}[1]{
\noindent{\bf Exercício #1: }}

\def\ano{2024}
\def\semestre{1}
\def\disciplina{Álgebra 1}
\def\nomeabreviado{Álgebra 1}
\def\turma{1}

\newcommand{\im}{{\rm Im\,}}
\newcommand{\dlim}[2]{\displaystyle\lim_{#1\rightarrow #2}}
\newcommand{\minf}{+\infty}
\newcommand{\ninf}{-\infty}
\newcommand{\cp}[1]{\mathbb{#1}}
\newcommand{\sub}{\subseteq}
\newcommand{\n}{\mathbb{N}}
\newcommand{\z}{\mathbb{Z}}
\newcommand{\rac}{\mathbb{Q}}
\newcommand{\real}{\mathbb{R}}
\newcommand{\complex}{\mathbb{C}}

\newcommand{\vesp}[1]{\vspace{ #1  cm}}

\newcommand{\compcent}[1]{\vcenter{\hbox{$#1\circ$}}}
\newcommand{\comp}{\mathbin{\mathchoice
        {\compcent\scriptstyle}{\compcent\scriptstyle}
        {\compcent\scriptscriptstyle}{\compcent\scriptscriptstyle}}}
\renewcommand{\sin}{{\rm sen\,}}
\renewcommand{\tan}{{\rm tg\,}}
\renewcommand{\csc}{{\rm cossec\,}}
\renewcommand{\cot}{{\rm cotg\,}}
\renewcommand{\sinh}{{\rm senh\,}}

\begin{document}
    \begin{center}
    {\Large\bf \disciplina\ - Turma \turma\ -- \semestre$^{o}$/\ano} \\ \vspace{9pt} {\large\bf
        Lista de Exerc{\'\i}cios -- Semana \numerosemana}\\ \vspace{9pt} Prof. Jos{\'e} Ant{\^o}nio O. Freitas
    \end{center}
    \hrule

    \vspace{.6cm}

    \textbf{Observação: }\textit{Nos casos em que n\~ao forem especificadas as opera\c{c}\~oes do anel, considere as opera\c{c}\~oes usuais.}

    \vspace{.6cm}

    \questao{} Determinar quais dos seguintes subconjuntos de $\rac$ s{\~a}o suban{\'e}is:
        \begin{multicols}{2}
            \begin{enumerate}[label=({\alph*})]
                \item $\z$
                \item $B = \{x \in \rac \mid x \notin \z\}$
                \item $C = \left\{\dfrac{a}{b} \in \rac \mid a \in \z,\ b \in \z,\ 2 |b \right\}$
                \item $D = \left\{\dfrac{a}{2^n} \in \rac \mid a \in \z \mbox{ e } n \in \z \right\}$
            \end{enumerate}
        \end{multicols}

    \vspace{.3cm}

    \questao{} No anel $(\z \times \z, \oplus, \otimes)$ onde as opera\c{c}\~oes $\oplus$ e $\otimes$ s\~ao definidas por
    \begin{align*}
        (a, b) \oplus (c, d) = (a + c, b + d)\\
        (a ,b) \otimes (c, d) = (ac - bd, ad + bc).
    \end{align*}
    Quais dos seguintes conjuntos s\~ao suban\'eis?
    \begin{enumerate}[label=({\alph*})]
        \item $A = \{(x, y) \in \z \times \z \mid x = 0\}$
        \item $B = \{(x, y) \in \z \times \z \mid y = 0\}$
        \item $C = \{(x, y) \in \z \times \z \mid x = y\}$
        \item $D = \{(x, y) \in \z \times \z \mid x = 2k,\ k \in \z\}$
        \item $E = \{(x, y) \in \z \times \z \mid y = 3k,\ k \in \z\}$
        \item $F = \{(x, y) \in \z \times \z \mid x + y = 2k,\ k \in \z\}$
    \end{enumerate}

    \vspace{.3cm}

    \questao{} No anel $(\rac, \star, \odot)$ onde as opera\c{c}\~oes $\star$ e $\odot$ em $\rac$ definidas por
    \begin{align*}
        x \star y = x + y - 6\\
        x \odot y = x + y - \dfrac{xy}{6}.
    \end{align*}
    Quais dos seguintes subconjuntos s\~ao suban\'eis?
    \begin{enumerate}[label=({\alph*})]
        \item $A = \z$
        \item $B = \{2k \mid k \in \z\}$
        \item $C = \{6k \mid k \in \z\}$
        \item $D = \{3k \mid k \in \z\}$
    \end{enumerate}

    \vspace{.3cm}

    \questao{} Quais dos conjuntos abaixo s\~ao suban\'eis de $M_2(\real)$?
    \begin{align*}
        L_1 &= \left\{\begin{pmatrix}
            a & 0\\
            b & 0
        \end{pmatrix} \mid a, b \in \real\right\}\\
        L_2 &= \left\{\begin{pmatrix}
            a & b\\
            0 & c
        \end{pmatrix} \mid a, b, c \in \real\right\}\\
        L_3 &= \left\{\begin{pmatrix}
            a & 0\\
            0 & b
        \end{pmatrix} \mid a, b \in \real\right\}\\
        L_4 &= \left\{\begin{pmatrix}
            0 & a\\
            c & b
        \end{pmatrix} \mid a, b, c \in \real\right\}\\
        L_5 &= \left\{\begin{pmatrix}
            0 & a\\
            0 & b^2 + 1
        \end{pmatrix} \mid a, b \in \real\right\}
    \end{align*}

    \vspace{.3cm}

    \questao{} Quais dos conjuntos abaixo s\~ao suban\'eis de $M_2(\z_2)$?
    \begin{align*}
        L_1 &= \left\{\begin{pmatrix}
            \overline{a} & \overline{0}\\
            \overline{b} & \overline{0}
        \end{pmatrix} \mid a, b \in \z_2\right\}\\
        L_2 &= \left\{\begin{pmatrix}
            \overline{a} & \overline{b}\\
            \overline{0} & \overline{c}
        \end{pmatrix} \mid a, b, c \in \z_2\right\}\\
        L_3 &= \left\{\begin{pmatrix}
            \overline{a} & \overline{0}\\
            \overline{0} & \overline{b}
        \end{pmatrix} \mid a, b \in \z_2\right\}\\
        L_4 &= \left\{\begin{pmatrix}
            \overline{0} & \overline{a}\\
            \overline{c} & \overline{b}
        \end{pmatrix} \mid a, b, c \in \z_2\right\}\\
        L_5 &= \left\{\begin{pmatrix}
            \overline{0} & \overline{a}\\
            \overline{0} & \overline{b}^2 + \overline{1}
        \end{pmatrix} \mid a, b \in \z_2\right\}
    \end{align*}

    \vspace{.3cm}

    \questao{} Determine todos os suban\'eis do anel $(\z_8, \oplus, \otimes)$.

    \vspace{.3cm}

    \questao{} Determine todos os suban\'eis do anel $(\z_{16}, \oplus, \otimes)$.

    \vspace{.3cm}

    \questao{} Mostre que a interse\c{c}\~ao de dois suban\'eis de um anel $A$ \'e ainda um subanel de $A$.

    \vspace{.3cm}

    \questao{} \'E verdade que a uni\~ao de suban\'eis \'e um subanel?

    \vspace{.3cm}

    \questao{} Seja $(A, + , \cdot)$ um anel e $x \in A$ fixo. Mostre que o conjunto
    \[
        N(x) = \{y \in A \mid xy = yx\}
    \]
    \'e um subanel de $A$.

    \vspace{.3cm}

    \questao{} Verifique se $L = \{ a + b\sqrt{2} \mid a, b \in \rac\}$ {\'e} um subanel
    do anel $\mathbb{R}$.

    \vspace{.3cm}

    \questao{} Seja $d \in \z$ e considere o subconjunto de $M_2(\z)$ dado por
    \[
        M_2^d(\z) = \left\{\begin{pmatrix} a & db \\ b & a \end{pmatrix} \mid a, b \in \z\right\}.
    \]
    Mostre que $M_2^d(\z)$ é um subanel de $M_2(\z)$.

    \vspace{.3cm}

    \questao{} Seja $X$ um conjunto infinito. Sabemos que $(\mathcal{P}(X), \Delta, \cap)$ é um anel com unidade. Seja
    \[
        R = \{A \in \mathcal{P}(X) \mid A \mbox{ é finito}\}.
    \]
    Prove as seguintes afirmações:
    \begin{enumerate}[label=({\alph*})]
        \item $R$ é um subanel de $\mathcal{P}(X)$.

        \item $R$ não possui unidade.

        \item Para todo $A \in R$, $A \ne \emptyset$ existe $B \in R$, $B \ne \emptyset$, tal que $A \cap B = \emptyset$.

        \item Para todo $A \in \mathcal{P}(X)$, $A \ne X$, $A \ne \emptyset$ existe $B \in \mathcal{P}(X)$, $B \ne \emptyset$, tal que $A \cap B = \emptyset$.
    \end{enumerate}

    \vspace{.6cm}

    \questao{} Verifique se s\~ao ideiais:
    \begin{enumerate}[label=({\alph*})]
        \item  $I = \{\overline{0}, \overline{2}, \overline{4}\}$ no anel $\z_6$;

        \item $I = m\z \times n\z$ no anel $\z \times \z$, em que $m$, $n \in \z$;

        \item $I = \{x \in \z \mid 25 \mbox{ divide } 35x\}$ no anel $\z$;

        \item $I = \{x \in \z \mid x \mbox{ divide } 24\}$ no anel $\z$;

        \item $I = \{x \in \z \mid 6 \mbox{ divide } x \mbox{ e } 24 \mbox{ divide } x^2\}$ no anel $\z$;

        \item $I = \z$ no anel $(\rac, \oplus, \odot)$ em que a $a \oplus b = a + b - 1$ e $a \odot b = a + b - ab$, para todos $a$, $b \in \rac$;

        \item $I = 2\z$ no anel $(\z, +, \cdot)$ em que a adi\c{c}\~ao \'e a usual e $a \cdot b = 0$, para quaisquer $a$, $b \in \z$.
    \end{enumerate}

    \vspace{.3cm}

    \questao{} Seja $(A, +, \cdot)$ um anel comutativo.
    \begin{enumerate}[label=({\alph*})]
        \item Mostre que a interse\c{c}\~ao de quaisquer dois ideais de $A$ \'e sempre um conjunto n\~ao vazio.

        \item Mostre que essa interse\c{c}\~ao \'e sempre um ideal.

        \item A uni\~ao de ideias \'e ainda um ideal?

        \item Sejam $J_1$, $J_2 \subset A$ ideiais tais que $J_1 \subset J_2$. Mostre que $J_1 \cup J_2$ \'e um ideal de $A$.

        \item Sejam $I$ e $J$ ideais de $A$. Mostre que
        \[
            I + J = \{x + y \mid x \in I, y \in J\}
        \]
        \'e um ideal de $A$.

    \item Se $I$ é um ideal de um anel comutativo $A$, mostre que $r(I) = \{x \in A \mid xy = 0 \mbox{ para todo } y \in I\}$, então $r(I)$ é um ideal de $A$.
    \end{enumerate}

    \vspace{.3cm}

    \questao{} Sendo $A$ um anel, n\~ao necessariamente comutativo, dizemos que $I \subset A$, $I \ne \emptyset$ \'e um \textbf{ideal \`a esquerda} em $A$ se, e somente se:
    \begin{enumerate}[label=({\roman*})]
        \item Para todos $x$, $y \in I$ temos $x - y \in I$;

        \item Para todo $\alpha \in A$ e todo $x \in I$ temos $\alpha x \in I$.
    \end{enumerate}

    Verifique se s\~ao ideiais \`a esquerda em $M_2(\real)$:
    \begin{enumerate}[label=({\alph*})]
        \item $L_1 = \left\{\begin{pmatrix}
            a & 0\\
            0 & b
        \end{pmatrix} \mid a, b \in \real\right\}$.

        \item $L_2 = \left\{\begin{pmatrix}
            a & b\\
            0 & c
        \end{pmatrix} \mid a, b, c \in \real\right\}$.

        \item $L_3 = \left\{\begin{pmatrix}
            a & 0\\
            b & 0
        \end{pmatrix} \mid a, b \in \real\right\}$.

        \item $L_4 = \left\{\begin{pmatrix}
            a & b\\
            0 & 0
        \end{pmatrix} \mid a, b \in \real\right\}$.
    \end{enumerate}

\end{document}
