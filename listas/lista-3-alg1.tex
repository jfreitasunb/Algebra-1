%!TEX program = xelatex
%!TEX encoding = ISO-8859-1
\documentclass[12pt]{article}

\usepackage{amssymb}
\usepackage{amsmath,amsfonts,amsthm,amstext,mathabx}
\usepackage[brazil]{babel}
%\usepackage[latin1]{inputenc}
\usepackage{graphicx}
\graphicspath{{/home/jfreitas/Dropbox/imagens-latex/}{/Volumes/Vader/Dropbox/imagens-latex/}{D:/Dropbox/imagens-latex/}}
\usepackage{enumitem}
\usepackage{multicol}
\usepackage[all]{xy}

\setlength{\topmargin}{-1.0in}
\setlength{\oddsidemargin}{0in}
\setlength{\textheight}{10.1in}
\setlength{\textwidth}{6.5in}
\setlength{\baselineskip}{12mm}

\newcounter{exercicios}
\setcounter{exercicios}{0}
\newcommand{\questao}{
\addtocounter{exercicios}{1}
\noindent{\bf Exerc{\'\i}cio \arabic{exercicios}: }}

\newcommand{\equi}{\Leftrightarrow}
\newcommand{\bic}{\leftrightarrow}
\newcommand{\cond}{\rightarrow}
\newcommand{\impl}{\Rightarrow}
\newcommand{\nao}{\sim}
\newcommand{\sub}{\subseteq}
\newcommand{\e}{\ \wedge\ }
\newcommand{\ou}{\ \vee\ }
\newcommand{\vaz}{\emptyset}
\newcommand{\nsub}{\nsubset}
\renewcommand{\sin}{{\rm sen\,}}

\newcommand{\n}{\mathbb{N}}
\newcommand{\z}{\mathbb{Z}}
\newcommand{\real}{\mathbb{R}}
\newcommand{\vesp}{\vspace{0.2cm}}
\newcommand{\subne}{\subsetneqq}


\newcommand{\compcent}[1]{\vcenter{\hbox{$#1\circ$}}}
\newcommand{\comp}{\mathbin{\mathchoice
{\compcent\scriptstyle}{\compcent\scriptstyle}
{\compcent\scriptscriptstyle}{\compcent\scriptscriptstyle}}}

\begin{document}

\pagestyle{empty}

\begin{figure}[h]
        \begin{minipage}[c]{1.7cm}
        \includegraphics[width=1.7cm]{unb.pdf}
        \end{minipage}%
        \hspace{0pt}
        \begin{minipage}[c]{4in}
          {Universidade de Brasília} \\
          {Departamento de Matemática}
\end{minipage}
\end{figure}
\vspace{-1cm}\hrule


\begin{center}
{\Large\bf {\'A}lgebra 1 - Turma D -- 2$^{o}$/2017} \\ \vspace{9pt} {\large\bf
  $3^{\underline{a}}$ Lista de Exerc{\'\i}cios -- Rela\c{c}\~oes de Ordem}\\
\vspace{9pt} Prof. Jos{\'e} Ant{\^o}nio O. Freitas
\end{center}
\hrule

\vspace{.6cm}

\questao Sejam $X$ um conjunto e $\mathcal{P}(X)$ o conjunto de todas as partes de $X$. Se $A \in \mathcal{P}(X)$ e $B \in \mathcal{P}(X)$, definimos a relação $R$ como
\[
	ARB \mbox{ se, e só se, } A \sub B.
\]
Mostre que $R$ é uma relação de ordem em $\mathcal{P}(X)$.

\vesp

\questao Se $a$, $b \in \z$ definimos
\[
	aRb \mbox{ se , e somente se, } b - a \in \n.
\]
Mostre que $R$ é uma relação de ordem sobre $\z$. O conjunto $\z$ é totalmente ordenado por $R$?

\vesp

\questao Seja $\complex$ o conjunto dos números complexos e sejam $x = a + bi$ e $y = c + di$ dois elementos de $\complex$.
Mostre que o conjunto $R$ definido por
\[
	xRy \mbox{ se, e somente se, } a \leqslant c \mbox{ e } b \leqslant d
\]
é uma relação de ordem parcial em $\complex$.

\vesp

\questao Em $\n \times \n$ defina $(a,b) \leqslant (c, d)$ se, e somente se, $a \mid c$ e $b \leqslant d$.
\begin{enumerate}[label={\alph*})]
	\item Mostre que $\leqslant$ é uma relação de ordem parcial em $\n \times \n$.
	\item Sendo $A = \{(2,1);(1,2)\}$ ache os limites superiores, limites inferiores, máximo e mínimo de $A$.
\end{enumerate}

\vesp

\questao Seja $\n$ ordenado pela relação de ordem $R$ dada por $xRy$ se, e somente se, $x \mid y$.
\begin{enumerate}[label={\alph*})]
	\item Determine se os pares de números a seguir são ou não comparáveis:
	\begin{multicols}{2}
		\begin{enumerate}[label={\roman*})]
			\item 2 e 8
			\item 9 e 3
			\item 18 e 24
			\item 5 e 15
			\item 18 e 6
			\item 11 e 121
			\item 15 e 51
			\item 17 e 149
		\end{enumerate}
	\end{multicols}
	\item Determine se os seguintes conjuntos são ou não totalmente ordenados:
	\begin{multicols}{2}
		\begin{enumerate}[label={\roman*})]
			\item $A = \{2,8,24\}$
			\item $B = \{3,6,8\}$
			\item $C = \{3,5,15\}$
			\item $D = \{7,49,147\}$
			\item $E = \{13\}$
			\item $F = \{3,51\}$
		\end{enumerate}
	\end{multicols}
\end{enumerate}

\vesp

\questao Mostre que
\[
	R = \{(a + bi, c + di) \in \complex \times \complex \mid a < c \mbox{ ou } (a = c \mbox{ e } b \leqslant d)\}
\]
é uma relação de ordem total em $\complex$.

\vesp

\questao Considere a relação $R$ definida em $\n \times \n$ por
\[
	(x,y)R(z,t) \mbox{ se, e somente se, } a \mid c \mbox{ e } b \mid d.
\]
\begin{enumerate}[label={\alph*})]
	\item Mostre que $R$ é uma relação de ordem em $\n \times \n$.
	\item $\n \times \n$ é totalmente ordendo por $R$?
\end{enumerate}
\end{document}