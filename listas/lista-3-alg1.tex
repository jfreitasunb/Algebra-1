%!TEX program = xelatex
%!TEX encoding = ISO-8859-1
\documentclass[12pt]{article}

\usepackage{amssymb}
\usepackage{amsmath,amsfonts,amsthm,amstext,mathabx}
\usepackage[brazil]{babel}
%\usepackage[latin1]{inputenc}
\usepackage{graphicx}
\graphicspath{{/home/jfreitas/Dropbox/imagens-latex/}{/Volumes/Vader/Dropbox/imagens-latex/}{D:/Dropbox/imagens-latex/}}
\usepackage{enumitem}
\usepackage{multicol}
\usepackage[all]{xy}

\setlength{\topmargin}{-1.0in}
\setlength{\oddsidemargin}{0in}
\setlength{\textheight}{10.1in}
\setlength{\textwidth}{6.5in}
\setlength{\baselineskip}{12mm}

\newcounter{exercicios}
\setcounter{exercicios}{0}
\newcommand{\questao}{
\addtocounter{exercicios}{1}
\noindent{\bf Exerc{\'\i}cio \arabic{exercicios}: }}

\newcommand{\equi}{\Leftrightarrow}
\newcommand{\bic}{\leftrightarrow}
\newcommand{\cond}{\rightarrow}
\newcommand{\impl}{\Rightarrow}
\newcommand{\nao}{\sim}
\newcommand{\sub}{\subseteq}
\newcommand{\e}{\ \wedge\ }
\newcommand{\ou}{\ \vee\ }
\newcommand{\vaz}{\emptyset}
\newcommand{\nsub}{\nsubset}
\renewcommand{\sin}{{\rm sen\,}}

\newcommand{\n}{\mathbb{N}}
\newcommand{\z}{\mathbb{Z}}
\newcommand{\real}{\mathbb{R}}
\newcommand{\vesp}{\vspace{0.2cm}}
\newcommand{\subne}{\subsetneqq}


\newcommand{\compcent}[1]{\vcenter{\hbox{$#1\circ$}}}
\newcommand{\comp}{\mathbin{\mathchoice
{\compcent\scriptstyle}{\compcent\scriptstyle}
{\compcent\scriptscriptstyle}{\compcent\scriptscriptstyle}}}

\begin{document}

\pagestyle{empty}

\begin{figure}[h]
        \begin{minipage}[c]{1.7cm}
        \includegraphics[width=1.7cm]{unb.pdf}
        \end{minipage}%
        \hspace{0pt}
        \begin{minipage}[c]{4in}
          {Universidade de Brasília} \\
          {Departamento de Matemática}
\end{minipage}
\end{figure}
\vspace{-1cm}\hrule


\begin{center}
{\Large\bf {\'A}lgebra 1 - Turma D -- 2$^{o}$/2017} \\ \vspace{9pt} {\large\bf
  $3^{\underline{a}}$ Lista de Exerc{\'\i}cios -- Rela\c{c}\~oes de Ordem}\\
\vspace{9pt} Prof. Jos{\'e} Ant{\^o}nio O. Freitas
\end{center}
\hrule

\vspace{.6cm}

\questao Sejam $X$ um conjunto e $\mathcal{P}(X)$ o conjunto de todas as partes de $X$. Se $A \in \mathcal{P}(X)$ e $B \in \mathcal{P}(X)$, definimos a relação $R$ como
\[
	ARB \mbox{ se, e só se, } A \sub B.
\]
Mostre que $R$ é uma relação de ordem em $\mathcal{P}(X)$.

\vesp

\questao Se $a$, $b \in \z$ definimos
\[
	aRb \mbox{ se , e somente se, } b - a \in \n.
\]
Mostre que $R$ é uma relação de ordem sobre $\z$. O conjunto $\z$ é totalmente ordenado por $R$?

\vesp

\questao Seja $\complex$ o conjunto dos números complexos e sejam $x = a + bi$ e $y = c + di$ dois elementos de $\complex$.
Mostre que o conjunto $R$ definido por
\[
	xRy \mbox{ se, e somente se, } a \leqslant c \mbox{ e } b \leqslant d
\]
é uma relação de ordem parcial em $\complex$.

\vesp

\questao Em $\n \times n$ defina $(a,b) \leqslant (c, d)$ se, e somente se, $a \mid c$ e $b \leqslant d$.
\begin{enumerate}
	\item Mostre que $\leqslant$ é uma relação de ordem parcial em $\n \times \n$.
	\item Sendo $A = \{(2,1);(1,2)\}$ ache os limites superiores, limites inferiores, máximo e mínimo de $A$.
\end{enumerate}

\vesp

\questao Considere $\n$ com a divisibilidade. Determine se os seguintes subconjuntos são ou não totalmente ordenados:
\begin{enumerate}[label={\alph*})]
	\item $A = \{24,2,6\}$
	\item $B = \{3,15,5\}$
	\item $C = \{15,5,30\}$
\end{enumerate}

\vesp

\questao Seja $f:E\to F$ uma fun{\c c}{\~a}o e sejam $A\subset
E$ e $X, Y\subset F$. Mostre que:
\begin{enumerate}[label={\alph*})]
\item Se $X\subset Y$, ent{\~a}o $f^{-1}(X)\subset f^{-1}(Y)$.
\item $f^{-1}(X\cup Y)=f^{-1}(X)\cup f^{-1}(Y)$.
\item $f^{-1}(X\cap Y)= f^{-1}(X)\cap f^{-1}(Y)$.
\item $A\subset f^{-1}(f(A))$.
\item $f(f^{-1}(X))= X \cap \mbox{Im}f$ e conclua que se $f$ {\'e} sobrejetora ent{\~a}o
$f(f^{-1}(X))=X$.
\end{enumerate}

\vesp

\questao Se as fun{\c c}{\~o}es $f:E\to F$ e $g: F\to E$ s{\~a}o
tais que $g\circ f$ {\'e} injetora, ent{\~a}o $f$ {\'e} injetora.

\vesp

\questao Se as fun{\c c}{\~o}es $f:E\to F$ e $g: F\to E$ s{\~a}o
tais que $g\circ f$ {\'e} sobrejetora, ent{\~a}o g {\'e} sobrejetora.

\vesp

\questao Mostrar que toda fun{\c c}{\~a}o injetora (sobrejetora) de um conjunto finito em si mesmo {\'e} tamb{\'e}m sobrejetora (injetora).

\vesp

\questao Seja $f: \real^2 \to \real$ dada por $f(x,y) = xy$.
\begin{enumerate}[label={\alph*})]
\item $f$ {\'e} injetora?
\item $f$ {\'e} sobrejetora?
\item Obter $f^{-1}({0})$.
\item Obter $f([0,1]\times [0,1])$.
\end{enumerate}

\vesp

\questao Considere a fun{\c c}{\~a}o $f : \z \times \z \to \z \times \z$ tal que $f(x,y) = (2x + 3, 4y + 5)$. Prove que $f$ {\'e} injetora. Verifique se $f$ {\'e} bijetora.

\vesp

\questao Mostre que a fun{\c c}{\~a}o $f : \real \to \real$ definida por $f(x) = ax + b$, com $a$ e $b$ constantes reais, $a \ne 0$, {\'e} uma bije{\c c}{\~a}o. Obter $f^{-1}$.

\vesp

\questao Mostrar que $f : \real - \left\{-\dfrac{d}{c}\right\} \to \real  - \left\{\dfrac{a}{c}\right\}$ dada por $f(x) =  \dfrac{ax + b}{cx + d}$, onde $a$, $b$, $c$, $d$ s{\~a}o n{\'u}meros reais constantes, $ad - bc \ne 0$, {\'e} uma bije{\c c}{\~a}o. Descrever a fun{\c c}{\~a}o $f^{-1}$.

\vesp

\questao Achar uma fun{\c c}{\~a}o $f : A \to B$, com $A$ e $B$ subconjuntos de $\real$, para cada caso abaixo:
\begin{enumerate}[label={\alph*})]
\item $A = \real$, $B \subne \real$ e $f$ injetora e n{\~a}o sobrejetora.
\item $A \subne \real$, $B = \real$ e $f$ injetora e n{\~a}o sobrejetora.
\item $A = \real$, $B \subne \real$ e $f$ sobrejetora e n{\~a}o injetora.
\item $A \subne \real$, $B = \real$ e $f$ sobrejetora e n{\~a}o injetora.
\end{enumerate}

\vesp

\questao Classificar (se poss{\'\i}vel) em injetora ou sobrejetora as seguintes fun{\c c}{\~o}es de $\real$ em $\real$.

\begin{enumerate}[label={\alph*})]
\begin{multicols}{2}
\item $y = x^3$
\item $y = x^2 - 5x - 6$
\item $y = 2^x$
\item $y = | \sin x |$
\item $y = x + | x |$
\item $y = x + 3$
\end{multicols}
\end{enumerate}

\vesp

\questao Seja a fun{\c c}{\~a}o $f : \real \to \real$ dada por $f(x) = | x |$. Determinar $f([-1,1])$, $f(]-1,2])$, $f(\real)$, $f^{-1}([0,3])$, $f^{-1}([-1,3])$ e $f^{-1}(\real_-^*)$.

\end{document}