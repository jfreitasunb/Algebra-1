%!TEX program = xelatex
%!TEX encoding = ISO-8859-1
\def\ano{2021}
\def\semestre{1}
\def\disciplina{\'Algebra 1}
\def\turma{C}
\def\numerosemana{07}

\documentclass[12pt]{exam}

\usepackage{caption}
\usepackage{amssymb}
\usepackage{amsmath,amsfonts,amsthm,amstext}
\usepackage[brazil]{babel}
% \usepackage[latin1]{inputenc}
\usepackage{graphicx}
\graphicspath{{/ArquivosLinux/OneDrive/imagens-latex/}{D:/OneDrive - unb.br/imagens-latex/}}
\usepackage{enumitem}
\usepackage{multicol}
\usepackage{answers}
\usepackage{tikz,ifthen}
\usetikzlibrary{lindenmayersystems}
\usetikzlibrary[shadings]
\Newassociation{solucao}{Solution}{ans}
\newtheorem{exercicio}{}

\setlength{\topmargin}{-1.0in}
\setlength{\oddsidemargin}{0in}
\setlength{\textheight}{10.1in}
\setlength{\textwidth}{6.5in}
\setlength{\baselineskip}{12mm}

\extraheadheight{0.7in}
\firstpageheadrule
\runningheadrule
\lhead{
        \begin{minipage}[c]{1.7cm}
        \includegraphics[width=1.7cm]{unb.pdf}
        \end{minipage}%
        \hspace{0pt}
        \begin{minipage}[c]{4in}
          {Universidade de Brasília} --
          {Departamento de Matemática}
\end{minipage}
\vspace*{-0.8cm}
}
% \chead{Universidade de Brasília - Departamento de Matemática}
% \rhead{}
% \vspace*{-2cm}

\extrafootheight{.5in}
\footrule
\lfoot{\disciplina\ - \semestre$^o$/\ano\ - Módulo \numeromodulo}
\cfoot{}
\rfoot{Página \thepage\ de \numpages}

\newcounter{exercicios}
\renewcommand{\theexercicios}{\arabic{exercicios}}

\newenvironment{questao}[1]{
\refstepcounter{exercicios}
\ifx&#1&
\else
   \label{#1}
\fi
\noindent\textbf{Exercício {\theexercicios}:}
}

\newcommand{\resp}[1]{
\noindent{\bf Exercício #1: }}

\def\ano{2024}
\def\semestre{1}
\def\disciplina{Álgebra 1}
\def\nomeabreviado{Álgebra 1}
\def\turma{1}

\newcommand{\im}{{\rm Im\,}}
\newcommand{\dlim}[2]{\displaystyle\lim_{#1\rightarrow #2}}
\newcommand{\minf}{+\infty}
\newcommand{\ninf}{-\infty}
\newcommand{\cp}[1]{\mathbb{#1}}
\newcommand{\sub}{\subseteq}
\newcommand{\n}{\mathbb{N}}
\newcommand{\z}{\mathbb{Z}}
\newcommand{\rac}{\mathbb{Q}}
\newcommand{\real}{\mathbb{R}}
\newcommand{\complex}{\mathbb{C}}

\newcommand{\vesp}[1]{\vspace{ #1  cm}}

\newcommand{\compcent}[1]{\vcenter{\hbox{$#1\circ$}}}
\newcommand{\comp}{\mathbin{\mathchoice
        {\compcent\scriptstyle}{\compcent\scriptstyle}
        {\compcent\scriptscriptstyle}{\compcent\scriptscriptstyle}}}
\renewcommand{\sin}{{\rm sen\,}}
\renewcommand{\tan}{{\rm tg\,}}
\renewcommand{\csc}{{\rm cossec\,}}
\renewcommand{\cot}{{\rm cotg\,}}
\renewcommand{\sinh}{{\rm senh\,}}

\begin{document}
    \begin{center}
        {\Large\bf \disciplina\ - Turma \turma\ -- \semestre$^{o}$/\ano} \\ \vspace{9pt} {\large\bf
            Lista de Exerc{\'\i}cios -- Semana \numerosemana}\\ \vspace{9pt} Prof. Jos{\'e} Ant{\^o}nio O. Freitas
    \end{center}
    \hrule

    \vspace{.6cm}

    \begin{center}
        \textit{Nota\c{c}\~oes:}
    \end{center}

    \begin{multicols}{2}
        \begin{enumerate}[label={\roman*})]
            \item $\n^* = \n - \{0\}$

            \item $\real^*_+ = \{x \in \real \mid x > 0\}$

            \item $\real^*_- = \{x \in \real \mid x < 0\}$

            \item $\real_+ = \{x \in \real \mid x \ge 0\}$

            \item $\real_- = \{x \in \real \mid x \le 0\}$

            \item $]c,d[\ = (c,d) = \{x \in \real \mid c < x < d\}$

            \item $[c,d] = \{x \in \real \mid c \le x \le d\}$

            \item $[c,d[\ = [c, d) = \{x \in \real \mid c \le x < d\}$

            \item $]c,d] = (c, d] = \{x \in \real \mid c < x \le d\}$

            \item Em $\z_m$ vamos denotar a opera\c{c}\~ao de soma $\oplus$ por $+$.

            \item Em $\z_m$ vamos denotar a opera\c{c}\~ao de multiplica\c{c}\~ao $\otimes$ por $\cdot$.
        \end{enumerate}
    \end{multicols}

    \vspace{.6cm}

    \questao{} Considere a fun\c{c}\~ao $f : \real \to \real$ dada por $f(x) = \left| x - \dfrac{5}{2}\right|$. Encontre:
    \begin{multicols}{3}
        \begin{enumerate}[label={\alph*})]
            \item $f(\{1\})$

            \item $f(\{-\sqrt{2}, 3\})$

            \item $f([-2,2])$

            \item $f((-3,5))$

            \item $f^{-1}(\{3\})$

            \item $f^{-1}(\{-3,5\})$

            \item $f^{-1}([0,2])$

            \item $f^{-1}([-3,3])$

            \item $f^{-1}(\real^*_-)$
        \end{enumerate}
    \end{multicols}

    \vspace{.3cm}

    \questao{} Seja $g : \real \to \real$ dada por
    \[
        g(x) = \begin{cases}
            x^3 - 2,& \mbox{ se } x \le 0\\
            \sqrt{x + 1}, & \mbox{ se } 0 < x \le 2\\
            \dfrac{1}{x - 2}, & \mbox{ se } x > 2.
        \end{cases}
    \]
    Encontre:
    \begin{multicols}{2}
        \begin{enumerate}[label={\alph*})]
            \item $g([-1,8])$

            \item $g(\real_+)$

            \item $g(]1, 3])$

            \item $g^{-1}([-1,16])$

            \item $g(\real_-)$

            \item $g^{-1}([1,25])$

            \item $g^{-1}(\real^*_-)$

            \item $g^{-1}([-1,5])$
        \end{enumerate}
    \end{multicols}

    \vspace{.3cm}

    \questao{} Seja $f: \real^2 \to \real$ dada por $f(x,y) = xy - 2$.
    \begin{enumerate}[label={\alph*})]
        \item Obter $f^{-1}({0})$.

        \item Obter $f([0,1]\times [0,1])$.

        \item Obter $f^{-1}([0,2])$.
    \end{enumerate}

    \vspace{.3cm}


    \questao{} Considere a fun{\c c}{\~a}o $f : \z_5 \times \z_9 \to \z_5 \times \z_9$ tal que $f(\overline{x},\overline{y}) = (\overline{2} \overline{x} + \overline{3}, \overline{4}\overline{y} + \overline{5})$. Verifique se $f$ possui inversa e em caso afirmativo, encontre sua inversa.

    \vspace{.3cm}

    \questao{} Para quais valores de $a \in \z$ a função $f : \z \to \z$ dada por $f(x)       = x + a$ possui inversa? Para cada valor de $a$, encontre a função inversa.

    \vspace{.3cm}

    \questao{} Para quais valores de $b \in \z$ a função $g : \z \to \z$ dada por $g(x)       = bx$ possui inversa? Para cada valor de $b$, encontre a função inversa.

    \vspace{.3cm}

    \questao{} Suponha que as funções $f : A \to B$ e $g : B \to C$ são inversíveis. Mos      tre que $g \circ f$ é inversível e que
    \[
        (g \circ f)^{-1} = f^{-1} \circ g^{-1}.
    \]

    \vspace{.3cm}

    \questao{} Considere a fun{\c c}{\~a}o $g : \z_6 \times \z_8 \to \z_6 \times \z_8$ tal que $g(\overline{x},\overline{y}) = (\overline{2} \overline{x} + \overline{3}, \overline{4}\overline{y} + \overline{5})$. Verifique se $g$ possui inversa e em caso afirmativo, encontre sua inversa.

    \vspace{.3cm}

    \questao{} Considere a fun{\c c}{\~a}o $h : \z_5 \times \z_7 \to \z_5 \times \z_7$ tal que $h(\overline{x},\overline{y}) = (\overline{x} + \overline{3}, \overline{6}\overline{y} + \overline{2})$. Determine a fun\c{c}\~ao inversa de $h$, se existir.

    \vspace{.3cm}

    \questao{} Considere a fun{\c c}{\~a}o $t : \z_8 \times \z_{10} \to \z_8 \times \z_{10}$ tal que $t(\overline{x},\overline{y}) = (\overline{x} + \overline{3}, \overline{6}\overline{y} + \overline{2})$. Determine a fun\c{c}\~ao inversa de $t$, se existir.

    \vspace{.3cm}

    \questao{} A fun{\c c}{\~a}o $f : \z \to \z$ definida por $f(x) = ax + b$, com $a$ e $b$ constantes, $a \ne 0$, {\'e} uma bije{\c c}{\~a}o? Caso afirmativo, obtenha $f^{-1}$.

    \vspace{.3cm}


    \questao{} Mostre que a fun{\c c}{\~a}o $f : \real \to \real$ definida por $f(x) = ax + b$, com $a$ e $b$ constantes reais, $a \ne 0$, {\'e} uma bije{\c c}{\~a}o. Obter $f^{-1}$.

    \vspace{.3cm}

    \questao{} Mostrar que $f : \real - \left\{-\dfrac{d}{c}\right\} \to \real  - \left\{\dfrac{a}{c}\right\}$ dada por $f(x) =  \dfrac{ax + b}{cx + d}$, onde $a$, $b$, $c$, $d$ s{\~a}o n{\'u}meros reais constantes, $ad - bc \ne 0$, {\'e} uma bije{\c c}{\~a}o. Descrever a fun{\c c}{\~a}o $f^{-1}$.

    \vspace{.3cm}

    \questao{} Seja $f : A \to B$ e $g : B \to A$ fun\c{c}\~oes tais que $g \circ f = i_A$. Quais das afirma\c{c}\~oes seguintes s\~ao verdadeiras?
    \begin{enumerate}[label={\alph*})]
        \item $g = f^{-1}$

        \item $f$ \'e sobrejetora

        \item $f$ \'e injetora

        \item $g$ \'e sobrejetora

        \item $g$ \'e injetora
    \end{enumerate}

    \vspace{.3cm}

    \questao{} Seja $f : A \to B$ uma fun\c{c}\~ao. Se $Y \sub B$, ent\~ao $f(f^{-1}(Y)) = Y$? Prove ou d\^e um contra-exemplo.

    \vspace{.3cm}

    \questao{} Seja $f : A \to B$ uma fun\c{c}\~ao e $X \sub A$. \'E verdade que
    \[
        f(f^{-1}(f(X))) = f(X)?
    \]
    Prove ou d\^e um contra-exemplo.

    \vspace{.3cm}

    \questao{} Seja $f : A \to B$ uma fun\c{c}\~ao. Dados conjuntos $X \sub A$ e $Y \sub B$ \'e verdade que
    \[
        f^{-1}(f(f^{-1}(f(X)))) = f^{-1}(Y)?
    \]
    Prove ou d\^e um contra-exemplo.

    \vspace{.3cm}

    \questao{} Seja $f : A \to B$ uma fun\c{c}\~ao. Prove que $f$ \'e injetiva se, e somente se, $X = f^{-1}(f(X))$ para todo $X \sub A$.

    \vspace{.3cm}

    \questao{} Dada uma fun\c{c}\~ao $f : A \to B$ e conjuntos $W$, $X \sub A$, ent\~ao a igualdade
    \[
        f(W \cap X) = f(W) \cap f(X)
    \]
    \textbf{em geral \'e falsa}. D\^e um exemplo em que essa igualdade seja falsa.

    \vspace{.3cm}

    \questao{} Seja $f : A \to B$ uma fun\c{c}\~ao e sejam $P$ e $Q$ subconjuntos de $A$. Mostre que:
    \begin{enumerate}[label={\alph*})]
        \item Se $P\sub Q$, ent{\~a}o $f(P)\sub f(Q)$.
        \item $f(P\cup Q) = f(P)\cup f(Q)$.

        \item $f(P\cap Q)\sub f(P)\cap f(Q)$.

        \item $f(P) - f(Q) \sub f(P - Q)$.

        \item Se $f$ {\'e} injetora, ent{\~a}o $f(P\cap Q) =  f(P)\cap f(Q)$.

        \item $f$ {\'e} bijetora se, e somente se, $f(P^C) = (f(P))^C$ para todo $P \sub A$. \textit{(Aqui $P^C$ \'e o complementar de $P$ em rela\c{c}\~ao \`a $A$.)}
    \end{enumerate}

    \vspace{.3cm}

    \questao{} Suponha que $f : A \to B$ e $g : B \to C$ são funções. Seja $P \sub A$, então
    \[
        (g \circ f)(A) = g(f(A)).
    \]

    \vspace{.3cm}

    \questao{} Sejam $P$, $Q$ conjuntos, $f : P \to Q$ uma função, $A \sub P$ e $B \sub Q$. Então $f(A) \cap B = f(A \cap f^{-1}(B))$.

    \vspace{.3cm}

    \questao{} Suponha que $f : A \to B$ e $g : B \to C$ são funções. Seja $P \sub C$. Então
    \[
        (g \circ f)^{-1}(C) = f^{-1}(g^{-1}(C)).
    \]

    \newpage

    \questao{} Seja $f : A \to B$ uma fun{\c c}{\~a}o e sejam $P \sub
    A$ e $X, Y\sub B$. Mostre que:
    \begin{enumerate}[label={\alph*})]
        \item Se $X\sub Y$, ent{\~a}o $f^{-1}(X)\sub f^{-1}(Y)$.

        \item $f^{-1}(X\cup Y)=f^{-1}(X)\cup f^{-1}(Y)$.

        \item $f^{-1}(X - Y) = f^{-1}(X) - f^{-1}(Y)$.

        \item $f^{-1}(X\cap Y)= f^{-1}(X)\cap f^{-1}(Y)$.

        \item $P\sub f^{-1}(f(P))$.

        \item $f(f^{-1}(X))= X \cap \mbox{Im}f$ e conclua que se $f$ {\'e} sobrejetora ent{\~a}o
        $f(f^{-1}(X))=X$.
        \item $f^{-1}(X^C) = (f^{-1}(X))^C$.

        \item $f$ \'e sobrejetora se, e somente se, $f^{-1}(T) \ne \emptyset$ para todo $T \sub B$.
    \end{enumerate}

\end{document}
