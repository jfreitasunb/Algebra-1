%!TEX program = xelatex
% !TEX encoding = ISO-8859-1
\def\ano{2020}
\def\semestre{1}
\def\disciplina{\'Algebra 1}
\def\turma{C}
\def\numerosemana{11}

\documentclass[12pt]{exam}

\usepackage{caption}
\usepackage{amssymb}
\usepackage{amsmath,amsfonts,amsthm,amstext}
\usepackage[brazil]{babel}
% \usepackage[latin1]{inputenc}
\usepackage{graphicx}
\graphicspath{{/ArquivosLinux/OneDrive/imagens-latex/}{D:/OneDrive - unb.br/imagens-latex/}}
\usepackage{enumitem}
\usepackage{multicol}
\usepackage{answers}
\usepackage{tikz,ifthen}
\usetikzlibrary{lindenmayersystems}
\usetikzlibrary[shadings]
\Newassociation{solucao}{Solution}{ans}
\newtheorem{exercicio}{}

\setlength{\topmargin}{-1.0in}
\setlength{\oddsidemargin}{0in}
\setlength{\textheight}{10.1in}
\setlength{\textwidth}{6.5in}
\setlength{\baselineskip}{12mm}

\extraheadheight{0.7in}
\firstpageheadrule
\runningheadrule
\lhead{
        \begin{minipage}[c]{1.7cm}
        \includegraphics[width=1.7cm]{unb.pdf}
        \end{minipage}%
        \hspace{0pt}
        \begin{minipage}[c]{4in}
          {Universidade de Brasília} --
          {Departamento de Matemática}
\end{minipage}
\vspace*{-0.8cm}
}
% \chead{Universidade de Brasília - Departamento de Matemática}
% \rhead{}
% \vspace*{-2cm}

\extrafootheight{.5in}
\footrule
\lfoot{\disciplina\ - \semestre$^o$/\ano\ - Módulo \numeromodulo}
\cfoot{}
\rfoot{Página \thepage\ de \numpages}

\newcounter{exercicios}
\renewcommand{\theexercicios}{\arabic{exercicios}}

\newenvironment{questao}[1]{
\refstepcounter{exercicios}
\ifx&#1&
\else
   \label{#1}
\fi
\noindent\textbf{Exercício {\theexercicios}:}
}

\newcommand{\resp}[1]{
\noindent{\bf Exercício #1: }}

\def\ano{2024}
\def\semestre{1}
\def\disciplina{Álgebra 1}
\def\nomeabreviado{Álgebra 1}
\def\turma{1}

\newcommand{\im}{{\rm Im\,}}
\newcommand{\dlim}[2]{\displaystyle\lim_{#1\rightarrow #2}}
\newcommand{\minf}{+\infty}
\newcommand{\ninf}{-\infty}
\newcommand{\cp}[1]{\mathbb{#1}}
\newcommand{\sub}{\subseteq}
\newcommand{\n}{\mathbb{N}}
\newcommand{\z}{\mathbb{Z}}
\newcommand{\rac}{\mathbb{Q}}
\newcommand{\real}{\mathbb{R}}
\newcommand{\complex}{\mathbb{C}}

\newcommand{\vesp}[1]{\vspace{ #1  cm}}

\newcommand{\compcent}[1]{\vcenter{\hbox{$#1\circ$}}}
\newcommand{\comp}{\mathbin{\mathchoice
        {\compcent\scriptstyle}{\compcent\scriptstyle}
        {\compcent\scriptscriptstyle}{\compcent\scriptscriptstyle}}}
\renewcommand{\sin}{{\rm sen\,}}
\renewcommand{\tan}{{\rm tg\,}}
\renewcommand{\csc}{{\rm cossec\,}}
\renewcommand{\cot}{{\rm cotg\,}}
\renewcommand{\sinh}{{\rm senh\,}}

\begin{document}

\begin{center}
    
    {\Large\bf \disciplina\ - Turma \turma\ -- \semestre$^{o}$/\ano} \\ \vspace{9pt} {\large\bf
        Lista de Exerc{\'\i}cios -- Semana \numerosemana}\\ \vspace{9pt} Prof. Jos{\'e} Ant{\^o}nio O. Freitas
    \end{center}
    \hrule

    \vspace{.6cm}
    \questao{} Verifique se os seguintes conjuntos com a opera\c{c}\~ao dada \'e ou n\~ao um grupo. Em caso afirmativo, o grupo \'e comutativo?
    \begin{enumerate}[label=({\alph*})]
        \item $(\z, \star)$, onde $x \star y = x + xy$, para $x$, $y \in \z$;
        \item $(\z, \star)$, onde $x \star y = x + y + xy$, para $x$, $y \in \z$;
        \item $(\z, \star)$, onde $x \star y = xy + 2x$, para $x$, $y \in \z$;
        \item $(\rac, \star)$, onde $x \star y = x + xy$, para $x$, $y \in \rac$;
        \item $(\real^*, \star)$, onde $x \star y = \dfrac{x}{y}$, para $x$, $y \in \real$;
        \item $(\real_+, \star)$, onde $x \star y = \sqrt{x^2 + y^2}$, para $x$, $y \in \real_+$;
        \item $(\real, \star)$, onde $x \star y = \sqrt[3]{x^3 + y^3}$, para $x$, $y \in \real$.
        \item $(G, \cdot)$, onde $G = \{x \in \rac \mid x > 0\}$ e $\cdot$ \'e a multiplica\c{c}\~ao de n\'umeros racionais.
        \item $(G, \cdot)$, onde $G = \left\{\dfrac{1 + 2m}{1 + 2n} \mid m, n \in \z\right\}$ e $\cdot$ \'e a multiplica\c{c}\~ao de n\'umeros racionais.
        \item $(G, +)$, onde $G = \{0, \pm 2, \pm 4, \pm 6, \dots\}$ e $+$ \'e a soma de n\'umeros inteiros.
        \item $(G, \star)$, onde $G = \{0, \pm 2, \pm 4, \pm 6, \dots\}$ e $\star$ \'e definida como $x \star y = x + y - xy$.
        \item $(G, +)$, onde $G = \{a + b\sqrt{2} \mid a, b \in \rac\}$ e $+$ \'e a soma de n\'umeros reais.
        \item $(G, \cdot)$, onde $G = \{a + b\sqrt{2} \in \real^* \mid a, b \in \rac\}$ e $\cdot$ \'e a multipli\c{c}\~ao de n\'umeros reais.
        \item $(G, +)$, onde $G = \{a + b\sqrt[3]{2} \mid a, b \in \rac\}$ e $+$ \'e a soma de n\'umeros reais.
        \item $(G, \cdot)$, onde $G = \{a + b\sqrt[3]{2} \in \real^* \mid a, b \in \rac\}$ e $\cdot$ \'e a multipli\c{c}\~ao de n\'umeros reais..
    \end{enumerate}

    \vspace{.3cm}

    \questao{} Seja
    \[
        \complex = \{a + bi \mid a, b \in \real\}
    \]
    e $i^2 = -1$. Mostre que:
    \begin{enumerate}[label=({\alph*})]
        \item $(\complex, +)$ \'e um grupo abeliano, onde
        \[
            (a + bi) + (c + di) = (a + c) + (b + d)i 
        \]
        para $a + bi$, $c + di \in \complex$.
        \item Para $\complex^* = \complex - \{0\}$, $(\complex^*, \cdot)$ \'e um grupo abeliano, onde
        \[
            (a + bi)\cdot (c + di) = (ac - bd) + (ad + bc)i 
        \]
        para $a + bi$, $c + di \in \complex$.
    \end{enumerate}

    \vspace{.3cm}

    \questao{} Verifique se o conjunto $\rac_{>0}$ dos n{\'u}meros racionais estritamente positivos com a
     opera{\c c}{\~a}o dada {\'e} ou n{\~a}o um grupo. Justifique sua
    resposta.
    \begin{multicols}{2}
    \begin{enumerate}[label=({\alph*})]
    \item $(\rac_{>0},\cdot)$
    \item $(\rac_{>0}, +)$
    \end{enumerate}
    \end{multicols}

    \vspace{.3cm}

    \questao{} Seja $z  = a + bi \in \mathbb{C}$, onde $a$, $b \in \real$. Definimos $|z| = \sqrt{a^2 + b^2}$. Prove que $G=\{z \in \mathbb{C} \mid |z| = 1\}$ {\'e} um grupo
    abeliano com a opera{\c c}{\~a}o de multiplica{\c c}{\~a}o de n{\'u}meros complexos.

    \vspace{.3cm}

    \questao{} Mostre que o conjunto $\rac[\sqrt{2}]^*=\{ a + b\sqrt{2} \in
    \mathbb{R}^* \mid  a, b \in \rac \}$ {\'e} um grupo multiplicativo abeliano.

    \vspace{.3cm}

    \questao{} No conjunto $\z \times \z$ considere a opera\c{c}\~ao de soma definida por
    \[
        (x, y) + (z, t) = (x + z, y + t)
    \]
    para $(x, y)$, $(z, t) \in \z \times \z$. Mostre que $(\z\times\z, +)$ \'e um grupo abeliano.

    \vspace{.3cm}

    \questao{} Quais dos seguintes subconjuntos $G$ de $\z_{13}$ s{\~a}o grupos
    com a opera{\c c}{\~a}o de multiplica{\c c}{\~a}o?
    \begin{multicols}{2}
    \begin{enumerate}[label=({\alph*})]
    \item $G=\{\overline{1},\overline{12}\}$;

    \item $G=\{\overline{1},\overline{5},\overline{8},\overline{12}\}$;

    \item $G=\{\overline{1},\overline{2},\overline{3},\overline{4}, \overline{5},\overline{6},\overline{7},
     \overline{8},\overline{9},\overline{10},\overline{11},\overline{12}\}$
    \item $G=\{\overline{1}, \overline{3},\overline{5},\overline{7},\overline{9},\overline{11}\}$.
    \end{enumerate}
    \end{multicols}

    \vspace{.3cm}

    \questao{} Determine $f$, $g \in S_3$ tais que:
    \begin{enumerate}[label=({\alph*})]
        \item $(f \comp g)^3 \ne f^3\comp g^3$
        \item $(f \comp g)^2 \ne f^2\comp g^2$
    \end{enumerate}

    \vspace{.3cm}

    \questao{} Determine $f$, $g \in S_4$ tais que:
    \begin{enumerate}[label=({\alph*})]
        \item $(f \comp g)^4 \ne f^4\comp g^4$
        \item $(f \comp g)^3 \ne f^3\comp g^3$
    \end{enumerate}

    \vspace{.3cm}

    \questao{} Considere o grupo $S_3$:
    \begin{enumerate}[label=({\alph*})]
        \item Determine todos os elementos $f \in S_3$ tais que $f^2 = Id$ e $f \ne Id$.
        \item Determine todos os elementos $g \in S_3$ tais que $g^3 = Id$ e $g \ne Id$.
    \end{enumerate}

    \vspace{.3cm}

    \questao{} Considere o grupo $S_4$:
    \begin{enumerate}[label=({\alph*})]
        \item Determine todos os elementos $f \in S_4$ tais que $f^2 = Id$ e $f \ne Id$.
        \item Determine todos os elementos $g \in S_4$ tais que $g^3 = Id$ e $g \ne Id$.
        \item Determine todos os elementos $g \in S_4$ tais que $g^4 = Id$ e $g \ne Id$.
    \end{enumerate}

    \vspace{.3cm}


    \questao{} Seja $V = \{1, f, g, h\}$ o seguinte subconjunto do grupo $S_4$:
    \begin{align*}
        1 = \begin{pmatrix}
            1 & 2 & 3 & 4\\
            1 & 2 & 3 & 4
        \end{pmatrix}; \quad f = \begin{pmatrix}
            1 & 2 & 3 & 4\\
            2 & 1 & 4 & 3
        \end{pmatrix}\\
        g = \begin{pmatrix}
            1 & 2 & 3 & 4\\
            3 & 4 & 1 & 2
        \end{pmatrix}; \quad h = \begin{pmatrix}
            1 & 2 & 3 & 4\\
            4 & 3 & 2 & 1
        \end{pmatrix}.
    \end{align*}
    \begin{enumerate}[label=({\alph*})]
        \item Prove que $(V, \comp)$ \'e um grupo contendo 4 elementos, onde $\comp$ \'e a opera\c{c}\~ao de $S_4$.
        \item Prove que $(V, \comp)$ \'e um grupo abeliano.
    \end{enumerate}

    \vspace{.3cm}

    \questao{} Considere o grupo $S_7$ e sejam
    \begin{align*}
        1 &= \begin{pmatrix}
            1 & 2 & 3 & 4 & 5 & 6 & 7\\
            1 & 2 & 3 & 4 & 5 & 6 & 7
        \end{pmatrix}\\
        \sigma &= \begin{pmatrix}
                1 & 2 & 3 & 4 & 5 & 6 & 7\\
                3 & 4 & 2 & 6 & 7 & 5 & 1
            \end{pmatrix}.\\
        \beta &= \begin{pmatrix}
                1 & 2 & 3 & 4 & 5 & 6 & 7\\
                3 & 1 & 2 & 6 & 7 & 4 & 5
            \end{pmatrix}.
    \end{align*}.
    \begin{enumerate}[label=({\alph*})]
        \item Encontre o menor $l \ge 0$ tal que $\sigma^l = 1$.
        \item Encontre $\delta \in S_7$ tal que $\sigma\comp\delta = 1$.
        \item Encontre o menor $k \ge 0$ tal que $\beta^k = 1$.
        \item Encontre $\gamma \in S_7$ tal que $\gamma\comp\beta = 1$.
    \end{enumerate}

    \vspace{.3cm}

    \questao{} Seja $(G,*)$ um grupo com elemento neutro $e$. Para $x\in
    G$, considere a nota{\c c}{\~a}o $x^n=x*x*\cdots *x$ ($n$ vezes).
    \begin{enumerate}[label=({\alph*})]
    \item Seja $G$ um grupo tendo $e$ como elemento neutro. Prove que se
    $x^2=e$, para todo $x\in G$, ent{\~a}o $G$ {\'e} um grupo abeliano.
    \item Mostre que se $x\in G$ {\'e} tal que $x^2=x$, ent{\~a}o $x$ {\'e} o elemento neutro.
    \end{enumerate}

    \vspace{.3cm}

    \questao{} Sejam $G$ um grupo e $x$, $y$, $z \in G$. Prove que:
    \begin{enumerate}[label=({\alph*})]
        \item Se $xy = xz$, ent\~ao $y = z$.
        \item Se $yx = zx$, ent\~ao $y = z$.
    \end{enumerate}

    \vspace{.3cm}
\end{document}