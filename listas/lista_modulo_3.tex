%!TEX program = xelatex

\def\numeromodulo{3}

\documentclass[12pt]{exam}

\def\ano{2022}
\def\semestre{1}
\def\disciplina{\'Algebra 1}
\def\turma{2}

\usepackage{caption}
\usepackage{amssymb}
\usepackage{amsmath,amsfonts,amsthm,amstext}
\usepackage[brazil]{babel}
% \usepackage[latin1]{inputenc}
\usepackage{graphicx}
\graphicspath{{/ArquivosLinux/OneDrive/imagens-latex/}{D:/OneDrive - unb.br/imagens-latex/}}
\usepackage{enumitem}
\usepackage{multicol}
\usepackage{answers}
\usepackage{tikz,ifthen}
\usetikzlibrary{lindenmayersystems}
\usetikzlibrary[shadings]
\Newassociation{solucao}{Solution}{ans}
\newtheorem{exercicio}{}

\setlength{\topmargin}{-1.0in}
\setlength{\oddsidemargin}{0in}
\setlength{\textheight}{10.1in}
\setlength{\textwidth}{6.5in}
\setlength{\baselineskip}{12mm}

\extraheadheight{0.7in}
\firstpageheadrule
\runningheadrule
\lhead{
        \begin{minipage}[c]{1.7cm}
        \includegraphics[width=1.7cm]{unb.pdf}
        \end{minipage}%
        \hspace{0pt}
        \begin{minipage}[c]{4in}
          {Universidade de Brasília} --
          {Departamento de Matemática}
\end{minipage}
\vspace*{-0.8cm}
}
% \chead{Universidade de Brasília - Departamento de Matemática}
% \rhead{}
% \vspace*{-2cm}

\extrafootheight{.5in}
\footrule
\lfoot{\disciplina\ - \semestre$^o$/\ano\ - Módulo \numeromodulo}
\cfoot{}
\rfoot{Página \thepage\ de \numpages}

\newcounter{exercicios}
\renewcommand{\theexercicios}{\arabic{exercicios}}

\newenvironment{questao}[1]{
\refstepcounter{exercicios}
\ifx&#1&
\else
   \label{#1}
\fi
\noindent\textbf{Exercício {\theexercicios}:}
}

\newcommand{\resp}[1]{
\noindent{\bf Exercício #1: }}

\def\ano{2024}
\def\semestre{1}
\def\disciplina{Álgebra 1}
\def\nomeabreviado{Álgebra 1}
\def\turma{1}

\newcommand{\im}{{\rm Im\,}}
\newcommand{\dlim}[2]{\displaystyle\lim_{#1\rightarrow #2}}
\newcommand{\minf}{+\infty}
\newcommand{\ninf}{-\infty}
\newcommand{\cp}[1]{\mathbb{#1}}
\newcommand{\sub}{\subseteq}
\newcommand{\n}{\mathbb{N}}
\newcommand{\z}{\mathbb{Z}}
\newcommand{\rac}{\mathbb{Q}}
\newcommand{\real}{\mathbb{R}}
\newcommand{\complex}{\mathbb{C}}

\newcommand{\vesp}[1]{\vspace{ #1  cm}}

\newcommand{\compcent}[1]{\vcenter{\hbox{$#1\circ$}}}
\newcommand{\comp}{\mathbin{\mathchoice
        {\compcent\scriptstyle}{\compcent\scriptstyle}
        {\compcent\scriptscriptstyle}{\compcent\scriptscriptstyle}}}
\renewcommand{\sin}{{\rm sen\,}}
\renewcommand{\tan}{{\rm tg\,}}
\renewcommand{\csc}{{\rm cossec\,}}
\renewcommand{\cot}{{\rm cotg\,}}
\renewcommand{\sinh}{{\rm senh\,}}

\begin{document}
    \begin{center}
    {\Large\bf \disciplina\ - Turma \turma\ -- \semestre$^{o}$/\ano} \\ \vspace{9pt} {\large\bf
        Lista de Exerc{\'\i}cios -- Módulo \numeromodulo}\\ \vspace{9pt} Prof. Jos{\'e} Ant{\^o}nio O. Freitas
    \end{center}

    \hrule

    \vspace{.6cm}

    \textbf{Observação: }\textit{Nos casos em que n\~ao forem especificadas as opera\c{c}\~oes do anel ou do grupo, considere as opera\c{c}\~oes usuais.}

    \vspace{.6cm}

    \questao{} Determinar quais dos seguintes subconjuntos de $\rac$ s{\~a}o suban{\'e}is:
        \begin{multicols}{2}
            \begin{enumerate}[label=({\alph*})]
                \item $\z$
                \item $B = \{x \in \rac \mid x \notin \z\}$
                \item $C = \left\{\dfrac{a}{b} \in \rac \mid a \in \z,\ b \in \z,\ 2 |b \right\}$
                \item $D = \left\{\dfrac{a}{2^n} \in \rac \mid a \in \z \mbox{ e } n \in \z \right\}$
            \end{enumerate}
        \end{multicols}

    \vspace{.3cm}

    \questao{} No anel $(\z \times \z, \oplus, \otimes)$ onde as opera\c{c}\~oes $\oplus$ e $\otimes$ s\~ao definidas por
    \begin{align*}
        (a, b) \oplus (c, d) = (a + c, b + d)\\
        (a ,b) \otimes (c, d) = (ac - bd, ad + bc).
    \end{align*}
    Quais dos seguintes conjuntos s\~ao suban\'eis?
    \begin{enumerate}[label=({\alph*})]
        \item $A = \{(x, y) \in \z \times \z \mid x = 0\}$
        \item $B = \{(x, y) \in \z \times \z \mid y = 0\}$
        \item $C = \{(x, y) \in \z \times \z \mid x = y\}$
        \item $D = \{(x, y) \in \z \times \z \mid x = 2k,\ k \in \z\}$
        \item $E = \{(x, y) \in \z \times \z \mid y = 3k,\ k \in \z\}$
        \item $F = \{(x, y) \in \z \times \z \mid x + y = 2k,\ k \in \z\}$
    \end{enumerate}

    \vspace{.3cm}

    \questao{} No anel $(\rac, \star, \odot)$ onde as opera\c{c}\~oes $\star$ e $\odot$ em $\rac$ definidas por
    \begin{align*}
        x \star y = x + y - 6\\
        x \odot y = x + y - \dfrac{xy}{6}.
    \end{align*}
    Quais dos seguintes subconjuntos s\~ao suban\'eis?
    \begin{enumerate}[label=({\alph*})]
        \item $A = \z$
        \item $B = \{2k \mid k \in \z\}$
        \item $C = \{6k \mid k \in \z\}$
        \item $D = \{3k \mid k \in \z\}$
    \end{enumerate}

    \vspace{.3cm}

    \questao{} Quais dos conjuntos abaixo s\~ao suban\'eis de $M_2(\real)$?
    \begin{align*}
        L_1 &= \left\{\begin{pmatrix}
            a & 0\\
            b & 0
        \end{pmatrix} \mid a, b \in \real\right\}\\
        L_2 &= \left\{\begin{pmatrix}
            a & b\\
            0 & c
        \end{pmatrix} \mid a, b, c \in \real\right\}\\
        L_3 &= \left\{\begin{pmatrix}
            a & 0\\
            0 & b
        \end{pmatrix} \mid a, b \in \real\right\}\\
        L_4 &= \left\{\begin{pmatrix}
            0 & a\\
            c & b
        \end{pmatrix} \mid a, b, c \in \real\right\}\\
        L_5 &= \left\{\begin{pmatrix}
            0 & a\\
            0 & b^2 + 1
        \end{pmatrix} \mid a, b \in \real\right\}
    \end{align*}

    \vspace{.3cm}

    \questao{} Quais dos conjuntos abaixo s\~ao suban\'eis de $M_2(\z_2)$?
    \begin{align*}
        L_1 &= \left\{\begin{pmatrix}
            \overline{a} & \overline{0}\\
            \overline{b} & \overline{0}
        \end{pmatrix} \mid a, b \in \z_2\right\}\\
        L_2 &= \left\{\begin{pmatrix}
            \overline{a} & \overline{b}\\
            \overline{0} & \overline{c}
        \end{pmatrix} \mid a, b, c \in \z_2\right\}\\
        L_3 &= \left\{\begin{pmatrix}
            \overline{a} & \overline{0}\\
            \overline{0} & \overline{b}
        \end{pmatrix} \mid a, b \in \z_2\right\}\\
        L_4 &= \left\{\begin{pmatrix}
            \overline{0} & \overline{a}\\
            \overline{c} & \overline{b}
        \end{pmatrix} \mid a, b, c \in \z_2\right\}\\
        L_5 &= \left\{\begin{pmatrix}
            \overline{0} & \overline{a}\\
            \overline{0} & \overline{b}^2 + \overline{1}
        \end{pmatrix} \mid a, b \in \z_2\right\}
    \end{align*}

    \vspace{.3cm}

    \questao{} Determine todos os suban\'eis do anel $(\z_8, \oplus, \otimes)$.

    \vspace{.3cm}

    \questao{} Determine todos os suban\'eis do anel $(\z_{16}, \oplus, \otimes)$.

    \vspace{.3cm}

    \questao{} Mostre que a interse\c{c}\~ao de dois suban\'eis de um anel $A$ \'e ainda um subanel de $A$.

    \vspace{.3cm}

    \questao{} \'E verdade que a uni\~ao de suban\'eis \'e um subanel?

    \vspace{.3cm}

    \questao{} Seja $(A, + , \cdot)$ um anel e $x \in A$ fixo. Mostre que o conjunto
    \[
        N(x) = \{y \in A \mid xy = yx\}
    \]
    \'e um subanel de $A$.

    \newpage

    \questao{} Verifique se $L = \{ a + b\sqrt{2} \mid a, b \in \rac\}$ {\'e} um subanel
    do anel $\mathbb{R}$.

    \vspace{.3cm}

    \questao{} Seja $d \in \z$ e considere o subconjunto de $M_2(\z)$ dado por
    \[
        M_2^d(\z) = \left\{\begin{pmatrix} a & db \\ b & a \end{pmatrix} \mid a, b \in \z\right\}.
    \]
    Mostre que $M_2^d(\z)$ é um subanel de $M_2(\z)$.

    \vspace{.3cm}

    \questao{} Seja $X$ um conjunto infinito. Sabemos que $(\mathcal{P}(X), \Delta, \cap)$ é um anel com unidade. Seja
    \[
        R = \{A \in \mathcal{P}(X) \mid A \mbox{ é finito}\}.
    \]
    Prove as seguintes afirmações:
    \begin{enumerate}[label=({\alph*})]
        \item $R$ é um subanel de $\mathcal{P}(X)$.

        \item $R$ não possui unidade.

        \item Para todo $A \in R$, $A \ne \emptyset$ existe $B \in R$, $B \ne \emptyset$, tal que $A \cap B = \emptyset$.

        \item Para todo $A \in \mathcal{P}(X)$, $A \ne X$, $A \ne \emptyset$ existe $B \in \mathcal{P}(X)$, $B \ne \emptyset$, tal que $A \cap B = \emptyset$.
    \end{enumerate}

    \vspace{.6cm}

    \questao{} Verifique se s\~ao ideiais:
    \begin{enumerate}[label=({\alph*})]
        \item  $I = \{\overline{0}, \overline{2}, \overline{4}\}$ no anel $\z_6$;

        \item $I = m\z \times n\z$ no anel $\z \times \z$, em que $m$, $n \in \z$;

        \item $I = \{x \in \z \mid 25 \mbox{ divide } 35x\}$ no anel $\z$;

        \item $I = \{x \in \z \mid x \mbox{ divide } 24\}$ no anel $\z$;

        \item $I = \{x \in \z \mid 6 \mbox{ divide } x \mbox{ e } 24 \mbox{ divide } x^2\}$ no anel $\z$;

        \item $I = \z$ no anel $(\rac, \oplus, \odot)$ em que a $a \oplus b = a + b - 1$ e $a \odot b = a + b - ab$, para todos $a$, $b \in \rac$;

        \item $I = 2\z$ no anel $(\z, +, \cdot)$ em que a adi\c{c}\~ao \'e a usual e $a \cdot b = 0$, para quaisquer $a$, $b \in \z$.
    \end{enumerate}

    \vspace{.3cm}

    \questao{} Seja $(A, +, \cdot)$ um anel comutativo.
    \begin{enumerate}[label=({\alph*})]
        \item Mostre que a interse\c{c}\~ao de quaisquer dois ideais de $A$ \'e sempre um conjunto n\~ao vazio.

        \item Mostre que essa interse\c{c}\~ao \'e sempre um ideal.

        \item A uni\~ao de ideias \'e ainda um ideal?

        \item Sejam $J_1$, $J_2 \subset A$ ideiais tais que $J_1 \subset J_2$. Mostre que $J_1 \cup J_2$ \'e um ideal de $A$.

        \item Sejam $I$ e $J$ ideais de $A$. Mostre que
        \[
            I + J = \{x + y \mid x \in I, y \in J\}
        \]
        \'e um ideal de $A$.

    \item Seja $I$ um ideal de $A$. Considere o conjunto $r(I) = \{x \in A \mid xy = 0_A \mbox{ para todo } y \in I\}$. Mostre que
        $r(I)$ é um ideal de $A$.
    \end{enumerate}

    \vspace{.3cm}

    \questao{} Sendo $A$ um anel, n\~ao necessariamente comutativo, dizemos que $I \subset A$, $I \ne \emptyset$ \'e um \textbf{ideal \`a esquerda} em $A$ se, e somente se:
    \begin{enumerate}[label=({\roman*})]
        \item Para todos $x$, $y \in I$ temos $x - y \in I$;

        \item Para todo $\alpha \in A$ e todo $x \in I$ temos $\alpha x \in I$.
    \end{enumerate}

    Verifique se s\~ao ideiais \`a esquerda em $M_2(\real)$:
    \begin{enumerate}[label=({\alph*})]
        \item $L_1 = \left\{\begin{pmatrix}
            a & 0\\
            0 & b
        \end{pmatrix} \mid a, b \in \real\right\}$.

        \item $L_2 = \left\{\begin{pmatrix}
            a & b\\
            0 & c
        \end{pmatrix} \mid a, b, c \in \real\right\}$.

        \item $L_3 = \left\{\begin{pmatrix}
            a & 0\\
            b & 0
        \end{pmatrix} \mid a, b \in \real\right\}$.

        \item $L_4 = \left\{\begin{pmatrix}
            a & b\\
            0 & 0
        \end{pmatrix} \mid a, b \in \real\right\}$.
    \end{enumerate}

    \vspace{.3cm}

    \questao{}  Seja $(A, +, \cdot)$ um anel unit\'ario. Mostre que o inverso multiplicativo de um elemento $x \in A$, se existir, \'e \'unico.

    \vspace{.3cm}

    \questao{inicioreferencia}{} Verificar se a fun\c{c}\~ao $f : A \to B$ \'e ou n\~ao um homomorfismo do anel $A$ no anel $B$, nos seguintes casos:
    \begin{enumerate}[label=({\alph*})]
        \item $A = \z$, $B = \z$ e $f(x) = x + 1$

        \item $A = \z$, $B = \z$ e $f(x) = 2x$

        \item $A = \z$, $B = \z \times \z$ e $f(x) = (x, 0)$

        \item $A = \z \times \z$, $B = \z$ e $f(x,y) = x$

        \item $A = \z \times \z$, $B = \z \times \z$ e $f(x,y) = (y,x)$

        \item $A = \z$, $B = \z_n$ e $f(x) = \overline{x}$

        \item $A = \complex$, $B = \complex$ e $f(a + bi) = a - bi$

        \item $A = M_2(\real)$, $B = \real$ e $f\left(\begin{bmatrix}
            x & y\\z & t
        \end{bmatrix}\right) = x$

        \item $A = M_2(\real)$, $B = \real$ e $f\left(\begin{bmatrix}
            x & y\\z & t
        \end{bmatrix}\right) = x + t$
    \end{enumerate}

    \vspace{.3cm}

    \questao{} Seja $(\z\times\z, +, \cdot)$ um anel com as seguintes opera\c{c}\~oes
    \begin{align*}
        (a, b) + (c, d) &= (a + c, b + d)\\
        (a, b)\cdot (c, d) &= (ac, 0)
    \end{align*}
    para todos $(a, b)$, $(c, d) \in \z\times\z$.
    Mostre que $ f : \z\times\z \to \z$ definada por $f(a, b) = a$ \'e um homomorfismo sobrejetor.

    \vspace{.3cm}

    \questao{} Seja
    \[
        T_2(\z) = \left\{\begin{bmatrix}a & b\\ 0 & c\end{bmatrix} \mid a, b, c \in \z\right\}
    \]
    um anel. Defina $f : T_2(\z) \to \z$ por
    \[
        f\left(\begin{bmatrix}a & b\\ 0 & c\end{bmatrix}\right) = a.
    \]
    Prove que $f$ é um homomorfismo de anéis.

    \vspace{.3cm}

    \questao{} Seja $(\z\times\z, +, \cdot)$ um anel com as seguintes opera\c{c}\~oes
    \begin{align*}
        (a, b) + (c, d) &= (a + c, b + d)\\
        (a, b)\cdot (c, d) &= (ac, ad + bc)
    \end{align*}
    para todos $(a, b)$, $(c, d) \in \z\times\z$.
    Mostre que $ f : \z\times\z \to \z$ definada por $f(a, b) = a$ \'e um homomorfismo.

    \vspace{.3cm}

    \questao{} Seja $(\z\times\z, +, \cdot)$ um anel com as seguintes opera\c{c}\~oes
    \begin{align*}
        (a, b) + (c, d) &= (a + c, b + d)\\
        (a, b)\cdot (c, d) &= (ac - bd, ad + bc)
    \end{align*}
    para todos $(a, b)$, $(c, d) \in \z\times\z$.
    Mostre que $ f : \z \to \z\times\z$ definada por $f(a) = (a, 0)$ \'e um homomorfismo.

    \vspace{.3cm}

    \questao{} Prove que $f : \rac \to M_3(\rac)$ dada por
    \[
        f(x) = \begin{pmatrix}
            x & 0 & 0\\
            0 & x & 0\\
            0 & 0 & x
        \end{pmatrix}
    \]
    \'e um homomorfismo de an\'eis.

    \newpage

    \questao{fimreferencia} Verifique se as seguintes fun\c{c}\~oes s{\~a}o homomorfismos de an\'eis:
    \begin{enumerate}[label=({\alph*})]
        \item $f : \z\times\z \to \z\times\z$ dado por $f(x,y) = (0,y)$

        \item $f : \z\times\z \to \z$ dado por $f(x,y) = y$

        \item $f : \z\to \z\times\z$ dado por $f(x) = (2x,0)$

        \item $f : \z\to \z_{12}\times\z_{12}$ dado por $f(x) = (\overline{2x},\overline{0})$

        \item $f : \z\times\z \to \z\times\z$ dado por $f(x,y) = (-y,-x)$

        \item $f : \z_6\times\z_6 \to \z_6\times\z_6$ dado por $f(x,y) = (\overline{5y},\overline{5x})$

        \item $f : \z \to \z\times\z$ dado por $f(x) = (0,x)$

        \item $f : \z \to \z_3\times\z_3$ dado por $f(x) = (\overline{0},\overline{x})$
    \end{enumerate}

    \vspace{.3cm}

    \questao{} Determine o kernel dos homomorfismos dos \textbf{Exerc{\'i}cios de \ref{inicioreferencia} a \ref{fimreferencia}}.

    \vspace{.3cm}

    \questao{} Nos \textbf{Exerc{\'\i}cios de \ref{inicioreferencia} a \ref{fimreferencia}} para as fun\c{c}\~oes que forem homomorfismos determine se elas tamb\'em s\~ao isomorfismos.

    \vspace{.3cm}

    \questao{} Seja $f : \complex \to M_2(\real)$ dada por
    \[
        f(a + bi) = \begin{bmatrix}
            a & -b\\
            b & a
        \end{bmatrix}.
    \]
    \begin{enumerate}[label=({\alph*})]
        \item Mostre que $f$ \'e um homomorfismo de an\'eis.

        \item Esse homomorfismo \'e injetor?

        \item \'E sobrejetor?
    \end{enumerate}

    \vspace{.3cm}

    \questao{} Seja $f : \z \to M_2(\z_3)$ dada por
    \[
        f(a) = \begin{bmatrix}
            \overline{a} & \overline{0}\\
            \overline{0} & \overline{a}
        \end{bmatrix}.
    \]
    \begin{enumerate}[label=({\alph*})]
        \item Mostre que $f$ \'e um homomorfismo de an\'eis.

        \item Esse homomorfismo \'e injetor?

        \item \'E sobrejetor?
    \end{enumerate}

    \vspace{.3cm}

    \questao{} Seja $f: A \to B$ um homomorfismo de an{\'e}is. Mostre que:
    \begin{enumerate}[label=({\alph*})]
        \item Se $C$  {\'e} um subanel de $A$, ent{\~a}o $f(C)$ {\'e} um subanel de $B$.

        \item Se $D$ {\'e} um subanel de $B$, ent{\~a}o $f^{-1}(D)$ {\'e} um subanel de $A$.

        \item Se $I$ {\'e} um ideal de $A$ e $f$ \'e sobrejetora, ent{\~a}o $f(I)$ {\'e} um ideal de $B$.

        \item Se $J$ {\'e} um ideal de $B$, ent{\~a}o $f^{-1}(J)$ {\'e} um ideal de $A$.
    \end{enumerate}

    \vspace{.3cm}

    \questao{} D{\^e} um exemplo de an{\'e}is $A$ e $B$ e um homomorfismo $f : A \to B$ tal que $f(1_A) \ne 1_B$.

    \vspace{.3cm}

    \questao{} Sejam os an{\'e}is $A = \{ a + b\sqrt{-2} \mid a,\ b \in \z\}$ e $B = M_2(\z_7)$.
    \begin{enumerate}[label=({\alph*})]
        \item Mostre que $f : A \to B$ dada por
        \[
        f(a + b\sqrt{-2}) =
        \begin{pmatrix}
        \overline{a} & \overline{5}\overline{b}\\
        \overline{b} & \overline{a}
        \end{pmatrix}
        \]
        {\'e} um homomorfismo.

        \item $f$ {\'e} um isomorfimo?
    \end{enumerate}

    \vspace{.3cm}

    \questao{} Considere o conjunto
    \[
        M^\sigma(\z) = \{a_1I_2 + a_2\sigma \mid a_1, a_2 \in \z\}
    \]
    onde
    \[
        I_2 = \begin{pmatrix} 1 & 0\\ 0 & 1\end{pmatrix}, \quad \sigma = \begin{pmatrix} 0 & 1\\ 1 & 0\end{pmatrix}.
    \]
    Este conjunto é um subanel de $M_2(\z)$? Caso afirmativo responda aos itens abaixo:
    \begin{enumerate}[label=({\alph*})]
        \item Se $f : M^\sigma(\z) \to \z$ é dada por $f(a_1I_2 + a_2\sigma) = a_1 + a_2$, então $f$ é um homomorfismo de anéis? Caso afirmativo, determine $\ker(f)$.

        \item Se $g : M^\sigma(\z) \to \z$ é dada por $g(a_1I_2 + a_2\sigma) = a_1 - a_2$, então $g$ é um homomorfismo de anéis? Caso afirmativo, determine $\ker(g)$.
    \end{enumerate}

    \vspace{.3cm}

    \questao{} {\'E} verdadeiro ou falso: $\z$ e $\z_{m}$ para $m > 1$ s{\~a}o an{\'e}is
    isomorfos.

    \vspace{.3cm}

    \questao{} Considere os seguintes an{\'e}is: $(\real, +, \cdot)$ e $(\real, \oplus, \odot)$, sendo $a \oplus b = a + b + 1$ e $a \odot b = a + b + ab$. Mostre que $f : \real \to \real$ dado por $f(x) = x + 1$, para todo $x \in \real$, {\'e} um isomorfimos de $(\real, \oplus, \odot)$ em $(\real, +, \cdot)$.

    \vspace{.3cm}

    %\questao{} Seja $A$ um anel de integridade. Mostre que se $x \in A$ \'e tal que $x^2 = 1$, ent\~ao $x = 1$ ou $x = -1$.

    %\vspace{.3cm}

    %\questao{} Seja $A$ \'e um anel de integridade. Mostre que se $x \in A$ \'e tal que $x ^2 = x$, ent\~ao $x = 0$ ou $x = 1$.

    %\vspace{.3cm}

    %\questao{} Seja $A$ um anel com unidade tal que $x^2 = x$ para todo $x \in A$. Mostre que $A$ \'e um anel de integridade se, e somente se, $A = \{0, 1\}$.

    %\vspace{.3cm}


%    \questao{} Seja $(A, +, \cdot)$ um anel comutativo e com unidade. Mostre que se $I$ {\'e} um ideal de $A$, ent\~ao
%    \[
%        \left(\dfrac{A}{I}, \oplus, \otimes\right)
%    \]
%    {\'e} um anel comutativo  e com unidade.

%    \vspace{.3cm}

%    \questao{} Suponha que $A$ {\'e} um anel com unidade e $I$ um ideal de $A$. Mostre que $a + I \in A/I$ {\'e} invers{\'\i}vel se, e somente se, existe $r \in A$ de modo que $ar - 1 \in I$.

%    \vspace{.3cm}

%    \questao{} D{\^e} um exemplo de um anel de integridade $A$ e de um ideal $I$ em $A$ tal que $A/I$ n{\~a}o {\'e} de integridade.

    \questao{} Verifique se os seguintes conjuntos com a opera\c{c}\~ao dada \'e ou n\~ao um grupo. Em caso afirmativo, o grupo \'e comutativo?
    \begin{enumerate}[label=({\alph*})]
        \item $(\z, \star)$, onde $x \star y = x + xy$, para $x$, $y \in \z$;

        \item $(\z, \star)$, onde $x \star y = x + y + xy$, para $x$, $y \in \z$;

        \item $(\z, \star)$, onde $x \star y = xy + 2x$, para $x$, $y \in \z$;

        \item $(\rac, \star)$, onde $x \star y = x + xy$, para $x$, $y \in \rac$;

        \item $(\real^*, \star)$, onde $x \star y = \dfrac{x}{y}$, para $x$, $y \in \real$;

        \item $(\real_+, \star)$, onde $x \star y = \sqrt{x^2 + y^2}$, para $x$, $y \in \real_+$;

        \item $(\real, \star)$, onde $x \star y = \sqrt[3]{x^3 + y^3}$, para $x$, $y \in \real$.

        \item $(G, \cdot)$, onde $G = \{x \in \rac \mid x > 0\}$ e $\cdot$ \'e a multiplica\c{c}\~ao de n\'umeros racionais.

        \item $(G, \cdot)$, onde $G = \left\{\dfrac{1 + 2m}{1 + 2n} \mid m, n \in \z\right\}$ e $\cdot$ \'e a multiplica\c{c}\~ao de n\'umeros racionais.

        \item $(G, +)$, onde $G = \{0, \pm 2, \pm 4, \pm 6, \dots\}$ e $+$ \'e a soma de n\'umeros inteiros.

        \item $(G, \star)$, onde $G = \{0, \pm 2, \pm 4, \pm 6, \dots\}$ e $\star$ \'e definida como $x \star y = x + y - xy$.

        \item $(G, +)$, onde $G = \{a + b\sqrt{2} \mid a, b \in \rac\}$ e $+$ \'e a soma de n\'umeros reais.

        \item $(G, \cdot)$, onde $G = \{a + b\sqrt{2} \in \real^* \mid a, b \in \rac\}$ e $\cdot$ \'e a multipli\c{c}\~ao de n\'umeros reais.

        \item $(G, +)$, onde $G = \{a + b\sqrt[3]{2} \mid a, b \in \rac\}$ e $+$ \'e a soma de n\'umeros reais.

        \item $(G, \cdot)$, onde $G = \{a + b\sqrt[3]{2} \in \real^* \mid a, b \in \rac\}$ e $\cdot$ \'e a multipli\c{c}\~ao de n\'umeros reais..
    \end{enumerate}

    \vspace{.3cm}

    \questao{} Seja
    \[
        \complex = \{a + bi \mid a, b \in \real\}
    \]
    e $i^2 = -1$. Mostre que:
    \begin{enumerate}[label=({\alph*})]
        \item $(\complex, +)$ \'e um grupo abeliano, onde
        \[
            (a + bi) + (c + di) = (a + c) + (b + d)i
        \]
        para $a + bi$, $c + di \in \complex$.
        \item Para $\complex^* = \complex - \{0\}$, $(\complex^*, \cdot)$ \'e um grupo abeliano, onde
        \[
            (a + bi)\cdot (c + di) = (ac - bd) + (ad + bc)i
        \]
        para $a + bi$, $c + di \in \complex$.
    \end{enumerate}

    \vspace{.3cm}

    \questao{} Verifique se o conjunto $\rac_{>0}$ dos n{\'u}meros racionais estritamente positivos com a
     opera{\c c}{\~a}o dada {\'e} ou n{\~a}o um grupo. Justifique sua
    resposta.
    \begin{multicols}{2}
        \begin{enumerate}[label=({\alph*})]
            \item $(\rac_{>0},\cdot)$

            \item $(\rac_{>0}, +)$
        \end{enumerate}
    \end{multicols}

    \vspace{.3cm}

    \questao{} Seja $z  = a + bi \in \mathbb{C}$, onde $a$, $b \in \real$. Definimos $|z| = \sqrt{a^2 + b^2}$. Prove que $G=\{z \in \mathbb{C} \mid |z| = 1\}$ {\'e} um grupo
    abeliano com a opera{\c c}{\~a}o de multiplica{\c c}{\~a}o de n{\'u}meros complexos.

    \vspace{.3cm}

    \questao{} Mostre que o conjunto $\rac[\sqrt{2}]^*=\{ a + b\sqrt{2} \in
    \mathbb{R}^* \mid  a, b \in \rac \}$ {\'e} um grupo multiplicativo abeliano.

    \vspace{.3cm}

    \questao{} No conjunto $\z \times \z$ considere a opera\c{c}\~ao de soma definida por
    \[
        (x, y) + (z, t) = (x + z, y + t)
    \]
    para $(x, y)$, $(z, t) \in \z \times \z$. Mostre que $(\z\times\z, +)$ \'e um grupo abeliano.

    \vspace{.3cm}

    \questao{} Considere o seguinte conjunto
    \[
        G = \left\{\begin{pmatrix} 1 & a & b\\ 0 & 1 & c \\ 0 & 0 & 1\end{pmatrix} \mid a, b, c \in \real\right\}.
    \]
    Mostre que $(G, \cdot)$, onde $\cdot$ é a multiplicação de matrizes, é um grupo. Esse grupo é comutativo?

    \vspace{.3cm}

    \questao{} Seja $p \ge 2$, um número primo. Mostre que o conjunto
    \[
        G = \left\{
                    \begin{pmatrix}
                        \overline{1} & \overline{a} & -\overline{a} & \overline{b}\\
                        \overline{0} & \overline{1} & \overline{0} & \overline{b}\\
                        \overline{0} & \overline{0} & \overline{1} & \overline{b}\\
                        \overline{0} & \overline{0} & \overline{0} & \overline{1}\\
                    \end{pmatrix}
                \mid \overline{a}, \overline{b} \in \z_p
            \right\}
    \]
    é um grupo com a multiplicação de matrizes. Esse grupo é comutativo?
    \vspace{.3cm}

    \questao{} Quais dos seguintes subconjuntos $G$ de $\z_{13}$ s{\~a}o grupos
    com a opera{\c c}{\~a}o de multiplica{\c c}{\~a}o?
    \begin{multicols}{2}
        \begin{enumerate}[label=({\alph*})]
            \item $G=\{\overline{1},\overline{12}\}$;

            \item $G=\{\overline{1},\overline{5},\overline{8},\overline{12}\}$;

            \item $G=\{\overline{1},\overline{2},\overline{3},\overline{4}, \overline{5},\overline{6},\overline{7}, \overline{8},\overline{9},\overline{10},\overline{11},\overline{12}\}$

            \item $G=\{\overline{1}, \overline{3},\overline{5},\overline{7},\overline{9},\overline{11}\}$.
        \end{enumerate}
    \end{multicols}

    \vspace{.3cm}

    %\questao{} Determine $f$, $g \in S_3$ tais que:
   % \begin{enumerate}[label=({\alph*})]
   %     \item $(f \comp g)^3 \ne f^3\comp g^3$

%        \item $(f \comp g)^2 \ne f^2\comp g^2$
%    \end{enumerate}

%    \vspace{.3cm}

%    \questao{} Determine $f$, $g \in S_4$ tais que:
%    \begin{enumerate}[label=({\alph*})]
%        \item $(f \comp g)^4 \ne f^4\comp g^4$

%        \item $(f \comp g)^3 \ne f^3\comp g^3$
%    \end{enumerate}

%    \vspace{.3cm}

%    \questao{} Considere o grupo $S_3$:
%    \begin{enumerate}[label=({\alph*})]
%        \item Determine todos os elementos $f \in S_3$ tais que $f^2 = Id$ e $f \ne Id$.

%        \item Determine todos os elementos $g \in S_3$ tais que $g^3 = Id$ e $g \ne Id$.
%    \end{enumerate}

%    \vspace{.3cm}

%    \questao{} Considere o grupo $S_4$:
%    \begin{enumerate}[label=({\alph*})]
%        \item Determine todos os elementos $f \in S_4$ tais que $f^2 = Id$ e $f \ne Id$.

%        \item Determine todos os elementos $g \in S_4$ tais que $g^3 = Id$ e $g \ne Id$.

%        \item Determine todos os elementos $g \in S_4$ tais que $g^4 = Id$ e $g \ne Id$.
%    \end{enumerate}

%    \vspace{.3cm}

%    \questao{} Seja $V = \{1, f, g, h\}$ o seguinte subconjunto do grupo $S_4$:
%    \begin{align*}
%        1 = \begin{pmatrix}
%            1 & 2 & 3 & 4\\
%            1 & 2 & 3 & 4
%        \end{pmatrix}; \quad f = \begin{pmatrix}
%            1 & 2 & 3 & 4\\
%            2 & 1 & 4 & 3
%        \end{pmatrix}\\
%        g = \begin{pmatrix}
%            1 & 2 & 3 & 4\\
%            3 & 4 & 1 & 2
%        \end{pmatrix}; \quad h = \begin{pmatrix}
%            1 & 2 & 3 & 4\\
%            4 & 3 & 2 & 1
%        \end{pmatrix}.
%    \end{align*}
%    \begin{enumerate}[label=({\alph*})]
%        \item Prove que $(V, \comp)$ \'e um grupo contendo 4 elementos, onde $\comp$ \'e a opera\c{c}\~ao de $S_4$.

%        \item Prove que $(V, \comp)$ \'e um grupo abeliano.
%    \end{enumerate}

%    \vspace{.3cm}

%    \questao{} Considere o grupo $S_7$ e sejam
%    \begin{align*}
%        1 &= \begin{pmatrix}
%            1 & 2 & 3 & 4 & 5 & 6 & 7\\
%            1 & 2 & 3 & 4 & 5 & 6 & 7
%        \end{pmatrix}\\
%        \sigma &= \begin{pmatrix}
%                1 & 2 & 3 & 4 & 5 & 6 & 7\\
%                3 & 4 & 2 & 6 & 7 & 5 & 1
%            \end{pmatrix}.\\
%        \beta &= \begin{pmatrix}
%                1 & 2 & 3 & 4 & 5 & 6 & 7\\
%                3 & 1 & 2 & 6 & 7 & 4 & 5
%            \end{pmatrix}.
%    \end{align*}.
%    \begin{enumerate}[label=({\alph*})]
%        \item Encontre o menor $l \ge 0$ tal que $\sigma^l = 1$.

%        \item Encontre $\delta \in S_7$ tal que $\sigma\comp\delta = 1$.

%        \item Encontre o menor $k \ge 0$ tal que $\beta^k = 1$.

%        \item Encontre $\gamma \in S_7$ tal que $\gamma\comp\beta = 1$.
%    \end{enumerate}

    \vspace{.3cm}

    \questao{} Seja $(G,*)$ um grupo. Mostre que:
    \begin{enumerate}[label={\alph*})]
        \item O elemento neutro de $G$ {\'e} {\'u}nico.

        \item Existe um {\'u}nico inverso para cada $x \in G$.

        \item Para todo $x \in G$, $(x^{-1})^{-1} = x$.
    \end{enumerate}

    \vspace{.3cm}

    \questao{} Seja $(G,*)$ um grupo com elemento neutro $e$. Para $x\in
    G$, considere a nota{\c c}{\~a}o $x^n=x*x*\cdots *x$ ($n$ vezes).
    \begin{enumerate}[label=({\alph*})]
        \item Prove que se
        $x^2 = e$, para todo $x\in G$, ent{\~a}o $G$ {\'e} um grupo abeliano.

        \item Mostre que se $x\in G$ {\'e} tal que $x^2 = x$, ent{\~a}o $x$ {\'e} o elemento neutro.
    \end{enumerate}

    \newpage

    \questao{} Sejam $G$ um grupo e $x$, $y$, $z \in G$. Prove que:
    \begin{enumerate}[label=({\alph*})]
        \item Se $xy = xz$, ent\~ao $y = z$.

        \item Se $yx = zx$, ent\~ao $y = z$.
    \end{enumerate}

    \vspace{.3cm}

    \questao{} Na tabela abaixo encontra-se representada a opera\c{c}\~ao $\cdot$ definida no conjunto $G = \{e, a , b, c, d, f\}$ de tal modo que $(G, \cdot)$ \'e um grupo:
    \[
        \begin{tabular}{|c|c|c|c|c|c|c|c|}
            \hline
            $\cdot$ & e & a & b & c & d & f\\
            \hline
            e & e & a & b & c & d & f\\
            \hline
            a & a & b & e & f & c & d\\
            \hline
            b & b & e & a & d & f & c\\
            \hline\
            c & c & d & f & e & a & b\\
            \hline
            d & d & f & c & b & e & a\\
            \hline
            f & f & c & d & a & b & e\\
            \hline
        \end{tabular}
    \]
    \begin{enumerate}[label=({\alph*})]
        \item Esse grupo \'e comutativo?

        \item Determine $x \in G$ de maneira que
        \[
            d^{-1}\cdot f \cdot x \cdot b = (a \cdot b \cdot c)^{-1}.
        \]
    \end{enumerate}

    \vspace{.3cm}

    \questao{} Na tabela abaixo encontra-se representada a opera\c{c}\~ao $\cdot$ definida no conjunto $G = \{e, a , b, c, d, f\}$ de tal modo que $(G, \cdot)$ \'e um grupo:
    \[
        \begin{tabular}{|c|c|c|c|c|c|c|c|}
            \hline
            $\cdot$ & e & a & b & c & d & f\\
            \hline
            e & e & a & b & c & d & f\\
            \hline
            a & a & b & c & d & f & e\\
            \hline
            b & b & c & d & f & e & a\\
            \hline\
            c & c & d & f & e & a & b\\
            \hline
            d & d & f & e & a & b & c\\
            \hline
            f & f & e & a & b & c & d\\
            \hline
        \end{tabular}
    \]
    \begin{enumerate}[label=({\alph*})]
        \item Esse grupo \'e comutativo?

        \item Determine $x \in G$ de maneira que
        \[
            (c \cdot d)^{-1}\cdot x^{-1} \cdot b  \cdot a^{-1} = f^{-1}.
        \]
    \end{enumerate}

    \vspace{.3cm}

    \questao{} Sejam $(G, \cdot)$ um grupo e $a$, $b \in G$. Determine $x \in G$, em termos de $a$ e $b$, tal que
    \[
        x\cdot a \cdot x = b \cdot b \cdot a^{-1}.
    \]

    \vspace{.3cm}

    \questao{} Sejam $(G, \cdot)$ um grupo e $a$, $b \in G$. Suponha que $a \cdot b = e$, onde $e$ \'e o elemento neutro de $G$. Prove que $b \cdot a = e$.

    \vspace{.3cm}

    \questao{} Sejam $(G, \cdot)$ um grupo e $a$, $b \in G$. Suponha que $a \cdot b \cdot a \cdot b = e$, onde $e$ \'e o elemento neutro de $G$. Prove que $b \cdot a \cdot b \cdot a= e$.

    \vspace{.3cm}

    \questao{} Verifique se s\~ao subgrupos:
    \begin{enumerate}[label=({\alph*})]
      \item $H = \{x \in \rac \mid x > 0\}$ de $(\rac^*,\cdot)$.

      \item $H = \left\{\dfrac{1 + 2m}{1 + 2n} \mid m, n \in \z\right\}$ de $(\rac^*,\cdot)$.

      \item $H = \{\cos\theta + i\sin\theta \mid \theta \in \rac\}$ de $(\complex^*,\cdot)$.

      \item $H = \{0, \pm 2, \pm 4, \pm 6, \dots\}$ de $(\z,+)$.

      \item $H = \{0, \pm 2, \pm 4, \pm 6, \dots\}$ do grupo $(\rac - \{1\},\star)$ onde $\star$ \'e definida como $x \star y = x + y - xy$.

      \item $H = \{a + b\sqrt{2} \mid a, b \in \rac\}$ de $(\real,+)$.

      \item $H = \{a + b\sqrt{2} \in \real^* \mid a, b \in \rac\}$ de $(\real^*,\cdot)$.

      \item $H = \{a + b\sqrt[3]{2} \mid a, b \in \rac\}$ de $(\real,+)$.

      \item $H = \{a + b\sqrt[3]{2} \in \real^* \mid a, b \in \rac\}$ de $(\real^*,\cdot)$.
    \end{enumerate}

    \vspace{.3cm}

    \questao{} Determine todos os subgrupos do grupo aditivo $\z_4$.

    \vspace{.3cm}

    \questao{} Seja
    \[
    	GL_2(\real) = GL(2, \real) = \left\{A = \begin{bmatrix}x & y\\z & t\end{bmatrix} \mid x, y, z, t \in \real,\ \det(A) \ne 0 \right\}.
    \]
    \begin{enumerate}[label=({\alph*})]
      \item Mostre que $GL_2(\real)$ com a opera\c{c}\~ao de multiplica\c{c}\~ao de matrizes \'e um grupo. Esse grupo \'e abeliano?

      \item Seja
      \[
          H = \left\{ \begin{pmatrix}
              \cos a & \sin a\\ - \sin a & \cos a
          \end{pmatrix} \mid a \in \real\right\}.
      \]
      Mostre que $H$ \'e um subgrupo de $GL_2(\real)$.

      \item Seja
      \[
          K = \left\{ \begin{pmatrix}
              a & b\\ -b & a
          \end{pmatrix} \mid a, b \in \real \mbox{ e n\~ao nulos simultaneamente}\right\}.
      \]
      Mostre que $K$ \'e um subgrupo de $GL_2(\real)$.
    \end{enumerate}

    \questao{} Sejam $H$ e $K$ subgrupos de um grupo $G$ (com nota{\c c}{\~a}o
    multiplicativa).
    \begin{enumerate}[label=({\alph*})]
      \item Mostre que $H\cap K$ tamb{\'e}m {\'e} subgrupo de $G$.

      \item Seja $g\in G$ um elemento fixado. Mostre que o conjunto
      $g^{-1}Hg=\{ g^{-1}xg \mid x\in H \} $ {\'e} um subgrupo de $G$.

      \item Prove que $H\cup K$ {\'e} subgrupo de $G$ se, e somente se,
      $H\subseteq K$ ou $K\subseteq H$.

      \item Demonstre que $HK=\{hk \mid h\in H, k\in K\}$ {\'e} subgrupo
      de $G$ se, e somente se, $HK=KH$.

      [\emph{Nota: $HK=KH$ \textbf{n{\~a}o} quer dizer que $hk=kh$,
      para todo $h\in H, k\in K$; significa que $hk=k_1h_1 \in KH$ e $kh=h_2k_2 \in
      HK$, para todo $h\in H, k\in K$.}]
    \end{enumerate}

    \vspace{.3cm}
    \questao{} Seja $G$ um grupo com nota\c{c}\~ao multiplicativa e $a$ um elemento de $G$. Prove que $N(a) = \{x \in G \mid ax = xa\}$ \'e um subgrupo de $G$.

    \vspace{.3cm}

    \questao{} Seja $G$ um grupo com nota\c{c}\~ao multiplicativa. Considere o subconjunto $Z(G) = \{x \in G \mid xh = hx, \mbox{ para todo } h \in G\}$. Mostre que:
    \begin{enumerate}[label=({\alph*})]
      \item $Z(G)$ \'e um subgrupo de $G$.

      \item $G$ \'e abeliano se, e somente se, $Z(G) = G$.
    \end{enumerate}

    \vspace{.3cm}

    \questao{} Considere o conjunto $\z_{24}$. Defina
    \[
        G = \{ \overline{a} \in \z_{24}^* \mid \mbox{ existe } \overline{b} \in \z_{24} \mbox{ tal que } \overline{a}\overline{b} = \overline{1}\}.
    \]
    \begin{enumerate}[label={\alph*})]
        \item  Mostre que $G$ é um grupo com a multiplicação de $\z_{24}$.

        \item Encontre todos os subgrupos de $G$.
    \end{enumerate}

    \vspace{.3cm}

    \questao{} Considere o conjunto $\z_{20}$. Defina
    \[
        G = \{ \overline{a} \in \z_{20}^* \mid \mbox{ existe } \overline{b} \in \z_{20} \mbox{ tal que } \overline{a}\overline{b} = \overline{1}\}.
    \]
    \begin{enumerate}[label={\alph*})]
        \item  Mostre que $G$ é um grupo com a multiplicação de $\z_{20}$.

        \item Encontre todos os subgrupos de $G$.
    \end{enumerate}

    \vspace{.3cm}

   \questao{nucleo_homomorfismo} Verificar em cada caso se $f$ \'e um homomorfismo de grupos.
    \begin{enumerate}[label=({\alph*})]
        \item $f: \z \to \z$ dada por $f(x) = kx$, sendo $\z$ o grupo aditivo dos inteiros e $k$ um n\'umero inteiro fixo.

        \item $f: \real^* \to \real^*$ dada por $f(x) = |x|$ sendo $\real^*$ o grupo multiplicativo dos reais.

        \item $f: \real \to \real$ dada por $f(x) = x + 1$, onde $\real$ \'e o grupo aditivo dos reais.

        \item $f: \z \to \z \times \z$ dada por $f(x) = (x, 0)$, onde $\z$ e $\z \times \z$ denotam grupos aditivos.

        \item $f: \z \times \z \to \z$ dada por $f(x,y) = x$, onde $\z \times \z$ e $\z$ s\~ao grupos aditivos.

        \item $f: \z \to \real^*_+$ dada por $f(x) = 2^x$, onde $\z$ \'e grupo aditivo e $\real^*_+$ \'e grupo multiplicativo.
    \end{enumerate}

    \newpage

    \questao{} Das fun\c{c}\~oes a seguir, algumas s\~ao homomorfismos do grupo multiplicativo $\complex^*$. \textbf{Descubra} quais e determine o n\'ucleo de cada uma.
    \begin{multicols}{2}
      \begin{enumerate}[label=({\alph*})]
        \item $f(z) = z^2$

        \item $f(z) = |z|$, aqui $|z| = \sqrt{a^2 + b^2}$, se $z = a + bi$.

        \item $f(z) = \dfrac{1}{z}$

        \item $f(z) = -\dfrac{1}{z}$

        \item $f(z) = -z$

        \item $f(z) = z^3$

        \item $f(z) = \overline{z}$ onde $\overline{z} = a - bi$, se $z = a + bi$.
      \end{enumerate}
    \end{multicols}

    \vspace{.3cm}

    \questao{} Considere o conjunto
    \[
        T = \left\{\begin{bmatrix}x & y\\0 & 1/x\end{bmatrix} \mid x \in \real^*,\ y \in \real\right\}.
    \]
    \begin{enumerate}[label=({\alph*})]
        \item Mostre que $T$ é um grupo não comutativo com a operação de multiplicação de matrizes.

        \item A função $f : \real \to T$ definida por
            \[
                f(t) = \begin{bmatrix}1 & t\\0 & 1\end{bmatrix}
            \]
        é um homomorfismo de grupos? Caso afirmativo, esse homomorfismo é injetor? É sobrejetor?

        \item A função $g : \real \to T$ definida por
            \[
                g(t) = \begin{bmatrix}e^t & 0\\0 & e^{-t}\end{bmatrix}
            \]
        é um homomorfismo de grupos? Caso afirmativo, esse homomorfismo é injetor? É sobrejetor?
    \end{enumerate}

    \vspace{.3cm}

    \questao{} Considere o conjunto
    \[
        SO(2) = \left\{\begin{bmatrix}x & -y\\y & x\end{bmatrix} \mid x, y \in \real,\ x^2 + y^2 = 1\right\}.
    \]
    \begin{enumerate}[label=({\alph*})]
        \item Mostre que $SO(2)$ é um grupo abeliano com a operação de multiplicação de matrizes.

        \item A função $f : \real \to SO(2)$ definida por
            \[
                f(t) = \begin{bmatrix}cos(t) & -\sin(t)\\\sin(t) & \cos(t)\end{bmatrix}
            \]
        é um homomorfismo de grupos? Caso afirmativo, esse homomorfismo é injetor? É sobrejetor?
    \end{enumerate}

    \vspace{.3cm}

    \questao{} Considere os grupos $\real$ e $GL_2(\real)$ com as operações usuais.
    \begin{enumerate}[label=({\alph*})]
        \item A função $f : \real \to GL_2(\real)$ definida por
            \[
                f(x) = \begin{bmatrix}x & 0\\0 & 1\end{bmatrix}
            \]
        é um homomorfismo de grupos? Caso afirmativo, esse homomorfismo é injetor? É sobrejetor?

        \item A função $g : \real \to GL_2(\real)$ definida por
            \[
                g(x) = \begin{bmatrix}1 & x\\0 & 1\end{bmatrix}
            \]
        é um homomorfismo de grupos? Caso afirmativo, esse homomorfismo é injetor? É sobrejetor?
    \end{enumerate}

    \vspace{.3cm}

    \questao{} Considere o conjunto
    \[
        T = \left\{\begin{bmatrix}a & b\\0 & d\end{bmatrix} \mid a, b, d \in \real,\ ad \ne 0 \right\}.
    \]
    \begin{enumerate}[label=({\alph*})]
        \item Mostre que $T$ é um subgrupo de $GL_2(\real)$.

        \item Seja $\phi : T \to \real^*$ definida por
            \[
                \phi\left(\begin{bmatrix}a & b\\0 & d\end{bmatrix}\right) = a^2
            \]
            Mostre que $\phi$ é um homomorfismo de grupos. Esse homomorfismo é injetor? É sobrejetor?
    \end{enumerate}

    \vspace{.3cm}

    \questao{} Sejam $(G, \cdot)$ e $(J, \cdot)$ grupos. Estabele\c{c}a quais das seguintes fun\c{c}\~oes s\~ao homomorfismos e determine seus n\'ucleos.

    \begin{enumerate}[label=({\alph*})]
      \item $f_1 : G \times J \to G$ dada por $f_1(x,y) = x$.

      \item $f_2 : G \times J \to J$ dada por $f_2(x,y) = y$.

      \item $f_3 : G  \to G \times J$ dada por $f_3(x) = (x, 1_J)$.

      \item $f_4 : G \times J \to J \times G$ dada por $f_4(x,y) = (y, x)$.

      \item $f_5 : J \to G \times J$ dada por $f_5(y) = (1_G, y)$.
    \end{enumerate}

    \vspace{.3cm}

    \questao{} Determine o n\'ucleo em cada homomorfismo do \textbf{Exerc{\'\i}cio \ref{nucleo_homomorfismo}}.

    \vspace{.3cm}

    \questao{} Determinar os homomorfismos injetores e os sobrejetores do \textbf{Exerc{\'\i}cio \ref{nucleo_homomorfismo}}.

    \vspace{.3cm}

    \questao{} Sejam $G$ e $J$ grupos multiplicativos, $f : G \to J$ um homomorfismo de grupos e $H$ um subgrupo de $J$. Mostre que $f^{-1}(H) = \{ x \in G \mid f(x) \in H\}$ {\'e} um subgrupo de $G$.

    \vspace{.3cm}

    \questao{} Prove que um grupo $G$ {\'e} abeliano se, e somente se, $f : G \to G$ definada por $f(x) = x^{-1}$ {\'e} um homomorfismo.

    \vspace{.3cm}

    \questao{} Seja $f: G\to H$ um homomorfismo de grupos e $K$ um subgrupo de $H$. Mostre que $\ker(f)\sub f^{-1}(K)$.

    \vspace{.3cm}

    \questao{} Se $\phi : G \to H$ é um homomorfismo de grupos e $G$ é abeliano, prove que $\phi(G)$ também é abeliano.

    \vspace{.3cm}

    \questao{} Seja $f: \z \times \z \to \z \times \z$ definida por $f(x, y) = (x - y, 0)$. Provar que $f$ \'e um homomorfismo do grupo aditivo $\z \times \z$ em si pr\'oprio. Obter $\ker(f)$.

    \vspace{.3cm}

    \questao{} Mostre que $f : \z \to 2\z$ dada por $f(n) = 2n$, para todo $n \in \z$, \'e um isomorfismo do grupo aditivo $\z$ no grupo aditivo $2\z = \{0, \pm 2, \pm 4, \dots\}$.

    \vspace{.3cm}

    \questao{} Seja $a \in \real_+^*$ e $a \ne 1$.
    \begin{enumerate}[label=({\alph*})]
      \item  Mostre que $G = \{a^n \mid n \in \z\}$ \'e um subgrupo de $(\real_+^*, \cdot)$.

      \item Mostre que $f : \z \to G$ tal que $f(n) = a^n$ \'e um isomorfismo de $(\z, +)$ em $(G, \cdot)$.
    \end{enumerate}

    \vspace{.3cm}

    \questao{} Mostre que $G = \{2^m3^n \mid m, n \in \z\}$ e $J = \{m + ni \mid m, n \in \z\}$ s\~ao subgrupos de $(\real_+^*, \cdot)$ e $(\complex, +)$, respectivamente, e que s\~ao isomorfos.

    \vspace{.3cm}

    \questao{} Seja $G = \real - \{-1\}$ e defina a operação binária em $G$ por
    \[
        a * b = a + b + ab.
    \]
    Prove que $G$ é um grupo com essa operação. Mostre também que $f : G \to \real^*$ dada por $f(x) = x + 1$ é um isomorfismo do grupo $G$ no grupo multiplicativo $\real^*$.
\end{document}
