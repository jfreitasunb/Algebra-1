%!TEX program = xelatex

\def\numerosemana{06}

\documentclass[12pt]{exam}

\def\ano{2022}
\def\semestre{1}
\def\disciplina{\'Algebra 1}
\def\turma{2}

\usepackage{caption}
\usepackage{amssymb}
\usepackage{amsmath,amsfonts,amsthm,amstext}
\usepackage[brazil]{babel}
% \usepackage[latin1]{inputenc}
\usepackage{graphicx}
\graphicspath{{/ArquivosLinux/OneDrive/imagens-latex/}{D:/OneDrive - unb.br/imagens-latex/}}
\usepackage{enumitem}
\usepackage{multicol}
\usepackage{answers}
\usepackage{tikz,ifthen}
\usetikzlibrary{lindenmayersystems}
\usetikzlibrary[shadings]
\Newassociation{solucao}{Solution}{ans}
\newtheorem{exercicio}{}

\setlength{\topmargin}{-1.0in}
\setlength{\oddsidemargin}{0in}
\setlength{\textheight}{10.1in}
\setlength{\textwidth}{6.5in}
\setlength{\baselineskip}{12mm}

\extraheadheight{0.7in}
\firstpageheadrule
\runningheadrule
\lhead{
        \begin{minipage}[c]{1.7cm}
        \includegraphics[width=1.7cm]{unb.pdf}
        \end{minipage}%
        \hspace{0pt}
        \begin{minipage}[c]{4in}
          {Universidade de Brasília} --
          {Departamento de Matemática}
\end{minipage}
\vspace*{-0.8cm}
}
% \chead{Universidade de Brasília - Departamento de Matemática}
% \rhead{}
% \vspace*{-2cm}

\extrafootheight{.5in}
\footrule
\lfoot{\disciplina\ - \semestre$^o$/\ano\ - Módulo \numeromodulo}
\cfoot{}
\rfoot{Página \thepage\ de \numpages}

\newcounter{exercicios}
\renewcommand{\theexercicios}{\arabic{exercicios}}

\newenvironment{questao}[1]{
\refstepcounter{exercicios}
\ifx&#1&
\else
   \label{#1}
\fi
\noindent\textbf{Exercício {\theexercicios}:}
}

\newcommand{\resp}[1]{
\noindent{\bf Exercício #1: }}

\def\ano{2024}
\def\semestre{1}
\def\disciplina{Álgebra 1}
\def\nomeabreviado{Álgebra 1}
\def\turma{1}

\newcommand{\im}{{\rm Im\,}}
\newcommand{\dlim}[2]{\displaystyle\lim_{#1\rightarrow #2}}
\newcommand{\minf}{+\infty}
\newcommand{\ninf}{-\infty}
\newcommand{\cp}[1]{\mathbb{#1}}
\newcommand{\sub}{\subseteq}
\newcommand{\n}{\mathbb{N}}
\newcommand{\z}{\mathbb{Z}}
\newcommand{\rac}{\mathbb{Q}}
\newcommand{\real}{\mathbb{R}}
\newcommand{\complex}{\mathbb{C}}

\newcommand{\vesp}[1]{\vspace{ #1  cm}}

\newcommand{\compcent}[1]{\vcenter{\hbox{$#1\circ$}}}
\newcommand{\comp}{\mathbin{\mathchoice
        {\compcent\scriptstyle}{\compcent\scriptstyle}
        {\compcent\scriptscriptstyle}{\compcent\scriptscriptstyle}}}
\renewcommand{\sin}{{\rm sen\,}}
\renewcommand{\tan}{{\rm tg\,}}
\renewcommand{\csc}{{\rm cossec\,}}
\renewcommand{\cot}{{\rm cotg\,}}
\renewcommand{\sinh}{{\rm senh\,}}

\begin{document}
    \begin{center}
        {\Large\bf \disciplina\ - Turma \turma\ -- \semestre$^{o}$/\ano} \\ \vspace{9pt} {\large\bf
            Lista de Exerc{\'\i}cios -- Semana \numerosemana}\\ \vspace{9pt} Prof. Jos{\'e} Ant{\^o}nio O. Freitas
    \end{center}
    \hrule

    \vspace{.6cm}

    \begin{center}
        \textit{Nota\c{c}\~oes:}
    \end{center}
    \begin{multicols}{2}
        \begin{enumerate}[label={\roman*})]
            \item $\n^* = \n - \{0\}$

            \item $\real^*_+ = \{x \in \real \mid x > 0\}$

            \item $\real^*_- = \{x \in \real \mid x < 0\}$

            \item $\real_+ = \{x \in \real \mid x \ge 0\}$

            \item $\real_- = \{x \in \real \mid x \le 0\}$

            \item $]c,d[\ = (c,d) = \{x \in \real \mid c < x < d\}$

            \item $[c,d] = \{x \in \real \mid c \le x \le d\}$

            \item $[c,d[\ = [c, d) = \{x \in \real \mid c \le x < d\}$

            \item $]c,d] = (c, d] = \{x \in \real \mid c < x \le d\}$

            \item Em $\z_m$ vamos denotar a opera\c{c}\~ao de soma $\oplus$ por $+$.

            \item Em $\z_m$ vamos denotar a opera\c{c}\~ao de multiplica\c{c}\~ao $\otimes$ por $\cdot$.
        \end{enumerate}
    \end{multicols}

    \vspace{.6cm}

    \questao{} Seja $f(x) = x^4$ e $g(x) = x^7$. Verifique que $(f\circ g)(x) = (g\circ f)(x)$.

    \vspace{.3cm}

    \questao{} Dadas as fun\c{c}\~oes $f(x) = 3x + m$ e $g(x) = ax + 2$, determine condi\c{c}\~oes sobre $a$ e $m$ para que $(f\circ g)(x) = (g\circ f)(x)$.

    \vspace{.3cm}

    \questao{} Sejam $f$, $g$ e $h$ fun\c{c}\~oes de $\real$ em $\real$ definidas por $f(x) = x + 2$, $g(x) = x^2 - 1$ e $h(x) = 3x$, respectivamente.
    \begin{enumerate}[label={\alph*})]
        \item Determine $f \circ g$, $f \circ h$, $g \circ f$, $g \circ h$ e $h \circ g$.
        \item Verifique que $(f \circ g)\circ h = f \circ (g \circ h)$.
    \end{enumerate}

    \vspace{.3cm}

    \questao{} Dada as fun\c{c}\~oes
    \[
        f(x) = \begin{cases}
            1, & \mbox{ se } x < 0\\
            2x^2, & \mbox{ se } 0 \le x \le 1\\
            0, & \mbox{ se } x > 1
        \end{cases} \qquad g(x) = \begin{cases}
            x, & \mbox{ se } x < 0\\
            0, & \mbox{ se } 0 \le x \le 1\\
            1, & \mbox{ se } x > 1.
        \end{cases}
    \]
    Determine $f\circ g$.

    \vspace{.3cm}

    \questao{} Dada as fun\c{c}\~oes
    \[
        f(x) = \begin{cases}
            x^2 + 2, & \mbox{ se } x \le -1\\
            \dfrac{1}{x - 2}, & \mbox{ se } -1 < x < 1\\
            4 - x^2, & \mbox{ se } x \ge 1
        \end{cases} \qquad g(x) = 2 - 3x.
    \]
    Determine $f\circ g$ e $g \circ f$.

    \newpage

    \questao{} Seja $f: \real^2 \to \real$ dada por $f(x,y) = xy$.
    \begin{enumerate}[label={\alph*})]
        \item $f$ {\'e} injetora?

        \item $f$ {\'e} sobrejetora?
    \end{enumerate}

    \vspace{.3cm}

    \questao{} Seja $f : A \to [-9,-1)$ dada por $f(x) = \dfrac{4x + 3}{3 - x}$.
    \begin{enumerate}[label={\alph*})]
        \item Determine $A$.

        \item Mostre que $f$ \'e injetora.

        \item \'E verdade que $f$ \'e sobrejetora?
    \end{enumerate}

    \vspace{.3cm}

    \questao{} Sejam $S = \{1,2\}$ e $T = \{a, b, c\}$.
    \begin{enumerate}[label={\alph*})]
        \item Existem quantas funções $f : S \to T$? Descreva cada uma delas.

        \item Quantas funções injetoras $f : S \to T$ existem? Descreva cada uma delas.

        \item Quantas funções sobrejetoras $f : S \to T$ existem? Descreva cada uma delas.

        \item Quantas funções bijetoras $f : S \to T$ existem? Descreva cada uma delas.
    \end{enumerate}

    \vspace{.3cm}

    \questao{} Seja $f : A \to (1,10]$ dada por $f(x) = \dfrac{4 - 11x}{4 - 2x}$.
    \begin{enumerate}[label={\alph*})]
        \item Determine $A$.

        \item Mostre que $f$ \'e injetora.

        \item \'E verdade que $f$ \'e sobrejetora?
    \end{enumerate}

    \vspace{.3cm}

    \questao{} Considere a fun{\c c}{\~a}o $f : \z \times \z \to \z \times \z$ tal que $f(x,y) = (2x + 3, 4y + 5)$. Prove que $f$ {\'e} injetora. Verifique se $f$ {\'e} bijetora.

    \vspace{.3cm}

    \questao{} Mostre que a fun\c{c}\~ao $f : \z \to \z$ dada por $f(n) = 2n$, \'e injetora mas n\~ao \'e sobrejetora.

    \vspace{.3cm}

    \questao{} Mostre que a fun\c{c}\~ao $h : \real^2 \to \real^2$ tal que $h(x, y) = (\sqrt[3]{x}, y^5)$ \'e sobrejetora.

    \vspace{.3cm}

    \questao{} Encontre exemplos de conjuntos $A$, $B$ e $C$ e funções $f : A \to B$ e $g : B \to C$ tais que $g \circ f : A \to C$ é sobrejetora mas $f : A \to B$ não é sobrejetora.

    \vspace{.3cm}

    \questao{} Considere a fun{\c c}{\~a}o $g : \z_5 \times \z_9 \to \z_5 \times \z_9$ tal que $f(\overline{x},\overline{y}) = (\overline{2} \overline{x} + \overline{3}, \overline{4}\overline{y} + \overline{5})$.
    \begin{enumerate}[label={\alph*})]
        \item $g$ \'e injetora?

        \item $g$ \'e sobrejetora?
    \end{enumerate}

    \vspace{.3cm}

    \questao{} Achar uma fun{\c c}{\~a}o $f : A \to B$, com $A$ e $B$ subconjuntos de $\real$, para cada caso abaixo:
    \begin{enumerate}[label={\alph*})]
        \item $A = \real$, $B \nsub \real$ e $f$ injetora e n{\~a}o sobrejetora.

        \item $A \nsub \real$, $B = \real$ e $f$ injetora e n{\~a}o sobrejetora.

        \item $A = \real$, $B \nsub \real$ e $f$ sobrejetora e n{\~a}o injetora.

        \item $A \nsub \real$, $B = \real$ e $f$ sobrejetora e n{\~a}o injetora.
    \end{enumerate}

    \vspace{.3cm}

    \questao{} Classificar (se poss{\'\i}vel) em injetora ou sobrejetora as seguintes fun{\c c}{\~o}es de $\real$ em $\real$.

    \begin{multicols}{2}
        \begin{enumerate}[label={\alph*})]
            \item $f(x) = x^3$

            \item $f(x) = x^2 - 5x - 6$

            \item $f(x) = 2^x$

            \item $f(x) = | \sin x |$

            \item $f(x) = x + | x |$

            \item $f(x) = x + 3$

            \item $f(x) = \mid x - 1\mid$

            \item $f(x) = \dfrac{1}{x}$

            \item $f(x) = 1 - x^2$

            \item $f(x) = |x|(x - 1)$
        \end{enumerate}
    \end{multicols}

    \vspace{.3cm}

    \questao{} Sejam $f : A \to B$, $g : A \to B$ e $h : B \to C$ fun\c{c}\~oes. Prove que se $h$ \'e injetora e $h \circ g = h \circ f$, ent\~ao $g = f$.

    \vspace{.3cm}

    \questao{} Se as fun{\c c}{\~o}es $f : A \to B$ e $g : B\to A$ s{\~a}o
    tais que $g\circ f$ {\'e} injetora, ent{\~a}o $f$ {\'e} injetora.

    \vspace{.3cm}

    \questao{} Dê um exemplo de funções $f$ e $g$ tais que $g \circ f$ é injetora mas $g$ não é injetora.

    \vspace{.3cm}

    \questao{} Se as fun{\c c}{\~o}es $f : A \to B$ e $g : B\to A$ s{\~a}o
    tais que $g\circ f$ {\'e} sobrejetora, ent{\~a}o g {\'e} sobrejetora.

    \vspace{.3cm}

    \questao{} Mostre que toda fun{\c c}{\~a}o injetora (sobrejetora) de um conjunto finito em si mesmo {\'e} tamb{\'e}m sobrejetora (injetora).

    \vspace{.3cm}

    \questao{} Suponha que $f$, $g : A \to A$ são funções. Se $f \circ f = g \circ g$, então $f = g$? Prove ou dê um contra-exemplo.

    \vspace{.3cm}

    \questao{} Uma função $f : \real \to \real$ é dita \textit{estritamente crescente} se para todo $x_1 < x_2$ temos $f(x_1) < f(x_2)$. Mostre que se $f : \real \to \real$ é estritamente crescente, então $f$ é injetora.
\end{document}
