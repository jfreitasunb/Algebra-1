%!TEX program = xelatex
% !TEX encoding = ISO-8859-1
\def\ano{2022}
\def\semestre{1}
\def\disciplina{\'Algebra 1}
\def\turma{2}

\documentclass[12pt]{exam}

\usepackage{caption}
\usepackage{amssymb}
\usepackage{amsmath,amsfonts,amsthm,amstext}
\usepackage[brazil]{babel}
% \usepackage[latin1]{inputenc}
\usepackage{graphicx}
\graphicspath{{/ArquivosLinux/OneDrive/imagens-latex/}{D:/OneDrive - unb.br/imagens-latex/}}
\usepackage{enumitem}
\usepackage{multicol}
\usepackage{answers}
\usepackage{tikz,ifthen}
\usetikzlibrary{lindenmayersystems}
\usetikzlibrary[shadings]
\Newassociation{solucao}{Solution}{ans}
\newtheorem{exercicio}{}

\setlength{\topmargin}{-1.0in}
\setlength{\oddsidemargin}{0in}
\setlength{\textheight}{10.1in}
\setlength{\textwidth}{6.5in}
\setlength{\baselineskip}{12mm}

\extraheadheight{0.7in}
\firstpageheadrule
\runningheadrule
\lhead{
        \begin{minipage}[c]{1.7cm}
        \includegraphics[width=1.7cm]{unb.pdf}
        \end{minipage}%
        \hspace{0pt}
        \begin{minipage}[c]{4in}
          {Universidade de Brasília} --
          {Departamento de Matemática}
\end{minipage}
\vspace*{-0.8cm}
}
% \chead{Universidade de Brasília - Departamento de Matemática}
% \rhead{}
% \vspace*{-2cm}

\extrafootheight{.5in}
\footrule
\lfoot{\disciplina\ - \semestre$^o$/\ano\ - Módulo \numeromodulo}
\cfoot{}
\rfoot{Página \thepage\ de \numpages}

\newcounter{exercicios}
\renewcommand{\theexercicios}{\arabic{exercicios}}

\newenvironment{questao}[1]{
\refstepcounter{exercicios}
\ifx&#1&
\else
   \label{#1}
\fi
\noindent\textbf{Exercício {\theexercicios}:}
}

\newcommand{\resp}[1]{
\noindent{\bf Exercício #1: }}

\def\ano{2024}
\def\semestre{1}
\def\disciplina{Álgebra 1}
\def\nomeabreviado{Álgebra 1}
\def\turma{1}

\newcommand{\im}{{\rm Im\,}}
\newcommand{\dlim}[2]{\displaystyle\lim_{#1\rightarrow #2}}
\newcommand{\minf}{+\infty}
\newcommand{\ninf}{-\infty}
\newcommand{\cp}[1]{\mathbb{#1}}
\newcommand{\sub}{\subseteq}
\newcommand{\n}{\mathbb{N}}
\newcommand{\z}{\mathbb{Z}}
\newcommand{\rac}{\mathbb{Q}}
\newcommand{\real}{\mathbb{R}}
\newcommand{\complex}{\mathbb{C}}

\newcommand{\vesp}[1]{\vspace{ #1  cm}}

\newcommand{\compcent}[1]{\vcenter{\hbox{$#1\circ$}}}
\newcommand{\comp}{\mathbin{\mathchoice
        {\compcent\scriptstyle}{\compcent\scriptstyle}
        {\compcent\scriptscriptstyle}{\compcent\scriptscriptstyle}}}
\renewcommand{\sin}{{\rm sen\,}}
\renewcommand{\tan}{{\rm tg\,}}
\renewcommand{\csc}{{\rm cossec\,}}
\renewcommand{\cot}{{\rm cotg\,}}
\renewcommand{\sinh}{{\rm senh\,}}
\renewcommand{\qedsymbol}{$\blacksquare$}
\begin{document}
    \begin{center}
        {\Large\bf \disciplina\ - Turma \turma\ -- \semestre$^{o}$/\ano} \\ \vspace{9pt} {\large\bf
        Solu\c{c}\~ao de Exerc{\'\i}cio}\\
        \vspace{9pt} Prof. Jos{\'e} Ant{\^o}nio O. Freitas
    \end{center}
    \hrule

    \vspace{.6cm}

    Solu\c{c}\~ao do Teste Discursivo realizado no dia \textbf{09/09/2022}.

    \vspace{.6cm}

    \questao{}
        \begin{enumerate}[label={\arabic*})]
            \item Sejam $(A, +, \cdot)$ um anel e $x \in A$ um elemento fixo. Mostre que o conjunto $N(x) = \{y \in A \mid xy = yx \}$ é um subanel de $A$.

            \item Suponha que $(A, +, \cdot)$ é um anel comutativo e que $I$ e $J$ são ideais de $A$. Mostre que $I \cap J$ é um ideal de $A$.
        \end{enumerate}

    \noindent\textbf{Solu\c{c}\~ao:}
    \begin{enumerate}[label=({\arabic*})]
        \item Primeiro precisamos mostrar que $N(x) \ne \emptyset$. Para isso, observe que em qualquer anel $A$ sempre vale que
            \[ y \cdot 0_A = 0_A = 0_A\cdot y\]
        para todo $y \in A$. Logo $0_A \in N(x)$ e assim $N(x) \ne \emptyset$.

        Agora sejam $a$, $b \in N(x)$. Precisamos mostrar que $a + (-b)\in N(x)$ e que $a\cdot b \in N(x)$. Como $a$ e $b$ estão em $N(x)$
        então
        \begin{align}
            xa = ax\label{p1}\\
            xb = bx\label{p2}.
        \end{align}
        Da definição do conjunto $N(x)$, precisamos mostrar que
        \begin{align*}
            x[a + (-b)] &= [a + (-b)]x\\
            x(ab) &= (ab)x
        \end{align*}
        para concluir que $a + (-b) \in N(x)$ e que $ab \in N(x)$. Mas
        \begin{align*}
            x[a + (-b)] &= xa + x(-b) \\ &\stackrel{\eqref{p1}}{=} ax + (-(xb)) \\ &\stackrel{\eqref{p2}}{=} xa + (-(bx)) \\ &= xa + (-b)x \\ &= [a + (-b)]x
        \end{align*}
        e assim $a + (-b) \in N(x)$, como queríamos. Além disso,
        \begin{align*}
            x(ab) = (xa)b \stackrel{\eqref{p1}}{=} (ax)b = a(xb) \stackrel{\eqref{p2}}{=} a(bx) = (ab)x
        \end{align*}
        e com isso $ab \in N(x)$.

        Portanto, $N(x)$ é de fato um subanel de $A$. \hspace{.1cm} \qedsymbol
        \item Sejam $I$ e $J$ ideais de um anel comutativo $(A, +, \cdot)$. Queremos mostrar que $I \cap J$ também é um ideal de $A$. Para
            isso o primeiro passo é mostrar que $I \cap J \ne \emptyset$. Mas $I$ e $J$ são ideias de $A$, logo $0_A \in I$ e $0_A \in J$.
            Logo $0_A \in I \cap J$, ou seja, $I \cap J \ne \emptyset$.

            Agora sejam $x$, $y \in I \cap J$. Assim $x$, $y \in I$ e $x$, $y \in J$. Mas $I$ e $J$ são ideais de $A$, daí
            \begin{align}
                x + (-y) \in I,\label{soma_I}\\
                \alpha x \in I,\label{produto_I}\\
                x + (-y) \in J,\label{soma_J}\\
                \alpha x \in J\label{produto_J},
            \end{align}
            para todo $\alpha \in A$. Logo de \eqref{soma_I} e \eqref{soma_J} segue que $x + (-y) \in I \cap J$, e de \eqref{produto_I} e
            \eqref{produto_J} segue que $\alpha x \in I \cap J$. Portanto $I \cap J$ é um ideal de $A$, como queríamos. \hspace{.1cm} \qedsymbol
    \end{enumerate}
\end{document}
