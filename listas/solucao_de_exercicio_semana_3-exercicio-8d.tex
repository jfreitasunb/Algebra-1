%!TEX program = xelatex
% !TEX encoding = ISO-8859-1
\def\ano{2020}
\def\semestre{1}
\def\disciplina{\'Algebra 1}
\def\turma{C}

\documentclass[12pt]{exam}

\usepackage{caption}
\usepackage{amssymb}
\usepackage{amsmath,amsfonts,amsthm,amstext}
\usepackage[brazil]{babel}
% \usepackage[latin1]{inputenc}
\usepackage{graphicx}
\graphicspath{{/ArquivosLinux/OneDrive/imagens-latex/}{D:/OneDrive - unb.br/imagens-latex/}}
\usepackage{enumitem}
\usepackage{multicol}
\usepackage{answers}
\usepackage{tikz,ifthen}
\usetikzlibrary{lindenmayersystems}
\usetikzlibrary[shadings]
\Newassociation{solucao}{Solution}{ans}
\newtheorem{exercicio}{}

\setlength{\topmargin}{-1.0in}
\setlength{\oddsidemargin}{0in}
\setlength{\textheight}{10.1in}
\setlength{\textwidth}{6.5in}
\setlength{\baselineskip}{12mm}

\extraheadheight{0.7in}
\firstpageheadrule
\runningheadrule
\lhead{
        \begin{minipage}[c]{1.7cm}
        \includegraphics[width=1.7cm]{unb.pdf}
        \end{minipage}%
        \hspace{0pt}
        \begin{minipage}[c]{4in}
          {Universidade de Brasília} --
          {Departamento de Matemática}
\end{minipage}
\vspace*{-0.8cm}
}
% \chead{Universidade de Brasília - Departamento de Matemática}
% \rhead{}
% \vspace*{-2cm}

\extrafootheight{.5in}
\footrule
\lfoot{\disciplina\ - \semestre$^o$/\ano\ - Módulo \numeromodulo}
\cfoot{}
\rfoot{Página \thepage\ de \numpages}

\newcounter{exercicios}
\renewcommand{\theexercicios}{\arabic{exercicios}}

\newenvironment{questao}[1]{
\refstepcounter{exercicios}
\ifx&#1&
\else
   \label{#1}
\fi
\noindent\textbf{Exercício {\theexercicios}:}
}

\newcommand{\resp}[1]{
\noindent{\bf Exercício #1: }}

\def\ano{2024}
\def\semestre{1}
\def\disciplina{Álgebra 1}
\def\nomeabreviado{Álgebra 1}
\def\turma{1}

\newcommand{\im}{{\rm Im\,}}
\newcommand{\dlim}[2]{\displaystyle\lim_{#1\rightarrow #2}}
\newcommand{\minf}{+\infty}
\newcommand{\ninf}{-\infty}
\newcommand{\cp}[1]{\mathbb{#1}}
\newcommand{\sub}{\subseteq}
\newcommand{\n}{\mathbb{N}}
\newcommand{\z}{\mathbb{Z}}
\newcommand{\rac}{\mathbb{Q}}
\newcommand{\real}{\mathbb{R}}
\newcommand{\complex}{\mathbb{C}}

\newcommand{\vesp}[1]{\vspace{ #1  cm}}

\newcommand{\compcent}[1]{\vcenter{\hbox{$#1\circ$}}}
\newcommand{\comp}{\mathbin{\mathchoice
        {\compcent\scriptstyle}{\compcent\scriptstyle}
        {\compcent\scriptscriptstyle}{\compcent\scriptscriptstyle}}}
\renewcommand{\sin}{{\rm sen\,}}
\renewcommand{\tan}{{\rm tg\,}}
\renewcommand{\csc}{{\rm cossec\,}}
\renewcommand{\cot}{{\rm cotg\,}}
\renewcommand{\sinh}{{\rm senh\,}}
\renewcommand{\qedsymbol}{$\blacksquare$}
\begin{document}
    \begin{center}
        {\Large\bf \disciplina\ - Turma \turma\ -- \semestre$^{o}$/\ano} \\ \vspace{9pt} {\large\bf
        Solu\c{c}\~ao de Exerc{\'\i}cio}\\
        \vspace{9pt} Prof. Jos{\'e} Ant{\^o}nio O. Freitas
    \end{center}
    \hrule

    \vspace{.6cm}

    Solu\c{c}\~ao do Exerc{\'\i}cio \textbf{8}, letra \textbf{d} da lista da Semana \textbf{03}.

    \vspace{.6cm}

    \questao{} Dados conjuntos $A$ e $B$, defina
    \[
        A \Delta B = (A - B) \cup (B - A).
    \]

    Verifique se a propriedade seguinte \'e verdadeira:    
    \[
        A \Delta (A \Delta B) = B.
    \]


    \noindent\textbf{Solu\c{c}\~ao:} \textit{Verdadeira}. Para prov\'a-la, precisamos mostrar que
    \begin{enumerate}[label=({\roman*})]
        \item $A \Delta (A \Delta B) \subset B$
        \item $B \subset A \Delta (A \Delta B)$.
    \end{enumerate}

    Para mostrar (i)  seja $x \in A \Delta (A \Delta B)$. Ent\~ao pela defini\c{c}\~ao da opera\c{c}\~ao $\Delta$, temos
    \[
        x \in [A - (A \Delta B)] \cup [(A \Delta B) - A].
    \]
    Assim $x \in [A - (A \Delta B)]$ ou $x \in [(A \Delta B) - A]$. Vamos analisar cada op\c{c}\~ao separadamente.

    Suponha que $x \in [A - (A \Delta B)]$. Com isso, $x \in A$ e $x \notin A \Delta B$. Agora, $A \Delta B$ envolve uma uni\~ao de conjuntos e para $x$ n\~ao pertencer a essa uni\~ao $x$ n\~ao pode pertencer a nenhum dos conjuntos. Logo $x \notin A - B$ e $x \notin B - A$. Mas para que $x \notin A - B$, devemos ter $x \notin A$ ou $x \in B$ e para que $x \notin B - A$, devemos ter $x \notin B$ ou $x \in A$. Mas $x \in A$, logo devemos ter obrigatoriamente $x \in B$.

    Agora suponha que $x \in [(A \Delta B) - A]$. Ent\~ao $x \in A \Delta B$ e $x \notin A$. Mas como $x \in A \Delta B$, ent\~ao $x \in (A - B) \cup (B - A)$ e como $x \notin A$ devemos ter obrigatoriamente $x \in B - A$. Com isso $x \in B$.

    Portanto, independente do caso, $x \in B$, isto \'e,
    \[
        A \Delta (A \Delta B) \subset B.
    \]

    Para mostrar (ii) seja $y \in B$. Agora temos duas possibilidades que s\~ao: $y \in A$ ou $y \notin A$. Vamos analisar cada uma separadamente.

    Primeiro suponha que $y \in A$. Como, por hip\'otese, $y \in B$ segue que
    \[
        y \notin A - B
    \]
    e de $y \in A$ segue que
    \[
        y \notin B - A.
    \]
    Assim $y \notin A \Delta B$. Mas $y \in A$, logo $y \in A - (A \Delta B)$. Da{\'\i} $y \in A \Delta (A \Delta B)$.

    Agora suponha que $y \notin A$. Novamente, por hip\'otese, $y \in B$. Assim $y \in B - A$. Com isso $y \in A \Delta B$. Mas ent\~ao $y \notin A - (A \Delta B)$ e $y \in (A \Delta B) - A$. Da{\'\i} $y \in A \Delta (A \Delta B)$.

    Portanto, independente do caso, sempre temos
    \[
        B \subset A \Delta (A \Delta B).
    \]

    Assim, como as duas inclus\~oes s\~ao verdadeiras, segue que
    \[
        A \Delta (A \Delta B) = B,
    \]
    como quer{\'\i}amos. \hspace{.1cm} \qedsymbol
\end{document}