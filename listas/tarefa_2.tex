%!TEX program = xelatex
%!TEX encoding = ISO-8859-1
\def\ano{2020}
\def\semestre{1}
\def\disciplina{\'Algebra 1}
\def\turma{C}
\def\numerosemana{04}

\documentclass[12pt]{exam}

\usepackage{caption}
\usepackage{amssymb}
\usepackage{amsmath,amsfonts,amsthm,amstext}
\usepackage[brazil]{babel}
% \usepackage[latin1]{inputenc}
\usepackage{graphicx}
\graphicspath{{/ArquivosLinux/OneDrive/imagens-latex/}{D:/OneDrive - unb.br/imagens-latex/}}
\usepackage{enumitem}
\usepackage{multicol}
\usepackage{answers}
\usepackage{tikz,ifthen}
\usetikzlibrary{lindenmayersystems}
\usetikzlibrary[shadings]
\def\ano{2024}
\def\semestre{1}
\def\disciplina{Álgebra 1}
\def\nomeabreviado{Álgebra 1}
\def\turma{1}

\newcommand{\im}{{\rm Im\,}}
\newcommand{\dlim}[2]{\displaystyle\lim_{#1\rightarrow #2}}
\newcommand{\minf}{+\infty}
\newcommand{\ninf}{-\infty}
\newcommand{\cp}[1]{\mathbb{#1}}
\newcommand{\sub}{\subseteq}
\newcommand{\n}{\mathbb{N}}
\newcommand{\z}{\mathbb{Z}}
\newcommand{\rac}{\mathbb{Q}}
\newcommand{\real}{\mathbb{R}}
\newcommand{\complex}{\mathbb{C}}

\newcommand{\vesp}[1]{\vspace{ #1  cm}}

\newcommand{\compcent}[1]{\vcenter{\hbox{$#1\circ$}}}
\newcommand{\comp}{\mathbin{\mathchoice
        {\compcent\scriptstyle}{\compcent\scriptstyle}
        {\compcent\scriptscriptstyle}{\compcent\scriptscriptstyle}}}
\renewcommand{\sin}{{\rm sen\,}}
\renewcommand{\tan}{{\rm tg\,}}
\renewcommand{\csc}{{\rm cossec\,}}
\renewcommand{\cot}{{\rm cotg\,}}
\renewcommand{\sinh}{{\rm senh\,}}

\begin{document}
    \vspace{.6cm}

    \begin{center}
        Lista Semana 07
    \end{center}
    \noindent \textbf{Questão 15:} Sejam $f : A \to B$, $g : A \to B$ e $h : B \to C$ fun\c{c}\~oes. Prove que se $h$ \'e injetora e $h \circ g = h \circ f$, ent\~ao $g = f$.
    
    \vspace{.3cm}

    \noindent \textbf{Questão 18:} Mostrar que toda fun{\c c}{\~a}o injetora (sobrejetora) de um conjunto finito em si mesmo {\'e} tamb{\'e}m sobrejetora (injetora).

    \hrule
    \begin{center}
        Lista Semana 08
    \end{center}

    \noindent \textbf{Questão 10:} Seja $f : A \to B$ uma fun{\c c}{\~a}o e sejam $P \sub
    A$ e $X, Y\sub B$. Mostre que:
    \begin{enumerate}
        \item[e)] $f(f^{-1}(X))= X \cap \mbox{Im}f$ e conclua que se $f$ {\'e} sobrejetora ent{\~a}o
    $f(f^{-1}(X))=X$.

        \item[g)] $f$ \'e sobrejetora se, e somente se, $f^{-1}(T) \ne \emptyset$ para todo $T \sub B$.
    \end{enumerate}

    \vspace{.3cm}

    \hrule
    \begin{center}
        Lista Semana 09
    \end{center}

    \noindent \textbf{Questão 5:} Prove que s\~ao an\'eis:
        \begin{enumerate}
            \item[b)] O conjunto $\rac$ com as opera\c{c}\~oes $x \oplus y = x + y - 1$ e $x \odot y = x + y - xy$.
            \item[c)] O conjunto $\z \times \z$ com as opera\c{c}\~oes:
            \begin{align*}
                (a, b) \oplus (c, d) = (a + c, b + d)\\
                (a ,b) \otimes (c, d) = (ac, ad + bc).
            \end{align*}
        \end{enumerate}
    Quais destes an\'eis s\~ao comutativos? Quais t\^em unidade?

    \vspace{.3cm}

    \noindent \textbf{Questão 12}: Quais dos conjuntos abaixo s\~ao suban\'eis de $M_2(\real)$?
    \begin{align*}
        L_1 &= \left\{\begin{pmatrix}
            a & 0\\
            b & 0
        \end{pmatrix} \mid a, b \in \real\right\}\\
        L_2 &= \left\{\begin{pmatrix}
            a & b\\
            0 & c
        \end{pmatrix} \mid a, b, c \in \real\right\}\\
        L_3 &= \left\{\begin{pmatrix}
            a & 0\\
            0 & b
        \end{pmatrix} \mid a, b \in \real\right\}\\
        L_4 &= \left\{\begin{pmatrix}
            0 & a\\
            c & b
        \end{pmatrix} \mid a, b, c \in \real\right\}\\
        L_5 &= \left\{\begin{pmatrix}
            0 & a\\
            0 & b^2 + 1
        \end{pmatrix} \mid a, b \in \real\right\}
    \end{align*}

    \hrule
    \begin{center}
        Lista Semana 10
    \end{center}
    
    \noindent \textbf{Questão 10}: Seja $f : \complex \to M_2(\real)$ dada por
    \[
        f(a + bi) = \begin{bmatrix}
            a & -b\\
            b & a
        \end{bmatrix}.
    \]
    \begin{enumerate}[label=({\alph*})]
        \item Mostre que $f$ \'e um homomorfismo de an\'eis.
        
        \item Esse homomorfismo \'e injetor?
        
        \item \'E sobrejetor?
    \end{enumerate}


    \vspace{.3cm}

    \noindent \textbf{Questão 12}: Seja $f: A \to B$ um homomorfismo de an{\'e}is. Mostre que:
    \begin{enumerate}
        \item[b)] Se $D$ {\'e} um subanel de $B$, ent{\~a}o $f^{-1}(D)$ {\'e} um subanel de $A$.
        
        \item[c)] Se $I$ {\'e} um ideal de $A$ e $f$ \'e sobrejetora, ent{\~a}o $f(I)$ {\'e} um ideal de $B$.
        
        \item[d)] Se $J$ {\'e} um ideal de $B$, ent{\~a}o $f^{-1}(J)$ {\'e} um ideal de $A$.
    \end{enumerate}

    \vspace{.3cm}

    \noindent \textbf{Questão 21}: Seja $(A, +, \cdot)$ um anel comutativo.
    \begin{enumerate}
        \item[d)] Sejam $J_1$, $J_2 \subset A$ ideiais tais que $J_1 \subset J_2$. Mostre que $J_1 \cup J_2$ \'e um ideal de $A$.

        \item[e)] Sejam $I$ e $J$ ideais de $A$. Mostre que
        \[
            I + J = \{x + y \mid x \in I, y \in J\}
        \]
        \'e um ideal de $A$.
    \end{enumerate}

    \vspace{.3cm}

    \noindent \textbf{Questão 23}: Seja $(A, +, \cdot)$ um anel comutativo e com unidade. Mostre que se $I$ {\'e} um ideal de $A$, ent\~ao
    \[
        \left(\dfrac{A}{I}, \oplus, \otimes\right) 
    \]
    
    {\'e} um anel comutativo  e com unidade.
    
\end{document}