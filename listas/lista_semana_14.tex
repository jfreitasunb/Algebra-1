%!TEX program = xelatex
\def\ano{2021}
\def\semestre{1}
\def\disciplina{\'Algebra 1}
\def\turma{C}
\def\numerosemana{14}

\documentclass[12pt]{exam}

\usepackage{caption}
\usepackage{amssymb}
\usepackage{amsmath,amsfonts,amsthm,amstext}
\usepackage[brazil]{babel}
% \usepackage[latin1]{inputenc}
\usepackage{graphicx}
\graphicspath{{/ArquivosLinux/OneDrive/imagens-latex/}{D:/OneDrive - unb.br/imagens-latex/}}
\usepackage{enumitem}
\usepackage{multicol}
\usepackage{answers}
\usepackage{tikz,ifthen}
\usetikzlibrary{lindenmayersystems}
\usetikzlibrary[shadings]
\Newassociation{solucao}{Solution}{ans}
\newtheorem{exercicio}{}

\setlength{\topmargin}{-1.0in}
\setlength{\oddsidemargin}{0in}
\setlength{\textheight}{10.1in}
\setlength{\textwidth}{6.5in}
\setlength{\baselineskip}{12mm}

\extraheadheight{0.7in}
\firstpageheadrule
\runningheadrule
\lhead{
        \begin{minipage}[c]{1.7cm}
        \includegraphics[width=1.7cm]{unb.pdf}
        \end{minipage}%
        \hspace{0pt}
        \begin{minipage}[c]{4in}
          {Universidade de Brasília} --
          {Departamento de Matemática}
\end{minipage}
\vspace*{-0.8cm}
}
% \chead{Universidade de Brasília - Departamento de Matemática}
% \rhead{}
% \vspace*{-2cm}

\extrafootheight{.5in}
\footrule
\lfoot{\disciplina\ - \semestre$^o$/\ano\ - Módulo \numeromodulo}
\cfoot{}
\rfoot{Página \thepage\ de \numpages}

\newcounter{exercicios}
\renewcommand{\theexercicios}{\arabic{exercicios}}

\newenvironment{questao}[1]{
\refstepcounter{exercicios}
\ifx&#1&
\else
   \label{#1}
\fi
\noindent\textbf{Exercício {\theexercicios}:}
}

\newcommand{\resp}[1]{
\noindent{\bf Exercício #1: }}

\def\ano{2024}
\def\semestre{1}
\def\disciplina{Álgebra 1}
\def\nomeabreviado{Álgebra 1}
\def\turma{1}

\newcommand{\im}{{\rm Im\,}}
\newcommand{\dlim}[2]{\displaystyle\lim_{#1\rightarrow #2}}
\newcommand{\minf}{+\infty}
\newcommand{\ninf}{-\infty}
\newcommand{\cp}[1]{\mathbb{#1}}
\newcommand{\sub}{\subseteq}
\newcommand{\n}{\mathbb{N}}
\newcommand{\z}{\mathbb{Z}}
\newcommand{\rac}{\mathbb{Q}}
\newcommand{\real}{\mathbb{R}}
\newcommand{\complex}{\mathbb{C}}

\newcommand{\vesp}[1]{\vspace{ #1  cm}}

\newcommand{\compcent}[1]{\vcenter{\hbox{$#1\circ$}}}
\newcommand{\comp}{\mathbin{\mathchoice
        {\compcent\scriptstyle}{\compcent\scriptstyle}
        {\compcent\scriptscriptstyle}{\compcent\scriptscriptstyle}}}
\renewcommand{\sin}{{\rm sen\,}}
\renewcommand{\tan}{{\rm tg\,}}
\renewcommand{\csc}{{\rm cossec\,}}
\renewcommand{\cot}{{\rm cotg\,}}
\renewcommand{\sinh}{{\rm senh\,}}

\begin{document}

\begin{center}

    {\Large\bf \disciplina\ - Turma \turma\ -- \semestre$^{o}$/\ano} \\ \vspace{9pt} {\large\bf
        Lista de Exerc{\'\i}cios -- Semana \numerosemana}\\ \vspace{9pt} Prof. Jos{\'e} Ant{\^o}nio O. Freitas
    \end{center}
    \hrule

    \vspace{.6cm}

    \questao{} Seja $G$ um grupo multiplicativo. Se $A \subset G$ e $A \ne \emptyset$, seja $A^{-1} = \{x^{-1} \mid x \in A\}$. Mostre que:
    \begin{enumerate}[label=({\alph*})]
      \item $(A^{-1})^{-1} = A$;

      \item Para todos $A$, $B \subset G$, $A \ne \emptyset$, $B \ne \emptyset$, tem-se que $(AB)^{-1} = B^{-1}A^{-1}$.
    \end{enumerate}

    \vspace{.3cm}

    \questao{} Sejam $G$ um grupo, $N$ um subgrupo normal e $H$ um subgrupo de $G$. Mostre que o subgrupo $H \cap N$ \'e normal em $H$.

    \vspace{.3cm}

    \questao{} Seja $G$ um grupo e $N$ um subgrupo de $G$. Mostre que $N$ \'e normal se, e somente se, $x^{-1}Nx = N$, para todo $x \in G$.
    \noindent\textit{Observa\c{c}\~ao: $x^{-1}Nx = \{x^{-1}nx \mid n \in N\}$}.

    \vspace{.3cm}

    \questao{} Mostre que, se $M$ e $N$ s\~ao subgrupos normais do grupo $G$ e $M \cap N = \{e\}$, ent\~ao $xy = yx$ para quaisquer $x \in M$ e $y \in N$.
    \noindent \textit{Sugest\~ao: Mostre que $(xy)(yx)^{-1} = e$.}

    \vspace{.3cm}

    \questao{} Seja $G$ um grupo multiplicativo. Se $a \in G$ é tal que o subgrupo $[a]$ é normal, mostre que $H = \{x \in G \mid xa = ax\}$ \'e um subgrupo normal de $G$.

    \vspace{.3cm}

    \questao{} Seja $f : G \to L$ um homomorfismo sobrejetor de grupos. Se $H$ \'e um subgrupo normal de $G$, mostre que $f(H)$ \'e um subgrupo normal de $L$.

    \vspace{.3cm}

    \questao{} Sejam $H$ e $K$ subgrupos normais de um grupo $G$, tais que $H \cap K = \{e\}$. Se $h_1$, $h_2 \in H$ e $k_1$, $k_2 \in K$ s\~ao tais que
    \[
      h_1k_1 = h_2k_2,
    \]
    prove que $h_1 = h_2$ e $k_1 = k_2$.

    \vspace{.3cm}

    \questao{} Determine todos os subgrupos n\~ao triviais do grupo aditivo $\z_8$. Para cada subgrupo $H$ encontrado, construa a tabela do grupo quociente $\z_6/H$.

    \vspace{.3cm}

    \questao{} Construa a t\'abua dos seguintes grupos quocientes:
    \begin{enumerate}[label=({\alph*})]
      \item $\z_9/H$ em que $H = \{\overline{0}, \overline{3}, \overline{6}\}$

      \item $\z/2\z$

      \item $(\z \times \z)/(3\z \times 2\z)$
    \end{enumerate}

    \vspace{.3cm}

    \questao{} Seja $G$ um grupo. Sabemos que o conjunto
    \[
        Z(G) = \{x \in G \mid xy = yx, \mbox{ para todo } y \in G\}
    \]
    é um subgrupo de $G$ e esse subgrupo é chamado de \textbf{centro} de $G$.
    \begin{enumerate}[label=({\alph*})]
        \item Mostre que $Z(G)$ é um subgrupo nomal de $G$.

        \item Seja $\phi : G \to G$ um automorfismo. Mostre que $\phi(Z(G)) = Z(G)$.

        \item Dê um exemplo de dois grupos $G$ e $H$ e um homomorfismo de grupos $\phi : G \to H$ tal que $\phi(Z(G)$ não está contido em $Z(H)$.
    \end{enumerate}
\end{document}
