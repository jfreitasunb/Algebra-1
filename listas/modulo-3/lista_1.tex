%!TEX program = xelatex
%!TEX encoding = UTF-8
\def\numeromodulo{3}
\def\numerolista{1}

\documentclass[12pt]{exam}

\def\ano{2022}
\def\semestre{1}
\def\disciplina{\'Algebra 1}
\def\turma{2}

\usepackage{caption}
\usepackage{amssymb}
\usepackage{amsmath,amsfonts,amsthm,amstext}
\usepackage[brazil]{babel}
% \usepackage[latin1]{inputenc}
\usepackage{graphicx}
\graphicspath{{/ArquivosLinux/OneDrive/imagens-latex/}{D:/OneDrive - unb.br/imagens-latex/}}
\usepackage{enumitem}
\usepackage{multicol}
\usepackage{answers}
\usepackage{tikz,ifthen}
\usetikzlibrary{lindenmayersystems}
\usetikzlibrary[shadings]
\Newassociation{solucao}{Solution}{ans}
\newtheorem{exercicio}{}

\setlength{\topmargin}{-1.0in}
\setlength{\oddsidemargin}{0in}
\setlength{\textheight}{10.1in}
\setlength{\textwidth}{6.5in}
\setlength{\baselineskip}{12mm}

\extraheadheight{0.7in}
\firstpageheadrule
\runningheadrule
\lhead{
        \begin{minipage}[c]{1.7cm}
        \includegraphics[width=1.7cm]{unb.pdf}
        \end{minipage}%
        \hspace{0pt}
        \begin{minipage}[c]{4in}
          {Universidade de Brasília} --
          {Departamento de Matemática}
\end{minipage}
\vspace*{-0.8cm}
}
% \chead{Universidade de Brasília - Departamento de Matemática}
% \rhead{}
% \vspace*{-2cm}

\extrafootheight{.5in}
\footrule
\lfoot{\disciplina\ - \semestre$^o$/\ano\ - Módulo \numeromodulo}
\cfoot{}
\rfoot{Página \thepage\ de \numpages}

\newcounter{exercicios}
\renewcommand{\theexercicios}{\arabic{exercicios}}

\newenvironment{questao}[1]{
\refstepcounter{exercicios}
\ifx&#1&
\else
   \label{#1}
\fi
\noindent\textbf{Exercício {\theexercicios}:}
}

\newcommand{\resp}[1]{
\noindent{\bf Exercício #1: }}

\def\ano{2024}
\def\semestre{1}
\def\disciplina{Álgebra 1}
\def\nomeabreviado{Álgebra 1}
\def\turma{1}

\newcommand{\im}{{\rm Im\,}}
\newcommand{\dlim}[2]{\displaystyle\lim_{#1\rightarrow #2}}
\newcommand{\minf}{+\infty}
\newcommand{\ninf}{-\infty}
\newcommand{\cp}[1]{\mathbb{#1}}
\newcommand{\sub}{\subseteq}
\newcommand{\n}{\mathbb{N}}
\newcommand{\z}{\mathbb{Z}}
\newcommand{\rac}{\mathbb{Q}}
\newcommand{\real}{\mathbb{R}}
\newcommand{\complex}{\mathbb{C}}

\newcommand{\vesp}[1]{\vspace{ #1  cm}}

\newcommand{\compcent}[1]{\vcenter{\hbox{$#1\circ$}}}
\newcommand{\comp}{\mathbin{\mathchoice
        {\compcent\scriptstyle}{\compcent\scriptstyle}
        {\compcent\scriptscriptstyle}{\compcent\scriptscriptstyle}}}
\renewcommand{\sin}{{\rm sen\,}}
\renewcommand{\tan}{{\rm tg\,}}
\renewcommand{\csc}{{\rm cossec\,}}
\renewcommand{\cot}{{\rm cotg\,}}
\renewcommand{\sinh}{{\rm senh\,}}

\begin{document}
    \begin{center}
    {\Large\bf \disciplina\ - Turma \turma\ -- \semestre$^{o}$/\ano} \\ \vspace{9pt} {\large\bf
        $\numerolista^a$ Lista de Exercícios -- Módulo \numeromodulo\\ Subanéis, ideais e homomorfismos}\\ \vspace{9pt} Prof. José Antônio O. Freitas
    \end{center}
    \hrule

    \vspace{.6cm}

    \textbf{Observação: }\textit{Nos casos em que não forem especificadas as operações do anel ou do grupo, considere as operações usuais.}

\vspace{.6cm}

\questao{} Determinar quais dos seguintes subconjuntos de $\rac$ são subanéis:
\begin{multicols}{2}
    \begin{enumerate}[label=({\alph*})]
        \item $\z$
        \item $B = \{x \in \rac \mid x \notin \z\}$
        \item $C = \left\{\dfrac{a}{b} \in \rac \mid a \in \z,\ b \in \z,\ 2 |b \right\}$
        \item $D = \left\{\dfrac{a}{2^n} \in \rac \mid a \in \z \mbox{ e } n \in \z \right\}$
    \end{enumerate}
\end{multicols}

\vspace{.3cm}

\questao{} No anel $(\z \times \z, \oplus, \otimes)$ onde as operações $\oplus$ e $\otimes$ são definidas por
\begin{align*}
    (a, b) \oplus (c, d) = (a + c, b + d)\\
    (a ,b) \otimes (c, d) = (ac - bd, ad + bc).
\end{align*}
Quais dos seguintes conjuntos são subanéis?
\begin{enumerate}[label=({\alph*})]
    \item $A = \{(x, y) \in \z \times \z \mid x = 0\}$
    \item $B = \{(x, y) \in \z \times \z \mid y = 0\}$
    \item $C = \{(x, y) \in \z \times \z \mid x = y\}$
    \item $D = \{(x, y) \in \z \times \z \mid x = 2k,\ k \in \z\}$
    \item $E = \{(x, y) \in \z \times \z \mid y = 3k,\ k \in \z\}$
    \item $F = \{(x, y) \in \z \times \z \mid x + y = 2k,\ k \in \z\}$
\end{enumerate}

\vspace{.3cm}

\questao{} No anel $(\rac, \star, \odot)$ onde as operações $\star$ e $\odot$ em $\rac$ definidas por
\begin{align*}
    x \star y = x + y - 6\\
    x \odot y = x + y - \dfrac{xy}{6}.
\end{align*}
Quais dos seguintes subconjuntos são subanéis?
\begin{enumerate}[label=({\alph*})]
    \item $A = \z$
    \item $B = \{2k \mid k \in \z\}$
    \item $C = \{6k \mid k \in \z\}$
    \item $D = \{3k \mid k \in \z\}$
\end{enumerate}

\vspace{.3cm}

\questao{} Quais dos conjuntos abaixo são subanéis de $M_2(\real)$?
\begin{align*}
    L_1 &= \left\{\begin{pmatrix}
        a & 0\\
        b & 0
    \end{pmatrix} \mid a, b \in \real\right\}\\
    L_2 &= \left\{\begin{pmatrix}
        a & b\\
        0 & c
    \end{pmatrix} \mid a, b, c \in \real\right\}\\
    L_3 &= \left\{\begin{pmatrix}
        a & 0\\
        0 & b
    \end{pmatrix} \mid a, b \in \real\right\}\\
    L_4 &= \left\{\begin{pmatrix}
        0 & a\\
        c & b
    \end{pmatrix} \mid a, b, c \in \real\right\}\\
    L_5 &= \left\{\begin{pmatrix}
        0 & a\\
        0 & b^2 + 1
    \end{pmatrix} \mid a, b \in \real\right\}
\end{align*}

\vspace{.3cm}

\questao{} Quais dos conjuntos abaixo são subanéis de $M_2(\z_2)$?
\begin{align*}
    L_1 &= \left\{\begin{pmatrix}
        \overline{a} & \overline{0}\\
        \overline{b} & \overline{0}
    \end{pmatrix} \mid a, b \in \z_2\right\}\\
    L_2 &= \left\{\begin{pmatrix}
        \overline{a} & \overline{b}\\
        \overline{0} & \overline{c}
    \end{pmatrix} \mid a, b, c \in \z_2\right\}\\
    L_3 &= \left\{\begin{pmatrix}
        \overline{a} & \overline{0}\\
        \overline{0} & \overline{b}
    \end{pmatrix} \mid a, b \in \z_2\right\}\\
    L_4 &= \left\{\begin{pmatrix}
        \overline{0} & \overline{a}\\
        \overline{c} & \overline{b}
    \end{pmatrix} \mid a, b, c \in \z_2\right\}\\
    L_5 &= \left\{\begin{pmatrix}
        \overline{0} & \overline{a}\\
        \overline{0} & \overline{b}^2 + \overline{1}
    \end{pmatrix} \mid a, b \in \z_2\right\}
\end{align*}

\vspace{.3cm}

\questao{} Determine todos os subanéis do anel $(\z_8, \oplus, \otimes)$.

\vspace{.3cm}

\questao{} Determine todos os subanéis do anel $(\z_{16}, \oplus, \otimes)$.

\vspace{.3cm}

\questao{} Mostre que a interseção de dois subanéis de um anel $A$ é ainda um subanel de $A$.

\vspace{.3cm}

\questao{} É verdade que a união de subanéis é um subanel?

\vspace{.3cm}

\questao{} Seja $(A, + , \cdot)$ um anel e $x \in A$ fixo. Mostre que o conjunto
\[
N(x) = \{y \in A \mid xy = yx\}
\]
é um subanel de $A$.

\newpage

\questao{} Verifique se $L = \{ a + b\sqrt{2} \mid a, b \in \rac\}$ é um subanel
do anel $\mathbb{R}$.

\vspace{.3cm}

\questao{} Seja $d \in \z$ e considere o subconjunto de $M_2(\z)$ dado por
\[
M_2^d(\z) = \left\{\begin{pmatrix} a & db \\ b & a \end{pmatrix} \mid a, b \in \z\right\}.
\]
Mostre que $M_2^d(\z)$ é um subanel de $M_2(\z)$.

\vspace{.3cm}

\questao{} Seja $X$ um conjunto infinito. Sabemos que $(\mathcal{P}(X), \Delta, \cap)$ é um anel com unidade. Seja
\[
R = \{A \in \mathcal{P}(X) \mid A \mbox{ é finito}\}.
\]
Prove as seguintes afirmações:
\begin{enumerate}[label=({\alph*})]
    \item $R$ é um subanel de $\mathcal{P}(X)$.

    \item $R$ não possui unidade.

    \item Para todo $A \in R$, $A \ne \emptyset$ existe $B \in R$, $B \ne \emptyset$, tal que $A \cap B = \emptyset$.

    \item Para todo $A \in \mathcal{P}(X)$, $A \ne X$, $A \ne \emptyset$ existe $B \in \mathcal{P}(X)$, $B \ne \emptyset$, tal que $A \cap B = \emptyset$.
\end{enumerate}

\vspace{.6cm}

\questao{} Verifique se são ideais:
\begin{enumerate}[label=({\alph*})]
    \item  $I = \{\overline{0}, \overline{2}, \overline{4}\}$ no anel $\z_6$;

    \item $I = m\z \times n\z$ no anel $\z \times \z$, em que $m$, $n \in \z$;

    \item $I = \{x \in \z \mid 25 \mbox{ divide } 35x\}$ no anel $\z$;

    \item $I = \{x \in \z \mid x \mbox{ divide } 24\}$ no anel $\z$;

    \item $I = \{x \in \z \mid 6 \mbox{ divide } x \mbox{ e } 24 \mbox{ divide } x^2\}$ no anel $\z$;

    \item $I = \z$ no anel $(\rac, \oplus, \odot)$ em que a $a \oplus b = a + b - 1$ e $a \odot b = a + b - ab$, para todos $a$, $b \in \rac$;

    \item $I = 2\z$ no anel $(\z, +, \cdot)$ em que a adição é a usual e $a \cdot b = 0$, para quaisquer $a$, $b \in \z$.
\end{enumerate}

\vspace{.3cm}

\questao{} Seja $(A, +, \cdot)$ um anel comutativo.
\begin{enumerate}[label=({\alph*})]
    \item Mostre que a interseção de quaisquer dois ideais de $A$ é sempre um conjunto não vazio.

    \item Mostre que essa interseção é sempre um ideal.

    \item A união de ideias é ainda um ideal?

    \item Sejam $J_1$, $J_2 \subset A$ ideais tais que $J_1 \subset J_2$. Mostre que $J_1 \cup J_2$ é um ideal de $A$.

    \item Sejam $I$ e $J$ ideais de $A$. Mostre que
    \[
    I + J = \{x + y \mid x \in I, y \in J\}
    \]
    é um ideal de $A$.

    \item Seja $I$ um ideal de $A$. Considere o conjunto $r(I) = \{x \in A \mid xy = 0_A \mbox{ para todo } y \in I\}$. Mostre que
    $r(I)$ é um ideal de $A$.
\end{enumerate}

\vspace{.3cm}

\questao{} Sendo $A$ um anel, não necessariamente comutativo, dizemos que $I \subset A$, $I \ne \emptyset$ é um \textbf{ideal à esquerda} em $A$ se, e somente se:
\begin{enumerate}[label=({\roman*})]
    \item Para todos $x$, $y \in I$ temos $x - y \in I$;

    \item Para todo $\alpha \in A$ e todo $x \in I$ temos $\alpha x \in I$.
\end{enumerate}

Verifique se são ideais à esquerda em $M_2(\real)$:
\begin{enumerate}[label=({\alph*})]
    \item $L_1 = \left\{\begin{pmatrix}
        a & 0\\
        0 & b
    \end{pmatrix} \mid a, b \in \real\right\}$.

    \item $L_2 = \left\{\begin{pmatrix}
        a & b\\
        0 & c
    \end{pmatrix} \mid a, b, c \in \real\right\}$.

    \item $L_3 = \left\{\begin{pmatrix}
        a & 0\\
        b & 0
    \end{pmatrix} \mid a, b \in \real\right\}$.

    \item $L_4 = \left\{\begin{pmatrix}
        a & b\\
        0 & 0
    \end{pmatrix} \mid a, b \in \real\right\}$.
\end{enumerate}

\vspace{.3cm}

\questao{}  Seja $(A, +, \cdot)$ um anel unitário. Mostre que o inverso multiplicativo de um elemento $x \in A$, se existir, é único.

\vspace{.3cm}

\questao{inicioreferencia}{} Verificar se a função $f : A \to B$ é ou não um homomorfismo do anel $A$ no anel $B$, nos seguintes casos:
\begin{enumerate}[label=({\alph*})]
    \item $A = \z$, $B = \z$ e $f(x) = x + 1$

    \item $A = \z$, $B = \z$ e $f(x) = 2x$

    \item $A = \z$, $B = \z \times \z$ e $f(x) = (x, 0)$

    \item $A = \z \times \z$, $B = \z$ e $f(x,y) = x$

    \item $A = \z \times \z$, $B = \z \times \z$ e $f(x,y) = (y,x)$

    \item $A = \z$, $B = \z_n$ e $f(x) = \overline{x}$

    \item $A = \complex$, $B = \complex$ e $f(a + bi) = a - bi$

    \item $A = M_2(\real)$, $B = \real$ e $f\left(\begin{bmatrix}
        x & y\\z & t
    \end{bmatrix}\right) = x$

    \item $A = M_2(\real)$, $B = \real$ e $f\left(\begin{bmatrix}
        x & y\\z & t
    \end{bmatrix}\right) = x + t$
\end{enumerate}

\vspace{.3cm}

\questao{} Seja $(\z\times\z, +, \cdot)$ um anel com as seguintes operações
\begin{align*}
    (a, b) + (c, d) &= (a + c, b + d)\\
    (a, b)\cdot (c, d) &= (ac, 0)
\end{align*}
para todos $(a, b)$, $(c, d) \in \z\times\z$.
Mostre que $ f : \z\times\z \to \z$ definida por $f(a, b) = a$ é um homomorfismo sobrejetor.

\vspace{.3cm}

\questao{} Seja
\[
T_2(\z) = \left\{\begin{bmatrix}a & b\\ 0 & c\end{bmatrix} \mid a, b, c \in \z\right\}
\]
um anel. Defina $f : T_2(\z) \to \z$ por
\[
f\left(\begin{bmatrix}a & b\\ 0 & c\end{bmatrix}\right) = a.
\]
Prove que $f$ é um homomorfismo de anéis.

\vspace{.3cm}

\questao{} Seja $(\z\times\z, +, \cdot)$ um anel com as seguintes operações
\begin{align*}
    (a, b) + (c, d) &= (a + c, b + d)\\
    (a, b)\cdot (c, d) &= (ac, ad + bc)
\end{align*}
para todos $(a, b)$, $(c, d) \in \z\times\z$.
Mostre que $ f : \z\times\z \to \z$ definida por $f(a, b) = a$ é um homomorfismo.

\vspace{.3cm}

\questao{} Seja $(\z\times\z, +, \cdot)$ um anel com as seguintes operações
\begin{align*}
    (a, b) + (c, d) &= (a + c, b + d)\\
    (a, b)\cdot (c, d) &= (ac - bd, ad + bc)
\end{align*}
para todos $(a, b)$, $(c, d) \in \z\times\z$.
Mostre que $ f : \z \to \z\times\z$ definida por $f(a) = (a, 0)$ é um homomorfismo.

\vspace{.3cm}

\questao{} Prove que $f : \rac \to M_3(\rac)$ dada por
\[
f(x) = \begin{pmatrix}
    x & 0 & 0\\
    0 & x & 0\\
    0 & 0 & x
\end{pmatrix}
\]
é um homomorfismo de anéis.

\newpage

\questao{fimreferencia} Verifique se as seguintes funções são homomorfismos de anéis:
\begin{enumerate}[label=({\alph*})]
    \item $f : \z\times\z \to \z\times\z$ dado por $f(x,y) = (0,y)$

    \item $f : \z\times\z \to \z$ dado por $f(x,y) = y$

    \item $f : \z\to \z\times\z$ dado por $f(x) = (2x,0)$

    \item $f : \z\to \z_{12}\times\z_{12}$ dado por $f(x) = (\overline{2x},\overline{0})$

    \item $f : \z\times\z \to \z\times\z$ dado por $f(x,y) = (-y,-x)$

    \item $f : \z_6\times\z_6 \to \z_6\times\z_6$ dado por $f(x,y) = (\overline{5y},\overline{5x})$

    \item $f : \z \to \z\times\z$ dado por $f(x) = (0,x)$

    \item $f : \z \to \z_3\times\z_3$ dado por $f(x) = (\overline{0},\overline{x})$
\end{enumerate}

\vspace{.3cm}

\questao{} Determine o kernel dos homomorfismos dos \textbf{Exercícios de \ref{inicioreferencia} a \ref{fimreferencia}}.

\vspace{.3cm}

\questao{} Nos \textbf{Exercícios de \ref{inicioreferencia} a \ref{fimreferencia}} para as funções que forem homomorfismos determine se elas também são isomorfismos.

\vspace{.3cm}

\questao{} Seja $f : \complex \to M_2(\real)$ dada por
\[
f(a + bi) = \begin{bmatrix}
    a & -b\\
    b & a
\end{bmatrix}.
\]
\begin{enumerate}[label=({\alph*})]
    \item Mostre que $f$ é um homomorfismo de anéis.

    \item Esse homomorfismo é injetor?

    \item É sobrejetor?
\end{enumerate}

\vspace{.3cm}

\questao{} Seja $f : \z \to M_2(\z_3)$ dada por
\[
f(a) = \begin{bmatrix}
    \overline{a} & \overline{0}\\
    \overline{0} & \overline{a}
\end{bmatrix}.
\]
\begin{enumerate}[label=({\alph*})]
    \item Mostre que $f$ é um homomorfismo de anéis.

    \item Esse homomorfismo é injetor?

    \item É sobrejetor?
\end{enumerate}

\vspace{.3cm}

\questao{} Seja $f: A \to B$ um homomorfismo de anéis. Mostre que:
\begin{enumerate}[label=({\alph*})]
    \item Se $C$  é um subanel de $A$, então $f(C)$ é um subanel de $B$.

    \item Se $D$ é um subanel de $B$, então $f^{-1}(D)$ é um subanel de $A$.

    \item Se $I$ é um ideal de $A$ e $f$ é sobrejetora, então $f(I)$ é um ideal de $B$.

    \item Se $J$ é um ideal de $B$, então $f^{-1}(J)$ é um ideal de $A$.
\end{enumerate}

\vspace{.3cm}

\questao{} Dê um exemplo de anéis $A$ e $B$ e um homomorfismo $f : A \to B$ tal que $f(1_A) \ne 1_B$.

\vspace{.3cm}

\questao{} Sejam os anéis $A = \{ a + b\sqrt{-2} \mid a,\ b \in \z\}$ e $B = M_2(\z_7)$.
\begin{enumerate}[label=({\alph*})]
    \item Mostre que $f : A \to B$ dada por
    \[
    f(a + b\sqrt{-2}) =
    \begin{pmatrix}
        \overline{a} & \overline{5}\overline{b}\\
        \overline{b} & \overline{a}
    \end{pmatrix}
    \]
    é um homomorfismo.

    \item $f$ é um isomorfismo?
\end{enumerate}

\vspace{.3cm}

\questao{} Considere o conjunto
\[
M^\sigma(\z) = \{a_1I_2 + a_2\sigma \mid a_1, a_2 \in \z\}
\]
onde
\[
I_2 = \begin{pmatrix} 1 & 0\\ 0 & 1\end{pmatrix}, \quad \sigma = \begin{pmatrix} 0 & 1\\ 1 & 0\end{pmatrix}.
\]
Este conjunto é um subanel de $M_2(\z)$? Caso afirmativo responda aos itens abaixo:
\begin{enumerate}[label=({\alph*})]
    \item Se $f : M^\sigma(\z) \to \z$ é dada por $f(a_1I_2 + a_2\sigma) = a_1 + a_2$, então $f$ é um homomorfismo de anéis? Caso afirmativo, determine $\ker(f)$.

    \item Se $g : M^\sigma(\z) \to \z$ é dada por $g(a_1I_2 + a_2\sigma) = a_1 - a_2$, então $g$ é um homomorfismo de anéis? Caso afirmativo, determine $\ker(g)$.
\end{enumerate}

\vspace{.3cm}

\questao{} {É} verdadeiro ou falso: $\z$ e $\z_{m}$ para $m > 1$ são anéis
isomorfos.

\vspace{.3cm}

\questao{} Considere os seguintes anéis: $(\real, +, \cdot)$ e $(\real, \oplus, \odot)$, sendo $a \oplus b = a + b + 1$ e $a \odot b = a + b + ab$. Mostre que $f : \real \to \real$ dado por $f(x) = x + 1$, para todo $x \in \real$, é um isomorfismos de $(\real, \oplus, \odot)$ em $(\real, +, \cdot)$.

\vspace{.3cm}

%\questao{} Seja $A$ um anel de integridade. Mostre que se $x \in A$ é tal que $x^2 = 1$, então $x = 1$ ou $x = -1$.

%\vspace{.3cm}

%\questao{} Seja $A$ é um anel de integridade. Mostre que se $x \in A$ é tal que $x ^2 = x$, então $x = 0$ ou $x = 1$.

%\vspace{.3cm}

%\questao{} Seja $A$ um anel com unidade tal que $x^2 = x$ para todo $x \in A$. Mostre que $A$ é um anel de integridade se, e somente se, $A = \{0, 1\}$.

%\vspace{.3cm}


%    \questao{} Seja $(A, +, \cdot)$ um anel comutativo e com unidade. Mostre que se $I$ é um ideal de $A$, então
%    \[
%        \left(\dfrac{A}{I}, \oplus, \otimes\right)
%    \]
%    é um anel comutativo  e com unidade.

%    \vspace{.3cm}

%    \questao{} Suponha que $A$ é um anel com unidade e $I$ um ideal de $A$. Mostre que $a + I \in A/I$ é inversível se, e somente se, existe $r \in A$ de modo que $ar - 1 \in I$.

%    \vspace{.3cm}

%    \questao{} Dê um exemplo de um anel de integridade $A$ e de um ideal $I$ em $A$ tal que $A/I$ não é de integridade.

\end{document}
