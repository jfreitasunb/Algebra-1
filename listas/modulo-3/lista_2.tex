%!TEX program = xelatex
%!TEX encoding = UTF-8
\def\numeromodulo{3}
\def\numerolista{2}

\documentclass[12pt]{exam}

\def\ano{2022}
\def\semestre{1}
\def\disciplina{\'Algebra 1}
\def\turma{2}

\usepackage{caption}
\usepackage{amssymb}
\usepackage{amsmath,amsfonts,amsthm,amstext}
\usepackage[brazil]{babel}
% \usepackage[latin1]{inputenc}
\usepackage{graphicx}
\graphicspath{{/ArquivosLinux/OneDrive/imagens-latex/}{D:/OneDrive - unb.br/imagens-latex/}}
\usepackage{enumitem}
\usepackage{multicol}
\usepackage{answers}
\usepackage{tikz,ifthen}
\usetikzlibrary{lindenmayersystems}
\usetikzlibrary[shadings]
\Newassociation{solucao}{Solution}{ans}
\newtheorem{exercicio}{}

\setlength{\topmargin}{-1.0in}
\setlength{\oddsidemargin}{0in}
\setlength{\textheight}{10.1in}
\setlength{\textwidth}{6.5in}
\setlength{\baselineskip}{12mm}

\extraheadheight{0.7in}
\firstpageheadrule
\runningheadrule
\lhead{
        \begin{minipage}[c]{1.7cm}
        \includegraphics[width=1.7cm]{unb.pdf}
        \end{minipage}%
        \hspace{0pt}
        \begin{minipage}[c]{4in}
          {Universidade de Brasília} --
          {Departamento de Matemática}
\end{minipage}
\vspace*{-0.8cm}
}
% \chead{Universidade de Brasília - Departamento de Matemática}
% \rhead{}
% \vspace*{-2cm}

\extrafootheight{.5in}
\footrule
\lfoot{\disciplina\ - \semestre$^o$/\ano\ - Módulo \numeromodulo}
\cfoot{}
\rfoot{Página \thepage\ de \numpages}

\newcounter{exercicios}
\renewcommand{\theexercicios}{\arabic{exercicios}}

\newenvironment{questao}[1]{
\refstepcounter{exercicios}
\ifx&#1&
\else
   \label{#1}
\fi
\noindent\textbf{Exercício {\theexercicios}:}
}

\newcommand{\resp}[1]{
\noindent{\bf Exercício #1: }}

\def\ano{2024}
\def\semestre{1}
\def\disciplina{Álgebra 1}
\def\nomeabreviado{Álgebra 1}
\def\turma{1}

\newcommand{\im}{{\rm Im\,}}
\newcommand{\dlim}[2]{\displaystyle\lim_{#1\rightarrow #2}}
\newcommand{\minf}{+\infty}
\newcommand{\ninf}{-\infty}
\newcommand{\cp}[1]{\mathbb{#1}}
\newcommand{\sub}{\subseteq}
\newcommand{\n}{\mathbb{N}}
\newcommand{\z}{\mathbb{Z}}
\newcommand{\rac}{\mathbb{Q}}
\newcommand{\real}{\mathbb{R}}
\newcommand{\complex}{\mathbb{C}}

\newcommand{\vesp}[1]{\vspace{ #1  cm}}

\newcommand{\compcent}[1]{\vcenter{\hbox{$#1\circ$}}}
\newcommand{\comp}{\mathbin{\mathchoice
        {\compcent\scriptstyle}{\compcent\scriptstyle}
        {\compcent\scriptscriptstyle}{\compcent\scriptscriptstyle}}}
\renewcommand{\sin}{{\rm sen\,}}
\renewcommand{\tan}{{\rm tg\,}}
\renewcommand{\csc}{{\rm cossec\,}}
\renewcommand{\cot}{{\rm cotg\,}}
\renewcommand{\sinh}{{\rm senh\,}}

\begin{document}
    \begin{center}
    {\Large\bf \disciplina\ - Turma \turma\ -- \semestre$^{o}$/\ano} \\ \vspace{9pt} {\large\bf
        $\numerolista^a$ Lista de Exercícios -- Módulo \numeromodulo\\ Grupos}\\ \vspace{9pt} Prof. José Antônio O. Freitas
    \end{center}
    \hrule

    \vspace{.6cm}

    \textbf{Observação: }\textit{Nos casos em que não forem especificadas as operações do grupo, considere as operações usuais.}

    \vspace{.6cm}

    \questao{} Seja $(G,*)$ um grupo. Mostre que:
    \begin{enumerate}[label={\alph*})]
        \item O elemento neutro de $G$ é único.

        \item Existe um único inverso para cada $x \in G$.

        \item Para todo $x \in G$, $(x^{-1})^{-1} = x$.
    \end{enumerate}

    \vspace{.3cm}

    \questao{} Seja $(G,*)$ um grupo com elemento neutro $e$. Para $x\in
    G$, considere a notação $x^n=x*x*\cdots *x$ ($n$ vezes).
    \begin{enumerate}[label=({\alph*})]
        \item Prove que se
        $x^2 = e$, para todo $x\in G$, então $G$ é um grupo abeliano.

        \item Mostre que se $x\in G$ é tal que $x^2 = x$, então $x$ é o elemento neutro.
    \end{enumerate}

    \vspace{.3cm}

    \questao{} Sejam $G$ um grupo e $x$, $y$, $z \in G$. Prove que:
    \begin{enumerate}[label=({\alph*})]
        \item Se $xy = xz$, então $y = z$.

        \item Se $yx = zx$, então $y = z$.
    \end{enumerate}

    \vspace{.3cm}

    \questao{} Na tabela abaixo encontra-se representada a operação $\cdot$ definida no conjunto $G = \{e, a , b, c, d, f\}$ de tal modo que $(G, \cdot)$ é um grupo:
    \[
    \begin{tabular}{|c|c|c|c|c|c|c|c|}
        \hline
        $\cdot$ & e & a & b & c & d & f\\
        \hline
        e & e & a & b & c & d & f\\
        \hline
        a & a & b & e & f & c & d\\
        \hline
        b & b & e & a & d & f & c\\
        \hline\
        c & c & d & f & e & a & b\\
        \hline
        d & d & f & c & b & e & a\\
        \hline
        f & f & c & d & a & b & e\\
        \hline
    \end{tabular}
    \]
    \begin{enumerate}[label=({\alph*})]
        \item Esse grupo é comutativo?

        \item Determine $x \in G$ de maneira que
        \[
        d^{-1}\cdot f \cdot x \cdot b = (a \cdot b \cdot c)^{-1}.
        \]
    \end{enumerate}

    \vspace{.3cm}

    \questao{} Na tabela abaixo encontra-se representada a operação $\cdot$ definida no conjunto $G = \{e, a , b, c, d, f\}$ de tal modo que $(G, \cdot)$ é um grupo:
    \[
    \begin{tabular}{|c|c|c|c|c|c|c|c|}
        \hline
        $\cdot$ & e & a & b & c & d & f\\
        \hline
        e & e & a & b & c & d & f\\
        \hline
        a & a & b & c & d & f & e\\
        \hline
        b & b & c & d & f & e & a\\
        \hline\
        c & c & d & f & e & a & b\\
        \hline
        d & d & f & e & a & b & c\\
        \hline
        f & f & e & a & b & c & d\\
        \hline
    \end{tabular}
    \]
    \begin{enumerate}[label=({\alph*})]
        \item Esse grupo é comutativo?

        \item Determine $x \in G$ de maneira que
        \[
        (c \cdot d)^{-1}\cdot x^{-1} \cdot b  \cdot a^{-1} = f^{-1}.
        \]
    \end{enumerate}

    \vspace{.3cm}

    \questao{} Sejam $(G, \cdot)$ um grupo e $a$, $b \in G$. Determine $x \in G$, em termos de $a$ e $b$, tal que
    \[
    x\cdot a \cdot x = b \cdot b \cdot a^{-1}.
    \]

    \vspace{.3cm}

    \questao{} Sejam $(G, \cdot)$ um grupo e $a$, $b \in G$. Suponha que $a \cdot b = e$, onde $e$ é o elemento neutro de $G$. Prove que $b \cdot a = e$.

    \vspace{.3cm}

    \questao{} Sejam $(G, \cdot)$ um grupo e $a$, $b \in G$. Suponha que $a \cdot b \cdot a \cdot b = e$, onde $e$ é o elemento neutro de $G$. Prove que $b \cdot a \cdot b \cdot a= e$.

    \vspace{.3cm}

    \questao{} Verifique se são subgrupos:
    \begin{enumerate}[label=({\alph*})]
        \item $H = \{x \in \rac \mid x > 0\}$ de $(\rac^*,\cdot)$.

        \item $H = \left\{\dfrac{1 + 2m}{1 + 2n} \mid m, n \in \z\right\}$ de $(\rac^*,\cdot)$.

        \item $H = \{\cos\theta + i\sin\theta \mid \theta \in \rac\}$ de $(\complex^*,\cdot)$.

        \item $H = \{0, \pm 2, \pm 4, \pm 6, \dots\}$ de $(\z,+)$.

        \item $H = \{0, \pm 2, \pm 4, \pm 6, \dots\}$ do grupo $(\rac - \{1\},\star)$ onde $\star$ é definida como $x \star y = x + y - xy$.

        \item $H = \{a + b\sqrt{2} \mid a, b \in \rac\}$ de $(\real,+)$.

        \item $H = \{a + b\sqrt{2} \in \real^* \mid a, b \in \rac\}$ de $(\real^*,\cdot)$.

        \item $H = \{a + b\sqrt[3]{2} \mid a, b \in \rac\}$ de $(\real,+)$.

        \item $H = \{a + b\sqrt[3]{2} \in \real^* \mid a, b \in \rac\}$ de $(\real^*,\cdot)$.
    \end{enumerate}

    \vspace{.3cm}

    \questao{} Determine todos os subgrupos do grupo aditivo $\z_4$.

    \vspace{.3cm}

    \questao{} Seja
    \[
    GL_2(\real) = GL(2, \real) = \left\{A = \begin{bmatrix}x & y\\z & t\end{bmatrix} \mid x, y, z, t \in \real,\ \det(A) \ne 0 \right\}.
    \]
    \begin{enumerate}[label=({\alph*})]
        \item Mostre que $GL_2(\real)$ com a operação de multiplicação de matrizes é um grupo. Esse grupo é abeliano?

        \item Seja
        \[
        H = \left\{ \begin{pmatrix}
            \cos a & \sin a\\ - \sin a & \cos a
        \end{pmatrix} \mid a \in \real\right\}.
        \]
        Mostre que $H$ é um subgrupo de $GL_2(\real)$.

        \item Seja
        \[
        K = \left\{ \begin{pmatrix}
            a & b\\ -b & a
        \end{pmatrix} \mid a, b \in \real \mbox{ e não nulos simultaneamente}\right\}.
        \]
        Mostre que $K$ é um subgrupo de $GL_2(\real)$.
    \end{enumerate}

    \questao{} Sejam $H$ e $K$ subgrupos de um grupo $G$ (com notação
    multiplicativa).
    \begin{enumerate}[label=({\alph*})]
        \item Mostre que $H\cap K$ também é subgrupo de $G$.

        \item Seja $g\in G$ um elemento fixado. Mostre que o conjunto
        $g^{-1}Hg=\{ g^{-1}xg \mid x\in H \} $ é um subgrupo de $G$.

        \item Prove que $H\cup K$ é subgrupo de $G$ se, e somente se,
        $H\subseteq K$ ou $K\subseteq H$.

        \item Demonstre que $HK=\{hk \mid h\in H, k\in K\}$ é subgrupo
        de $G$ se, e somente se, $HK=KH$.

        [\emph{Nota: $HK=KH$ \textbf{não} quer dizer que $hk=kh$,
            para todo $h\in H, k\in K$; significa que $hk=k_1h_1 \in KH$ e $kh=h_2k_2 \in
            HK$, para todo $h\in H, k\in K$.}]
    \end{enumerate}

    \vspace{.3cm}
    \questao{} Seja $G$ um grupo com notação multiplicativa e $a$ um elemento de $G$. Prove que $N(a) = \{x \in G \mid ax = xa\}$ é um subgrupo de $G$.

    \vspace{.3cm}

    \questao{} Seja $G$ um grupo com notação multiplicativa. Considere o subconjunto $Z(G) = \{x \in G \mid xh = hx, \mbox{ para todo } h \in G\}$. Mostre que:
    \begin{enumerate}[label=({\alph*})]
        \item $Z(G)$ é um subgrupo de $G$.

        \item $G$ é abeliano se, e somente se, $Z(G) = G$.
    \end{enumerate}

    \vspace{.3cm}

    \questao{} Considere o conjunto $\z_{24}$. Defina
    \[
    G = \{ \overline{a} \in \z_{24}^* \mid \mbox{ existe } \overline{b} \in \z_{24} \mbox{ tal que } \overline{a}\overline{b} = \overline{1}\}.
    \]
    \begin{enumerate}[label={\alph*})]
        \item  Mostre que $G$ é um grupo com a multiplicação de $\z_{24}$.

        \item Encontre todos os subgrupos de $G$.
    \end{enumerate}

    \vspace{.3cm}

    \questao{} Considere o conjunto $\z_{20}$. Defina
    \[
    G = \{ \overline{a} \in \z_{20}^* \mid \mbox{ existe } \overline{b} \in \z_{20} \mbox{ tal que } \overline{a}\overline{b} = \overline{1}\}.
    \]
    \begin{enumerate}[label={\alph*})]
        \item  Mostre que $G$ é um grupo com a multiplicação de $\z_{20}$.

        \item Encontre todos os subgrupos de $G$.
    \end{enumerate}

    \vspace{.3cm}

    \questao{nucleo_homomorfismo} Verificar em cada caso se $f$ é um homomorfismo de grupos.
    \begin{enumerate}[label=({\alph*})]
        \item $f: \z \to \z$ dada por $f(x) = kx$, sendo $\z$ o grupo aditivo dos inteiros e $k$ um número inteiro fixo.

        \item $f: \real^* \to \real^*$ dada por $f(x) = |x|$ sendo $\real^*$ o grupo multiplicativo dos reais.

        \item $f: \real \to \real$ dada por $f(x) = x + 1$, onde $\real$ é o grupo aditivo dos reais.

        \item $f: \z \to \z \times \z$ dada por $f(x) = (x, 0)$, onde $\z$ e $\z \times \z$ denotam grupos aditivos.

        \item $f: \z \times \z \to \z$ dada por $f(x,y) = x$, onde $\z \times \z$ e $\z$ são grupos aditivos.

        \item $f: \z \to \real^*_+$ dada por $f(x) = 2^x$, onde $\z$ é grupo aditivo e $\real^*_+$ é grupo multiplicativo.
    \end{enumerate}

    \vspace{.3cm}

    \questao{} Das funções a seguir, algumas são homomorfismos do grupo multiplicativo $\complex^*$. \textbf{Descubra} quais e determine o núcleo de cada uma.
    \begin{multicols}{2}
        \begin{enumerate}[label=({\alph*})]
            \item $f(z) = z^2$

            \item $f(z) = |z|$, aqui $|z| = \sqrt{a^2 + b^2}$, se $z = a + bi$.

            \item $f(z) = \dfrac{1}{z}$

            \item $f(z) = -\dfrac{1}{z}$

            \item $f(z) = -z$

            \item $f(z) = z^3$

            \item $f(z) = \overline{z}$ onde $\overline{z} = a - bi$, se $z = a + bi$.
        \end{enumerate}
    \end{multicols}

    \vspace{.3cm}

    \questao{} Considere o conjunto
    \[
    T = \left\{\begin{bmatrix}x & y\\0 & 1/x\end{bmatrix} \mid x \in \real^*,\ y \in \real\right\}.
    \]
    \begin{enumerate}[label=({\alph*})]
        \item Mostre que $T$ é um grupo não comutativo com a operação de multiplicação de matrizes.

        \item A função $f : \real \to T$ definida por
        \[
        f(t) = \begin{bmatrix}1 & t\\0 & 1\end{bmatrix}
        \]
        é um homomorfismo de grupos? Caso afirmativo, esse homomorfismo é injetor? É sobrejetor?

        \item A função $g : \real \to T$ definida por
        \[
        g(t) = \begin{bmatrix}e^t & 0\\0 & e^{-t}\end{bmatrix}
        \]
        é um homomorfismo de grupos? Caso afirmativo, esse homomorfismo é injetor? É sobrejetor?
    \end{enumerate}

    \vspace{.3cm}

    \questao{} Considere o conjunto
    \[
    SO(2) = \left\{\begin{bmatrix}x & -y\\y & x\end{bmatrix} \mid x, y \in \real,\ x^2 + y^2 = 1\right\}.
    \]
    \begin{enumerate}[label=({\alph*})]
        \item Mostre que $SO(2)$ é um grupo abeliano com a operação de multiplicação de matrizes.

        \item A função $f : \real \to SO(2)$ definida por
        \[
        f(t) = \begin{bmatrix}cos(t) & -\sin(t)\\\sin(t) & \cos(t)\end{bmatrix}
        \]
        é um homomorfismo de grupos? Caso afirmativo, esse homomorfismo é injetor? É sobrejetor?
    \end{enumerate}

    \vspace{.3cm}

    \questao{} Considere os grupos $\real$ e $GL_2(\real)$ com as operações usuais.
    \begin{enumerate}[label=({\alph*})]
        \item A função $f : \real \to GL_2(\real)$ definida por
        \[
        f(x) = \begin{bmatrix}x & 0\\0 & 1\end{bmatrix}
        \]
        é um homomorfismo de grupos? Caso afirmativo, esse homomorfismo é injetor? É sobrejetor?

        \item A função $g : \real \to GL_2(\real)$ definida por
        \[
        g(x) = \begin{bmatrix}1 & x\\0 & 1\end{bmatrix}
        \]
        é um homomorfismo de grupos? Caso afirmativo, esse homomorfismo é injetor? É sobrejetor?
    \end{enumerate}

    \vspace{.3cm}

    \questao{} Considere o conjunto
    \[
    T = \left\{\begin{bmatrix}a & b\\0 & d\end{bmatrix} \mid a, b, d \in \real,\ ad \ne 0 \right\}.
    \]
    \begin{enumerate}[label=({\alph*})]
        \item Mostre que $T$ é um subgrupo de $GL_2(\real)$.

        \item Seja $\phi : T \to \real^*$ definida por
        \[
        \phi\left(\begin{bmatrix}a & b\\0 & d\end{bmatrix}\right) = a^2
        \]
        Mostre que $\phi$ é um homomorfismo de grupos. Esse homomorfismo é injetor? É sobrejetor?
    \end{enumerate}

    \vspace{.3cm}

    \questao{} Sejam $(G, \cdot)$ e $(J, \cdot)$ grupos. Estabeleça quais das seguintes funções são homomorfismos e determine seus núcleos.

    \begin{enumerate}[label=({\alph*})]
        \item $f_1 : G \times J \to G$ dada por $f_1(x,y) = x$.

        \item $f_2 : G \times J \to J$ dada por $f_2(x,y) = y$.

        \item $f_3 : G  \to G \times J$ dada por $f_3(x) = (x, 1_J)$.

        \item $f_4 : G \times J \to J \times G$ dada por $f_4(x,y) = (y, x)$.

        \item $f_5 : J \to G \times J$ dada por $f_5(y) = (1_G, y)$.
    \end{enumerate}

    \vspace{.3cm}

    \questao{} Determine o núcleo em cada homomorfismo do \textbf{Exercício \ref{nucleo_homomorfismo}}.

    \vspace{.3cm}

    \questao{} Determinar os homomorfismos injetores e os sobrejetores do \textbf{Exercício \ref{nucleo_homomorfismo}}.

    \vspace{.3cm}

    \questao{} Sejam $G$ e $J$ grupos multiplicativos, $f : G \to J$ um homomorfismo de grupos e $H$ um subgrupo de $J$. Mostre que $f^{-1}(H) = \{ x \in G \mid f(x) \in H\}$ é um subgrupo de $G$.

    \vspace{.3cm}

    \questao{} Prove que um grupo $G$ é abeliano se, e somente se, $f : G \to G$ definida por $f(x) = x^{-1}$ é um homomorfismo.

    \vspace{.3cm}

    \questao{} Seja $f: G\to H$ um homomorfismo de grupos e $K$ um subgrupo de $H$. Mostre que $\ker(f)\sub f^{-1}(K)$.

    \vspace{.3cm}

    \questao{} Se $\phi : G \to H$ é um homomorfismo de grupos e $G$ é abeliano, prove que $\phi(G)$ também é abeliano.

    \vspace{.3cm}

    \questao{} Seja $f: \z \times \z \to \z \times \z$ definida por $f(x, y) = (x - y, 0)$. Provar que $f$ é um homomorfismo do grupo aditivo $\z \times \z$ em si pr\'oprio. Obter $\ker(f)$.

    \vspace{.3cm}

    \questao{} Mostre que $f : \z \to 2\z$ dada por $f(n) = 2n$, para todo $n \in \z$, é um isomorfismo do grupo aditivo $\z$ no grupo aditivo $2\z = \{0, \pm 2, \pm 4, \dots\}$.

    \vspace{.3cm}

    \questao{} Seja $a \in \real_+^*$ e $a \ne 1$.
    \begin{enumerate}[label=({\alph*})]
        \item  Mostre que $G = \{a^n \mid n \in \z\}$ é um subgrupo de $(\real_+^*, \cdot)$.

        \item Mostre que $f : \z \to G$ tal que $f(n) = a^n$ é um isomorfismo de $(\z, +)$ em $(G, \cdot)$.
    \end{enumerate}

    \vspace{.3cm}

    \questao{} Mostre que $G = \{2^m3^n \mid m, n \in \z\}$ e $J = \{m + ni \mid m, n \in \z\}$ são subgrupos de $(\real_+^*, \cdot)$ e $(\complex, +)$, respectivamente, e que são isomorfos.

    \vspace{.3cm}

    \questao{} Seja $G = \real - \{-1\}$ e defina a operação binária em $G$ por
    \[
    a * b = a + b + ab.
    \]
    Prove que $G$ é um grupo com essa operação. Mostre também que $f : G \to \real^*$ dada por $f(x) = x + 1$ é um isomorfismo do grupo $G$ no grupo multiplicativo $\real^*$.


    \questao{} Considere o grupo $S_3$:
    \begin{enumerate}[label=({\alph*})]
        \item Determine todos os elementos $f \in S_3$ tais que $f^2 = Id$ e $f \ne Id$.

        \item Determine todos os elementos $g \in S_3$ tais que $g^3 = Id$ e $g \ne Id$.
    \end{enumerate}

    \vspace{.3cm}

    \questao{} Considere o grupo $S_4$:
    \begin{enumerate}[label=({\alph*})]
        \item Determine todos os elementos $f \in S_4$ tais que $f^2 = Id$ e $f \ne Id$.

        \item Determine todos os elementos $g \in S_4$ tais que $g^3 = Id$ e $g \ne Id$.

        \item Determine todos os elementos $g \in S_4$ tais que $g^4 = Id$ e $g \ne Id$.
    \end{enumerate}

    \vspace{.3cm}

    \questao{} Seja $V = \{1, f, g, h\}$ o seguinte subconjunto do grupo $S_4$:
    \begin{align*}
        1 = \begin{pmatrix}
            1 & 2 & 3 & 4\\
            1 & 2 & 3 & 4
        \end{pmatrix}; \quad f = \begin{pmatrix}
            1 & 2 & 3 & 4\\
            2 & 1 & 4 & 3
        \end{pmatrix}\\
        g = \begin{pmatrix}
            1 & 2 & 3 & 4\\
            3 & 4 & 1 & 2
        \end{pmatrix}; \quad h = \begin{pmatrix}
            1 & 2 & 3 & 4\\
            4 & 3 & 2 & 1
        \end{pmatrix}.
    \end{align*}
    \begin{enumerate}[label=({\alph*})]
        \item Prove que $(V, \comp)$ é um grupo contendo 4 elementos, onde $\comp$ é a operação de $S_4$.

        \item Prove que $(V, \comp)$ é um grupo abeliano.
    \end{enumerate}

    \vspace{.3cm}

    \questao{} Considere o grupo $S_7$ e sejam
    \begin{align*}
        1 &= \begin{pmatrix}
            1 & 2 & 3 & 4 & 5 & 6 & 7\\
            1 & 2 & 3 & 4 & 5 & 6 & 7
        \end{pmatrix}\\
        \sigma &= \begin{pmatrix}
            1 & 2 & 3 & 4 & 5 & 6 & 7\\
            3 & 4 & 2 & 6 & 7 & 5 & 1
        \end{pmatrix}.\\
        \beta &= \begin{pmatrix}
            1 & 2 & 3 & 4 & 5 & 6 & 7\\
            3 & 1 & 2 & 6 & 7 & 4 & 5
        \end{pmatrix}.
    \end{align*}.
    \begin{enumerate}[label=({\alph*})]
        \item Encontre o menor $l \ge 0$ tal que $\sigma^l = 1$.

        \item Encontre $\delta \in S_7$ tal que $\sigma\comp\delta = 1$.

        \item Encontre o menor $k \ge 0$ tal que $\beta^k = 1$.

        \item Encontre $\gamma \in S_7$ tal que $\gamma\comp\beta = 1$.
    \end{enumerate}

    \vspace{.3cm}

    \vspace{.3cm}
\end{document}
