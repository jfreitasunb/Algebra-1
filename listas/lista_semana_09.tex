%!TEX program = xelatex
%!TEX encoding = ISO-8859-1
\def\ano{2021}
\def\semestre{1}
\def\disciplina{\'Algebra 1}
\def\turma{C}
\def\numerosemana{09}

\documentclass[12pt]{exam}

\usepackage{caption}
\usepackage{amssymb}
\usepackage{amsmath,amsfonts,amsthm,amstext}
\usepackage[brazil]{babel}
% \usepackage[latin1]{inputenc}
\usepackage{graphicx}
\graphicspath{{/ArquivosLinux/OneDrive/imagens-latex/}{D:/OneDrive - unb.br/imagens-latex/}}
\usepackage{enumitem}
\usepackage{multicol}
\usepackage{answers}
\usepackage{tikz,ifthen}
\usetikzlibrary{lindenmayersystems}
\usetikzlibrary[shadings]
\Newassociation{solucao}{Solution}{ans}
\newtheorem{exercicio}{}

\setlength{\topmargin}{-1.0in}
\setlength{\oddsidemargin}{0in}
\setlength{\textheight}{10.1in}
\setlength{\textwidth}{6.5in}
\setlength{\baselineskip}{12mm}

\extraheadheight{0.7in}
\firstpageheadrule
\runningheadrule
\lhead{
        \begin{minipage}[c]{1.7cm}
        \includegraphics[width=1.7cm]{unb.pdf}
        \end{minipage}%
        \hspace{0pt}
        \begin{minipage}[c]{4in}
          {Universidade de Brasília} --
          {Departamento de Matemática}
\end{minipage}
\vspace*{-0.8cm}
}
% \chead{Universidade de Brasília - Departamento de Matemática}
% \rhead{}
% \vspace*{-2cm}

\extrafootheight{.5in}
\footrule
\lfoot{\disciplina\ - \semestre$^o$/\ano\ - Módulo \numeromodulo}
\cfoot{}
\rfoot{Página \thepage\ de \numpages}

\newcounter{exercicios}
\renewcommand{\theexercicios}{\arabic{exercicios}}

\newenvironment{questao}[1]{
\refstepcounter{exercicios}
\ifx&#1&
\else
   \label{#1}
\fi
\noindent\textbf{Exercício {\theexercicios}:}
}

\newcommand{\resp}[1]{
\noindent{\bf Exercício #1: }}

\def\ano{2024}
\def\semestre{1}
\def\disciplina{Álgebra 1}
\def\nomeabreviado{Álgebra 1}
\def\turma{1}

\newcommand{\im}{{\rm Im\,}}
\newcommand{\dlim}[2]{\displaystyle\lim_{#1\rightarrow #2}}
\newcommand{\minf}{+\infty}
\newcommand{\ninf}{-\infty}
\newcommand{\cp}[1]{\mathbb{#1}}
\newcommand{\sub}{\subseteq}
\newcommand{\n}{\mathbb{N}}
\newcommand{\z}{\mathbb{Z}}
\newcommand{\rac}{\mathbb{Q}}
\newcommand{\real}{\mathbb{R}}
\newcommand{\complex}{\mathbb{C}}

\newcommand{\vesp}[1]{\vspace{ #1  cm}}

\newcommand{\compcent}[1]{\vcenter{\hbox{$#1\circ$}}}
\newcommand{\comp}{\mathbin{\mathchoice
        {\compcent\scriptstyle}{\compcent\scriptstyle}
        {\compcent\scriptscriptstyle}{\compcent\scriptscriptstyle}}}
\renewcommand{\sin}{{\rm sen\,}}
\renewcommand{\tan}{{\rm tg\,}}
\renewcommand{\csc}{{\rm cossec\,}}
\renewcommand{\cot}{{\rm cotg\,}}
\renewcommand{\sinh}{{\rm senh\,}}

\begin{document}
    \begin{center}
    {\Large\bf \disciplina\ - Turma \turma\ -- \semestre$^{o}$/\ano} \\ \vspace{9pt} {\large\bf
        Lista de Exerc{\'\i}cios -- Semana \numerosemana}\\ \vspace{9pt} Prof. Jos{\'e} Ant{\^o}nio O. Freitas
    \end{center}
    \hrule

    \vspace{.6cm}

    \questao{}  Seja $(A, +, \cdot)$ um anel unit\'ario. Mostre que o inverso multiplicativo de um elemento $x \in A$, se existir, \'e \'unico.

    \vspace{.3cm}

    \questao{inicioreferencia}{} Verificar se a fun\c{c}\~ao $f : A \to B$ \'e ou n\~ao um homomorfismo do anel $A$ no anel $B$, nos seguintes casos:
    \begin{enumerate}[label=({\alph*})]
        \item $A = \z$, $B = \z$ e $f(x) = x + 1$

        \item $A = \z$, $B = \z$ e $f(x) = 2x$

        \item $A = \z$, $B = \z \times \z$ e $f(x) = (x, 0)$

        \item $A = \z \times \z$, $B = \z$ e $f(x,y) = x$

        \item $A = \z \times \z$, $B = \z \times \z$ e $f(x,y) = (y,x)$

        \item $A = \z$, $B = \z_n$ e $f(x) = \overline{x}$

        \item $A = \complex$, $B = \complex$ e $f(a + bi) = a - bi$

        \item $A = M_2(\real)$, $B = \real$ e $f\left(\begin{bmatrix}
            x & y\\z & t
        \end{bmatrix}\right) = x$

        \item $A = M_2(\real)$, $B = \real$ e $f\left(\begin{bmatrix}
            x & y\\z & t
        \end{bmatrix}\right) = x + t$
    \end{enumerate}

    \vspace{.3cm}

    \questao{} Seja $(\z\times\z, +, \cdot)$ um anel com as seguintes opera\c{c}\~oes
    \begin{align*}
        (a, b) + (c, d) &= (a + c, b + d)\\
        (a, b)\cdot (c, d) &= (ac, 0)
    \end{align*}
    para todos $(a, b)$, $(c, d) \in \z\times\z$.
    Mostre que $ f : \z\times\z \to \z$ definada por $f(a, b) = a$ \'e um homomorfismo sobrejetor.

    \vspace{.3cm}

    \questao{} Seja $(\z\times\z, +, \cdot)$ um anel com as seguintes opera\c{c}\~oes
    \begin{align*}
        (a, b) + (c, d) &= (a + c, b + d)\\
        (a, b)\cdot (c, d) &= (ac, ad + bc)
    \end{align*}
    para todos $(a, b)$, $(c, d) \in \z\times\z$.
    Mostre que $ f : \z\times\z \to \z$ definada por $f(a, b) = a$ \'e um homomorfismo.

    \vspace{.3cm}

    \questao{} Seja $(\z\times\z, +, \cdot)$ um anel com as seguintes opera\c{c}\~oes
    \begin{align*}
        (a, b) + (c, d) &= (a + c, b + d)\\
        (a, b)\cdot (c, d) &= (ac - bd, ad + bc)
    \end{align*}
    para todos $(a, b)$, $(c, d) \in \z\times\z$.
    Mostre que $ f : \z \to \z\times\z$ definada por $f(a) = (a, 0)$ \'e um homomorfismo.

    \vspace{.3cm}

    \questao{} Prove que $f : \rac \to M_3(\rac)$ dada por
    \[
        f(x) = \begin{pmatrix}
            x & 0 & 0\\
            0 & x & 0\\
            0 & 0 & x
        \end{pmatrix}
    \]
    \'e um homomorfismo de an\'eis.

    \vspace{.3cm}


    \questao{fimreferencia} Verifique se as seguintes fun\c{c}\~oes s{\~a}o homomorfismos de an\'eis:
    \begin{enumerate}[label=({\alph*})]
        \item $f : \z\times\z \to \z\times\z$ dado por $f(x,y) = (0,y)$

        \item $f : \z\times\z \to \z$ dado por $f(x,y) = y$

        \item $f : \z\to \z\times\z$ dado por $f(x) = (2x,0)$

        \item $f : \z\to \z_{12}\times\z_{12}$ dado por $f(x) = (\overline{2}\overline{x},\overline{0})$

        \item $f : \z\times\z \to \z\times\z$ dado por $f(x,y) = (-y,-x)$

        \item $f : \z_6\times\z_6 \to \z_6\times\z_6$ dado por $f(x,y) = (\overline{5}\overline{y},\overline{5}\overline{x})$

        \item $f : \z \to \z\times\z$ dado por $f(x) = (0,x)$

        \item $f : \z \to \z_3\times\z_3$ dado por $f(x) = (\overline{0},\overline{x})$
    \end{enumerate}

    \vspace{.3cm}

    \questao{} Determine o kernel dos homomorfismos dos \textbf{Exerc{\'i}cios de \ref{inicioreferencia} a \ref{fimreferencia}}.

    \vspace{.3cm}

    \questao{} Nos \textbf{Exerc{\'\i}cios de \ref{inicioreferencia} a \ref{fimreferencia}} para as fun\c{c}\~oes que forem homomorfismos determine se elas tamb\'em s\~ao isomorfismos.

    \vspace{.3cm}

    \questao{} Seja $f : \complex \to M_2(\real)$ dada por
    \[
        f(a + bi) = \begin{bmatrix}
            a & -b\\
            b & a
        \end{bmatrix}.
    \]
    \begin{enumerate}[label=({\alph*})]
        \item Mostre que $f$ \'e um homomorfismo de an\'eis.

        \item Esse homomorfismo \'e injetor?

        \item \'E sobrejetor?
    \end{enumerate}

    \vspace{.3cm}

    \questao{} Seja $f : \z \to M_2(\z_3)$ dada por
    \[
        f(a) = \begin{bmatrix}
            \overline{a} & \overline{0}\\
            \overline{0} & \overline{a}
        \end{bmatrix}.
    \]
    \begin{enumerate}[label=({\alph*})]
        \item Mostre que $f$ \'e um homomorfismo de an\'eis.

        \item Esse homomorfismo \'e injetor?

        \item \'E sobrejetor?
    \end{enumerate}

    \vspace{.3cm}

    \questao{} Seja $f: A \to B$ um homomorfismo de an{\'e}is. Mostre que:
    \begin{enumerate}[label=({\alph*})]
        \item Se $C$  {\'e} um subanel de $A$, ent{\~a}o $f(C)$ {\'e} um subanel de $B$.

        \item Se $D$ {\'e} um subanel de $B$, ent{\~a}o $f^{-1}(D)$ {\'e} um subanel de $A$.

        \item Se $I$ {\'e} um ideal de $A$ e $f$ \'e sobrejetora, ent{\~a}o $f(I)$ {\'e} um ideal de $B$.

        \item Se $J$ {\'e} um ideal de $B$, ent{\~a}o $f^{-1}(J)$ {\'e} um ideal de $A$.
    \end{enumerate}

    \vspace{.3cm}

    \questao{} D{\^e} um exemplo de an{\'e}is $A$ e $B$ e um homomorfismo $f : A \to B$ tal que $f(1_A) \ne 1_B$.

    \vspace{.3cm}

    \questao{} Sejam os an{\'e}is $A = \{ a + b\sqrt{-2} \mid a,\ b \in \z\}$ e $B = M_2(\z_7)$.
    \begin{enumerate}[label=({\alph*})]
        \item Mostre que $f : A \to B$ dada por
        \[
        f(a + b\sqrt{-2}) =
        \begin{pmatrix}
        \overline{a} & \overline{5}\overline{b}\\
        \overline{b} & \overline{a}
        \end{pmatrix}
        \]
        {\'e} um homomorfismo.

        \item $f$ {\'e} um isomorfimo?
    \end{enumerate}

    \vspace{.3cm}

    \questao{} {\'E} verdadeiro ou falso: $\z$ e $\z_{m}$ para $m > 1$ s{\~a}o an{\'e}is
    isomorfos.

    \vspace{.3cm}

    \questao{} Considere os seguintes an{\'e}is: $(\real, +, \cdot)$ e $(\real, \oplus, \odot)$, sendo $a \oplus b = a + b + 1$ e $a \odot b = a + b + ab$. Mostre que $f : \real \to \real$ dado por $f(x) = x + 1$, para todo $x \in \real$, {\'e} um isomorfimos de $(\real, \oplus, \odot)$ em $(\real, +, \cdot)$.

    \vspace{.3cm}

    \questao{} Seja $A$ um anel de integridade. Mostre que se $x \in A$ \'e tal que $x^2 = 1$, ent\~ao $x = 1$ ou $x = -1$.

    \vspace{.3cm}

    \questao{} Seja $A$ \'e um anel de integridade. Mostre que se $x \in A$ \'e tal que $x ^2 = x$, ent\~ao $x = 0$ ou $x = 1$.

    \vspace{.3cm}

    \questao{} Seja $A$ um anel com unidade tal que $x^2 = x$ para todo $x \in A$. Mostre que $A$ \'e um anel de integridade se, e somente se, $A = \{0, 1\}$.

    \newpage

    \questao{} Verifique se s\~ao ideiais:
    \begin{enumerate}[label=({\alph*})]
        \item  $I = \{\overline{0}, \overline{2}, \overline{4}\}$ no anel $\z_6$;

        \item $I = m\z \times n\z$ no anel $\z \times \z$, em que $m$, $n \in \z$;

        \item $I = \{x \in \z \mid 25 \mbox{ divide } 35x\}$ no anel $\z$;

        \item $I = \{x \in \z \mid x \mbox{ divide } 24\}$ no anel $\z$;

        \item $I = \{x \in \z \mid 6 \mbox{ divide } x \mbox{ e } 24 \mbox{ divide } x^2\}$ no anel $\z$;

        \item $I = \z$ no anel $(\rac, \oplus, \odot)$ em que a $a \oplus b = a + b - 1$ e $a \odot b = a + b - ab$, para todos $a$, $b \in \rac$;

        \item $I = 2\z$ no anel $(\z, +, \cdot)$ em que a adi\c{c}\~ao \'e a usual e $a \cdot b = 0$, para quaisquer $a$, $b \in \z$.
    \end{enumerate}

    \vspace{.3cm}

    \questao{} Seja $(A, +, \cdot)$ um anel comutativo.
    \begin{enumerate}[label=({\alph*})]
        \item Mostre que a interse\c{c}\~ao de quaisquer dois ideais de $A$ \'e sempre um conjunto n\~ao vazio.

        \item Mostre que essa interse\c{c}\~ao \'e sempre um ideal.

        \item A uni\~ao de ideias \'e ainda um ideal?

        \item Sejam $J_1$, $J_2 \subset A$ ideiais tais que $J_1 \subset J_2$. Mostre que $J_1 \cup J_2$ \'e um ideal de $A$.

        \item Sejam $I$ e $J$ ideais de $A$. Mostre que
        \[
            I + J = \{x + y \mid x \in I, y \in J\}
        \]
        \'e um ideal de $A$.

    \item Se $I$ é um ideal $A$, mostre que $r(I) = \{x \in A \mid xy = 0 \mbox{ para todo } y \in I\}$, então $r(I)$ é um ideal de $A$.
    \end{enumerate}

    \vspace{.3cm}

    \questao{} Considere o conjunto
    \[
        M^\sigma(\z) = \{a_1I_2 + a_2\sigma \mid a_1, a_2 \in \z\}
    \]
    onde
    \[
        I_2 = \begin{pmatrix} 1 & 0\\ 0 & 1\end{pmatrix}, \quad \sigma = \begin{pmatrix} 0 & 1\\ 1 & 0\end{pmatrix}.
    \]
    Este conjunto é um subanel de $M_2(\z)$? Caso afirmativo responda aos itens abaixo:
    \begin{enumerate}[label=({\alph*})]
        \item Se $f : M^\sigma(\z) \to \z$ é dada por $f(a_1I_2 + a_2\sigma) = a_1 + a_2$, então $f$ é um homomorfismo de anéis? Caso afirmativo, determine $\ker(f)$.

        \item Se $g : M^\sigma(\z) \to \z$ é dada por $g(a_1I_2 + a_2\sigma) = a_1 - a_2$, então $g$ é um homomorfismo de anéis? Caso afirmativo, determine $\ker(g)$.
    \end{enumerate}
    \vspace{.3cm}

    \questao{} Sendo $A$ um anel, n\~ao necessariamente comutativo, dizemos que $I \subset A$, $I \ne \emptyset$ \'e um \textbf{ideal \`a esquerda} em $A$ se, e somente se:
    \begin{enumerate}[label=({\roman*})]
        \item Para todos $x$, $y \in I$ temos $x - y \in I$;

        \item Para todo $\alpha \in A$ e todo $x \in I$ temos $\alpha x \in I$.
    \end{enumerate}

    Verifique se s\~ao ideiais \`a esquerda em $M_2(\real)$:
    \begin{enumerate}[label=({\alph*})]
        \item $L_1 = \left\{\begin{pmatrix}
            a & 0\\
            0 & b
        \end{pmatrix} \mid a, b \in \real\right\}$.

        \item $L_2 = \left\{\begin{pmatrix}
            a & b\\
            0 & c
        \end{pmatrix} \mid a, b, c \in \real\right\}$.

        \item $L_3 = \left\{\begin{pmatrix}
            a & 0\\
            b & 0
        \end{pmatrix} \mid a, b \in \real\right\}$.

        \item $L_4 = \left\{\begin{pmatrix}
            a & b\\
            0 & 0
        \end{pmatrix} \mid a, b \in \real\right\}$.
    \end{enumerate}

    \vspace{.3cm}

    \questao{} Seja $(A, +, \cdot)$ um anel comutativo e com unidade. Mostre que se $I$ {\'e} um ideal de $A$, ent\~ao
    \[
        \left(\dfrac{A}{I}, \oplus, \otimes\right)
    \]

    {\'e} um anel comutativo  e com unidade.

    \vspace{.3cm}

    \questao{} Suponha que $A$ {\'e} um anel com unidade e $I$ um ideal de $A$. Mostre que $a + I \in A/I$ {\'e} invers{\'\i}vel se, e somente se, existe $r \in A$ de modo que $ar - 1 \in I$.

    \vspace{.3cm}

    \questao{} D{\^e} um exemplo de um anel de integridade $A$ e de um ideal $I$ em $A$ tal que $A/I$ n{\~a}o {\'e} de integridade.
\end{document}
