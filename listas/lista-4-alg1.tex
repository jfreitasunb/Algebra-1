%!TEX program = xelatex
%!TEX encoding = ISO-8859-1
\documentclass[12pt]{article}

\usepackage{amssymb}
\usepackage{amsmath,amsfonts,amsthm,amstext,mathabx}
\usepackage[brazil]{babel}
%\usepackage[latin1]{inputenc}
\usepackage{graphicx}
\graphicspath{{/home/jfreitas/Dropbox/imagens-latex/}{/Volumes/Vader/Dropbox/imagens-latex/}{D:/Dropbox/imagens-latex/}}
\usepackage{enumitem}
\usepackage{multicol}
\usepackage[all]{xy}

\setlength{\topmargin}{-1.0in}
\setlength{\oddsidemargin}{0in}
\setlength{\textheight}{10.1in}
\setlength{\textwidth}{6.5in}
\setlength{\baselineskip}{12mm}

\newcounter{exercicios}
\setcounter{exercicios}{0}
\newcommand{\questao}{
\addtocounter{exercicios}{1}
\noindent{\bf Exerc{\'\i}cio \arabic{exercicios}: }}

\newcommand{\equi}{\Leftrightarrow}
\newcommand{\bic}{\leftrightarrow}
\newcommand{\cond}{\rightarrow}
\newcommand{\impl}{\Rightarrow}
\newcommand{\nao}{\sim}
\newcommand{\sub}{\subseteq}
\newcommand{\e}{\ \wedge\ }
\newcommand{\ou}{\ \vee\ }
\newcommand{\vaz}{\emptyset}
\newcommand{\nsub}{\nsubset}
\renewcommand{\sin}{{\rm sen\,}}

\newcommand{\n}{\mathbb{N}}
\newcommand{\z}{\mathbb{Z}}
\newcommand{\real}{\mathbb{R}}
\newcommand{\vesp}{\vspace{0.2cm}}
\newcommand{\subne}{\subsetneqq}


\newcommand{\compcent}[1]{\vcenter{\hbox{$#1\circ$}}}
\newcommand{\comp}{\mathbin{\mathchoice
{\compcent\scriptstyle}{\compcent\scriptstyle}
{\compcent\scriptscriptstyle}{\compcent\scriptscriptstyle}}}

\begin{document}

\pagestyle{empty}

\begin{figure}[h]
        \begin{minipage}[c]{1.7cm}
        \includegraphics[width=1.7cm]{unb.pdf}
        \end{minipage}%
        \hspace{0pt}
        \begin{minipage}[c]{4in}
          {Universidade de Brasília} \\
          {Departamento de Matemática}
\end{minipage}
\end{figure}
\vspace{-1cm}\hrule


\begin{center}
 {\Large\bf {\'A}lgebra 1 - Turma C -- 1$^{o}$/2018} \\
 \vspace{9pt} {\large\bf $4^{\underline{a}}$ Lista de Exerc{\'\i}cios -- An\'eis}\\
 \vspace{9pt} Prof. Jos{\'e} Ant{\^o}nio O. Freitas
\end{center}
\hrule


\begin{center}
\Large{\bf An{\'e}is}
\end{center}

\vspace{.6cm}

\questao Consideremos em $\z \times \z$ as opera\c{c}\~oes $\oplus$ e $\otimes$ definidas por
\begin{align*}
	(a, b) \oplus (c, d) = (a + c, b + d)\\
	(a ,b) \otimes (c, d) = (ac - bd, ad + bc).
\end{align*}
Mostre que $(\z \times \z, \oplus, \otimes)$ \'e um anel comutativo e com unidade.

\vesp

\questao Considere as opera\c{c}\~oes $\star$ e $\odot$ em $\rac$ definidas por
\begin{align*}
	x \star y = x + y - 3\\
	x \odot y = x + y - \dfrac{xy}{3}.
\end{align*}
Mostre que $(\rac, \star, \odot)$ \'e um anel comutativo e com unidade.

\vesp

\questao Prove que s\~ao an\'eis:
\begin{enumerate}[label={\alph*})]
	\item O conjunto $\z$ com a adi\c{c}\~ao usual e o produto $x \otimes y = 0$, para todo $x$, $y \in \z$.
	\item O conjunto $\rac$ com as opera\c{c}\~oes $x \oplus y = x + y - 1$ e $x \odot y = x + y - xy$.
	\item O conjunto $\z \times \z$ com as opera\c{c}\~oes:
	\begin{align*}
		(a, b) \oplus (c, d) = (a + c, b + d)\\
		(a ,b) \otimes (c, d) = (ac, ad + bc).
	\end{align*}
\end{enumerate}
Quais destes an\'eis s\~ao comutativos? Quais t\^em unidade?

\vesp

\questao Determinar quais dos seguintes conjuntos são an{\'e}is com as operações usuais em $\rac$:
	\begin{multicols}{2}
		\begin{enumerate}[label=({\alph*})]
			\item $\z$
			\item $B = \{x \in \rac \mid x \notin \z\}$
			\item $C = \left\{\dfrac{a}{b} \in \rac \mid a \in \z,\ b \in \z,\ 2 |b \right\}$
			\item $D = \left\{\dfrac{a}{2^n} \in \rac \mid a \in \z \mbox{ e } n \in \z \right\}$
		\end{enumerate}
	\end{multicols}

\vesp

\questao Quais dos conjuntos abaixo s\~ao an\'eis de com as operações usuais de $M_2(\real)$?
\begin{align*}
	L_1 &= \left\{\begin{pmatrix}
		a & 0\\
		b & 0
	\end{pmatrix} \mid a, b \in \real\right\}\\
	L_2 &= \left\{\begin{pmatrix}
		a & b\\
		0 & c
	\end{pmatrix} \mid a, b, c \in \real\right\}\\
	L_3 &= \left\{\begin{pmatrix}
		a & 0\\
		0 & b
	\end{pmatrix} \mid a, b \in \real\right\}\\
	L_4 &= \left\{\begin{pmatrix}
		0 & a\\
		c & b
	\end{pmatrix} \mid a, b, c \in \real\right\}
\end{align*}

\vesp

\questao Considere o conjunto $L = \{ a + b\sqrt{2} \mid a, b \in \rac\} \sub \real$. Para $a + b\sqrt{2}$, $c + d\sqrt{2} \in L$ definimos:
\begin{align*}
	a + b\sqrt{2} &= c + d\sqrt{2} \quad a = c \mbox{ e } b = d\\
	(a + b\sqrt{2}) \oplus (c + d\sqrt{2}) = (a + c) + (b + d)\sqrt{2}\\
	(a + b\sqrt{2}) \otimes (c + d\sqrt{2}) = (ac + 2bd) + (ad + bc\sqrt{2}).
\end{align*}

Mostre que $(L, \oplus, \otimes)$ é um anel. Esse anel é comutativo? Possui unidade?

\end{document}