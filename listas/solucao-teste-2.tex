%!TEX program = xelatex
%!TEX encoding = ISO-8859-1
\documentclass[12pt]{article}

\usepackage{amssymb}
\usepackage{amsmath,amsfonts,amsthm,amstext,mathabx}
\usepackage[brazil]{babel}
%\usepackage[latin1]{inputenc}
\usepackage{graphicx}
\graphicspath{{/home/jfreitas/Dropbox/imagens-latex/}{/Users/jfreitas/Dropbox/imagens-latex/}{D:/Dropbox/imagens-latex/}}
\usepackage{enumitem}
\usepackage{multicol}
\usepackage[all]{xy}

\setlength{\topmargin}{-1.0in}
\setlength{\oddsidemargin}{0in}
\setlength{\textheight}{10.1in}
\setlength{\textwidth}{6.5in}
\setlength{\baselineskip}{12mm}

\newcounter{exercicios}
\setcounter{exercicios}{0}
\newcommand{\questao}{
\addtocounter{exercicios}{1}
\noindent{\bf Exerc{\'\i}cio \arabic{exercicios}: }}

\newcommand{\equi}{\Leftrightarrow}
\newcommand{\bic}{\leftrightarrow}
\newcommand{\cond}{\rightarrow}
\newcommand{\impl}{\Rightarrow}
\newcommand{\nao}{\sim}
\newcommand{\sub}{\subseteq}
\newcommand{\e}{\ \wedge\ }
\newcommand{\ou}{\ \vee\ }
\newcommand{\vaz}{\emptyset}
\newcommand{\nsub}{\nsubset}
\renewcommand{\sin}{{\rm sen\,}}

\newcommand{\n}{\mathbb{N}}
\newcommand{\z}{\mathbb{Z}}
\newcommand{\q}{\mathbb{Q}}
\newcommand{\cmp}{\mathbb{C}}
\newcommand{\real}{\mathbb{R}}
\newcommand{\vesp}{\vspace{0.2cm}}
\newcommand{\subne}{\subsetneqq}


\newcommand{\compcent}[1]{\vcenter{\hbox{$#1\circ$}}}
\newcommand{\comp}{\mathbin{\mathchoice
{\compcent\scriptstyle}{\compcent\scriptstyle}
{\compcent\scriptscriptstyle}{\compcent\scriptscriptstyle}}}

\begin{document}
\pagestyle{empty}

\begin{figure}[h]
        \begin{minipage}[c]{1.7cm}
        \includegraphics[width=1.7cm]{unb.pdf}
        \end{minipage}%
        \hspace{0pt}
        \begin{minipage}[c]{4in}
          {Universidade de Bras{\'\i}lia} \\
          {Departamento de Matem{\'a}tica}
\end{minipage}
\end{figure}
\vspace{-1cm}\hrule

\begin{center}
{\Large\bf {\'A}lgebra 1 - Turma B -- 2$^{o}$/2015} \\ \vspace{9pt} {\large\bf
  $2^{\underline{o}}$ Teste - Resolução}\\
\vspace{9pt} Prof. Jos{\'e} Ant{\^o}nio O. Freitas
\end{center}
\hrule

\vspace{.6cm}

\questao Mostre que $9^n - 1$ {\'e} m\'ultiplo de 8 para todo $n \ge 0$.

\noindent\textbf{Solucao:}

\vesp

\questao Seja $A=\z\times \z^*$, onde
$\mathbb{Z}^*=\mathbb{Z}\setminus \{0\}$. Para $(a,b), (c,d) \in
A$, considere a seguinte rela{\c c}{\~a}o
\[
(a,b)R (c,d) \Leftrightarrow ad=bc.
\]
\begin{enumerate}[label={\alph*})]
\item Mostre que $R$ {\'e} uma rela{\c c}{\~a}o de equival{\^e}ncia sobre $A$.
\item Descreva a classe de equival{\^e}ncia $\overline{(0,1)}$, $\overline{(1,1)}$, $\overline{(1,2)}$, $\overline{(2,1)}$, $\overline{(2,2)}$, $\overline{(2,3)}$.
\end{enumerate}


\end{document}