%!TEX program = xelatex
%!TEX encoding = UTF-8
\def\numeromodulo{1}
\def\numerolista{2}

\documentclass[12pt]{exam}

\def\ano{2022}
\def\semestre{1}
\def\disciplina{\'Algebra 1}
\def\turma{2}

\usepackage{caption}
\usepackage{amssymb}
\usepackage{amsmath,amsfonts,amsthm,amstext}
\usepackage[brazil]{babel}
% \usepackage[latin1]{inputenc}
\usepackage{graphicx}
\graphicspath{{/ArquivosLinux/OneDrive/imagens-latex/}{D:/OneDrive - unb.br/imagens-latex/}}
\usepackage{enumitem}
\usepackage{multicol}
\usepackage{answers}
\usepackage{tikz,ifthen}
\usetikzlibrary{lindenmayersystems}
\usetikzlibrary[shadings]
\Newassociation{solucao}{Solution}{ans}
\newtheorem{exercicio}{}

\setlength{\topmargin}{-1.0in}
\setlength{\oddsidemargin}{0in}
\setlength{\textheight}{10.1in}
\setlength{\textwidth}{6.5in}
\setlength{\baselineskip}{12mm}

\extraheadheight{0.7in}
\firstpageheadrule
\runningheadrule
\lhead{
        \begin{minipage}[c]{1.7cm}
        \includegraphics[width=1.7cm]{unb.pdf}
        \end{minipage}%
        \hspace{0pt}
        \begin{minipage}[c]{4in}
          {Universidade de Brasília} --
          {Departamento de Matemática}
\end{minipage}
\vspace*{-0.8cm}
}
% \chead{Universidade de Brasília - Departamento de Matemática}
% \rhead{}
% \vspace*{-2cm}

\extrafootheight{.5in}
\footrule
\lfoot{\disciplina\ - \semestre$^o$/\ano\ - Módulo \numeromodulo}
\cfoot{}
\rfoot{Página \thepage\ de \numpages}

\newcounter{exercicios}
\renewcommand{\theexercicios}{\arabic{exercicios}}

\newenvironment{questao}[1]{
\refstepcounter{exercicios}
\ifx&#1&
\else
   \label{#1}
\fi
\noindent\textbf{Exercício {\theexercicios}:}
}

\newcommand{\resp}[1]{
\noindent{\bf Exercício #1: }}

\def\ano{2024}
\def\semestre{1}
\def\disciplina{Álgebra 1}
\def\nomeabreviado{Álgebra 1}
\def\turma{1}

\newcommand{\im}{{\rm Im\,}}
\newcommand{\dlim}[2]{\displaystyle\lim_{#1\rightarrow #2}}
\newcommand{\minf}{+\infty}
\newcommand{\ninf}{-\infty}
\newcommand{\cp}[1]{\mathbb{#1}}
\newcommand{\sub}{\subseteq}
\newcommand{\n}{\mathbb{N}}
\newcommand{\z}{\mathbb{Z}}
\newcommand{\rac}{\mathbb{Q}}
\newcommand{\real}{\mathbb{R}}
\newcommand{\complex}{\mathbb{C}}

\newcommand{\vesp}[1]{\vspace{ #1  cm}}

\newcommand{\compcent}[1]{\vcenter{\hbox{$#1\circ$}}}
\newcommand{\comp}{\mathbin{\mathchoice
        {\compcent\scriptstyle}{\compcent\scriptstyle}
        {\compcent\scriptscriptstyle}{\compcent\scriptscriptstyle}}}
\renewcommand{\sin}{{\rm sen\,}}
\renewcommand{\tan}{{\rm tg\,}}
\renewcommand{\csc}{{\rm cossec\,}}
\renewcommand{\cot}{{\rm cotg\,}}
\renewcommand{\sinh}{{\rm senh\,}}

\begin{document}
    \begin{center}
    {\Large\bf \disciplina\ - Turma \turma\ -- \semestre$^{o}$/\ano} \\ \vspace{9pt} {\large\bf
        $\numerolista^a$ Lista de Exercícios -- Módulo \numeromodulo\\ Relações}\\ \vspace{9pt} Prof. José Antônio O. Freitas
    \end{center}
    \hrule

    \vspace{.6cm}
        \vspace{.3cm}

    \questao{} Quais das relações abaixo são relações de equivalência sobre $E = \{a,b,c\}$?
    \begin{enumerate}[label={\alph*})]
        \item $R_1 = \{(a,a);(a,b);(b,a);(b,b);(c,c)\}$

        \item $R_2 = \{(a,a);(a,b);(b,a);(b,b);(b,c)\}$

        \item $R_3 = \{(a,a);(b,b);(b,c);(c,b);(a,c);(c,a);(c,c)\}$

        \item $R_3 = \{(a,a);(b,b);(b,c);(c,b);(a,c);(c,a)\}$

        \item $R_4 = E \times E$

        \item $R_5 = \vaz$
    \end{enumerate}

    \vspace{.3cm}

    \questao{} Determinar todas as relações de equivalência
    $R$ sobre $A$ e os respectivos conjuntos quocientes $A/R$ para:
    \begin{enumerate}[label={\alph*})]
        \item $A=\{a\}$;

        \item $A=\{a,b\}$;

        \item $A=\{a,b,c\}$;

        \item $A=\{a,b,c,d\}$.
    \end{enumerate}

    \vspace{.3cm}

    \questao{} Seja $m \in \z$, $m > 1$. Defina $R \sub \z\times\z$ como
    \[
    R = \{(x,y) \in \z\times\z \mid x - y = km, \mbox{ para algum } k \in \z\}.
    \]
    Mostre que $R$ é uma relação de equivalência sobre $\z$.

    \vspace{.3cm}

    \questao{} Para $x$, $y \in \real$ defina a relação $R$ como
    \[
    R = \{(x,y) \in \real \times \real \mid x - y = 2n\pi, \mbox{ para algum } n \in \z\}.
    \]
    Mostre que $R$ é uma relação de equivalência sobre $\real$.

    \vspace{.3cm}

    \questao{} Quais das seguintes sentenças definem uma relação de equivalência no conjunto $A$ dado?
    \begin{enumerate}[label={\alph*})]
        \item $aRb$ se, e s{\'o} se, existe $k \in \n$ tal que $a - b = 3k$, $A = \n$.

        \item $aRb$ quando existe $k \in \n$ tal que $b = k a$, $A = \n$.

        \item $aRb$ quando $a \le b$, $A = \z$.

        \item $xRy$ quando $xy > 0$, $ A = \real$.

        \item $xRy$ quando $xy \ge 0$, $ A = \real$.

        \item $x \sim y$ quando $x + y$ é par, onde $x$, $y \in \z$.

        \item $x \sim y$ quando  $x + y = km$, para algum $k \in \z$, $A = \z$.

        \item $x \sim y$ quando $x + y$ é ímpar, onde $x$, $y \in \z$.

        \item $x \sim y$ quando $\dfrac{x}{y} \in \rac$, onde $x$, $y \in \real^* = \real - \{0\}$.

        \item $x \sim y$ quando $\dfrac{x}{y} \in \z$, onde $x$, $y \in \real^* = \real - \{0\}$.
    \end{enumerate}

    \vspace{.3cm}

    \questao{} Seja $\mathcal{A} = \{1,2,3,4\}$ e suponha que $R$ é uma relação de equivalência em $\mathcal{A}$. Sabendo que $1R2$ e $2R3$ mostre que existem exatamente duas possibilidades para a relação $R$, e descreva as duas.

    \vspace{.3cm}

    \questao{} Sejam $A$ e $B$ conjuntos. Definia a relação $\mathcal{R}$ por
    \[
    A \mathcal{R} B \mbox{ quando } A \subseteq B.
    \]
    Então $\mathcal{R}$ é uma relação de equivalência?

    \vspace{.3cm}

    \questao{} Seja $A = \n \times \n^*$. Considere a seguinte
    relação sobre $A$:
    \[
    (a,b) R (c,d) \mbox{ quando } a + b = c + d.
    \]
    Mostre que $R$ é uma relação de equivalência sobre $A$.

    \vspace{.3cm}

    \questao{} Seja $A = \real^2$ e considere o conjunto definido por
    \[
    (a,b)R(c,d) \mbox{ quando } 2a - b = 2c - d.
    \]
    Mostre que $R$ é uma relação de equivalência sobre $\real^2$.

    \vspace{.3cm}

    \questao{} Seja $A = \rac^2$ e considere o conjunto definido por
    \[
    (a,b)S(c,d) \mbox{ quando } 3a - b = 3c - d.
    \]
    Então $S$ é uma relação de equivalência sobre $\rac^2$?

    \vspace{.3cm}

    \questao{} Seja $A = \real^3$. Dados $(x, y, z)$, $(\alpha, \beta, \gamma) \in \real^3$, defina $(x, y, z) R (\alpha, \beta, \gamma)$
    quando $y = \beta$. Mostre que $R$ é uma relação de equivalência sobre $\real^3.$

    \vspace{.3cm}

    \questao{} Seja $A = \real^3$. Dados $u = (x_1, y_1, z_1)$, $v = (x_2, y_2, z_2) \in \real^3$ defina
    \[
    u\cdot v = x_1x_2 + y_1y_2 + z_1z_2.
    \]
    Tome um elemento fixo $w = (\alpha, \beta, \gamma) \in \real^3$ e defina
    \[
    u \sim v \mbox{ quando } u \cdot w = v \cdot w.
    \]

    \begin{enumerate}[label={\alph*})]
        \item Mostre que $\sim$ é uma relação de equivalência sobre $\real^3$.

        \item Encontre a classe de equivalência dos elementos $(1, 0, 0)$, $(0, 1, 0)$, $(0, 0, 1)$ e $(1, 1, 1)$.
    \end{enumerate}
    \vspace{.3cm}

    \questao{} Em $\z$ defina a relação $R$ por
    \[
    a R b \mbox{ se, e somente se, } |a + b| \mbox{ é par}
    \]
    para $a$, $b \in \z$.
    \begin{enumerate}[label={\alph*})]
        \item Mostre que $R$ é uma relação de equivalência em $\z$.

        \item Determine todas as classes de equivalência dessa relação.

        \item A qual classe de equivalência sua matrícula pertence.
    \end{enumerate}

    \vspace{.3cm}

    \questao{} Para pontos $(a, b)$, $(c, d) \in \real^2$ defina $(a, b) S (c, d)$ quando $a^2 + b^2 = c^2 + d^2$.
    \begin{enumerate}[label={\alph*})]
        \item Prove que $S$ é uma relação de equivalência em $\real^2$.

        \item Liste todos os elementos no conjunto $\{(x, y) \in \real \mid (x, y) S (0, 0)\}$.

        \item Liste cinco elementos distintos no conjunto $\{(x, y) \in \real \mid (x, y) S (1, 0)\}$.
    \end{enumerate}

    \vspace{.3cm}

    \questao{} Sejam $E = \{-5, -4, -3, -2, -1, 0, 1, 2, 3, 5, 5\}$ e $R = \{(x, y) \in E \times E : x + |x| = y + |y|\}$.
    \begin{enumerate}[label={\alph*})]
        \item Mostre que $R$ é uma relação de equivalência.

        \item Descreva o conjunto quociente $E/R$.
    \end{enumerate}
    \vspace{.3cm}

    \vspace{.3cm}

    \questao{} Considere a seguinte relação sobre $\mathbb{C}$:
    \[
    (x+yi) R (r+si) \mbox{ quando } x^2+y^2=r^2+s^2.
    \]
    \begin{enumerate}[label={\alph*})]
        \item Mostre que $R$ é relação de equivalência.

        \item Descreva a classe de equivalência de $2 - 3i$.
    \end{enumerate}

    \vspace{.3cm}

    \questao{} Considere a seguinte relação sobre $\z$:
    \[
    x \sim y \mbox{ quando }  x^2 - y^2 = 4k, \mbox{ para algum } k \in \z.
    \]

    \begin{enumerate}[label={\alph*})]
        \item Mostre que $\sim$ é relação de equivalência.

        \item Determine as classes de equivalência dessa relação.

        \item A qual classe de equivalência sua matrícula pertence.
    \end{enumerate}

    \vspace{.3cm}

    \questao{} Considera a relação em $\real$ dada por
    \[
    a \sim b \mbox{ quando } a^2 + a = b^2 + b.
    \]
    \begin{enumerate}[label={\alph*})]
        \item Mostre que $\sim$ é uma relação de equivalência em $\real$.

        \item Determine as classes de equivalência dessa relação.
    \end{enumerate}

    \vspace{.3cm}

    \questao{} Seja $U$ um conjunto não vazio e considere $\mathcal{P}(U)$ o conjunto das partes de $U$. Fixado um conjunto $T$ defina a relação $\sim$ em $\mathcal{P}(U)$ por
    \[
    X \sim Y \mbox{ quando } X \cap T = Y \cap T,
    \]
    para $X$, $Y \in \mathcal{P}(U)$. Mostre que $\sim$ é uma relação de equivalência. Em particular, tome $U = \{1,2,3,4,5\}$ e $T = \{1,3\}$.
    \begin{enumerate}[label={\alph*})]
        \item Verdadeiro ou falso: $\{1,2,4\} \sim \{1,4,5\}$? Justifique.

        \item Verdadeiro ou falso: $\{1,2,4\} \sim \{1,3,4\}$? Justifique.

        \item Verdadeiro ou falso: $\{1,3,4\} \sim \{1,3,5\}$? Justifique.

        \item Encontre $\overline{\{1,5\}}$.

        \item Encontre $\overline{\{2,3\}}$.

    \end{enumerate}

    \vspace{.3cm}

    \questao{} Seja $\sim$ uma relação sobre $\rac$ definida da seguinte forma:
    \[
    x \sim y \mbox{ quando } x - y \in \mathbb{Z}.
    \]
    \begin{enumerate}[label={\alph*})]
        \item Prove que $\sim$ é uma relação de equivalência sobre $\rac$.

        \item Descreva a classe $\bar{1}$.

        \item Descreva a classe $\overline{1/2}$.
    \end{enumerate}

    \vspace{.3cm}

    \questao{} Defina
    \[
    H = \{3^m \mid m \in \z\} \mbox{ e } \rac^+ = \{x \in \rac \mid x > 0\}.
    \]
    Seja $R$ dado por
    \[
    R = \left\{(x,y) \in \rac^+\times \rac^+ : \frac{x}{y} \in H\right\}.
    \]
    \begin{enumerate}[label={\alph*})]
        \item Mostre que $R$ é uma relação de equivalência em $\rac^+$.

        \item Determine a classe de equivalência de $5$.
    \end{enumerate}

    \vspace{.3cm}

    \questao{} Mostre que as seguintes condições definem uma relação de equivalência em $\z$ e encontre suas classes de equivalência:
    \begin{enumerate}[label={\alph*})]
        \item $mRn$ quando $3 | (m + 2n)$

        \item $mSn$ quando $5 | (2m + 3n)$.
    \end{enumerate}

    \vspace{.3cm}

    \questao{} A divisibilidade (ou seja, a relação definida por $xRy$ se, e s{\'o}
    se, $x \mid y$) é uma relação de equivalência sobre $\z$?

    \vspace{.3cm}

    \questao{} Seja $R$ a seguinte relação sobre $\z^*$:
    \[
    x R y \mbox{ quando existem }  k, l \in \z \mbox{ tais que } y = kx \mbox{ e } x = ly.
    \]
    Mostre que $R$ é uma relação de equivalência sobre $\z^*$.

    \vspace{.3cm}

    \questao{} Seja $R = \{ (x, y) \in \real^2 \mid x - y \in \rac\}$. Prove que $R$ é uma relação de equivalência e descreva as classes representadas por $1/2$ e $\sqrt{2}$.

    \vspace{.3cm}

    \questao{} Seja $A$ um conjunto não vazio. Suponha que $R \sub A \times A$ é tal que:
    \begin{enumerate}[label={\alph*})]
        \item Para todo $x \in A$, $(x,x) \in R$;

        \item Para todos $x$, $y$, $z \in A$ se $(x, y) \in R$ e $(x,z) \in R$, então $(y,z) \in R$.
    \end{enumerate}
    Mostre que $R$ é uma relação de equivalência sobre $A$.

    \vspace{.3cm}

    \questao{} Sejam $R$ e $S$ relações de equivalência em um conjunto $A$. Defina $T = R \cap S$. Assim $aTb$ se, e somente se, $aRb$ e $aSb$. Mostre que $T$ é uma relação de equivalência em $A$.

    \vspace{.3cm}

    \questao{} É verdade que a união de duas relações de equivalência ainda é uma relação de equivalência?

\end{document}
