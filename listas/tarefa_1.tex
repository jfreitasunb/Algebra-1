%!TEX program = xelatex
%!TEX encoding = ISO-8859-1
\def\ano{2020}
\def\semestre{1}
\def\disciplina{\'Algebra 1}
\def\turma{C}
\def\numerosemana{04}

\documentclass[12pt]{exam}

\usepackage{caption}
\usepackage{amssymb}
\usepackage{amsmath,amsfonts,amsthm,amstext}
\usepackage[brazil]{babel}
% \usepackage[latin1]{inputenc}
\usepackage{graphicx}
\graphicspath{{/home/jfreitas/GitHub_Repos/Algebra-1/Pictures/}{D:/OneDrive - unb.br/imagens-latex/}}
\usepackage{enumitem}
\usepackage{multicol}
\usepackage{answers}
\usepackage{tikz,ifthen}
\usetikzlibrary{lindenmayersystems}
\usetikzlibrary[shadings]
\newcommand{\nsub}{\varsubsetneq}
\newcommand{\vaz}{\emptyset}
\newcommand{\im}{{\rm Im\,}}
\newcommand{\sub}{\subseteq}
\newcommand{\n}{\mathbb{N}}
\newcommand{\z}{\mathbb{Z}}
\newcommand{\rac}{\mathbb{Q}}
\newcommand{\real}{\mathbb{R}}
\newcommand{\complex}{\mathbb{C}}
\newcommand{\cp}[1]{\mathbb{#1}}
\newcommand{\ch}{\mbox{\textrm{car\,}}\nobreak}
\newcommand{\vesp}[1]{\vspace{ #1  cm}}
\newcommand{\compcent}[1]{\vcenter{\hbox{$#1\circ$}}}
\newcommand{\comp}{\mathbin{\mathchoice
{\compcent\scriptstyle}{\compcent\scriptstyle}
{\compcent\scriptscriptstyle}{\compcent\scriptscriptstyle}}}

\begin{document}
    \vspace{.6cm}

    \begin{center}
        Lista Semana 02
    \end{center}
    \noindent \textbf{Questão 3}
    \begin{enumerate}
        \item[f)] Se $A$, $B$ e $C$ s\~ao conjuntos, ent\~ao $A \cup (B \cap C) = (A \cup B) \cap C$.
    \end{enumerate}
    
    \vspace{.3cm}

    \noindent \textbf{Questão 4}
    \begin{enumerate}
        \item[e)] $A \cup (B \cap (A \cup C)) = A \cup (B \cap C)$.
    \end{enumerate}

    \hrule
    \begin{center}
        Lista Semana 03
    \end{center}

    \noindent \textbf{Questão 7}
    \begin{enumerate}
        \item[c)] $(A - B) - C = A - (B \cup C)$.

        \item[j] $(A \cap B) \cap (A - B) = (A - B) \cap (B - A) = \vaz$.
    \end{enumerate}

    \vspace{.3cm}

    \noindent \textbf{Questão 11}
    \begin{enumerate}
        \item[c)] $A \times (B \cap C) = (A \times B) \cap (A \times C)$
    \end{enumerate}

    \vspace{.3cm}

    \noindent \textbf{Questão 12}
    \begin{enumerate}
        \item[b)] Suponha $A \ne \emptyset$ e $C \ne \emptyset$, com $A \ne C$. Mostre que $A \sub B$ e $C \sub D$ se, e somente se, $A \times C \sub B \times D$.
    \end{enumerate}

    \hrule
    \begin{center}
        Lista Semana 04
    \end{center}

    \noindent \textbf{Questão 6}
    Seja $A = \real^2$ e considere o conjunto definido por
    \[
      (a,b)R(c,d) \mbox{ quando } 2a - b = 2c - d.
    \]
    Mostre que $R$ \'e uma rela\c{c}\~ao de equival\^encia sobre $\real^2$.
    
    \vspace{.3cm}

    \noindent \textbf{Questão 8}
    Seja $A = \real^3$. Dados $u = (x_1, y_1, z_1)$, $v = (x_2, y_2, z_2) \in \real^3$ defina
    \[
        u\cdot v = x_1x_2 + y_1y_2 + z_1z_2.
    \]
    Tome um elemento fixo $w = (\alpha, \beta, \gamma) \in \real^3$ e defina
    \[
        u \sim v \mbox{ quando } u \cdot w = v \cdot w.
    \]
    Mostre que $\sim$ \'e uma rela\c{c}\~ao de equival\^encia sobre $\real^3$.
    
    \vspace{.3cm}
    
    \noindent \textbf{Questão 11}
    Seja $A = \z \times \z^*$, onde $\mathbb{Z}^* = \mathbb{Z} \setminus \{0\}$. Para $(a,b)$, $(c,d) \in A$, considere a seguinte rela{\c c}{\~a}o
    \[
        (a,b) R (c,d) \mbox{ quando } ad = bc.
    \]
    \begin{enumerate}
        \item[b)] Descreva a classe de equival{\^e}ncia $\overline{(0,1)}$, $\overline{(1,1)}$, $\overline{(1,2)}$, $\overline{(2,1)}$, $\overline{(2,2)}$, $\overline{(2,3)}$.
    \end{enumerate}

    \vspace{.3cm}

    \noindent \textbf{Questão 15}
    Defina
    \[
        H = \{2^m \mid m \in \z\} \mbox{ e } \rac^+ = \{x \in \rac \mid x > 0\}.
    \]
    Seja $R$ dado por
    \[
        R = \left\{(x,y) \in \rac^+\times \rac^+ : \frac{x}{y} \in H\right\}.
    \]
    \begin{enumerate}[label={\alph*})]
        \item Mostre que $R$ \'e uma rela\c{c}\~ao de equival\^encia em $\rac^+$.
        \item Determine a classe de equival\^encia de $3$.
    \end{enumerate}
    
\end{document}