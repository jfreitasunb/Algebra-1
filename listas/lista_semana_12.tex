%!TEX program = xelatex
% !TEX encoding = ISO-8859-1
\def\ano{2020}
\def\semestre{1}
\def\disciplina{\'Algebra 1}
\def\turma{C}
\def\numerosemana{12}

\documentclass[12pt]{exam}

\usepackage{caption}
\usepackage{amssymb}
\usepackage{amsmath,amsfonts,amsthm,amstext}
\usepackage[brazil]{babel}
% \usepackage[latin1]{inputenc}
\usepackage{graphicx}
\graphicspath{{/ArquivosLinux/OneDrive/imagens-latex/}{D:/OneDrive - unb.br/imagens-latex/}}
\usepackage{enumitem}
\usepackage{multicol}
\usepackage{answers}
\usepackage{tikz,ifthen}
\usetikzlibrary{lindenmayersystems}
\usetikzlibrary[shadings]
\Newassociation{solucao}{Solution}{ans}
\newtheorem{exercicio}{}

\setlength{\topmargin}{-1.0in}
\setlength{\oddsidemargin}{0in}
\setlength{\textheight}{10.1in}
\setlength{\textwidth}{6.5in}
\setlength{\baselineskip}{12mm}

\extraheadheight{0.7in}
\firstpageheadrule
\runningheadrule
\lhead{
        \begin{minipage}[c]{1.7cm}
        \includegraphics[width=1.7cm]{unb.pdf}
        \end{minipage}%
        \hspace{0pt}
        \begin{minipage}[c]{4in}
          {Universidade de Brasília} --
          {Departamento de Matemática}
\end{minipage}
\vspace*{-0.8cm}
}
% \chead{Universidade de Brasília - Departamento de Matemática}
% \rhead{}
% \vspace*{-2cm}

\extrafootheight{.5in}
\footrule
\lfoot{\disciplina\ - \semestre$^o$/\ano\ - Módulo \numeromodulo}
\cfoot{}
\rfoot{Página \thepage\ de \numpages}

\newcounter{exercicios}
\renewcommand{\theexercicios}{\arabic{exercicios}}

\newenvironment{questao}[1]{
\refstepcounter{exercicios}
\ifx&#1&
\else
   \label{#1}
\fi
\noindent\textbf{Exercício {\theexercicios}:}
}

\newcommand{\resp}[1]{
\noindent{\bf Exercício #1: }}

\def\ano{2024}
\def\semestre{1}
\def\disciplina{Álgebra 1}
\def\nomeabreviado{Álgebra 1}
\def\turma{1}

\newcommand{\im}{{\rm Im\,}}
\newcommand{\dlim}[2]{\displaystyle\lim_{#1\rightarrow #2}}
\newcommand{\minf}{+\infty}
\newcommand{\ninf}{-\infty}
\newcommand{\cp}[1]{\mathbb{#1}}
\newcommand{\sub}{\subseteq}
\newcommand{\n}{\mathbb{N}}
\newcommand{\z}{\mathbb{Z}}
\newcommand{\rac}{\mathbb{Q}}
\newcommand{\real}{\mathbb{R}}
\newcommand{\complex}{\mathbb{C}}

\newcommand{\vesp}[1]{\vspace{ #1  cm}}

\newcommand{\compcent}[1]{\vcenter{\hbox{$#1\circ$}}}
\newcommand{\comp}{\mathbin{\mathchoice
        {\compcent\scriptstyle}{\compcent\scriptstyle}
        {\compcent\scriptscriptstyle}{\compcent\scriptscriptstyle}}}
\renewcommand{\sin}{{\rm sen\,}}
\renewcommand{\tan}{{\rm tg\,}}
\renewcommand{\csc}{{\rm cossec\,}}
\renewcommand{\cot}{{\rm cotg\,}}
\renewcommand{\sinh}{{\rm senh\,}}

\begin{document}

\begin{center}
    
    {\Large\bf \disciplina\ - Turma \turma\ -- \semestre$^{o}$/\ano} \\ \vspace{9pt} {\large\bf
        Lista de Exerc{\'\i}cios -- Semana \numerosemana}\\ \vspace{9pt} Prof. Jos{\'e} Ant{\^o}nio O. Freitas
    \end{center}
    \hrule

    \vspace{.6cm}
    
    \questao{} Verifique se s\~ao subgrupos:
    \begin{enumerate}[label=({\alph*})]
      \item $H = \{x \in \rac \mid x > 0\}$ de $(\rac^*,\cdot)$.
      \item $H = \left\{\dfrac{1 + 2m}{1 + 2n} \mid m, n \in \z\right\}$ de $(\rac^*,\cdot)$.
      \item $H = \{\cos\theta + i\sin\theta \mid \theta \in \rac\}$ de $(\complex^*,\cdot)$.
      \item $H = \{0, \pm 2, \pm 4, \pm 6, \dots\}$ de $(\z,+)$.
      \item $H = \{0, \pm 2, \pm 4, \pm 6, \dots\}$ do grupo $(\rac - \{1\},\star)$ onde $\star$ \'e definida como $x \star y = x + y - xy$.
      \item $H = \{a + b\sqrt{2} \mid a, b \in \rac\}$ de $(\real,+)$.
      \item $H = \{a + b\sqrt{2} \in \real^* \mid a, b \in \rac\}$ de $(\real^*,\cdot)$.
      \item $H = \{a + b\sqrt[3]{2} \mid a, b \in \rac\}$ de $(\real,+)$.
      \item $H = \{a + b\sqrt[3]{2} \in \real^* \mid a, b \in \rac\}$ de $(\real^*,\cdot)$.
    \end{enumerate}

    \vspace{.3cm}

    \questao{} Determine todos os subgrupos do grupo aditivo $\z_4$.

    \vspace{.3cm}

    \questao{} Determine todos os subgrupos de $S_3$.

    \vspace{.3cm}

    \questao{} Seja
    \[
      GL_2(\real) = \left\{A = \begin{pmatrix}
          x & y\\z & t
      \end{pmatrix} \mid x, y, z, t \in \real,\ \det(A) \ne 0\right\}.
    \]
    \begin{enumerate}[label=({\alph*})]
      \item Mostre que $GL_2(\real)$ com a opera\c{c}\~ao de multiplica\c{c}\~ao de matrizes \'e um grupo. Esse grupo \'e abeliano?
      \item Seja
      \[
          H = \left\{ A = \begin{pmatrix}
              \cos a & \sin a\\ -\sin a & \cos a
          \end{pmatrix} \mid a \in \real\right\}.
      \]
      Mostre que $H$ \'e um subgrupo de $GL_2(\real)$.
      \item Seja
      \[
          K = \left\{ A = \begin{pmatrix}
              a & b\\ -b & a
          \end{pmatrix} \mid a, b \in \real \mbox{ e n\~ao nulos simultaneamente}\right\}.
      \]
      Mostre que $K$ \'e um subgrupo de $GL_2(\real)$.
    \end{enumerate}

    \questao{} Sejam $H$ e $K$ subgrupos de um grupo $G$ (com nota{\c c}{\~a}o
    multiplicativa).
    \begin{enumerate}[label=({\alph*})]
    \item Mostre que $H\cap K$ tamb{\'e}m {\'e} subgrupo de $G$.
    \item Seja $g\in G$ um elemento fixado. Mostre que o conjunto
    $g^{-1}Hg=\{ g^{-1}xg \mid x\in H \} $ {\'e} um subgrupo de $G$.
    \item Prove que $H\cup K$ {\'e} subgrupo de $G$ se, e somente se,
    $H\subseteq K$ ou $K\subseteq H$.
    \item Demonstre que $HK=\{hk \mid h\in H, k\in K\}$ {\'e} subgrupo
    de $G$ se, e somente se, $HK=KH$.

    [\emph{Nota: $HK=KH$ \textbf{n{\~a}o} quer dizer que $hk=kh$,
    para todo $h\in H, k\in K$; significa que $hk=k_1h_1 \in KH$ e $kh=h_2k_2 \in
    HK$, para todo $h\in H, k\in K$.}]
    \end{enumerate}

    \vspace{.3cm}
    \questao{} Seja $G$ um grupo com nota\c{c}\~ao multiplicativa e $a$ um elemento de $G$. Prove que $N(a) = \{x \in G \mid ax = xa\}$ \'e um subgrupo de $G$.

    \vspace{.3cm}

    \questao{} Seja $G$ um grupo com nota\c{c}\~ao multiplicativa. Considere o subconjunto $Z(G) = \{x \in G \mid xg = gx, \mbox{ para todo } x \in G\}$. Mostre que:
    \begin{enumerate}[label=({\alph*})]
      \item $Z(G)$ \'e um subgrupo de $G$.
      \item $G$ \'e abeliano se, e somente se, $Z(G) = G$.
    \end{enumerate}

    \vspace{.3cm}

    \questao{} Seja $G$ um grupo. Dado $H \subset G$ um subgrupo defina
    \begin{align}\label{relacao_equivalencia_subgrupo}
        x \cong y \mbox{ se, e somente se, } xy^{-1} \in H
    \end{align}
    para todos $x$, $y \in G$.
    \begin{enumerate}[label={\alph*})]
        \item A rela\c{c}\~ao definida em \eqref{relacao_equivalencia_subgrupo} \'e uma rela\c{c}\~ao de equival\^encia.

        \item Se $y \in G$, ent\~ao a classe de equival\^encia determinada por $y$ \'e o conjunto
        \begin{align*}\label{classe_equivalencia_subgrupo}
            Hy = \{ty \mid t \in H\}.
        \end{align*}
    \end{enumerate}

    \vspace{.3cm}

    \questao{} Seja $H$ um subgrupo de $G$. Denote por
    \begin{align*}
        \mathcal{P} &= \{aH \mid a \in G\}\\
        \mathcal{Q} &= \{Hb \mid b \in G\}.
    \end{align*}
    Mostre que a função $f : \mathcal{P} \to \mathcal{Q}$ dada por $f(aH) = Ha^{-1}$ é uma bijeção.

    \vspace{.3cm}

    \questao{} Construa os seguintes subgrupos:
    \begin{enumerate}[label={\alph*})]
        \item $[-1]$ em $(\rac, +)$

        \item $[6]$ em $(\z, +)$

        \item $[6]$ em $(\rac^*, \cdot)$

        \item $[i]$ em $(\complex, +)$
    \end{enumerate}

    \vspace{.3cm}

    \questao{} Mostre que os elementos não nulos de $\z_{13}$ formam um grupo multiplicativo cíclico.

    \vspace{.3cm}

    \questao{} Seja $a \ne e$ um elemento do grupo $G$. Mostre que $o(a) = 2$ se, e somente se, $a = a^{-1}$.

    \vspace{.3cm}

    \questao{} Seja $G$ um grupo multiplicativo $x \in G$. Mostre que, se existe um inteiro $n \ge 1$ tal que $x^n = e$, então existe um inteiro $m \ge 1$ tal que $x^{-1} = x^m$.

    \vspace{.3cm}

    \questao{} Seja $G$ um grupo multiplicativo e suponha que $a \in G$. Mostre que $o(a) = o(a^{-1}) = o(xax^{-1})$.

    \vspace{.3cm}

    \questao{} Seja $G$ um grupo finito e denote por $e$ seu elemento neutro. Se $x \in G$, mostre que existe $n \in \z$ tal que $x^n = e$.

    \vspace{.3cm}

    \questao{} Sejam $a$ e $b$ elementos de um grupo $G$ finito. Se $[a] = [b]$, prove que $a$ e $b$ têm a mesma ordem.
\end{document}