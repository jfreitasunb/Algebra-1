%!TEX program = xelatex

\def\numerosemana{03}

\documentclass[12pt]{exam}

\def\ano{2022}
\def\semestre{1}
\def\disciplina{\'Algebra 1}
\def\turma{2}

\usepackage{caption}
\usepackage{amssymb}
\usepackage{amsmath,amsfonts,amsthm,amstext}
\usepackage[brazil]{babel}
% \usepackage[latin1]{inputenc}
\usepackage{graphicx}
\graphicspath{{/ArquivosLinux/OneDrive/imagens-latex/}{D:/OneDrive - unb.br/imagens-latex/}}
\usepackage{enumitem}
\usepackage{multicol}
\usepackage{answers}
\usepackage{tikz,ifthen}
\usetikzlibrary{lindenmayersystems}
\usetikzlibrary[shadings]
\Newassociation{solucao}{Solution}{ans}
\newtheorem{exercicio}{}

\setlength{\topmargin}{-1.0in}
\setlength{\oddsidemargin}{0in}
\setlength{\textheight}{10.1in}
\setlength{\textwidth}{6.5in}
\setlength{\baselineskip}{12mm}

\extraheadheight{0.7in}
\firstpageheadrule
\runningheadrule
\lhead{
        \begin{minipage}[c]{1.7cm}
        \includegraphics[width=1.7cm]{unb.pdf}
        \end{minipage}%
        \hspace{0pt}
        \begin{minipage}[c]{4in}
          {Universidade de Brasília} --
          {Departamento de Matemática}
\end{minipage}
\vspace*{-0.8cm}
}
% \chead{Universidade de Brasília - Departamento de Matemática}
% \rhead{}
% \vspace*{-2cm}

\extrafootheight{.5in}
\footrule
\lfoot{\disciplina\ - \semestre$^o$/\ano\ - Módulo \numeromodulo}
\cfoot{}
\rfoot{Página \thepage\ de \numpages}

\newcounter{exercicios}
\renewcommand{\theexercicios}{\arabic{exercicios}}

\newenvironment{questao}[1]{
\refstepcounter{exercicios}
\ifx&#1&
\else
   \label{#1}
\fi
\noindent\textbf{Exercício {\theexercicios}:}
}

\newcommand{\resp}[1]{
\noindent{\bf Exercício #1: }}

\def\ano{2024}
\def\semestre{1}
\def\disciplina{Álgebra 1}
\def\nomeabreviado{Álgebra 1}
\def\turma{1}

\newcommand{\im}{{\rm Im\,}}
\newcommand{\dlim}[2]{\displaystyle\lim_{#1\rightarrow #2}}
\newcommand{\minf}{+\infty}
\newcommand{\ninf}{-\infty}
\newcommand{\cp}[1]{\mathbb{#1}}
\newcommand{\sub}{\subseteq}
\newcommand{\n}{\mathbb{N}}
\newcommand{\z}{\mathbb{Z}}
\newcommand{\rac}{\mathbb{Q}}
\newcommand{\real}{\mathbb{R}}
\newcommand{\complex}{\mathbb{C}}

\newcommand{\vesp}[1]{\vspace{ #1  cm}}

\newcommand{\compcent}[1]{\vcenter{\hbox{$#1\circ$}}}
\newcommand{\comp}{\mathbin{\mathchoice
        {\compcent\scriptstyle}{\compcent\scriptstyle}
        {\compcent\scriptscriptstyle}{\compcent\scriptscriptstyle}}}
\renewcommand{\sin}{{\rm sen\,}}
\renewcommand{\tan}{{\rm tg\,}}
\renewcommand{\csc}{{\rm cossec\,}}
\renewcommand{\cot}{{\rm cotg\,}}
\renewcommand{\sinh}{{\rm senh\,}}

\begin{document}
    \begin{center}
        {\Large\bf \disciplina\ - Turma \turma\ -- \semestre$^{o}$/\ano} \\ \vspace{9pt} {\large\bf
            Lista de Exerc{\'\i}cios -- Semana \numerosemana}\\ \vspace{9pt} Prof. Jos{\'e} Ant{\^o}nio O. Freitas
    \end{center}
    \hrule

    \vspace{.6cm}

    \begin{center}
        \textit{Observa\c{c}\~ao: A nota\c{c}\~ao $A \nsub B$ significa que $A \subset B$ e $A \ne B$.}
    \end{center}

    \vspace{.3cm}

    \questao{} Dê exemplos de conjuntos $A$, $B$ e $C$ tais que as interseções $A \cap B$, $A \cap C$ e $B \cap C$ não são o conjunto vazio mas $A \cap B \cap C$ é o conjunto vazio.

    \vspace{.3cm}

    \questao{} Determinar os elementos dos conjuntos $A$, $B$ e $E$ tais que:
    \[
        A \cap B = \{b, c\}, \quad C_E(A) = \{d, e, f\} \quad \mbox{e}\quad C_E(B) = \{a          , e, f\}.
    \]

    \vspace{.3cm}

    \questao{} Dados os conjuntos $A = \{a, b, c, d\}$, $B = \{c, d, e, f\}$, $C = \{d,       e, f, g\}$, determine:
    \begin{enumerate}[label={\alph*})]
        \item $A - (B - C)$ e $(A - B) - C$

        \item $(A \cap B) - (B \cup C)$ e $(A \cup C) - (A \cup B)$

        \item $A \cap (B - C)$ e $(A \cap B) - C$

        \item $A - (B \cup C)$ e $(A - B) \cap (B - C)$
    \end{enumerate}

    \questao{} Dados os conjuntos $E = \{1,2,3,4,5,a,b,c,d\}$, $A = \{1,2,4,d\}$ e $B =        \{a,2,4,b,5\}$, determine:
    \begin{multicols}{2}
        \begin{enumerate}[label={\alph*})]
            \item $A \cup B$

            \item $A \cup C_E(B)$

            \item $C_E(A) \cup B$

            \item $C_E(A \cup B)$
        \end{enumerate}
    \end{multicols}

    \vspace{.3cm}

    \questao{} Dados os conjuntos $A = \{3,6,9,12,15,18\}$, $B = \{2,3,5,7,11,13,17,19\}      $ e $C = \{1,2,3,4,5,\dots,20\}$ calcule:
    \begin{multicols}{3}
        \begin{enumerate}[label={\alph*})]
            \item $A - B$

            \item $A - C$

            \item $B - C$

            \item $(A - B) - C$

            \item $A - (B - C)$

            \item $(A \cup B) - C$

            \item $A - (B \cap C)$

            \item $(A \cup C) - B$

            \item $(B \cap C) - A$
        \end{enumerate}
    \end{multicols}

    \questao{} Determine os conjuntos $E$, $A$ e $B$ se:
    \begin{align*}
        C_E(A) &= \{2, 5, 9, 13, 18, 20\}\\
        C_E(B) &= \{2, 6, 18, 20\}\\
        A \cup B &= \{1, 5, 6, 9, 13, 14\}.
    \end{align*}

    \newpage

    \questao{} Demonstre que:
    \begin{enumerate}[label={\alph*})]

        \item Se $A \sub B$ e $C = B - A$, ent\~ao $A = B - C$.

        \item Se $A \cap B = \vaz$ e $A \cup B = C$, ent\~ao $A = C - B$.

        \item Se $A\cup B = E$, ent\~ao $C_E(A) \sub B$. \textit{(Aqui suponha que $A$,            $B \sub E$.)}

        \item Se $A \cap B = \vaz$, ent\~ao $A \cup C_E(B) = C_E(B)$. \textit{(Aqui supo           nha que $A$, $B \sub E$.)}

        \item Suponha que $A \subseteq E$. Mostre que $C_E((C_E(A) \cap E)) = A$.

        \item $A = B$ se, e somente se, $A - B = B - A$.

        \item $A \sub B$ se, e somente se, $A - B = \vaz$.

        \item $C_E(A) \sub C_E(B)$ se, e somente se, $A \cup B = A$. \textit{(Aqui suponha que $A$, $B \sub E$.)}

        \item $C_E(A) \sub C_E(B)$ se, e somente se, $A \cap B = B$. \textit{(Aqui suponha que $A$, $B \sub E$.)}

    \end{enumerate}

    \vspace{.3cm}

    \questao{} Demonstre as seguintes igualdades.
    \begin{enumerate}[label={\alph*})]
        \item $A \cup (C_E(A) \cap B) = A \cup B$. \textit{(Aqui suponha que $A$, $B \sub E$.)}

        \item $A \cap (C_E(A) \cup B) = A \cap B$. \textit{(Aqui suponha que $A$, $B \sub E$.)}

        \item $(A \cap C_E(B)) \cup (C_E(A) \cap B) = (A \cup B) \cap C_E(A \cap B)$. \textit{(Aqui suponha que $A$, $B \sub E$.)}
        \item $(A - B) - C = A - (B \cup C)$.

        \item $A \cup (B - C) = (A \cup B) - (C - A)$.

        \item $A \cap (B - C) = (A \cap B) - (A \cap C)$.

        \item $A - (B \cup C) = (A - B) \cap (A - C)$.

        \item $A - (B \cap C) = (A - B) \cup (A - C)$.

        \item $(A \cup B) - C = (A - C) \cup (B - C)$.

        \item $(A \cap B) - C = (A - C) \cap (B - C)$.

        \item $(A \cap B) \cap (A - B) = (A - B) \cap (B - A) = \vaz$.

        \item $(A - B) \cup (B - A) = (A \cup B) - (A \cap B)$.

        \item $A - (A - B) = A \cap B$.

        \item $(A - B) - (B - C) = A - B$.

        \item $(A - B) - B = A - B$.

        \item $A \cup (B \cap (A \cup C)) = A \cup (B \cap C)$.

        \item $(A \cap B) \cup C_E(A \cup C_E(B)) = B$. \textit{(Aqui suponha que $A$, $B \sub E$.)}

        \item $C_E( A \cap [B \cup C]) = C_E(A) \cup [C_E(B)\cap C_E(C)]$. \textit{(Aqui suponha que $A$, $B$, $C \sub E$.)}

        \item $C_E( A \cap B \cap C) = C_E(A) \cup C_E(B) \cup C_E(C)$. \textit{(Aqui suponha que $A$, $B$, $C \sub E$.)}

        \item $C_E( A \cup B \cup C) = C_E(A) \cap C_E(B) \cap C_E(C)$. \textit{(Aqui suponha que $A$, $B$, $C \sub E$.)}
    \end{enumerate}

    \vspace{.3cm}

    %\questao{} Dados conjuntos $A$ e $B$, defina:
    %\[
    %    A \Delta B = (A - B) \cup (B - A).
   % \]
    %Agora considerando os conjuntos $A$, $B$ e $C$ verifique se as seguintes propriedades s\~ao verdadeiras:
    %\begin{multicols}{2}
    %    \begin{enumerate}[label={\alph*})]
    %    \item $A \Delta A = \emptyset$

    %    \item $A \Delta B = B \Delta A$

    %    \item $A \Delta \emptyset = \emptyset \Delta A = A$

    %    \item $A \Delta (A \Delta B) = B$

    %    \item $(A \Delta B) \Delta C = A \Delta (B \Delta C)$

    %    \item $A \cap (B \Delta C) = (A \cap B) \Delta (A \cap C)$

    %    \item $A \Delta B = (A \cup B) - (A \cap B)$
    %\end{enumerate}
    %\end{multicols}

    %\vspace{.3cm}

    \questao{} Dados os conjuntos $A = \{1, 2, 3, r, s\}$, $B = \{r, s, 2, 3, 4\}$, $C = \{s, t, 4, 5\}$, $D = \{s, t, 5, 6\}$ e $E = \{1, 2, 3, 4, 5, 6, r, s, t\}$, determine:
    \begin{multicols}{2}
        \begin{enumerate}[label={\alph*})]
            \item $(A \times A) \cap (B \times B)$

            \item $A^2 \times C^2$

            \item $(A - B) \times (C - D)$

            \item $(A \cap B) \times (C \cap B)$

            \item $(A \cup B) \times (B \cup D)$

            \item $(A \times B) - (C \times D)$

            \item $(A - B) \times (C \cap D)$

            \item $(A - C) \times (B - D)$

            \item $(A - (C - D)) \times ((D - B) \cup A)$

            %\item $(A \Delta B) \times (D \Delta B)$

            \item $C_E(A) \times C_E(B)$

            \item $C_{E\times E}(A \times B)$

            \item $(A - B) \times (C_E(B) - C_E(A))$
        \end{enumerate}

    \end{multicols}

    \vspace{.3cm}

    %\questao{} Determine conjuntos $A$, $B \sub E$ tais que:
    %\begin{align*}
    %    A \Delta B &= \{2, 4, 5, 8, 9, 10\}\\
    %    A \cap B &= \{1, 3\}\\
    %    C_E(A) &= \{5, 6, 7, 9, 10\}\\
    %    E &= \{1, 2, 3, 4, 5, 6, 7, 8, 9, 10\}.
    %\end{align*}

    \questao{} Sejam $A$ e $B$ conjuntos tais que $A$, $B \subseteq E$. Mostre que $C_E(A) \times C_E(B) \subseteq C_{E \times E}(A \times B)$.

    \vspace{.3cm}

    \questao{} Seja $A$ e $B$ conjuntos tais que $A$, $B \subseteq E$. É verdade que
    \[
        C_{E\times E}(A \times B) = C_E(A) \times C_E(B)?
    \]
    Prove ou apresente um contra-exemplo.

    \vspace{.3cm}

    \questao{} Sejam $A$, $B$, $C$ e $D$ conjuntos quaisquer. A igualdade
    \[
        (A \cup B) \times (C \cup D) = (A \times C) \cup (B \times D)
    \]
    é válida? Justifique.

    \vspace{.3cm}

    \questao{} Sejam $A$, $B$, $C$ e $D$ conjuntos. Prove que:
    \begin{enumerate}[label={\alph*})]
        \item Se $A \sub C$ e $B \sub D$, ent\~ao $A \times B \sub C \times D$.

        \item $A \times (B \cup C) = (A \times B) \cup (A \times C)$

        \item $A \times (B \cap C) = (A \times B) \cap (A \times C)$

        \item $A \times B = \emptyset$ se, e somente se, $A = \emptyset$ ou $B = \emptyset$.

        \item $A \times B \ne \emptyset$ se, e somente se, $A \ne \emptyset$ e $B \ne \emptyset$.

        \item $(A \cap B) \times (C \cap D) = (A \times C) \cap (B \times D)$

        \item $(A \times B) - (C_E(A) \times C_E(B)) = [(A - C_E(A)) \times (B - C_E(B))] \cup [(A - C_E(A)) \times B]$  \textit{(Aqui suponha que $A$, $B \sub E$.)}

        \item $A \times (B - C) = (A \times B) - (A \times C)$

        \item $(A \times B) - (C \times D) = [A \times (B - D)] \cup [(A - C) \times B]$
    \end{enumerate}

    \vspace{.3cm}

    \questao{} Sejam $A$, $B$, $C$ e $D$ conjuntos quaisquer. É válido que
    \[
        (A \times B) - (C \times D) = (A - C) \times (B - D)?
    \]
    Prove ou dê um contra-exemplo.

    \vspace{.3cm}

    \questao{} Sejam $A$, $B$, $C$ e $D$ conjuntos.
    \begin{enumerate}[label={\alph*})]
        \item Mostre que $A$ e $B$ s\~ao disjuntos se, e somente se, para todo conjunto n\~ao vazio $C$, $A \times C$ e $B \times C$ s\~ao disjuntos.

        \item Suponha $A \ne \emptyset$ e $C \ne \emptyset$, com $A \ne C$. Mostre que $A \sub B$ e $C \sub D$ se, e somente se, $A \times C \sub B \times D$.
    \end{enumerate}

    \vspace{.3cm}

    \questao{} Sejam $X_1$, $X_1$, $Y_1$ e $Y_2$ subconjuntos contidos num conjunto $E$. Suponha que $X_1 \cup X_2 = E$, $Y_1 \cap Y_2 = \emptyset$, $X_1 \sub Y_1$ e $X_2 \sub Y_2$. Prove que $X_1 = Y_1$ e $X_2 = Y_2$.

    \vspace{.3cm}

    \questao{} Sejam $A$ e $B$ conjuntos quaisquer. Mostre que $A \subset B$ se, e somente se, $\mathcal{P}(A) \subset \mathcal{P}(B)$.

    \vspace{.3cm}

    \questao{} Sejam $A$ e $B$ conjuntos quaisquer. Mostre que $\mathcal{P}(A \cap B) = \mathcal{P}(A) \cap \mathcal{P}(B)$.

    \vspace{.3cm}

    \questao{} Sejam $A$ e $B$ conjuntos quaisquer. É verdade que
    \[
       \mathcal{P}(A \cup B) = \mathcal{P}(A) \cup \mathcal{P}(B)?
    \]
    Prove ou apresente um contra-exemplo.
\end{document}
