%!TEX program = xelatex
% !TEX encoding = ISO-8859-1
\def\ano{2019}
\def\semestre{2}
\def\disciplina{\'Algebra 1}
\def\turma{C}
\def\numerolista{6}

\documentclass[12pt]{exam}

\usepackage{caption}
\usepackage{amssymb}
\usepackage{amsmath,amsfonts,amsthm,amstext}
\usepackage[brazil]{babel}
% \usepackage[latin1]{inputenc}
\usepackage{graphicx}
\graphicspath{{/ArquivosLinux/OneDrive/imagens-latex/}{D:/OneDrive - unb.br/imagens-latex/}}
\usepackage{enumitem}
\usepackage{multicol}
\usepackage{answers}
\usepackage{tikz,ifthen}
\usetikzlibrary{lindenmayersystems}
\usetikzlibrary[shadings]
\Newassociation{solucao}{Solution}{ans}
\newtheorem{exercicio}{}

\setlength{\topmargin}{-1.0in}
\setlength{\oddsidemargin}{0in}
\setlength{\textheight}{10.1in}
\setlength{\textwidth}{6.5in}
\setlength{\baselineskip}{12mm}

\extraheadheight{0.7in}
\firstpageheadrule
\runningheadrule
\lhead{
        \begin{minipage}[c]{1.7cm}
        \includegraphics[width=1.7cm]{unb.pdf}
        \end{minipage}%
        \hspace{0pt}
        \begin{minipage}[c]{4in}
          {Universidade de Brasília} --
          {Departamento de Matemática}
\end{minipage}
\vspace*{-0.8cm}
}
% \chead{Universidade de Brasília - Departamento de Matemática}
% \rhead{}
% \vspace*{-2cm}

\extrafootheight{.5in}
\footrule
\lfoot{\disciplina\ - \semestre$^o$/\ano\ - Módulo \numeromodulo}
\cfoot{}
\rfoot{Página \thepage\ de \numpages}

\newcounter{exercicios}
\renewcommand{\theexercicios}{\arabic{exercicios}}

\newenvironment{questao}[1]{
\refstepcounter{exercicios}
\ifx&#1&
\else
   \label{#1}
\fi
\noindent\textbf{Exercício {\theexercicios}:}
}

\newcommand{\resp}[1]{
\noindent{\bf Exercício #1: }}

\def\ano{2024}
\def\semestre{1}
\def\disciplina{Álgebra 1}
\def\nomeabreviado{Álgebra 1}
\def\turma{1}

\newcommand{\im}{{\rm Im\,}}
\newcommand{\dlim}[2]{\displaystyle\lim_{#1\rightarrow #2}}
\newcommand{\minf}{+\infty}
\newcommand{\ninf}{-\infty}
\newcommand{\cp}[1]{\mathbb{#1}}
\newcommand{\sub}{\subseteq}
\newcommand{\n}{\mathbb{N}}
\newcommand{\z}{\mathbb{Z}}
\newcommand{\rac}{\mathbb{Q}}
\newcommand{\real}{\mathbb{R}}
\newcommand{\complex}{\mathbb{C}}

\newcommand{\vesp}[1]{\vspace{ #1  cm}}

\newcommand{\compcent}[1]{\vcenter{\hbox{$#1\circ$}}}
\newcommand{\comp}{\mathbin{\mathchoice
        {\compcent\scriptstyle}{\compcent\scriptstyle}
        {\compcent\scriptscriptstyle}{\compcent\scriptscriptstyle}}}
\renewcommand{\sin}{{\rm sen\,}}
\renewcommand{\tan}{{\rm tg\,}}
\renewcommand{\csc}{{\rm cossec\,}}
\renewcommand{\cot}{{\rm cotg\,}}
\renewcommand{\sinh}{{\rm senh\,}}

\begin{document}

\begin{center}
{\Large\bf \disciplina\ - Turma \turma\ -- \semestre$^{o}$/\ano} \\ \vspace{9pt} {\large\bf
  $\numerolista^{\underline{a}}$ Lista de Exerc{\'\i}cios -- Subanéis e Homomorfismos}\\ \vspace{9pt} Prof. Jos{\'e} Ant{\^o}nio O. Freitas
\end{center}
\hrule
\vspace{.6cm}

\textit{Nos casos em que n\~ao forem especificadas as opera\c{c}\~oes do anel, considere as opera\c{c}\~oes usuais.}

\vspace{.6cm}

\questao{} Determinar quais dos seguintes subconjuntos de $\rac$ s{\~a}o suban{\'e}is:
	\begin{multicols}{2}
		\begin{enumerate}[label=({\alph*})]
			\item $\z$
			\item $B = \{x \in \rac \mid x \notin \z\}$
			\item $C = \left\{\dfrac{a}{b} \in \rac \mid a \in \z,\ b \in \z,\ 2 |b \right\}$
			\item $D = \left\{\dfrac{a}{2^n} \in \rac \mid a \in \z \mbox{ e } n \in \z \right\}$
		\end{enumerate}
	\end{multicols}
 

\vspace{.3cm}

\questao{} No anel $(\z \times \z, \oplus, \otimes)$ onde as opera\c{c}\~oes $\oplus$ e $\otimes$ s\~ao definidas por
\begin{align*}
	(a, b) \oplus (c, d) = (a + c, b + d)\\
	(a ,b) \otimes (c, d) = (ac - bd, ad + bc).
\end{align*}
Quais dos seguintes conjuntos s\~ao suban\'eis?
\begin{enumerate}[label=({\alph*})]
	\item $A = \{(x, y) \in \z \times \z \mid x = 0\}$
	\item $B = \{(x, y) \in \z \times \z \mid y = 0\}$
	\item $C = \{(x, y) \in \z \times \z \mid x = y\}$
	\item $D = \{(x, y) \in \z \times \z \mid x = 2k,\ k \in \z\}$
	\item $E = \{(x, y) \in \z \times \z \mid y = 3k,\ k \in \z\}$
	\item $F = \{(x, y) \in \z \times \z \mid x + y = 2k,\ k \in \z\}$
\end{enumerate}

\vspace{.3cm}

\questao{} No anel $(\rac, \star, \odot)$ onde as opera\c{c}\~oes $\star$ e $\odot$ em $\rac$ definidas por
\begin{align*}
	x \star y = x + y - 6\\
	x \odot y = x + y - \dfrac{xy}{6}.
\end{align*}
Quais dos seguintes subconjuntos s\~ao suban\'eis?
\begin{enumerate}[label=({\alph*})]
	\item $A = \z$
	\item $B = \{2k \mid k \in \z\}$
	\item $C = \{6k \mid k \in \z\}$
	\item $D = \{3k \mid k \in \z\}$
\end{enumerate}


\vspace{.3cm}

\questao{} Quais dos conjuntos abaixo s\~ao suban\'eis de $M_2(\real)$?
\begin{align*}
	L_1 &= \left\{\begin{pmatrix}
		a & 0\\
		b & 0
	\end{pmatrix} \mid a, b \in \real\right\}\\
	L_2 &= \left\{\begin{pmatrix}
		a & b\\
		0 & c
	\end{pmatrix} \mid a, b, c \in \real\right\}\\
	L_3 &= \left\{\begin{pmatrix}
		a & 0\\
		0 & b
	\end{pmatrix} \mid a, b \in \real\right\}\\
	L_4 &= \left\{\begin{pmatrix}
		0 & a\\
		c & b
	\end{pmatrix} \mid a, b, c \in \real\right\}
\end{align*}


\vspace{.3cm}

\questao{} Determine todos os suban\'eis do anel $(\z_{16}, \oplus, \otimes)$.


\vspace{.3cm}

\questao{} Mostre que a interse\c{c}\~ao de dois suban\'eis de um anel $A$ \'e ainda um subanel de $A$.

\vspace{.3cm}

\questao{} Seja $(A, + , \cdot)$ um anel e $x \in A$ fixo. Mostre que o conjunto
	\[
		N(x) = \{y \in A \mid xy = yx\}
	\]
	\'e um subanel de $A$.


\questao{} Verifique se $L = \{ a + b\sqrt{2} \mid a, b \in \rac\}$ {\'e} um subanel
do anel $\mathbb{R}$.

\vspace{.3cm}

\questao{inicioreferencia}{} Verificar se a fun\c{c}\~ao $f : A \to B$ \'e ou n\~ao um homomorfismo do anel $A$ no anel $B$, nos seguintes casos:
\begin{enumerate}[label=({\alph*})]
\item $A = \z$, $B = \z$ e $f(x) = x + 1$
\item $A = \z$, $B = \z$ e $f(x) = 2x$
\item $A = \z$, $B = \z \times \z$ e $f(x) = (x, 0)$
\item $A = \z \times \z$, $B = \z$ e $f(x,y) = x$
\item $A = \z \times \z$, $B = \z \times \z$ e $f(x,y) = (y,x)$
\item $A = \z$, $B = \z_n$ e $f(x) = \overline{x}$
\item $A = \complex$, $B = \complex$ e $f(a + bi) = a - bi$
\item $A = M_2(\real)$, $B = \real$ e $f\left(\begin{bmatrix}
	x & y\\z & t
\end{bmatrix}\right) = x$
\item $A = M_2(\real)$, $B = \real$ e $f\left(\begin{bmatrix}
	x & y\\z & t
\end{bmatrix}\right) = x + t$
\end{enumerate}


\vspace{.3cm}

\questao{} Prove que $f : \rac \to M_3(\rac)$ dada por
\[
	f(x) = \begin{pmatrix}
		x & 0 & 0\\
		0 & x & 0\\
		0 & 0 & x
	\end{pmatrix}
\]
\'e um homomorfismo de an\'eis.

\vspace{.3cm}

\questao{fimreferencia} Considere os an{\'e}is $\z$ e $\z\times \z$. Verifique se as seguintes fun\c{c}\~oes s{\~a}o homomorfismos:
\begin{enumerate}[label=({\alph*})]
\item $f : \z\times\z \to \z\times\z$ dado por $f(x,y) = (0,y)$
\item $f : \z\times\z \to \z$ dado por $f(x,y) = y$
\item $f : \z\to \z\times\z$ dado por $f(x) = (2x,0)$
\item $f : \z\times\z \to \z\times\z$ dado por $f(x,y) = (-y,-x)$
\item $f : \z \to \z\times\z$ dado por $f(x) = (0,x)$
\end{enumerate}

\vspace{.3cm}

\questao{} Determine o kernel dos homomorfismos dos \textbf{Exerc{\'i}cios de \ref{inicioreferencia} a \ref{fimreferencia}}.

\vspace{.3cm}

\questao{} Nos \textbf{Exerc{\'\i}cios de \ref{inicioreferencia} a \ref{fimreferencia}} para as fun\c{c}\~oes que forem homomorfismos determine se elas tamb\'em s\~ao isomorfismos.

\vspace{.3cm}

\questao{} Seja $f : \complex \to M_2(\real)$ dada por
\[
	f(a + bi) = \begin{bmatrix}
		a & -b\\
		b & a
	\end{bmatrix}.
\]
\begin{enumerate}[label=({\alph*})]
	\item Mostre que $f$ \'e um homomorfismo de an\'eis.
	\item Esse homomorfismo \'e injetor?
	\item \'E sobrejetor?
\end{enumerate}

\vspace{.3cm}

\questao{} Seja $f: A \to B$ um homomorfismo de an{\'e}is. Mostre que:
\begin{enumerate}[label=({\alph*})]
\item Se $C$  {\'e} um subanel de $A$, ent{\~a}o $f(C)$ {\'e} um subanel de $B$.
\item Se $D$ {\'e} um subanel de $B$, ent{\~a}o $f^{-1}(D)$ {\'e} um subanel de $A$.
\end{enumerate}

\vspace{.3cm}

\questao{} D{\^e} um exemplo de an{\'e}is $A$ e $B$ e um homomorfismo $f : A \to B$ tal que $f(1_A) \ne 1_B$.

\vspace{.3cm}

\questao{} Sejam os an{\'e}is $A = \{ a + b\sqrt{-2} \mid a,\ b \in \rac\}$ e $B = M_2(\rac)$.
\begin{enumerate}[label=({\alph*})]
\item Mostre que $f : A \to B$ dada por
\[
f(a + b\sqrt{-2}) =
\begin{pmatrix}
a & -2b\\
b & a
\end{pmatrix}
\]
{\'e} um homomorfismo.
\item $f$ {\'e} um isomorfimo?
\end{enumerate}

\vspace{.3cm}

\questao{} {\'E} verdadeiro ou falso: $\z$ e $\z_{m}$ para $m > 1$ s{\~a}o an{\'e}is
isomorfos.

\vspace{.3cm}

\questao{} Considere os seguintes an{\'e}is: $(\real, +, \cdot)$ e $(\real, \oplus, \odot)$, sendo $a \oplus b = a + b + 1$ e $a \odot b = a + b + ab$. Mostre que $f : \real \to \real$ dado por $f(x) = x + 1$, para todo $x \in \real$, {\'e} um isomorfimos de $(\real, \oplus, \odot)$ em $(\real, +, \cdot)$.

\end{document}