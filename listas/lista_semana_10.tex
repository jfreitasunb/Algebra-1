%!TEX program = xelatex

\def\numerosemana{10}

\documentclass[12pt]{exam}

\def\ano{2022}
\def\semestre{1}
\def\disciplina{\'Algebra 1}
\def\turma{2}

\usepackage{caption}
\usepackage{amssymb}
\usepackage{amsmath,amsfonts,amsthm,amstext}
\usepackage[brazil]{babel}
% \usepackage[latin1]{inputenc}
\usepackage{graphicx}
\graphicspath{{/ArquivosLinux/OneDrive/imagens-latex/}{D:/OneDrive - unb.br/imagens-latex/}}
\usepackage{enumitem}
\usepackage{multicol}
\usepackage{answers}
\usepackage{tikz,ifthen}
\usetikzlibrary{lindenmayersystems}
\usetikzlibrary[shadings]
\Newassociation{solucao}{Solution}{ans}
\newtheorem{exercicio}{}

\setlength{\topmargin}{-1.0in}
\setlength{\oddsidemargin}{0in}
\setlength{\textheight}{10.1in}
\setlength{\textwidth}{6.5in}
\setlength{\baselineskip}{12mm}

\extraheadheight{0.7in}
\firstpageheadrule
\runningheadrule
\lhead{
        \begin{minipage}[c]{1.7cm}
        \includegraphics[width=1.7cm]{unb.pdf}
        \end{minipage}%
        \hspace{0pt}
        \begin{minipage}[c]{4in}
          {Universidade de Brasília} --
          {Departamento de Matemática}
\end{minipage}
\vspace*{-0.8cm}
}
% \chead{Universidade de Brasília - Departamento de Matemática}
% \rhead{}
% \vspace*{-2cm}

\extrafootheight{.5in}
\footrule
\lfoot{\disciplina\ - \semestre$^o$/\ano\ - Módulo \numeromodulo}
\cfoot{}
\rfoot{Página \thepage\ de \numpages}

\newcounter{exercicios}
\renewcommand{\theexercicios}{\arabic{exercicios}}

\newenvironment{questao}[1]{
\refstepcounter{exercicios}
\ifx&#1&
\else
   \label{#1}
\fi
\noindent\textbf{Exercício {\theexercicios}:}
}

\newcommand{\resp}[1]{
\noindent{\bf Exercício #1: }}

\def\ano{2024}
\def\semestre{1}
\def\disciplina{Álgebra 1}
\def\nomeabreviado{Álgebra 1}
\def\turma{1}

\newcommand{\im}{{\rm Im\,}}
\newcommand{\dlim}[2]{\displaystyle\lim_{#1\rightarrow #2}}
\newcommand{\minf}{+\infty}
\newcommand{\ninf}{-\infty}
\newcommand{\cp}[1]{\mathbb{#1}}
\newcommand{\sub}{\subseteq}
\newcommand{\n}{\mathbb{N}}
\newcommand{\z}{\mathbb{Z}}
\newcommand{\rac}{\mathbb{Q}}
\newcommand{\real}{\mathbb{R}}
\newcommand{\complex}{\mathbb{C}}

\newcommand{\vesp}[1]{\vspace{ #1  cm}}

\newcommand{\compcent}[1]{\vcenter{\hbox{$#1\circ$}}}
\newcommand{\comp}{\mathbin{\mathchoice
        {\compcent\scriptstyle}{\compcent\scriptstyle}
        {\compcent\scriptscriptstyle}{\compcent\scriptscriptstyle}}}
\renewcommand{\sin}{{\rm sen\,}}
\renewcommand{\tan}{{\rm tg\,}}
\renewcommand{\csc}{{\rm cossec\,}}
\renewcommand{\cot}{{\rm cotg\,}}
\renewcommand{\sinh}{{\rm senh\,}}

\begin{document}

\begin{center}

    {\Large\bf \disciplina\ - Turma \turma\ -- \semestre$^{o}$/\ano} \\ \vspace{9pt} {\large\bf
        Lista de Exerc{\'\i}cios -- Semana \numerosemana}\\ \vspace{9pt} Prof. Jos{\'e} Ant{\^o}nio O. Freitas
    \end{center}
    \hrule

    \vspace{.6cm}

    \questao{} Verifique se os seguintes conjuntos com a opera\c{c}\~ao dada \'e ou n\~ao um grupo. Em caso afirmativo, o grupo \'e comutativo?
    \begin{enumerate}[label=({\alph*})]
        \item $(\z, \star)$, onde $x \star y = x + xy$, para $x$, $y \in \z$;

        \item $(\z, \star)$, onde $x \star y = x + y + xy$, para $x$, $y \in \z$;

        \item $(\z, \star)$, onde $x \star y = xy + 2x$, para $x$, $y \in \z$;

        \item $(\rac, \star)$, onde $x \star y = x + xy$, para $x$, $y \in \rac$;

        \item $(\real^*, \star)$, onde $x \star y = \dfrac{x}{y}$, para $x$, $y \in \real$;

        \item $(\real_+, \star)$, onde $x \star y = \sqrt{x^2 + y^2}$, para $x$, $y \in \real_+$;

        \item $(\real, \star)$, onde $x \star y = \sqrt[3]{x^3 + y^3}$, para $x$, $y \in \real$.

        \item $(G, \cdot)$, onde $G = \{x \in \rac \mid x > 0\}$ e $\cdot$ \'e a multiplica\c{c}\~ao de n\'umeros racionais.

        \item $(G, \cdot)$, onde $G = \left\{\dfrac{1 + 2m}{1 + 2n} \mid m, n \in \z\right\}$ e $\cdot$ \'e a multiplica\c{c}\~ao de n\'umeros racionais.

        \item $(G, +)$, onde $G = \{0, \pm 2, \pm 4, \pm 6, \dots\}$ e $+$ \'e a soma de n\'umeros inteiros.

        \item $(G, \star)$, onde $G = \{0, \pm 2, \pm 4, \pm 6, \dots\}$ e $\star$ \'e definida como $x \star y = x + y - xy$.

        \item $(G, +)$, onde $G = \{a + b\sqrt{2} \mid a, b \in \rac\}$ e $+$ \'e a soma de n\'umeros reais.

        \item $(G, \cdot)$, onde $G = \{a + b\sqrt{2} \in \real^* \mid a, b \in \rac\}$ e $\cdot$ \'e a multipli\c{c}\~ao de n\'umeros reais.

        \item $(G, +)$, onde $G = \{a + b\sqrt[3]{2} \mid a, b \in \rac\}$ e $+$ \'e a soma de n\'umeros reais.

        \item $(G, \cdot)$, onde $G = \{a + b\sqrt[3]{2} \in \real^* \mid a, b \in \rac\}$ e $\cdot$ \'e a multipli\c{c}\~ao de n\'umeros reais..
    \end{enumerate}

    \vspace{.3cm}

    \questao{} Seja
    \[
        \complex = \{a + bi \mid a, b \in \real\}
    \]
    e $i^2 = -1$. Mostre que:
    \begin{enumerate}[label=({\alph*})]
        \item $(\complex, +)$ \'e um grupo abeliano, onde
        \[
            (a + bi) + (c + di) = (a + c) + (b + d)i
        \]
        para $a + bi$, $c + di \in \complex$.
        \item Para $\complex^* = \complex - \{0\}$, $(\complex^*, \cdot)$ \'e um grupo abeliano, onde
        \[
            (a + bi)\cdot (c + di) = (ac - bd) + (ad + bc)i
        \]
        para $a + bi$, $c + di \in \complex$.
    \end{enumerate}

    \vspace{.3cm}

    \questao{} Verifique se o conjunto $\rac_{>0}$ dos n{\'u}meros racionais estritamente positivos com a
     opera{\c c}{\~a}o dada {\'e} ou n{\~a}o um grupo. Justifique sua
    resposta.
    \begin{multicols}{2}
        \begin{enumerate}[label=({\alph*})]
            \item $(\rac_{>0},\cdot)$

            \item $(\rac_{>0}, +)$
        \end{enumerate}
    \end{multicols}

    \vspace{.3cm}

    \questao{} Seja $z  = a + bi \in \mathbb{C}$, onde $a$, $b \in \real$. Definimos $|z| = \sqrt{a^2 + b^2}$. Prove que $G=\{z \in \mathbb{C} \mid |z| = 1\}$ {\'e} um grupo
    abeliano com a opera{\c c}{\~a}o de multiplica{\c c}{\~a}o de n{\'u}meros complexos.

    \vspace{.3cm}

    \questao{} Mostre que o conjunto $\rac[\sqrt{2}]^*=\{ a + b\sqrt{2} \in
    \mathbb{R}^* \mid  a, b \in \rac \}$ {\'e} um grupo multiplicativo abeliano.

    \vspace{.3cm}

    \questao{} No conjunto $\z \times \z$ considere a opera\c{c}\~ao de soma definida por
    \[
        (x, y) + (z, t) = (x + z, y + t)
    \]
    para $(x, y)$, $(z, t) \in \z \times \z$. Mostre que $(\z\times\z, +)$ \'e um grupo abeliano.

    \vspace{.3cm}

    \questao{} Considere o seguinte conjunto
    \[
        G = \left\{\begin{pmatrix} 1 & a & b\\ 0 & 1 & c \\ 0 & 0 & 1\end{pmatrix} \mid a, b, c \in \real\right\}.
    \]
    Mostre que $(G, \cdot)$, onde $\cdot$ é a multiplicação de matrizes, é um grupo. Esse grupo é comutativo?

    \vspace{.3cm}

    \questao{} Seja $p \ge 2$, um número primo. Mostre que o conjunto
    \[
        G = \left\{
                    \begin{pmatrix}
                        \overline{1} & \overline{a} & -\overline{a} & \overline{b}\\
                        \overline{0} & \overline{1} & \overline{0} & \overline{b}\\
                        \overline{0} & \overline{0} & \overline{1} & \overline{b}\\
                        \overline{0} & \overline{0} & \overline{0} & \overline{1}\\
                    \end{pmatrix}
                \mid \overline{a}, \overline{b} \in \z_p
            \right\}
    \]
    é um grupo com a multiplicação de matrizes. Esse grupo é comutativo?
    \vspace{.3cm}

    \questao{} Quais dos seguintes subconjuntos $G$ de $\z_{13}$ s{\~a}o grupos
    com a opera{\c c}{\~a}o de multiplica{\c c}{\~a}o?
    \begin{multicols}{2}
        \begin{enumerate}[label=({\alph*})]
            \item $G=\{\overline{1},\overline{12}\}$;

            \item $G=\{\overline{1},\overline{5},\overline{8},\overline{12}\}$;

            \item $G=\{\overline{1},\overline{2},\overline{3},\overline{4}, \overline{5},\overline{6},\overline{7}, \overline{8},\overline{9},\overline{10},\overline{11},\overline{12}\}$

            \item $G=\{\overline{1}, \overline{3},\overline{5},\overline{7},\overline{9},\overline{11}\}$.
        \end{enumerate}
    \end{multicols}

    \vspace{.3cm}

    %\questao{} Determine $f$, $g \in S_3$ tais que:
   % \begin{enumerate}[label=({\alph*})]
   %     \item $(f \comp g)^3 \ne f^3\comp g^3$

%        \item $(f \comp g)^2 \ne f^2\comp g^2$
%    \end{enumerate}

%    \vspace{.3cm}

%    \questao{} Determine $f$, $g \in S_4$ tais que:
%    \begin{enumerate}[label=({\alph*})]
%        \item $(f \comp g)^4 \ne f^4\comp g^4$

%        \item $(f \comp g)^3 \ne f^3\comp g^3$
%    \end{enumerate}

%    \vspace{.3cm}

%    \questao{} Considere o grupo $S_3$:
%    \begin{enumerate}[label=({\alph*})]
%        \item Determine todos os elementos $f \in S_3$ tais que $f^2 = Id$ e $f \ne Id$.

%        \item Determine todos os elementos $g \in S_3$ tais que $g^3 = Id$ e $g \ne Id$.
%    \end{enumerate}

%    \vspace{.3cm}

%    \questao{} Considere o grupo $S_4$:
%    \begin{enumerate}[label=({\alph*})]
%        \item Determine todos os elementos $f \in S_4$ tais que $f^2 = Id$ e $f \ne Id$.

%        \item Determine todos os elementos $g \in S_4$ tais que $g^3 = Id$ e $g \ne Id$.

%        \item Determine todos os elementos $g \in S_4$ tais que $g^4 = Id$ e $g \ne Id$.
%    \end{enumerate}

%    \vspace{.3cm}

%    \questao{} Seja $V = \{1, f, g, h\}$ o seguinte subconjunto do grupo $S_4$:
%    \begin{align*}
%        1 = \begin{pmatrix}
%            1 & 2 & 3 & 4\\
%            1 & 2 & 3 & 4
%        \end{pmatrix}; \quad f = \begin{pmatrix}
%            1 & 2 & 3 & 4\\
%            2 & 1 & 4 & 3
%        \end{pmatrix}\\
%        g = \begin{pmatrix}
%            1 & 2 & 3 & 4\\
%            3 & 4 & 1 & 2
%        \end{pmatrix}; \quad h = \begin{pmatrix}
%            1 & 2 & 3 & 4\\
%            4 & 3 & 2 & 1
%        \end{pmatrix}.
%    \end{align*}
%    \begin{enumerate}[label=({\alph*})]
%        \item Prove que $(V, \comp)$ \'e um grupo contendo 4 elementos, onde $\comp$ \'e a opera\c{c}\~ao de $S_4$.

%        \item Prove que $(V, \comp)$ \'e um grupo abeliano.
%    \end{enumerate}

%    \vspace{.3cm}

%    \questao{} Considere o grupo $S_7$ e sejam
%    \begin{align*}
%        1 &= \begin{pmatrix}
%            1 & 2 & 3 & 4 & 5 & 6 & 7\\
%            1 & 2 & 3 & 4 & 5 & 6 & 7
%        \end{pmatrix}\\
%        \sigma &= \begin{pmatrix}
%                1 & 2 & 3 & 4 & 5 & 6 & 7\\
%                3 & 4 & 2 & 6 & 7 & 5 & 1
%            \end{pmatrix}.\\
%        \beta &= \begin{pmatrix}
%                1 & 2 & 3 & 4 & 5 & 6 & 7\\
%                3 & 1 & 2 & 6 & 7 & 4 & 5
%            \end{pmatrix}.
%    \end{align*}.
%    \begin{enumerate}[label=({\alph*})]
%        \item Encontre o menor $l \ge 0$ tal que $\sigma^l = 1$.

%        \item Encontre $\delta \in S_7$ tal que $\sigma\comp\delta = 1$.

%        \item Encontre o menor $k \ge 0$ tal que $\beta^k = 1$.

%        \item Encontre $\gamma \in S_7$ tal que $\gamma\comp\beta = 1$.
%    \end{enumerate}

    \vspace{.3cm}

    \questao{} Seja $(G,*)$ um grupo. Mostre que:
    \begin{enumerate}[label={\alph*})]
        \item O elemento neutro de $G$ {\'e} {\'u}nico.

        \item Existe um {\'u}nico inverso para cada $x \in G$.

        \item Para todo $x \in G$, $(x^{-1})^{-1} = x$.
    \end{enumerate}

    \vspace{.3cm}

    \questao{} Seja $(G,*)$ um grupo com elemento neutro $e$. Para $x\in
    G$, considere a nota{\c c}{\~a}o $x^n=x*x*\cdots *x$ ($n$ vezes).
    \begin{enumerate}[label=({\alph*})]
        \item Prove que se
        $x^2 = e$, para todo $x\in G$, ent{\~a}o $G$ {\'e} um grupo abeliano.

        \item Mostre que se $x\in G$ {\'e} tal que $x^2 = x$, ent{\~a}o $x$ {\'e} o elemento neutro.
    \end{enumerate}

    \vspace{.3cm}

    \questao{} Sejam $G$ um grupo e $x$, $y$, $z \in G$. Prove que:
    \begin{enumerate}[label=({\alph*})]
        \item Se $xy = xz$, ent\~ao $y = z$.

        \item Se $yx = zx$, ent\~ao $y = z$.
    \end{enumerate}

    \vspace{.3cm}

    \questao{} Na tabela abaixo encontra-se representada a opera\c{c}\~ao $\cdot$ definida no conjunto $G = \{e, a , b, c, d, f\}$ de tal modo que $(G, \cdot)$ \'e um grupo:
    \[
        \begin{tabular}{|c|c|c|c|c|c|c|c|}
            \hline
            $\cdot$ & e & a & b & c & d & f\\
            \hline
            e & e & a & b & c & d & f\\
            \hline
            a & a & b & e & f & c & d\\
            \hline
            b & b & e & a & d & f & c\\
            \hline\
            c & c & d & f & e & a & b\\
            \hline
            d & d & f & c & b & e & a\\
            \hline
            f & f & c & d & a & b & e\\
            \hline
        \end{tabular}
    \]
    \begin{enumerate}[label=({\alph*})]
        \item Esse grupo \'e comutativo?

        \item Determine $x \in G$ de maneira que
        \[
            d^{-1}\cdot f \cdot x \cdot b = (a \cdot b \cdot c)^{-1}.
        \]
    \end{enumerate}

    \vspace{.3cm}

    \questao{} Na tabela abaixo encontra-se representada a opera\c{c}\~ao $\cdot$ definida no conjunto $G = \{e, a , b, c, d, f\}$ de tal modo que $(G, \cdot)$ \'e um grupo:
    \[
        \begin{tabular}{|c|c|c|c|c|c|c|c|}
            \hline
            $\cdot$ & e & a & b & c & d & f\\
            \hline
            e & e & a & b & c & d & f\\
            \hline
            a & a & b & c & d & f & e\\
            \hline
            b & b & c & d & f & e & a\\
            \hline\
            c & c & d & f & e & a & b\\
            \hline
            d & d & f & e & a & b & c\\
            \hline
            f & f & e & a & b & c & d\\
            \hline
        \end{tabular}
    \]
    \begin{enumerate}[label=({\alph*})]
        \item Esse grupo \'e comutativo?

        \item Determine $x \in G$ de maneira que
        \[
            (c \cdot d)^{-1}\cdot x^{-1} \cdot b  \cdot a^{-1} = f^{-1}.
        \]
    \end{enumerate}

    \vspace{.3cm}

    \questao{} Sejam $(G, \cdot)$ um grupo e $a$, $b \in G$. Determine $x \in G$, em termos de $a$ e $b$, tal que
    \[
        x\cdot a \cdot x = b \cdot b \cdot a^{-1}.
    \]

    \vspace{.3cm}

    \questao{} Sejam $(G, \cdot)$ um grupo e $a$, $b \in G$. Suponha que $a \cdot b = e$, onde $e$ \'e o elemento neutro de $G$. Prove que $b \cdot a = e$.

    \vspace{.3cm}

    \questao{} Sejam $(G, \cdot)$ um grupo e $a$, $b \in G$. Suponha que $a \cdot b \cdot a \cdot b = e$, onde $e$ \'e o elemento neutro de $G$. Prove que $b \cdot a \cdot b \cdot a= e$.

    \vspace{.3cm}

    \questao{} Verifique se s\~ao subgrupos:
    \begin{enumerate}[label=({\alph*})]
      \item $H = \{x \in \rac \mid x > 0\}$ de $(\rac^*,\cdot)$.

      \item $H = \left\{\dfrac{1 + 2m}{1 + 2n} \mid m, n \in \z\right\}$ de $(\rac^*,\cdot)$.

      \item $H = \{\cos\theta + i\sin\theta \mid \theta \in \rac\}$ de $(\complex^*,\cdot)$.

      \item $H = \{0, \pm 2, \pm 4, \pm 6, \dots\}$ de $(\z,+)$.

      \item $H = \{0, \pm 2, \pm 4, \pm 6, \dots\}$ do grupo $(\rac - \{1\},\star)$ onde $\star$ \'e definida como $x \star y = x + y - xy$.

      \item $H = \{a + b\sqrt{2} \mid a, b \in \rac\}$ de $(\real,+)$.

      \item $H = \{a + b\sqrt{2} \in \real^* \mid a, b \in \rac\}$ de $(\real^*,\cdot)$.

      \item $H = \{a + b\sqrt[3]{2} \mid a, b \in \rac\}$ de $(\real,+)$.

      \item $H = \{a + b\sqrt[3]{2} \in \real^* \mid a, b \in \rac\}$ de $(\real^*,\cdot)$.
    \end{enumerate}

    \vspace{.3cm}

    \questao{} Determine todos os subgrupos do grupo aditivo $\z_4$.

    \vspace{.3cm}

    \questao{} Seja
    \[
      GL_2(\real) = \left\{ \begin{pmatrix}
          x & y\\z & t
      \end{pmatrix} \mid x, y, z, t \in \real,\ \det(A) \ne 0\right\}.
    \]
    \begin{enumerate}[label=({\alph*})]
      \item Mostre que $GL_2(\real)$ com a opera\c{c}\~ao de multiplica\c{c}\~ao de matrizes \'e um grupo. Esse grupo \'e abeliano?

      \item Seja
      \[
          H = \left\{ \begin{pmatrix}
              \cos a & \sin a\\ - \sin a & \cos a
          \end{pmatrix} \mid a \in \real\right\}.
      \]
      Mostre que $H$ \'e um subgrupo de $GL_2(\real)$.

      \item Seja
      \[
          K = \left\{ \begin{pmatrix}
              a & b\\ -b & a
          \end{pmatrix} \mid a, b \in \real \mbox{ e n\~ao nulos simultaneamente}\right\}.
      \]
      Mostre que $K$ \'e um subgrupo de $GL_2(\real)$.
    \end{enumerate}

    \questao{} Sejam $H$ e $K$ subgrupos de um grupo $G$ (com nota{\c c}{\~a}o
    multiplicativa).
    \begin{enumerate}[label=({\alph*})]
      \item Mostre que $H\cap K$ tamb{\'e}m {\'e} subgrupo de $G$.

      \item Seja $g\in G$ um elemento fixado. Mostre que o conjunto
      $g^{-1}Hg=\{ g^{-1}xg \mid x\in H \} $ {\'e} um subgrupo de $G$.

      \item Prove que $H\cup K$ {\'e} subgrupo de $G$ se, e somente se,
      $H\subseteq K$ ou $K\subseteq H$.

      \item Demonstre que $HK=\{hk \mid h\in H, k\in K\}$ {\'e} subgrupo
      de $G$ se, e somente se, $HK=KH$.

      [\emph{Nota: $HK=KH$ \textbf{n{\~a}o} quer dizer que $hk=kh$,
      para todo $h\in H, k\in K$; significa que $hk=k_1h_1 \in KH$ e $kh=h_2k_2 \in
      HK$, para todo $h\in H, k\in K$.}]
    \end{enumerate}

    \vspace{.3cm}
    \questao{} Seja $G$ um grupo com nota\c{c}\~ao multiplicativa e $a$ um elemento de $G$. Prove que $N(a) = \{x \in G \mid ax = xa\}$ \'e um subgrupo de $G$.

    \vspace{.3cm}

    \questao{} Seja $G$ um grupo com nota\c{c}\~ao multiplicativa. Considere o subconjunto $Z(G) = \{x \in G \mid xh = hx, \mbox{ para todo } h \in G\}$. Mostre que:
    \begin{enumerate}[label=({\alph*})]
      \item $Z(G)$ \'e um subgrupo de $G$.

      \item $G$ \'e abeliano se, e somente se, $Z(G) = G$.
    \end{enumerate}

    \vspace{.3cm}

    \questao{} Seja $G$ um grupo. Dado $H \subset G$ um subgrupo defina
    \begin{align}\label{relacao_equivalencia_subgrupo}
        x \cong y \mbox{ se, e somente se, } xy^{-1} \in H
    \end{align}
    para todos $x$, $y \in G$.
    \begin{enumerate}[label={\alph*})]
        \item Mostre que a rela\c{c}\~ao definida em \eqref{relacao_equivalencia_subgrupo} \'e uma rela\c{c}\~ao de equival\^encia.

        \item Se $y \in G$, ent\~ao a classe de equival\^encia determinada por $y$ \'e o conjunto
        \begin{align*}\label{classe_equivalencia_subgrupo}
            Hy = \{ty \mid t \in H\}.
        \end{align*}
    \end{enumerate}

    \vspace{.3cm}

    \questao{} Seja $H$ um subgrupo de $G$. Denote por
    \begin{align*}
        \mathcal{P} &= \{aH \mid a \in G\}\\
        \mathcal{Q} &= \{Hb \mid b \in G\}.
    \end{align*}
    Mostre que a fun\c{c}\~ao $f : \mathcal{P} \to \mathcal{Q}$ dada por $f(aH) = Ha^{-1}$ \'e uma bije\c{c}\~ao.

    \vspace{.3cm}

    \questao{} Considere o conjunto $\z_{24}$. Defina
    \[
        G = \{ \overline{a} \in \z_{24}^* \mid \mbox{ existe } \overline{b} \in \z_{24} \mbox{ tal que } \overline{a}\overline{b} = \overline{1}\}.
    \]
    \begin{enumerate}[label={\alph*})]
        \item  Mostre que $G$ é um grupo com a multiplicação de $\z_{24}$.

        \item Encontre todos os subgrupos de $G$.
    \end{enumerate}

    \vspace{.3cm}

    \questao{} Considere o conjunto $\z_{20}$. Defina
    \[
        G = \{ \overline{a} \in \z_{20}^* \mid \mbox{ existe } \overline{b} \in \z_{20} \mbox{ tal que } \overline{a}\overline{b} = \overline{1}\}.
    \]
    \begin{enumerate}[label={\alph*})]
        \item  Mostre que $G$ é um grupo com a multiplicação de $\z_{20}$.

        \item Encontre todos os subgrupos de $G$.
    \end{enumerate}
\end{document}
