%!TEX program = xelatex
%!TEX encoding = ISO-8859-1
\documentclass[12pt]{article}

\usepackage{amssymb}
\usepackage{amsmath,amsfonts,amsthm,amstext,mathabx}
\usepackage[brazil]{babel}
%\usepackage[latin1]{inputenc}
\usepackage{graphicx}
\graphicspath{{/home/jfreitas/Dropbox/imagens-latex/}{/Volumes/Vader/Dropbox/imagens-latex/}{D:/Dropbox/imagens-latex/}}
\usepackage{enumitem}
\usepackage{multicol}
\usepackage[all]{xy}

\setlength{\topmargin}{-1.0in}
\setlength{\oddsidemargin}{0in}
\setlength{\textheight}{10.1in}
\setlength{\textwidth}{6.5in}
\setlength{\baselineskip}{12mm}

\newcounter{exercicios}
\setcounter{exercicios}{0}
\newcommand{\questao}{
\addtocounter{exercicios}{1}
\noindent{\bf Exerc{\'\i}cio \arabic{exercicios}: }}

\newcommand{\equi}{\Leftrightarrow}
\newcommand{\bic}{\leftrightarrow}
\newcommand{\cond}{\rightarrow}
\newcommand{\impl}{\Rightarrow}
\newcommand{\nao}{\sim}
\newcommand{\sub}{\subseteq}
\newcommand{\e}{\ \wedge\ }
\newcommand{\ou}{\ \vee\ }
\newcommand{\vaz}{\emptyset}
\newcommand{\nsub}{\nsubset}
\renewcommand{\sin}{{\rm sen\,}}

\newcommand{\n}{\mathbb{N}}
\newcommand{\z}{\mathbb{Z}}
\newcommand{\real}{\mathbb{R}}
\newcommand{\vesp}{\vspace{0.2cm}}
\newcommand{\subne}{\subsetneqq}


\newcommand{\compcent}[1]{\vcenter{\hbox{$#1\circ$}}}
\newcommand{\comp}{\mathbin{\mathchoice
{\compcent\scriptstyle}{\compcent\scriptstyle}
{\compcent\scriptscriptstyle}{\compcent\scriptscriptstyle}}}

\begin{document}

\pagestyle{empty}

\begin{figure}[h]
        \begin{minipage}[c]{1.7cm}
        \includegraphics[width=1.7cm]{unb.pdf}
        \end{minipage}%
        \hspace{0pt}
        \begin{minipage}[c]{4in}
          {Universidade de Brasília} \\
          {Departamento de Matemática}
\end{minipage}
\end{figure}
\vspace{-1cm}\hrule


\begin{center}
{\Large\bf {\'A}lgebra 1 - Turma D -- 2$^{o}$/2017} \\ \vspace{9pt} {\large\bf
  $3^{\underline{a}}$ Lista de Exerc{\'\i}cios -- Operações em $\z_m$}\\
\vspace{9pt} Prof. Jos{\'e} Ant{\^o}nio O. Freitas
\end{center}
\hrule

\vspace{.6cm}

\questao Construa a tabela de soma, $\oplus$, e multiplicação, $\otimes$, nos seguintes conjuntos:
\begin{multicols}{2}
	\begin{enumerate}[label=({\alph*})]
		\item $\z_5$
		\item $\z_8$
		\item $\z_9$
		\item $\z_{11}$
	\end{enumerate}
\end{multicols}

\vesp

\questao Considere os seguintes subconjuntos $G$ de $\z_{12}$:
\begin{multicols}{2}
	\begin{enumerate}[label=({\alph*})]
		\item $G=\{\overline{1},\overline{11}\}$;

		\item $G=\{\overline{0},\overline{4},\overline{8}\}$;

		\item $G=\{\overline{0},\overline{2},\overline{4},\overline{6},\overline{8},\overline{10}\}$
		\item $G=\{\overline{1}, \overline{3},\overline{5},\overline{7},\overline{9},\overline{11}\}$.
	\end{enumerate}
\end{multicols}

Para quais desses conjuntos vale as duas propriedades seguintes:
\begin{enumerate}[label=({\roman*})]
	\item $x \oplus y \in G$, para todos $x$, $y \in G$;
	\item Para todo $x \in G$, existe $y \in G$ tal que $x \oplus y = \overline{0}$.
\end{enumerate}

\vesp

\questao Considere os seguintes subconjuntos $G$ de $\z_{13}$:
\begin{multicols}{2}
	\begin{enumerate}[label=({\alph*})]
		\item $G=\{\overline{1},\overline{12}\}$;

		\item $G=\{\overline{1},\overline{5},\overline{8},\overline{12}\}$;

		\item $G=\{\overline{1},\overline{2},\overline{3},\overline{4}, \overline{5},\overline{6},\overline{7},
		 \overline{8},\overline{9},\overline{10},\overline{11},\overline{12}\}$
		\item $G=\{\overline{1}, \overline{3},\overline{5},\overline{7},\overline{9},\overline{11}\}$.
	\end{enumerate}
\end{multicols}

Para quais desses conjuntos vale as duas propriedades seguintes:
\begin{enumerate}[label=({\roman*})]
	\item $x \otimes y \in G$, para todos $x$, $y \in G$;
	\item Para todo $x \in G$, existe $y \in G$ tal que $x \otimes y = \overline{1}$
\end{enumerate}

\vesp

\questao Resolva as seguintes equa\c{c}\~oes:
\begin{enumerate}[label={\alph*})]
	\item $x \oplus \overline{2} = \overline{0}$ no conjunto $\z_4$;
	\item $x^2 \oplus \overline{2}x = \overline{2}$ no conjunto $\z_5$;
	\item $x^3 \oplus \overline{2}x = \overline{3}$ no conjunto $\z_7$.
\end{enumerate}

\vesp

\questao Resolva o sistema de equa\c{c}\~oes
  \[
    \begin{cases}
      x \oplus y = \overline{0}\\
      \overline{2}x \oplus y = \overline{2}
    \end{cases}
  \]
no conjunto $\z_4$. Este sistema possui solu\c{c}\~ao em $\z_3$?

\vesp

\questao Resolva a equa\c{c}\~ao $x^2 \oplus \overline{2}x \oplus \overline{1} = \overline{0}$ nos conjuntos $\z_7$ e $\z_{11}$, caso ela tenha ra{\'\i}zes.

\vesp

\questao Para quais valores de $\overline{c}$ a equa\c{c}\~ao $\overline{2}\otimes x = \overline{c}$ tem solu\c{c}\~ao no conjunto $\z_5$.

\end{document}