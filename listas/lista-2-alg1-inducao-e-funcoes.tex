%!TEX program = xelatex
%!TEX encoding = ISO-8859-1
\documentclass[12pt]{article}

\usepackage{amssymb}
\usepackage{amsmath,amsfonts,amsthm,amstext,mathabx}
\usepackage[brazil]{babel}
%\usepackage[latin1]{inputenc}
\usepackage{graphicx}
\graphicspath{{/home/jfreitas/Dropbox/imagens-latex/}{/Users/jfreitas/Dropbox/imagens-latex/}{D:/Dropbox/imagens-latex/}}
\usepackage{enumitem}
\usepackage{multicol}
\usepackage[all]{xy}

\setlength{\topmargin}{-1.0in}
\setlength{\oddsidemargin}{0in}
\setlength{\textheight}{10.1in}
\setlength{\textwidth}{6.5in}
\setlength{\baselineskip}{12mm}

\newcounter{exercicios}
\setcounter{exercicios}{0}
\newcommand{\questao}{
\addtocounter{exercicios}{1}
\noindent{\bf Exerc{\'\i}cio \arabic{exercicios}: }}

\newcommand{\equi}{\Leftrightarrow}
\newcommand{\bic}{\leftrightarrow}
\newcommand{\cond}{\rightarrow}
\newcommand{\impl}{\Rightarrow}
\newcommand{\nao}{\sim}
\newcommand{\sub}{\subseteq}
\newcommand{\e}{\ \wedge\ }
\newcommand{\ou}{\ \vee\ }
\newcommand{\vaz}{\emptyset}
\newcommand{\nsub}{\nsubset}
\renewcommand{\sin}{{\rm sen\,}}

\newcommand{\n}{\mathbb{N}}
\newcommand{\z}{\mathbb{Z}}
\newcommand{\real}{\mathbb{R}}
\newcommand{\vesp}{\vspace{0.2cm}}
\newcommand{\subne}{\subsetneqq}


\newcommand{\compcent}[1]{\vcenter{\hbox{$#1\circ$}}}
\newcommand{\comp}{\mathbin{\mathchoice
{\compcent\scriptstyle}{\compcent\scriptstyle}
{\compcent\scriptscriptstyle}{\compcent\scriptscriptstyle}}}

\begin{document}
\pagestyle{empty}

\begin{figure}[h]
        \begin{minipage}[c]{1.7cm}
        \includegraphics[width=1.7cm]{unb.pdf}
        \end{minipage}%
        \hspace{0pt}
        \begin{minipage}[c]{4in}
          {Universidade de Bras{\'\i}lia} \\
          {Departamento de Matem{\'a}tica}
\end{minipage}
\end{figure}
\vspace{-1cm}\hrule

\begin{center}
{\Large\bf {\'A}lgebra 1 - Turma B -- 2$^{o}$/2015} \\ \vspace{9pt} {\large\bf
  $2^{\underline{a}}$ Lista de Exerc{\'\i}cios -- Indução e Relações}\\
\vspace{9pt} Prof. Jos{\'e} Ant{\^o}nio O. Freitas
\end{center}
\hrule

\vspace{.6cm}

\questao Prove por indu{\c c}{\~a}o que:
\begin{enumerate}[label={\alph*})]
\item $1^2 + 2^2 + \cdots + n^2 = \dfrac{n(n + 1)(2n + 1)}{6}$, $n \in \n$.

%\item $(1 + p)^n \ge 1 + np$, $n \in \n$, $p > -1$.

\item $1 + 3 + 5 + \cdots + (2n - 1) = n^2$, $n \ge 1$.

\item $2.1 + 2.2 + 2.3 + \cdots + 2n = n^2 + n$, $n \ge 1$.

\item Qualquer número inteiro positivo $n \ge 8$ pode ser escrito como a soma de $3'$s e $5'$s.

\item $1 + q + q^2 + \cdots + q^n = \dfrac{1 - q^{n + 1}}{1 - q}$, $q \ne
  1$.

\item $3^{2n + 1} + 2^{n + 2}$ {\'e} múltiplo de $7$, para todo $n \in \n$.

\item Para todo $n \ge 0$, $9^n - 1$ {\'e} múltiplo de 8.

\item $1^3 + 2^3 + \cdots + n^3 = \left[\dfrac{n(n+1)}{2}\right]^2$, $n \ge 1$.
\end{enumerate}

\vesp

\questao Mostre que o quadrado de um n{\'u}mero {\'\i}mpar {\'e} da forma $8q + 1$, $q
\in \z$.

\vesp

\questao Sejam $a$, $b \in \z$ tais que $mdc(a,b) = 1$. Se $a \mid c$ e $b \mid
c$, mostre que $ab \mid c$.

\vesp

\questao Use o resultado do exerc{\'\i}cio anterior para provar que $6 \mid n(2n + 7)(7n + 1)$ para todo $n \in \z$.

\vesp

\questao Sejam $a$, $b$, $c \in \z$ tais que $a \mid bc$ e $mdc(a,b) =
1$. Prove que $a \mid c$.

\vesp

\questao Mostre que $a^3 - a$ {\'e} multiplo de 3 para todo $a \in \z$.

\vesp
%\questao Quais das rela{\c c}{\~o}es abaixo s{\~a}o rela{\c c}{\~o}es de equival{\^e}ncia sobre $E = \{a,b,c\}$?
%\begin{enumerate}[label={\alph*})]
%\item $R_1 = \{(a,a),(a,b),(b,a),(b,b),(c,c)\}$;
%\item $R_2 = \{(a,a),(a,b),(b,a),(b,b),(b,c)\}$;
%\item $R_3 = \{(a,a),(b,b),(b,c),(c,b),(a,c),(c,a)\}$;
%\item $R_4 = E \times E$;
%\item $R_5 = \vaz$.
%\end{enumerate}
%\questao Determinar todas as rela{\c c}{\~o}es de equival{\^e}ncia
%$R$ sobre $A$ e os respectivos conjuntos quocientes $A/R$ para
%
%\begin{enumerate}[label={\alph*})]
%\item $A=\{a\}$;
%\item $A=\{a,b\}$;
%\item $A=\{a,b,c\}$;
%\item $A=\{a,b,c,d\}$.
%\end{enumerate}
%
%\vesp

%\questao Seja $A$ um conjunto n{\~a}o vazio e $B\subseteq
%A$. Considere $E=\mathcal{P}(A)$, o conjunto das partes de $A$,
%e a seguinte rela{\c c}{\~a}o sobre $E$:
%\[
%X~R~Y \Leftrightarrow X\cap B=Y\cap B.
%\]
%\begin{enumerate}[label={\alph*})]
%\item Mostre que $R$ {\'e} uma rela{\c c}{\~a}o de equival{\^e}ncia sobre $E$.
%\item Descreva a classe de equival{\^e}ncia $\overline{X}$ de $X \in
% E$.
%\end{enumerate}
%
%\vesp
%
%\questao Seja $A=\n\times \n^*$. Considere a seguinte
%rela{\c c}{\~a}o sobre $A$:
%\[
%(a,b)R (c,d) \Leftrightarrow a + b = c + d.
%\]
%Mostre que $R$ {\'e} uma rela{\c c}{\~a}o de equival{\^e}ncia sobre $A$.
%
%\vesp
%
%\questao Seja $A=\z\times \z^*$, onde
%$\mathbb{Z}^*=\mathbb{Z}\setminus \{0\}$. Para $(a,b), (c,d) \in
%A$, considere a seguinte rela{\c c}{\~a}o
%\[
%(a,b)R (c,d) \Leftrightarrow ad=bc.
%\]
%Mostre que $R$ {\'e} uma rela{\c c}{\~a}o de equival{\^e}ncia sobre $A$.
%
%\vesp
%
%\questao Considere a seguinte rela{\c c}{\~a}o sobre $\mathbb{C}$:
%\[
%(x+yi)R(r+si) \Leftrightarrow x^2+y^2=r^2+s^2.
%\]
%
%\begin{enumerate}[label={\alph*})]
%\item Mostre que $R$ {\'e} rela{\c c}{\~a}o de equival{\^e}ncia.
%\item Descreva a classe de equival{\^e}ncia de $1+i$.
%\end{enumerate}
%
%\vesp
%
%\questao Seja $R$ uma rela{\c c}{\~a}o sobre $\mathbb{Q}$
%definida da seguinte forma:
%\[
%xRy \Leftrightarrow x-y \in \mathbb{Z}.
%\]
%\begin{enumerate}[label={\alph*})]
%\item Prove que $R$ {\'e} uma rela{\c c}{\~a}o de equival{\^e}ncia sobre $\mathbb{Q}$.
%\item Descreva a classe $\bar{1}$.
%\item Descreva a classe $\overline{1/2}$.
%\end{enumerate}
%
%\vesp
%
%\questao A divisibilidade (ou seja, a rela{\c c}{\~a}o definida por $xRy$ se, e s{\'o}
%se, $x \mid y$) {\'e} uma rela{\c c}{\~a}o de equival{\^e}ncia sobre $\z$?
%
%\vesp
%
%\questao Seja $R$ a seguinte rela{\c c}{\~a}o sobre $\z^*$:
%\[
%xRy \Leftrightarrow x\mid y \mbox{ e } y\mid x.
%\]
%Mostre que $R$ {\'e} uma rela{\c c}{\~a}o de equival{\^e}ncia sobre $\z^*$ e
%descreva o conjunto quociente $\z^*/R$.
%
%\vesp

\questao Seja $f:E\to F$ uma aplica{\c c}{\~a}o e sejam $A$ e $B$ subconjuntos de $E$. Mostre que
\begin{enumerate}[label={\alph*})]
\item Se $A\subset B$, ent{\~a}o $f(A)\subset f(B)$.
\item $f(A\cup B)=f(A)\cup f(B)$.
\item $f(A\cap B)\subset f(A)\cap f(B)$.
\item Se $f$ {\'e} injetora, ent{\~a}o $f(A\cap B) =  f(A)\cap f(B)$.
\item $f$ {\'e} bijetora se, e somente se, $f(A^C) = (f(A))^C$ para todo $A \sub E$.
\end{enumerate}

\vesp

\questao Seja $f:E\to F$ uma aplica{\c c}{\~a}o e sejam $A\subset
E$ e $X, Y\subset F$. Mostre que
\begin{enumerate}[label={\alph*})]
\item Se $X\subset Y$, ent{\~a}o $f^{-1}(X)\subset f^{-1}(Y)$.
\item $f^{-1}(X\cup Y)=f^{-1}(X)\cup f^{-1}(Y)$.
\item $f^{-1}(X\cap Y)= f^{-1}(X)\cap f^{-1}(Y)$.
\item $A\subset f^{-1}(f(A))$.
\item $f(f^{-1}(X))= X \cap \mbox{Im}f$ e conclua que se $f$ {\'e} sobrejetora ent{\~a}o
$f(f^{-1}(X))=X$.
\end{enumerate}

\vesp

\questao Se as aplica{\c c}{\~o}es $f:E\to F$ e $g: F\to E$ s{\~a}o
tais que $g\circ f$ {\'e} injetora, ent{\~a}o $f$ {\'e} injetora.

\vesp

\questao Se as aplica{\c c}{\~o}es $f:E\to F$ e $g: F\to E$ s{\~a}o
tais que $g\circ f$ {\'e} sobrejetora, ent{\~a}o g {\'e} sobrejetora.

\vesp

\questao Mostrar que toda aplica{\c c}{\~a}o injetora (sobrejetora) de um conjunto finito em si mesmo {\'e} tamb{\'e}m sobrejetora (injetora).

\vesp

\questao Seja $f: \real^2 \to \real$ dada por $f(x,y) = xy$.
\begin{enumerate}[label={\alph*})]
\item $f$ {\'e} injetora?
\item $f$ {\'e} sobrejetora?
\item Obter $f^{-1}({0})$.
\item Obter $f([0,1]\times [0,1])$.
\end{enumerate}

\vesp

\questao Considere a aplica{\c c}{\~a}o $f : \z \times \z \to \z \times \z$ tal que $f(x,y) = (2x + 3, 4y + 5)$. Prove que $f$ {\'e} injetora. Verifique se $f$ {\'e} bijetora.

\vesp

\questao Mostra que a fun{\c c}{\~a}o $f : \real \to \real$ definida por $f(x) = ax + b$, com $a$ e $b$ constantes reais, $a \ne 0$, {\'e} uma bije{\c c}{\~a}o. Obter $f^{-1}$.

\vesp

\questao Mostrar que $f : \real - \left\{-\dfrac{d}{c}\right\} \to \real  - \left\{\dfrac{a}{c}\right\}$ dada por $f(x) =  \dfrac{ax + b}{cx + d}$, onde $a$, $b$, $c$, $d$ s{\~a}o n{\'u}meros reais constantes, $ad - bc \ne 0$, {\'e} uma bije{\c c}{\~a}o. Descrever a aplica{\c c}{\~a}o $f^{-1}$.

\vesp

\questao Achar uma fun{\c c}{\~a}o $f : A \to B$, com $A$ e $B$ subconjuntos de $\real$, para cada caso abaixo:
\begin{enumerate}[label={\alph*})]
\item $A = \real$, $B \subne \real$ e $f$ injetora e n{\~a}o sobrejetora.
\item $A \subne \real$, $B = \real$ e $f$ injetora e n{\~a}o sobrejetora.
\item $A = \real$, $B \subne \real$ e $f$ sobrejetora e n{\~a}o injetora.
\item $A \subne \real$, $B = \real$ e $f$ sobrejetora e n{\~a}o injetora.
\end{enumerate}

\vesp

\questao Classificar (se poss{\'\i}vel) em injetora ou sobrejetora as seguintes fun{\c c}{\~o}es de $\real$ em $\real$.

\begin{enumerate}[label={\alph*})]
\begin{multicols}{2}
\item $y = x^3$
\item $y = x^2 - 5x - 6$
\item $y = 2^x$
\item $y = | \sin x |$
\item $y = x + | x |$
\item $y = x + 3$
\end{multicols}
\end{enumerate}

\vesp

\questao Seja a fun{\c c}{\~a}o $f : \real \to \real$ dada por $f(x) = | x |$. Determinar $f([-1,1])$, $f(]-1,2])$, $f(\real)$, $f^{-1}([0,3])$, $f^{-1}([-1,3])$ e $f^{-1}(\real_-^*)$.


% \questao Seja $R$ uma rela{\c c}{\~a}o de equival{\^e}ncia
% sobre um conjunto n{\~a}o vazio $E$. Considere o conjunto quociente
% $E/R$ e  a aplica{\c c}{\~a}o quociente $q:E\to E/R$, ou seja, a aplica{\c c}{\~a}o
% que a cada $x\in E$ associa sua classe $\bar{x}$ m{\'o}dulo $R$. Seja
% $f:E\to F$ uma aplica{\c c}{\~a}o de $E$ em um conjunto $F$.

% \begin{enumerate}[label={\alph*})]
% \item Mostre que existe uma aplica{\c c}{\~a}o $g$ tal que $g\circ
% q=f$ se, e somente se, a seguinte condi{\c c}{\~a}o estiver verificada:
% para quaisquer $x,y\in E$, se $x\equiv y~(\mathrm{mod}~R)$ ent{\~a}o
% $f(x)=f(y).$
% \[
% \xymatrix{
% E  \ar[d]_-{f} \ar[r]^-{q}& E/R \ar@{-->}[dl]^-{g}  \\
% F  &  \\
% }
% \]
% \item Supondo que exista $g$ tal que $g\circ q=f$, mostre que
% $g$ {\'e} {\'u}nica.
% \item Nas condi{\c c}{\~o}es do item anterior, mostre que $g$ {\'e}
% sobrejetora se, e somente se, $f$ {\'e} sobrejetora.
% \end{enumerate}

%\vesp
%
%\questao Seja $f:E\to F$ e considere a seguinte
%rela{\c c}{\~a}o sobre $E$:
%\[
%xRy\Longleftrightarrow f(x)=f(y).
%\]
%\begin{enumerate}[label={\alph*})]
%\item Mostre que $R$ {\'e} uma rela{\c c}{\~a}o de equival{\^e}ncia.
%\item Mostre que se $f$ {\'e} sobrejetora, ent{\~a}o existe uma {\'u}nica
%bije{\c c}{\~a}o $g: E/R \to F$ tal que $g\circ \pi = f$, onde $\pi: E\to
%E/R$ {\'e} a aplica{\c c}{\~a}o quociente.
%\end{enumerate}
%
%\vesp
%
%\questao As seguintes afirma{\c c}{\~o}es s{\~a}o verdadeiras ou falsas?
%\begin{multicols}{2}
%\begin{enumerate}[label={\alph*})]
%\item $2\z \sub 4\z$;
%\item$3\z \sub 7\z$;
%\item$8\z \sub 2\z$;
%\item$2\z \cap 8\z = 8\z$;
%\item$3\z \cap 7\z = 21\z$;
%\item$n\z \cap m\z = mn\z$;
%\item$2\z \cup 3\z = 6\z$;
%\item$2\z \cup 8\z = 2\z$;
%\item$n\z \cup m\z = mdc(m,n)\z$.
%\end{enumerate}
%\end{multicols}
%
%\vesp
%
%\questao Seja $p$ um n{\'u}mero primo. Seja $A$ um ideal de $\z$, tal que $p\z
%\sub A$. Mostre que $A = \z$ ou $A = p\z$.
%
%\vesp
%
%\questao Sejam $m$, $n$ inteiros positivos. Mostre que a interse{\c c}{\~a}o $m\z
%\cap n\z$ {\'e} um ideal de $\z$. {\'E} poss{\'\i}vel achar o gerador desse ideal?
%
%\vesp
%
%\questao {\'E} verdadeiro ou falso que a uni{\~a}o de dois ideais de $\z$ {\'e} um ideal
%de $\z$?
%
%\vesp
%
%\questao Sejam $A$ e $B$ dois ideais de $\z$. Definimos a {\bf soma de
%  ideais}
%\[
%A + B = \{ a + b \mid a \in A, b \in B\}
%\]
%quando $a$ varia sobre todos os elementos de $A$ e $b$ varia sobre todos os
%elementos de $B$. Mostre que a soma de dois ideais de $\z$ {\'e} um ideal de $\z$.

\end{document}