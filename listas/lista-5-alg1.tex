%!TEX program = xelatex
%!TEX encoding = ISO-8859-1
\documentclass[12pt]{article}

\usepackage{amssymb}
\usepackage{amsmath,amsfonts,amsthm,amstext}
\usepackage[brazil]{babel}
%\usepackage[latin1]{inputenc}
\usepackage{graphicx}
\graphicspath{{/home/jfreitas/Dropbox/imagens-latex/}{/Users/jfreitas/Dropbox/imagens-latex/}{D:/Dropbox/imagens-latex/}}
\usepackage{enumitem}
\usepackage{multicol}
\usepackage[all]{xy}

\setlength{\topmargin}{-1.0in}
\setlength{\oddsidemargin}{0in}
\setlength{\textheight}{10.1in}
\setlength{\textwidth}{6.5in}
\setlength{\baselineskip}{12mm}

\newcounter{exercicios}
\setcounter{exercicios}{0}
\newcommand{\questao}{
\addtocounter{exercicios}{1}
\noindent{\bf Exerc{\'\i}cio \arabic{exercicios}: }}

\newcommand{\equi}{\Leftrightarrow}
\newcommand{\bic}{\leftrightarrow}
\newcommand{\cond}{\rightarrow}
\newcommand{\impl}{\Rightarrow}
\newcommand{\nao}{\sim}
\newcommand{\sub}{\subseteq}
\newcommand{\e}{\ \wedge\ }
\newcommand{\ou}{\ \vee\ }
\newcommand{\vaz}{\emptyset}

\newcommand{\real}{\mathbb{R}}
\newcommand{\vesp}{\vspace{0.2cm}}
\newcommand{\z}{\mathbb{Z}}
\newcommand{\n}{\mathbb{N}}
\newcommand{\q}{\mathbb{Q}}
\newcommand{\complex}{\mathbb{C}}
\renewcommand{\sin}{{\rm sen\,}}
\newtheorem{defin}{Defini{\c c}{\~a}o}

\newcommand{\compcent}[1]{\vcenter{\hbox{$#1\circ$}}}
\newcommand{\comp}{\mathbin{\mathchoice
{\compcent\scriptstyle}{\compcent\scriptstyle}
{\compcent\scriptscriptstyle}{\compcent\scriptscriptstyle}}}

\begin{document}
\pagestyle{empty}

\begin{figure}[h]
        \begin{minipage}[c]{1.7cm}
        \includegraphics[width=1.7cm]{unb.pdf}
        \end{minipage}%
        \hspace{0pt}
        \begin{minipage}[c]{4in}
          {Universidade de Bras{\'\i}lia} \\
          {Departamento de Matem{\'a}tica}
\end{minipage}
\end{figure}
\vspace{-1cm}\hrule

\begin{center}
 {\Large\bf {\'A}lgebra 1 - Turma D -- 1$^{o}$/2016} \\
 \vspace{9pt} {\large\bf $5^{\underline{a}}$ Lista de Exerc{\'\i}cios -- Grupos: continua\c{c}\~ao}\\
 \vspace{9pt} Prof. Jos{\'e} Ant{\^o}nio O. Freitas
\end{center}
\hrule

\vspace{0.6cm}

\questao No conjunto $\z \times \z$ considere a opera\c{c}\~ao de soma definida por
\[
	(x, y) + (z, t) = (x + z, y + t)
\]
para $(x, y)$, $(z, t) \in \z \times \z$. Mostre que $(\z\times\z, +)$ \'e um grupo abeliano.

\vesp

\questao Verifique se s\~ao subgrupos:
\begin{enumerate}[label=({\alph*})]
	\item $H = \{x \in \q \mid x > 0\}$ de $(\q^*,\cdot)$.
	\item $H = \left\{\dfrac{1 + 2m}{1 + 2n} \mid m, n \in \z\right\}$ de $(\q^*,\cdot)$.
	\item $H = \{\cos\theta + i\sin\theta \mid \theta \in \q\}$ de $(\complex^*,\cdot)$.
	\item $H = \{0, \pm 2, \pm 4, \pm 6, \dots\}$ de $(\z,+)$.
	\item $H = \{0, \pm 2, \pm 4, \pm 6, \dots\}$ do grupo $(\q - \{1\},\star)$ onde $\star$ \'e definida como $x \star y = x + y - xy$.
	\item $H = \{a + b\sqrt{2} \mid a, b \in \q\}$ de $(\real,+)$.
	\item $H = \{a + b\sqrt{2} \in \real^* \mid a, b \in \q\}$ de $(\real^*,\cdot)$.
	\item $H = \{a + b\sqrt[3]{2} \mid a, b \in \q\}$ de $(\real,+)$.
	\item $H = \{a + b\sqrt[3]{2} \in \real^* \mid a, b \in \q\}$ de $(\real^*,\cdot)$.
\end{enumerate}

\vesp

\questao Determine todos os subgrupos do grupo aditivo $\z_4$.

\vesp

\questao Determine todos os subgrupos de $S_3$.

\vesp

\questao Seja
\[
	GL_2(\real) = \left\{A = \begin{pmatrix}
		x & y\\z & t
	\end{pmatrix} \mid x, y, z, t \in \real,\ \det(A) \ne 0\right\}.
\]
\begin{enumerate}[label=({\alph*})]
	\item Mostre que $GL_2(\real)$ com a opera\c{c}\~ao de multiplica\c{c}\~ao de matrizes \'e um grupo. Esse grupo \'e abeliano?
	\item Seja
	\[
		H = \left\{ A = \begin{pmatrix}
			\cos a & \sin a\\ -\sin a & \cos a
		\end{pmatrix} \mid a \in \real\right\}.
	\]
	Mostre que $H$ \'e um subgrupo de $GL_2(\real)$.
	\item Seja
	\[
		K = \left\{ A = \begin{pmatrix}
			a & b\\ -b & a
		\end{pmatrix} \mid a, b \in \real \in \real \mbox{ e n\~ao nulos simultaneamente}\right\}.
	\]
	Mostre que $K$ \'e um subgrupo de $GL_2(\real)$.
\end{enumerate}

\vesp

\questao Sejam $H$ e $K$ subgrupos de um grupo $G$ (com nota{\c c}{\~a}o
multiplicativa).
\begin{enumerate}[label=({\alph*})]
\item Mostre que $H\cap K$ tamb{\'e}m {\'e} subgrupo de $G$.
\item Seja $g\in G$ um elemento fixado. Mostre que o conjunto
$g^{-1}Hg=\{ g^{-1}xg \mid x\in H \} $ {\'e} um subgrupo de $G$.
\item Prove que $H\cup K$ {\'e} subgrupo de $G$ se, e somente se,
$H\subseteq K$ ou $K\subseteq H$.
%\item Demonstre que $HK=\{hk \mid h\in H, k\in K\}$ {\'e} subgrupo
%de $G$ se, e somente se, $HK=KH$.

%[\emph{Nota: $HK=KH$ \textbf{n{\~a}o} quer dizer que $hk=kh$,
%para todo $h\in H, k\in K$; significa que $hk=k_1h_1 \in KH$ e $kh=h_2k_2 \in
%HK$, para todo $h\in H, k\in K$.}]
\end{enumerate}

\vesp

\questao Seja $G$ um grupo com nota\c{c}\~ao multiplicativa e $a$ um elemento de $G$. Prove que $N(a) = \{x \in G \mid ax = xa\}$ \'e um subgrupo de $G$.

\vesp

\questao Seja $G$ um grupo com nota\c{c}\~ao multiplicativa. Considere o subconjunto $Z(G) = \{x \in G \mid xg = gx, \mbox{ para todo } x \in G\}$. Mostre que:
\begin{enumerate}[label=({\alph*})]
	\item $Z(G)$ \'e um subgrupo de $G$.
	\item $G$ \'e abeliano se, e somente se, $Z(G) = G$.
\end{enumerate}

\vesp

\questao Verificar em cada caso se $f$ \'e um homomorfismo de grupos.
\begin{enumerate}[label=({\alph*})]
	\item $f: \z \to \z$ dada por $f(x) = kx$, sendo $\z$ o grupo aditivo dos inteiros e $k$ um n\'umero inteiro fixo.
	\item $f: \real^* \to \real^*$ dada por $f(x) = |x|$ sendo $\real^*$ o grupo multiplicativo dos reais.
	\item $f: \real \to \real$ dada por $f(x) = x + 1$, onde $\real$ \'e o grupo aditivo dos reais.
	\item $f: \z \to \z \times \z$ dada por $f(x) = (x, 0)$, onde $\z$ e $\z \times \z$ denotam grupos aditivos.
	\item $f: \z \times \z \to \z$ dada por $f(x,y) = x$, onde $\z \times \z$ e $\z$ s\~ao grupos aditivos.
	\item $f: \z \to \real^*_+$ dada por $f(x) = 2^x$, onde $\z$ \'e grupo aditivo e $\real^*_+$ \'e grupo multiplicativo.
\end{enumerate}

\vesp

\questao Determinar os homomorfismos injetores e os sobrejetores do exerc{\'\i}cio anterior.

\vesp

\questao Determine o n\'ucleo em cada homomorfismo do \textbf{Exerc{\'\i}cio 9}.

\vesp

\questao Seja $f: \z \times \z \to \z \times \z$ definida por $f(x, y) = (x - y, 0)$. Provar que $f$ \'e um homomorfimo do grupo aditivo $\z \times \z$ em si pr\'oprio. Obter $\ker(f)$.

\vesp

\questao Seja $(G, *)$ um grupo e $g\in G$ um elemento fixado e $g^{-1}$
seu inverso. Mostre que a aplica{\c c}{\~a}o $i_g: G\to G$ definida por
$i_g(x)=g^{-1}*x*g$, para todo $x \in G$, {\'e} um isomorfismo.

\vesp

\questao Sejam $f:G\to H$ e $g:H\to J$ isomorfismos de grupos. Mostre
que
\begin{enumerate}[label=({\alph*})]
\item $g\circ f$ {\'e} tamb{\'e}m isomorfismo de grupos;
\item $f^{-1}$ {\'e} isomorfismo de grupos.
\end{enumerate}

\vesp

\questao Sejam $G$ e $J$ grupos multiplicativos, $f : G \to J$ um homomorfismo de grupos e $H$ um subgrupo de $J$. Mostre que $f^{-1}(H) = \{ x \in G \mid f(x) \in H\}$ {\'e} um subgrupo de $G$.

\vesp

\questao Prove que um grupo $G$ {\'e} abeliano se, e somente se, $f : G \to G$ definada por $f(x) = x^{-1}$ {\'e} um homomorfismo.

\vesp

\questao Seja $f: G\to H$ um
homomorfismo sobrejetivo de grupos e $K$ um subgrupo de $H$.
\begin{enumerate}[label=({\alph*})]
\item Mostre que $f^{-1}(K)$ {\'e} um subgrupo de $G$, onde $f^{-1}(K)$ {\'e} a imagem inversa de $K$.
\item Mostre que o n{\'u}cleo Ker$(f)$ de $f$ {\'e} um subgrupo de
$G$  e que Ker$(f)\subset f^{-1}(K)$.
%\item Seja $N$ um subgrupo de $G$. Mostre que $\mbox{Ker}(f)\cdot
%N = f^{-1}(f(N))$, onde $f^{-1}(f(N))$ {\'e} a imagem inversa de $f(N)$.

%[Lembre que $\mbox{Ker}(f)\cdot
%N=\{xy ~|~ x\in \mbox{Ker}(f), y\in N\}$ e mostre que $\mbox{Ker}(f)\cdot
%N \subseteq f^{-1}(f(N))$. Para ver que $f^{-1}(f(N))\subseteq \mbox{Ker}(f)\cdot
%N$, tome $x\in f^{-1}(f(N))$. Ent{\~a}o $f(x)\in f(N)$, ou seja, $f(x)=f(y)$, para algum $y\in N$. Assim,
%$f(xy^{-1})=e$ e portanto $xy^{-1}=z \in \mbox{Ker}(f)$. Ent{\~a}o $x=zy \in \mbox{Ker}(f)\cdot
%N$.]
\end{enumerate}

\vesp

\questao Seja $f: G \to J$ um homomorfismo de grupos e $g\in G$ tal que $o(g)=n$.
\begin{enumerate}[label=({\alph*})]
\item Mostre que $f(g)$ tem ordem positiva e que a ordem de $f(g)$ divide $n$.
\item Mostre que se $f$ {\'e} isomorfismo, ent{\~a}o $o(f(g))=n$.
\end{enumerate}

\vesp

\questao Sabendo que $G = \{e, a, b, c, d, f\}$ {\'e} um grupo multiplicativo isomorfo ao grupo $(\z_6, +)$, pede-se:
\begin{enumerate}[label=({\alph*})]
\item construir uma tabela de multiplica{\c c}{\~a}o para $G$;
\item calcular $a^2$, $b^{-2}$ e $c^{-3}$;
\item obter $x \in G$ tal que $dxf = a^{-1}$.
\end{enumerate}

\vesp


\questao Seja $G$ um grupo finito com elemento
neutro $e$ e suponha que $H$ e $K$ s{\~a}o subgrupos de $G$ tais que
$|H|=p$ e $|K|=q$, onde $p$ e $q$ s{\~a}o primos distintos. Mostre que
$H\cap K=\{e\}$.

\vesp


\end{document}