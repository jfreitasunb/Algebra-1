%!TEX program = xelatex
%!TEX encoding = ISO-8859-1
\documentclass[12pt]{article}

\usepackage{amssymb}
\usepackage{amsmath,amsfonts,amsthm,amstext,mathabx}
\usepackage[brazil]{babel}
%\usepackage[latin1]{inputenc}
\usepackage{graphicx}
\graphicspath{{/home/jfreitas/Dropbox/imagens-latex/}{/Volumes/Vader/Dropbox/imagens-latex/}{D:/Dropbox/imagens-latex/}}
\usepackage{enumitem}
\usepackage{multicol}
\usepackage[all]{xy}

\setlength{\topmargin}{-1.0in}
\setlength{\oddsidemargin}{0in}
\setlength{\textheight}{10.1in}
\setlength{\textwidth}{6.5in}
\setlength{\baselineskip}{12mm}

\newcounter{exercicios}
\setcounter{exercicios}{0}
\newcommand{\questao}{
\addtocounter{exercicios}{1}
\noindent{\bf Exerc{\'\i}cio \arabic{exercicios}: }}

\newcommand{\equi}{\Leftrightarrow}
\newcommand{\bic}{\leftrightarrow}
\newcommand{\cond}{\rightarrow}
\newcommand{\impl}{\Rightarrow}
\newcommand{\nao}{\sim}
\newcommand{\sub}{\subseteq}
\newcommand{\e}{\ \wedge\ }
\newcommand{\ou}{\ \vee\ }
\newcommand{\vaz}{\emptyset}
\newcommand{\nsub}{\nsubset}
\renewcommand{\sin}{{\rm sen\,}}

\newcommand{\n}{\mathbb{N}}
\newcommand{\z}{\mathbb{Z}}
\newcommand{\real}{\mathbb{R}}
\newcommand{\vesp}{\vspace{0.2cm}}
\newcommand{\subne}{\subsetneqq}


\newcommand{\compcent}[1]{\vcenter{\hbox{$#1\circ$}}}
\newcommand{\comp}{\mathbin{\mathchoice
{\compcent\scriptstyle}{\compcent\scriptstyle}
{\compcent\scriptscriptstyle}{\compcent\scriptscriptstyle}}}

\begin{document}

\pagestyle{empty}

\begin{figure}[h]
        \begin{minipage}[c]{1.7cm}
        \includegraphics[width=1.7cm]{unb.pdf}
        \end{minipage}%
        \hspace{0pt}
        \begin{minipage}[c]{4in}
          {Universidade de Brasília} \\
          {Departamento de Matemática}
\end{minipage}
\end{figure}
\vspace{-1cm}\hrule


\begin{center}
 {\Large\bf {\'A}lgebra 1 - Turma C -- 1$^{o}$/2018} \\
 \vspace{9pt} {\large\bf $5^{\underline{a}}$ Lista de Exerc{\'\i}cios -- Anéis e Homomorfismos}\\
 \vspace{9pt} Prof. Jos{\'e} Ant{\^o}nio O. Freitas
\end{center}
\hrule

\vspace{.6cm}

\textit{Nos casos em que não forem especificadas as operações do anel, considere as operações usuais.}

\vspace{.6cm}

\questao{}
	Determinar quais dos seguintes subconjuntos de $\rac$ s{\~a}o suban{\'e}is:
	\begin{multicols}{2}
		\begin{enumerate}[label=({\alph*})]
			\item $\z$
			\item $B = \{x \in \rac \mid x \notin \z\}$
			\item $C = \left\{\dfrac{a}{b} \in \rac \mid a \in \z,\ b \in \z,\ 2 |b \right\}$
			\item $D = \left\{\dfrac{a}{2^n} \in \rac \mid a \in \z \mbox{ e } n \in \z \right\}$
		\end{enumerate}
	\end{multicols}
 

\vesp

\questao{} No anel $(\z \times \z, \oplus, \otimes)$ onde as opera\c{c}\~oes $\oplus$ e $\otimes$ são definidas por
\begin{align*}
	(a, b) \oplus (c, d) = (a + c, b + d)\\
	(a ,b) \otimes (c, d) = (ac - bd, ad + bc).
\end{align*}
Quais dos seguintes conjuntos são subanéis?
\begin{enumerate}
	\item $A = \{(x, y) \in \z \times \z \mid x = 0\}$
	\item $A = \{(x, y) \in \z \times \z \mid y = 0\}$
	\item $A = \{(x, y) \in \z \times \z \mid x = y\}$
	\item $A = \{(x, y) \in \z \times \z \mid x = 2k,\ k \in \z\}$
	\item $A = \{(x, y) \in \z \times \z \mid y = 3k,\ k \in \z\}$
	\item $A = \{(x, y) \in \z \times \z \mid x + y = 2k,\ k \in \z\}$
\end{enumerate}


\questao{}
	Quais dos conjuntos abaixo s\~ao suban\'eis de $M_2(\real)$?
\begin{align*}
	L_1 &= \left\{\begin{pmatrix}
		a & 0\\
		b & 0
	\end{pmatrix} \mid a, b \in \real\right\}\\
	L_2 &= \left\{\begin{pmatrix}
		a & b\\
		0 & c
	\end{pmatrix} \mid a, b, c \in \real\right\}\\
	L_3 &= \left\{\begin{pmatrix}
		a & 0\\
		0 & b
	\end{pmatrix} \mid a, b \in \real\right\}\\
	L_4 &= \left\{\begin{pmatrix}
		0 & a\\
		c & b
	\end{pmatrix} \mid a, b, c \in \real\right\}
\end{align*}


\vesp

\questao{}
Determine todos os suban\'eis do anel $(\z_{16}, \oplus, \otimes)$.


\vesp

\questao{}
	Mostre que a interseção de dois subanéis de um anel $A$ é ainda um subanel de $A$.


\vesp

\questao
	Seja $(A, + , \cdot)$ um anel e $x \in A$ fixo. Mostre que o conjunto
	\[
		N(x) = \{y \in A \mid xy = yx\}
	\]
	é um subanel de $A$.


\questao{}
Verifique se $L = \{ a + b\sqrt{2} \mid a, b \in \rac\}$ {\'e} um subanel
do anel $\mathbb{R}$.


\vesp

\questao{inicio_referencia}{}
	Verificar se a função $f : A \to B$ é ou não um homomorfismo do anel $A$ no anel $B$, nos seguintes casos:
\begin{enumerate}[label=({\alph*})]
\item $A = \z$, $B = \z$ e $f(x) = x + 1$
\item $A = \z$, $B = \z$ e $f(x) = 2x$
\item $A = \z$, $B = \z \times \z$ e $f(x) = (x, 0)$
\item $A = \z \times \z$, $B = \z$ e $f(x,y) = x$
\item $A = \z \times \z$, $B = \z \times \z$ e $f(x,y) = (y,x)$
\item $A = \z$, $B = \z_n$ e $f(x) = \overline{x}$
\item $A = \complex$, $B = \complex$ e $f(a + bi) = a - bi$
\item $A = M_2(\real)$, $B = \real$ e $f\left(\begin{bmatrix}
	x & y\\z & t
\end{bmatrix}\right) = x$
\item $A = M_2(\real)$, $B = \real$ e $f\left(\begin{bmatrix}
	x & y\\z & t
\end{bmatrix}\right) = x + t$
\end{enumerate}


\vesp

\questao{}
	Prove que $f : \rac \to M_3(\rac)$ dada por
\[
	f(x) = \begin{pmatrix}
		x & 0 & 0\\
		0 & x & 0\\
		0 & 0 & x
	\end{pmatrix}
\]
é um homomorfismo de anéis.


\vesp

\questao{fim_referencia}
	 Considere os an{\'e}is $\z$ e $\z\times \z$. Verifique se as seguintes funções s{\~a}o homomorfismos:
\begin{enumerate}[label=({\alph*})]
\item $f : \z\times\z \to \z\times\z$ dado por $f(x,y) = (0,y)$
\item $f : \z\times\z \to \z$ dado por $f(x,y) = y$
\item $f : \z\to \z\times\z$ dado por $f(x) = (2x,0)$
\item $f : \z\times\z \to \z\times\z$ dado por $f(x,y) = (-y,-x)$
\item $f : \z \to \z\times\z$ dado por $f(x) = (0,x)$
\end{enumerate}


\vesp

\questao{}
	Determine o kernel dos homomorfismos dos \textbf{Exerc{\'i}cios de \ref{inicio_referencia} a \ref{fim_referencia}}.


\vesp

\questao{}
	Nos \textbf{Exercícios de \ref{inicio_referencia} a \ref{fim_referencia}} para as funções que forem homomorfismos determine se elas também são isomorfismos.


\vesp

\questao{}
	Seja $f : \complex \to M_2(\real)$ dada por
\[
	f(a + bi) = \begin{bmatrix}
		a & -b\\
		b & a
	\end{bmatrix}.
\]
\begin{enumerate}[label=({\alph*})]
	\item Mostre que $f$ é um homomorfismo de anéis.
	\item Esse homomorfismo é injetor?
	\item É sobrejetor?
\end{enumerate}


\vesp

\questao{}
	Seja $f: A \to B$ um homomorfismo de an{\'e}is. Mostre que:
\begin{enumerate}[label=({\alph*})]
\item Se $C$  {\'e} um subanel de $A$, ent{\~a}o $f(C)$ {\'e} um subanel de $B$.
\item Se $D$ {\'e} um subanel de $B$, ent{\~a}o $f^{-1}(D)$ {\'e} um subanel de $A$.
\end{enumerate}


\vesp

\questao{}
	D{\^e} um exemplo de an{\'e}is $A$ e $B$ e um homomorfismo $f : A \to B$ tal que $f(1_A) \ne 1_B$.


\vesp

\questao{}
	Sejam os an{\'e}is $A = \{ a + b\sqrt{-2} \mid a,\ b \in \rac\}$ e $B = M_2(\rac)$.
\begin{enumerate}[label=({\alph*})]
\item Mostre que $f : A \to B$ dada por
\[
f(a + b\sqrt{-2}) =
\begin{pmatrix}
a & -2b\\
b & a
\end{pmatrix}
\]
{\'e} um homomorfismo.
\item $f$ {\'e} um isomorfimo?
\end{enumerate}


\vesp

\questao{}
	{\'E} verdadeiro ou falso: $\z$ e $\z_{m}$ para $m > 1$ s{\~a}o an{\'e}is
isomorfos.


\vesp

\questao{}
	Considere os seguintes an{\'e}is: $(\real, +, \cdot)$ e $(\real, \oplus, \odot)$, sendo $a \oplus b = a + b + 1$ e $a \odot b = a + b + ab$. Mostre que $f : \real \to \real$ dado por $f(x) = x + 1$, para todo $x \in \real$, {\'e} um isomorfimos de $(\real, \oplus, \odot)$ em $(\real, +, \cdot)$.

\end{document}