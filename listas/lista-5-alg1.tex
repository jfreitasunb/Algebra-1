%!TEX program = xelatex 
% !TEX encoding = ISO-8859-1
\documentclass[12pt]{article}

\usepackage{amssymb}
\usepackage{amsmath,amsfonts,amsthm,amstext}
\usepackage[brazil]{babel}
%\usepackage[latin1]{inputenc}
\usepackage{graphicx}
\graphicspath{{/home/jfreitas/Dropbox/imagens-latex/}{/Users/jfreitas/Dropbox/imagens-latex/}{D:/Dropbox/imagens-latex/}}
\usepackage{enumitem}
\usepackage{multicol}
\usepackage{color}
\usepackage[all]{xy}

\setlength{\topmargin}{-1.0in}
\setlength{\oddsidemargin}{0in}
\setlength{\textheight}{10.1in}
\setlength{\textwidth}{6.5in}
\setlength{\baselineskip}{12mm}

\newcounter{exercicios}
\setcounter{exercicios}{0}
\newcommand{\questao}{
\addtocounter{exercicios}{1}
\noindent{\bf Exerc{\'\i}cio \arabic{exercicios}: }}

\newcommand{\equi}{\Leftrightarrow}
\newcommand{\bic}{\leftrightarrow}
\newcommand{\cond}{\rightarrow}
\newcommand{\impl}{\Rightarrow}
\newcommand{\nao}{\sim}
\newcommand{\sub}{\subseteq}
\newcommand{\e}{\ \wedge\ }
\newcommand{\ou}{\ \vee\ }
\newcommand{\vaz}{\emptyset}

\newcommand{\real}{\mathbb{R}}
\newcommand{\vesp}{\vspace{0.2cm}}
\newcommand{\z}{\mathbb{Z}}
\newcommand{\n}{\mathbb{N}}
\newcommand{\q}{\mathbb{Q}}
\newtheorem{defin}{Defini{\c c}{\~a}o}

\newcommand{\compcent}[1]{\vcenter{\hbox{$#1\circ$}}}
\newcommand{\comp}{\mathbin{\mathchoice
{\compcent\scriptstyle}{\compcent\scriptstyle}
{\compcent\scriptscriptstyle}{\compcent\scriptscriptstyle}}}

\begin{document}
\pagestyle{empty}

\begin{figure}[h]
        \begin{minipage}[c]{1.7cm}
        \includegraphics[width=1.7cm]{unb.pdf}
        \end{minipage}%
        \hspace{0pt}
        \begin{minipage}[c]{4in}
          {Universidade de Bras{\'\i}lia} \\
          {Departamento de Matem{\'a}tica}
\end{minipage}
\end{figure}
\vspace{-1cm}\hrule

\begin{center}
 {\Large\bf {\'A}lgebra 1 - Turma B -- 2$^{o}$/2015} \\
 \vspace{9pt} {\large\bf $5^{\underline{a}}$ Lista de Exerc{\'\i}cios -- An\'eis}\\
 \vspace{9pt} Prof. Jos{\'e} Ant{\^o}nio O. Freitas
\end{center}
\hrule

\begin{center}
\Large{\bf An{\'e}is}
\end{center}

\vspace{.6cm}

\questao Consideremos em $\z \times \z$ as opera\c{c}\~oes $\oplus$ e $\otimes$ definidas por
\begin{align*}
	(a, b) \oplus (c, d) = (a + c, b + d)\\
	(a ,b) \otimes (c, d) = (ac - bd, ad + bc).
\end{align*}
Mostre que $(\z \times \z, \oplus, \otimes)$ \'e um anel comutativo e com unidade.

\vesp

\questao Considere as opera\c{c}\~oes $\star$ e $\odot$ em $\q$ definidas por
\begin{align*}
	x \star y = x + y - 3\\
	x \odot y = x + y - \dfrac{xy}{3}.
\end{align*}
Mostre que $(\q, \star, \odot)$ \'e um anel comutativo e com unidade.

\vesp

\questao Prove que s\~ao an\'eis:
\begin{enumerate}[label={\alph*})]
	\item O conjunto $\z$ com a adi\c{c}\~ao usual e o produto $x \otimes y = 0$, para todo $x$, $y \in \z$.
	\item O conjunto $\q$ com as opera\c{c}\~oes $x \oplus y = x + y - 1$ e $x \odot y = x + y - xy$.
	\item O conjunto $\z \times \z$ com as opera\c{c}\~oes:
	\begin{align*}
		(a, b) \oplus (c, d) = (a + c, b + d)\\
		(a ,b) \otimes (c, d) = (ac, ad + bc).
	\end{align*}
\end{enumerate}
Quais destes an\'eis s\~ao comutativos. Quais t\^em unidade.

\vesp

% \questao Seja $(A, + , \cdot)$ um anel. Mostre que subconjunto n{\~a}o vazio $C\subseteq A$ {\'e} um \textbf{subanel} de $A$ se, e somente se, $x - y \in C$ e $x\cdot y \in C$ para todos $x$, $y \in C$.

% \vesp

\questao Ache os elementos invers{\'\i}veis dos seguintes an\'eis:
\begin{enumerate}
	\item $(\q, \oplus, \otimes)$ onde $a \oplus b = a + b - 1$ e $a \otimes b = a + b - ab$;
	\item $(\z \times \z, \star, \odot)$ onde $(a, b) \star (c, d) = (a + c, b + d)$ e $(a, b) \odot (c, d) = (ac, ad + bc)$.
\end{enumerate}

\vesp

\questao Determinar quais dos seguintes subconjuntos de $\q$ s{\~a}o suban{\'e}is:
	\begin{multicols}{2}
		\begin{enumerate}[label=({\alph*})]
			\item $\z$
			\item $B = \{x \in \q \mid x \notin \z\}$
			\item $C = \left\{\dfrac{a}{b} \in \q \mid a \in \z,\ b \in \z,\ 2 |b \right\}$
			\item $D = \left\{\dfrac{a}{2^n} \in \q \mid a \in \z \mbox{ e } n \in \z \right\}$
		\end{enumerate}
	\end{multicols}

\vesp

\questao Quais dos conjuntos abaixo s\~ao suban\'eis de $M_2(\real)$?
\begin{align*}
	L_1 &= \left\{\begin{pmatrix}
		a & 0\\
		b & 0
	\end{pmatrix} \mid a, b \in \real\right\}\\
	L_2 &= \left\{\begin{pmatrix}
		a & b\\
		0 & c
	\end{pmatrix} \mid a, b, c \in \real\right\}\\
	L_3 &= \left\{\begin{pmatrix}
		a & 0\\
		0 & b
	\end{pmatrix} \mid a, b \in \real\right\}\\
	L_4 &= \left\{\begin{pmatrix}
		0 & a\\
		c & b
	\end{pmatrix} \mid a, b, c \in \real\right\}
\end{align*}

\vesp

\questao Determine todos os suban\'eis do anel $(\z_{16}, \oplus, \otimes)$.

\vesp

\questao Verifique se $L = \{ a + b\sqrt{2} \mid a, b \in \q\}$ {\'e} um subanel
do anel $\mathbb{R}$.

\vesp

\questao Ache todos os ideais do anel $\z_4$, do anel $\z_{12}$ e do anel
$\z_{16}$.

\vesp

\questao Mostre que a interse{\c c}{\~a}o de quaisquer dois ideais de um anel
comutativo {\'e} sempre um conjunto n{\~a}o vazio. Mostre que esse conjunto n{\~a}o
vazio {\'e} um ideal tamb{\'e}m.

\vesp

\questao Ache os ideais primos dos an{\'e}is $\z_2$, $\z_4$, $\z_5$,
$\z_8$ e $\z_{10}$. O ideal $\{\overline{0}\}$ do anel $\z_4$ {\'e} primo?

\vesp

% \questao Seja $I$ um ideal de $\z$. Mostre que se todos os elementos n{\~a}o
% nulos de $\z/I$ s{\~a}o invers{\'\i}veis, ent{\~a}o $I$ {\'e} um ideal primo.

% \vesp

\questao As seguintes afirma{\c c}{\~o}es s{\~a}o verdadeiras ou falsas?
\begin{multicols}{2}
\begin{enumerate}[label={\alph*})]
\item $2\z \sub 4\z$;
\item$3\z \sub 7\z$;
\item$8\z \sub 2\z$;
\item$2\z \cap 8\z = 8\z$;
\item$3\z \cap 7\z = 21\z$;
\item$n\z \cap m\z = mn\z$;
\item$2\z \cup 3\z = 6\z$;
\item$2\z \cup 8\z = 2\z$;
%\item$n\z \cup m\z = mdc(m,n)\z$.
\end{enumerate}
\end{multicols}

\vesp

\questao Seja $p$ um n{\'u}mero primo. Seja $A$ um ideal de $\z$, tal que $p\z
\sub A$. Mostre que $A = \z$ ou $A = p\z$.

\vesp

\questao Sejam $m$, $n$ inteiros positivos. Mostre que a interse{\c c}{\~a}o $m\z
\cap n\z$ {\'e} um ideal de $\z$.

\vesp

\questao {\'E} verdadeiro ou falso que a uni{\~a}o de dois ideais de $\z$ {\'e} um ideal
de $\z$?

\vesp

\questao Sejam $A$ e $B$ dois ideais de $\z$. Definimos a {\bf soma de
  ideais}
\[
A + B = \{ a + b \mid a \in A, b \in B\}
\]
quando $a$ varia sobre todos os elementos de $A$ e $b$ varia sobre todos os
elementos de $B$. Mostre que a soma de dois ideais de $\z$ {\'e} um ideal de $\z$.

\vesp

\questao Considere os an{\'e}is $\z$ e $\z\times \z$. Verifique se s{\~a}o homomorfismos:
\begin{enumerate}[label=({\alph*})]
\item $f : \z\times\z \to \z\times\z$ dado por $f(x,y) = (0,y)$
\item $f : \z\times\z \to \z$ dado por $f(x,y) = y$
\item $f : \z\to \z\times\z$ dado por $f(x) = (2x,0)$
\item $f : \z\times\z \to \z\times\z$ dado por $f(x,y) = (-y,-x)$
\item $f : \z \to \z\times\z$ dado por $f(x) = (0,x)$
\end{enumerate}

\vesp

\questao Determine o kernel dos homomorfismos do Exerc{\'i}cio 17.

\vesp

\questao Seja $f: A \to B$ um homomorfismo de an{\'e}is. Mostre que:
\begin{enumerate}[label=({\alph*})]
\item Se $C$  {\'e} um subanel de $A$, ent{\~a}o $f(C)$ {\'e} um subanel de $B$.
\item Se $D$ {\'e} um subanel de $B$, ent{\~a}o $f^{-1}(D)$ {\'e} um subanel de $A$.
\item Se $I$ {\'e} um ideal de $A$, ent{\~a}o $f(I)$ {\'e} um ideal de $B$.
\item Se $J$ {\'e} um ideal de $B$ e $f$ é sobrejetora, ent{\~a}o $f^{-1}(J)$ {\'e} um ideal de $A$.
\end{enumerate}

\vesp

\questao D{\^e} um exemplo de an{\'e}is $A$ e $B$ e um homomorfismo $f : A \to B$ tal que $f(1_A) \ne 1_B$.

\vesp

%\questao Sejam $(A, +, \cdot)$ e $(B, \oplus, \otimes)$ dois an{\'e}is e $f: A
%\to B$ um homomorfismo. Definimos o {\it n{\'u}cleo} ou o {\it
%  kernel} de $f$ como o conjunto
%  \[
%  Ker(f) = \{a \in A \mid f(a) = 0_B\}.
%  \]
%Mostre que o kernel {\'e} um ideal de $A$.

%\vesp

% \questao Mostre que o kernel de $f$ pode ser considerado como a imagem
% inversa do subconjunto $\{0_B\}$ pela fun{\c c}{\~a}o $f$.

% {\it Sugest{\~a}o: Seja $f: X \to Y$ uma fun{\c c}{\~a}o}. A imagem inversa {\it
%   de um subconjunto $U \sub Y$ {\'e} o subconjunto $X$ definido por $f^{-1}(U) = \{x \in X \mid f(x) \in U\}$. }

%\vesp

%\questao Mostre que um homomorfismo do anel $A$ no anel $B$ {\'e} um
%monomorfismo se, e somente se, $Ker(f) = \{0_A\}$.

%\vesp

\questao Sejam os an{\'e}is $A = \{ a + b\sqrt{-2} \mid a,\ b \in \q\}$ e $B = M_2(\q)$.
\begin{enumerate}[label=({\alph*})]
\item Mostre que $f : A \to B$ dada por
\[
f(a + b\sqrt{-2}) =
\begin{pmatrix}
a & -2b\\
b & a
\end{pmatrix}
\]
{\'e} um homomorfismo.
\item $f$ {\'e} um isomorfimo?
\end{enumerate}

\questao {\'E} verdadeiro ou falso: $\z$ e $\z_{m}$ para $m > 1$ s{\~a}o an{\'e}is
isomorfos.

\vesp

\questao Considere os seguintes an{\'e}is: $(\real, +, \cdot)$ e $(\real, \oplus, \odot)$, sendo $a \oplus b = a + b + 1$ e $a \odot b = a + b + ab$. Mostre que $f : \real \to \real$ dado por $f(x) = x + 1$, para todo $x \in \real$, {\'e} um isomorfimos de $(\real, \oplus, \odot)$ em $(\real, +, \cdot)$.

\vesp

\questao Seja $A$ um anel de integridade. Mostre que se $x \in A$ \'e tal que $x^2 = 1$, ent\~ao $x = 1$ ou $x = -1$.

\vesp

\questao Seja $A$ \'e um anel de integridade. Mostre que se $x \in A$ \'e tal que $x ^2 = x$, ent\~ao $x = 0$ ou $x = -1$.

\vesp

\questao Seja $A$ um anel com unidade tal que $x^2 = x$ para todo $x \in A$. Mostre que $A$ \'e um anel de integridade se, e somente se, $A = \{0, 1\}$.

% \questao Mostre que nenhuma aplica{\c c}{\~a}o $f : A \to B$, onde $A = \{ x + y\sqrt{2} \mid x,\ y \in \q\}$ e $B = \{ x + y\sqrt{3} \mid x,\ y \in \q\}$ {\'e} isomorfismo.

% {\it Sugest{\~a}o: Observe que se $f$ fosse um isomorfimo de $A$ em $B$, ent{\~a}o $f(\sqrt{2}) = a + b\sqrt{3}$. Calcule a seguir $f(2) = 2$ a partir de $f(\sqrt{2}) = a + b\sqrt{3}$}

% \vesp

% \questao Seja $I$ um ideal de um anel $A$. Mostre que se $A$ possui unidade, ent{\~a}o $A/I$ tamb{\'e}m possui unidade.

% \vesp

% \questao Suponha que $A$ {\'e} um anel com unidade e $I$ um ideal de $A$. Mostre que $a + I \in A/I$ {\'e} invers{\'\i}vel se, e somente se, existe $r \in A$ de modo que $ar - 1 \in I$.

% \vesp

% \questao D{\^e} um exemplo de um anel de integridade $A$ e de um ideal $I$ em $A$ tal que $A/I$ n{\~a}o {\'e} de integridade.

\end{document}