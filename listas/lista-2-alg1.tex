%!TEX program = xelatex
%!TEX encoding = ISO-8859-1
\documentclass[12pt]{article}

\usepackage{amssymb}
\usepackage{amsmath,amsfonts,amsthm,amstext,mathabx}
\usepackage[brazil]{babel}
%\usepackage[latin1]{inputenc}
\usepackage{graphicx}
\graphicspath{{/home/jfreitas/Dropbox/imagens-latex/}{/Volumes/Vader/Dropbox/imagens-latex/}{D:/Dropbox/imagens-latex/}}
\usepackage{enumitem}
\usepackage{multicol}
\usepackage[all]{xy}

\setlength{\topmargin}{-1.0in}
\setlength{\oddsidemargin}{0in}
\setlength{\textheight}{10.1in}
\setlength{\textwidth}{6.5in}
\setlength{\baselineskip}{12mm}

\newcounter{exercicios}
\setcounter{exercicios}{0}
\newcommand{\questao}{
\addtocounter{exercicios}{1}
\noindent{\bf Exerc{\'\i}cio \arabic{exercicios}: }}

\newcommand{\equi}{\Leftrightarrow}
\newcommand{\bic}{\leftrightarrow}
\newcommand{\cond}{\rightarrow}
\newcommand{\impl}{\Rightarrow}
\newcommand{\nao}{\sim}
\newcommand{\sub}{\subseteq}
\newcommand{\e}{\ \wedge\ }
\newcommand{\ou}{\ \vee\ }
\newcommand{\vaz}{\emptyset}
\newcommand{\nsub}{\nsubset}
\renewcommand{\sin}{{\rm sen\,}}

\newcommand{\n}{\mathbb{N}}
\newcommand{\z}{\mathbb{Z}}
\newcommand{\real}{\mathbb{R}}
\newcommand{\vesp}{\vspace{0.2cm}}
\newcommand{\subne}{\subsetneqq}


\newcommand{\compcent}[1]{\vcenter{\hbox{$#1\circ$}}}
\newcommand{\comp}{\mathbin{\mathchoice
{\compcent\scriptstyle}{\compcent\scriptstyle}
{\compcent\scriptscriptstyle}{\compcent\scriptscriptstyle}}}

\begin{document}

\pagestyle{empty}

\begin{figure}[h]
        \begin{minipage}[c]{1.7cm}
        \includegraphics[width=1.7cm]{unb.pdf}
        \end{minipage}%
        \hspace{0pt}
        \begin{minipage}[c]{4in}
          {Universidade de Brasília} \\
          {Departamento de Matemática}
\end{minipage}
\end{figure}
\vspace{-1cm}\hrule


\begin{center}
{\Large\bf {\'A}lgebra 1 - Turma C -- 1$^{o}$/2018} \\ \vspace{9pt} {\large\bf
  $2^{\underline{a}}$ Lista de Exerc{\'\i}cios -- Rela\c{c}\~oes de equival{\^e}ncia}\\
\vspace{9pt} Prof. Jos{\'e} Ant{\^o}nio O. Freitas
\end{center}
\hrule

\vspace{.6cm}

\questao Quais das rela{\c c}{\~o}es abaixo s{\~a}o rela{\c c}{\~o}es de equival{\^e}ncia sobre $E = \{a,b,c\}$?
\begin{enumerate}[label={\alph*})]
\item $R_1 = \{(a,a),(a,b),(b,a),(b,b),(c,c)\}$;
\item $R_2 = \{(a,a),(a,b),(b,a),(b,b),(b,c)\}$;
\item $R_3 = \{(a,a),(b,b),(b,c),(c,b),(a,c),(c,a)\}$;
\item $R_4 = E \times E$;
\item $R_5 = \vaz$.
\end{enumerate}

\vesp

\questao Seja $m \in \z$, $m > 1$. Defina $R \sub \z\times\z$ como
\[
  R = \{(x,y) \in \z\times\z \mid x - y = km, \mbox{ para algum } k \in \z\}.
\]
Mostre que $R$ \'e uma rela\c{c}\~ao de equival\^encia sobre $\z$.

\vesp

\questao Determinar todas as rela{\c c}{\~o}es de equival{\^e}ncia
$R$ sobre $A$ e os respectivos conjuntos quocientes $A/R$ para:
\begin{enumerate}[label={\alph*})]
\item $A=\{a\}$;
\item $A=\{a,b\}$;
\item $A=\{a,b,c\}$;
\item $A=\{a,b,c,d\}$.
\end{enumerate}

\vesp

\questao Quais das seguintes senten{\c c}as definem uma rela{\c c}{\~a}o de equival{\^e}ncia no conjunto $A$ dado?
\begin{enumerate}[label={\alph*})]
\item $aRb$ se, e s{\'o} se, existe $k \in \n$ tal que $a - b = 3k$, $A = \n$.
\item $aRb$ quando $a \mid b$, $A = \n$.
\item $aRb$ quando $a \le b$, $A = \z$.
\item $xRy$ quando $xy \ge 0$, $ A = \real$.
\end{enumerate}

\vesp


\questao Seja $A=\n\times \n^*$. Considere a seguinte
rela{\c c}{\~a}o sobre $A$:
\[
(a,b)R (c,d) \Leftrightarrow a + b = c + d.
\]
Mostre que $R$ {\'e} uma rela{\c c}{\~a}o de equival{\^e}ncia sobre $A$.

\vesp

\questao Seja $A = \real$ e considere o conjunto definido por
\[
  (a,b)R(c,d) \Leftrightarrow 2a - b = 2c - d.
\]
Mostre que $R$ \'e uma rela\c{c}\~ao de equival\^encia sobre $\real$.

\vesp

\questao Para pontos $(a, b)$, $(c, d) \in \real^2$ defina $(a, b) S (c, d)$ quando $a^2 + b^2 = c^2 + d^2$.
\begin{enumerate}[label={\alph*})]
  \item Prove que $S$ \'e uma rela\c{c}\~ao de equival\^encia em $\real^2$.
  \item Liste todos os elementos no conjunto $\{(x, y) \in \real \mid (x, y) S (0, 0)\}$.
  \item Liste cinco elementos distintos no conjunto $\{(x, y) \in \real \mid (x, y) S (1, 0)\}$.
\end{enumerate}

\vesp

\questao Sejam $E = \{-3, -2, -1, 0, 1, 2, 3\}$ e $R = \{(x, y) \in E \times E \mid x + |x| = y + |y|\}$. Mostrar que $R$ \'e uma rela\c{c}\~ao de equival\^encia e descrever $E/R$.

\vesp

\questao Seja $A=\z\times \z^*$, onde
$\mathbb{Z}^*=\mathbb{Z}\setminus \{0\}$. Para $(a,b), (c,d) \in
A$, considere a seguinte rela{\c c}{\~a}o
\[
(a,b)R (c,d) \mbox{ quando } ad=bc.
\]
\begin{enumerate}[label={\alph*})]
\item Mostre que $R$ {\'e} uma rela{\c c}{\~a}o de equival{\^e}ncia sobre $A$.
\item Descreva a classe de equival{\^e}ncia $\overline{(0,1)}$, $\overline{(1,1)}$, $\overline{(1,2)}$, $\overline{(2,1)}$, $\overline{(2,2)}$, $\overline{(2,3)}$.
\end{enumerate}

\vesp

\questao Considere a seguinte rela{\c c}{\~a}o sobre $\mathbb{C}$:
\[
(x+yi)R(r+si) \mbox{ quando } x^2+y^2=r^2+s^2.
\]
\begin{enumerate}[label={\alph*})]
\item Mostre que $R$ {\'e} rela{\c c}{\~a}o de equival{\^e}ncia.
\item Descreva a classe de equival{\^e}ncia de $1+i$.
\end{enumerate}

\vesp

\questao Considere a seguinte relação sobre $\z$:
\[
	xRy \mbox{ quando }  x^2 - y^2 = 4k, \mbox{ para algum } k \in \z.
\]
\begin{enumerate}[label={\alph*})]
	\item Mostre que $R$ {\'e} rela{\c c}{\~a}o de equival{\^e}ncia.
	\item Determine todas as classes de equivalência de $R$.
\end{enumerate}
\vesp

\questao Seja $R$ uma rela{\c c}{\~a}o sobre $\rac$
definida da seguinte forma:
\[
xRy \mbox{ quando } x-y \in \mathbb{Z}.
\]
\begin{enumerate}[label={\alph*})]
\item Prove que $R$ {\'e} uma rela{\c c}{\~a}o de equival{\^e}ncia sobre $\rac$.
\item Descreva a classe $\bar{1}$.
\item Descreva a classe $\overline{1/2}$.
\end{enumerate}

\vesp

\questao A divisibilidade (ou seja, a rela{\c c}{\~a}o definida por $xRy$ se, e s{\'o}
se, $x \mid y$) {\'e} uma rela{\c c}{\~a}o de equival{\^e}ncia sobre $\z$?

\vesp

\questao Seja $R$ a seguinte rela{\c c}{\~a}o sobre $\z^*$:
\[
xRy \mbox{ quando } x\mid y \mbox{ e } y\mid x.
\]
Mostre que $R$ {\'e} uma rela{\c c}{\~a}o de equival{\^e}ncia sobre $\z^*$ e
descreva o conjunto quociente $\z^*/R$.

\vesp

\questao Seja $R = \{ (x, y) \in \real^2 \mid x - y \in \rac\}$. Provar que $R$ {\'e} uma rela{\c c}{\~a}o de equival{\^e}ncia e descrever as classes representadas por $1/2$ e $\sqrt{2}$.

\end{document}