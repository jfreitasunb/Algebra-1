%!TEX program = xelatex
%!TEX encoding = ISO-8859-1
%!TEX program = xelatex
% !TEX encoding = ISO-8859-1
\def\ano{2019}
\def\semestre{2}
\def\disciplina{\'Algebra 1}
\def\turma{C}

\documentclass[12pt]{exam}

\usepackage{caption}
\usepackage{amssymb}
\usepackage{amsmath,amsfonts,amsthm,amstext}
\usepackage[brazil]{babel}
% \usepackage[latin1]{inputenc}
\usepackage{graphicx}
\graphicspath{{/ArquivosLinux/OneDrive/imagens-latex/}{D:/OneDrive - unb.br/imagens-latex/}}
\usepackage{enumitem}
\usepackage{multicol}
\usepackage{answers}
\usepackage{tikz,ifthen}
\usetikzlibrary{lindenmayersystems}
\usetikzlibrary[shadings]
\Newassociation{solucao}{Solution}{ans}
\newtheorem{exercicio}{}

\setlength{\topmargin}{-1.0in}
\setlength{\oddsidemargin}{0in}
\setlength{\textheight}{10.1in}
\setlength{\textwidth}{6.5in}
\setlength{\baselineskip}{12mm}

\extraheadheight{0.7in}
\firstpageheadrule
\runningheadrule
\lhead{
        \begin{minipage}[c]{1.7cm}
        \includegraphics[width=1.7cm]{unb.pdf}
        \end{minipage}%
        \hspace{0pt}
        \begin{minipage}[c]{4in}
          {Universidade de Brasília} --
          {Departamento de Matemática}
\end{minipage}
\vspace*{-0.8cm}
}
% \chead{Universidade de Brasília - Departamento de Matemática}
% \rhead{}
% \vspace*{-2cm}

\extrafootheight{.5in}
\footrule
\lfoot{\disciplina\ - \semestre$^o$/\ano\ - Módulo \numeromodulo}
\cfoot{}
\rfoot{Página \thepage\ de \numpages}

\newcounter{exercicios}
\renewcommand{\theexercicios}{\arabic{exercicios}}

\newenvironment{questao}[1]{
\refstepcounter{exercicios}
\ifx&#1&
\else
   \label{#1}
\fi
\noindent\textbf{Exercício {\theexercicios}:}
}

\newcommand{\resp}[1]{
\noindent{\bf Exercício #1: }}

\def\ano{2024}
\def\semestre{1}
\def\disciplina{Álgebra 1}
\def\nomeabreviado{Álgebra 1}
\def\turma{1}

\newcommand{\im}{{\rm Im\,}}
\newcommand{\dlim}[2]{\displaystyle\lim_{#1\rightarrow #2}}
\newcommand{\minf}{+\infty}
\newcommand{\ninf}{-\infty}
\newcommand{\cp}[1]{\mathbb{#1}}
\newcommand{\sub}{\subseteq}
\newcommand{\n}{\mathbb{N}}
\newcommand{\z}{\mathbb{Z}}
\newcommand{\rac}{\mathbb{Q}}
\newcommand{\real}{\mathbb{R}}
\newcommand{\complex}{\mathbb{C}}

\newcommand{\vesp}[1]{\vspace{ #1  cm}}

\newcommand{\compcent}[1]{\vcenter{\hbox{$#1\circ$}}}
\newcommand{\comp}{\mathbin{\mathchoice
        {\compcent\scriptstyle}{\compcent\scriptstyle}
        {\compcent\scriptscriptstyle}{\compcent\scriptscriptstyle}}}
\renewcommand{\sin}{{\rm sen\,}}
\renewcommand{\tan}{{\rm tg\,}}
\renewcommand{\csc}{{\rm cossec\,}}
\renewcommand{\cot}{{\rm cotg\,}}
\renewcommand{\sinh}{{\rm senh\,}}

\begin{document}

\begin{center}
{\Large\bf \disciplina\ - Turma \turma\ -- \semestre$^{o}$/\ano} \\ \vspace{9pt} {\large\bf
  $2^{\underline{a}}$ Lista de Exerc{\'\i}cios -- Conjuntos}\\ \vspace{9pt} Prof. Jos{\'e} Ant{\^o}nio O. Freitas
\end{center}
\hrule

\vspace{.6cm}

\questao Quais das rela{\c c}{\~o}es abaixo s{\~a}o rela{\c c}{\~o}es de equival{\^e}ncia sobre $E = \{a,b,c\}$?
\begin{enumerate}[label={\alph*})]
\item $R_1 = \{(a,a);(a,b);(b,a);(b,b);(c,c)\}$
\item $R_2 = \{(a,a);(a,b);(b,a);(b,b);(b,c)\}$
\item $R_3 = \{(a,a);(b,b);(b,c);(c,b);(a,c);(c,a)\}$
\item $R_4 = E \times E$
\item $R_5 = \vaz$
\end{enumerate}

\vspace{.3cm}

\questao Seja $m \in \z$, $m > 1$. Defina $R \sub \z\times\z$ como
\[
  R = \{(x,y) \in \z\times\z \mid x - y = km, \mbox{ para algum } k \in \z\}.
\]
Mostre que $R$ \'e uma rela\c{c}\~ao de equival\^encia sobre $\z$.

\vspace{.3cm}

\questao Determinar todas as rela{\c c}{\~o}es de equival{\^e}ncia
$R$ sobre $A$ e os respectivos conjuntos quocientes $A/R$ para:
\begin{enumerate}[label={\alph*})]
\item $A=\{a\}$;
\item $A=\{a,b\}$;
\item $A=\{a,b,c\}$;
\item $A=\{a,b,c,d\}$.
\end{enumerate}

\vspace{.3cm}

\questao Quais das seguintes senten{\c c}as definem uma rela{\c c}{\~a}o de equival{\^e}ncia no conjunto $A$ dado?
\begin{enumerate}[label={\alph*})]
\item $aRb$ se, e s{\'o} se, existe $k \in \n$ tal que $a - b = 3k$, $A = \n$.
\item $aRb$ quando $a \mid b$, $A = \n$.
\item $aRb$ quando $a \le b$, $A = \z$.
\item $xRy$ quando $xy \ge 0$, $ A = \real$.
\end{enumerate}

\vspace{.3cm}


\questao Seja $A=\n\times \n^*$. Considere a seguinte
rela{\c c}{\~a}o sobre $A$:
\[
(a,b)R (c,d) \mbox{ quando } a + b = c + d.
\]
Mostre que $R$ {\'e} uma rela{\c c}{\~a}o de equival{\^e}ncia sobre $A$.

\vspace{.3cm}

\questao Seja $A = \real$ e considere o conjunto definido por
\[
  (a,b)R(c,d) \mbox{ quando } 2a - b = 2c - d.
\]
Mostre que $R$ \'e uma rela\c{c}\~ao de equival\^encia sobre $\real$.

\vspace{.3cm}

\questao Seja $A = \real^3$. Dados $(x, y, z)$, $(\alpha, \beta, \gamma) \in \real^3$, defina $(x, y, z) R (\alpha, \beta, \gamma)$ quando $z = \gamma$. Mostre que $R$ é uma relação de equivalência sobre $\real^3.$

\vspace{.3cm}

\questao Seja $A = \real^3$. Dados $u = (x_1, y_1, z_1)$, $v = (x_2, y_2, z_2) \in \real^3$ defina
\[
    u\cdot v = x_1x_2 + y_1y_2 + z_1z_2.
\]
Tome um elemento fixo $w = (\alpha, \beta, \gamma) \in \real^3$ e defina
\[
    u \sim v \mbox{ quando } u \cdot w = v \cdot w.
\]
Mostre que $\sim$ é uma relação de equivalência sobre $\real^3$.

\vspace{.3cm}

\questao Para pontos $(a, b)$, $(c, d) \in \real^2$ defina $(a, b) S (c, d)$ quando $a^2 + b^2 = c^2 + d^2$.
\begin{enumerate}[label={\alph*})]
  \item Prove que $S$ \'e uma rela\c{c}\~ao de equival\^encia em $\real^2$.
  \item Liste todos os elementos no conjunto $\{(x, y) \in \real \mid (x, y) S (0, 0)\}$.
  \item Liste cinco elementos distintos no conjunto $\{(x, y) \in \real \mid (x, y) S (1, 0)\}$.
\end{enumerate}

\vspace{.3cm}

\questao Sejam $E = \{-3, -2, -1, 0, 1, 2, 3\}$ e $R = \{(x, y) \in E \times E \mid x + |x| = y + |y|\}$. Mostrar que $R$ \'e uma rela\c{c}\~ao de equival\^encia e descrever $E/R$.

\vspace{.3cm}

\questao Seja $A=\z\times \z^*$, onde
$\mathbb{Z}^*=\mathbb{Z}\setminus \{0\}$. Para $(a,b), (c,d) \in
A$, considere a seguinte rela{\c c}{\~a}o
\[
(a,b)R (c,d) \mbox{ quando } ad=bc.
\]
\begin{enumerate}[label={\alph*})]
\item Mostre que $R$ {\'e} uma rela{\c c}{\~a}o de equival{\^e}ncia sobre $A$.
\item Descreva a classe de equival{\^e}ncia $\overline{(0,1)}$, $\overline{(1,1)}$, $\overline{(1,2)}$, $\overline{(2,1)}$, $\overline{(2,2)}$, $\overline{(2,3)}$.
\end{enumerate}

\vspace{.3cm}

\questao Considere a seguinte rela{\c c}{\~a}o sobre $\mathbb{C}$:
\[
(x+yi)R(r+si) \mbox{ quando } x^2+y^2=r^2+s^2.
\]
\begin{enumerate}[label={\alph*})]
\item Mostre que $R$ {\'e} rela{\c c}{\~a}o de equival{\^e}ncia.
\item Descreva a classe de equival{\^e}ncia de $1+i$.
\end{enumerate}

\vspace{.3cm}

\questao Considere a seguinte rela\c{c}\~ao sobre $\z$:
\[
	xRy \mbox{ quando }  x^2 - y^2 = 4k, \mbox{ para algum } k \in \z.
\]
\begin{enumerate}[label={\alph*})]
	\item Mostre que $R$ {\'e} rela{\c c}{\~a}o de equival{\^e}ncia.
	\item Determine todas as classes de equival\^encia de $R$.
\end{enumerate}
\vspace{.3cm}

\questao Seja $R$ uma rela{\c c}{\~a}o sobre $\rac$
definida da seguinte forma:
\[
xRy \mbox{ quando } x-y \in \mathbb{Z}.
\]
\begin{enumerate}[label={\alph*})]
\item Prove que $R$ {\'e} uma rela{\c c}{\~a}o de equival{\^e}ncia sobre $\rac$.
\item Descreva a classe $\bar{1}$.
\item Descreva a classe $\overline{1/2}$.
\end{enumerate}

\vspace{.3cm}

\questao Defina
\[
	H = \{2^m \mid m \in \z\} \mbox{ e } \rac^+ = \{x \in \rac \mid x > 0\}.
\]
Seja $R$ dado por
\[
	R = \left\{(x,y) \in \rac^+\times \rac^+ : \frac{x}{y} \in H\right\}.
\]
\begin{enumerate}[label={\alph*})]
	\item Mostre que $R$ \'e uma rela\c{c}\~ao de equival\^encia em $\rac^+$.
	\item Determine a classe de equival\^encia de $3$.
\end{enumerate}
\vspace{.3cm}

\questao A divisibilidade (ou seja, a rela{\c c}{\~a}o definida por $xRy$ se, e s{\'o}
se, $x \mid y$) {\'e} uma rela{\c c}{\~a}o de equival{\^e}ncia sobre $\z$?

\vspace{.3cm}

\questao Seja $R$ a seguinte rela{\c c}{\~a}o sobre $\z^*$:
\[
xRy \mbox{ quando } x\mid y \mbox{ e } y\mid x.
\]
Mostre que $R$ {\'e} uma rela{\c c}{\~a}o de equival{\^e}ncia sobre $\z^*$ e
descreva o conjunto quociente $\z^*/R$.

\vspace{.3cm}

\questao Seja $R = \{ (x, y) \in \real^2 \mid x - y \in \rac\}$. Prove que $R$ {\'e} uma rela{\c c}{\~a}o de equival{\^e}ncia e descrever as classes representadas por $1/2$ e $\sqrt{2}$.

\vspace{.3cm}

\questao Seja $A$ um conjunto não vazio. Suponha que $R \sub A \times A$ é tal que:
\begin{enumerate}
    \item Para todo $x \in A$, $(x,x) \in R$;
    \item Para todos $x$, $y$, $z \in A$ se $(x, y) \in R$ e $(x,z) \in R$, então $(y,z) \in R$.
\end{enumerate}
Mostre que $R$ é uma relação de equivalência sobre $A$.

\vspace{.3cm}

\questao Seja $A$ um conjunto não vazio. Suponha que $R_1$ e $R_2$ são relações de equivalências sobre $A$. Defina
\[
    R = \{(x,y) \in A\times A \mid (x,y) \in R_1 \mbox{ e } (x,y) \in R_2\}.
\]
Mostre que:
\begin{enumerate}
    \item $R$ é uma relação de equivalência sobre $A$.
    \item Se $c_R(x)$, $c_{R_1}(x)$ e $c_{R_2}(x)$ denota a classe de equivalência de $x$ com respeito a $R$, $R_1$ e $R_2$, mostre que $c_R(a) = c_{R_1}(a) \cap c_{R_2}(a)$.
\end{enumerate}
\end{document}