%!TEX program = xelatex
%!TEX encoding = ISO-8859-1
\documentclass[12pt]{article}

\usepackage{amssymb}
\usepackage{amsmath,amsfonts,amsthm,amstext,mathabx}
\usepackage[brazil]{babel}
%\usepackage[latin1]{inputenc}
\usepackage{graphicx}
\graphicspath{{/home/jfreitas/Dropbox/imagens-latex/}{/Users/jfreitas/Dropbox/imagens-latex/}{D:/Dropbox/imagens-latex/}}
\usepackage{enumitem}
\usepackage{multicol}
\usepackage[all]{xy}

\setlength{\topmargin}{-1.0in}
\setlength{\oddsidemargin}{0in}
\setlength{\textheight}{10.1in}
\setlength{\textwidth}{6.5in}
\setlength{\baselineskip}{12mm}

\newcounter{exercicios}
\setcounter{exercicios}{0}
\newcommand{\questao}{
\addtocounter{exercicios}{1}
\noindent{\bf Exerc{\'\i}cio \arabic{exercicios}: }}

\newcommand{\equi}{\Leftrightarrow}
\newcommand{\bic}{\leftrightarrow}
\newcommand{\cond}{\rightarrow}
\newcommand{\impl}{\Rightarrow}
\newcommand{\nao}{\sim}
\newcommand{\sub}{\subseteq}
\newcommand{\e}{\ \wedge\ }
\newcommand{\ou}{\ \vee\ }
\newcommand{\vaz}{\emptyset}
\newcommand{\nsub}{\nsubset}
\renewcommand{\sin}{{\rm sen\,}}

\newcommand{\n}{\mathbb{N}}
\newcommand{\z}{\mathbb{Z}}
\newcommand{\q}{\mathbb{Q}}
\newcommand{\cmp}{\mathbb{C}}
\newcommand{\real}{\mathbb{R}}
\newcommand{\vesp}{\vspace{0.2cm}}
\newcommand{\subne}{\subsetneqq}


\newcommand{\compcent}[1]{\vcenter{\hbox{$#1\circ$}}}
\newcommand{\comp}{\mathbin{\mathchoice
{\compcent\scriptstyle}{\compcent\scriptstyle}
{\compcent\scriptscriptstyle}{\compcent\scriptscriptstyle}}}

\begin{document}
\pagestyle{empty}

\begin{figure}[h]
        \begin{minipage}[c]{1.7cm}
        \includegraphics[width=1.7cm]{unb.pdf}
        \end{minipage}%
        \hspace{0pt}
        \begin{minipage}[c]{4in}
          {Universidade de Bras{\'\i}lia} \\
          {Departamento de Matem{\'a}tica}
\end{minipage}
\end{figure}
\vspace{-1cm}\hrule

\begin{center}
{\Large\bf {\'A}lgebra 1 - Turma D -- 1$^{o}$/2016} \\ \vspace{9pt} {\large\bf
  $2^{\underline{a}}$ Lista de Exerc{\'\i}cios -- Rela\c{c}\~oes}\\
\vspace{9pt} Prof. Jos{\'e} Ant{\^o}nio O. Freitas
\end{center}
\hrule

\vspace{.6cm}

% \questao Prove por indu{\c c}{\~a}o que:
% \begin{enumerate}[label={\alph*})]
% \item $1^2 + 2^2 + \cdots + n^2 = \dfrac{n(n + 1)(2n + 1)}{6}$, $n \in \n$.

% %\item $(1 + p)^n \ge 1 + np$, $n \in \n$, $p > -1$.

% \item $1 + 3 + 5 + \cdots + (2n - 1) = n^2$, $n \ge 1$.

% \item $2.1 + 2.2 + 2.3 + \cdots + 2n = n^2 + n$, $n \ge 1$.

% \item Qualquer n\'umero inteiro positivo $n \ge 8$ pode ser escrito como a soma de $3'$s e $5'$s.

% \item $1 + q + q^2 + \cdots + q^n = \dfrac{1 - q^{n + 1}}{1 - q}$, $q \ne
%   1$.

% \item $3^{2n + 1} + 2^{n + 2}$ {\'e} m\'ultiplo de $7$, para todo $n \in \n$.

% \item Para todo $n \ge 0$, $9^n - 1$ {\'e} m\'ultiplo de 8.

% \item $1^3 + 2^3 + \cdots + n^3 = \left[\dfrac{n(n+1)}{2}\right]^2$, $n \ge 1$.
% \end{enumerate}


\vesp

\questao Dizer se cada um dos seguintes subconjuntos de $\n$ \'e ou n\~ao totalmente ordenado pela rela\c{c}\~ao de divisibilidade:
\begin{enumerate}[label={\alph*})]
  \item $\{24, 2, 6\}$
  \item $\{3,15,5\}$
  \item $\{15,5,30\}$
  \item $\n$
\end{enumerate}

\vesp

\questao Seja $\cmp$ o conjunto dos n\'umeros complexos e sejam $x = a + bi$ e $y = c + di$ dois elementos de $\cmp$. Defina $x R y$ se, e somente se, $a \le c$ e $b \le d$. Mostre que $R$ \'e uma rela\c{c}\~ao de ordem parcial em $\cmp$.

\vesp

\questao Em $\n \times \n$ defina $(a, b) \le (c, d)$ se, e somente se, $a \mid c$ e $b \le d$. Mostre que $\le$ \'e uma rela\c{c}\~ao de ordem parcial em $\n \times \n$.

\questao Quais das rela{\c c}{\~o}es abaixo s{\~a}o rela{\c c}{\~o}es de equival{\^e}ncia sobre $E = \{a,b,c\}$?
\begin{enumerate}[label={\alph*})]
\item $R_1 = \{(a,a),(a,b),(b,a),(b,b),(c,c)\}$;
\item $R_2 = \{(a,a),(a,b),(b,a),(b,b),(b,c)\}$;
\item $R_3 = \{(a,a),(b,b),(b,c),(c,b),(a,c),(c,a)\}$;
\item $R_4 = E \times E$;
\item $R_5 = \vaz$.
\end{enumerate}
\questao Determinar todas as rela{\c c}{\~o}es de equival{\^e}ncia
$R$ sobre $A$ e os respectivos conjuntos quocientes $A/R$ para:
\begin{enumerate}[label={\alph*})]
\item $A=\{a\}$;
\item $A=\{a,b\}$;
\item $A=\{a,b,c\}$;
\item $A=\{a,b,c,d\}$.
\end{enumerate}

\vesp

\questao Quais das seguintes senten{\c c}as definem uma rela{\c c}{\~a}o de equival{\^e}ncia em $\n$?
\begin{enumerate}[label={\alph*})]
\item $aRb$ se, e s{\'o} se, existe $k \in \n$ tal que $a - b = 3k$.
\item $a \mid b$.
\item $a \le b$.
%\item $mdc(a, b) = 1$.
\item $x + y =10$.
\end{enumerate}
\vesp

\questao Seja $E$ um conjunto n{\~a}o vazio. Dados $X$, $Y \in \mathcal{P}(E)$, o conjunto das partes de $E$, mostre que as rela{\c c}{\~o}es $R$ e $S$ s{\~a}o de equival{\^e}ncia em  $\mathcal{P}(E)$.
\begin{enumerate}[label={\alph*})]
\item $X~R~Y \Leftrightarrow X\cap A=Y\cap A$.
\item $X~R~Y \Leftrightarrow X\cup A=Y\cup A$.
\end{enumerate}
onde $A$ {\'e} um subconjunto fixo de $E$.

\vesp

\questao Seja $A=\n\times \n^*$. Considere a seguinte
rela{\c c}{\~a}o sobre $A$:
\[
(a,b)R (c,d) \Leftrightarrow a + b = c + d.
\]
Mostre que $R$ {\'e} uma rela{\c c}{\~a}o de equival{\^e}ncia sobre $A$.

\vesp

\questao Seja $A=\z\times \z^*$, onde
$\mathbb{Z}^*=\mathbb{Z}\setminus \{0\}$. Para $(a,b), (c,d) \in
A$, considere a seguinte rela{\c c}{\~a}o
\[
(a,b)R (c,d) \Leftrightarrow ad=bc.
\]
\begin{enumerate}[label={\alph*})]
\item Mostre que $R$ {\'e} uma rela{\c c}{\~a}o de equival{\^e}ncia sobre $A$.
\item Descreva a classe de equival{\^e}ncia $\overline{(0,1)}$, $\overline{(1,1)}$, $\overline{(1,2)}$, $\overline{(2,1)}$, $\overline{(2,2)}$, $\overline{(2,3)}$.
\end{enumerate}

\vesp

\questao Considere a seguinte rela{\c c}{\~a}o sobre $\mathbb{C}$:
\[
(x+yi)R(r+si) \Leftrightarrow x^2+y^2=r^2+s^2.
\]
\begin{enumerate}[label={\alph*})]
\item Mostre que $R$ {\'e} rela{\c c}{\~a}o de equival{\^e}ncia.
\item Descreva a classe de equival{\^e}ncia de $1+i$.
\end{enumerate}

\vesp

\questao Seja $R$ uma rela{\c c}{\~a}o sobre $\q$
definida da seguinte forma:
\[
xRy \Leftrightarrow x-y \in \mathbb{Z}.
\]
\begin{enumerate}[label={\alph*})]
\item Prove que $R$ {\'e} uma rela{\c c}{\~a}o de equival{\^e}ncia sobre $\q$.
\item Descreva a classe $\bar{1}$.
\item Descreva a classe $\overline{1/2}$.
\end{enumerate}

\vesp

\questao A divisibilidade (ou seja, a rela{\c c}{\~a}o definida por $xRy$ se, e s{\'o}
se, $x \mid y$) {\'e} uma rela{\c c}{\~a}o de equival{\^e}ncia sobre $\z$?

\vesp

\questao Seja $R$ a seguinte rela{\c c}{\~a}o sobre $\z^*$:
\[
xRy \Leftrightarrow x\mid y \mbox{ e } y\mid x.
\]
Mostre que $R$ {\'e} uma rela{\c c}{\~a}o de equival{\^e}ncia sobre $\z^*$ e
descreva o conjunto quociente $\z^*/R$.

\vesp

\questao Seja $R = \{ (x, y) \in \real^2 \mid x - y \in \q\}$. Provar que $R$ {\'e} uma rela{\c c}{\~a}o de equival{\^e}ncia e descrever as classes representadas por $1/2$ e $\sqrt{2}$.

\end{document}