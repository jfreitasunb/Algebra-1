%!TEX program = xelatex

\def\numerosemana{13}

\documentclass[12pt]{exam}

\def\ano{2022}
\def\semestre{1}
\def\disciplina{\'Algebra 1}
\def\turma{2}

\usepackage{caption}
\usepackage{amssymb}
\usepackage{amsmath,amsfonts,amsthm,amstext}
\usepackage[brazil]{babel}
% \usepackage[latin1]{inputenc}
\usepackage{graphicx}
\graphicspath{{/ArquivosLinux/OneDrive/imagens-latex/}{D:/OneDrive - unb.br/imagens-latex/}}
\usepackage{enumitem}
\usepackage{multicol}
\usepackage{answers}
\usepackage{tikz,ifthen}
\usetikzlibrary{lindenmayersystems}
\usetikzlibrary[shadings]
\Newassociation{solucao}{Solution}{ans}
\newtheorem{exercicio}{}

\setlength{\topmargin}{-1.0in}
\setlength{\oddsidemargin}{0in}
\setlength{\textheight}{10.1in}
\setlength{\textwidth}{6.5in}
\setlength{\baselineskip}{12mm}

\extraheadheight{0.7in}
\firstpageheadrule
\runningheadrule
\lhead{
        \begin{minipage}[c]{1.7cm}
        \includegraphics[width=1.7cm]{unb.pdf}
        \end{minipage}%
        \hspace{0pt}
        \begin{minipage}[c]{4in}
          {Universidade de Brasília} --
          {Departamento de Matemática}
\end{minipage}
\vspace*{-0.8cm}
}
% \chead{Universidade de Brasília - Departamento de Matemática}
% \rhead{}
% \vspace*{-2cm}

\extrafootheight{.5in}
\footrule
\lfoot{\disciplina\ - \semestre$^o$/\ano\ - Módulo \numeromodulo}
\cfoot{}
\rfoot{Página \thepage\ de \numpages}

\newcounter{exercicios}
\renewcommand{\theexercicios}{\arabic{exercicios}}

\newenvironment{questao}[1]{
\refstepcounter{exercicios}
\ifx&#1&
\else
   \label{#1}
\fi
\noindent\textbf{Exercício {\theexercicios}:}
}

\newcommand{\resp}[1]{
\noindent{\bf Exercício #1: }}

\def\ano{2024}
\def\semestre{1}
\def\disciplina{Álgebra 1}
\def\nomeabreviado{Álgebra 1}
\def\turma{1}

\newcommand{\im}{{\rm Im\,}}
\newcommand{\dlim}[2]{\displaystyle\lim_{#1\rightarrow #2}}
\newcommand{\minf}{+\infty}
\newcommand{\ninf}{-\infty}
\newcommand{\cp}[1]{\mathbb{#1}}
\newcommand{\sub}{\subseteq}
\newcommand{\n}{\mathbb{N}}
\newcommand{\z}{\mathbb{Z}}
\newcommand{\rac}{\mathbb{Q}}
\newcommand{\real}{\mathbb{R}}
\newcommand{\complex}{\mathbb{C}}

\newcommand{\vesp}[1]{\vspace{ #1  cm}}

\newcommand{\compcent}[1]{\vcenter{\hbox{$#1\circ$}}}
\newcommand{\comp}{\mathbin{\mathchoice
        {\compcent\scriptstyle}{\compcent\scriptstyle}
        {\compcent\scriptscriptstyle}{\compcent\scriptscriptstyle}}}
\renewcommand{\sin}{{\rm sen\,}}
\renewcommand{\tan}{{\rm tg\,}}
\renewcommand{\csc}{{\rm cossec\,}}
\renewcommand{\cot}{{\rm cotg\,}}
\renewcommand{\sinh}{{\rm senh\,}}

\begin{document}

\begin{center}

    {\Large\bf \disciplina\ - Turma \turma\ -- \semestre$^{o}$/\ano} \\ \vspace{9pt} {\large\bf
        Lista de Exerc{\'\i}cios -- Semana \numerosemana}\\ \vspace{9pt} Prof. Jos{\'e} Ant{\^o}nio O. Freitas
    \end{center}
    \hrule

    \vspace{.6cm}

    \questao{} Se $G = [a]$ \'e um grupo c{\'\i}clico tal que
    \[
        a^r \ne a^s \mbox{ sempre que } r \ne s.
    \]
    Mostre que a fun\c{c}\~ao $f : \z \to G$ por $f(r) = a^r$ \'e um isomorfismo de grupos.

    \vspace{.3cm}

    \questao{} Seja $G = [a]$ um grupo c{\'\i}clico de ordem finita igual a $m$. Mostre que a fun\c{c}\~ao $f : \z_m \to G$ dada por $f(\overline{x}) = a^x$ \'e um isomorfismo de grupos.

    \vspace{.3cm}

    \questao{} Construa a tabela de multiplica\c{c}\~ao de um grupo $G = \{e, a, b, c\}$ que seja isomorfo ao grupo multiplicativo $H = \{1, i, -1, -i\}$.
    \textit{Sugest\~ao:} Fa\c{c}a a seguinte associa\c{c}\~ao $e \to 1$, $a \to i$, $b \to -1$, $c \to -i$.

    \vspace{.3cm}

    \questao{} Construa a tabela de multiplica\c{c}\~ao de um grupo $G = \{e, a, b, c\}$ que seja isomorfo ao grupo $(\z_5^*, \otimes)$. Em seguida, resolva em $G$ a equa\c{c}\~ao $axb^{-1} = c^2$.
    \textit{Sugest\~ao:} Fa\c{c}a a seguinte associa\c{c}\~ao $e \to \overline{1}$, $a \to \overline{2}$, $b \to \overline{3}$, $c \to \overline{4}$.

    \vspace{.3cm}

    \questao{} Sabendo que $G = \{e, a, b, c, d, f\}$ \'e um grupo multiplicativo isomorfo ao grupo aditivo $\z_6$, onde $e$ \'e o elemento neutro de $G$ responda aos seguintes itens:
    \begin{enumerate}[label=({\alph*})]
      \item Construa a tabela de multiplica\c{c}\~ao de $G$.

      \item Calcule $a^2$, $b^{-2}$ e $c^{-3}$.

      \item Obtenha $x \in G$ tal que $bxc = a^{-1}$.
    \end{enumerate}
    \textit{Sugest\~ao:} Fa\c{c}a a seguinte associa\c{c}\~ao $e \to \overline{0}$, $a \to \overline{1}$, $b \to \overline{2}$, $c \to \overline{2}$, $d \to \overline{4}$, $f \to \overline{5}$.

    \vspace{.3cm}

    \questao{} Mostre que $f : \z \to 2\z$ dada por $f(n) = 2n$, para todo $n \in \z$, \'e um isomorfismo do grupo aditivo $\z$ no grupo aditivo $2\z = \{0, \pm 2, \pm 4, \dots\}$.

    \vspace{.3cm}

    \questao{} Seja $a \in \real_+^*$ e $a \ne 1$.
    \begin{enumerate}[label=({\alph*})]
      \item  Mostre que $G = \{a^n \mid n \in \z\}$ \'e um subgrupo de $(\real_+^*, \cdot)$.

      \item Mostre que $f : \z \to G$ tal que $f(n) = a^n$ \'e um isomorfismo de $(\z, +)$ em $(G, \cdot)$.
    \end{enumerate}

    \vspace{.3cm}

    \questao{} Mostre que $G = \{2^m3^n \mid m, n \in \z\}$ e $J = \{m + ni \mid m, n \in \z\}$ s\~ao subgrupos de $(\real_+^*, \cdot)$ e $(\complex, +)$, respectivamente, e que s\~ao isomorfos.

    \newpage

    \questao{} Seja $G$ um grupo. Considere o conjunto
    \[
        Aut(G) = \{f : G \to G \mid f \mbox{ \'e um isomorfismo}\}.
    \]
    \begin{enumerate}[label=({\alph*})]
      \item Mostre que $(Aut(G), \circ)$ \'e um grupo.

      \item Mostre que $Aut(\z) \cong \z_2$.
    \end{enumerate}

    \vspace{.3cm}

    \questao{} Considere o conjunto
    \[
      J = \left\{\begin{bmatrix}
        1 & 0\\0 & 1
      \end{bmatrix}, \begin{bmatrix}
        1 & 0\\0 & -1
      \end{bmatrix}, \begin{bmatrix}
        -1 & 0\\0 & 1
      \end{bmatrix}, \begin{bmatrix}
        -1 & 0\\0 & -1
      \end{bmatrix}\right\}.
    \]
    \begin{enumerate}[label=({\alph*})]
      \item Mostre que $J$ com a opera\c{c}\~ao de multiplica\c{c}\~ao de matrizes \'e um grupo.

      \item Esse grupo \'e abeliano?

      \item É verdade que $J \cong \z_4$? \textit{(Sugestão: observe a ordem dos elementos de $J$ e de $\z_4$.)}
    \end{enumerate}

    \vspace{.3cm}

    \questao{} Os grupos $(\z_7^*, \otimes)$ e $(\z_6, \oplus)$ s\~ao isomorfos?

    \vspace{.3cm}

    \questao{} Sejam $G_1$, $G_2$, $H_1$, $H_2$ grupos, suponha que $f : G_1 \to H_1$ e $g : G_2 \to H_2$ são isomorfismos de grupos. Defina $\phi : G_1 \times G_2 \to H_1 \times H_2$ por
    \[
        \phi(x_1, x_2) = (f(x_1), g(x_2))
    \]
    para todo $(x_1, x_2) \in G_1 \times G_2$. Prove que $\phi$ é um isomorfismo de grupos.

    \vspace{.3cm}

    \questao{} Defina que $f : \z_{30} \times \z_2 \to \z_{10} \times \z_6$ por $f(\overline{x}, \overline{y}) = (\overline{x}, \overline{y})$. Prove que $f$ é um isomorfismo de grupos.

    \vspace{.3cm}

    \questao{} Prove que $\complex^* = \complex - \{0\}$ é isomorfo ao subgrupo de $GL_2(\real)$ consistindo de matrizes da forma
    \[
        \begin{pmatrix}
            a & b\\
            -b & a
        \end{pmatrix}.
    \]

    \vspace{.3cm}

    \questao{} Seja $G = \real - \{-1\}$ e defina a operação binária em $G$ por
    \[
        a * b = a + b + ab.
    \]
    Prove que $G$ é um grupo com essa operação. Mostre também que $f : G \to \real^*$ dada por $f(x) = x + 1$ é um isomorfismo do grupo $G$ no grupo multiplicativo $\real^*$.

    \vspace{.3cm}

    \questao{} Sejam $G$ e $H$ grupos. Prove que $G \times H$ é isomorfo a $H \times G$.

    \vspace{.3cm}

    \questao{} Sejam $G_1$, $G_2$, $H_1$, $H_2$ grupos. Mostre que se $G_1 \cong G_2$ e $H_1 \cong H_2$, então $G_1 \times H_1 \cong G_2 \times H_2$.
\end{document}
