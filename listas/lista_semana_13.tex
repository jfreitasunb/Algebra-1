%!TEX program = xelatex
%!TEX encoding = ISO-8859-1
\def\ano{2020}
\def\semestre{1}
\def\disciplina{\'Algebra 1}
\def\turma{C}
\def\numerosemana{13}

\documentclass[12pt]{exam}

\usepackage{caption}
\usepackage{amssymb}
\usepackage{amsmath,amsfonts,amsthm,amstext}
\usepackage[brazil]{babel}
% \usepackage[latin1]{inputenc}
\usepackage{graphicx}
\graphicspath{{/ArquivosLinux/OneDrive/imagens-latex/}{D:/OneDrive - unb.br/imagens-latex/}}
\usepackage{enumitem}
\usepackage{multicol}
\usepackage{answers}
\usepackage{tikz,ifthen}
\usetikzlibrary{lindenmayersystems}
\usetikzlibrary[shadings]
\Newassociation{solucao}{Solution}{ans}
\newtheorem{exercicio}{}

\setlength{\topmargin}{-1.0in}
\setlength{\oddsidemargin}{0in}
\setlength{\textheight}{10.1in}
\setlength{\textwidth}{6.5in}
\setlength{\baselineskip}{12mm}

\extraheadheight{0.7in}
\firstpageheadrule
\runningheadrule
\lhead{
        \begin{minipage}[c]{1.7cm}
        \includegraphics[width=1.7cm]{unb.pdf}
        \end{minipage}%
        \hspace{0pt}
        \begin{minipage}[c]{4in}
          {Universidade de Brasília} --
          {Departamento de Matemática}
\end{minipage}
\vspace*{-0.8cm}
}
% \chead{Universidade de Brasília - Departamento de Matemática}
% \rhead{}
% \vspace*{-2cm}

\extrafootheight{.5in}
\footrule
\lfoot{\disciplina\ - \semestre$^o$/\ano\ - Módulo \numeromodulo}
\cfoot{}
\rfoot{Página \thepage\ de \numpages}

\newcounter{exercicios}
\renewcommand{\theexercicios}{\arabic{exercicios}}

\newenvironment{questao}[1]{
\refstepcounter{exercicios}
\ifx&#1&
\else
   \label{#1}
\fi
\noindent\textbf{Exercício {\theexercicios}:}
}

\newcommand{\resp}[1]{
\noindent{\bf Exercício #1: }}

\def\ano{2024}
\def\semestre{1}
\def\disciplina{Álgebra 1}
\def\nomeabreviado{Álgebra 1}
\def\turma{1}

\newcommand{\im}{{\rm Im\,}}
\newcommand{\dlim}[2]{\displaystyle\lim_{#1\rightarrow #2}}
\newcommand{\minf}{+\infty}
\newcommand{\ninf}{-\infty}
\newcommand{\cp}[1]{\mathbb{#1}}
\newcommand{\sub}{\subseteq}
\newcommand{\n}{\mathbb{N}}
\newcommand{\z}{\mathbb{Z}}
\newcommand{\rac}{\mathbb{Q}}
\newcommand{\real}{\mathbb{R}}
\newcommand{\complex}{\mathbb{C}}

\newcommand{\vesp}[1]{\vspace{ #1  cm}}

\newcommand{\compcent}[1]{\vcenter{\hbox{$#1\circ$}}}
\newcommand{\comp}{\mathbin{\mathchoice
        {\compcent\scriptstyle}{\compcent\scriptstyle}
        {\compcent\scriptscriptstyle}{\compcent\scriptscriptstyle}}}
\renewcommand{\sin}{{\rm sen\,}}
\renewcommand{\tan}{{\rm tg\,}}
\renewcommand{\csc}{{\rm cossec\,}}
\renewcommand{\cot}{{\rm cotg\,}}
\renewcommand{\sinh}{{\rm senh\,}}

\begin{document}

\begin{center}
    
    {\Large\bf \disciplina\ - Turma \turma\ -- \semestre$^{o}$/\ano} \\ \vspace{9pt} {\large\bf
        Lista de Exerc{\'\i}cios -- Semana \numerosemana}\\ \vspace{9pt} Prof. Jos{\'e} Ant{\^o}nio O. Freitas
    \end{center}
    \hrule

    \vspace{.6cm}
    
    \questao{nucleo_homomorfismo} Verificar em cada caso se $f$ \'e um homomorfismo de grupos.
    \begin{enumerate}[label=({\alph*})]
      \item $f: \z \to \z$ dada por $f(x) = kx$, sendo $\z$ o grupo aditivo dos inteiros e $k$ um n\'umero inteiro fixo.
      
      \item $f: \real^* \to \real^*$ dada por $f(x) = |x|$ sendo $\real^*$ o grupo multiplicativo dos reais.
      
      \item $f: \real \to \real$ dada por $f(x) = x + 1$, onde $\real$ \'e o grupo aditivo dos reais.
      
      \item $f: \z \to \z \times \z$ dada por $f(x) = (x, 0)$, onde $\z$ e $\z \times \z$ denotam grupos aditivos.
      
      \item $f: \z \times \z \to \z$ dada por $f(x,y) = x$, onde $\z \times \z$ e $\z$ s\~ao grupos aditivos.
      
      \item $f: \z \to \real^*_+$ dada por $f(x) = 2^x$, onde $\z$ \'e grupo aditivo e $\real^*_+$ \'e grupo multiplicativo.
    \end{enumerate}

    \vspace{.3cm}

    \questao{} Das funções a seguir, algumas são homomorfismos do grupo multiplicativo $\complex^*$. \textbf{Descubra} quais e determine o núcleo de cada uma.
    \begin{multicols}{2}
      \begin{enumerate}[label=({\alph*})]
        \item $f(z) = z^2$

        \item $f(z) = |z|$

        \item $f(z) = \dfrac{1}{z}$

        \item $f(z) = -\dfrac{1}{z}$

        \item $f(z) = -z$

        \item $f(z) = z^3$

        \item $f(z) = \overline{z}$ onde $\overline{z} = a - bi$, se $z = a + bi$.
      \end{enumerate}
    \end{multicols}

    \vspace{.3cm}

    \questao{} Sejam $(G, \cdot)$ e $(J, \cdot)$ grupos. Estabeleça quais das seguintes funções são homomorfismos e determine seus núcleos.

    \begin{enumerate}[label=({\alph*})]
      \item $f_1 : G \times J \to G$ dada por $f_1(x,y) = x$.

      \item $f_2 : G \times J \to J$ dada por $f_2(x,y) = y$.

      \item $f_3 : G  \to G \times J$ dada por $f_3(x) = (x, 1_J)$.

      \item $f_4 : G \times J \to J \times G$ dada por $f_4(x,y) = (y, x)$.

      \item $f_5 : J \to G \times J$ dada por $f_5(y) = (1_G, y)$.
    \end{enumerate}

    \vspace{.3cm}

    \questao{} Determine o n\'ucleo em cada homomorfismo do \textbf{Exerc{\'\i}cio \ref{nucleo_homomorfismo}}.

    \vspace{.3cm}

    \questao{} Determinar os homomorfismos injetores e os sobrejetores do \textbf{Exerc{\'\i}cio \ref{nucleo_homomorfismo}}.

    \vspace{.3cm}

    \questao{} Sejam $G$ e $J$ grupos multiplicativos, $f : G \to J$ um homomorfismo de grupos e $H$ um subgrupo de $J$. Mostre que $f^{-1}(H) = \{ x \in G \mid f(x) \in H\}$ {\'e} um subgrupo de $G$.

    \vspace{.3cm}

    \questao{} Prove que um grupo $G$ {\'e} abeliano se, e somente se, $f : G \to G$ definada por $f(x) = x^{-1}$ {\'e} um homomorfismo.

    \vspace{.3cm}

    \questao{} Seja $f: G\to H$ um homomorfismo de grupos e $K$ um subgrupo de $H$. Mostre que Ker$(f)\subset f^{-1}(K)$.

    \vspace{.3cm}

    \questao{} Seja $f: \z \times \z \to \z \times \z$ definida por $f(x, y) = (x - y, 0)$. Provar que $f$ \'e um homomorfimo do grupo aditivo $\z \times \z$ em si pr\'oprio. Obter $\ker(f)$.

    \vspace{.3cm}

    \questao{} Seja $f: G \to J$ um homomorfismo de grupos e $g\in G$ tal que $o(g)=n$.
    \begin{enumerate}[label=({\alph*})]
      \item Mostre que $f(g)$ tem ordem positiva e que a ordem de $f(g)$ divide $n$.
      
      \item Mostre que se $f$ {\'e} isomorfismo, ent{\~a}o $o(f(g))=n$.
    \end{enumerate}

    \vspace{.3cm}

    \questao{} Seja $G = [a]$ um grupo cíclico de ordem $s$ e $H = [b]$ um grupo cíclico de ordem $t$. Demonstre que $\phi : G \to H$, definida por $\phi(a) = b^k$ é um homomorfismo de grupos se, e somente se, $sk$ é um múltiplo de $t$.
\end{document}