%!TEX program = xelatex

\def\numeromodulo{1}

\documentclass[12pt]{exam}

\def\ano{2022}
\def\semestre{1}
\def\disciplina{\'Algebra 1}
\def\turma{2}

\usepackage{caption}
\usepackage{amssymb}
\usepackage{amsmath,amsfonts,amsthm,amstext}
\usepackage[brazil]{babel}
% \usepackage[latin1]{inputenc}
\usepackage{graphicx}
\graphicspath{{/ArquivosLinux/OneDrive/imagens-latex/}{D:/OneDrive - unb.br/imagens-latex/}}
\usepackage{enumitem}
\usepackage{multicol}
\usepackage{answers}
\usepackage{tikz,ifthen}
\usetikzlibrary{lindenmayersystems}
\usetikzlibrary[shadings]
\Newassociation{solucao}{Solution}{ans}
\newtheorem{exercicio}{}

\setlength{\topmargin}{-1.0in}
\setlength{\oddsidemargin}{0in}
\setlength{\textheight}{10.1in}
\setlength{\textwidth}{6.5in}
\setlength{\baselineskip}{12mm}

\extraheadheight{0.7in}
\firstpageheadrule
\runningheadrule
\lhead{
        \begin{minipage}[c]{1.7cm}
        \includegraphics[width=1.7cm]{unb.pdf}
        \end{minipage}%
        \hspace{0pt}
        \begin{minipage}[c]{4in}
          {Universidade de Brasília} --
          {Departamento de Matemática}
\end{minipage}
\vspace*{-0.8cm}
}
% \chead{Universidade de Brasília - Departamento de Matemática}
% \rhead{}
% \vspace*{-2cm}

\extrafootheight{.5in}
\footrule
\lfoot{\disciplina\ - \semestre$^o$/\ano\ - Módulo \numeromodulo}
\cfoot{}
\rfoot{Página \thepage\ de \numpages}

\newcounter{exercicios}
\renewcommand{\theexercicios}{\arabic{exercicios}}

\newenvironment{questao}[1]{
\refstepcounter{exercicios}
\ifx&#1&
\else
   \label{#1}
\fi
\noindent\textbf{Exercício {\theexercicios}:}
}

\newcommand{\resp}[1]{
\noindent{\bf Exercício #1: }}

\def\ano{2024}
\def\semestre{1}
\def\disciplina{Álgebra 1}
\def\nomeabreviado{Álgebra 1}
\def\turma{1}

\newcommand{\im}{{\rm Im\,}}
\newcommand{\dlim}[2]{\displaystyle\lim_{#1\rightarrow #2}}
\newcommand{\minf}{+\infty}
\newcommand{\ninf}{-\infty}
\newcommand{\cp}[1]{\mathbb{#1}}
\newcommand{\sub}{\subseteq}
\newcommand{\n}{\mathbb{N}}
\newcommand{\z}{\mathbb{Z}}
\newcommand{\rac}{\mathbb{Q}}
\newcommand{\real}{\mathbb{R}}
\newcommand{\complex}{\mathbb{C}}

\newcommand{\vesp}[1]{\vspace{ #1  cm}}

\newcommand{\compcent}[1]{\vcenter{\hbox{$#1\circ$}}}
\newcommand{\comp}{\mathbin{\mathchoice
        {\compcent\scriptstyle}{\compcent\scriptstyle}
        {\compcent\scriptscriptstyle}{\compcent\scriptscriptstyle}}}
\renewcommand{\sin}{{\rm sen\,}}
\renewcommand{\tan}{{\rm tg\,}}
\renewcommand{\csc}{{\rm cossec\,}}
\renewcommand{\cot}{{\rm cotg\,}}
\renewcommand{\sinh}{{\rm senh\,}}

\begin{document}
    \begin{center}
    {\Large\bf \disciplina\ - Turma \turma\ -- \semestre$^{o}$/\ano} \\ \vspace{9pt} {\large\bf
        Lista de Exerc{\'\i}cios -- Módulo \numeromodulo}\\ \vspace{9pt} Prof. Jos{\'e} Ant{\^o}nio O. Freitas
    \end{center}
    \hrule

    \vspace{.6cm}

    \begin{center}
        \textit{Observa\c{c}\~oes: \begin{enumerate} \item A nota\c{c}\~ao $A \nsub B$ significa que $A \subset B$ e $A \ne B$, isto é, existe $x \in B$ com $x \notin A$. \item Seja $A$ um conjunto com uma quantidade finita de elementos, denotamos por $|A|$ a quantidade de elementos no conjunto $A$.\end{enumerate}}
    \end{center}
    \questao{} Dados os conjuntos $A = \{0,1,2\}$, $B = \{0,2,3\}$ e $C = \{0,1,2,3,4\}$, classifique as afirma\c{c}\~oes a seguir em verdadeira ou falsa, justificando:
    \begin{multicols}{2}
        \begin{enumerate}[label={\arabic*})]
            \item $A \sub B$

            \item $A \in C$

            \item $B \sub C$

            \item $A \in \{A, B, \{A, C\}\}$

            \item $\{A\} \sub \{B, \{A, B\}\}$

            \item $\{0,2\} \sub B$

            \item $\{0,2\} \sub C$

            \item $1 \sub C$

            \item $\{1,4\} \in C$

            \item $\{0,3\} \sub B$

            \item $B \sub A$

            \item $C \sub \{0, 1, 2, 3, 4, C, \{A, B, C\}\}$

            \item $\{\vaz\} \sub A$

            \item $\vaz \in C$

            \item $A \sub C$

            \item $x \neq \{x\}$

            \item $\emptyset = \{\emptyset\}$

            \item $\emptyset \in \{\emptyset\}$

            \item $\emptyset \sub \{A, B, \{C\}\}$

            \item $\{\} = \{0\}$

            \item $A \sub \emptyset$

            \item $\{\emptyset\} \sub \emptyset$

            \item $\{1, 3, 3\} = \{1, \{2, 3\}, 3\}$

            \item $\{a, b, c\} = \{A, B, C\}$

            \item $\{\{a\}, \{b\}\} = \{\{a\}, b\}$

            \item $\{1, 2, 3, 4, 5\} = \{4, 1, 3, 5, 2, 4, 5\}$

            \item $\{\} \sub \{\emptyset\}$

            \item $|\emptyset| = 1$

            \item $|\{x,x\}| = 2$

            \item $|A \cap \emptyset | = 0$

            \item $|\{A, B, C\}| = 11$

            \item $|\{A, B, C\}| = 3$

            \item $|\{A, \{B, C\}, C, \{A, B, C\}\}| = 4$

        \end{enumerate}
    \end{multicols}

    \questao{} D\^e exemplos de conjuntos n\~ao vazios $A$, $B$ e $C$ tais que:
    \begin{enumerate}[label={\alph*})]
        \item $A \sub B$, $C \sub B$ e $A \cap C = \emptyset$.

        \item $A \sub B$, $C \nsub B$ e $A \cap C = \emptyset$.

        \item $A \sub C$, $A \ne C$ e $B \cap C = \emptyset$.

        \item $A \sub (B \cap C)$, $B \sub C$, $B \ne C$ e $A \ne C$.

        \item $A \in B$, $B \sub C$ e $A \nsubseteq C$.

        \item $A \subsetneq B$, $B \in C$ e $A \in C$.

        \item $A \cap B \sub C$, $A \nsubseteq C$ e $B \nsubseteq C$.

        \item $A \cap C = \emptyset$, $A \sub B$ e $|B \cap C| = 3$.
    \end{enumerate}

    \vspace{.3cm}

    \questao{} Determine $A \cup B$ e $A \cap B$ se $A = \{x \in \real \mid x^2 = 1\}$ e $B = \z$.

    \vspace{.3cm}

    \questao{} Em cada um dos seguintes itens, determine se a afirma\c{c}\~ao \'e
    verdadeira ou falsa. Se for verdadeira, demonstre-a. Se for falsa, exiba um
    exemplo mostrando que a afirma\c{c}\~ao \'e falsa.
    \begin{enumerate}[label={\alph*})]
        \item Se $x \in A$ e $A \sub B$, ent\~ao $x \in B$.

        \item Se $A \nsub B$ e $B \sub C$, ent\~ao $A \nsub C$.

        \item Se $A \nsub B$ e $B \nsub C$, ent\~ao $A \nsub C$.

        \item Se $x \in A$ e $A \nsub B$, ent\~ao $x \notin B$.

        \item Se $A \sub B$ e $x \notin B$, ent\~ao $x \notin A$.

        \item Se $A$, $B$ e $C$ s\~ao conjuntos, ent\~ao $A \cup (B \cap C) = (A \cup B) \cap C$.

        \item $A \cap B = B \cap A$.

        \item Se $A \cap B = A \cap C$, então $B = C$.

        \item Se $A \cup B = A \cup C$, então $B = C$.

        \item $A \cup B = B \cup A$.

        \item Se $A \cap B = A$, então $A \subset B$.

        \item Se $A \cup B = B$, então $A \subset B$.

        \item $A \cap B = A$ se, e somente se, $A \cup B = B$

        \item Se $A \sub B$, $B \sub C$ e $C \sub A$, então $A = B = C$.
    \end{enumerate}

    \vspace{.3cm}

    \questao{} Considere os seguintes conjuntos:
    \begin{align*}
        A &= \{x \in \z \mid x = 2m,\ m \in \z\}\\
        B &= \{y \in \z \mid y = 2(n - 1),\ n \in \z\}.
    \end{align*}
    Os conjuntos $A$ e $B$ são iguais? Justifique sua resposta.

    \vspace{.3cm}

    \questao{} Sejam $a$, $b \in \real$ com $a < b$. O intervalo fechado $[a, b]$ é o conjunto $\{x \in \real \mid a \le x \le b\}$. De modo semelhantes, o intervalo aberto $(a, b)$ é o conjunto $\{y \in \real \mid a < y < b\}$. Sejam $P = [3, 7]$, $Q = [7,9]$ e $R = [-3, 8]$. Descreva os seguintes conjuntos:
    \begin{enumerate}[label={\alph*})]
        \item $P \cap Q$.

        \item $P \cup Q$.

        \item $(P \cup Q) \cap R$.

        \item $P \cup (Q \cap R)$.
    \end{enumerate}

    \vspace{.3cm}

    \questao{} Sejam $A$ e $B$ conjuntos e suponha que $A$ não é um subconjunto de $B$. Quais das seguintes afirmações são verdadeiras? Encontre todas as respostas corretas.

    \begin{enumerate}[label={\roman*})]
        \item Se $x \in A$, então $x \notin B$.

        \item Se $x \in B$, ent~ao $x \in A$.

        \item Existe um elemento $x \in A$ tal que $x \notin B$.

        \item Existe um elemento $x \in B$, tal que $x \notin B$.
    \end{enumerate}
    \questao{} Demonstre que:
    \begin{enumerate}[label={\alph*})]
        \item Se $A \sub B$ e $C \sub D$, ent\~ao $A \cap C \sub B \cap D$.

        \item $A \cup B = \vaz$ se, e somente se, $A = \vaz$ e $B = \vaz$.

        \item $A \cup B = A \cap B$ se, e somente se,  $A = B$.

        \item $A \sub B$ se, e somente se,  $A \cap B = A$.

        \item $A \cup (B \cap (A \cup C)) = A \cup (B \cap C)$.

        \item $A \cup B = B$ se, e somente se, $A \subset B$.

        \item $(A \cup B) \cap C \sub A \cup (B \cap C)$.

        \item $A \sub B \cap C$ se, e somente se, $A \sub B$ e $A \sub C$.

        \item Se $A \sub B \cup C$ e $A \cap B = \emptyset$, então $A \sub C$.
    \end{enumerate}

\end{document}
