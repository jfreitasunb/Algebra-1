%!TEX program = xelatex
%!TEX encoding = UTF-8
\def\numeromodulo{2}
\def\numerolista{2}

\documentclass[12pt]{exam}

\def\ano{2022}
\def\semestre{1}
\def\disciplina{\'Algebra 1}
\def\turma{2}

\usepackage{caption}
\usepackage{amssymb}
\usepackage{amsmath,amsfonts,amsthm,amstext}
\usepackage[brazil]{babel}
% \usepackage[latin1]{inputenc}
\usepackage{graphicx}
\graphicspath{{/ArquivosLinux/OneDrive/imagens-latex/}{D:/OneDrive - unb.br/imagens-latex/}}
\usepackage{enumitem}
\usepackage{multicol}
\usepackage{answers}
\usepackage{tikz,ifthen}
\usetikzlibrary{lindenmayersystems}
\usetikzlibrary[shadings]
\Newassociation{solucao}{Solution}{ans}
\newtheorem{exercicio}{}

\setlength{\topmargin}{-1.0in}
\setlength{\oddsidemargin}{0in}
\setlength{\textheight}{10.1in}
\setlength{\textwidth}{6.5in}
\setlength{\baselineskip}{12mm}

\extraheadheight{0.7in}
\firstpageheadrule
\runningheadrule
\lhead{
        \begin{minipage}[c]{1.7cm}
        \includegraphics[width=1.7cm]{unb.pdf}
        \end{minipage}%
        \hspace{0pt}
        \begin{minipage}[c]{4in}
          {Universidade de Brasília} --
          {Departamento de Matemática}
\end{minipage}
\vspace*{-0.8cm}
}
% \chead{Universidade de Brasília - Departamento de Matemática}
% \rhead{}
% \vspace*{-2cm}

\extrafootheight{.5in}
\footrule
\lfoot{\disciplina\ - \semestre$^o$/\ano\ - Módulo \numeromodulo}
\cfoot{}
\rfoot{Página \thepage\ de \numpages}

\newcounter{exercicios}
\renewcommand{\theexercicios}{\arabic{exercicios}}

\newenvironment{questao}[1]{
\refstepcounter{exercicios}
\ifx&#1&
\else
   \label{#1}
\fi
\noindent\textbf{Exercício {\theexercicios}:}
}

\newcommand{\resp}[1]{
\noindent{\bf Exercício #1: }}

\def\ano{2024}
\def\semestre{1}
\def\disciplina{Álgebra 1}
\def\nomeabreviado{Álgebra 1}
\def\turma{1}

\newcommand{\im}{{\rm Im\,}}
\newcommand{\dlim}[2]{\displaystyle\lim_{#1\rightarrow #2}}
\newcommand{\minf}{+\infty}
\newcommand{\ninf}{-\infty}
\newcommand{\cp}[1]{\mathbb{#1}}
\newcommand{\sub}{\subseteq}
\newcommand{\n}{\mathbb{N}}
\newcommand{\z}{\mathbb{Z}}
\newcommand{\rac}{\mathbb{Q}}
\newcommand{\real}{\mathbb{R}}
\newcommand{\complex}{\mathbb{C}}

\newcommand{\vesp}[1]{\vspace{ #1  cm}}

\newcommand{\compcent}[1]{\vcenter{\hbox{$#1\circ$}}}
\newcommand{\comp}{\mathbin{\mathchoice
        {\compcent\scriptstyle}{\compcent\scriptstyle}
        {\compcent\scriptscriptstyle}{\compcent\scriptscriptstyle}}}
\renewcommand{\sin}{{\rm sen\,}}
\renewcommand{\tan}{{\rm tg\,}}
\renewcommand{\csc}{{\rm cossec\,}}
\renewcommand{\cot}{{\rm cotg\,}}
\renewcommand{\sinh}{{\rm senh\,}}

\begin{document}
    \begin{center}
    {\Large\bf \disciplina\ - Turma \turma\ -- \semestre$^{o}$/\ano} \\ \vspace{9pt} {\large\bf
        $\numerolista^a$ Lista de Exercícios -- Módulo \numeromodulo\\ Funções}\\ \vspace{9pt} Prof. José Antônio O. Freitas
    \end{center}
    \hrule

    \vspace{.6cm}

\begin{center}
    \textit{Notações:}
\end{center}
\begin{multicols}{2}
    \begin{enumerate}[label={\roman*})]
        \item $\n^* = \n - \{0\}$

        \item $\real^*_+ = \{x \in \real \mid x > 0\}$

        \item $\real^*_- = \{x \in \real \mid x < 0\}$

        \item $\real_+ = \{x \in \real \mid x \ge 0\}$

        \item $\real_- = \{x \in \real \mid x \le 0\}$

        \item $]c,d[\ = (c,d) = \{x \in \real \mid c < x < d\}$

        \item $[c,d] = \{x \in \real \mid c \le x \le d\}$

        \item $[c,d[\ = [c, d) = \{x \in \real \mid c \le x < d\}$

        \item $]c,d] = (c, d] = \{x \in \real \mid c < x \le d\}$

        \item Em alguns exercícios vamos denotar a operação de soma $\oplus$ em $\z_m$ simplesmente por $+$.

        \item Em alguns exercícios vamos denotar a operação de multiplicação $\otimes$ em $\z_m$ simplesmente por $\cdot$.

    \end{enumerate}
\end{multicols}

\vspace{.6cm}

\questao{} Seja $f: \real^2 \to \real$ dada por $f(x,y) = xy$.
\begin{enumerate}[label={\alph*})]
    \item $f$ é injetora?

    \item $f$ é sobrejetora?
\end{enumerate}

\vspace{.3cm}

\questao{} Seja $f: \z_4 \times \z_8 \to \z_{12}$ dada por $f(\overline{x},\overline{y}) = \overline{3xy}$.
\begin{enumerate}[label={\alph*})]
    \item $f$ é injetora?

    \item $f$ é sobrejetora?
\end{enumerate}

\vspace{.3cm}

\questao{} Seja $f: \z_4 \to \z_9 \times \z_4$ dada por $f(\overline{x}) = (\overline{9 + x},\overline{x+1})$.
\begin{enumerate}[label={\alph*})]
    \item $f$ é injetora?

    \item $f$ é sobrejetora?
\end{enumerate}

\vspace{.3cm}

\questao{} Seja $f : A \to [-9,-1)$ dada por $f(x) = \dfrac{4x + 3}{3 - x}$.
\begin{enumerate}[label={\alph*})]
    \item Determine $A$.

    \item Mostre que $f$ é injetora.

    \item É verdade que $f$ é sobrejetora?
\end{enumerate}

\vspace{.3cm}

\questao{} Sejam $S = \{1, 2, 3\}$ e $T = \{a, b, c, d\}$.
\begin{enumerate}[label={\alph*})]
    \item Quantas funções $f : S \to T$ existem? Descreva cada uma delas.

    \item Quantas funções injetoras $f : S \to T$ existem? Descreva cada uma delas.

    \item Quantas funções sobrejetoras $f : S \to T$ existem? Descreva cada uma delas.

    \item Quantas funções bijetoras $f : S \to T$ existem? Descreva cada uma delas.
\end{enumerate}

\vspace{.3cm}

\questao{} Seja $f : A \to (1,10]$ dada por $f(x) = \dfrac{4 - 11x}{4 - 2x}$.
\begin{enumerate}[label={\alph*})]
    \item Determine $A$.

    \item Mostre que $f$ é injetora.

    \item É verdade que $f$ é sobrejetora?
\end{enumerate}

\vspace{.3cm}

\questao{} Considere a função $f : \z \times \z \to \z \times \z$ tal que $f(x,y) = (2x + 3, 4y + 5)$. Prove que $f$ é injetora. Verifique se $f$ é bijetora.

\vspace{.3cm}

\questao{} Mostre que a função $f : \z \to \z$ dada por $f(n) = 2n$, é injetora mas não é sobrejetora.

\vspace{.3cm}

\questao{} Mostre que a função $h : \real^2 \to \real^2$ tal que $h(x, y) = (\sqrt[3]{x}, y^5)$ é sobrejetora. Essa função é bijetora? Justifique.

\vspace{.3cm}

\questao{} Classificar (se possível) em injetora ou sobrejetora as seguintes funç{õ}es de $f : \z \times \z \to \z$.

\begin{multicols}{2}
    \begin{enumerate}[label={\alph*})]
        \item $f(m, n) = m^2 + n^2$

        \item $f(m, n) = m$

        \item $f(m, n) = |n|$

        \item $f(m, n) = m - n$
    \end{enumerate}
\end{multicols}

\vspace{.3cm}

\questao{} Considere a função $g : \z_5 \times \z_9 \to \z_5 \times \z_9$ tal que $g(\overline{x},\overline{y}) = (\overline{2x + 3}, \overline{4y + 5})$.
\begin{enumerate}[label={\alph*})]
    \item $g$ é injetora?

    \item $g$ é sobrejetora?
\end{enumerate}

\vspace{.3cm}

\questao{} Achar uma função $f : A \to B$, com $A$ e $B$ subconjuntos de $\real$, para cada caso abaixo:
\begin{enumerate}[label={\alph*})]
    \item $A = \real$, $B \nsub \real$ e $f$ injetora e não sobrejetora.

    \item $A \nsub \real$, $B = \real$ e $f$ injetora e não sobrejetora.

    \item $A = \real$, $B \nsub \real$ e $f$ sobrejetora e não injetora.

    \item $A \nsub \real$, $B = \real$ e $f$ sobrejetora e não injetora.
\end{enumerate}

\vspace{.3cm}

\questao{} Classificar (se possível) em injetora ou sobrejetora as seguintes funç{õ}es de $\real$ em $\real$.

\begin{multicols}{2}
    \begin{enumerate}[label={\alph*})]
        \item $f(x) = x^3$

        \item $f(x) = x^2 - 5x - 6$

        \item $f(x) = 2^x$

        \item $f(x) = x + | x |$

        \item $f(x) = x + 3$

        \item $f(x) = \mid x - 1\mid$

        \item $f(x) = 1/x$

        \item $f(x) = 1 - x^2$

        \item $f(x) = x^2/(1 + x^2)$
    \end{enumerate}
\end{multicols}

\vspace{.3cm}

\questao{} Seja $f : A \to B$ uma função. Defina $P_f : \mathcal{P}(A) \to \mathcal{P}(B)$ por
\[
P_f(C) = \{\beta \in B \mid \mbox{ existe } \alpha \in C \mbox{ tal que } f(\alpha) = \beta\}
\]
para $C \sub A$. Prove que:
\begin{enumerate}[label={\roman*})]

    \item $f : A \to B$ é injetora se, e somente se, $P_f : \mathcal{P}(A) \to \mathcal{P}(B)$ é injetora.

    \item $f : A \to B$ é sobrejetora se, e somente se, $P_f : \mathcal{P}(A) \to \mathcal{P}(B)$ é sobrejetora.
\end{enumerate}


\vspace{.3cm}

\questao{} Sejam $f$, $g : \real \to \real$ dadas por $f(x) = x^4$ e $g(x) = x^7$. Verifique que $(f\circ g)(x) = (g\circ f)(x)$.

\vspace{.3cm}

\questao{} Dadas as funções $f(x) = 3x + m$ e $g(x) = ax + 2$, determine condições sobre $a$ e $m$ para que $(f\circ g)(x) = (g\circ f)(x)$, quando:
\begin{enumerate}[label={\roman*})]
    \item $f : \z \to \z$ e $g : \z \to \z$
    \item $f : \real \to \real$ e $g : \real \to \real$
    \item $f : \z_6 \to \z_6$ e $g : \z_6 \to \z_6$
\end{enumerate}

\vspace{.3cm}

\questao{} Sejam $f$, $g$ e $h$ funções de $\real$ em $\real$ definidas por $f(x) = x + 2$, $g(x) = x^2 - 1$ e $h(x) = 3x$, respectivamente.
\begin{enumerate}[label={\alph*})]
    \item Determine $f \circ g$, $f \circ h$, $g \circ f$, $g \circ h$ e $h \circ g$.
    \item Verifique que $(f \circ g)\circ h = f \circ (g \circ h)$.
\end{enumerate}

\vspace{.3cm}

\questao{} Dada as funções
\[
f(x) = \begin{cases}
    1, & \mbox{ se } x < 0\\
    2x^2, & \mbox{ se } 0 \le x \le 1\\
    0, & \mbox{ se } x > 1
\end{cases} \qquad g(x) = \begin{cases}
    x, & \mbox{ se } x < 0\\
    0, & \mbox{ se } 0 \le x \le 1\\
    1, & \mbox{ se } x > 1.
\end{cases}
\]
Determine $f\circ g$.

\vspace{.3cm}

\questao{} Dada as funções
\[
f(x) = \begin{cases}
    x^2 + 2, & \mbox{ se } x \le -1\\
    \dfrac{1}{x - 2}, & \mbox{ se } -1 < x < 1\\
    4 - x^2, & \mbox{ se } x \ge 1
\end{cases} \qquad g(x) = 2 - 3x.
\]
Determine $f\circ g$ e $g \circ f$.

\vspace{.3cm}

\questao{} Seja $f : A \to B$ uma função. Quando a composição $f \circ f$ está definida?

\vspace{.3cm}

\questao{} Encontre exemplos de conjuntos $A$, $B$ e $C$ e funções $f : A \to B$ e $g : B \to C$ tais que $g \circ f : A \to C$ é sobrejetora mas $f : A \to B$ não é sobrejetora.

\vspace{.3cm}

\questao{} Sejam $f : A \to B$, $g : A \to B$ e $h : B \to C$ funções. Prove que se $h$ é injetora e $h \circ g = h \circ f$, então $g = f$.

\vspace{.3cm}

\questao{} Se as funç{õ}es $f : A \to B$ e $g : B\to A$ são
tais que $g\circ f$ é injetora, então $f$ é injetora.

\vspace{.3cm}

\questao{} Dê um exemplo de funções $f$ e $g$ tais que $g \circ f$ é injetora mas $g$ não é injetora.

\vspace{.3cm}

\questao{} Se as funç{õ}es $f : A \to B$ e $g : B\to A$ são
tais que $g\circ f$ é sobrejetora, então g é sobrejetora.

\vspace{.3cm}

\questao{} Mostre que toda função injetora (sobrejetora) de um conjunto finito em si mesmo é também sobrejetora (injetora).

\vspace{.3cm}

\questao{} Suponha que $f$, $g : A \to A$ são funções. Se $f \circ f = g \circ g$, então $f = g$? Prove ou dê um contra-exemplo.

\vspace{.3cm}

\questao{} Uma função $f : \real \to \real$ é dita \textit{estritamente crescente} se para todo $x_1 < x_2$ temos $f(x_1) < f(x_2)$. Mostre que se $f : \real \to \real$ é estritamente crescente, então $f$ é injetora.

\vspace{.3cm}

\questao{} Considere a função $f : \real \to \real$ dada por $f(x) = \left| x - \dfrac{5}{2}\right|$. Encontre:
\begin{multicols}{3}
    \begin{enumerate}[label={\alph*})]
        \item $f(\{1\})$

        \item $f(\{-\sqrt{2}, 3\})$

        \item $f([-2,2])$

        \item $f((-3,5))$

        \item $f^{-1}(\{3\})$

        \item $f^{-1}(\{-3,5\})$

        \item $f^{-1}([0,2])$

        \item $f^{-1}([-3,3])$

        \item $f^{-1}(\real^*_-)$
    \end{enumerate}
\end{multicols}

\vspace{.3cm}

\questao{} Seja $g : \real \to \real$ dada por
\[
g(x) = \begin{cases}
    x^3 - 2,& \mbox{ se } x \le 0\\
    \sqrt{x + 1}, & \mbox{ se } 0 < x \le 2\\
    \dfrac{1}{x - 2}, & \mbox{ se } x > 2.
\end{cases}
\]
Encontre:
\begin{multicols}{2}
    \begin{enumerate}[label={\alph*})]
        \item $g([-1,8])$

        \item $g(\real_+)$

        \item $g(]1, 3])$

        \item $g^{-1}([-1,16])$

        \item $g(\real_-)$

        \item $g^{-1}([1,25])$

        \item $g^{-1}(\real^*_-)$

        \item $g^{-1}([-1,5])$
    \end{enumerate}
\end{multicols}

\vspace{.3cm}

\questao{} Seja $f: \real^2 \to \real$ dada por $f(x,y) = xy - 2$.
\begin{multicols}{2}
    \begin{enumerate}[label={\alph*})]
        \item Obter $f^{-1}(\{0\})$.

        \item Obter $f([0,1]\times [0,1])$.

        \item Obter $f^{-1}([0,2])$.

        \item Obter $f(]-3,4]\times [2,4[)$.
    \end{enumerate}
\end{multicols}

\vspace{.3cm}

\questao{} Seja $f : \z \to \z_{12}$ dada por $f(x) = \overline{4x + 3}$.
\begin{enumerate}[label={\alph*})]
    \item Obter $f^{-1}(\{\overline{0}\})$.

    \item Seja $I = \{8k \mid k \in \z\}$. Encontre $f(I)$.

    \item Obter $f^{-1}(\{\overline{0}, \overline{2}, \overline{7}, \overline{10}\})$.

    \item Seja $J = \{5k \mid k \in \z\} \cup \{10l \mid l \in \z\}$. Encontre $f(I)$.

    \item Obter $f^{-1}(\{\overline{1}, \overline{3}, \overline{4}, \overline{8}\})$.
\end{enumerate}

\vspace{.3cm}

\questao{} Considere a função $f : \z_5 \times \z_9 \to \z_5 \times \z_9$ tal que $f(\overline{x},\overline{y}) = (\overline{2x + 3}, \overline{4y + 5})$. Verifique se $f$ possui inversa e em caso afirmativo, encontre sua inversa.

\vspace{.3cm}

\questao{} Para quais valores de $a \in \z$ a função $f : \z \to \z$ dada por $f(x)       = x + a$ possui inversa? Para cada valor de $a$, encontre a função inversa.

\vspace{.3cm}

\questao{} Para quais valores de $b \in \z$ a função $g : \z \to \z$ dada por $g(x)       = bx$ possui inversa? Para cada valor de $b$, encontre a função inversa.

\vspace{.3cm}

\questao{} Suponha que as funções $f : A \to B$ e $g : B \to C$ são inversíveis. Mos      tre que $g \circ f$ é inversível e que
\[
(g \circ f)^{-1} = f^{-1} \circ g^{-1}.
\]

\vspace{.3cm}

\questao{} Considere a função $g : \z_6 \times \z_8 \to \z_6 \times \z_8$ tal que $g(\overline{x},\overline{y}) = (\overline{2x + 3}, \overline{4y + 5})$. Verifique se $g$ possui inversa e em caso afirmativo, encontre sua inversa.

\vspace{.3cm}

\questao{} Considere a função $h : \z_5 \times \z_7 \to \z_5 \times \z_7$ tal que $h(\overline{x},\overline{y}) = (\overline{x + 3}, \overline{6y + 2})$. Determine a função inversa de $h$, se existir.

\vspace{.3cm}

\questao{} Considere a função $t : \z_8 \times \z_{10} \to \z_8 \times \z_{10}$ tal que $t(\overline{x},\overline{y}) = (\overline{x + 3}, \overline{6y + 2})$. Determine a função inversa de $t$, se existir.

\vspace{.3cm}

\questao{} A função $f : \z \to \z$ definida por $f(x) = ax + b$, com $a$ e $b$ constantes, $a \ne 0$, é uma bijeção? Caso afirmativo, obtenha $f^{-1}$.

\vspace{.3cm}


\questao{} Mostre que a função $f : \real \to \real$ definida por $f(x) = ax + b$, com $a$ e $b$ constantes reais, $a \ne 0$, é uma bijeção. Obter $f^{-1}$.

\vspace{.3cm}

\questao{} Mostrar que $f : \real - \left\{-\dfrac{d}{c}\right\} \to \real  - \left\{\dfrac{a}{c}\right\}$ dada por $f(x) =  \dfrac{ax + b}{cx + d}$, onde $a$, $b$, $c$, $d$ são n{\'u}meros reais constantes, $ad - bc \ne 0$, é uma bijeção. Descrever a função $f^{-1}$.

\vspace{.3cm}

\questao{} Seja $f : A \to B$ e $g : B \to A$ funções tais que $g \circ f = i_A$. Quais das afirmações seguintes são verdadeiras?
\begin{enumerate}[label={\alph*})]
    \item $g = f^{-1}$

    \item $f$ é sobrejetora

    \item $f$ é injetora

    \item $g$ é sobrejetora

    \item $g$ é injetora
\end{enumerate}

\vspace{.3cm}

\questao{} Seja $f : A \to B$ uma função. Se $Y \sub B$, então $f(f^{-1}(Y)) = Y$? Prove ou d\^e um contra-exemplo.

\vspace{.3cm}

\questao{} Seja $f : A \to B$ uma função e $X \sub A$. É verdade que
\[
f(f^{-1}(f(X))) = f(X)?
\]
Prove ou d\^e um contra-exemplo.

\vspace{.3cm}

\questao{} Seja $f : A \to B$ uma função. Dados conjuntos $X \sub A$ e $Y \sub B$ é verdade que
\[
f^{-1}(f(f^{-1}(f(X)))) = f^{-1}(Y)?
\]
Prove ou d\^e um contra-exemplo.

\vspace{.3cm}

\questao{} Seja $f : A \to B$ uma função. Prove que $f$ é injetiva se, e somente se, $X = f^{-1}(f(X))$ para todo $X \sub A$.

\vspace{.3cm}

\questao{} Dada uma função $f : A \to B$ e conjuntos $W$, $X \sub A$, então a igualdade
\[
f(W \cap X) = f(W) \cap f(X)
\]
\textbf{em geral é falsa}. D\^e um exemplo em que essa igualdade seja falsa.

\vspace{.3cm}

\questao{} Seja $f : A \to B$ uma função e sejam $P$ e $Q$ subconjuntos de $A$. Mostre que:
\begin{enumerate}[label={\alph*})]
    \item Se $P\sub Q$, então $f(P)\sub f(Q)$.
    \item $f(P\cup Q) = f(P)\cup f(Q)$.

    \item $f(P\cap Q)\sub f(P)\cap f(Q)$.

    \item $f(P) - f(Q) \sub f(P - Q)$.

    \item Mostre que se $f$ é bijetora, então $f(P - Q) = f(P) - f(Q)$.

    \item Se $f$ é injetora, então $f(P\cap Q) =  f(P)\cap f(Q)$.

    \item $f$ é bijetora se, e somente se, $f(P^C) = (f(P))^C$ para todo $P \sub A$. \textit{(Aqui $P^C$ é o complementar de $P$ em relação \`a $A$.)}
\end{enumerate}

\vspace{.3cm}

\questao{} Suponha que $f : A \to B$ e $g : B \to C$ são funções. Seja $P \sub A$, então
\[
(g \circ f)(A) = g(f(A)).
\]

\vspace{.3cm}

\questao{} Sejam $P$, $Q$ conjuntos, $f : P \to Q$ uma função, $A \sub P$ e $B \sub Q$. Então $f(A) \cap B = f(A \cap f^{-1}(B))$.

\vspace{.3cm}

\questao{} Suponha que $f : A \to B$ e $g : B \to C$ são funções. Seja $P \sub C$. Então
\[
(g \circ f)^{-1}(C) = f^{-1}(g^{-1}(C)).
\]

\questao{} Seja $f : A \to B$ uma função e sejam $P \sub
A$ e $X, Y\sub B$. Mostre que:
\begin{enumerate}[label={\alph*})]
    \item Se $X\sub Y$, então $f^{-1}(X)\sub f^{-1}(Y)$.

    \item $f^{-1}(X\cup Y)=f^{-1}(X)\cup f^{-1}(Y)$.

    \item $f^{-1}(X - Y) = f^{-1}(X) - f^{-1}(Y)$.

    \item $f^{-1}(X\cap Y)= f^{-1}(X)\cap f^{-1}(Y)$.

    \item $P\sub f^{-1}(f(P))$.

    \item $f(f^{-1}(X))= X \cap \mbox{Im}f$ e conclua que se $f$ é sobrejetora então
    $f(f^{-1}(X))=X$.
    \item $f^{-1}(X^C) = (f^{-1}(X))^C$.

    \item $f$ é sobrejetora se, e somente se, $f^{-1}(T) \ne \emptyset$ para todo $T \sub B$.
\end{enumerate}

\vspace{.3cm}

\questao{} Seja $f : A \to B$ uma função. Mostre que $f$ é injetora se, e somente se, para todos conjuntos $P$, $Q \sub A$ temos $f(P \cap Q) = f(P) \cap f(Q)$.

\vspace{.3cm}

\questao{} Seja $f : A \to B$ uma função. Prove que $f$ é sobrejetora se, e somente se, $B - f(P) \sub f(A - P)$, para todo conjunto $P \sub A$.

\vspace{.3cm}

\questao{} Seja $f: A \to B$ uma função. Prove que $f$ é injetora se, e somente se, $f(A - P) = B - f(P)$ para todo conjunto $P \sub A$.

\end{document}
