%!TEX program = xelatex
%!TEX encoding = UTF-8
\def\numeromodulo{2}
\def\numerolista{3}

\documentclass[12pt]{exam}

\def\ano{2022}
\def\semestre{1}
\def\disciplina{\'Algebra 1}
\def\turma{2}

\usepackage{caption}
\usepackage{amssymb}
\usepackage{amsmath,amsfonts,amsthm,amstext}
\usepackage[brazil]{babel}
% \usepackage[latin1]{inputenc}
\usepackage{graphicx}
\graphicspath{{/ArquivosLinux/OneDrive/imagens-latex/}{D:/OneDrive - unb.br/imagens-latex/}}
\usepackage{enumitem}
\usepackage{multicol}
\usepackage{answers}
\usepackage{tikz,ifthen}
\usetikzlibrary{lindenmayersystems}
\usetikzlibrary[shadings]
\Newassociation{solucao}{Solution}{ans}
\newtheorem{exercicio}{}

\setlength{\topmargin}{-1.0in}
\setlength{\oddsidemargin}{0in}
\setlength{\textheight}{10.1in}
\setlength{\textwidth}{6.5in}
\setlength{\baselineskip}{12mm}

\extraheadheight{0.7in}
\firstpageheadrule
\runningheadrule
\lhead{
        \begin{minipage}[c]{1.7cm}
        \includegraphics[width=1.7cm]{unb.pdf}
        \end{minipage}%
        \hspace{0pt}
        \begin{minipage}[c]{4in}
          {Universidade de Brasília} --
          {Departamento de Matemática}
\end{minipage}
\vspace*{-0.8cm}
}
% \chead{Universidade de Brasília - Departamento de Matemática}
% \rhead{}
% \vspace*{-2cm}

\extrafootheight{.5in}
\footrule
\lfoot{\disciplina\ - \semestre$^o$/\ano\ - Módulo \numeromodulo}
\cfoot{}
\rfoot{Página \thepage\ de \numpages}

\newcounter{exercicios}
\renewcommand{\theexercicios}{\arabic{exercicios}}

\newenvironment{questao}[1]{
\refstepcounter{exercicios}
\ifx&#1&
\else
   \label{#1}
\fi
\noindent\textbf{Exercício {\theexercicios}:}
}

\newcommand{\resp}[1]{
\noindent{\bf Exercício #1: }}

\def\ano{2024}
\def\semestre{1}
\def\disciplina{Álgebra 1}
\def\nomeabreviado{Álgebra 1}
\def\turma{1}

\newcommand{\im}{{\rm Im\,}}
\newcommand{\dlim}[2]{\displaystyle\lim_{#1\rightarrow #2}}
\newcommand{\minf}{+\infty}
\newcommand{\ninf}{-\infty}
\newcommand{\cp}[1]{\mathbb{#1}}
\newcommand{\sub}{\subseteq}
\newcommand{\n}{\mathbb{N}}
\newcommand{\z}{\mathbb{Z}}
\newcommand{\rac}{\mathbb{Q}}
\newcommand{\real}{\mathbb{R}}
\newcommand{\complex}{\mathbb{C}}

\newcommand{\vesp}[1]{\vspace{ #1  cm}}

\newcommand{\compcent}[1]{\vcenter{\hbox{$#1\circ$}}}
\newcommand{\comp}{\mathbin{\mathchoice
        {\compcent\scriptstyle}{\compcent\scriptstyle}
        {\compcent\scriptscriptstyle}{\compcent\scriptscriptstyle}}}
\renewcommand{\sin}{{\rm sen\,}}
\renewcommand{\tan}{{\rm tg\,}}
\renewcommand{\csc}{{\rm cossec\,}}
\renewcommand{\cot}{{\rm cotg\,}}
\renewcommand{\sinh}{{\rm senh\,}}

\begin{document}
    \begin{center}
    {\Large\bf \disciplina\ - Turma \turma\ -- \semestre$^{o}$/\ano} \\ \vspace{9pt} {\large\bf
        $\numerolista^a$ Lista de Exercícios -- Módulo \numeromodulo\\ Anéis}\\ \vspace{9pt} Prof. José Antônio O. Freitas
    \end{center}
    \hrule

    \vspace{.6cm}

        \questao{} Verifique se o conjunto $A$ com as operações dadas é um anel. Em caso afirmativo, esse anel é comutativo? Possui unidade?
    \begin{enumerate}[label={\alph*})]
        \item $A = \real$, $x \oplus y = \dfrac{x + y}{2}$, $x \otimes y = x$, para todos $x$, $y \in A$.
        \item $A = \real$, $x \oplus y = \sqrt[3]{x^3 + y^3}$, $x \otimes y = x$, para todos $x$, $y \in A$.
        \item $A = \rac^*$, $x \oplus y = x + y + xy$, $x \otimes y = \dfrac{x}{y}$, para todos $x$, $y \in A$.
        \item $A = \z$, $x \oplus y = xy + 2x$, $x \otimes y = x + xy$, para todos $x$, $y \in A$.
        \item $A = \real_+$, $x \oplus y = \dfrac{x + y}{1 + xy}$, $x \otimes y = x^2 + y^2 + 2xy$, para todos $x$, $y \in A$.
        \item $A = \z \times \z$, $(x, y) \oplus (z, t) = (x + y, z + t)$, $(x, y) \otimes (z, t) = (xz, 0)$, para todos $(x, y)$, $(z, t) \in A$.
    \end{enumerate}

    \vspace{.3cm}

    \questao{} Consideremos em $\z \times \z$ as operações $\oplus$ e $\otimes$ definidas por
    \begin{align*}
        (a, b) \oplus (c, d) = (a + c, b + d)\\
        (a ,b) \otimes (c, d) = (ac - bd, ad + bc).
    \end{align*}
    Mostre que $(\z \times \z, \oplus, \otimes)$ é um anel comutativo e com unidade.

    \vspace{.3cm}

    \questao{} Considere as operações $\star$ e $\odot$ em $\rac$ definidas por
    \begin{align*}
        x \star y = x + y - 8\\
        x \odot y = x + y - \dfrac{xy}{8}.
    \end{align*}
    Mostre que $(\rac, \star, \odot)$ é um anel comutativo e com unidade.

    \vspace{.3cm}

    \questao{} Seja $m > 1$, $m \in \z$. No conjunto $\z_m$ considere as operações usuais de soma $\oplus$ e multiplicação $\otimes$. Mostre que $(\z_m, \oplus, \otimes)$ é um anel comutativo com unidade.

    \vspace{.3cm}

    \questao{} Prove que são anéis:
    \begin{enumerate}[label={\alph*})]
        \item O conjunto $\z$ com a adição usual e o produto $x \otimes y = 0$, para todo $x$, $y \in \z$.
        \item O conjunto $\rac$ com as operações $x \oplus y = x + y - 1$ e $x \odot y = x + y - xy$.
        \item O conjunto $\z \times \z$ com as operações:
        \begin{align*}
            (a, b) \oplus (c, d) = (a + c, b + d)\\
            (a ,b) \otimes (c, d) = (ac, ad + bc).
        \end{align*}
    \end{enumerate}
    Quais destes anéis são comutativos? Quais t\^em unidade?

    \vspace{.3cm}

    \questao{} Determinar quais dos seguintes conjuntos são anéis com as operações usuais em $\rac$:
    \begin{multicols}{2}
        \begin{enumerate}[label=({\alph*})]
            \item $\z$
            \item $2\z = \{2k \mid k \in \z\}$
            \item $B = \{x \in \rac \mid x \notin \z\}$
            \item $C = \left\{\dfrac{a}{b} \in \rac \mid a \in \z,\ b \in \z,\ 2 |b \right\}$
            \item $D = \left\{\dfrac{a}{2^n} \in \rac \mid a \in \z \mbox{ e } n \in \z \right\}$
        \end{enumerate}
    \end{multicols}

    \vspace{.3cm}

    \questao{} Seja
    \[
    M_2(\real) = \left\{\begin{pmatrix}
        x & y\\z & t
    \end{pmatrix} \mid x, y, z, t \in \real \right\}.
    \]
    Dados $A = \begin{pmatrix}
        x & y\\z & t
    \end{pmatrix}$, $B = \begin{pmatrix}
        a & b\\c & d
    \end{pmatrix} \in M_2(\real)$ definimos
    \begin{align*}
        A &= B \mbox{ quando } x = a, y = b, z = c, t = d\\
        A + B &= \begin{pmatrix}
            x & y\\z & t
        \end{pmatrix} + \begin{pmatrix}
            a & b\\c & d
        \end{pmatrix} = \begin{pmatrix}
            x + a & y + b\\z + c & t + d
        \end{pmatrix}\\
        A \cdot B &= \begin{pmatrix}
            x & y\\z & t
        \end{pmatrix} \cdot \begin{pmatrix}
            a & b\\c & d
        \end{pmatrix} = \begin{pmatrix}
            xa + yc & xb + yd\\za + tc & zb + td
        \end{pmatrix}.
    \end{align*}
    \begin{enumerate}[label=({\alph*})]
        \item Mostre que $(M_2(\real), + , \cdot)$ é um anel. Esse anel é comutativo? Possui unidade?
        \item Quais dos seguintes conjuntos são anéis com as operações de $M_2(\real)$? Quais são comutativos? Quais possuem unidade?

        \begin{align*}
            L_1 &= \left\{\begin{pmatrix}
                a & 0\\
                b & 0
            \end{pmatrix} \mid a, b \in \real\right\},\qquad
            L_2 = \left\{\begin{pmatrix}
                a & b\\
                0 & c
            \end{pmatrix} \mid a, b, c \in \real\right\}\\
            L_3 &= \left\{\begin{pmatrix}
                a & 0\\
                0 & b
            \end{pmatrix} \mid a, b \in \real\right\},\qquad
            L_4 = \left\{\begin{pmatrix}
                0 & a\\
                c & b
            \end{pmatrix} \mid a, b, c \in \real\right\}
        \end{align*}
    \end{enumerate}

    \vspace{.3cm}

    \questao{} Considere o conjunto
    \[
    T_2(\z) = \left\{\begin{bmatrix}a & b\\ 0 & c\end{bmatrix} \mid a, b, c \in \z\right\}.
    \]
    Mostre que $T_2(\z)$ é um anel, com as operações usuais de soma e multiplicação de matrizes. Esse anel possui unidade? É comutativo?

    \vspace{.3cm}

    \questao{} Seja
    \[
    M_2(\z_m) = \left\{\begin{pmatrix}
        \overline{x} & \overline{y}\\\overline{z} & \overline{t}
    \end{pmatrix} \mid \overline{x}, \overline{y}, \overline{z}, \overline{t} \in \z_m \right\}
    \]
    onde $m \in \z$, $m > 1$.
    Dados $A = \begin{pmatrix}
        \overline{x} & \overline{y}\\\overline{z} & \overline{t}
    \end{pmatrix}$, $B = \begin{pmatrix}
        \overline{a} & \overline{b}\\\overline{c} & \overline{d}
    \end{pmatrix} \in M_2(\z_m)$ definimos
    \begin{align*}
        A &= B \mbox{ quando } \overline{x} = \overline{a}, \overline{y} = \overline{b}, \overline{z} = \overline{c}, \overline{t} = \overline{d}\\
        A + B &= \begin{pmatrix}
            \overline{x}& \overline{y}\\\overline{z} &\overline{t}
        \end{pmatrix} + \begin{pmatrix}
            \overline{a} & \overline{b}\\\overline{c} & \overline{d}
        \end{pmatrix} = \begin{pmatrix}
            \overline{x + a} & \overline{y + b}\\\overline{z + c} &\overline{t + d}
        \end{pmatrix}\\
        A \cdot B &= \begin{pmatrix}
            \overline{x}& \overline{y}\\\overline{z} &\overline{t}
        \end{pmatrix} \cdot \begin{pmatrix}
            \overline{a} & \overline{b}\\\overline{c} & \overline{d}
        \end{pmatrix} = \begin{pmatrix}
            \overline{xa + yc} & \overline{xb + yd}\\\overline{za + tc} & \overline{zb + td}
        \end{pmatrix}.
    \end{align*}
    Mostre que $(M_2(\z_m), + , \cdot)$ é um anel. Esse anel é comutativo? Possui unidade?

    \vspace{.2cm}

    \textbf{Observação: }\textit{No caso geral, o conjunto $(M_n(\z_m), + , \cdot)$ também é um anel.}

    \vspace{.3cm}

    \questao{} Considere o conjunto $L = \{ a + b\sqrt{2} \mid a, b \in \rac\} \sub \real$. Para $a + b\sqrt{2}$, $c + d\sqrt{2} \in L$ definimos:
    \begin{align*}
        a + b\sqrt{2} &= c + d\sqrt{2} \quad \mbox{ quando } a = c \mbox{ e } b = d\\
        (a + b\sqrt{2}) \oplus (c + d\sqrt{2}) &= (a + c) + (b + d)\sqrt{2}\\
        (a + b\sqrt{2}) \otimes (c + d\sqrt{2}) &= (ac + 2bd) + (ad + bc)\sqrt{2}.
    \end{align*}

    Mostre que $(L, \oplus, \otimes)$ é um anel. Esse anel é comutativo? Possui unidade?

    \vspace{.3cm}

    \questao{} Seja $(R, \oplus, \otimes)$ um anel. Dados $a$, $b$, $c$, $d \in R$ calcule $(a \oplus b)\otimes (c \oplus d)$.

    \vspace{.3cm}

    \questao{} Seja $(R, +, \cdot)$ um anel. Dados $a$, $b \in R$ prove que
    \[
    (a + b)^2 = a^2 + ab + ba + b^2,
    \]
    onde $x^2 = xx$.

    \vspace{.3cm}

    \questao{} Seja $(R, +, \cdot)$ um anel. Prove que se para todo $x \in R$, vale que $x^2 = x$, então $R$ é um anel comutativo.

    \vspace{.3cm}

    \questao{} Seja $(R, +, \cdot)$ um anel. Prove que $R$ é comutativo se, e somente se, $(a + b)^2 = a^2 + 2ab + b^2$ para todos $a$, $b \in R$.

    \vspace{.3cm}

    \questao{} Seja $(R, +, \cdot)$ um anel. Prove que $R$ é comutativo se, e somente se, $a^2 - b^2 = (a - b)(a + b)$ para todos $a$, $b \in R$.

    \vspace{.3cm}

    \questao{} Seja $(R, +, \cdot)$ um anel. Se $a^3 = a$ para todo $a \in R$, mostre que $R$ é comutativo.

\end{document}
