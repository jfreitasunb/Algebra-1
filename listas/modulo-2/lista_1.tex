%!TEX program = xelatex
%!TEX encoding = UTF-8
\def\numeromodulo{2}
\def\numerolista{1}

\documentclass[12pt]{exam}

\def\ano{2022}
\def\semestre{1}
\def\disciplina{\'Algebra 1}
\def\turma{2}

\usepackage{caption}
\usepackage{amssymb}
\usepackage{amsmath,amsfonts,amsthm,amstext}
\usepackage[brazil]{babel}
% \usepackage[latin1]{inputenc}
\usepackage{graphicx}
\graphicspath{{/ArquivosLinux/OneDrive/imagens-latex/}{D:/OneDrive - unb.br/imagens-latex/}}
\usepackage{enumitem}
\usepackage{multicol}
\usepackage{answers}
\usepackage{tikz,ifthen}
\usetikzlibrary{lindenmayersystems}
\usetikzlibrary[shadings]
\Newassociation{solucao}{Solution}{ans}
\newtheorem{exercicio}{}

\setlength{\topmargin}{-1.0in}
\setlength{\oddsidemargin}{0in}
\setlength{\textheight}{10.1in}
\setlength{\textwidth}{6.5in}
\setlength{\baselineskip}{12mm}

\extraheadheight{0.7in}
\firstpageheadrule
\runningheadrule
\lhead{
        \begin{minipage}[c]{1.7cm}
        \includegraphics[width=1.7cm]{unb.pdf}
        \end{minipage}%
        \hspace{0pt}
        \begin{minipage}[c]{4in}
          {Universidade de Brasília} --
          {Departamento de Matemática}
\end{minipage}
\vspace*{-0.8cm}
}
% \chead{Universidade de Brasília - Departamento de Matemática}
% \rhead{}
% \vspace*{-2cm}

\extrafootheight{.5in}
\footrule
\lfoot{\disciplina\ - \semestre$^o$/\ano\ - Módulo \numeromodulo}
\cfoot{}
\rfoot{Página \thepage\ de \numpages}

\newcounter{exercicios}
\renewcommand{\theexercicios}{\arabic{exercicios}}

\newenvironment{questao}[1]{
\refstepcounter{exercicios}
\ifx&#1&
\else
   \label{#1}
\fi
\noindent\textbf{Exercício {\theexercicios}:}
}

\newcommand{\resp}[1]{
\noindent{\bf Exercício #1: }}

\def\ano{2024}
\def\semestre{1}
\def\disciplina{Álgebra 1}
\def\nomeabreviado{Álgebra 1}
\def\turma{1}

\newcommand{\im}{{\rm Im\,}}
\newcommand{\dlim}[2]{\displaystyle\lim_{#1\rightarrow #2}}
\newcommand{\minf}{+\infty}
\newcommand{\ninf}{-\infty}
\newcommand{\cp}[1]{\mathbb{#1}}
\newcommand{\sub}{\subseteq}
\newcommand{\n}{\mathbb{N}}
\newcommand{\z}{\mathbb{Z}}
\newcommand{\rac}{\mathbb{Q}}
\newcommand{\real}{\mathbb{R}}
\newcommand{\complex}{\mathbb{C}}

\newcommand{\vesp}[1]{\vspace{ #1  cm}}

\newcommand{\compcent}[1]{\vcenter{\hbox{$#1\circ$}}}
\newcommand{\comp}{\mathbin{\mathchoice
        {\compcent\scriptstyle}{\compcent\scriptstyle}
        {\compcent\scriptscriptstyle}{\compcent\scriptscriptstyle}}}
\renewcommand{\sin}{{\rm sen\,}}
\renewcommand{\tan}{{\rm tg\,}}
\renewcommand{\csc}{{\rm cossec\,}}
\renewcommand{\cot}{{\rm cotg\,}}
\renewcommand{\sinh}{{\rm senh\,}}

\begin{document}
    \begin{center}
    {\Large\bf \disciplina\ - Turma \turma\ -- \semestre$^{o}$/\ano} \\ \vspace{9pt} {\large\bf
        $\numerolista^a$ Lista de Exercícios -- Módulo \numeromodulo\\ Operações em $\z_m$}\\ \vspace{9pt} Prof. José Antônio O. Freitas
    \end{center}
    \hrule

    \vspace{.6cm}

    \questao{} Construa a tabela de soma, $\oplus$, e multiplicação, $\otimes$, nos seguintes conjuntos:
    \begin{multicols}{3}
        \begin{enumerate}[label=({\alph*})]
            \item $\z_5$
            \item $\z_6$
            \item $\z_8$
            \item $\z_9$
            \item $\z_{11}$
            \item $\z_{12}$
        \end{enumerate}
    \end{multicols}

    \vspace{.3cm}

    \questao{} Considere os seguintes subconjuntos $G$ de $\z_{12}$:
    \begin{multicols}{2}
        \begin{enumerate}[label=({\alph*})]
            \item $G=\{\overline{1},\overline{11}\}$;

            \item $G=\{\overline{0},\overline{4},\overline{8}\}$;

            \item $G=\{\overline{0},\overline{2},\overline{4},\overline{6},\overline{8},\overline{10}\}$

            \item $G=\{\overline{1}, \overline{3},\overline{5},\overline{7},\overline{9},\overline{11}\}$.
        \end{enumerate}
    \end{multicols}

    Para quais desses conjuntos valem as duas propriedades seguintes:
    \begin{enumerate}[label=({\roman*})]
        \item $x \oplus y \in G$, para todos $x$, $y \in G$.
        \item Para todo $x \in G$, existe $y \in G$ tal que $x \oplus y = \overline{0}$.
    \end{enumerate}

    \vspace{.3cm}

    \questao{} Considere os seguintes subconjuntos $G$ de $\z_{13}$:
    \begin{multicols}{2}
        \begin{enumerate}[label=({\alph*})]
            \item $G=\{\overline{1},\overline{12}\}$;

            \item $G=\{\overline{1},\overline{5},\overline{8},\overline{12}\}$;

            \item $G=\{\overline{1},\overline{2},\overline{3},\overline{4}, \overline{5},\overline{6},\overline{7},
            \overline{8},\overline{9},\overline{10},\overline{11},\overline{12}\}$

            \item $G=\{\overline{1}, \overline{3},\overline{5},\overline{7},\overline{9},\overline{11}\}$.
        \end{enumerate}
    \end{multicols}

    Para quais desses conjuntos valem as duas propriedades seguintes:
    \begin{enumerate}[label=({\roman*})]
        \item $x \otimes y \in G$, para todos $x$, $y \in G$.
        \item Para todo $x \in G$, existe $y \in G$ tal que $x \otimes y = \overline{1}$.
    \end{enumerate}

    \vspace{.3cm}

    \questao{} Resolva as seguintes equações:
    \begin{enumerate}[label={\alph*})]
        \item $x \oplus \overline{2} = \overline{0}$ no conjunto $\z_4$;
        \item $x^2 \oplus \overline{2}x = \overline{2}$ no conjunto $\z_5$;
        \item $x^3 \oplus \overline{2}x = \overline{3}$ no conjunto $\z_7$.
    \end{enumerate}

    \vspace{.3cm}

    \questao{} Resolva o sistema de equações
    \[
    \begin{cases}
        x \oplus y = \overline{0}\\
        \overline{2}x \oplus y = \overline{2}
    \end{cases}
    \]
    nos conjuntos:
    \begin{multicols}{2}
        \begin{enumerate}[label=({\alph*})]
            \item $\z_4$
            \item $\z_6$
            \item $\z_3$
            \item $\z_5$
        \end{enumerate}
    \end{multicols}

    \vspace{.3cm}

    \questao{} Resolva a equação $x^2 \oplus \overline{2}x \oplus \overline{1} = \overline{0}$ nos conjuntos $\z_7$ e $\z_{11}$, caso ela tenha raízes.

    \vspace{.3cm}

    \questao{} Para quais valores de $\overline{c}$ a equação $\overline{2}\otimes x = \overline{c}$ tem solução no conjunto $\z_5$.

    \vspace{.3cm}

    \questao{} Mostre que a relação de congruência módulo $m$ satisfaz as seguintes propriedades:
    \begin{enumerate}[label=({\alph*})]
        \item $a\equiv b\pmod{m}$ se, e somente se, $a - b\equiv 0\pmod{m}$.
        \item Se $a_{1}\equiv b_{1}\pmod{m}$ e $a_{2}\equiv b_{2}\pmod{m}$, então $a_{1}+a_{2}\equiv b_{1}+b_{2}\pmod{m}$.
        \item Se $a_{1}\equiv b_{1}\pmod{m}$ e $a_{2}\equiv b_{2}\pmod{m}$, então $a_{1}a_{2}\equiv b_{1}b_{2}\pmod{m}$.\label{item_provado}
        \item Se $a\equiv b\pmod{m}$, então $ax\equiv bx\pmod{m}$, para todo $x \in \z$.
        \item Vale a lei do cancelamento: se $d \in \z$ e $mdc(d,m) = 1$ então $ad \equiv bd \pmod{m}$ implica $a\equiv b \pmod{m}$.
    \end{enumerate}

    \vspace{.3cm}

    \questao{} Sejam $\overline{a}$, $\overline{b}$, $\overline{c} \in \z_m$. Prove que
    \begin{enumerate}[label=({\alph*})]
        \item $(\overline{a} \oplus \overline{b}) \otimes \overline{c} = (\overline{a} \otimes \overline{c}) \oplus (\overline{b} \otimes \overline{c})$

        \item $\overline{a} \otimes (\overline{b} \oplus \overline{c}) = (\overline{a} \otimes \overline{b}) \oplus (\overline{a} \otimes \overline{c})$
    \end{enumerate}
\end{document}
