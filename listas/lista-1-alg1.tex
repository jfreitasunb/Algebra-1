%!TEX program = xelatex
%!TEX encoding = ISO-8859-1
%!TEX program = xelatex
% !TEX encoding = ISO-8859-1
\def\ano{2019}
\def\semestre{2}
\def\disciplina{\'Algebra 1}
\def\turma{C}

\documentclass[12pt]{exam}

\usepackage{caption}
\usepackage{amssymb}
\usepackage{amsmath,amsfonts,amsthm,amstext}
\usepackage[brazil]{babel}
% \usepackage[latin1]{inputenc}
\usepackage{graphicx}
\graphicspath{{/ArquivosLinux/OneDrive/imagens-latex/}{D:/OneDrive - unb.br/imagens-latex/}}
\usepackage{enumitem}
\usepackage{multicol}
\usepackage{answers}
\usepackage{tikz,ifthen}
\usetikzlibrary{lindenmayersystems}
\usetikzlibrary[shadings]
\Newassociation{solucao}{Solution}{ans}
\newtheorem{exercicio}{}

\setlength{\topmargin}{-1.0in}
\setlength{\oddsidemargin}{0in}
\setlength{\textheight}{10.1in}
\setlength{\textwidth}{6.5in}
\setlength{\baselineskip}{12mm}

\extraheadheight{0.7in}
\firstpageheadrule
\runningheadrule
\lhead{
        \begin{minipage}[c]{1.7cm}
        \includegraphics[width=1.7cm]{unb.pdf}
        \end{minipage}%
        \hspace{0pt}
        \begin{minipage}[c]{4in}
          {Universidade de Brasília} --
          {Departamento de Matemática}
\end{minipage}
\vspace*{-0.8cm}
}
% \chead{Universidade de Brasília - Departamento de Matemática}
% \rhead{}
% \vspace*{-2cm}

\extrafootheight{.5in}
\footrule
\lfoot{\disciplina\ - \semestre$^o$/\ano\ - Módulo \numeromodulo}
\cfoot{}
\rfoot{Página \thepage\ de \numpages}

\newcounter{exercicios}
\renewcommand{\theexercicios}{\arabic{exercicios}}

\newenvironment{questao}[1]{
\refstepcounter{exercicios}
\ifx&#1&
\else
   \label{#1}
\fi
\noindent\textbf{Exercício {\theexercicios}:}
}

\newcommand{\resp}[1]{
\noindent{\bf Exercício #1: }}

\def\ano{2024}
\def\semestre{1}
\def\disciplina{Álgebra 1}
\def\nomeabreviado{Álgebra 1}
\def\turma{1}

\newcommand{\im}{{\rm Im\,}}
\newcommand{\dlim}[2]{\displaystyle\lim_{#1\rightarrow #2}}
\newcommand{\minf}{+\infty}
\newcommand{\ninf}{-\infty}
\newcommand{\cp}[1]{\mathbb{#1}}
\newcommand{\sub}{\subseteq}
\newcommand{\n}{\mathbb{N}}
\newcommand{\z}{\mathbb{Z}}
\newcommand{\rac}{\mathbb{Q}}
\newcommand{\real}{\mathbb{R}}
\newcommand{\complex}{\mathbb{C}}

\newcommand{\vesp}[1]{\vspace{ #1  cm}}

\newcommand{\compcent}[1]{\vcenter{\hbox{$#1\circ$}}}
\newcommand{\comp}{\mathbin{\mathchoice
        {\compcent\scriptstyle}{\compcent\scriptstyle}
        {\compcent\scriptscriptstyle}{\compcent\scriptscriptstyle}}}
\renewcommand{\sin}{{\rm sen\,}}
\renewcommand{\tan}{{\rm tg\,}}
\renewcommand{\csc}{{\rm cossec\,}}
\renewcommand{\cot}{{\rm cotg\,}}
\renewcommand{\sinh}{{\rm senh\,}}

\begin{document}

\begin{center}
{\Large\bf \disciplina\ - Turma \turma\ -- \semestre$^{o}$/\ano} \\ \vspace{9pt} {\large\bf
  $1^{\underline{a}}$ Lista de Exerc{\'\i}cios -- Conjuntos}\\ \vspace{9pt} Prof. Jos{\'e} Ant{\^o}nio O. Freitas
\end{center}
\hrule


\vspace{.6cm}

\questao Dados os conjuntos $A = \{0,1,2\}$, $B = \{0,2,3\}$ e $C = \{0,1,2,3,4\}$, classifique as afirma\c{c}\~oes a seguir em verdadeira ou falsa, justificando:
\begin{multicols}{2}
	\begin{enumerate}[label={\alph*})]
		\item $A \sub B$
		\item $A \in C$
		\item $B \sub C$
		\item $\{0,2\} \sub B$
		\item $\{0,2\} \sub C$
		\item $1 \sub C$
		\item $\{1,4\} \in C$
		\item $\{0,3\} \sub B$
		\item $B \sub A$
		\item $\{\vaz\} \sub A$
		\item $\vaz \in C$
		\item $A \sub C$
	\end{enumerate}	
\end{multicols}

\questao D\^e exemplos de conjuntos n\~ao vazios $A$, $B$ e $C$ tais que:
\begin{enumerate}[label={\alph*})]
	\item $A \sub B$, $C \sub B$ e $A \cap C = \emptyset$.
	\item $A \sub B$, $C \nsub B$ e $A \cap C = \emptyset$.
	\item $A \sub C$, $A \ne C$ e $B \cap C = \emptyset$.
	\item $A \sub (B \cap C)$, $B \sub C$, $B \ne C$ e $A \ne C$.
\end{enumerate}

\vspace{.3cm}

\questao Em cada um dos seguintes itens, determine se a afirma\c{c}\~ao \'e
verdadeira ou falsa. Se for verdadeira, demonstre-a. Se for falsa, exiba um
exemplo mostrando que a afirma\c{c}\~ao \'e falsa.
\begin{enumerate}[label={\alph*})]
\item Se $x \in A$ e $A \sub B$, ent\~ao $x \in B$.

\item Se $A \nsub B$ e $B \sub C$, ent\~ao $A \nsub C$.

\item Se $A \nsub B$ e $B \nsub C$, ent\~ao $A \nsub C$.

\item Se $x \in A$ e $A \nsub B$, ent\~ao $x \notin B$.

\item Se $A \sub B$ e $x \notin B$, ent\~ao $x \notin A$.

\item Se $A \sub E$, ent\~ao $A \sub C_E(A)$.

\item Se $A$, $B$ e $C$ s\~ao conjuntos, ent\~ao $A \cup (B \cap C) = (A \cup B) \cap C$.
\end{enumerate}

\vspace{.3cm}

\questao Determinar os elementos dos conjuntos $A$, $B$ e $E$ tais que:
\[
	A \cap B = \{b, c\}, \quad C_E(A) = \{d, e, f\} \quad \mbox{e}\quad C_E(B) = \{a, e, f\}.
\]

\vspace{.3cm}

\questao Dados os conjuntos $A = \{a, b, c, d\}$, $B = \{c, d, e, f\}$, $C = \{d, e, f, g\}$, determine:
\begin{enumerate}[label={\alph*})]
	\item $A - (B - C)$ e $(A - B) - C$
	\item $(A \cap B) - (B \cup C)$ e $(A \cup C) - (A \cup B)$
	\item $A \cap (B - C)$ e $(A \cap B) - C$
	\item $A - (B \cup C)$ e $(A - B) \cap (B - C)$
\end{enumerate}

\questao Dados os conjuntos $E = \{1,2,3,4,5,a,b,c,d\}$, $A = \{1,2,4,d\}$ e $B = \{a,2,4,b,5\}$, determine:
\begin{multicols}{2}
	\begin{enumerate}[label={\alph*})]
		\item $A \cup B$
		\item $A \cup C_E(B)$
		\item $C_E(A) \cup B$
		\item $C_E(A \cup B)$
	\end{enumerate}
\end{multicols}

\vspace{.3cm}

\questao Dados os conjuntos $A = \{3,6,9,12,15,18\}$, $B = \{2,3,5,7,11,13,17,19\}$ e $C = \{1,2,3,4,5,\dots,20\}$ calcule:
\begin{multicols}{3}
	\begin{enumerate}[label={\alph*})]
		\item $A - B$
		\item $A - C$
		\item $B - C$
		\item $(A - B) - C$
		\item $A - (B - C)$
		\item $(A \cup B) - C$
		\item $A - (B \cap C)$
		\item $(A \cup C) - B$
		\item $(B \cap C) - A$
	\end{enumerate}
\end{multicols}

\vspace{.3cm}

\questao Demonstre que:
\begin{enumerate}[label={\alph*})]
\item Se $A \sub B$ e $C \sub D$, ent\~ao $A \cap C \sub B \cap D$.

\item Se $A \sub B$ e $C = B - A$, ent\~ao $A = B - C$.

\item Se $A \cap B = \vaz$ e $A \cup B = C$, ent\~ao $A = C - B$.

\item Se $A\cup B = E$, ent\~ao $C_E(A) \sub B$. \textit{(Aqui suponha que $A$, $B \sub E$.)}

\item Se $A \cap B = \vaz$, ent\~ao $A \cup C_E(B) = C_E(B)$. \textit{(Aqui suponha que $A$, $B \sub E$.)}
  
\item $A = B$ se, e somente se, $A - B = B - A$.

\item $A \sub B$ se, e somente se, $A - B = \vaz$.

\item $A \cup B = \vaz$ se, e somente se, $A = \vaz$ e $B = \vaz$.

\item $A \cup B = A \cap B$ se, e somente se,  $A = B$.

\item $C_E(A) \sub C_E(B)$ se, e somente se, $A \cup B = A$. \textit{(Aqui suponha que $A$, $B \sub E$.)}

\item $C_E(A) \sub C_E(B)$ se, e somente se, $A \cap B = B$. \textit{(Aqui suponha que $A$, $B \sub E$.)}

\item $A \sub B$ se, e somente se,  $A \cap B = A$.
\end{enumerate}

\vspace{.3cm}

\questao Demonstre as seguintes igualdades.
\begin{enumerate}[label={\alph*})]
\item $A \cup (C_E(A) \cap B) = A \cup B$. \textit{(Aqui suponha que $A$, $B \sub E$.)}

\item $A \cap (C_E(A) \cup B) = A \cap B$. \textit{(Aqui suponha que $A$, $B \sub E$.)}

\item $(A - B) - C = A - (B \cup C)$.

\item $A \cup (B - C) = (A \cup B) - (C - A)$.

\item $A \cap (B - C) = (A \cap B) - (A \cap C)$.

\item $A - (B \cup C) = (A - B) \cap (A - C)$.

\item $A - (B \cap C) = (A - B) \cup (A - C)$.

\item $(A \cup B) - C = (A - C) \cup (B - C)$.

\item $(A \cap B) - C = (A - C) \cap (B - C)$.

\item $(A \cap B) \cap (A - B) = (A - B) \cap (B - A) = \vaz$.

\item $(A - B) \cup (B - A) = (A \cup B) - (A \cap B)$.

\item $A - (A - B) = A \cap B$.

\item $(A - B) - (B - C) = A - B$.

\item $(A - B) - B = A - B$.

\item $A \cup (B \cap (A \cup C)) = A \cup (B \cap C)$.

\item $C_E( A \cap [B \cup C]) = C_E(A) \cup [C_E(B)\cap C_E(C)]$. \textit{(Aqui suponha que $A$, $B$, $C \sub E$.)}

\item $C_E( A \cap B \cap C) = C_E(A) \cup C_E(B) \cup C_E(C)$. \textit{(Aqui suponha que $A$, $B$, $C \sub E$.)}

\item $C_E( A \cup B \cup C) = C_E(A) \cap C_E(B) \cap C_E(C)$. \textit{(Aqui suponha que $A$, $B$, $C \sub E$.)}
\end{enumerate}

\newpage

\questao Sejam $A$, $B$, $C$ e $D$ conjuntos.
\begin{enumerate}[label={\alph*})]
	\item Mostre que $A$ e $B$ s\~ao disjuntos se, e somente se, para todo conjunto n\~ao vazio $C$, $A \times C$ e $B \times C$ s\~ao disjuntos.
	\item Suponha $A \ne \emptyset$ e $C \ne \emptyset$, com $A \ne C$. Mostre que $A \sub B$ e $C \sub D$ se, e somente se, $A \times C \sub B \times D$.
\end{enumerate}

\vspace{.3cm}

\questao Sejam $X_1$, $X_1$, $Y_1$ e $Y_2$ subconjuntos contidos num conjunto $E$. Suponha que $X_1 \cup X_2 = E$, $Y_1 \cap Y_2 = \emptyset$, $X_1 \sub Y_1$ e $X_2 \sub Y_2$. Prove que $X_1 = Y_1$ e $X_2 = Y_2$.

\end{document}