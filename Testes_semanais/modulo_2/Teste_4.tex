%!TEX program = xelatex
%!TEX encoding = UTF-8
%Definições de Ano, Semestre, número e data da avaliação
\def\numeroteste{4}
\def\modulo{2}
\def\dataavaliacao{02/08/2024}

\documentclass[12pt]{exam}

\usepackage{caption}
\usepackage{amssymb}
\usepackage{amsmath,amsfonts,amsthm,amstext}
\usepackage[brazil]{babel}
% \usepackage[latin1]{inputenc}
\usepackage{graphicx}
\graphicspath{{/ArquivosLinux/OneDrive/imagens-latex/}{D:/OneDrive - unb.br/imagens-latex/}}
\usepackage{enumitem}
\usepackage{multicol}
\usepackage{answers}
\usepackage{tikz,ifthen}
\usetikzlibrary{lindenmayersystems}
\usetikzlibrary[shadings]
\def\ano{2024}
\def\semestre{1}
\def\disciplina{Álgebra 1}
\def\nomeabreviado{Álgebra 1}
\def\turma{1}

\newcommand{\im}{{\rm Im\,}}
\newcommand{\dlim}[2]{\displaystyle\lim_{#1\rightarrow #2}}
\newcommand{\minf}{+\infty}
\newcommand{\ninf}{-\infty}
\newcommand{\cp}[1]{\mathbb{#1}}
\newcommand{\sub}{\subseteq}
\newcommand{\n}{\mathbb{N}}
\newcommand{\z}{\mathbb{Z}}
\newcommand{\rac}{\mathbb{Q}}
\newcommand{\real}{\mathbb{R}}
\newcommand{\complex}{\mathbb{C}}

\newcommand{\vesp}[1]{\vspace{ #1  cm}}

\newcommand{\compcent}[1]{\vcenter{\hbox{$#1\circ$}}}
\newcommand{\comp}{\mathbin{\mathchoice
        {\compcent\scriptstyle}{\compcent\scriptstyle}
        {\compcent\scriptscriptstyle}{\compcent\scriptscriptstyle}}}
\renewcommand{\sin}{{\rm sen\,}}
\renewcommand{\tan}{{\rm tg\,}}
\renewcommand{\csc}{{\rm cossec\,}}
\renewcommand{\cot}{{\rm cotg\,}}
\renewcommand{\sinh}{{\rm senh\,}}
\newtheorem{definicao}{Definição}[section]
\newtheorem{definicoes}{Definições}[section]
\newtheorem{exemplo}{Exemplo}[section]
\newtheorem{exemplos}{Exemplos}[section]
\newtheorem{exercicio}{Exercício}
\newtheorem{observacao}{Observação:}[section]
\newtheorem{observacoes}{Observações:}[section]
\newtheorem*{solucao}{Solução:}
\newtheorem{proposicao}{Proposição}
\newtheorem{lema}{Lema}
\newtheorem{teorema}{Teorema}
\newtheorem{corolario}{Corolário}
\newenvironment{prova}[1][Prova]{\noindent\textbf{#1:} }{\qedsymbol}%{\ \rule{0.5em}{0.5em}}

\begin{document}

    \begin{figure}[h]
        \begin{minipage}[c]{1.7cm}
            \includegraphics[width=1.7cm]{unb.pdf}
        \end{minipage}
        \hspace{0pt}
        \begin{minipage}[c]{4in}
            {Universidade de Brasília} \\
            {Departamento de Matemática}
        \end{minipage}
    \end{figure}
    \vesp{-1}
    \hrule
    \begin{center}
        {\Large\bf \disciplina\ - Turma \turma}  \\
        \vesp{0.2} {\large\bf Teste \numeroteste\ - Módulo\ \modulo\ -\ \dataavaliacao}
    \end{center}

    \noindent{\bf \underline{Nome: \hspace{9.8cm} Mat.:\qquad \
            \hspace{2.1cm}}}\\
    \vspace*{.01cm}

    \noindent{\bf \underline{Nome: \hspace{9.8cm} Mat.:\qquad \
            \hspace{2.1cm}}}
    \vspace{.4cm}

    \questao{} Considere o conjunto $L = \{ a + b\sqrt{2} \mid a, b \in \rac\} \sub \real$. Para $a + b\sqrt{2}$, $c + d\sqrt{2} \in L$ definimos:
    \begin{align*}
        a + b\sqrt{2} &= c + d\sqrt{2} \quad \mbox{ quando } a = c \mbox{ e } b = d\\
        (a + b\sqrt{2}) \oplus (c + d\sqrt{2}) &= (a + c) + (b + d)\sqrt{2}\\
        (a + b\sqrt{2}) \otimes (c + d\sqrt{2}) &= (ac + 2bd) + (ad + bc)\sqrt{2}.
    \end{align*}

    Mostre que $(L, \oplus, \otimes)$ é um anel. Esse anel é comutativo? Justifique. Possui unidade? Justifique.

    \solucao Sejam $a + b\sqrt{2}$, $c + d\sqrt{2}$, $e + f\sqrt{2} \in L$. Temos:
    \begin{align*}
        ((a + b\sqrt{2}) \oplus (c + d\sqrt{2})) \oplus (e + f\sqrt{2}) &= ((a + c) + (b + d)\sqrt{2}) \oplus (e + f\sqrt{2})\\ &= (a + c) + e + ((b + d) + f)\sqrt{2} \\ &= a + (b + c) + (b + (d + f))\sqrt{2} \\ &= (a + b\sqrt{2}) \oplus ((b + c) + (d + f)\sqrt{2}) \\ &= (a + b\sqrt{2}) \oplus ((c + d\sqrt{2}) \oplus (e + f\sqrt{2})),
    \end{align*}
    como queríamos.

    Agora, dados $a + b\sqrt{2}$, $c + d\sqrt{2} \in L$ temos:
    \begin{align*}
        (a + b\sqrt{2}) \oplus (c + d\sqrt{2}) &= (a + c) + (b + d)\sqrt{2} \\ &= (c + a) + (b + d)\sqrt{2} \\ &= (c + d\sqrt{2}) \oplus (a + b\sqrt{2}),
    \end{align*}
    e com isso a soma em $L$ é comutativa.

    Agora tome $0_L = 0 + 0\sqrt{2} \in L$. Para todo $a + b\sqrt{2} \in L$ temos:
    \begin{align*}
        (a + b\sqrt{2}) \oplus 0_L = (a + b\sqrt{2}) + (0 + 0\sqrt{2}) = (a + 0) + (b + 0)\sqrt{2} = a + b\sqrt{2}.
    \end{align*}
    Logo $0_L = 0 + 0\sqrt{2}$ é o elemento neutro da soma $\oplus$ em $L$.

    Dado $a + b\sqrt{2} \in L$, tome $-a - b\sqrt{2} \in L$. Temos:
    \begin{align*}
        (a + b\sqrt{2}) \oplus (-a - b\sqrt{2}) = (a + (-a)) + (b + (-b))\sqrt{2} = 0 + 0\sqrt{2} = 0_L,
    \end{align*}
    assim $-a - b\sqrt{2} \in L$ é o oposto de $a + b\sqrt{2}$ na operação $\oplus$ em $L$.

    Sejam $a + b\sqrt{2}$, $c + d\sqrt{2}$, $e + f\sqrt{2} \in L$. Temos:
    \begin{align*}
        ((a + b\sqrt{2}) \otimes (c + d\sqrt{2})) \otimes (e + f\sqrt{2}) &= ((ac + 2bd) + (ad + bc)\sqrt{2}) \otimes (e + f\sqrt{2}) \\ &= ((ac + 2bd)e + 2(ad + bc)f) + ((ac + 2bd)f + (ad + bc)e)\sqrt{2},
    \end{align*}
    e pelo outro lado temos:
    \begin{align*}
        (a + b\sqrt{2}) \otimes ((c + d\sqrt{2}) \otimes (e + f\sqrt{2})) &= (a + b\sqrt{2}) \otimes ((ce + 2df) + (cf + de)\sqrt{2}) \\ &= (a(ce + 2df) + 2b(cf + de)) + (a(cf + de) + b(ce + 2df))\sqrt{2}.
    \end{align*}

    Com isso,
    \[
        ((a + b\sqrt{2}) \otimes (c + d\sqrt{2})) \otimes (e + f\sqrt{2}) = (a + b\sqrt{2}) \otimes ((c + d\sqrt{2}) \otimes (e + f\sqrt{2})).
    \]


    Sejam $a + b\sqrt{2}$, $c + d\sqrt{2}$, $e + f\sqrt{2} \in L$. Temos:
    \begin{align*}
        ((a + b\sqrt{2}) \oplus (c + d\sqrt{2})) \otimes (e + f\sqrt{2}) &= ((a + c) + (b + d)\sqrt{2}) \otimes (e + f\sqrt{2}) \\ &= ((a + c)e + 2(b + d)f) + ((a + c)f + (b + d)e)\sqrt{2}
    \end{align*}
    e pelo outro lado temos:
    \begin{align*}
        ((a + b\sqrt{2}) \otimes (e + f\sqrt{2})) \oplus ((c + d\sqrt{2}) \otimes (e + f\sqrt{2})) &= ((ae + 2bf) +(af + be)\sqrt{2}) \\ &\oplus ((ce + 2df) + (cf + de)\sqrt{2}) \\ &= (ae + ebf + ce + edf) + (af + be + cf + de)\sqrt{2}.
    \end{align*}
    Logo
    \[
        ((a + b\sqrt{2}) \oplus (c + d\sqrt{2})) \otimes (e + f\sqrt{2}) = ((a + b\sqrt{2}) \otimes (e + f\sqrt{2})) \oplus ((c + d\sqrt{2}) \otimes (e + f\sqrt{2})),
    \]
    como queríamos.

    Finalmente, sejam $a + b\sqrt{2}$, $c + d\sqrt{2}$, $e + f\sqrt{2} \in L$. Temos:
    \begin{align*}
        (a + b\sqrt{2}) \otimes ((c + d\sqrt{2}) \oplus (e + f\sqrt{2})) &= (a + b \sqrt{2}) \otimes ((c + e) + (d + f)\sqrt{2}) \\ &= (a(c + e) + 2b(d + f)) + (a(d + f) + b(c + e))\sqrt{2} \\ &= (ac + ae + 2bd + 2bf) + (ad + af + bc + be)\sqrt{2},
    \end{align*}
    e pelo outro lado temos:
    \begin{align*}
        ((a + b\sqrt{2}) \otimes (c + d\sqrt{2})) \oplus ((a + b\sqrt{2}) \otimes (e + f\sqrt{2})) &= ((ac + 2bd) + (ad + bc)\sqrt{2}) \\ &\oplus ((ae + 2bf) + (af + be)\sqrt{2}) \\ &= (ac + 2bd + ae + 2bf) + (ad + bc + af + be)\sqrt{2}.
    \end{align*}
    Logo
    \[
        (a + b\sqrt{2}) \otimes ((c + d\sqrt{2}) \oplus (e + f\sqrt{2})) = ((a + b\sqrt{2}) \otimes (c + d\sqrt{2})) \oplus ((a + b\sqrt{2}) \otimes (e + f\sqrt{2})),
    \]
    como queríamos.

    Portanto $(L, \oplus, \otimes)$ é um anel.

    Para verificar se esse anel é comutativo, sejam $a + b\sqrt{2}$, $c + d\sqrt{2} \in L$. Temos:
    \begin{align*}
        (a + b\sqrt{2}) \otimes (c + d\sqrt{2}) &= (ac + 2bd) +(ad + bc)\sqrt{2} \\ &= (ca + 2bd) + (da + cb)\sqrt{2} \\ &= (c + d)\sqrt{2} \otimes (a + b\sqrt{2}).
    \end{align*}

    Logo $(L, \oplus, \otimes)$ é um anel comutativo.

    Finalmente, tome $1_L = 1 + 0\sqrt{2} \in L$. Para todo $a + b\sqrt{2} \in L$ temos
    \begin{align*}
        (a + b\sqrt{2}) \otimes 1_L &= (a + b\sqrt{2}) \otimes (1 + 0\sqrt{2}) \\ &= (a\cdot 1 + 2b\cdot 0) + (a\cdot 0 + b\cdot 1)\sqrt{2} \\ &= a + b\sqrt{2}.
    \end{align*}

    Logo $1_L = 1 + 0\sqrt{2}$ é o elemento neutro da operação $\otimes$ em $L$. Assim $(L, \oplus, \otimes)$ é um anel com unidade.

    Portanto $(L, \oplus, \otimes)$ é um anel comutativo unitário.

\end{document}
