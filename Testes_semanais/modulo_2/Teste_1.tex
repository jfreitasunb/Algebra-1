%!TEX program = xelatex
%!TEX encoding = UTF-8
%Definições de Ano, Semestre, número e data da avaliação
\def\numeroteste{1}
\def\modulo{2}
\def\dataavaliacao{12/07/2024}

\documentclass[12pt]{exam}

\usepackage{caption}
\usepackage{amssymb}
\usepackage{amsmath,amsfonts,amsthm,amstext}
\usepackage[brazil]{babel}
% \usepackage[latin1]{inputenc}
\usepackage{graphicx}
\graphicspath{{/ArquivosLinux/OneDrive/imagens-latex/}{D:/OneDrive - unb.br/imagens-latex/}}
\usepackage{enumitem}
\usepackage{multicol}
\usepackage{answers}
\usepackage{tikz,ifthen}
\usetikzlibrary{lindenmayersystems}
\usetikzlibrary[shadings]
\def\ano{2024}
\def\semestre{1}
\def\disciplina{Álgebra 1}
\def\nomeabreviado{Álgebra 1}
\def\turma{1}

\newcommand{\im}{{\rm Im\,}}
\newcommand{\dlim}[2]{\displaystyle\lim_{#1\rightarrow #2}}
\newcommand{\minf}{+\infty}
\newcommand{\ninf}{-\infty}
\newcommand{\cp}[1]{\mathbb{#1}}
\newcommand{\sub}{\subseteq}
\newcommand{\n}{\mathbb{N}}
\newcommand{\z}{\mathbb{Z}}
\newcommand{\rac}{\mathbb{Q}}
\newcommand{\real}{\mathbb{R}}
\newcommand{\complex}{\mathbb{C}}

\newcommand{\vesp}[1]{\vspace{ #1  cm}}

\newcommand{\compcent}[1]{\vcenter{\hbox{$#1\circ$}}}
\newcommand{\comp}{\mathbin{\mathchoice
        {\compcent\scriptstyle}{\compcent\scriptstyle}
        {\compcent\scriptscriptstyle}{\compcent\scriptscriptstyle}}}
\renewcommand{\sin}{{\rm sen\,}}
\renewcommand{\tan}{{\rm tg\,}}
\renewcommand{\csc}{{\rm cossec\,}}
\renewcommand{\cot}{{\rm cotg\,}}
\renewcommand{\sinh}{{\rm senh\,}}
\newtheorem{definicao}{Definição}[section]
\newtheorem{definicoes}{Definições}[section]
\newtheorem{exemplo}{Exemplo}[section]
\newtheorem{exemplos}{Exemplos}[section]
\newtheorem{exercicio}{Exercício}
\newtheorem{observacao}{Observação:}[section]
\newtheorem{observacoes}{Observações:}[section]
\newtheorem*{solucao}{Solução:}
\newtheorem{proposicao}{Proposição}
\newtheorem{lema}{Lema}
\newtheorem{teorema}{Teorema}
\newtheorem{corolario}{Corolário}
\newenvironment{prova}[1][Prova]{\noindent\textbf{#1:} }{\qedsymbol}%{\ \rule{0.5em}{0.5em}}

\begin{document}

    \begin{figure}[h]
        \begin{minipage}[c]{1.7cm}
            \includegraphics[width=1.7cm]{unb.pdf}
        \end{minipage}
        \hspace{0pt}
        \begin{minipage}[c]{4in}
            {Universidade de Brasília} \\
            {Departamento de Matemática}
        \end{minipage}
    \end{figure}
    \vesp{-1}
    \hrule
    \begin{center}
        {\Large\bf \disciplina\ - Turma \turma}  \\
        \vesp{0.2} {\large\bf Teste \numeroteste\ - Módulo\ \modulo\ -\ \dataavaliacao}
    \end{center}

    \noindent{\bf \underline{Nome: \hspace{9.8cm} Mat.:\qquad \
            \hspace{2.1cm}}}\\
    \vspace*{.01cm}

    \noindent{\bf \underline{Nome: \hspace{9.8cm} Mat.:\qquad \
            \hspace{2.1cm}}}
    \vspace{.4cm}

    \noindent Considere a função $g : \z_5 \times \z_9 \to \z_5 \times \z_9$ tal que $g(\overline{x},\overline{y}) = (\overline{2x + 3}, \overline{4y + 5})$.

    \vspace{.5cm}
    \questao $g$ é injetora?

    \solucao Para mostrar que $g$ é injetora começamos tomando dois elementos de $\z_5 \times \z_9$, digamos $(\overline{x}_1, \overline{y}_1)$, $(\overline{x}_2, \overline{y}_2) \in \z_5 \times \z_9$, tais que
    \begin{align*}
        g(\overline{x}_1, \overline{y}_1) &= g(\overline{x}_2, \overline{y}_2)\\
        (\overline{2x_1 + 3}, \overline{4y_1 + 5}) &= (\overline{2x_2 + 3}, \overline{4y_1 + 5})
    \end{align*}

    Daí devemos ter:
    \begin{align}
        \overline{2x_1 + 3} &= \overline{2x_2 + 3}\label{primeiraequacao}\\
        \overline{4y_1 + 5} &= \overline{4y_1 + 5}\label{segundaequacao}
    \end{align}

    Devemos resolver a equação \eqref{primeiraequacao} em $\z_5$. Primeiro, podemos escrever essa equação como
    \[
        \overline{2}\overline{x}_1 \oplus \overline{3} = \overline{2}\overline{x}_2 \oplus \overline{3}.
    \]

    Somando cada lado dessa equação com $\overline{2}$, obtemos
    \[
        \overline{2}\overline{x}_1 = \overline{2}\overline{x}_2
    \]
    pois $\overline{2} \oplus \overline{3} = \overline{0}$ em $\z_5$.


    Agora, multiplicando cada lado da última equação por $\overline{3}$ obtemos
    \[
        \overline{x}_1 = \overline{x}_2
    \]
    pois em $\z_5$ temos $\overline{2} \otimes \overline{3} = \overline{1}$.

    Agora vamos resolver a equação \eqref{segundaequacao} em $\z_9$. Primeiro, podemos escrever essa equação como
    \[
        \overline{4}\overline{y}_1 \oplus \overline{5} = \overline{4}\overline{y}_2 \oplus \overline{5}.
    \]

    Somando cada lado dessa equação com $\overline{4}$, obtemos
    \[
        \overline{4}\overline{y}_1 = \overline{4}\overline{y}_2
    \]
    pois $\overline{4} \oplus \overline{5} = \overline{0}$ em $\z_9$.


    Agora, multiplicando cada lado da última equação por $\overline{7}$ obtemos
    \[
        \overline{y}_1 = \overline{y}_2
    \]
    pois em $\z_9$ temos $\overline{4} \otimes \overline{7} = \overline{1}$.

    Portanto $(\overline{x}_1, \overline{y}_1) = (\overline{x}_2, \overline{y}_2)$ e então $g$ é injetora.

    \vspace{1cm}
    \questao $g$ é sobrejetora?

    \solucao Dado $(\overline{c}, \overline{d}) \in \z_5 \times \z_9$ tome $(\overline{a}, \overline{b}) \in  \z_5 \times \z_9$ dado por
    \[
        (\overline{a}, \overline{b}) = (\overline{3c + 1}, \overline{7d + 1}).
    \]
    Assim temos
    \begin{align*}
        g(\overline{a}, \overline{b}) &= (\overline{2a + 3}, \overline{4b + 5}) \\ &= (\overline{2(3c + 1) + 3}, \overline{4(7d + 1) + 5})\\ &= (\overline{6c + 2 + 3)}, \overline{28d + 4 + 5}) = (\overline{6c + 5}, \overline{28d + 9})\\ &= (\overline{6}\overline{c} \oplus \overline{5}, \overline{28}\overline{d} \oplus \overline{9})\\ &= (\overline{c}, \overline{d}).
    \end{align*}

    Portanto, $g$ é sobrejetora.
\vspace{1cm}
\end{document}
