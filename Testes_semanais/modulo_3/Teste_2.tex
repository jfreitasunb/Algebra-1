%!TEX program = xelatex
%!TEX encoding = UTF-8
%Definições de Ano, Semestre, número e data da avaliação
\def\numeroteste{2}
\def\modulo{3}
\def\dataavaliacao{23/08/2024}

\documentclass[12pt]{exam}

\usepackage{caption}
\usepackage{amssymb}
\usepackage{amsmath,amsfonts,amsthm,amstext}
\usepackage[brazil]{babel}
% \usepackage[latin1]{inputenc}
\usepackage{graphicx}
\graphicspath{{/ArquivosLinux/OneDrive/imagens-latex/}{D:/OneDrive - unb.br/imagens-latex/}}
\usepackage{enumitem}
\usepackage{multicol}
\usepackage{answers}
\usepackage{tikz,ifthen}
\usetikzlibrary{lindenmayersystems}
\usetikzlibrary[shadings]
\def\ano{2024}
\def\semestre{1}
\def\disciplina{Álgebra 1}
\def\nomeabreviado{Álgebra 1}
\def\turma{1}

\newcommand{\im}{{\rm Im\,}}
\newcommand{\dlim}[2]{\displaystyle\lim_{#1\rightarrow #2}}
\newcommand{\minf}{+\infty}
\newcommand{\ninf}{-\infty}
\newcommand{\cp}[1]{\mathbb{#1}}
\newcommand{\sub}{\subseteq}
\newcommand{\n}{\mathbb{N}}
\newcommand{\z}{\mathbb{Z}}
\newcommand{\rac}{\mathbb{Q}}
\newcommand{\real}{\mathbb{R}}
\newcommand{\complex}{\mathbb{C}}

\newcommand{\vesp}[1]{\vspace{ #1  cm}}

\newcommand{\compcent}[1]{\vcenter{\hbox{$#1\circ$}}}
\newcommand{\comp}{\mathbin{\mathchoice
        {\compcent\scriptstyle}{\compcent\scriptstyle}
        {\compcent\scriptscriptstyle}{\compcent\scriptscriptstyle}}}
\renewcommand{\sin}{{\rm sen\,}}
\renewcommand{\tan}{{\rm tg\,}}
\renewcommand{\csc}{{\rm cossec\,}}
\renewcommand{\cot}{{\rm cotg\,}}
\renewcommand{\sinh}{{\rm senh\,}}
\newtheorem{definicao}{Definição}[section]
\newtheorem{definicoes}{Definições}[section]
\newtheorem{exemplo}{Exemplo}[section]
\newtheorem{exemplos}{Exemplos}[section]
\newtheorem{exercicio}{Exercício}
\newtheorem{observacao}{Observação:}[section]
\newtheorem{observacoes}{Observações:}[section]
\newtheorem*{solucao}{Solução:}
\newtheorem{proposicao}{Proposição}
\newtheorem{lema}{Lema}
\newtheorem{teorema}{Teorema}
\newtheorem{corolario}{Corolário}
\newenvironment{prova}[1][Prova]{\noindent\textbf{#1:} }{\qedsymbol}%{\ \rule{0.5em}{0.5em}}

\begin{document}

    \begin{figure}[h]
        \begin{minipage}[c]{1.7cm}
            \includegraphics[width=1.7cm]{unb.pdf}
        \end{minipage}
        \hspace{0pt}
        \begin{minipage}[c]{4in}
            {Universidade de Brasília} \\
            {Departamento de Matemática}
        \end{minipage}
    \end{figure}
    \vesp{-1}
    \hrule
    \begin{center}
        {\Large\bf \disciplina\ - Turma \turma}  \\
        \vesp{0.2} {\large\bf Teste \numeroteste\ - Módulo\ \modulo\ -\ \dataavaliacao}
    \end{center}

    \noindent{\bf \underline{Nome: \hspace{9.8cm} Mat.:\qquad \
            \hspace{2.1cm}}}\\
    \vspace*{.01cm}

    \noindent{\bf \underline{Nome: \hspace{9.8cm} Mat.:\qquad \
            \hspace{2.1cm}}}

    \vspace{.4cm}


    \questao Seja $f : \z \to M_2(\z_3)$ dada por
    \[
        f(a) = \begin{bmatrix}
            \overline{a} & \overline{0}\\
            \overline{0} & \overline{a}
        \end{bmatrix}.
    \]
    \begin{enumerate}[label=({\alph*})]
        \item Mostre que $f$ é um homomorfismo de anéis.

        \item $f$ é um isomorfismo? Justifique.
    \end{enumerate}

    \solucao
    \begin{enumerate}[label=({\alph*})]
        \item Sejam $x$, $y \in \z$. Temos
        \begin{align*}
            f(x + y) &= \begin{bmatrix}
                \overline{x + y} & \overline{0}\\
                \overline{0} & \overline{x + y}
            \end{bmatrix} = \begin{bmatrix}
            \overline{x} + \overline{y} & \overline{0}\\
            \overline{0} & \overline{x} + \overline{y}
            \end{bmatrix} = \begin{bmatrix}
            \overline{x} & \overline{0}\\
            \overline{0} & \overline{x}
            \end{bmatrix} + \begin{bmatrix}
            \overline{y} & \overline{0}\\
            \overline{0} & \overline{y} = f(x) + f(y)
            \end{bmatrix}\\
            f(xy) &= \begin{bmatrix}
                \overline{xy} & \overline{0}\\
                \overline{0} & \overline{xy}
            \end{bmatrix} =
            \begin{bmatrix}
                \overline{x} & \overline{0}\\
                \overline{0} & \overline{x}
            \end{bmatrix}\begin{bmatrix}
                \overline{y} & \overline{0}\\
                \overline{0} & \overline{y}
            \end{bmatrix} = f(x)f(y)
        \end{align*}
        Logo $f$ é um homomorfismo de anéis.

        \item Para que $f$ seja um isomorfismo é preciso que $f$ seja bijetora. Assim $f$ deve ser injetora e sobrejetora.

        Para determinar se $f$ é injetora podemos encontrar o kernel de $f$:
        \[
            \ker(f) = \left\{x \in \z \mid f(x) = \begin{bmatrix}
                \overline{0} & \overline{0}\\\overline{0} & \overline{0}
            \end{bmatrix}\right\}.
        \]
        Agora
        \begin{align*}
            f(x) &= \begin{bmatrix}
                \overline{0} & \overline{0}\\\overline{0} & \overline{0}
            \end{bmatrix}\\
            \begin{bmatrix}
                \overline{x} & \overline{0}\\\overline{0} & \overline{x}
            \end{bmatrix} &= \begin{bmatrix}
            \overline{0} & \overline{0}\\\overline{0} & \overline{0}
            \end{bmatrix}
        \end{align*}
        Assim devemos ter $\overline{x} = \overline{0}$ e com isso
        \[
            \ker(f) = \{x \in \z \mid \overline{x} = \overline{0}\} = \{0, \pm 3, \pm 6, \dots\}
        \]
        e então $f$ não é injetora. Portanto $f$ não é um isomorfismo.
    \end{enumerate}
    \vspace{.4cm}

    \questao Sejam os anéis $A = \{ a + b\sqrt{-2} \mid a,\ b \in \z\}$ e $B = M_2(\z_7)$.
    \begin{enumerate}[label=({\alph*})]
        \item Mostre que $g : A \to B$ dada por
        \[
            g(a + b\sqrt{-2}) =
            \begin{pmatrix}
                \overline{a} & \overline{5}\overline{b}\\
                \overline{b} & \overline{a}
            \end{pmatrix}
        \]
        é um homomorfismo.

        \item $g$ é um isomorfismo? Justifique.
    \end{enumerate}

    \solucao
    \begin{enumerate}[label=({\alph*})]
        \item Sejam $a + b\sqrt{-2}$, $c + d\sqrt{-2} \in A$. Temos:
        \begin{align*}
            g((a + b\sqrt{-2}) + (c + d\sqrt{-2})) &= g((a + c) + (b + d)\sqrt{-2}) = \begin{pmatrix}
                \overline{a + c} & \overline{5}\overline{(b + d)}\\\overline{b + d} & \overline{a + c}
            \end{pmatrix}\\ &= \begin{pmatrix}
            \overline{a} & \overline{5}\overline{b}\\\overline{b} & \overline{a}
            \end{pmatrix} + \begin{pmatrix}
            \overline{c} & \overline{5}\overline{d}\\\overline{d} & \overline{c}
            \end{pmatrix} \\ &= g(a + b\sqrt{-2}) + g(c + d\sqrt{-2})\\
            g((a + b\sqrt{-2})(c + d\sqrt{-2})) &= g((ac - 2bd) + (ad + bc)\sqrt{-2}) = \begin{pmatrix}
                \overline{ac - 2bd} & \overline{5}(\overline{ad + bc})\\\overline{ad + bc} & \overline{ac - 2bd}
            \end{pmatrix} \\ &= \begin{pmatrix}
            \overline{ac} + \overline{-2}\overline{bd} & \overline{5}\overline{ad} + \overline{5}\overline{bc}\\\overline{ad} + \overline{bc} & \overline{ac} + \overline{-2}\overline{bd}
            \end{pmatrix}
        \end{align*}
        Agora, observe que $5 \equiv -2 \pmod 7$, assim $\overline{5} = \overline{-2}$ e daí podemos escrever
        \begin{align*}
            g((a + b\sqrt{-2})(c + d\sqrt{-2})) &= \begin{pmatrix}
                \overline{ac} + \overline{-2}\overline{bd} & \overline{5}\overline{ad} + \overline{5}\overline{cd}\\\overline{ad} + \overline{bc} & \overline{ac} + \overline{-2}\overline{bd}
            \end{pmatrix} = \begin{pmatrix}
            \overline{ac} + \overline{5}\overline{bd} & \overline{5}\overline{ad} + \overline{5}\overline{bc}\\\overline{ad} + \overline{bc} & \overline{ac} + \overline{5}\overline{bd}
            \end{pmatrix} \\ &= \begin{pmatrix}
            \overline{a} & \overline{5}\overline{b}\\\overline{b} & \overline{a}
            \end{pmatrix} \begin{pmatrix}
            \overline{c} & \overline{5}\overline{d}\\\overline{d} & \overline{c}
            \end{pmatrix} \\ &= g(a + b\sqrt{-2})g(c + d\sqrt{-2})
        \end{align*}
        e portanto $g$ é um homomorfismo de anéis.

         \item Para que $g$ seja um isomorfismo é preciso que $g$ seja bijetora. Assim $g$ deve ser injetora e sobrejetora.

        Para determinar se $g$ é injetora podemos encontrar o kernel de $g$:
        \[
            \ker(g) = \left\{a + b\sqrt{-2} \in A \mid g(a + b\sqrt{-2}) = \begin{pmatrix}
            \overline{0} & \overline{0}\\\overline{0} & \overline{0}
        \end{pmatrix}\right\}.
        \]
        Agora
        \begin{align*}
            g(x) &= \begin{bmatrix}\overline{0} & \overline{0}\\\overline{0} & \overline{0}\end{bmatrix}\\
            \begin{bmatrix}\overline{a} & \overline{5}\overline{b}\\\overline{b} & \overline{a}
            \end{bmatrix} &= \begin{bmatrix}
                \overline{0} & \overline{0}\\\overline{0} & \overline{0}
            \end{bmatrix}
        \end{align*}
        Assim devemos ter $\overline{a} = \overline{b} = \overline{0}$ e com isso
        \[
            \ker(g) = \{a + b\sqrt{-2} \in A \mid \overline{a} = \overline{b} = \overline{0}\} = \{a,\ b \in \z \mid a = 7k,\ b = 7l, k,l \in \z\}
        \]
        e então $g$ não é injetora. Portanto $g$ não é um isomorfismo.
    \end{enumerate}
    \vspace{1cm}
\end{document}
