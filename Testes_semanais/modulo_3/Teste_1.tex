%!TEX program = xelatex
%!TEX encoding = UTF-8
%Definições de Ano, Semestre, número e data da avaliação
\def\numeroteste{1}
\def\modulo{3}
\def\dataavaliacao{16/08/2024}

\documentclass[12pt]{exam}

\usepackage{caption}
\usepackage{amssymb}
\usepackage{amsmath,amsfonts,amsthm,amstext}
\usepackage[brazil]{babel}
% \usepackage[latin1]{inputenc}
\usepackage{graphicx}
\graphicspath{{/ArquivosLinux/OneDrive/imagens-latex/}{D:/OneDrive - unb.br/imagens-latex/}}
\usepackage{enumitem}
\usepackage{multicol}
\usepackage{answers}
\usepackage{tikz,ifthen}
\usetikzlibrary{lindenmayersystems}
\usetikzlibrary[shadings]
\def\ano{2024}
\def\semestre{1}
\def\disciplina{Álgebra 1}
\def\nomeabreviado{Álgebra 1}
\def\turma{1}

\newcommand{\im}{{\rm Im\,}}
\newcommand{\dlim}[2]{\displaystyle\lim_{#1\rightarrow #2}}
\newcommand{\minf}{+\infty}
\newcommand{\ninf}{-\infty}
\newcommand{\cp}[1]{\mathbb{#1}}
\newcommand{\sub}{\subseteq}
\newcommand{\n}{\mathbb{N}}
\newcommand{\z}{\mathbb{Z}}
\newcommand{\rac}{\mathbb{Q}}
\newcommand{\real}{\mathbb{R}}
\newcommand{\complex}{\mathbb{C}}

\newcommand{\vesp}[1]{\vspace{ #1  cm}}

\newcommand{\compcent}[1]{\vcenter{\hbox{$#1\circ$}}}
\newcommand{\comp}{\mathbin{\mathchoice
        {\compcent\scriptstyle}{\compcent\scriptstyle}
        {\compcent\scriptscriptstyle}{\compcent\scriptscriptstyle}}}
\renewcommand{\sin}{{\rm sen\,}}
\renewcommand{\tan}{{\rm tg\,}}
\renewcommand{\csc}{{\rm cossec\,}}
\renewcommand{\cot}{{\rm cotg\,}}
\renewcommand{\sinh}{{\rm senh\,}}
\newtheorem{definicao}{Definição}[section]
\newtheorem{definicoes}{Definições}[section]
\newtheorem{exemplo}{Exemplo}[section]
\newtheorem{exemplos}{Exemplos}[section]
\newtheorem{exercicio}{Exercício}
\newtheorem{observacao}{Observação:}[section]
\newtheorem{observacoes}{Observações:}[section]
\newtheorem*{solucao}{Solução:}
\newtheorem{proposicao}{Proposição}
\newtheorem{lema}{Lema}
\newtheorem{teorema}{Teorema}
\newtheorem{corolario}{Corolário}
\newenvironment{prova}[1][Prova]{\noindent\textbf{#1:} }{\qedsymbol}%{\ \rule{0.5em}{0.5em}}

\begin{document}

    \begin{figure}[h]
        \begin{minipage}[c]{1.7cm}
            \includegraphics[width=1.7cm]{unb.pdf}
        \end{minipage}
        \hspace{0pt}
        \begin{minipage}[c]{4in}
            {Universidade de Brasília} \\
            {Departamento de Matemática}
        \end{minipage}
    \end{figure}
    \vesp{-1}
    \hrule
    \begin{center}
        {\Large\bf \disciplina\ - Turma \turma}  \\
        \vesp{0.2} {\large\bf Teste \numeroteste\ - Módulo\ \modulo\ -\ \dataavaliacao}
    \end{center}

    \noindent{\bf \underline{Nome: \hspace{9.8cm} Mat.:\qquad \
            \hspace{2.1cm}}}\\
    \vspace*{.01cm}

    \noindent{\bf \underline{Nome: \hspace{9.8cm} Mat.:\qquad \
            \hspace{2.1cm}}}
    \vspace{.4cm}


    \questao Seja $d \in \z$ e considere o subconjunto de $M_2(\z)$ dado por
    \[
    M_2^d(\z) = \left\{\begin{pmatrix} a & db \\ b & a \end{pmatrix} \mid a, b \in \z\right\}.
    \]
    Mostre que $M_2^d(\z)$ é um subanel de $M_2(\z)$.
\vspace{1cm}
\end{document}
