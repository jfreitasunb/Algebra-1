%!TEX program = xelatex
%!TEX encoding = UTF-8
%Definições de Ano, Semestre, número e data da avaliação
\def\numeroteste{3}
\def\modulo{3}
\def\dataavaliacao{30/08/2024}

\documentclass[12pt]{exam}

\usepackage{caption}
\usepackage{amssymb}
\usepackage{amsmath,amsfonts,amsthm,amstext}
\usepackage[brazil]{babel}
% \usepackage[latin1]{inputenc}
\usepackage{graphicx}
\graphicspath{{/ArquivosLinux/OneDrive/imagens-latex/}{D:/OneDrive - unb.br/imagens-latex/}}
\usepackage{enumitem}
\usepackage{multicol}
\usepackage{answers}
\usepackage{tikz,ifthen}
\usetikzlibrary{lindenmayersystems}
\usetikzlibrary[shadings]
\def\ano{2024}
\def\semestre{1}
\def\disciplina{Álgebra 1}
\def\nomeabreviado{Álgebra 1}
\def\turma{1}

\newcommand{\im}{{\rm Im\,}}
\newcommand{\dlim}[2]{\displaystyle\lim_{#1\rightarrow #2}}
\newcommand{\minf}{+\infty}
\newcommand{\ninf}{-\infty}
\newcommand{\cp}[1]{\mathbb{#1}}
\newcommand{\sub}{\subseteq}
\newcommand{\n}{\mathbb{N}}
\newcommand{\z}{\mathbb{Z}}
\newcommand{\rac}{\mathbb{Q}}
\newcommand{\real}{\mathbb{R}}
\newcommand{\complex}{\mathbb{C}}

\newcommand{\vesp}[1]{\vspace{ #1  cm}}

\newcommand{\compcent}[1]{\vcenter{\hbox{$#1\circ$}}}
\newcommand{\comp}{\mathbin{\mathchoice
        {\compcent\scriptstyle}{\compcent\scriptstyle}
        {\compcent\scriptscriptstyle}{\compcent\scriptscriptstyle}}}
\renewcommand{\sin}{{\rm sen\,}}
\renewcommand{\tan}{{\rm tg\,}}
\renewcommand{\csc}{{\rm cossec\,}}
\renewcommand{\cot}{{\rm cotg\,}}
\renewcommand{\sinh}{{\rm senh\,}}
\newtheorem{definicao}{Definição}[section]
\newtheorem{definicoes}{Definições}[section]
\newtheorem{exemplo}{Exemplo}[section]
\newtheorem{exemplos}{Exemplos}[section]
\newtheorem{exercicio}{Exercício}
\newtheorem{observacao}{Observação:}[section]
\newtheorem{observacoes}{Observações:}[section]
\newtheorem*{solucao}{Solução:}
\newtheorem{proposicao}{Proposição}
\newtheorem{lema}{Lema}
\newtheorem{teorema}{Teorema}
\newtheorem{corolario}{Corolário}
\newenvironment{prova}[1][Prova]{\noindent\textbf{#1:} }{\qedsymbol}%{\ \rule{0.5em}{0.5em}}

\begin{document}

    \begin{figure}[h]
        \begin{minipage}[c]{1.7cm}
            \includegraphics[width=1.7cm]{unb.pdf}
        \end{minipage}
        \hspace{0pt}
        \begin{minipage}[c]{4in}
            {Universidade de Brasília} \\
            {Departamento de Matemática}
        \end{minipage}
    \end{figure}
    \vesp{-1}
    \hrule
    \begin{center}
        {\Large\bf \disciplina\ - Turma \turma}  \\
        \vesp{0.2} {\large\bf Teste \numeroteste\ - Módulo\ \modulo\ -\ \dataavaliacao}
    \end{center}

    \noindent{\bf \underline{Nome: \hspace{9.8cm} Mat.:\qquad \
            \hspace{2.1cm}}}\\
    \vspace*{.01cm}

    \noindent{\bf \underline{Nome: \hspace{9.8cm} Mat.:\qquad \
            \hspace{2.1cm}}}

    \vspace{.6cm}


    \questao Seja $G = \real$ e a operação $\star$ dada por $x \star y = \sqrt[3]{x^3 + y^3}$, para $x$, $y \in \real$. Mostre que $(\real, \star)$ é um grupo. Esse grupo é comutativo? Justifique.

    \noindent\solucao Primeiro, para todos $x$, $y \in \real$ temos
    \begin{align*}
        x \star y = \sqrt[3]{x^3 + y^3} = \sqrt[3]{y^3 + x^3} = y \star x
    \end{align*}
    e assim a operação $\star$ é comutativa.

    Agora sejam $x$, $y$, $z \in \real$. Temos
    \begin{align*}
        (x\star y) \star z &= \sqrt[3]{x^3 + y^3} \star z = \sqrt[3]{\sqrt[3]{x^3 + y^3}^3 + z^3} \\ &= \sqrt[3]{x^3 + y^3 + z^3} = \sqrt[3]{x^3 + \sqrt[3]{y^3 + z^3}^3} \\ &= x \star \sqrt[3]{y^3 + z^3} \\ &= x \star (y \star z)
    \end{align*}
    e assim a operação $\star$ é associativa.

    Tome $e = 0$. Para todo $x \in \real$ temos
    \[
        x\star e = x \star 0 = \sqrt[3]{x^3 + 0^3} = \sqrt[3]{x^3} = x
    \]
    e assim $e = 0$ é o elemento neutro da operação $\star$.

    Finalmente, dado $x \in \real$, tome $y = -x \in \real$. Daí
    \[
        x \star y = x \star (-x) = \sqrt[3]{x^3 + (-x)^3} = \sqrt[3]{x^3 - x^3} = 0
    \]
    e então $-x$ é o oposto de $x$ na operação $\star$.

    Portanto, $(\real, \star)$ é um grupo e esse grupo é comutativo.
    \vspace{1cm}

    \questao Considere o seguinte conjunto
    \[
        G = \left\{\begin{pmatrix} 1 & a & b\\ 0 & 1 & c \\ 0 & 0 & 1\end{pmatrix} \mid a, b, c \in \real\right\}.
    \]
    Mostre que $(G, \cdot)$, onde $\cdot$ é a multiplicação de matrizes, é um grupo. Esse grupo é comutativo? Justifique.

    \noindent\solucao Primeiro, observe que tomando
    \[
        A = \begin{pmatrix}1 & 2 & 3\\0 & 1 & 2\\0 & 0 & 1\end{pmatrix},\\
        B = \begin{pmatrix}1 & 3 & 2\\0 & 1 & 1\\0 & 0 & 1\end{pmatrix}\\
    \]
    temos
    \begin{align*}
        AB &= \begin{pmatrix}1 & 2 & 3\\0 & 1 & 2\\0 & 0 & 1\end{pmatrix}\begin{pmatrix}1 & 3 & 2\\0 & 1 & 1\\0 & 0 & 1\end{pmatrix} = \begin{pmatrix}1 & 5 & 7\\0 & 1 & 3\\0 & 0 & 1\end{pmatrix}\\
        BA &= \begin{pmatrix}1 & 3 & 2\\0 & 1 & 1\\0 & 0 & 1\end{pmatrix}\begin{pmatrix}1 & 2 & 3\\0 & 1 & 2\\0 & 0 & 1\end{pmatrix} = \begin{pmatrix}1 & 5 & 11\\0 & 1 & 3\\0 & 0 & 1\end{pmatrix}.
    \end{align*}
    Com isso $AB \ne BA$ e a operação de multiplicação de matrizes em $G$ não é comutativa.

    Agora sejam $A$, $B$, $C \in G$ com
    \[
        A = \begin{pmatrix} 1 & a & b\\0 & 1 & c\\0 & 0 & 1\end{pmatrix},
        B = \begin{pmatrix} 1 & d & e\\0 & 1 & f\\0 & 0 & 1\end{pmatrix},
        C = \begin{pmatrix} 1 & g & h\\0 & 1 & j\\0 & 0 & 1\end{pmatrix}.
    \]
    Temos
    \begin{align*}
        [AB]C &= \left[\begin{pmatrix} 1 & a & b\\0 & 1 & c\\0 & 0 & 1\end{pmatrix}\begin{pmatrix} 1 & d & e\\0 & 1 & f\\0 & 0 & 1\end{pmatrix}\right]\begin{pmatrix} 1 & g & h\\0 & 1 & j\\0 & 0 & 1\end{pmatrix} \\ &= \begin{pmatrix} 1 & d + a & e + af + b\\0 & 1 & c + f\\0 & 0 & 1\end{pmatrix}\begin{pmatrix} 1 & g & h\\0 & 1 & j\\0 & 0 & 1\end{pmatrix} \\ &= \begin{pmatrix} 1 & d + a + g & h + (d + a)j + e + af + b\\0 & 1 & c + f + j\\0 & 0 & 1\end{pmatrix}\\
        A[BC] &= \begin{pmatrix} 1 & a & b\\0 & 1 & c\\0 & 0 & 1\end{pmatrix}\left[\begin{pmatrix} 1 & d & e\\0 & 1 & f\\0 & 0 & 1\end{pmatrix}\begin{pmatrix} 1 & g & h\\0 & 1 & j\\0 & 0 & 1\end{pmatrix}\right] \\ &=\begin{pmatrix} 1 & a & b\\0 & 1 & c\\0 & 0 & 1\end{pmatrix}\begin{pmatrix} 1 & dg & h + dj + e\\0 & 1 & fj\\0 & 0 & 1\end{pmatrix} \\ &= \begin{pmatrix} 1 & d + g + a & h + (d + a)j + e + af + b\\0 & 1 & c + f + j\\0 & 0 & 1\end{pmatrix}
    \end{align*}
    logo $(AB)C = A(BC)$ e com isso a operação é associativa.

    Seja
    \[
        I = \begin{pmatrix} 1 & 0 & 0\\0 & 1 & 0\\0 & 0 & 1\end{pmatrix} \in G.
    \]
    Para toda $A \in G$,
    \[
        A = \begin{pmatrix} 1 & a & b\\0 & 1 & c\\0 & 0 & 1\end{pmatrix}
    \]
    temos
    \begin{align*}
        AI &= \begin{pmatrix} 1 & a & b\\0 & 1 & c\\0 & 0 & 1\end{pmatrix} \begin{pmatrix} 1 & 0 & 0\\0 & 1 & 0\\0 & 0 & 1\end{pmatrix} = \begin{pmatrix} 1 & a & b\\0 & 1 & c\\0 & 0 & 1\end{pmatrix} = A\\
        IA &= \begin{pmatrix} 1 & 0 & 0\\0 & 1 & 0\\0 & 0 & 1\end{pmatrix}\begin{pmatrix} 1 & a & b\\0 & 1 & c\\0 & 0 & 1\end{pmatrix} = \begin{pmatrix} 1 & a & b\\0 & 1 & c\\0 & 0 & 1\end{pmatrix} = A,
    \end{align*}
    logo $I$ é o elemento neutro de $G$.

    Finalmente, dada $A \in G$,
    \[
        A = \begin{pmatrix} 1 & a & b\\0 & 1 & c\\0 & 0 & 1\end{pmatrix} \in G
    \]
    tome
    \[
        B = \begin{pmatrix} 1 & -a & -b + ac\\0 & 1 & -c\\0 & 0 & 1\end{pmatrix}.
    \]
    Temos
    \begin{align*}
        AB &= \begin{pmatrix} 1 & a & b\\0 & 1 & c\\0 & 0 & 1\end{pmatrix} \begin{pmatrix} 1 & -a & -b + ac\\0 & 1 & -c\\0 & 0 & 1\end{pmatrix} = \begin{pmatrix} 1 & 0 & 0\\0 & 1 & 0\\0 & 0 & 1\end{pmatrix} = I\\
        BA &= \begin{pmatrix} 1 & -a & -b + ac\\0 & 1 & -c\\0 & 0 & 1\end{pmatrix}\begin{pmatrix} 1 & a & b\\0 & 1 & c\\0 & 0 & 1\end{pmatrix} = \begin{pmatrix} 1 & 0 & 0\\0 & 1 & 0\\0 & 0 & 1\end{pmatrix} = I
    \end{align*}
    com isso $B$ é o inverso de $A$.

    Portanto $G$ é um grupo e esse grupo não é comutativo.
\end{document}
