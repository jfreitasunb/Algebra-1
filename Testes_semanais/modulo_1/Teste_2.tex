%!TEX program = xelatex
%!TEX encoding = UTF-8
%Definições de Ano, Semestre, número e data da avaliação
\def\numeroteste{2}
\def\modulo{1}
\def\dataavaliacao{12/04/2024}

\documentclass[12pt]{exam}

\usepackage{caption}
\usepackage{amssymb}
\usepackage{amsmath,amsfonts,amsthm,amstext}
\usepackage[brazil]{babel}
% \usepackage[latin1]{inputenc}
\usepackage{graphicx}
\graphicspath{{/ArquivosLinux/OneDrive/imagens-latex/}{D:/OneDrive - unb.br/imagens-latex/}}
\usepackage{enumitem}
\usepackage{multicol}
\usepackage{answers}
\usepackage{tikz,ifthen}
\usetikzlibrary{lindenmayersystems}
\usetikzlibrary[shadings]
\def\ano{2024}
\def\semestre{1}
\def\disciplina{Álgebra 1}
\def\nomeabreviado{Álgebra 1}
\def\turma{1}

\newcommand{\im}{{\rm Im\,}}
\newcommand{\dlim}[2]{\displaystyle\lim_{#1\rightarrow #2}}
\newcommand{\minf}{+\infty}
\newcommand{\ninf}{-\infty}
\newcommand{\cp}[1]{\mathbb{#1}}
\newcommand{\sub}{\subseteq}
\newcommand{\n}{\mathbb{N}}
\newcommand{\z}{\mathbb{Z}}
\newcommand{\rac}{\mathbb{Q}}
\newcommand{\real}{\mathbb{R}}
\newcommand{\complex}{\mathbb{C}}

\newcommand{\vesp}[1]{\vspace{ #1  cm}}

\newcommand{\compcent}[1]{\vcenter{\hbox{$#1\circ$}}}
\newcommand{\comp}{\mathbin{\mathchoice
        {\compcent\scriptstyle}{\compcent\scriptstyle}
        {\compcent\scriptscriptstyle}{\compcent\scriptscriptstyle}}}
\renewcommand{\sin}{{\rm sen\,}}
\renewcommand{\tan}{{\rm tg\,}}
\renewcommand{\csc}{{\rm cossec\,}}
\renewcommand{\cot}{{\rm cotg\,}}
\renewcommand{\sinh}{{\rm senh\,}}
\newtheorem{definicao}{Definição}[section]
\newtheorem{definicoes}{Definições}[section]
\newtheorem{exemplo}{Exemplo}[section]
\newtheorem{exemplos}{Exemplos}[section]
\newtheorem{exercicio}{Exercício}
\newtheorem{observacao}{Observação:}[section]
\newtheorem{observacoes}{Observações:}[section]
\newtheorem*{solucao}{Solução:}
\newtheorem{proposicao}{Proposição}
\newtheorem{lema}{Lema}
\newtheorem{teorema}{Teorema}
\newtheorem{corolario}{Corolário}
\newenvironment{prova}[1][Prova]{\noindent\textbf{#1:} }{\qedsymbol}%{\ \rule{0.5em}{0.5em}}

\begin{document}

    \begin{figure}[h]
        \begin{minipage}[c]{1.7cm}
            \includegraphics[width=1.7cm]{unb.pdf}
        \end{minipage}
        \hspace{0pt}
        \begin{minipage}[c]{4in}
            {Universidade de Brasília} \\
            {Departamento de Matemática}
        \end{minipage}
    \end{figure}
    \vesp{-1}
    \hrule
    \begin{center}
        {\Large\bf \disciplina\ - Turma \turma}  \\
        \vesp{0.2} {\large\bf Teste \numeroteste\ - Módulo\ \modulo\ -\ \dataavaliacao}
    \end{center}

    \noindent{\bf \underline{Nome: \hspace{9.8cm} Mat.:\qquad \
            \hspace{2.1cm}}}\\
    \vspace*{.01cm}

    \noindent{\bf \underline{Nome: \hspace{9.8cm} Mat.:\qquad \
            \hspace{2.1cm}}}
    \vspace{.4cm}

    \questao{} Determine se a afirmação dada é verdadeira ou falsa. Caso seja verdadeira, demonstre-a. Se for falsa, apresente um exemplo mostrando que a afirmação é falsa.
    \begin{center}
        Se $A \cap B = A \cap C$, então $B = C$.
    \end{center}
    \solucao Essa afirmação é \textbf{falsa}, pois tomando $A = \emptyset$, $B = \{1, 2\}$ e $C = \{1, 3\}$ temos
    \[
        A \cap B = \emptyset = A \cap C
    \]
    e no entanto $B \ne C$.

    \vspace*{.75cm}

    \questao{} Sejam $A$, $B$ e $C$ conjuntos. Prove que:
    \[
        A - (B \cap C) = (A - B) \cup (A - C)
    \]
    \solucao Precisamos mostrar que
    \begin{enumerate}[label={\roman*})]
        \item $A - (B \cap C) \sub (A - B) \cup (A - C)$
        \item $(A - B) \cup (A - C) \sub A - (B \cap C)$
    \end{enumerate}

    Para mostrar (i) seja $x \in A - (B \cap C)$. Assim $x \in A$ e $x \notin B \cap C$. Para que $x \notin B \cap C$, devemos ter $x \notin B$ ou $x \notin C$.

    Se $x \notin B$, então como $x \in A$, temos $x \in A - B$. Logo $x \in (A - B) \cup (A - C)$.

    Se $x \notin C$, então como $x \in A$, temos $x \in A - C$. Logo $x \in (A - B) \cup (A - C)$.

    Independente do, caso sempre temos  $x \in (A - B) \cup (A - B)$. Logo $A - (B \cap C) \sub (A - B) \cup (A - C)$.

    Agora, para provar (ii) seja $y \in (A - B) \cup (A - C)$. Assim $y \in A - B$ ou $y \in A - C$.

    Se $y \in A - B$, então $y \in A$ e $y \notin B$. Como $y \notin B$, então $y \notin (B \cap C)$. Mas $y \in A$, daí $y \in A - (B \cap C)$.

    Se $y \in A - C$, então $y \in A$ e $y \notin C$. Como $y \notin C$, então $y \notin (B \cap C)$. Mas $y \in A$, daí $y \in A - (B \cap C)$.

    Independente do caso, sempre temos $y \in A - (B \cap C)$. Logo $(A - B) \cup (A - C) \sub A - (B \cap C)$.

    Portanto de (i) e (ii) temos
    \[
        A \cap B = \emptyset = A \cap C.
    \]

\end{document}
