%!TEX program = xelatex
%!TEX encoding = ISO-8859-1
\documentclass[12pt]{article}
%\usepackage[leqno]{amsmath}
%\usepackage{makeidx,graphics}
\usepackage{graphicx}
\graphicspath{{/Users/jfreitas/Dropbox/imagens-latex/}{/home/jfreitas/Dropbox/imagens-latex/}}
\usepackage{amssymb,amsmath,amsfonts,amsthm,amstext}
%\usepackage{color}
%\usepackage[latin1]{inputenc}
\usepackage[portuges]{babel}
\usepackage{enumerate}
\usepackage{url}


\newcommand{\real}{\mathbb{R}}

%\DeclareGraphicsRule{jpg}{*[}{}{`jpeg2eps #1.jpg}
%\input{seteps}
%\input{setbmp-dvips}

\setlength{\topmargin}{-1.0in}
\setlength{\oddsidemargin}{0in}
\setlength{\textheight}{10.1in}
\setlength{\textwidth}{6.5in}
\setlength{\baselineskip}{12mm}

\begin{document}
\pagestyle{empty}

\begin{figure}[h]
    \begin{minipage}[c]{1.7cm}
    \includegraphics[width=1.7cm]{unb.pdf}
    \end{minipage}%
    \hspace{0pt}
    \begin{minipage}[c]{4in}
    {Universidade de Bras{\'\i}lia} \\
    {Departamento de Matem{\'a}tica}
    \end{minipage}
\end{figure}
\vspace{-0.9cm}
\hrule

\begin{center}
{\large\bf Plano de Ensino -- 2$^{o}$/2015} \\
{\large\bf \'Algebra 1 -- Turma B}\\
Prof. Jos{\'e} Ant{\^o}nio O. Freitas
\end{center}
\hrule
\vspace{0.25cm}
\noindent {\bf{PROGRAMA:}} 17 semanas dividas em 3 m\'{o}dulos.
\begin{enumerate}[1)]

\item No\c{c}\~oes B\'asicas: Opera\c{c}\~oes l\'ogicas elementares; Conjuntos; Produtos Cartesianos Finitos; Rela\c{c}\~oes: de ordem e de equival\^encia, conjunto quociente; fun\c{c}\~oes: injetoras, sobrejetoras, bijetoras; 

\item Estruturas \'algebricas: grupos, grupos abelianos, subgrupos, rela\c{c}\~oes de equival\^encia m\'odulo um subgrupo e o Teorema de Lagrange, Grupos C{\'\i}clicos, Grupos de Permuta\c{c}\~oes; An\'eis, Suban\'eis. Ideais, Dom{\'\i}nio de Integridade, An\'eis de divis\~ao; Corpos de fra\c{c}\~oes de um dom{\'\i}nio de integridade

\item Polin\^omios sobre dom{\'\i}nios de integridade: forma e fun\c{c}\~ao polinomial, o algor{\'\i}tmo de Euclides para polin\^omios sobre um corpo, polin\^omios irredut{\'\i}veis. Outros t\'opicos: constru\c{c}\~ao dos n\'umeros reais (por sequ\^encias de Cauchy), caracteriza\c{c}\~ao do corpo dos n\'umeros reais. Estruturas alg\'ebricas simples.
\end{enumerate}

\vspace{0.5cm}
\noindent {\bf{BIBLIOGRAFIA:}}
\begin{itemize}

\item S. Shokranian: {\it {\'A}lgebra 1}, Ci{\^e}ncia Moderna, 2010.

\item H. H. Domingues, G. Iezzi: {\it {\'A}lgebra Moderna}, $2^a$
  Ed., Atual, 1982.

\item Adilson Gon{\c c}alves: {\it Introdu{\c c}{\~a}o {\`a} {\'A}lgebra}, $5^a$ Ed., IMPA,
  2003.

\item G. Birkhoff, S. MacLane: {\it {\'A}lgebra Moderna B{\'a}sica}, $4^a$ Ed.,
  Guanabara Dois, 1980.

\end{itemize}

\noindent {\bf{SISTEMA DE AVALIA\c{C}\~{A}O:}} O curso \'{e} dividido em tr\^{e}s
m\'{o}dulos.  Em cada um dos módulos o aluno receberá uma nota $M_i$, $i = 1$, 2, 3 dada por
\[
    M_i = 15\%T_i + 85\%P_i, \quad 0 \le M_i \le 10
\]
onde $T_i$  é a média aritmética dos testes em sala e $P_i$ é a nota da prova. A partir das notas dos módulos, a nota final ($NF$) de cada estudante é dada por:
\[
    NF = \dfrac{2M_1 + 3M_2 + 4M_3}{9}, \quad 0 \le NF \le 10.
\]
Será considerado aprovado o estudante que obtiver $NF$ maior ou igual a 5.

\vspace{0.5cm}
\noindent\textit{Teste em sala}: estes testes serão marcados com pelo menos uma semana de antecedência pelo professor e serão feitos nos 40 minutos finais da aula. Os teste serão feitos em grupos  e recolhidos para correção de parte das questões. Uma nota entre 0 e 10 será atribuída ao grupo.

\vspace{0.5cm}

\noindent\textit{Provas:} as provas serão realizadas no horário da aula e as datas estão listadas abaixo e podem, a critério do professor, ser mudadas.

\begin{center}
    \begin{tabular}{c|c|c}
        \hline\hline
        \hspace{1cm}{\bf Prova}\hspace{1cm} & \hspace{3cm}{\bf Data}\hspace{3cm} & \hspace{1.7cm}{\bf Hor\'{a}rio}\hspace{1.7cm} \\
        \hline\hline
        $P_1$ & 25/09/15 (sexta-feira) \phantom{x} & 19:00 - 20:40 \\
        \hline
        $P_2$ & 30/10/15 (sexta-feira) \phantom{x} & 19:00 - 20:40 \\
        \hline
        $P_3$ & 04/12/15 (sexta-feira) \phantom{x} & 19:00 - 20:40 \\
        \hline\hline
        %$PF$ &  11/07/11 (segunda-fei) \phantom{x} & 8:00 - 9:50 \\
        %\hline\hline
    \end{tabular}
\end{center}

\vspace{0.5cm}
{\bf \noindent Men\c{c}\~{a}o Final:} ser\'{a} obtida da $NF$ de
acordo com as normas da UnB.
\begin{center}
    \begin{tabular}{c|c}
        \hline\hline
        \hspace{1cm}{Nota}\hspace{1cm} & \hspace{0.25cm}{Men\c{c}\~{a}o}\hspace{0.25cm}\\
        \hline\hline
        9,00 a 10,0 & SS \\
        \hline
        7,00 a 8,99 & MS \\
        \hline
        5,00 a 6,99 & MM \\
        \hline
        3,00 a 4,99 & MI \\
        \hline
        0,00 a 2,99  & II \\
        \hline\hline
    \end{tabular}
\end{center}
Receber{\'a} a men{\c c}{\~a}o {\bf SR} quem estiver reprovado por faltar mais de 25\%
das aulas.

\noindent {\bf{P\'{A}GINA DA TURMA:}} Todos estudantes devem
obrigatoriamente se cadastrar no MOODLE para ter acesso a listas de exerc{\'\i}cios, comunica\c{c}\~oes oficiais da disciplina, notas de aula entre outras informa\c{c}\~oes. Para tal o(a) aluno(a) deve acessar o endere\c{c}o
\begin{center}
    \url{e-mural.mat.unb.br/cadastro.php}
\end{center}
 para efetuar o cadastro. Nesta p\'agina ser\'a solicitada a matr{\'\i}cula, e-mail e defini\c{c}\~ao da senha de acesso. Para validar o cadastro o(a) aluno(a) dever\'a informar uma disciplina e turma do MAT na qual est\'a regularmente matriculado. Ap\'os a efetiva\c{c}\~ao do cadastro, dever\'a acessar o endere\c{c}o 
 \begin{center}
     \url{moodle.mat.unb.br}
 \end{center}
  para ter acesso \`a p\'agina da disciplina e a seu conte\'udo.


\begin{itemize}
\item Toda a comunica\c{c}\~{a}o oficial do curso, inclusive a divulga\c{c}\~{a}o de
notas e gabaritos, se dar\'{a} atrav\'{e}s do {\em F\'{o}rum de Not\'{\i}cias} do
MOODLE.\vspace{-0.20cm}
\item No {\em F\'{o}rum de Debates} do MOODLE poder\~{a}o ser
postadas d\'{u}vidas que ser\~{a}o respondidas on-line pelos seus
colegas ou pelo monitor dessa turma.
\end{itemize}

\noindent {\bf{OBSERVA\c{C}\~{O}ES IMPORTANTES:}}

\begin{itemize}
\item[1)] As provas ser\~{a}o individuais e sem qualquer tipo de
aux\'{\i}lio (calculadora, livros etc.), sendo vedado o empr\'{e}stimo de
qualquer material entre os alunos durante as avalia\c{c}\~{o}es. As
tentativas de fraude ser\~{a}o reprimidas com m\'{a}ximo rigor.
\vspace{-0.25cm}

\item[2)] \'{E} vedado o uso de telefones celulares e quaisquer dispositivos eletr\^{o}nicos pessoais durante a realiza\c{c}\~{a}o das atividades do curso em sala de aula. \vspace{-0.25cm}

\item[3)] Ser\'{a} exigido documento de identifica\c{c}\~{a}o dos estudantes nos
dias de provas e testes. \vspace{-0.25cm}

\item[4)] A aus\^{e}ncia acarretar\'{a} nota zero em qualquer uma das
avalia\c{c}\~{o}es. \vspace{-0.25cm}

\item[5)] A crit\'{e}rio do professor, as datas das provas poder\~{a}o
ser alteradas. \vspace{-0.25cm}

\item[6)] A lista de presen\c{c}a ser\'{a} passada apenas uma vez
durante cada aula e est\'{a} sujeita a confirma\c{c}\~{a}o oral. O
estudante deve assin\'{a}-la com sua rubrica. {\'E} proibido assinar
com suas iniciais e \'{e} proibido assinar por outra pessoa.
\vspace{-0.25cm}

\item[7)] Haver{\'a} avalia{\c c}{\~a}o quanto {\`a} clareza, apresenta{\c
c}{\~a}o e formaliza{\c c}{\~a}o na  resolu{\c c}{\~a}o das quest{\~o}es de
cada prova. A nota do aluno poder{\'a} ser alterada em raz{\~a}o da
inobserv{\^a}ncia desses par{\^a}metros.

\item[8)] A comunica\c{c}\~ao entre o professor/monitores e estudantes ser\'a, preferencialmente,
 estabelecida pelo f\'orum do MOODLE.
\end{itemize}

\vfill
\hrule
\end{document}
