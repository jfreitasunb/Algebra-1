%!TEX program = xelatex
%!TEX root = Algebra_1.tex
\chapter{Fun{\c c}{\~o}es}

\begin{definicao}
Uma \textbf{fun{\c c}{\~a}o} $f$ de um conjunto $A$ em um conjunto $B$ {\'e} uma rela{\c c}{\~a}o de $A$ em $B$ satisfazendo:
	\begin{enumerate}
		\item Para todo $x \in A$, existe $y \in B$ tal que $f(x) = y$.
		\item  Se $x \in A$ é tal que $f(x) = y_{1}$ e $f(x) = y_{2}$ com $y_1$, $y_2 \in B$, então $y_{1} = y_{2}$.
	\end{enumerate}
Nesse caso $y$ é chamado de \textbf{imagem} de $x$ segundo $f$.
\end{definicao}

O conjunto $A$ {\'e} chamado de \textbf{dom{\'\i}nio} de $f$ e será denotado por $\dom(f)$.O conjunto $B$ {\'e} chamado de \textbf{contra-dom{\'\i}nio} de $f$. O conjunto
\[
	\im(f) = \{f(x) \mid x \in A\} \sub B
\]
é chamado \textbf{imagem} de $f$.

\begin{exemplos}
	\begin{enumerate}[label={\arabic*})]
	\item Sejam $A = \{0,1,2,3\}$ e $B = \{4,5,6,7,8\}$. Quais das seguintes rela{\c c}{\~o}es s{\~a}o fun{\c c}{\~o}es?
	\begin{enumerate}[label={\alph*})]
		\item $R_{1} = \{(0,5),(1,6),(2,7)\}$
		\item $R_{2} = \{(0,4),(1,5),(1,6),(2,7),(3,8)\}$
		\item $R_{3} = \{(0,4),(1,5),(2,7),(3,8)\}$
		\item $R_{4} = \{(0,5),(1,5),(2,6),(3,7)\}$
	\end{enumerate}
	\begin{solucao}
		\begin{enumerate}[label={\alph*})]
			\item Não é função pois $3 \in A$ e $3$ não está associado {\`a} nenhum elemento de $B$.
			\item Não é função pois $1 \in A$ está associado a dois elementos diferentes em $B$.
			\item É uma função.
			\item É uma função.
		\end{enumerate}
	\end{solucao}
	\item $R_{5} = \{(x,y) \in \real  \times \real  \mid y^2 = x^2\}$
	\begin{solucao}
	 	Não é função pois, por exemplo, para $x = 1$ temos $y = -1$ ou $y = 1$.
	 \end{solucao}
	\item $R_{6} = \{(x,y) \in \real  \times \real  \mid x^2 + y^2 = 1\}$.
	\begin{solucao}
	 	Não é função pois, por exemplo, para $x = 0$ temos $y = 1$ ou $y = -1$.
	 \end{solucao}
	\item  $R_{7} = \{(x,y) \in \real  \times \real \mid y = x^2\}$
	\begin{solucao}
	 	É uma função.
	\end{solucao}
	\end{enumerate}	
\end{exemplos}


\begin{definicao}
	Seja $f : A \to B$ uma função.
	\begin{enumerate}
		\item Dizemos que $f$ é \textbf{injetora} se dados $x_1$, $x_2 \in A$ tais que $f(x_1) = f(x_2)$, então $x_1 = x_2$. De modo equivalente, dizemos que $f$ e \textbf{injerota} se dados $x_1$, $x_2 \in A$ tais que $x_1 \ne x_2$, então $f(x_1) \ne f(x_2)$.
		\item Dizemos que $f$ é \textbf{sobrejetora} se para todo $y \in B$, existe $x \in A$ tal que $f(x) = y$.
		\item Dizemos que $f$ e \textbf{bijetora} se $f$ for \textbf{injetora} e \textbf{sobrejtora} simultaneamente.
	\end{enumerate}
\end{definicao}

\begin{exemplos}
	Verifique se as seguintes funções são injetoras ou sobrejetoras:
	\begin{enumerate}
		\item $f : \z \to \z$ dada por $f(x) = 3x + 1$
		\begin{solucao}
			De fato, dados $x_1$, $x_2 \in \z$ tais que $f(x_1) = f(x_2)$ temos
			\begin{align*}
				f(x_1) &= f(x_2)\\
				3x_1 + 1 &= 3x_2 + 1\\
				3x_1 &= 3x_2\\
				3(x_1 - x_2) &= 0.
			\end{align*}
			Assim $x_1 - x_2 = 0$, isto é, $x_1 = x_2$. Logo $f$ é injetora.

			Para determinar se $f$ é sobrejetora seja $y \in \z$. Precisamos determinar se é possível encontrar algum $x \in \z$ tal que $f(x) = y$. Ou seja, precisamos saber se a equação $3x + 1 = y$ tem solução em $\z$ para qualquer valor de $y$.

			Se tomarmos $y = 2$ temos
			\begin{align*}
				3x + 1 &= 2\\
				3x = 1
			\end{align*}
			e essa última equação não possui solução em $\z$. Logo para $y = 2$ \textbf{não} existe $x \in \z$ de modo que $f(x) = 2$. Logo $f$ não é sobrejetora.
		\end{solucao}

		\item $g : \rac \to \rac$ dada por $f(x) = 3x + 1$
		\begin{solucao}
			A prova que $g$ é injetora é idêntica ao caso anterior.

			Para determinar se $g$ é sobrejetora seja $y \in \rac$. Precisamos determinar se é possível encontrar algum $x \in \rac$ tal que $g(x) = y$. Ou seja, precisamos saber se a equação $3x + 1 = y$ tem solução em $\rac$ para qualquer valor de $y$. Mas
			\begin{align*}
				&3x + 1 = y\\
				&3x = y - 1\\
				&x = \dfrac{y - 1}{3} \in \rac
			\end{align*}
			para qualquer valor de $y \in \rac$. Assim dado $y \in \rac$ tome $x = (y - 1)/3 \in \rac$. Daí
			\[
				g(x) = g\left(\dfrac{y - 1}{3}\right) = 3\left(\dfrac{y - 1}{3}\right) + 1 = y - 1 + 1 = y. 
			\]
			Logo $g$ é sobrejetora.
		\end{solucao}
		
		\item A função $h :\real \to \real$ dada por $h(x) = x^2$
		\begin{solucao}
			A função $h$ não é injetora pois, por exemplo, $h(-1) = 1 = h(1)$ e $1\neq -1$.

			A função $h$ não é sobrejetora pois, por exemplo, para $y = -1$ \textbf{não} existe $x\in\mathbb{R}$ tal que $h(x) = -1$.
		\end{solucao}
		\end{enumerate}
\end{exemplos}

\begin{definicao}
	Sejam $f : A \to B$ e $g : B \to C$ funções. Definimos a \textbf{função composta} de $g$ com $f$ como sendo a função denotada por $g \circ f : A \to C$ tal que $(g\circ f)(x) = g(f(x))$ para todo $x \in A$.
\end{definicao}

\begin{exemplos}
	\begin{enumerate}
		\item Sejam $f : \real \to \real$ e $g : \real \to \real$ dadas por $f(x) = x^2$ e $g(x) = x + 1$. Assim podemos definir $g \circ f$ e $f \circ g$ e
		\begin{align*}
			(g\circ f)(x) = g(f(x)) = g(x^2) = x^2 + 1\\
			(f\circ g)(x) = f(g(x)) = f(x + 1) = (x + 1)^2
		\end{align*}
		Assim em geral $f \circ g \ne g \circ f$.

		\item $f : \real_- \to \real^*_+$ e $g : \real^*_+ \to \real$ dadas por $f(x) = x^2 + 1$ e $g(x) = \ln x$. Nesse caso só podemos definir $g \circ f : \real_- \to \real$ e
		\[
			(g\circ f)(x) = g(f(x)) = g(x^2 + 1) = \ln(x^2 + 1).
		\]
	\end{enumerate}
\end{exemplos}

\begin{proposicao}
	Se $f : A \to B$ e $g : B \to C$ s{\~a}o fun{\c c}{\~o}es injetoras, ent{\~a}o $g\circ f$ {\'e} injetora.
\end{proposicao}
\begin{prova}
	Dados $x_{1}$, $x_{2} \in A$ tais que $(g\circ f)(x_{1}) = (g\circ f)(x_{2})$ queremos mostrar que $x_1 = x_2$. Temos:
	\begin{align*}
		(g\circ f)(x_{1}) &= (g\circ f)(x_{2})\\
		g(f(x_1)) &= g(f(x_2)).
	\end{align*}
	Como por hipótese $g$ é injetora, dessa última igualdade segue que $f(x_1) = f(x_2)$. Mas $f$ também é injetora, por hipótese, daí $x_1 = x_2$, como queríamos. Portanto $g\circ f$ é injetora.
\end{prova}

\begin{proposicao}
	Se $f : A \to B$ e $g : B \to C$ s{\~a}o funções sobrejetoras, ent{\~a}o $g\circ f$ {\'e} sobrejetora.
\end{proposicao}
\begin{prova}
 	Para mostrar que $g \circ f$ é sobrejetora, precisamos mostrar que para todo $y \in C$, existe $x \in A$ tal que $(g\circ f)(x) = y$.

 	Assim seja $y \in C$. Como $g : B \to C$ é sobrejetora, existe $z \in A$ tal que $g(z) = y$. Mas $z \in B$ e $f : A \to B$ é sobrejetora e assim existe $x \in A$ tal que $f(x) = z$. Logo
 	\[
 		(g\circ f)(x) = g(f(x)) = g(z) = y.
 	\]
 	Portanto $g \circ f$ é sobrejetora.
 \end{prova}

Dado $f:A \to B$ uma funç{\~a}o, considere a relaç{\~a}o $f^{-1}\subseteq B$x$A$ tal que $(b,a)\in f^{-1}$ se $(a,b)\in f$, ou seja, $f^{-1}(b)=a$ se $f(a)=b$.

Pode ocorrer que $f^{-1}$ n{\~a}o seja funç{\~a}o, mesmo $f$ sendo uma funç{\~a}o. Por exemplo:

$f:\{0,1,2,3\} \to\{4,5,6,7,8\}$ dada por:\\
$f(0)=5$\\
$f(1)=5$\\
$f(2)=6$\\
$f(3)=7$

Neste caso, $f^{-1}$ {\'e} dado por:\\
$f^{-1}(5)=0$\\
$f^{-1}(5)=1$\\
$f^{-1}(6)=2$\\
$f^{-1}(7)=3$

\begin{teorema}
Dada $f:A\rightarrow B$ fun{\c c}{\~a}o tome $f^{-1}:B\rightarrow A$. Definida com o $f^{-1}(b)=a$ se $f(a)=b$. Ent{\~a}o $f^{-1}$ {\'e} uma fun{\c c}{\~a}o se, e somente se, $f$ {\'e} bijetora.
\end{teorema}

\textbf{Demonstra{\c c}{\~a}o}: Suponha $f^{-1}$ {\'e} fun{\c c}{\~a}o. Precisamos provar que $f$ {\'e} injetora e sobrejetora.

Dados $a_{1},a_{2}\in A$ tais que $f(a_{1})=b=f(a_{2})$. Como $f(a_{1})=b$ temos $f^{-1}(b)=a_{1}$, al{\'e}m disso, $f^{-1}(b)=a_{2}$. Mas $f^{-1}$ {\'e} fun{\c c}{\~a}o, da{\'\i} $a_{1}=a_{2}$, ou seja, $f$ {\'e} injetora.

Dado $b\in B$, como $f^{-1}$ {\'e} uma fun{\c c}{\~a}o, $\forall b\in B, f^{-1}(b)=a\in A$, logo $f(a)=b$ e assim $f$ {\'e} sobrejetora.

Portanto $f$ {\'e} bijetora.

Agora suponha que $f$ {\'e} bijetora.

Primeiramente, dado $b\in B$, como $f$ {\'e} sobrejetora, existe $a\in A$ tal que $f(a)=b$, ou seja, $f^{-1}(b)=a\in A$.

Suponha que $f^{-1}(b)=a_{1}$ e $f^{-1}(b)=a_{2}$. Da{\'\i}, $f(a_{1})=b\wedge f(a_{2})=b$. Mas $f$ {\'e} injetora, assim $a_{1}=a_{2}$ e ent{\~a}o $f^{-1}(b)=a_{1}=a_{2}$.

Portanto $f^{-1}$ {\'e} fun{\c c}{\~a}o. \#



\subsection{Fun{\c c}{\~a}o Identidade}
\subsubsection{Defini{\c c}{\~a}o}
\begin{definicao}[Fun{\c c}{\~a}o Identidade] Dado um conjunto $A\neq\emptyset$, a fun{\c c}{\~a}o $i_{A}:A\rightarrow A$ dada por $i_{A}(x)=(x)$ {\'e} chamada de fun{\c c}{\~a}o identidade.\end{definicao}

\begin{proposicao}
Se $f:A\rightarrow B$ {\'e} bijetora, ent{\~a}o $f\circ f^{-1}=i_{B}\wedge f^{-1}\circ f=i_{A}$.
\end{proposicao}

\textbf{Demonstra{\c c}{\~a}o}: Temos $i_{F}:F\rightarrow F$ e $i_{E}:E\rightarrow E$. Al{\'e}m disso, $f\circ f^{-1}:F\rightarrow F$ e $f^{-1}\circ f:E\rightarrow E$, da{\'\i} $D(f\circ f^{-1})=D(i_{F})$\footnote{$D(f(x))$ {\'e} o dom{\'\i}nio da fun{\c c}{\~a}o $f$} e $D(f^{-1}\circ f)=D(i_{E})$. Dado $x\in F, (f\circ f^{-1})(x)=f(f^{-1}(x))=x=i_{F}(x)$. Dado $x\in E, (f^{-1}\circ f)(x)=f^{-1}(f(x))=x=i_{E}(x)$.\#

\subsubsection{Propriedades}
\begin{proposicao} Se $f:A\rightarrow B$ e $g:B\rightarrow A$ s{\~a}o fun{\c c}{\~o}es, ent{\~a}o:
\begin{enumerate}
\item $f\circ i_{A}=f, i_{B}\circ f=f, g\circ i_{B}=g, i_{E}\circ g=g$
\item Se $g\circ f=i_{A}$, e $f\circ g=i_{B}$, ent{\~a}o $f$ e $g$ s{\~a}o bijetoras e $g=f^{-1}$
\end{enumerate}
\end{proposicao}

\textbf{Demonstra{\c c}{\~a}o}:
\begin{enumerate}
\item Provemos que $f\circ i_{A}=f$.\\
Primeiro temos $f:A\rightarrow B$ e $i_{A}:A\rightarrow A$. Da{\'\i}, $f\circ i_{A}:A\rightarrow B$, ou seja, $D(f\circ i_{A})=D(f)$. Dado $x\in A$, temos $(f\circ i_{A})(x)=f(i_{A}(x))=f(x)$. Portanto, $f\circ i_{A}=f$.
\item Provemos que $f$ {\'e} bijetora.\\
Dados $x_{1}$, $x_{2}\in B$ tais que $f(x_{1})=f(x_{2})$. Como $f:A\rightarrow B$ e $g:B\rightarrow A$, ent{\~a}o $g(f(x_{1}))=g(f(x_{2}))$, ou seja, $(g\circ f)(x_{1})=(g\circ f)(x_{2})$. Da{\'\i}, $i_{A}(x_{1})=i_{A}(x_{2})$. Logo, $x_{1}=x_{2}$, isto {\'e}, $f$ {\'e} injetora.

Agora, dado $y\in B$, segue que $y=i_{B}(y)$. Mas $i_{B}=f\circ g$. Da{\'\i}, $y=i_{B}(y)=(f\circ g)(y)=f(g(y))$. Assim, $x=g(y)\in A$ e $f(x)=y$.

Logo $f$ {\'e} sobrejetora. Portanto $f$ {\'e} bijetora. Analogamente, prova-se que g {\'e} bijetora. Provemos que $g=f^{-1}$. Temos  $f^{-1}:B\rightarrow A$, da{\'\i}, $D(g) = B = D(f^{-1})$. Agora, $f\circ g = i_{B} = f\circ f^{-1}$. Assim, para todo $x\in F$, $(f\circ g)(x)=(f\circ f^{-1})(x)$. Isto {\'e}, $f(g(x))=f(f^{-1}(x))$. Portanto, $g(x)=f^{-1}(x)\forall x\in B$. Logo, $g=f^{-1}$. \#
\end{enumerate}

\begin{definicao}
	Seja $f : A \to B$ uma fun{\c c}{\~a}o.
	\begin{enumerate}
		\item Dado $P \sub A$, chama-se \textbf{imagem direta} de $P$  \textbf{segundo} $f$ e indica-se por $f(P)$ o subconjunto de $B$ dado por
		\[
			f(P) = \{f(x) \mid x \in P\},
		\]
		isto {\'e}, $f(P)$ {\'e} o conjunto das imagens por $f$ dos elementos de $P$.

		\item Dado $Q \sub B$, chama-se \textbf{imagem inversa} de $Q$ \textbf{segundo} $f$ e indica-se por $f^{-1}(Q)$ o subconjunto de $A$ dado por
		\[
			f^{-1}(Q) = \{x \in E \mid f(x) \in Q\},
		\]
		isto {\'e}, $f^{-1}(Q)$ {\'e} o conjunto dos elementos de $A$ que tem imagem em $Q$ atrav{\'e}s de $f$.
	\end{enumerate}
\end{definicao}

{\it Exemplos:}
\begin{enumerate}
	\item Seja $A = \{1, 3, 5, 7, 9 \}$ e $B = \{0, 1, 2, 3, \dots, 10\}$ e $f : A \to B$ dada por $f(x) = x + 1$. Temos que
	\begin{itemize}
		\item $f(\{3, 5, 7\}) = \{f(3), f(5), f(7)\} = \{4, 6, 8\}$

		\item $f(A) = \{f(1), f(3), f(5), f(7), f(9)\} = \{2, 4, 6, 8, 10\}$

		\item $f(\emptyset) = \emptyset$

		\item $f^{-1}(\{2, 4, 10\}) = \{x \in A \mid f(x) \in \{2, 4, 10\}\} = \{1, 3, 9\}$

		\item $f^{-1}(\{0, 1, 3, 5, 7, 9\}) = \{x \in A \mid f(x) \in \{0, 1, 3, 5, 7, 9\}\} = \emptyset$
	\end{itemize}

	\item Sejam $A = B= \real$ e $f : \real \to \real$ dada por $f(x) = x^2$. Temos
	\begin{itemize}
		\item $f(\{1, 2, 3\}) = \{1, 4, 9\}$

		\item $f([0,2]) = \{f(x) \in \real \mid 0 \le x \le 2 \} = \{x^2 \mid 0 \le x \le 2\} = [0, 4]$

		\item $f^{-1}([1, 9]) = \{ x \in \real \mid 1 \le f(x) \le 9\} = \{x \in \real \mid 1 \le x^2 \le 9\} = [-1, -3] \cup [1, 3]$
	\end{itemize}
\end{enumerate}

\begin{proposicao}
	Seja $f : A \to B$ uma fun{\c c}{\~a}o e sejam $P$, $Q \sub E$, $X$, $Y \sub B$.
	\begin{enumerate}
		\item Se $P \sub Q$, ent{\~a}o $f(P) \sub f(Q)$.

		\item $f^{-1}(X \cup Y) = f^{-1}(X) \cup f^{-1}(Y)$.
	\end{enumerate}
\end{proposicao}
\begin{prova}
	\begin{enumerate}
		\item Se $y \in f(P)$, ent{\~a}o existe $x \in P$ tal que $f(x) = y$. Mas como $P \sub Q$, ent{\~a}o $x \in Q$ e da{\'\i} $y \in f(Q)$. Logo $f(P) \sub f(Q)$.

		\item Seja $z \in f^{-1}(X \cup Y)$. Ent{\~a}o $f(z) \in X \cup Y$. Se $f(z) \in X$, entao $z \in f^{-1}(X)$ e da{\'\i} $z \in f^{-1}(X) \cup f^{-1}(Y)$. Se $f(z) \in Y$, ent{\~a}o $z \in f^{-1}(Y)$ e assim $z \in f^{-1}(X) \cup f^{-1}(Y)$. Logo, $f^{-1}(X \cup Y) \sub f^{-1}(X) \cup f^{-1}(Y)$.

		Agora, seja $z \in f^{-1}(X) \cup f^{-1}(Y)$. Se $z \in f^{-1}(X)$, ent{\~a}o $f(z) \in X$, da{\'\i} $f(z) \in X \cup Y$, isto {\'e}, $z \in f^{-1}(X \cup Y)$. Se $z \in f^{-1}(Y)$, ent{\~a}o $f(z) \in Y$ e assim $f(z) \in X \cup Y$, isto {\'e}, $z \in f^{-1}(X \cup Y)$. Logo $f^{-1}(X) \cup f^{-1}(Y) \sub f^{-1}(X \cup Y)$.

		Portanto, $f^{-1}(X \cup Y) = f^{-1}(X) \cup f^{-1}(Y)$.
	\end{enumerate}
\end{prova}
