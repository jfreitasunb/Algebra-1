%!TEX program = xelatex
%!TEX root = Algebra_1.tex
\chapter{Fun{\c c}{\~o}es}

\begin{definicao}
Uma \textbf{fun{\c c}{\~a}o} $f : A \to B$, de um conjunto $A$ em um conjunto $B$, {\'e} uma rela{\c c}{\~a}o que associa os elementos de $A$ com os elementos em $B$ satisfazendo as seguintes condições:
	\begin{enumerate}[label={\roman*})]
		\item Para todo $x \in A$, existe $y \in B$ tal que $f(x) = y$.
		\item  Se $x \in A$ \'e tal que $f(x) = y_1$ e $f(x) = y_2$ com $y_1$, $y_2 \in B$, ent\~ao $y_1 = y_2$.
	\end{enumerate}
Nesse caso $y$ \'e chamado de \textbf{imagem} de $x$ segundo $f$.
\end{definicao}

O conjunto $A$ {\'e} chamado de \textbf{dom{\'\i}nio} de $f$ e ser\'a denotado por $\dom(f)$. O conjunto $B$ {\'e} chamado de \textbf{contra-dom{\'\i}nio} de $f$. O conjunto
\[
	\im(f) = \{f(x) \mid x \in A\} \sub B
\]
\'e chamado \textbf{imagem} de $f$.

\begin{exemplos}
	\begin{enumerate}[label={\arabic*})]
	\item Sejam $A = \{0,1,2,3\}$ e $B = \{4,5,6,7,8\}$. Quais das seguintes rela{\c c}{\~o}es s{\~a}o fun{\c c}{\~o}es?
	\begin{enumerate}[label={\alph*})]
		\item $R_1 = \{(0,5),(1,6),(2,7)\}$
		\item $R_2 = \{(0,4),(1,5),(1,6),(2,7),(3,8)\}$
		\item $R_3 = \{(0,4),(1,5),(2,7),(3,8)\}$
		\item $R_4 = \{(0,5),(1,5),(2,6),(3,7)\}$
	\end{enumerate}
	\begin{solucao}
		\begin{enumerate}[label={\alph*})]
			\item N\~ao \'e fun\c{c}\~ao pois $3 \in A$ e $3$ n\~ao est\'a associado {\`a} nenhum elemento de $B$.
			\item N\~ao \'e fun\c{c}\~ao pois $1 \in A$ est\'a associado a dois elementos diferentes em $B$.
			\item \'E uma fun\c{c}\~ao.
			\item \'E uma fun\c{c}\~ao.
		\end{enumerate}
	\end{solucao}
	\item $R_{5} = \{(x,y) \in \real  \times \real  \mid y^2 = x^2\}$
	\begin{solucao}
	 	N\~ao \'e fun\c{c}\~ao pois, por exemplo, para $x = 1$ temos $y = -1$ ou $y = 1$.
	 \end{solucao}
	\item $R_{6} = \{(x,y) \in \real  \times \real  \mid x^2 + y^2 = 1\}$.
	\begin{solucao}
	 	N\~ao \'e fun\c{c}\~ao pois, por exemplo, para $x = 0$ temos $y = 1$ ou $y = -1$.
	 \end{solucao}
	\item  $R_{7} = \{(x,y) \in \real  \times \real \mid y = x^2\}$
	\begin{solucao}
	 	\'E uma fun\c{c}\~ao.
	\end{solucao}
	\end{enumerate}	
\end{exemplos}


\begin{definicao}
	Seja $f : A \to B$ uma fun\c{c}\~ao.
	\begin{enumerate}[label={\roman*})]
		\item Dizemos que $f$ \'e \textbf{injetora} se dados $x_1$, $x_2 \in A$ tais que $f(x_1) = f(x_2)$, ent\~ao $x_1 = x_2$. De modo equivalente, dizemos que $f$ e \textbf{injetora} se dados $x_1$, $x_2 \in A$ tais que $x_1 \ne x_2$, ent\~ao $f(x_1) \ne f(x_2)$.
		\item Dizemos que $f$ \'e \textbf{sobrejetora} se para todo $y \in B$, existe $x \in A$ tal que $f(x) = y$.
		\item Dizemos que $f$ e \textbf{bijetora} se $f$ for \textbf{injetora} e \textbf{sobrejetora} simultaneamente.
	\end{enumerate}
\end{definicao}

\begin{exemplos}
	Verifique se as seguintes fun\c{c}\~oes s\~ao injetoras ou sobrejetoras:
	\begin{enumerate}[label={\arabic*})]
		\item $f : \z \to \z$ dada por $f(x) = 3x + 1$
		\begin{solucao}
			De fato, dados $x_1$, $x_2 \in \z$ tais que $f(x_1) = f(x_2)$ temos
			\begin{align*}
				f(x_1) &= f(x_2)\\
				3x_1 + 1 &= 3x_2 + 1\\
				3x_1 &= 3x_2\\
				3(x_1 - x_2) &= 0.
			\end{align*}
			Assim $x_1 - x_2 = 0$, isto \'e, $x_1 = x_2$. Logo $f$ \'e injetora.

			Para determinar se $f$ \'e sobrejetora seja $y \in \z$. Precisamos determinar se \'e poss{\'\i}vel encontrar algum $x \in \z$ tal que $f(x) = y$. Ou seja, precisamos saber se a equa\c{c}\~ao $3x + 1 = y$ tem solu\c{c}\~ao em $\z$ para qualquer valor de $y$.

			Se tomarmos $y = 2$ temos
			\begin{align*}
				3x + 1 &= 2\\
				3x = 1
			\end{align*}
			e essa \'ultima equa\c{c}\~ao n\~ao possui solu\c{c}\~ao em $\z$. Logo para $y = 2$ \textbf{n\~ao} existe $x \in \z$ de modo que $f(x) = 2$. Logo $f$ n\~ao \'e sobrejetora.
		\end{solucao}

		\item $g : \rac \to \rac$ dada por $f(x) = 3x + 1$
		\begin{solucao}
			A prova que $g$ \'e injetora \'e id\^entica ao caso anterior.

			Para determinar se $g$ \'e sobrejetora seja $y \in \rac$. Precisamos determinar se \'e poss{\'\i}vel encontrar algum $x \in \rac$ tal que $g(x) = y$. Ou seja, precisamos saber se a equa\c{c}\~ao $3x + 1 = y$ tem solu\c{c}\~ao em $\rac$ para qualquer valor de $y$. Mas
			\begin{align*}
				&3x + 1 = y\\
				&3x = y - 1\\
				&x = \dfrac{y - 1}{3} \in \rac
			\end{align*}
			para qualquer valor de $y \in \rac$. Assim dado $y \in \rac$ tome $x = (y - 1)/3 \in \rac$. Da{\'\i}
			\[
				g(x) = g\left(\dfrac{y - 1}{3}\right) = 3\left(\dfrac{y - 1}{3}\right) + 1 = y - 1 + 1 = y. 
			\]
			Logo $g$ \'e sobrejetora.
		\end{solucao}
		
		\item A fun\c{c}\~ao $h :\real \to \real$ dada por $h(x) = x^2$
		\begin{solucao}
			A fun\c{c}\~ao $h$ n\~ao \'e injetora pois, por exemplo, $h(-1) = 1 = h(1)$ e $1\neq -1$.

			A fun\c{c}\~ao $h$ n\~ao \'e sobrejetora pois, por exemplo, para $y = -1$ \textbf{n\~ao} existe $x\in\mathbb{R}$ tal que $h(x) = -1$.
		\end{solucao}
		\end{enumerate}
\end{exemplos}

\begin{definicao}
	Sejam $f : A \to B$ e $g : B \to C$ fun\c{c}\~oes. Definimos a \textbf{fun\c{c}\~ao composta} de $g$ com $f$ como sendo a fun\c{c}\~ao denotada por $g \circ f : A \to C$ tal que $(g\circ f)(x) = g(f(x))$ para todo $x \in A$.
\end{definicao}

\begin{exemplos}
	\begin{enumerate}[label={\arabic*})]
		\item Sejam $f : \real \to \real$ e $g : \real \to \real$ dadas por $f(x) = x^2$ e $g(x) = x + 1$. Assim podemos definir $g \circ f$ e $f \circ g$ e
		\begin{align*}
			(g\circ f)(x) = g(f(x)) = g(x^2) = x^2 + 1\\
			(f\circ g)(x) = f(g(x)) = f(x + 1) = (x + 1)^2
		\end{align*}
		Assim em geral $f \circ g \ne g \circ f$.

		\item $f : \real_- \to \real^*_+$ e $g : \real^*_+ \to \real$ dadas por $f(x) = x^2 + 1$ e $g(x) = \ln x$. Nesse caso s\'o podemos definir $g \circ f : \real_- \to \real$ e
		\[
			(g\circ f)(x) = g(f(x)) = g(x^2 + 1) = \ln(x^2 + 1).
		\]
	\end{enumerate}
\end{exemplos}

\begin{proposicao}
	Se $f : A \to B$ e $g : B \to C$ s{\~a}o fun{\c c}{\~o}es injetoras, ent{\~a}o $g\circ f : A \to C$ {\'e} injetora.
\end{proposicao}
\begin{prova}
	Dados $x_1$, $x_2 \in A$ tais que $(g\circ f)(x_1) = (g\circ f)(x_2)$ queremos mostrar que $x_1 = x_2$. Temos:
	\begin{align*}
		(g\circ f)(x_1) &= (g\circ f)(x_2)\\
		g(f(x_1)) &= g(f(x_2)).
	\end{align*}
	Como por hip\'otese $g$ \'e injetora, dessa \'ultima igualdade segue que $f(x_1) = f(x_2)$. Mas $f$ tamb\'em \'e injetora, por hip\'otese, da{\'\i} $x_1 = x_2$, como quer{\'\i}amos. Portanto $g\circ f$ \'e injetora.
\end{prova}

\begin{proposicao}
	Se $f : A \to B$ e $g : B \to C$ s{\~a}o fun\c{c}\~oes sobrejetoras, ent{\~a}o $g\circ f : A \to C$ {\'e} sobrejetora.
\end{proposicao}
\begin{prova}
 	Para mostrar que $g \circ f : A \to C$ \'e sobrejetora, precisamos mostrar que para todo $y \in C$, existe $x \in A$ tal que $(g\circ f)(x) = y$.

 	Assim seja $y \in C$. Como $g : B \to C$ \'e sobrejetora, existe $z \in B$ tal que $g(z) = y$. Mas $z \in B$ e $f : A \to B$ \'e sobrejetora e assim existe $x \in A$ tal que $f(x) = z$. Logo
 	\[
 		(g\circ f)(x) = g(f(x)) = g(z) = y.
 	\]
 	Portanto $g \circ f$ \'e sobrejetora.
\end{prova}

\begin{definicao}
	Seja $f : A \to B$ uma fun{\c c}{\~a}o.
	\begin{enumerate}[label={\roman*})]
		\item Dado $P \sub A$, chama-se \textbf{imagem direta} de $P$  \textbf{segundo} $f$ e indica-se por $f(P)$ o subconjunto de $B$ dado por
		\[
			f(P) = \{f(x) \mid x \in P\},
		\]
		isto {\'e}, $f(P)$ {\'e} o conjunto das imagens por $f$ dos elementos de $P$.

		\item Dado $Q \sub B$, chama-se \textbf{imagem inversa} de $Q$ \textbf{segundo} $f$ e indica-se por $f^{-1}(Q)$ o subconjunto de $A$ dado por
		\[
			f^{-1}(Q) = \{x \in A \mid f(x) \in Q\},
		\]
		isto {\'e}, $f^{-1}(Q)$ {\'e} o conjunto dos elementos de $A$ que tem imagem em $Q$ atrav{\'e}s de $f$.
	\end{enumerate}
\end{definicao}

\begin{exemplos}
	\begin{enumerate}[label={\arabic*})]
		\item Seja $A = \{1, 3, 5, 7, 9 \}$ e $B = \{0, 1, 2, 3, \dots, 10\}$ e $f : A \to B$ dada por $f(x) = x + 1$. Temos:
		\begin{itemize}
			\item $f(\{1\}) = \{f(1)\} = \{2\}$

			\item $f(\{3, 5, 7\}) = \{f(3), f(5), f(7)\} = \{4, 6, 8\}$

			\item $f(A) = \{f(1), f(3), f(5), f(7), f(9)\} = \{2, 4, 6, 8, 10\}$

			\item $f(\emptyset) = \emptyset$

			\item $f^{-1}(\{2, 4, 10\}) = \{x \in A \mid f(x) \in \{2, 4, 10\}\} = \{1, 3, 9\}$

			\item $f^{-1}(\{0, 1, 3, 5, 7, 9\}) = \{x \in A \mid f(x) \in \{0, 1, 3, 5, 7, 9\}\} = \emptyset$
		\end{itemize}

		\item Sejam $A = B = \real$ e $f : \real \to \real$ dada por $f(x) = x^2$. Temos:
		\begin{itemize}
			\item $f(\{1, 2, 3\}) = \{1, 4, 9\}$

			\item $f([0,2]) = \{f(x) \in \real \mid 0 \le x \le 2 \} = \{x^2 \mid 0 \le x \le 2\} = [0, 4]$

			\item $f^{-1}([1, 9]) = \{x \in \real \mid f(x) \in [1, 9]\} = \{ x \in \real \mid 1 \le f(x) \le 9\} = \{x \in \real \mid 1 \le x^2 \le 9\} = [-1, -3] \cup [1, 3]$
		\end{itemize}
	\end{enumerate}
\end{exemplos}

\begin{proposicao}
	Seja $f : A \to B$ uma fun{\c c}{\~a}o e sejam $P$, $Q \sub A$, $X$, $Y \sub B$.
	\begin{enumerate}[label={\roman*})]
		\item Se $P \sub Q$, ent{\~a}o $f(P) \sub f(Q)$.

		\item $f^{-1}(X \cup Y) = f^{-1}(X) \cup f^{-1}(Y)$.
	\end{enumerate}
\end{proposicao}
\begin{prova}
	\begin{enumerate}[label={\roman*})]
		\item Se $y \in f(P)$, ent{\~a}o existe $x \in P$ tal que $f(x) = y$. Mas como $P \sub Q$, ent{\~a}o $x \in Q$ e da{\'\i} $y \in f(Q)$. Logo $f(P) \sub f(Q)$.

		\item Seja $z \in f^{-1}(X \cup Y)$. Ent{\~a}o $f(z) \in X \cup Y$. Se $f(z) \in X$, ent\~ao $z \in f^{-1}(X)$ e da{\'\i} $z \in f^{-1}(X) \cup f^{-1}(Y)$. Se $f(z) \in Y$, ent{\~a}o $z \in f^{-1}(Y)$ e assim $z \in f^{-1}(X) \cup f^{-1}(Y)$. Logo, $f^{-1}(X \cup Y) \sub f^{-1}(X) \cup f^{-1}(Y)$.

		Agora, seja $z \in f^{-1}(X) \cup f^{-1}(Y)$. Se $z \in f^{-1}(X)$, ent{\~a}o $f(z) \in X$, da{\'\i} $f(z) \in X \cup Y$, isto {\'e}, $z \in f^{-1}(X \cup Y)$. Se $z \in f^{-1}(Y)$, ent{\~a}o $f(z) \in Y$ e assim $f(z) \in X \cup Y$, isto {\'e}, $z \in f^{-1}(X \cup Y)$. Logo $f^{-1}(X) \cup f^{-1}(Y) \sub f^{-1}(X \cup Y)$.

		Portanto, $f^{-1}(X \cup Y) = f^{-1}(X) \cup f^{-1}(Y)$.
	\end{enumerate}
\end{prova}


Dado $f : A \to B$ uma fun\c{c}{\~a}o, queremos construir uma fun\c{c}\~ao $g : B \to A$ de modo que
\[
	g(f(x)) = x
\]
para todo $x \in A$. Mas $f(x) = y$ com $y \in B$. Assim podemos tentar definir $g$ como
\begin{align}\label{condicao_funcao_inversa}
	g(y) = x,\ y \in B \mbox{ se, e somente se, } f(x) = y.
\end{align}
Com essa defini\c{c}\~ao $g$ \'e uma fun\c{c}\~ao? Vejamos um exemplo: definia $f : \{0,1,2,3\} \to \{4,5,6,7,8\}$ por:
\begin{align*}
	f(0) &= 5\\
	f(1) &= 5\\
	f(2) &= 6\\
	f(3) &= 7.	
\end{align*}

A partir da defini\c{c}\~ao \eqref{condicao_funcao_inversa} temos
\begin{align*}
	g(5) &= 0\\
	g(5) &= 1\\
	g(6) &= 2\\
	g(7) &= 3.
\end{align*}

Assim $g$ definida pela condi\c{c}\~ao \eqref{condicao_funcao_inversa} n\~ao \'e uma fun\c{c}\~ao pois $g$ atribui ao n\'umero 5 dois poss{\'\i}veis valores: 0 e 1. Isso ocorre pois $f$ n\~ao \'e injetora. Vamos ent\~ao redefinir $f$ de modo a torn\'a-la injetora:
\begin{align*}
	f(0) &= 5\\
	f(1) &= 4\\
	f(2) &= 6\\
	f(3) &= 7.	
\end{align*}

Agora $g$ torna-se:
\begin{align*}
	g(5) &= 0\\
	g(4) &= 1\\
	g(6) &= 2\\
	g(7) &= 3.
\end{align*}

Ainda assim $g$ n\~ao \'e fun\c{c}\~ao pois $g$ n\~ao associa $8 \in B$ com nenhum elemento em $A$. Isso ocorre pois $f$ n\~ao \'e sobrejetora.

Portanto para que a condi\c{c}\~ao \eqref{condicao_funcao_inversa} defina uma fun\c{c}\~ao \'e necess\'ario que $f$ seja bijetora. Temos ent\~ao o seguinte teorema:

\begin{teorema}\label{teorema_funcao_inversa}
	Seja $f: A \to B$ fun{\c c}{\~a}o. Defina $g : B \to A$ por
	\begin{align}\label{funcao_inversa}
		g(y) = x,\ y \in B \mbox{ se, e somente se, } f(x) = y.
	\end{align}
	Ent{\~a}o $g$ {\'e} uma fun{\c c}{\~a}o se, e somente se, $f$ {\'e} bijetora.
\end{teorema}
\begin{prova}
	Precisamos mostrar que:
	\begin{enumerate}[label={\roman*})]
		\item Se $g$ definida por \eqref{funcao_inversa} \'e uma fun\c{c}\~ao, ent\~ao $f$ \'e bijetora.
		\item Se $f$ \'e bijetora, ent\~ao $g$ definida por \eqref{funcao_inversa} \'e uma fun\c{c}\~ao.
	\end{enumerate}

	Provemos a primeira afirma\c{c}\~ao: suponha que $g$ \'e uma fun\c{c}\~ao. Precisamos provar que $f$ {\'e} injetora e sobrejetora.

	Sejam $x_1$, $x_2 \in A$ tais que $f(x_1) = y = f(x_2)$. Como $f(x_1) = y$ temos $g(y) = x_1$, al{\'e}m disso, $g(y) = x_2$. Mas $g$ {\'e} uma fun{\c c}{\~a}o, da{\'\i} $x_1 = x_2$, ou seja, $f$ {\'e} injetora.

	Dado $y \in B$, como $g$ {\'e} uma fun{\c c}{\~a}o, existe $x \in A$, tal que $g(y) = x$, logo $f(x) = y$ e assim $f$ {\'e} sobrejetora.

	Portanto $f$ {\'e} bijetora.

	Agora vamos provar a segunda afirma\c{c}\~ao: suponha que $f$ \'e bijetora. Precisamos mostrar que $g$ \'e uma fun\c{c}\~ao. Primeiramente, dado $y \in B$, como $f$ {\'e} sobrejetora, existe $x \in A$ tal que $f(x) = y$. Logo por \eqref{funcao_inversa} segue que $g(y) = x \in A$. Logo $g$ associa cada elemento de $B$ com algum elemento em $A$.

	Suponha que $g(y) = x_1$ e que $g(y) = x_2$. Da{\'\i}, de \eqref{funcao_inversa} temos $f(x_1) = y$ e $f(x_2) = y$. Mas $f$ {\'e} injetora, logo $x_1 = x_2$ e ent{\~a}o $g(y) = x_1 = x_2$. Assim $g$ associa cada elemento de $B$ com somente um elemento em $A$.

	Portanto $g$ {\'e} fun{\c c}{\~a}o.
\end{prova}


\begin{definicao}
	A fun\c{c}\~ao $g : B \to A$ do teorema \ref{teorema_funcao_inversa} \'e chamada de \textbf{fun\c{c}\~ao inversa} de $f : A \to B$ e ser\'a denotada por $g = f^{-1}$.
\end{definicao}


\begin{definicao}
	Dado um conjunto $A \ne \emptyset$, a fun{\c c}{\~a}o $i_{A}: A \to A$ dada por $i_{A}(x) = x$ {\'e} chamada de \textbf{fun{\c c}{\~a}o identidade}.
\end{definicao}

\begin{proposicao}
	Se $f : A \to B$ {\'e} bijetora, ent{\~a}o $f\circ f^{-1} = i_{B}$ e $f^{-1}\circ f = i_{A}$.
\end{proposicao}
\begin{prova}
	Temos $i_{B} : B \to B$ e $i_{A} : A \to A$. Al{\'e}m disso, $f\circ f^{-1} : B \to B$ e $f^{-1}\circ f : A \to A$, da{\'\i} $\dom(f\circ f^{-1}) = \dom(i_{B})$ e $\dom(f^{-1}\circ f) = \dom(i_{A})$. Agora, $y \in B$, $(f\circ f^{-1})(y) = f(f^{-1}(y)) = y = i_{B}(y)$. E se $x \in A$, $(f^{-1}\circ f)(x) = f^{-1}(f(x)) = x = i_{A}(x)$. Portanto $f\circ f^{-1} = i_{B}$ e $f^{-1}\circ f = i_{A}$ como quer{\'\i}amos.
\end{prova}

\begin{proposicao}\label{propriedades_identidade}
	Se $f : A \to B$ e $g : B \to A$ s{\~a}o fun{\c c}{\~o}es, ent{\~a}o:
	\begin{enumerate}[label={\roman*})]
		\item $f\circ i_{A} = f$
		\item $i_{B}\circ f = f$
		\item $g\circ i_{B} = g$
		\item $i_{A}\circ g = g$
		\item Se $g\circ f = i_{A}$ e $f\circ g = i_{B}$, ent{\~a}o $f$ e $g$ s{\~a}o bijetoras e $g=f^{-1}$.
	\end{enumerate}
\end{proposicao}
\begin{prova}
	\begin{enumerate}[label={\roman*})]
		\item Primeiro temos $f: A \to B$ e $i_{A} : A \to A$ e $f\circ i_{A} : A \to B$. Assim $\dom(f\circ i_{A}) = \dom(f)$. Agora dado $x \in A$, temos $(f\circ i_{A})(x) = f(i_{A}(x)) = f(x)$. Portanto, $f\circ i_{A} = f$.
		\item Segue de forma semelhante ao caso anteiror.
		\item Segue de forma semelhante ao primeiro caso.
		\item Segue de forma semelhante ao primeiro caso.
		\item Provemos que $f$ \'e bijetora: sejam $x_1$, $x_2 \in B$ tais que $f(x_1) = f(x_2)$. Como $f : A \to B$ e $g : B \to A$, ent{\~a}o $g(f(x_1)) = g(f(x_2))$, ou seja, $(g\circ f)(x_1) = (g\circ f)(x_2)$. Da{\'\i}, $i_{A}(x_1) = i_{A}(x_2)$. Logo, $x_1 = x_2$. Logo $f$ {\'e} injetora.

		Agora, dado $y \in B$, segue que $y = i_{B}(y)$. Mas $i_{B} = f\circ g$. Da{\'\i}, $y = i_{B}(y) = (f\circ g)(y) = f(g(y))$. Assim, $x = g(y)\in A$ e $f(x) = y$. Logo $f$ {\'e} sobrejetora.

		Portanto $f$ {\'e} bijetora. Analogamente, prova-se que g {\'e} bijetora. 

		Provemos agora que $g = f^{-1}$. Para isso, primeiro temos  $f^{-1} : B \to A$ e ent\~ao $\dom(g) = B = \dom(f^{-1})$. Agora, $f\circ g = i_{B} = f\circ f^{-1}$. Assim, para todo $x \in B$, $(f\circ g)(x) = (f\circ f^{-1})(x)$. Isto {\'e}, $f(g(x)) = f(f^{-1}(x))$. Portanto como $f$ \'e injetora, $g(x) = f^{-1}(x)$ para todo $x\in B$. Logo $g = f^{-1}$ como quer{\'\i}amos.
	\end{enumerate}
\end{prova}