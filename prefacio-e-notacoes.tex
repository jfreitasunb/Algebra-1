%!TEX program = xelatex
%!TEX root = Algebra_1.tex
%%Usar makeindex -s indexstyle.ist arquivo.idx no terminal para gerar o {\'\i}ndice remissivo agrupado por inicial
%%Ap\'os executar pdflatex arquivo
\begin{center}
\Huge \textbf{Pref{\'a}cio}
\end{center}

Essas notas de Aula s{\~a}o referentes {\`a} mat{\'e}ria {\'A}lgebra 1,
ministrada na UnB - Universidade de Bras{\'\i}lia - durante o 2$^o$ Semestre de 2010
pelo professor Jos{\'e} Ant{\^o}nio O. de Freitas, Departamento de Matem{\'a}tica. Tais
notas foram transcritas e editadas pelo graduando em Ci{\^e}ncias Econ{\^o}micas
Luiz Eduardo Sol R. da Silva\footnote{luizeduardosol@hotmail.com}.

Revisão e ampliação das notas feita por José Antônio O. de Freitas.


{\'E} livre a reprodu{\c c}{\~a}o, distribui{\c c}{\~a}o e edi{\c c}{\~a}o deste material, desde que citadas as suas fontes e autores. Cr{\'\i}ticas e sugest{\~o}es s{\~a}o bem vindas.
\vspace{20cm}




% \begin{center} \textbf{\Large Nota{\c c}{\~o}es e express{\~o}es}
% \end{center}
% \begin{minipage}[l]{0,5\textwidth}
% \begin{itemize}
% \item $\neg$ N{\~a}o
% \item $\forall$ Para todo
% \item $/$ Tal que
% \item $|$ Divide
% \item $\Rightarrow$ Implica
% \item $\in$ Pertence
% \item $\emptyset$ Vazio
% \item $\subseteq$ Contido ou igual a
% \item $\supseteq$ Cont{\'e}m ou igual a
% \item $\wedge$ E
% \item $\vee$ Ou
% \item $=$ Igual
% \item $\neq$ Diferente
% \item $\mathbb{Z}$ N{\'u}meros Inteiros
% \item $\mathbb{R}$ N{\'u}meros Reais
% \item $\cap$ Intersec{\c c}{\~a}o
% \item $>$ Maior que
% \item $\geq$ Maior ou igual a
% \item $\displaystyle\bigcup_{i=1}^{n}$ Uni{\~a}o de $n$ conjuntos
% \item $\displaystyle\bigsqcup_{i=1}^{n}$ Uni{\~a}o disjunta de $n$ conjuntos


% \end{itemize}
% \end{minipage}
% \begin{minipage}[r]{0,5\textwidth}
% \begin{itemize}

% \item $\leftrightarrow$ Se, e somente se
% \item $\veebar$ Ou...,ou..., mas nunca ambos
% \item $\rightarrow$ Se,... ent{\~a}o...
% \item $\exists$ Existe
% \item $\Leftrightarrow$ Equivalente a
% \item $\notin$ N{\~a}o pertence
% \item \# Fim da demonstra{\c c}{\~a}o
% \item $\mathbb{N}$ N{\'u}meros Naturais
% \item $\mathbb{Q}$ N{\'u}meros Racionais
% \item $\nsubseteq$ N{\~a}o cont{\'e}m ou {\'e} igual a
% \item $\cup$ Uni{\~a}o
% \item $\sqcup$ Uni{\~a}o Disjunta
% \item $<$ Menor que
% \item $\leq$ Menor ou igual a
% \item $\displaystyle\bigcap_{i=1}^{n}$ Intersec{\c c}{\~a}o de $n$ conjuntos
% \item Q.E.D. (\textit{Quod Erat Demonstrandum}): Como se queria demonstrar
% \item P.B.O.: Princ{\'\i}pio da boa ordena{\c c}{\~a}o
% \item H.I.: Hip{\'o}tese de Indu{\c c}{\~a}o
% \item \textit{Mutatis Mutandis}:  Mudando o que tem que ser mudado

% \end{itemize}
% \end{minipage}